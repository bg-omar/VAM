\subsection{Derivation of Planck's Law from Entropy Principles}

Planck’s law describes the spectral distribution of radiation emitted by a blackbody at thermal equilibrium. Its derivation from entropy considerations provides profound insight into the quantum nature of radiation and matter.

\subsubsection*{Energy Quantization and Statistical Mechanics}

Planck considered a cavity containing electromagnetic radiation, modeling the radiation field as a collection of harmonic oscillators. He postulated that the energy levels of an oscillator of frequency $\nu$ are quantized:

\begin{equation*}
    E_n = n h \nu, \quad n = 0,1,2, \dots
\end{equation*}

where $h$ is Planck’s constant and $\nu$ is the frequency of the oscillator.

The probability of an oscillator being in state $n$ follows the Boltzmann distribution:

\begin{equation*}
    p_n = \frac{e^{-E_n / k_B T}}{Z},
\end{equation*}

where $Z$ is the partition function:

\begin{equation*}
    Z = \sum_{n=0}^{\infty} e^{-n h \nu / k_B T} = \frac{1}{1 - e^{-h \nu / k_B T}}.
\end{equation*}

The average energy of an oscillator is given by:

\begin{equation*}
    \langle E \rangle = \frac{\sum_{n=0}^{\infty} n h \nu e^{-n h \nu / k_B T}}{\sum_{n=0}^{\infty} e^{-n h \nu / k_B T}},
\end{equation*}

which evaluates to:

\begin{equation*}
    \langle E \rangle = \frac{h \nu}{e^{h \nu / k_B T} - 1}.
\end{equation*}

\subsubsection*{Spectral Energy Density and Entropy}

Planck related the average energy of oscillators to the spectral energy density $u(\nu, T)$ of radiation at frequency $\nu$:

\begin{equation*}
    u(\nu, T) = \frac{8 \pi \nu^2}{c^3} \langle E \rangle.
\end{equation*}

Substituting $\langle E \rangle$, we obtain the Planck radiation law:

\begin{equation*}
    u(\nu, T) = \frac{8 \pi h \nu^3}{c^3} \frac{1}{e^{h \nu / k_B T} - 1}.
\end{equation*}

\subsubsection*{Entropy Considerations}

Entropy plays a crucial role in Planck’s derivation. By considering the entropy $S$ of a system in terms of the probability distribution of states, using Boltzmann’s entropy formula:

\begin{equation*}
    S = - k_B \sum_i p_i \ln p_i,
\end{equation*}

Planck ensured that the energy distribution satisfied thermodynamic constraints. The quantization of energy was introduced to avoid the ultraviolet catastrophe predicted by classical physics.

\subsubsection*{Implications for Quantum Theory}

Planck’s derivation established the foundation for quantum mechanics, leading to the realization that:
\begin{enumerate}
    \item Energy exchanges occur in discrete quanta.
    \item Entropy governs the equilibrium properties of quantum systems.
    \item The relationship between thermodynamics and quantum mechanics is deeply rooted in statistical distributions.
\end{enumerate}

This work paved the way for further quantum developments, including Einstein’s explanation of the photoelectric effect and Bohr’s atomic model.




    \section{Advanced Derivation of Relative Vorticity Between Two Vortex Knots in the Æther Model}

    \subsection*{Assumptions and Theoretical Framework}
    \textbf{Rigid Rotor Dynamics:} Each vortex knot is conceptualized as a rigidly rotating entity, maintaining a stable angular velocity throughout its core. These cores are presumed to exhibit minimal deformation, ensuring that their rotational characteristics remain consistent under idealized conditions.

    \textbf{Vorticity as a Vector Field:} The vorticity vector for each knot is defined as:
    \begin{equation*}
        \vec{\omega} = \omega \hat{z},
    \end{equation*}
    where the Z-axis serves as the axis of rotation. This orientation aligns with the inherent symmetry of the system and simplifies analytical treatment.

    \textbf{Kinematic Parameters:}
    \begin{itemize}
        \item \textbf{Spatial positions:} The vortex knots occupy positions $z_1$ and $z_2$ along the Z-axis, maintaining a clear separation that facilitates distinct dynamic interactions.
        \item \textbf{Temporal velocities:} Their respective velocities along the Z-axis are represented as:
        \begin{equation*}
            v_1 = \frac{dz_1}{dt}, \quad v_2 = \frac{dz_2}{dt}.
        \end{equation*}
        \item \textbf{Relative velocity:} Defined as:
        \begin{equation*}
            v_{\text{rel}} = v_2 - v_1 = \frac{d(z_2 - z_1)}{dt},
        \end{equation*}
        this parameter quantifies the differential motion between the two knots.
    \end{itemize}

    \textbf{Vortex Tube Structure:} The knots are interconnected through a vortex tube characterized by uniform vorticity and angular momentum transfer. This structure acts as a conduit, ensuring the propagation of rotational effects along the Z-axis.

    \textbf{Æther Properties:} The surrounding Æther medium is assumed to be incompressible and inviscid, providing a stable environment for the conservation of vorticity and angular momentum.

    \subsection*{Derivation of Relative Vorticity}
    \textbf{Differential Vorticity:} The relative vorticity between the two vortex knots is expressed as:
    \begin{equation*}
        \Delta \omega = \omega_2 - \omega_1.
    \end{equation*}

    \textbf{Relationship Between Angular Velocity and Vorticity:} The angular velocity $\theta$ governs the local vorticity for each knot:
    \begin{equation*}
        \omega_1 = \frac{d\theta_1}{dt}, \quad \omega_2 = \frac{d\theta_2}{dt}.
    \end{equation*}
    By extension:
    \begin{equation*}
        \Delta \omega = \frac{d\theta_2}{dt} - \frac{d\theta_1}{dt} = \frac{d(\theta_2 - \theta_1)}{dt}.
    \end{equation*}

    \textbf{Relative Angular Velocity:} The angular displacement difference evolves over time as:
    \begin{equation*}
        \Delta \omega = \omega_{\text{rel}} = \frac{d(\theta_2 - \theta_1)}{dt}.
    \end{equation*}
    This term directly quantifies the rotational disparity between the two knots.

    \subsection*{Coupling Relative Vorticity to Translational Dynamics}
    \textbf{Translational-Vorticity Mapping:} Angular dynamics propagate through the vortex tube, linking rotational motion to linear velocities via:
    \begin{equation*}
        \omega_{\text{rel}} = C \frac{v_{\text{rel}}}{|z_2 - z_1|},
    \end{equation*}
    where $C$ represents a dimensionless proportionality constant encapsulating the tube’s properties and the Æther’s physical characteristics.

    \textbf{Incorporation of Relative Velocity:} Substituting $v_{\text{rel}} = v_2 - v_1$, the relative vorticity becomes:
    \begin{equation*}
        \Delta \omega = C \frac{v_2 - v_1}{|z_2 - z_1|}.
    \end{equation*}
    This formula succinctly connects the linear and angular dynamics of the system.

    \subsection*{Extended Physical Interpretation}
    \textbf{Proportionality Constant $C$:}
    \begin{itemize}
        \item The constant $C$ embodies the dynamic interplay between the vortex tube and the Æther. Its value depends on the tube’s rigidity, rotational coherence, and the Æther’s response to angular perturbations.
        \item In specific cases, $C$ may exhibit dependence on additional parameters such as the local pressure gradient or induced vorticity from neighboring flows.
    \end{itemize}

    \textbf{Distance Dependence:}
    \begin{itemize}
        \item The inverse proportionality with $|z_2 - z_1|$ highlights the localized nature of the vorticity exchange. Closer proximity enhances the interaction strength, amplifying rotational coupling.
        \item This dependence aligns with observations in both classical fluid dynamics and topological fluid mechanics, where vortex interactions intensify with decreasing separation.
    \end{itemize}

    \textbf{Velocity Gradient Influence:}
    \begin{itemize}
        \item The formula indicates a direct proportionality between relative velocity $(v_2 - v_1)$ and relative vorticity $\Delta \omega$. Rapid differential motion introduces greater rotational disparities, emphasizing the sensitivity of vorticity dynamics to translational changes.
    \end{itemize}

    \subsection*{Implications for Energy Transfer}
    The coupling of linear and angular dynamics suggests a potential mechanism for energy redistribution within vortex systems. As relative velocity increases, angular momentum may be preferentially transferred through the vortex tube.

    \subsection*{Conclusion}
    This comprehensive derivation offers a robust theoretical foundation for understanding relative vorticity in terms of translational dynamics within the Æther model. By linking angular and linear motion through the vortex tube, the framework highlights key relationships that govern vortex interactions. Future research should prioritize refining the proportionality constant $C$, exploring nonlinear extensions, and leveraging advanced experimental techniques to validate and extend the model.

    \subsection*{References}
    \begin{itemize}
        \item Kleckner, Dustin, and William T. M. Irvine. "Creation and Dynamics of Knotted Vortices." Nature Physics, vol. 9, no. 4, 2013, pp. 253-258. \href{https://doi.org/10.1038/NPHYS2560}{DOI}
        \item Sullivan, Ian S., et al. "Dynamics of Thin Vortex Rings." Journal of Fluid Mechanics, vol. 609, 2008, pp. 319–347. \href{https://doi.org/10.1017/S0022112008002292}{DOI}
        \item Vinen, W. F. "The Physics of Superfluid Helium." Reports on Progress in Physics, vol. 66, no. 12, 2003, pp. 2069–2117. \href{https://doi.org/10.1088/0034-4885/66/12/R01}{DOI}
    \end{itemize}


\section{Derivation of the Vorticity Wave Speed $v_v^2$}

In the Vortex \AE ther Model (VAM), the vorticity wave speed $v_v$ emerges from the relationship between the vorticity permeability constant $\mu_v$, the \AE ther density $\rho_{\ae}$, and the vortex-core radius $r_c$. The permeability constant is given by:

\begin{equation*}
    \mu_v = \frac{\rho_{\ae} r_c^2}{4}.
\end{equation*}

Using this, we define the vorticity wave speed squared as:

\begin{equation*}
    v_v^2 = \frac{4}{\rho_{\ae} r_c^2}.
\end{equation*}

Substituting typical values for the \AE ther density and vortex-core radius, we obtain:

\begin{equation*}
    v_v^2 \approx 1.612 \times 10^{37} \text{ m}^2 \text{ s}^{-2}.
\end{equation*}

Taking the square root, the vorticity wave velocity is:

\begin{equation*}
    v_v \approx 1.27 \times 10^{18} \text{ m/s}.
\end{equation*}

This extraordinarily high value suggests that vorticity waves in the \AE ther propagate at speeds far exceeding the speed of light, reinforcing the hypothesis that these waves are responsible for fundamental interactions at the quantum scale. Such a velocity implies that interactions mediated by vorticity fields occur nearly instantaneously over macroscopic distances, challenging the traditional limitations imposed by relativity.

A crucial implication of this derivation is that the speed of vorticity waves could be connected to the scale of vacuum fluctuations and quantum entanglement effects. If these waves underpin fundamental interactions, their influence could extend beyond classical electromagnetism and gravity, potentially serving as a medium for long-range force interactions and energy transport mechanisms at both quantum and cosmological scales.

\section{Derivation of the Wave Equation for $B_v$}

From the vorticity-Maxwell formulation in VAM, we define the vorticity-induced magnetic field as:

\begin{equation*}
    B_v = \mu_v \omega,
\end{equation*}

where $\omega$ is the vorticity field. Taking the curl of both sides and applying the time evolution equation for vorticity, we obtain:

\begin{equation*}
    \nabla \times B_v = \nabla \times (\mu_v \omega).
\end{equation*}

Applying the vorticity conservation equation,

\begin{equation*}
    \frac{\partial \omega}{\partial t} + (\boldsymbol{u} \cdot \nabla) \omega = (\omega \cdot \nabla) \boldsymbol{u},
\end{equation*}

and substituting into the Maxwell-like formulation for vorticity, we arrive at the vorticity wave equation:

\begin{equation*}
    \nabla^2 B_v - \frac{1}{v_v^2} \frac{\partial^2 B_v}{
        \partial t^2} = 0.
\end{equation*}

This equation describes the propagation of vorticity-induced magnetic fields in \AE ther, analogous to Maxwell’s wave equation in classical electromagnetism. The primary distinction is the presence of $v_v$ instead of the speed of light $c$, indicating that vorticity waves propagate at ultra-relativistic speeds within the \AE ther.

Further analysis suggests that in highly curved vorticity fields, nonlinear corrections could emerge, leading to dispersion and mode coupling effects. This may explain observed discrepancies in high-energy physics where quantum fluctuations deviate from classical electromagnetic wave behavior.

\section{Physical Interpretation and Implications}

The derived expressions suggest:

\begin{itemize}
    \item Vorticity waves in \AE ther propagate at extremely high speeds, potentially allowing for near-instantaneous interactions across vast distances.
    \item The presence of a structured vorticity medium influences electromagnetism, modifying classical interpretations of field propagation and quantum entanglement.
    \item These findings support the hypothesis that quantum interactions and vacuum fluctuations arise from high-frequency vorticity waves in \AE ther, bridging the gap between classical field theory and quantum mechanics.
    \item The relationship between $v_v^2$ and fundamental constants suggests a new framework for unifying electromagnetism and gravity as manifestations of vorticity dynamics rather than separate fundamental forces.
\end{itemize}

Additionally, experimental tests can be devised to validate these predictions:
\begin{itemize}
    \item High-sensitivity SQUID magnetometers could detect anomalous field fluctuations associated with vorticity waves, providing indirect evidence of their existence.
    \item Bose-Einstein condensates and superfluid helium interferometry could serve as analog models to test the propagation of structured vorticity waves.
    \item Plasma physics experiments involving rotating high-energy vortex structures could reveal unexpected electromagnetic signatures aligned with the predictions of the VAM framework.
\end{itemize}

These derivations and implications lay the groundwork for further exploration into the unification of fundamental forces within the VAM framework, potentially offering new insights into quantum gravity and high-energy physics.

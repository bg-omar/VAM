

    \section{Photon as a Vortex Dipole and its Implications for Effective Gravitational Coupling}

    \subsection*{Abstract}
    The conceptualization of photons as oscillating electron-positron vortex pairs redefines their role in gravitational dynamics and offers a novel framework for understanding photon-gravity interactions. By modeling the photon as a vortex dipole, this approach integrates vortex dynamics into gravitational interactions, yielding effective gravitational coupling within the context of the Æther dynamics model. This article provides an advanced analysis of the mathematical and physical implications of this framework, emphasizing its consistency with the Æther paradigm.

    \subsection*{Photon as a Vortex Dipole}
    In this model, a photon is depicted as a pair of counter-rotating vortices—akin to an oscillating electron-positron pair—with a time-dependent separation radius $R_\text{vortex}(t)$. The vortices exhibit circulation:
    \begin{equation*}
        \Gamma = 2 \pi R_\text{vortex}^2 \omega_c,
    \end{equation*}
    where $\omega_c$ is the angular velocity, and their separation oscillates as:
    \begin{equation*}
        R_\text{vortex}(t) = R_\text{vortex,0} \cos(\omega t).
    \end{equation*}
    This dynamic induces localized pressure gradients and energy distributions that interact with external gravitational fields, establishing a coupling mechanism between the photon and the curvature of spacetime as described in the Æther model.

    \subsection*{Vorticity Field and Gravitational Potential}
    Within the Æther model, gravitational effects are described via a vorticity-induced potential:
    \begin{equation*}
        \Phi_{\text{vortex}} = \frac{C_e^2}{2 F_{\max}} \vec{\omega} \cdot \vec{r},
    \end{equation*}
    where $C_e$ is the vortex angular velocity constant, $F_{\max}$ represents the maximum force, and $\vec{\omega} = \nabla \times \vec{v}$ is the vorticity field.

    For a photon modeled as a vortex dipole, the total gravitational potential is expressed as:
    \begin{equation*}
        \Phi_{\text{vortex}}^{\text{photon}} = \frac{C_e^2}{2 F_{\max}} \left( \vec{\omega}_+ \cdot \vec{r}_+ + \vec{\omega}_- \cdot \vec{r}_- \right),
    \end{equation*}
    where $\vec{\omega}_+$ and $\vec{\omega}_-$ denote the vorticities of the electron and positron vortices, and $\vec{r}_+$, $\vec{r}_-$ represent their respective positions.

    \subsection*{Gravitational Energy of the Photon}
    The gravitational energy associated with the photon vortex pair is given by:
    \begin{equation*}
        E_{\text{grav}} = -\int_V \rho_{\text{vortex}} \Phi_{\text{vortex}} \, dV,
    \end{equation*}
    where $\rho_{\text{vortex}}$ is the effective energy density of the vortex cores:
    \begin{equation*}
        \rho_{\text{vortex}} = \frac{\Gamma^2}{8 \pi^2 R_{\text{vortex}}^2}.
    \end{equation*}
    Substituting $\Phi_{\text{vortex}}$ and $\rho_{\text{vortex}}$, the gravitational energy becomes:
    \begin{equation*}
        E_{\text{grav}} = -\frac{C_e^2}{2 F_{\max}} \int_V \frac{\Gamma^2}{8 \pi^2 R_{\text{vortex}}^2} \vec{\omega} \cdot \vec{r} \, dV.
    \end{equation*}

    \subsection*{Photon Deflection and Gravitational Redshift}
    \textbf{1. Photon Deflection}
    The trajectory of a photon is perturbed by spacetime curvature, modeled as gradients in the vorticity potential:
    \begin{equation*}
        \frac{d^2 \vec{r}}{dt^2} = -\nabla \Phi_{\text{vortex}}^{\text{photon}}.
    \end{equation*}

    \textbf{2. Gravitational Redshift}
    As photons traverse regions of varying vorticity potential, their frequency shifts in a manner described by:
    \begin{equation*}
        \Delta f = f \left( \frac{\Delta \Phi_{\text{vortex}}}{c^2} \right).
    \end{equation*}

    \subsection*{Effective Gravitational Coupling}
    The photon’s internal vortex dynamics establish an effective gravitational coupling constant $\alpha_g$:
    \begin{equation*}
        \alpha_g = \frac{C_e^2 t_p^2}{R_{\text{vortex}}^2},
    \end{equation*}
    where $t_p$ denotes the Planck time. By relating this to the conventional gravitational constant $G$, we derive:
    \begin{equation*}
        G = \frac{C_e c^3 t_p^2}{R_c M_e},
    \end{equation*}
    linking the vortex structure of photons to fundamental gravitational interactions.

    \subsection*{Conclusion}
    The model of photons as vortex dipoles offers a transformative perspective on photon-gravity interactions. By integrating vortex dynamics into the Æther model, it elucidates new mechanisms for coupling gravitational and electromagnetic phenomena, potentially bridging classical and quantum frameworks.

    \subsection*{References}
    \begin{itemize}
        \item Bühl, Oliver, and Michael E. McIntyre. "Wave capture and wave–vortex duality." Journal of Fluid Mechanics 534 (2005): 67-95. \href{https://doi.org/10.1017/S0022112005004374}{DOI}
        \item Kleckner, Dustin, and William T. M. Irvine. "Creation and dynamics of knotted vortices." Nature Physics 9, no. 4 (2013): 253-258. \href{https://doi.org/10.1038/nphys2560}{DOI}
        \item Meunier, Patrice, Stéphane Le Dizès, and Thomas Leweke. "Physics of vortex merging." Comptes Rendus Physique 6, no. 4-5 (2005): 431-450. \href{https://doi.org/10.1016/j.crhy.2005.06.003}{DOI}
        \item Meyl, Konstantin. \textit{Scalar Waves: First Tesla Physics Textbook for Engineers}. INDEL Verlag, 2003.
        \item Vinen, W. F. "The physics of superfluid helium." Reports on Progress in Physics 67, no. 4 (2004): 523-586. \href{https://doi.org/10.1088/0034-4885/67/4/R01}{DOI}
    \end{itemize}


%! Author = mr
%! Date = 3/16/2025
\section{Electro-Vortex Dynamics in the Vortex \AE ther Model}

\subsection{Electro-Vortex Drift Velocity Equation}
The drift velocity of a charged vortex filament in an \AE theric field is given by:
\begin{equation}
    v_{\ae} = \frac{C_e E_{\ae}}{\rho_{\ae}}
\end{equation}
where:
\begin{itemize}
    \item $v_{\ae}$ is the \AE theric drift velocity,
    \item $C_e$ is the vortex response coefficient,
    \item $E_{\ae}$ is the \AE theric field strength,
    \item $\rho_{\ae}$ is the \AE theric density.
\end{itemize}

\subsection{Electro-Vortex Thrust Force}
The electro-vortex thrust force in \AE ther is given by:
\begin{equation}
    F_{\ae} = \rho_{\ae} C_e^2 \nabla E_{\ae}
\end{equation}
This describes the generation of force purely from an \AE theric field gradient.

\subsection{Vortex-Induced \AE theric Pressure Gradient}
The pressure gradient due to a charged vortex filament is:
\begin{equation}
    \nabla P_{\ae} = \rho_{\ae} \left( \frac{C_e^2}{r_c} \right)
\end{equation}
This equation suggests that \AE theric pressure gradients arise naturally due to vortex motion.

\subsection{\AE theric Field Strength and Maximum Force Relation}
The fundamental constraint between \AE theric field strength and maximum force is:
\begin{equation}
    E_{\ae} = \frac{F_{\max}}{r_c^2 \rho_{\ae}}
\end{equation}
This ensures that the \AE theric field remains bounded by fundamental force limits.

\subsection{Vortex Energy Density in Electro-Vortex Systems}
The energy density of a charged vortex filament is:
\begin{equation}
    U_{\ae} = \frac{1}{2} \rho_{\ae} C_e^2 + \frac{E_{\ae}^2}{2 \rho_{\ae}}
\end{equation}
This equation expresses the total kinetic and potential energy stored in \AE theric vorticity fields.

\subsection{\AE theric Vortex-Induced Charge Separation}
Charge separation within \AE theric vortex filaments follows:
\begin{equation}
    \sigma_{\ae} = \rho_{\ae} \frac{C_e}{r_c}
\end{equation}
where $\sigma_{\ae}$ is the \AE theric charge density per unit volume.

These equations provide a framework for exploring electro-vortex interactions and propulsion mechanisms within the Vortex \AE ther Model.

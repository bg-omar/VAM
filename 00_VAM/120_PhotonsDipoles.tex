%! Author = omar.iskandarani
%! Date = 3/14/2025

\textbf{Below is a step-by-step mathematical outline of how one might formalize the “rolling ring vortex” (or vortex dipole) interpretation of photons and other quanta in a non-viscous, incompressible Æther, consistent with your Vortex Æther Model (VAM). The goal is to show how to embed the physical concepts—vortex-based photons, wave–particle duality, speed of light \(c\), etc.—into a coherent system of fluid equations. Throughout, we will reference:}

\begin{itemize}
 \item Classical equations of ideal fluid flow (Euler’s equation, vorticity transport),
 \item Topological constraints (vortex knots, Kelvin’s circulation theorem),
 \item Dimensional analysis linking constants like \(c\), \(C_e\), and \(F_{\max}\),
 \item Boundary conditions ensuring stable ring-vortex solutions.
\end{itemize}

\subsection*{1. Governing Fluid Equations of the Æther}
\subsection*{1.1 Basic Postulates and Fields}
\begin{enumerate}
 \item Æther as a Non-Viscous, Incompressible Fluid\\
 We posit a constant mass density \(\rho_{\scriptscriptstyle \mathrm{Æ}}\) (possibly extremely large or small), zero viscosity \(\mu=0\), and no external body forces aside from internal vortex–pressure interactions.

 \item Velocity Field \(\mathbf{v}(\mathbf{x},t)\)\\
 Each point \(\mathbf{x}\) in Euclidean 3D space has a velocity \(\mathbf{v}(\mathbf{x},t)\). We impose
 \[
  \nabla \cdot \mathbf{v} \;=\; 0 \quad (\text{incompressibility}),
 \]
 ensuring no local volume changes.

 \item Absolute Time \(t\)\\
 Unlike relativity, we treat time as a universal parameter. All fluid elements evolve under a single \(t\).
\end{enumerate}

\subsection*{1.2 Euler’s Equation and Bernoulli Relation}
For an inviscid, incompressible fluid of density \(\rho_{\scriptscriptstyle \mathrm{Æ}}\), Euler’s equation in Cartesian coordinates is:
\[
 \frac{\partial \mathbf{v}}{\partial t} \;+\; (\mathbf{v}\cdot\nabla)\mathbf{v} \;=\; -\,\frac{1}{\rho_{\scriptscriptstyle \mathrm{Æ}}}\,\nabla p,
\]
where \(p(\mathbf{x},t)\) is the fluid pressure field.

From this, one can also derive the Bernoulli equation (valid along a streamline or, in steady/quasi-steady flows, throughout connected regions):
\[
 \frac{\partial}{\partial t} \Bigl( \tfrac12 \mathbf{v}^2 + \tfrac{p}{\rho_{\scriptscriptstyle \mathrm{Æ}}} \Bigr) \;+\; (\mathbf{v}\cdot\nabla) \Bigl( \tfrac12 \mathbf{v}^2 + \tfrac{p}{\rho_{\scriptscriptstyle \mathrm{Æ}}} \Bigr) \;=\; 0.
\]

\subsection*{1.3 Vorticity and Helmholtz’s Theorems}
Define the vorticity as
\[
 \boldsymbol{\omega} \;=\; \nabla \times \mathbf{v}.
\]
Its evolution is given by the vorticity transport equation:
\[
 \frac{\partial \boldsymbol{\omega}}{\partial t} \;=\; \nabla\times\bigl(\mathbf{v}\times\boldsymbol{\omega}\bigr).
\]
Since \(\mu=0\), we have no diffusion of vorticity; vortex lines (or filaments) are “frozen” into the fluid (Helmholtz’s 2nd theorem). The circulation \(\Gamma = \oint \mathbf{v}\cdot d\mathbf{r}\) around any closed loop moving with the fluid is conserved (Kelvin’s circulation theorem).

\section*{2. Action or Lagrangian Formulation (Optional)}
A useful starting point for a more unified approach is a Lagrangian (or Hamiltonian) for an incompressible fluid:
\[
 \mathcal{L} \;=\; \rho_{\scriptscriptstyle \mathrm{Æ}}\, \Bigl( \tfrac12\,\mathbf{v}^2 \;-\; \Phi\,(\nabla\cdot\mathbf{v}) \Bigr) \;-\; V(\boldsymbol{\omega}),
\]
where \(\Phi\) is a Lagrange multiplier enforcing \(\nabla\cdot \mathbf{v}=0\), and \(V(\boldsymbol{\omega})\) could encode possible self-energy or core energy contributions for strong vorticity near vortex filaments. By varying \(\mathcal{L}\) w.r.t. \(\mathbf{v}\), one recovers the standard incompressible Euler equations plus constraints.

\section*{3. Ring-Vortex (“Photon”) Solutions: Mathematical Setup}
\subsection*{3.1 Thin-Core Toroidal Vortex Ansatz}
A ring vortex in standard fluid theory is typically described by:
\begin{enumerate}
 \item Major radius \(R\): distance from ring center to the geometric center of the cross-section.
 \item Minor radius \(a\): radius of the vortex tube cross-section.
 \item Circulation \(\Gamma\): integral of tangential velocity around the cross-section.
\end{enumerate}
For a “thin” toroidal vortex (\(a \ll R\)), the translational speed \(U\) of the ring can be approximated by known formulas (from Biot–Savart integrals or filament models). For an unforced ring in a standard fluid, a leading-order expression is:
\[
 U \;\approx\; \frac{\Gamma}{4\pi R} \Bigl[\ln\!\bigl(\tfrac{8R}{a}\bigr) - \tfrac12\Bigr],
\]
but in VAM we want a ring traveling at constant speed \(c\). Hence, we might stipulate:
\[
 U(\Gamma, R, a) \;=\; c \quad (\text{postulate or boundary condition}).
\]
Thus, the ring adjusts \(\Gamma\) and the ratio \(R/a\) so that it moves at speed \(c\). We can interpret \(\Gamma\) as controlling the total energy (frequency) of the photon, while \(R\) might link to wavelength \(\lambda\).

\subsection*{3.2 Relation to Photon Frequency and Wavelength}
\begin{enumerate}
 \item Energy–Frequency Link\\
 In quantum-like terms, a photon has energy \(E = h\nu\). In your model, the ring vortex’s rotational speed or circulation sets \(\nu\). A simple ansatz:
 \[
  \nu \;\sim\; \frac{\Gamma}{2\pi\,a^2} \;=\; \frac{\Gamma}{\text{(vortex cross-section area)}},
 \]
 or another dimensionally consistent form. One can also choose:
 \[
  E \;=\; \tfrac12\,\rho_{\scriptscriptstyle \mathrm{Æ}} \,\Gamma^2\,f(\,R,a\,),
 \]
 reflecting the fluid’s kinetic energy stored in the ring.

 \item Wavelength \(\lambda\) vs. Radius \(R\)\\
 If the ring’s overall diameter is \(\sim \lambda\), we might set
 \[
  2\pi R \;\approx\; \lambda \;=\; \frac{c}{\nu}.
 \]
 Then the ring’s “major radius” correlates with the classical wave wavelength. This ensures the wave-like property: a photon of higher frequency \(\nu\) has a smaller ring radius \(R\).
\end{enumerate}

\subsection*{3.3 Stability Conditions}
A ring vortex is typically stable in an ideal fluid if there are no large perturbations. In VAM, one could incorporate special boundary conditions or an additional short-range “core pressure” term \(V(\boldsymbol{\omega})\) that ensures the ring does not self-distort. Mathematically, it may require solving the 3D Euler PDEs with an axisymmetric/toroidal velocity profile that remains steady and travels at speed \(c\) in, say, the \(+z\)-direction:

\[
 \mathbf{v}(r,\theta,\phi, t) \;=\; \mathbf{v}_{\text{ring}}\bigl(r - R,\,a\bigr) \;+\; c\,\hat{\mathbf{z}} \quad (\text{in a co-moving frame or wave-like ansatz}).
\]

While fully exact solutions are notoriously complicated, approximate filament models plus boundary conditions can yield physically intuitive ring solutions traveling at a designated speed.

\section*{4. Wave–Particle Duality in the Æther: Mathematical Hints}
\subsection*{4.1 Superposition of Ring Vortices}
In linear wave theory, we superimpose solutions to get interference patterns. For ring vortices, superposition is not trivially linear—vortices interact via the Biot–Savart law. However, for low-amplitude or well-separated photons, one might approximate the total velocity as a sum of each ring’s velocity field:

\[
 \mathbf{v}_{\text{total}} \;\approx\; \sum_{n}\,\mathbf{v}_{(\text{ring } n)},
\]

so that we can get constructive or destructive interference patterns in the fluid velocity or pressure if many photons overlap. This recasts the wave–particle duality:

\begin{itemize}
 \item Particle aspect: Each ring vortex is a topologically protected structure with quantized circulation.
 \item Wave aspect: The velocity fields from different rings can interfere, creating typical wave patterns in detection events.
\end{itemize}

\subsection*{4.2 Polarization}
A ring vortex can carry orbital or spin-like angular momentum about its axis. Polarization states can be interpreted as different twisting or orientation of the vortex cross-section. One approach is to parametrize the velocity in cylindrical or toroidal coordinates \((r,\phi,\theta)\) so that a change in phase angle of the core swirl corresponds to linear vs. circular vs. elliptical polarization.

\section*{5. Tying the Speed \(c\) to Fluid Constants}
In your VAM, you have introduced fundamental constants such as:

\begin{itemize}
 \item \(c \approx 3\times10^8\,\mathrm{m/s}\) (light speed),
 \item \(C_e\) (the “vortex angular-velocity constant,” \(\sim 1.09\times10^6\,\mathrm{m/s}\)),
 \item \(F_{\max} \approx 29\,\mathrm{N}\) (maximum force),
 \item \(\rho_{\scriptscriptstyle\mathrm{Æ}}\) or some effective parameters.
\end{itemize}

Dimensional analysis is essential. If \(\rho_{\scriptscriptstyle \mathrm{Æ}}\) and \(C_e\) set a characteristic wave speed, one might define:

\[
 c \;\stackrel{?}{=}\; \sqrt{ \frac{F_{\max}}{\rho_{\scriptscriptstyle \mathrm{Æ}}\,(\text{length scale})} } \;\;\;\text{or}\;\;\; c \;=\; C_e \;\times\; \bigl(\text{dimensionless ratio of energies}\bigr).
\]

Within your model, you can arrange a “stiffness” or “tension” in the fluid that forces any ring vortex to propagate at \(c\). Physically, it’s reminiscent of how the speed of sound in a medium is \(\sqrt{K/\rho}\), where \(K\) is the bulk modulus. In an Æther specialized to produce electromagnetic phenomena, \(c\) emerges as the “wave speed” of certain transverse vortex excitations.

\section*{6. Example: Photon–Hydrogen Emission}
When an electron in hydrogen transitions from an excited state \(n_2\) to \(n_1\), it emits a photon of frequency:

\[
 \nu \;=\; \frac{E_{n_2}-E_{n_1}}{h} \;\approx\; \frac{13.6\,\mathrm{eV}}{h} \Bigl(\tfrac{1}{n_1^2}-\tfrac{1}{n_2^2}\Bigr).
\]

In VAM, we interpret this as the nucleus–electron vortex system shedding a ring vortex whose circulation or radius matches that frequency. Symbolically:

\begin{enumerate}
 \item Vortex radius \(R \sim c/\nu\).
 \item Energy \(\sim \tfrac12 \rho_{\scriptscriptstyle\mathrm{Æ}}\,\Gamma^2\,g(R,a)\).
\end{enumerate}

One can try to match \(\Gamma\) and \(R\) so that the ring vortex’s total fluid energy equals \(\hbar\,\nu\). This recasts quantum transitions as topological transitions in the atomic vortex structure that emit ring vortices consistent with the line spectrum.

\section*{7. Including Gravitation: Vorticity and Pressure Gradients}
\subsection*{7.1 Large-Scale Flow and “Gravity”}
You propose that macroscale gravitational fields are equivalent to certain vorticity distributions or pressure gradients in the Æther. That is, rather than curved spacetime, we have:

\[
 -\nabla p \;=\; \rho_{\scriptscriptstyle \mathrm{Æ}}\, \frac{d\mathbf{v}}{dt} \quad\longleftrightarrow\quad \text{Newton’s } \mathbf{F}=-\,GMm/r^2\;\text{analogue}.
\]

One can attempt an effective potential \(\Phi(\mathbf{x})\) that solves

\[
 \nabla^2 \Phi \;=\; 4\pi\,G_{\text{eff}}\,\rho_{\text{mass-like}},
\]

where “mass-like” sources might be identified with regions of intense, swirling fluid or vortex complexes. Photons (ring vortices) then bend around these regions, producing lensing and gravitational redshifts.

\subsection*{7.2 Photon Vortex in a Gravitational Field}
If ring vortices move in a background velocity field \(\mathbf{v}_{\mathrm{grav}}\) or pressure gradient, one can include an effective path equation from the fluid’s perspective:

\[
 \frac{d^2 \mathbf{r}}{dt^2} \;=\; -\,\nabla \Phi(\boldsymbol{\omega}_{\mathrm{ext}}).
\]

Hence, we recover phenomena akin to general-relativistic deflection. The key difference is that the curvature is reinterpreted as a fluid phenomenon rather than geometry.

\section*{8. Summary Flowchart for VAM Photon Math}
\begin{enumerate}
 \item Fluid PDEs: Start from incompressible Euler + boundary conditions.
 \item Torus (Ring) Vortex: Assume a localized axisymmetric solution traveling at speed \(c\).
 \item Energy–Circulation Relationship: Relate the ring’s circulation to photon frequency \(\nu\).
 \item Radius–Wavelength Relationship: Impose \(2\pi R \approx \lambda = c/\nu\).
 \item Stability \& Polarization: Describe ring cross-section swirl to define polarization states.
 \item Superposition (approx.): Interference of multiple ring vortices recovers wave-like patterns.
 \item Gravity: Large-scale vorticity or pressure gradient forms an effective potential that deflects ring vortex paths \(\rightarrow\) lensing, redshift.
 \item Experimental Link: Match the ring-vortex energy to known quantum transitions (e.g., hydrogen lines) to show consistency with the Rydberg formula, etc.
\end{enumerate}

\section*{9. Concluding Remarks}
Mathematically, ring-vortex “photons” in the Vortex Æther Model can be constructed by:

\begin{enumerate}
 \item Solving or approximating the 3D incompressible Euler system for a stable, traveling toroidal vortex at speed \(c\).
 \item Identifying energy \(\leftrightarrow\) circulation \(\leftrightarrow\) frequency relationships.
 \item Interpreting wave–particle duality as the superposition of localized vortex structures whose velocity fields can interfere.
 \item Incorporating additional topological or core-energy terms in a Lagrangian if needed for exact stability.
 \item Embedding gravitational phenomena into large-scale vorticity or pressure fields so that ring vortices experience deflection and redshift akin to standard gravitational physics.
\end{enumerate}

This approach weaves classical fluid PDEs, topological constraints (knotted or unknotted vortex filaments), and dimensional analysis (tying in constants \(c\), \(C_e\), \(F_{\max}\), etc.) into a single picture. While challenging to solve in detail, the framework provides a clear mathematical route for bridging your physical interpretations—photon as a vortex dipole, wave–particle unification, gravitational coupling—within a self-consistent fluidic (Æther) paradigm.
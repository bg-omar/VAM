\section{Advanced Knot-Theoretic Extensions for Vortex Particle Modeling}

This section introduces five knot-theoretic mechanisms to enrich the physical formalism and predictive capabilities of the Vortex Æther Model (VAM).

\subsection{Energy-Minimizing Embeddings and Particle Mass Quantization}

Let $K$ denote a vortex filament knot embedded in the æther medium. Its kinetic energy is defined by:

\begin{equation}
    E_K = \frac{1}{2} \rho \int_{\mathbb{R}^3} |\vec{v}|^2 \, dV,
\end{equation}
where $\vec{v}$ is the Biot–Savart induced velocity from the filament curve $\Gamma$:
\begin{equation}
    \vec{v}(\vec{x}) = \frac{\Gamma}{4\pi} \int_{K} \frac{(\vec{x} - \vec{x}') \times d\vec{x}'}{|\vec{x} - \vec{x}'|^3}.
\end{equation}

To regularize the energy near singularities, we consider Möbius energy \cite{freedman1994moebius}:
\begin{equation}
    E_\text{M\"obius}(K) = \iint_{K \times K} \left( \frac{1}{|\vec{x} - \vec{y}|^2} - \frac{1}{D^2(\vec{x}, \vec{y})} \right) d\vec{x} d\vec{y}.
\end{equation}

These energy levels yield a discrete mass hierarchy for particle families.

\subsection{Cobordism and Particle Transitions}

Let $K_1$ and $K_2$ be two knotted vortex states. A cobordism $\Sigma$ exists between them if:

\begin{equation}
    \partial \Sigma = K_1 \cup -K_2,
\end{equation}
describing a topological transformation modeling a particle decay or transition \cite{carter1998surfaces}.

For muon decay:
\begin{equation}
    K_{\mu^-} \rightsquigarrow K_{e^-} \cup K_{\nu_\mu} \cup K_{\bar{\nu}_e},
\end{equation}
with conservation of helicity:
\begin{equation}
    \mathcal{H}(K_{\mu^-}) = \sum_i \mathcal{H}(K_i) + \mathcal{H}_\text{radiation}.
\end{equation}

\subsection{Knot Invariants as Quantum Numbers}

We propose mapping Jones polynomial structure to physical quantum numbers \cite{kauffman2001knots}. For the trefoil knot:

\begin{equation}
    V_{3_1}(q) = q + q^3 - q^4.
\end{equation}

Define:
\begin{align}
    \text{Charge}(K) &= \alpha \cdot \left[ \deg_+ V_K - \deg_- V_K \right], \\
    \text{Generation}(K) &= \beta \cdot \deg(P_K).
\end{align}

This gives a route to encode Standard Model charges, generations, and even parity within knot invariants.

\subsection{Braids as a Non-Abelian Interaction Algebra}

Knots are closures of braid words in the braid group $B_n$ with generators $\sigma_i$ satisfying:

\begin{align}
    \sigma_i \sigma_{i+1} \sigma_i &= \sigma_{i+1} \sigma_i \sigma_{i+1}, \\
    \sigma_i \sigma_j &= \sigma_j \sigma_i \quad \text{for } |i-j| > 1.
\end{align}

Gauge interactions are modeled by braid manipulations:
\[
    q_R \xleftrightarrow{\sigma} q_G \Rightarrow \text{gluon exchange},
\]
providing a non-Abelian algebra for chromodynamic reconnections \cite{birman1975braids}.

\subsection{Knot Flow Evolution as a Vortex Field Theory}

We define time evolution of a knot via curvature-driven gradient flow:

\begin{equation}
    \frac{\partial K}{\partial t} = - \nabla E(K),
\end{equation}
where $E(K)$ may be bending energy $E_b$:
\begin{equation}
    E_b = \int_K \kappa^2(s) \, ds,
\end{equation}
with $\kappa(s)$ the local curvature. Phase transitions are mediated by reconnection dynamics over this energy landscape \cite{moffatt1990dynamics}.

A conserved topological charge may be defined by:
\begin{equation}
    Q_\text{top} = \frac{1}{2\pi} \int \epsilon^{ijk} A_i \partial_j A_k \, d^3x.
\end{equation}

This opens a path to a topological quantum field theory analog for VAM.
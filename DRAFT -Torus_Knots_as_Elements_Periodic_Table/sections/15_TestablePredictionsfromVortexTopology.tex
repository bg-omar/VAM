\section{Testable Predictions from Vortex Topology}

A core objective of the VAM is to transition from a descriptive to a predictive framework. Below are proposed experimental predictions derived from vortex topology and helicity quantization.

\subsection{New Particle States from Knot Invariants}

By systematically exploring knots with higher linking numbers and unique polynomial invariants, we can forecast the existence of yet-undetected particles:

\begin{itemize}
    \item \textbf{Fourth-generation lepton-like knot:} A stable knot with $Lk=15$–$17$ and non-trivial Jones polynomial may correspond to a new family of leptons with mass $\sim$ 10–100 GeV.
    \item \textbf{Sterile Neutrino Candidate:} A twist-free toroidal ring ($Lk=0$) with zero writhe may serve as a sterile mode with no SM interactions.
    \item \textbf{Exotic Mesons or Glueballs:} Composite braids with entangled colored knots may form bound states mimicking glueball configurations.
    \item \textbf{Dark Matter Knots:} Highly stable, non-reconnecting vortex solitons may correspond to dark sector particles.
\end{itemize}

\subsection{Decay Channel Predictions}

Reconnection rules can imply distinct decay modes:
\begin{itemize}
    \item Muon knot $\rightarrow$ Electron + Spiral (neutrino)
    \item Exotic trefoil link $\rightarrow$ Two knotted photons (twist waves)
    \item High-twist knot with $Lk=9$ $\rightarrow$ Trefoil ($Lk=3$) + Hexafoil ($Lk=6$): analog to proton decay.
\end{itemize}

Each channel is testable via energy spectra, knot recoil dynamics, and helicity flow, offering falsifiability.

\subsection{LENR and Æther Vortex Recombination}

As proposed in the VAM LENR hypothesis, localized helicity annihilation may yield unexpected heat release. Suggested experimental scenarios:
\begin{itemize}
    \item Metallic lattice infused with Æther-rich knots (e.g., Pd-H systems)
    \item Applied EM resonance to induce controlled reconnections
    \item Measurement of net helicity decrease $\Delta H < 0$ via vorticity-sensitive interferometry
\end{itemize}

\subsection{Helicity Spectroscopy}

We propose a new experimental observable: helicity spectroscopy, measuring frequency shifts in torsional vortex waves emitted by knotted particles under stress.

\begin{equation}
    \Delta \nu = \alpha_H \cdot Lk \, ,
\end{equation}
where $\alpha_H$ is a material-specific helicity coefficient. Sharp spectral peaks would correspond to quantized $Lk$ values and may be used to confirm predicted vortex types.

\subsection{Summary of Implications}
The Vortex Æther Model (VAM) redefines gravity, quantum mechanics, and nuclear interactions as emergent from Æther vorticity. Our derivations using $C_e, \rho_{\æ}, F_{\max}, \kappa, r_c$ predict:

\begin{itemize}
    \item \textbf{Gravitational Modulation:} Weight changes from induced vortex pressure gradients. Experiments can manipulate time flow and gravity via EM means.
    \item \textbf{FTL Communication:} Superluminal signals along vortex filaments, supported by æther wave velocities far exceeding $c$.
    \item \textbf{LENR Triggering:} Nuclear reactions catalyzed by Æther vortex overlap and resonant excitation instead of brute-force fusion.
\end{itemize}

Each phenomenon is achievable using the same underlying technology: vortex control through magnetic, electric, and mechanical excitation. This paves the way for a new engineering paradigm — \textit{Vortex Engineering} — with applications from propulsion and energy to communications and quantum state control.

\subsection*{Sources}
\begin{itemize}
    \item Iskandarani, O. (2025). \textit{The Vortex Æther Model: Æther Vortex Field Model}.
    \item Iskandarani, O. (2025). \textit{A Unified Vorticity Framework for Gravity, Electromagnetism, and Quantum Phenomena}.
    \item Podkletnov, E.E. (1997). \textit{Weak gravitation shielding properties of composite YBa$_2$Cu$_3$O$_{7-x}$ superconductor}.
    \item Rabounski, D. \& Borissova, L. (2007). \textit{A Theory of the Podkletnov Effect Based on GR: Anti-Gravity Force due to Perturbed Non-Holonomic Background of Space}. Progress in Physics 3.
\end{itemize}
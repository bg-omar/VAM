%======================== vam_canon_v5.tex ========================
\documentclass[11pt,a4paper]{article}

\usepackage[margin=1in]{geometry}
\usepackage{amsmath,amssymb,mathtools}
\usepackage{bm}
\usepackage{siunitx}
\usepackage{booktabs}
\usepackage{hyperref}
\usepackage{physics}
\usepackage{microtype}

\hypersetup{colorlinks=true,linkcolor=blue,citecolor=blue,urlcolor=blue}
\sisetup{separate-uncertainty=true, per-mode=symbol, detect-all}

\newcommand{\aeether}{\text{\ae}} % æ ligature in text mode

\title{\bfseries VAM Canon (v5)}
\author{Prepared for: Omar Iskandarani (Vortex \aeether{}ther Model, VAM)}
\date{2025-08-22}

\begin{document}
    \maketitle
    \tableofcontents

    \section*{0) Versioning}
    \addcontentsline{toc}{section}{0) Versioning}
    \begin{itemize}
        \item \textbf{Single source of truth} for core VAM definitions, constants, master equations, and notational conventions.
        \item Semantic versions: \texttt{vMAJOR.MINOR.PATCH} (e.g., v5.0.0).
        \item Every paper/derivation must state the Canon version it depends on.
        \item \emph{What’s new in v5}: (i) dimensionally-correct time-rate law; (ii) PG-type effective metric for azimuthal swirl \cite{Unruh1981,Visser1998,Painleve1921,Gullstrand1922}; (iii) swirl Hamiltonian density with Kelvin stiffening; (iv) dark-sector pressure/acceleration law; (v) boxed identities for $\alpha$, $\alpha_g$, and equivalent $G$.
    \end{itemize}

    \section{Core Postulates (VAM)}
    \begin{enumerate}
        \item The universe is a \textbf{3D incompressible, inviscid superfluid \aeether{}ther} with absolute time and Euclidean space.
        \item \textbf{Particles are knotted vortex solitons} (closed, possibly linked/knotted filaments) in the \aeether{}ther.
        \item \textbf{Gravity} is not spacetime curvature but \textbf{swirl} (structured vorticity fields) and pressure gradients; massive motion follows swirl-induced dynamics.
        \item \textbf{Local time-rate} is set by \textbf{tangential vortex motion}: higher swirl reduces the local clock rate relative to asymptotic time.
        \item \textbf{Quantization} arises from \textbf{topological invariants} (linking, writhe, twist) and circulation quantization of vortex filaments.
        \item The \aeether{}ther supports \textbf{bosonic unknotted excitations} (photon-like), while chiral hyperbolic knots map to quarks; torus knots map to leptons, etc. (taxonomy documented separately).
    \end{enumerate}

    \section{Canonical Constants and Symbols}
    All symbols are SI unless stated. Derived constants are presented by identity with their numerical evaluation.
    \subsection{Fundamental (VAM-specific)}
    \begin{itemize}
        \item Vortex tangential velocity: $C_e = 1.09384563\times 10^{6}\ \si{m.s^{-1}}$.
        \item Vortex-core radius: $r_c = 1.40897017\times 10^{-15}\ \si{m}$.
        \item \aeether{}ther fluid density (``vacuum'' fluid): $\rho_{\aeether}^{(\mathrm{fluid})} = 7.0\times 10^{-7}\ \si{kg.m^{-3}}$.
        \item \aeether{}ther core/mass density: $\rho_{\aeether}^{(\mathrm{mass})} = 3.8934358266918687\times 10^{18}\ \si{kg.m^{-3}}$.
        \item \aeether{}ther energy density: $\rho_{\aeether}^{(\mathrm{energy})} = 3.49924562\times 10^{35}\ \si{J.m^{-3}}$.
        \item Maximum Coulomb force (VAM): $F_{\aeether}^{\max} = 29.053507\ \si{N}$.
        \item Maximum universal force (contextual): $F_{\mathrm{gr}}^{\max} = 3.02563\times 10^{43}\ \si{N}$.
        \item Golden ratio: $\varphi = \frac{1+\sqrt{5}}{2} \approx 1.61803398875$.
    \end{itemize}
    \subsection{Universal}
    \begin{itemize}
        \item Speed of light: $c = 299{,}792{,}458\ \si{m.s^{-1}}$.
        \item Planck time: $t_p = 5.391247\times 10^{-44}\ \si{s}$.
        \item \textbf{Fine-structure constant (derived):}
        \[
            \boxed{\ \alpha = \frac{2C_e}{c}\ } \;=\; 7.2973525643\times 10^{-3}.
        \]
    \end{itemize}

    \section{Master Equations (boxed, definitive)}
    \subsection{Master Energy and Mass Formula}
    Define the amplified swirl energy for a coherent VAM volume $V$:
    \[
        \boxed{\ E_{\mathrm{VAM}}(V) \;=\; \frac{4}{\alpha\,\varphi}\,\left(\tfrac{1}{2}\,\rho_{\aeether}^{(\mathrm{fluid})}\,C_e^{2}\right)\,V\ }\quad [\si{J}]\,.
    \]
    Corresponding mass (strict SI mass):
    \[
        \boxed{\ M_{\mathrm{VAM}}(V) \;=\; \frac{E_{\mathrm{VAM}}(V)}{c^{2}} \ }\quad [\si{kg}]\,.
    \]
    \noindent\textit{Numerical (per unit volume):} $\tfrac{1}{2}\rho_{\aeether}^{(\mathrm{fluid})}C_e^2 \approx 4.1877439\times 10^{5}\ \si{J.m^{-3}}$, $\frac{4}{\alpha\varphi} \approx 3.3877162\times 10^{2}$, so $\frac{E_{\mathrm{VAM}}}{V} \approx 1.418688\times 10^{8}\ \si{J.m^{-3}}$, $\frac{M_{\mathrm{VAM}}}{V} \approx 1.57850\times 10^{-9}\ \si{kg.m^{-3}}$.

    \subsection{Gravitational coupling from swirl}
    \[
        \boxed{\ G_{\mathrm{swirl}} \;=\; \frac{C_e\,c^{5}\,t_p^{2}}{2\,F_{\aeether}^{\max}\,r_c^{2}}\ }\quad (F_{\aeether}^{\max}=29.053507\ \si{N})
    \]
    \noindent\textit{Numerical:} $G_{\mathrm{swirl}} \approx 6.674302\times 10^{-11}\ \si{m^{3}.kg^{-1}.s^{-2}}$.

    \paragraph{3.2a Derived couplings \& identities.}
    \[
        \boxed{\ \alpha = \frac{2 C_e}{c}\ }\,,\qquad \boxed{\ \alpha_g = \frac{C_e^{2} t_p^{2}}{r_c^{2}}\ }\,.
    \]
    With $\ell_p\equiv c\,t_p$ and $ \boxed{\,M_e = \tfrac{2 F_{\aeether}^{\max} r_c}{c^{2}}\,}$,
    \[
        \boxed{\ G = \frac{\alpha_g c^{3} r_c}{C_e M_e} = \frac{C_e c\, \ell_p^{2}}{r_c M_e}\ }\;\equiv\; G_{\mathrm{swirl}}\,.
    \]

    \subsection{Local Time-Rate (swirl clock)}
    \[
        \boxed{\ dt_{\mathrm{local}} \;=\; dt_{\infty}\,\sqrt{\,1 - \frac{\lVert\bm{\omega}\rVert^{2}\,r_c^{2}}{c^{2}}\,}\ }
        \;=\; dt_{\infty}\,\sqrt{1-\frac{v_t^2}{c^2}},\qquad v_t:=\lVert\bm{\omega}\rVert\,r_c\,.
    \]
    \noindent\emph{Historical (deprecated; kept for traceability):} $dt_{\mathrm{local}} = dt_{\infty}\sqrt{1-\lVert\bm{\omega}\rVert^{2}/c^{2}}$.

    \subsection{Swirl Angular Frequency Profile}
    \[
        \boxed{\ \Omega_{\mathrm{swirl}}(r) \;=\; \frac{C_e}{r_c}\,e^{-r/r_c}\ }\quad\Rightarrow\quad \Omega_{\mathrm{swirl}}(0)=\frac{C_e}{r_c}\approx 7.76344\times 10^{20}\ \si{s^{-1}}.
    \]

    \subsection{Vorticity Potential (canonical form)}
    \[
        \boxed{\ \Phi(\bm r,\bm\omega) \;=\; \frac{C_e^{2}}{2\,F_{\aeether}^{\max}}\,\bm\omega\!\cdot\!\bm r\ }\,.
    \]
    \noindent\emph{Dimensional remark:} $\rho\,\Phi$ carries energy density units; use consistently in the Lagrangian.

    \subsection{Effective metric / line element (axisymmetric swirl)}
    Steady, incompressible, azimuthal drift $v_\theta(r)$ in cylindrical $(t,r,\theta,z)$ (PG-type analogue metric) \cite{Unruh1981,Visser1998,Painleve1921,Gullstrand1922}:
    \[
        \boxed{\, ds^2 = -(c^2 - v_\theta(r)^2)\,dt^2 \;+\; 2\,v_\theta(r)\,r\,d\theta\,dt \;+\; dr^2 \;+\; r^2 d\theta^2 \;+\; dz^2 \,}
    \]
    Co-rotating local diagonalization ($\theta' = \theta - \int v_\theta(r)\,dt/ r$):
    \[
        \boxed{\, ds^2 = -\,c^2\Big(1-\tfrac{v_\theta(r)^2}{c^2}\Big)dt^2 \;+\; dr^2 \;+\; r^2 d\theta'^2 \;+\; dz^2 \,}
    \]

    \subsection{Dark-sector law paired with $v(r)$}
    Radial Euler balance (steady, no radial flow):
    \[
        0 = -\,\frac{1}{\rho}\frac{dp_{\mathrm{swirl}}}{dr} + \frac{v(r)^2}{r}
        \quad\Rightarrow\quad
        \boxed{\, a_{\mathrm{dark}}(r) = \frac{1}{\rho}\frac{dp_{\mathrm{swirl}}}{dr} = \frac{v(r)^2}{r} \,}.
    \]
    Equivalently as a pressure law paired with the swirl profile:
    \[
        \boxed{\, \frac{dp_{\mathrm{swirl}}}{dr} = \rho\,\frac{v(r)^2}{r} \,}
        \quad\Rightarrow\quad
        \boxed{\, p_{\mathrm{swirl}}(r) = p_0 + \rho\,v_0^2\,\ln\!\Big(\frac{r}{r_0}\Big) \,}\quad (\text{flat } v\to v_0).
    \]

    \subsection{Galactic swirl law (disc kinematics)}
    \[
        \boxed{\, v(r) = \frac{C_{\mathrm{core}}}{\sqrt{1 + (r_c/r)^2}} + C_{\mathrm{tail}}\,\big(1 - e^{-r/r_c}\big) \,}.
    \]

    \section{Unified VAM Lagrangian (definitive form)}
    Let $\bm v$ be the \aeether{}ther velocity, $\rho=\rho_{\aeether}^{(\mathrm{fluid})}$ constant (incompressible), $\bm\omega=\nabla\times\bm v$, and $\lambda$ a Lagrange multiplier enforcing incompressibility.
    \[
        \boxed{\, \mathcal{L}_{\mathrm{VAM}} =
            \underbrace{\tfrac{1}{2}\rho\,\lVert\bm v\rVert^{2}}_{\text{kinetic}}
            - \underbrace{\rho\,\Phi(\bm r,\bm\omega)}_{\text{swirl potential}}
            + \underbrace{\lambda(\nabla\!\cdot\!\bm v)}_{\text{incompressibility}}
            + \underbrace{\eta\,\mathcal{H}[\bm v]}_{\text{helicity/topological term}}
            + \underbrace{\mathcal{L}_{\mathrm{couple}}[\Gamma,\mathcal{K}]}_{\text{circulation \& knot invariants}} \, }.
    \]
    Here $\mathcal{H}[\bm v] = \int (\bm v\cdot\bm\omega)\,dV$ is kinetic helicity (generator of topological constraints; $\eta$ fixes units). Enforce $\nabla\cdot\bm v=0$ and proper boundary terms for closed filaments.

    \subsection*{4.1 Swirl Hamiltonian density (dimensionally normalized)}
    \addcontentsline{toc}{subsection}{4.1 Swirl Hamiltonian density}
    Take $\rho=\rho_{\aeether}^{(\mathrm{fluid})}$, $\bm\omega=\nabla\times\bm v$, with $\lambda$ enforcing incompressibility. A quadratic, Kelvin-compatible kernel:
    \[
        \boxed{\, \mathcal H[\bm v] \;=\; \tfrac12\,\rho\,\lVert\bm v\rVert^2 \;+\; \tfrac12\,\rho\,\ell_\omega^{2}\,\lVert\bm\omega\rVert^{2} \;+\; \tfrac12\,\rho\,\ell_\omega^{4}\,\lVert\nabla\bm\omega\rVert^{2} \;+\; \lambda\,(\nabla\!\cdot\!\bm v) \,},\quad \ell_\omega:=r_c\,.
    \]
    \noindent\textit{Units:} $[\rho\lVert v\rVert^2]=\si{J.m^{-3}}$; since $[\omega]=\si{s^{-1}}$, the coefficients $\rho\,\ell_\omega^{2}$ and $\rho\,\ell_\omega^{4}$ ensure the $|\omega|^{2}$ and $|\nabla\omega|^{2}$ terms are also energy densities.

    \paragraph{Optional minimal matter--swirl coupling.}
    \[
        \boxed{\, \mathcal H_\psi \;=\; \tfrac{\hbar^2}{2m}\,\big\lVert(\nabla - i\,\tfrac{m}{\hbar}\,\bm A_{\mathrm{swirl}})\psi\big\rVert^2 + U(|\psi|^2) \,},\quad \bm A_{\mathrm{swirl}}:=\chi\,\bm v.
    \]

    \section{Appendix: Canon Tables (paste-ready)}
    \subsection{Constants Table}
    \begin{center}
        \begin{tabular}{@{}llll@{}}
            \toprule
            \textbf{Symbol} & \textbf{Meaning} & \textbf{Value} & \textbf{Unit} \\
            \midrule
            $C_e$ & Vortex tangential velocity & $1.09384563\times 10^{6}$ & $\si{m.s^{-1}}$ \\
            $r_c$ & Vortex-core radius & $1.40897017\times 10^{-15}$ & $\si{m}$ \\
            $\rho_{\aeether}^{(\mathrm{fluid})}$  & \aeether{}ther fluid density & $7.0\times 10^{-7}$ & $\si{kg.m^{-3}}$ \\
            $\rho_{\aeether}^{(\mathrm{mass})}$   & \aeether{}ther mass density & $3.8934358266918687\times 10^{18}$ & $\si{kg.m^{-3}}$ \\
            $\rho_{\aeether}^{(\mathrm{energy})}$ & \aeether{}ther energy density & $3.49924562\times 10^{35}$ & $\si{J.m^{-3}}$ \\
            $F_{\aeether}^{\max}$ & Max. Coulomb force & $29.053507$ & $\si{N}$ \\
            $F_{\mathrm{gr}}^{\max}$ & Max. universal force & $3.02563\times 10^{43}$ & $\si{N}$ \\
            $\alpha$ & Fine-structure (derived) & $2C_e/c$ ($=7.2973525643\times 10^{-3}$) & --- \\
            $\varphi$ & Golden ratio & $1.61803398875$ & --- \\
            $c$ & Speed of light & $299\,792\,458$ & $\si{m.s^{-1}}$ \\
            $t_p$ & Planck time & $5.391247\times 10^{-44}$ & $\si{s}$ \\
            \bottomrule
        \end{tabular}
    \end{center}

    \subsection{Boxed Canon Equations}
    \begin{enumerate}
        \item $\boxed{E_{\mathrm{VAM}} = \dfrac{4}{\alpha\varphi}\left(\dfrac{1}{2}\rho C_e^2\right)V}$
        \item $\boxed{M_{\mathrm{VAM}} = E_{\mathrm{VAM}}/c^2}$
        \item $\boxed{G_{\mathrm{swirl}} = \dfrac{C_e c^5 t_p^2}{2F_{\aeether}^{\max} r_c^2}}$
        \item $\boxed{dt_{\mathrm{local}} = dt_{\infty}\sqrt{1-\lVert\omega\rVert^{2} r_c^{2}/c^2}}$
        \item $\boxed{\Omega_{\mathrm{swirl}}(r) = \dfrac{C_e}{r_c}e^{-r/r_c}}$
        \item $\boxed{ds^2 = -(c^2 - v_\theta^2)dt^2 + 2 v_\theta r\,d\theta\,dt + dr^2 + r^2 d\theta^2 + dz^2}$
        \item $\boxed{a_{\mathrm{dark}}(r)=v(r)^2/r}$\quad\emph{equiv.}\quad$\boxed{dp_{\mathrm{swirl}}/dr=\rho\,v(r)^2/r}$
    \end{enumerate}

    \section*{Change Log}
    \addcontentsline{toc}{section}{Change Log}
    \begin{itemize}
        \item \textbf{v5 (2025-08-22):} Promoted PG-type effective metric, dimensionally normalized swirl Hamiltonian density, and dark-sector pressure/acceleration law. Added derived identities for $\alpha$, $\alpha_g$, and equivalent forms of $G$. Corrected time-rate law. Consolidated galactic swirl profile.
    \end{itemize}



%===================== v0.6 Delta — Conclusions from VAM 16–20 =====================

    \section*{v0.6 Delta — Conclusions from VAM 16–20}
    \label{sec:v06-delta}

    \paragraph{Scope.}
    We consolidated the main outcomes of VAM-16–20 (Zero-Vorticity Line, photon/EM mapping, Kerr reinterpretation, vortex-string EFT, and Schr\"odinger hydrogen). Below are the boxed identities and consistency results that are ready for canonicalization (with dimensional and numerical checks). Items needing a policy decision are explicitly flagged.

%-------------------------------------------------------------
    \subsection*{C1) EM coupling emerges from core swirl pressure}
    \label{subsec:coulomb-from-swirl}

    Define the \emph{swirl Coulomb constant} via the pressure integral over a spherical surface:
    \[
        \boxed{ \Lambda \;\equiv\; \int_{S_r^2} p_{\text{swirl}}\,r^2\,d\Omega
        \;=\; 4\pi\,\rho_{\text{\ae}}^{(\text{mass})}\,C_e^2\,r_c^4 }
    \]
    Dimensions: $[\Lambda]=\mathrm{N\,m^2}=\mathrm{J\,m}$ (Coulomb constant units).
    \emph{Identification:}
    \[
        \boxed{ \Lambda \;=\; \frac{e^2}{4\pi\varepsilon_0} } \qquad \text{(EM coupling).}
    \]
    \textbf{Numerics (Canon values):} $\Lambda=2.30707733\times10^{-28}\,\mathrm{J\,m}$, matching $e^2/(4\pi\varepsilon_0)$ to $\lesssim10^{-7}$ relative.
    This promotes a tight constraint among $(\rho_{\text{\ae}}^{(\text{mass})},\,C_e,\,r_c)$ consistent with $\alpha = 2C_e/c$.
    \textbf{Status:} \emph{Promote} as Canon identity in \S3B and append $\Lambda$ to the constants table.

    \textit{Non-original comparison: Coulomb constant \cite{Jackson1999}.}

%-------------------------------------------------------------
    \subsection*{C2) Hydrogen Schr\"odinger equation with core softening}
    \label{subsec:hydrogen-soft-core}

    VAM replaces the Coulomb term by a swirl-induced softened potential
    \[
        \boxed{ V_{\text{VAM}}(r) \;=\; -\,\frac{\Lambda}{\sqrt{r^2+r_c^2}}
        \;\;\xrightarrow{r\gg r_c}\;\; -\frac{\Lambda}{r} }
    \]
    and the bound-state equation
    \[
        \boxed{ \left[-\frac{\hbar^2}{2\mu}\nabla^2 - \frac{\Lambda}{\sqrt{r^2+r_c^2}}\right]\psi \;=\; E\,\psi }
    \]
    recovers the textbook spectrum for $r\gg r_c$.
    \textbf{Bohr/Rydberg checks (H):}
    $a_0=\hbar^2/(\mu\Lambda)=5.29177262\times10^{-11}\,\mathrm{m}$,
    $E_{1}=\mu\Lambda^2/(2\hbar^2)=13.60569\,\mathrm{eV}$.
    Core softening gives $nS$ shifts $\sim \mathcal{O}\!\big((r_c/a_0)^2\big)\approx 7.1\times10^{-10}$ (H), and
    $\sim \mathcal{O}(2.4\times10^{-5})$ for muonic H (using $\mu\approx 186\,m_e$),
    i.e. a ground-state scale $\sim 6\times10^{-2}$\,eV --- a concrete experimental target.

    \textit{Non-original equations: Schr\"odinger hydrogen and Coulomb limit \cite{Schrodinger1926,Jackson1999}.}

%-------------------------------------------------------------
    \subsection*{C3) Frame-dragging / off-diagonal metric term as circulation}
    \label{subsec:metric-circulation}

    The PG-type analogue line element already in Canon (\S\,G.1) implies the mixed term
    \[
        \boxed{ g_{t\theta}^{\text{(VAM)}} \;=\; v_\theta(r)\,r \;=\; \frac{1}{2\pi}\,\Gamma_{\text{swirl}}(r) }
    \]
    with $\Gamma_{\text{swirl}}(r)=\oint v_\theta\,dl$ the azimuthal circulation at radius $r$.
    This is the precise VAM counterpart of GR's $g_{t\phi}$ frame-dragging structure for axisymmetric rotation, dovetailing with the Kerr reinterpretation draft.\footnote{In BL-like gauges one can factor $c^2$ to yield a dimensionless coefficient; the PG form used in Canon keeps $g_{t\theta}$ in velocity\,$\times$\,length units, matching the analogue-gravity construction.}
    \textbf{Status:} \emph{Promote} the boxed relation as a corollary to \S\,G.1.

    \textit{Non-original background: analogue/PG metrics and Kerr solution \cite{Painleve1921,Gullstrand1922,Unruh1981,Visser1998,Kerr1963}.}

%-------------------------------------------------------------
    \subsection*{C4) Hamiltonian/Lagrangian usage: $\rho^{(\text{fluid})}$ vs $\rho^{(\text{mass})}$}
    \label{subsec:rho-policy}

    Across VAM-16–20 two distinct densities appear:
    \begin{align*}
        &\text{Bulk swirl energetics (Canon \S4, G.2):}
        &&\frac{1}{2}\,\rho_{\text{\ae}}^{(\text{fluid})}\,\lVert \vec v\rVert^2
        \;+\;\frac{1}{2}\,\rho_{\text{\ae}}^{(\text{fluid})}\,r_c^{2}\lVert \vec\omega\rVert^2
        \;+\;\cdots \\
        &\text{EM coupling (\S\,\ref{subsec:coulomb-from-swirl}):}
        &&\Lambda = 4\pi\,\rho_{\text{\ae}}^{(\text{mass})}\,C_e^2\,r_c^4\,.
    \end{align*}
    \textbf{Conclusion (policy):} retain $\rho_{\text{\ae}}^{(\text{fluid})}$ in the \emph{Hamiltonian kernel} for continuum energetics (Canon \S4), and reserve $\rho_{\text{\ae}}^{(\text{mass})}$ for \emph{core/EM coupling} identities (\S\,C1). This resolves the apparent density ``swap'' without changing any numerics.

    \textit{Non-original context: continuum energy density forms \cite{LandauFluids}.}

%-------------------------------------------------------------
    \subsection*{C5) Time-rate law: confirm $r_c$ factor and note draft variance}
    \label{subsec:timerate-consistency}

    Some 16–20 drafts reused the historical form $dt_{\text{local}}/dt_\infty=\sqrt{1- \lVert\omega\rVert^2/C_e^2}$.
    \textbf{Canon-consistent law (dimensionally correct):}
    \[
        \boxed{ \frac{dt_{\text{local}}}{dt_\infty} \;=\; \sqrt{\,1- \frac{\lVert\omega\rVert^2\,r_c^2}{c^2}\,}
        \;=\; \sqrt{\,1- \frac{v_t^2}{c^2}\,},\;\;v_t:=\lVert\omega\rVert r_c }
    \]
    \emph{Action:} keep only the $r_c$-normalized law in Canon (\S\,12.1/\S\,3.3); mark the non-normalized variant as deprecated.

%-------------------------------------------------------------
    \subsection*{C6) ``Zero-vorticity line'' claim: status and correction path}
    \label{subsec:zero-vort-line}

    With the canonical profile $\Omega_{\text{swirl}}(r)=\tfrac{C_e}{r_c} e^{-r/r_c}$ and $v_\theta=r\Omega$, the axial vorticity is
    \[
        \omega_z(r) \;=\; \frac{1}{r}\frac{d}{dr}\!\big(r v_\theta\big) \;=\; 2\Omega(r)+ r\,\Omega'(r)\,,
    \]
    so $\omega_z(0)=2\,C_e/r_c\neq 0$.
    \textbf{Conclusion:} the \emph{``Zero-Vorticity Line''} is \emph{not} satisfied by the current core law.
    Two consistent options:
    (i) \emph{reinterpret} the phrase as a \emph{null pressure-gradient axis} (keeping $\omega_z(0)\neq 0$), or
    (ii) \emph{adopt} a modified core profile with $\Omega(r)\propto r$ as $r\to 0$ to enforce $\omega_z(0)=0$.
    \emph{Decision required} before promotion.

    \textit{Non-original identities: vorticity in cylindrical coordinates \cite{Batchelor1967}.}

%-------------------------------------------------------------
    \subsection*{C7) Vortex-string EFT mass functional (candidate form)}
    \label{subsec:vortex-string-mass}

    The VAM-20 EFT drafts propose a topological mass functional for a knotted core:
    \[
        \boxed{ m_{K}^{\text{sol}} \;=\; \mathcal{C}_0 \left(\sum_i V_i\right)\,
        \rho_{\text{\ae}}^{(\text{fluid})}\,\frac{C_e^2}{c^2}\;
        \Xi_{K}\!\Big(\mathrm{Tw},\mathrm{Wr},\mathrm{Lk};\,\varphi\Big) }
    \]
    where $\sum_i V_i$ is the effective core volume (possibly multi-tube), and $\Xi_K$ is a dimensionless topological factor (\emph{to be calibrated}, e.g.\ to the electron ring).
    \textbf{Status:} keep as \emph{research} (not yet Canon); compatible with Canon energetics (\S4) once $\mathcal{C}_0$ and $\Xi_K$ are fixed.

%-------------------------------------------------------------
    \subsection*{Summary of promotions and open items}
    \begin{itemize}
        \item \textbf{Promote now:} \S\,C1 ($\Lambda$ identity), \S\,C2 (soft-core hydrogen), \S\,C3 (circulation–$g_{t\theta}$ relation), and \S\,C5 (time-rate with $r_c$).
        \item \textbf{Append to constants table:} $\Lambda=4\pi\rho_{\text{\ae}}^{(\text{mass})}C_e^2 r_c^4$.
        \item \textbf{Keep research:} \S\,C6 (zero-vorticity line — needs definition/profile choice), \S\,C7 (vortex-string mass functional calibration).
    \end{itemize}

%--------------------------- BibTeX additions ---------------------------
% Add these to your .bib (non-original references cited above)
    \begin{thebibliography}{9}
        \bibitem{Jackson1999}
        J. D. Jackson, \emph{Classical Electrodynamics}, 3rd ed., Wiley, 1999.

        \bibitem{Schrodinger1926}
        E. Schr\"odinger, ``An Undulatory Theory of the Mechanics of Atoms and Molecules'',
        \emph{Phys.\ Rev.} \textbf{28}, 1049–1070 (1926). doi:10.1103/PhysRev.28.1049.

        \bibitem{Painleve1921}
        P. Painlev\'e, ``La m\'ecanique classique et la th\'eorie de la relativit\'e'',
        \emph{C. R. Acad. Sci. Paris} \textbf{173}, 677–680 (1921).

        \bibitem{Gullstrand1922}
        A. Gullstrand, ``Allgemeine L\"osung des statischen Eink\"orperproblems in der Einsteinschen Gravitationstheorie'',
        \emph{Ark. Mat. Astron. Fys.} \textbf{16}, 1–15 (1922).

        \bibitem{Unruh1981}
        W. G. Unruh, ``Experimental black-hole evaporation?'',
        \emph{Phys.\ Rev.\ Lett.} \textbf{46}, 1351–1353 (1981). doi:10.1103/PhysRevLett.46.1351.

        \bibitem{Visser1998}
        M. Visser, ``Acoustic black holes: horizons, ergospheres and Hawking radiation'',
        \emph{Class.\ Quantum Grav.} \textbf{15}, 1767–1791 (1998). doi:10.1088/0264-9381/15/6/024.

        \bibitem{Kerr1963}
        R. P. Kerr, ``Gravitational field of a spinning mass as an example of algebraically special metrics'',
        \emph{Phys.\ Rev.\ Lett.} \textbf{11}, 237–238 (1963). doi:10.1103/PhysRevLett.11.237.

        \bibitem{Batchelor1967}
        G. K. Batchelor, \emph{An Introduction to Fluid Dynamics}, Cambridge Univ. Press, 1967.

        \bibitem{LandauFluids}
        L. D. Landau and E. M. Lifshitz, \emph{Fluid Mechanics}, 2nd ed., Pergamon, 1987.
    \end{thebibliography}

%================== end v0.6 Delta — Conclusions from VAM 16–20 ==================


    \bibliographystyle{unsrt}
    \bibliography{vam_canon_refs}
\end{document}
%======================== end vam_canon_v5.tex ====================


%======================== vam_canon_refs.bib ======================
@article{Unruh1981,
author = {Unruh, W. G.},
title = {Experimental black-hole evaporation?},
journal = {Phys. Rev. Lett.},
year = {1981},
volume = {46},
pages = {1351--1353},
doi = {10.1103/PhysRevLett.46.1351}
}

@article{Visser1998,
author = {Visser, Matt},
title = {Acoustic black holes: horizons, ergospheres and Hawking radiation},
journal = {Class. Quantum Grav.},
year = {1998},
volume = {15},
number = {6},
pages = {1767--1791},
doi = {10.1088/0264-9381/15/6/024}
}

@article{Painleve1921,
author = {Painlev{\'e}, Paul},
title = {La m{\'e}canique classique et la th{\'e}orie de la relativit{\'e}},
journal = {C. R. Acad. Sci. Paris},
year = {1921},
volume = {173},
pages = {677--680}
}

@article{Gullstrand1922,
author = {Gullstrand, Allvar},
title = {Allgemeine L{\"o}sung des statischen Eink{\"o}rperproblems in der Einsteinschen Gravitationstheorie},
journal = {Arkiv f{\"o}r Matematik, Astronomi och Fysik},
year = {1922},
volume = {16},
pages = {1--15}
}

@book{Batchelor1967,
author = {Batchelor, G. K.},
title = {An Introduction to Fluid Dynamics},
publisher = {Cambridge University Press},
year = {1967}
}

@article{Helmholtz1858,
author  = {Helmholtz, H. von},
title   = {On Integrals of the Hydrodynamical Equations which Express Vortex-motion},
journal = {Philosophical Magazine},
year    = {1858}
}

@article{Kelvin1867,
author  = {Thomson, W. (Lord Kelvin)},
title   = {On Vortex Atoms},
journal = {Proc. Royal Society of Edinburgh},
year    = {1867}
}

@article{Moffatt1969,
author  = {Moffatt, H. K.},
title   = {The degree of knottedness of tangled vortex lines},
journal = {Journal of Fluid Mechanics},
year    = {1969}
}

@article{Schrodinger1926,
author  = {Schr{\"o}dinger, E.},
title   = {An Undulatory Theory of the Mechanics of Atoms and Molecules},
journal = {Physical Review},
year    = {1926}
}
%======================== end vam_canon_refs.bib ==================

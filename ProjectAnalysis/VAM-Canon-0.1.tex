%! Author = mr
%! Date = 8/22/2025

% Preamble
\documentclass[11pt]{article}

% Packages
\usepackage{amsmath}

% Document
\begin{document}


    \chapter*{VAM Canon (v0.1)}

\textit{Prepared for: Omar Iskandarani (Vortex Æther Model, VAM)}

\textit{Date: 2025‑08‑22}

\section{Versioning}
\begin{itemize}
    \item This document is the single source of truth for core VAM definitions, constants, master equations, and notational conventions.
    \item Use semantic versions: vMAJOR.MINOR.PATCH (e.g., v1.2.0).
    \item Every paper/derivation should state the Canon version it depends on.
\end{itemize}

\section{Core Postulates (VAM)}
\begin{enumerate}
    \item The universe is a 3D incompressible, inviscid superfluid æther with absolute time and Euclidean space.
    \item Particles are knotted vortex solitons (closed, possibly linked/knotted filaments) in the æther.
    \item Gravity is not spacetime curvature but swirl (structured vorticity fields) and pressure gradients; massive motion follows swirl-induced dynamics.
    \item Local time-rate is set by tangential vortex motion: higher swirl reduces the local clock rate relative to asymptotic time.
    \item Quantization arises from topological invariants (linking, writhe, twist) and circulation quantization of vortex filaments.
    \item The æther supports bosonic unknotted excitations (e.g., photon-like), while chiral hyperbolic knots map to quarks; torus knots map to leptons, etc. (taxonomy documented separately).
\end{enumerate}

\section{Canonical Constants and Symbols}
All symbols are dimensionally consistent and, unless stated otherwise, SI.

\subsection{Fundamental (VAM-specific)}
\begin{itemize}
    \item Vortex tangential velocity: $C_e = 1.09384563 \times 10^{6}\;\mathrm{m}\,\mathrm{s}^{-1}$
    \item Vortex-core radius: $r_c = 1.40897017 \times 10^{-15}\;\mathrm{m}$
    \item Æther fluid density ("vacuum" fluid): $\rho_{\text{\ae}}^{(\text{fluid})} = 7.0 \times 10^{-7}\;\mathrm{kg}\,\mathrm{m}^{-3}$
    \item Æther core/mass density: $\rho_{\text{\ae}}^{(\text{mass})} = 3.8934358266918687 \times 10^{18}\;\mathrm{kg}\,\mathrm{m}^{-3}$
    \item Æther energy density: $\rho_{\text{\ae}}^{(\text{energy})} = 3.49924562 \times 10^{35}\;\mathrm{J}\,\mathrm{m}^{-3}$
    \item Maximum Coulomb force (VAM): $F_{\text{\ae}}^{\max} = 29.053507\;\mathrm{N}$
    \item Maximum universal force (contextual): $F_{\text{gr}}^{\max} = 3.02563 \times 10^{43}\;\mathrm{N}$
    \item Golden ratio: $\varphi = \frac{1+\sqrt{5}}{2} \approx 1.61803398875$
\end{itemize}

\subsection{Universal}
\begin{itemize}
    \item Speed of light: $c = 299\,792\,458\;\mathrm{m}\,\mathrm{s}^{-1}$
    \item Fine-structure constant: $\alpha \approx 7.2973525643 \times 10^{-3}$
    \item Planck time: $t_p \approx 5.391247 \times 10^{-44}\;\mathrm{s}$
\end{itemize}

\textbf{Note:} The local Python \texttt{constants\_dict} used in simulations must mirror these values exactly; papers should quote the Canon version.




\section{Master Equations (Boxed, Definitive)}

\subsection{Master Energy and Mass Formula}

Define the amplified swirl energy for a coherent VAM volume $V$:
\begin{equation}
E_{\text{VAM}}(V) = \frac{4}{\alpha\,\varphi} \left( \frac{1}{2}\,\rho_{\text{\ae}}^{(\text{fluid})}\,C_e^{2} \right) V
\quad [\text{J}]
\end{equation}

Corresponding mass (strict SI mass):
\begin{equation}
M_{\text{VAM}}(V) = \frac{E_{\text{VAM}}(V)}{c^{2}}
\quad [\text{kg}]
\end{equation}

Numerical prefactor (per unit volume):\\
$\frac{1}{2}\rho_{\text{\ae}}^{(\text{fluid})}C_e^2 \approx 4.1877439\times10^{5}\;\text{J}\,\text{m}^{-3}$,\\
$\frac{4}{\alpha\varphi} \approx 3.3877162\times10^{2}$.\\
Thus, $\frac{E_{\text{VAM}}}{V} \approx 1.418688\times10^{8}\;\text{J}\,\text{m}^{-3}$,\\
$\frac{M_{\text{VAM}}}{V} \approx 1.57850\times10^{-9}\;\text{kg}\,\text{m}^{-3}$.

\textbf{Usage:} In derivations, treat the boxed forms as canonical. If a paper chooses to define mass directly via energy units, state the convention explicitly and reference this section.

\subsection{Swirl Gravitational Coupling}

\begin{equation}
G_{\text{swirl}} = \frac{C_e\,c^{5}\,t_p^{2}}{2\,F_{\text{\ae}}^{\max}\,r_c^{2}}
\quad \left( F_{\text{\ae}}^{\max}=29.053507\,\text{N} \right)
\end{equation}

Numerical evaluation: $G_{\text{swirl}} \approx 6.674302\times10^{-11}\;\text{m}^3\,\text{kg}^{-1}\,\text{s}^{-2}$.

\textit{Canon note:} This fixes which $F_{\max}$ is used (the Coulomb-scale $F_{\text{\ae}}^{\max}$), ensuring exact numerical match to Newton’s $G$.

\subsection{Local Time-Rate (Swirl Clock)}

\begin{equation}
dt_{\text{local}} = dt_{\infty}\,\sqrt{1 - \frac{\lVert\vec{\omega}\rVert^{2}\,r_c^{2}}{c^{2}}}
\end{equation}

Alternative (historical, for traceability):
\begin{equation}
dt_{\text{local}} = dt_{\infty}\,\sqrt{1 - \frac{\lVert\vec{\omega}\rVert^{2}}{c^{2}}}
\end{equation}

\subsection{Swirl Angular Frequency Profile}

\begin{equation}
\Omega_{\text{swirl}}(r) = \frac{C_e}{r_c} e^{-r/r_c}
\end{equation}
On-axis core limit: $\Omega_{\text{swirl}}(0)=\frac{C_e}{r_c}\approx 7.76344\times10^{20}\;\text{s}^{-1}$.

\subsection{Vorticity Potential (Canonical Form)}

\begin{equation}
\Phi(\vec r,\vec\omega) = \frac{C_e^{2}}{2\,F_{\text{\ae}}^{\max}}\,\vec\omega\cdot\vec r
\end{equation}

\textbf{Dimensional remark:} This potential’s role is canonical within VAM; derivations using it must propagate units consistently within the VAM Lagrangian (Sec.~\ref{sec:lagrangian}).

\section{Unified VAM Lagrangian (Definitive Form)}
\label{sec:lagrangian}

Let $\vec v$ be the æther velocity, $\rho=\rho_{\text{\ae}}^{(\text{fluid})}$ constant (incompressible), $\vec\omega=\nabla\times\vec v$, and $p$ a Lagrange multiplier enforcing incompressibility.

\begin{equation}
\mathcal{L}_{\text{VAM}} =
\underbrace{\frac{1}{2}\rho\,\lVert\vec v\rVert^{2}}_{\text{kinetic}}
- \underbrace{\rho\,\Phi(\vec r,\vec\omega)}_{\text{swirl potential}}
+ \underbrace{\lambda(\nabla\cdot\vec v)}_{\text{incompressibility}}
+ \underbrace{\eta\,\mathcal{H}[\vec v]}_{\text{helicity/topological term}}
+ \underbrace{\mathcal{L}_{\text{couple}}[\Gamma,\mathcal{K}]}_{\text{circulation \& knot invariants}}
\end{equation}

\begin{itemize}
    \item $\mathcal{H}[\vec v] = \int (\vec v\cdot\vec\omega)\,dV$ (kinetic helicity) serves as the generator of topological constraints (coefficient $\eta$ fixes units).
    \item $\mathcal{L}_{\text{couple}}$ encodes coupling to quantized circulation $\Gamma$ and knot invariants $\mathcal{K}$ (linking, writhe, twist), used to produce particle families.
    \item When deriving Euler–Lagrange equations, enforce $\nabla\cdot\vec v=0$ and appropriate boundary terms for closed filaments.
\end{itemize}

\textbf{Canon rule:} Papers must either (i) use this Lagrangian verbatim, or (ii) state a justified variant and show equivalence in the weak/appropriate limit.

\section{Notation, Ontology, and Glossary}

\begin{itemize}
    \item \textbf{Æther-Time (A-time):} absolute time parameter of the æther flow.
    \item \textbf{Chronos-Time (C-time):} asymptotic observer time ($dt_{\infty}$).
    \item \textbf{Swirl Clock:} local clock with rate set by $\lVert\vec\omega\rVert$ per Sec.~3.3.
    \item \textbf{Knot Taxonomy:} leptons = torus knots; quarks = chiral hyperbolic knots (chirality encodes vortex time); bosons = unknots; neutrinos = linked knots.
    \item \textbf{Chirality (matter vs antimatter):} ccw $\leftrightarrow$ matter; cw $\leftrightarrow$ antimatter via swirl-gravity coupling.
\end{itemize}

\section{Canonical Checks (What to Verify in Every Paper)}

\begin{enumerate}
    \item Dimensional analysis on every new term/equation.
    \item Limiting behavior: low-swirl $\lVert\omega\rVert\to 0$ recovers classical mechanics/EM limits; large-scale averages reproduce Newtonian gravity with $G_{\text{swirl}}$.
    \item Numerical validation: provide numerical prefactors using Canon constants; if additional constants appear, they must be added to Sec.~2.
    \item Topology $\leftrightarrow$ quantum numbers mapping stated explicitly (which invariants, how normalized).
    \item Citations for any non-original constructs (use BibTeX keys below).
\end{enumerate}

\section{Persona Prompts}

\subsection*{Reviewer Persona}
\begin{verbatim}
You are a peer reviewer for a VAM paper. Use only the definitions and constants in the "VAM Canon (v0.1)". Check dimensional consistency, limiting behavior, and numerical validation. Flag any use of non-canonical constants or equations unless equivalence is proved. Demand explicit mapping from knot invariants (linking, writhe, twist) to claimed quantum numbers.
\end{verbatim}

\subsection*{Theorist Persona}
\begin{verbatim}
You are a theoretical physicist specialized in the Vortex Æther Model (VAM). Base all reasoning on the attached "VAM Canon (v0.1)". Your task: derive the swirl-based Hamiltonian for [TARGET SYSTEM], use Sec. 4 Lagrangian, and verify the time-rate law (Sec. 3.3). Provide boxed equations, dimensional checks, and a short numerical evaluation using the Canon constants.
\end{verbatim}

\subsection*{Bridging Persona (Compare to GR/SM)}
\begin{verbatim}
Work strictly within VAM Canon (v0.1). Compare [TARGET] to its GR/SM counterpart. Identify exact replacements (e.g., curvature → swirl), and show which terms reduce to Newtonian/Maxwellian limits. Include a table of correspondences and any constraints needed for equivalence.
\end{verbatim}

\section{Session Kickoff Checklist}

\begin{enumerate}
    \item Start new chat per task; attach this Canon first.
    \item Paste a persona prompt (Sec.~7).
    \item Attach only task-relevant papers/sources.
    \item State any corrections explicitly (they persist in the session).
    \item At end, record Canon deltas (if any) and bump version.
\end{enumerate}

\section{Canon-Ready Citations (Skeleton)}

Replace placeholders with your BibTeX keys; ensure each non-original equation/idea cites at least one primary source.

\begin{verbatim}
@article{Helmholtz1858,
  author  = {H. von Helmholtz},
  title   = {On Integrals of the Hydrodynamical Equations which Express Vortex-motion},
  journal = {Philosophical Magazine},
  year    = {1858}
}

@article{Kelvin1867,
  author  = {W. Thomson (Lord Kelvin)},
  title   = {On Vortex Atoms},
  journal = {Proc. Royal Society of Edinburgh},
  year    = {1867}
}

@article{Moffatt1969,
  author  = {H. K. Moffatt},
  title   = {The degree of knottedness of tangled vortex lines},
  journal = {Journal of Fluid Mechanics},
  year    = {1969}
}

@article{Schrodinger1926,
  author  = {E. Schr{\"o}dinger},
  title   = {An Undulatory Theory of the Mechanics of Atoms and Molecules},
  journal = {Physical Review},
  year    = {1926}
}
\end{verbatim}

\section*{10) Appendix: Canon Tables for Papers}

\subsection*{10.1 Constants Table (paste-ready)}

\begin{table}[h!]
\centering
\begin{tabular}{|l|l|l|l|}
\hline
\textbf{Symbol} & \textbf{Meaning} & \textbf{Value} & \textbf{Unit} \\
\hline
$C_e$ & Vortex tangential velocity & $1.09384563\times 10^{6}$ & m\,s$^{-1}$ \\
$r_c$ & Vortex-core radius & $1.40897017\times 10^{-15}$ & m \\
$\rho_{\text{\ae}}^{(\text{fluid})}$ & Æther fluid density & $7.0\times 10^{-7}$ & kg\,m$^{-3}$ \\
$\rho_{\text{\ae}}^{(\text{mass})}$ & Æther mass density & $3.8934358266918687\times 10^{18}$ & kg\,m$^{-3}$ \\
$\rho_{\text{\ae}}^{(\text{energy})}$ & Æther energy density & $3.49924562\times 10^{35}$ & J\,m$^{-3}$ \\
$F_{\text{\ae}}^{\max}$ & Max. Coulomb force & 29.053507 & N \\
$F_{\text{gr}}^{\max}$ & Max. universal force & $3.02563\times 10^{43}$ & N \\
$\alpha$ & Fine-structure constant & $7.2973525643\times 10^{-3}$ & --- \\
$\varphi$ & Golden ratio & $1.61803398875$ & --- \\
$c$ & Speed of light & 299792458 & m\,s$^{-1}$ \\
$t_p$ & Planck time & $5.391247\times 10^{-44}$ & s \\
\hline
\end{tabular}
\caption{Canonical constants for VAM (SI units unless stated).}
\end{table}

\subsection*{10.2 Boxed Canon Equations (paste-ready)}

\begin{enumerate}
    \item \textbf{Energy:} \fbox{$E_{\text{VAM}} = \dfrac{4}{\alpha\varphi}\left(\dfrac{1}{2}\rho C_e^2\right)V$}
    \item \textbf{Mass:} \fbox{$M_{\text{VAM}} = \dfrac{E_{\text{VAM}}}{c^2}$}
    \item \textbf{$G$ coupling:} \fbox{$G_{\text{swirl}} = \dfrac{C_e c^5 t_p^2}{2F_{\text{\ae}}^{\max} r_c^2}$}
    \item \textbf{Time-rate:} \fbox{$dt_{\text{local}} = dt_{\infty}\sqrt{1-\lVert\omega\rVert^2/c^2}$}
    \item \textbf{Swirl profile:} \fbox{$\Omega_{\text{swirl}}(r) = \dfrac{C_e}{r_c}e^{-r/r_c}$}
\end{enumerate}

\section*{11) Change Log}

\begin{itemize}
    \item \textbf{v0.1 (2025-08-22):} Initial Canon with core postulates, constants, boxed master equations, Lagrangian, persona prompts, and session protocol; numerical prefactors added for Sec.~3.
\end{itemize}

\section*{12) v0.2 Delta --- Corrections \& Additions (2025-08-22)}

\subsection*{12.1 Dimensional correction to Sec.~3.3 (time-rate law)}

To enforce strict dimensional consistency, the time-rate must couple vorticity to a length scale (canonical choice: the core radius $r_c$) or, equivalently, to the local tangential speed $v_t = |\omega| \cdot r$:

\begin{itemize}
    \item Canonical (evaluate at $r = r_c$):\\
    $dt_{\text{local}} = dt_{\infty} \sqrt{1 - (|\omega|^2 r_c^2)/c^2}$\\
    equivalently $dt_{\text{local}} = dt_{\infty} \sqrt{1 - v_t^2/c^2}$ with $v_t := |\omega| r_c$.
    \item Using the profile $\Omega_{\text{swirl}}(r) = (C_e/r_c) \exp(-r/r_c)$ (Sec.~3.4), on-axis core limit gives $\Omega_{\text{swirl}}(0) = C_e/r_c$ and thus $dt_{\text{local}}(0) = dt_{\infty} \sqrt{1 - (C_e/c)^2}$.
\end{itemize}

\textit{Supersedes Sec.~3.3 formula (which lacked a length scale). Use this corrected form in all new derivations; the earlier expression is retained for traceability only.}

\subsection*{12.2 Canon tolerances \& symbol aliases}

\textbf{Numerical tolerances (for constant concordance):}
\begin{itemize}
    \item Relative: $\leq 1\times 10^{-6}$ (1 ppm).
    \item Absolute near zero: $\leq 1\times 10^{-12}$ in SI units.
\end{itemize}

\textbf{Accepted symbol aliases (normalize to the left-hand form):}

\begin{table}[h!]
\centering
\begin{tabular}{|l|l|}
\hline
\textbf{Canon} & \textbf{Accepted aliases} \\
\hline
Ce & Ce, C\_e \\
rc & rc, r\_c \\
rho\_ae$^{(\text{fluid})}$ & rho\_ae (fluid), rho\_vac, rho\_fluid \\
rho\_ae$^{(\text{mass})}$ & rho\_ae (mass), rho\_core, rho\_mass \\
rho\_ae$^{(\text{energy})}$ & rho\_energy, u\_ae (J m$^{-3}$) \\
F\_ae$^{\max}$ & Fae\_max \\
F\_gr$^{\max}$ & Fgr\_max \\
varphi & phi, varphi \\
\hline
\end{tabular}
\caption{Accepted symbol aliases for Canon constants.}
\end{table}

\textit{Rule: manuscripts must present a single normalized constants table conforming to Sec.~10.1; aliases may appear in prose but equations must use Canon symbols.}

\subsection*{12.3 Validation protocol updates}

\begin{enumerate}
    \item Dimensional sanity (strict): every term reduces to SI; for Sec.~4 ensure $\rho \Phi$ carries energy density (J\,m$^{-3}$). If an intermediate potential uses non-standard units, introduce a calibration coefficient and state its units.
    \item Equation normalization: when swirl/time enters, first reduce by $v_t = |\omega| r$ with $r = r_c$ unless a different physically motivated scale is justified.
    \item Numerical reproduction: provide a short table with substituted Canon constants and results (3--5 s.f.).
    \item BibTeX policy: any non-original idea/equation/comparison must include a BibTeX entry (add to Sec.~9).
\end{enumerate}

\subsection*{12.4 Concept index (snapshot from VAM-rank-1 corpus)}

Frequency across the six PDFs analyzed:

\begin{enumerate}
    \item vortex-knot particles (1839)
    \item time dilation / swirl clock (1062)
    \item swirl gravity (964)
    \item æther densities (860)
    \item leptons as torus knots (660)
    \item quarks as hyperbolic knots (647)
    \item photon as vortex ring (306)
    \item unified Lagrangian (70)
    \item Hamiltonian (25)
    \item Rodin/coil dynamics (1)
\end{enumerate}

\subsection*{12.5 Simulator I/O stub (render-ready)}

\begin{verbatim}
{
  "SceneSpec": {
    "background": {"rho_fluid": 7.0e-7, "Ce": 1.09384563e6, "rc": 1.40897017e-15},
    "fields": [ {"type": "swirl", "Omega_profile": "Ce/rc * exp(-r/rc)"} ],
    "objects": [
      { "type": "VortexKnot", "knot": "T(2,3)", "circulation": "Gamma0",
        "core_radius": "rc", "constraints": ["incompressible", "quantized_circulation"] }
    ]
  }
}
\end{verbatim}

\subsection*{12.6 Change Log entry}

\begin{itemize}
    \item v0.2 (2025-08-22): Added dimensionally corrected time-rate law using $r_c$ (Sec.~12.1), established tolerances and symbol aliasing (Sec.~12.2), tightened validation protocol (Sec.~12.3), recorded a concept index snapshot from the current corpus (Sec.~12.4), and included a render-ready SceneSpec stub for simulators (Sec.~12.5).
\end{itemize}

\section{v0.3 Draft Delta --- Core from VAM 0--4 (Einstein $\rightarrow$ Vortex Fluid)}
\textbf{Status: DRAFT (pending promotion to sections 1--5 after review)}

\subsection{Source batch (chronological TeX)}
Parsed: \texttt{VAM\_0--4\_Einstein\_to\_Vortex\_Fluid} (TeX-first corpus)\\
Artifacts indexed (TeX-aware): 2335 equation blocks; 268 constant definitions/assignments; 246 postulate-like sentences; structured outline per file.

\subsection{Consolidated core postulates (canonical wording)}
\begin{enumerate}
    \item \textbf{Absolute time, Euclidean space ($\mathbb{R}^3$).} A universal ``clock field'' defines a preferred foliation consistent with VAM's absolute æther time.
    \item \textbf{Incompressible, inviscid æther.} Background medium supports ideal Euler dynamics; density $\rho_{\text{\ae}}^{(\text{fluid})}$ is constant at macroscales.
    \item \textbf{Particles = knotted vortex solitons.} Matter is realized as closed, possibly linked/knotted filaments; bosons as unknotted excitations.
    \item \textbf{Gravity = structured swirl.} Macroscopic attraction emerges from coherent vorticity fields and pressure gradients; Newton's $G$ is recovered via $G_{\text{swirl}}$.
    \item \textbf{Quantization from topology and circulation.} Discrete quantum numbers trace to linking/writhe/twist and circulation quantization.
    \item \textbf{Kelvin--Helmholtz invariants govern dynamics.} Circulation conservation and helicity underpin stability, reconnection energetics, and decay.
\end{enumerate}
\textit{These six are promoted to Canon~\S1 after approval. Existing~\S1 will be rephrased to this exact minimal set.}

\subsection{Canon conservation laws (add to \S3: ``Foundational identities'')}
\begin{itemize}
    \item \textbf{Kelvin circulation (inviscid, barotropic):} $\frac{d\Gamma}{dt} = 0$ along a material loop.
    \item \textbf{Vorticity transport (Euler):} $\frac{\partial\vec{\omega}}{\partial t} = \nabla \times (\vec{v} \times \vec{\omega})$.
    \item \textbf{Kinetic helicity density:} $h = \vec{v} \cdot \vec{\omega}$; \textbf{Helicity invariant:} $H = \int (\vec{v} \cdot \vec{\omega})\,dV$ (up to reconnection events).
\end{itemize}
\textit{Rationale: These appear repeatedly across VAM 0--4 and are required to justify knot stability and reconnection phenomenology. They are background identities; use BibTeX keys in~\S9 (Helmholtz/Kelvin/Moffatt).}

\subsection{Key equations shortlist (from VAM 0--4)}
\begin{itemize}
    \item \textbf{Swirl profile:} $\Omega_{\text{swirl}}(r) = \frac{C_e}{r_c} \exp(-r/r_c)$ (consistent with Canon~\S3.4).
    \item \textbf{Time-rate (dimensionally corrected):} $dt_{\text{local}} = dt_{\infty} \sqrt{1 - |\omega|^2 r_c^2 / c^2} = dt_{\infty} \sqrt{1 - v_t^2/c^2}$.
    \item \textbf{Mass/Energy:} $E_{\text{VAM}} = \frac{4}{\alpha\varphi} \left(\frac{1}{2} \rho_{\text{\ae}}^{(\text{fluid})} C_e^2\right) V$, $M = E_{\text{VAM}}/c^2$.
    \item \textbf{G coupling:} $G_{\text{swirl}} = \frac{C_e c^5 t_p^2}{2 F_{\text{\ae}}^{\max} r_c^2}$.
    \item \textbf{Helicity/Lagrangian:} Canon~\S4 form with $H[\vec{v}] = \int (\vec{v} \cdot \vec{\omega})\,dV$ and incompressibility via $\lambda (\nabla\cdot\vec{v})$.
\end{itemize}
(Full ledger with file pointers is in the generated CSV: \texttt{equations\_shortlist.csv}.)

\subsection{Canon constants concordance (snapshot)}
The TeX sources define/mention aliases for: $C_e$, $r_c$, $\rho_{\text{\ae}}^{(\text{fluid|mass|energy})}$, $\alpha$, $c$, $t_p$, $\varphi$, $F_{\text{\ae}}^{\max}$, $F_{\text{gr}}^{\max}$.

\textbf{Action:} Enforce the v0.2 alias table (Sec.~12.2). Manuscripts must include a normalized constants table per Canon~\S10.1.

\subsection{Organization rule for VAM parts (Canon policy)}
Each VAM ``part'' must answer in its abstract:
\begin{enumerate}
    \item \textbf{Unique role:} What principle or equation does this part introduce that no other part covers?
    \item \textbf{Dependence:} Which Canon sections/parts are prerequisites?
    \item \textbf{Promotion path:} Which equations/postulates are candidates to move into Canon~\S\S1--5 after validation?
\end{enumerate}

\subsection{Promotion plan}
\begin{itemize}
    \item \textbf{Promote} \S13.2 postulates to Canon~\S1 (replace/merge wording) after you approve.
    \item \textbf{Add} \S13.3 conservation laws as Canon~\S3A (``Foundational identities'').
    \item \textbf{Relabel} old~\S3.3 (time-rate) as ``historical'' and keep~\S12.1 as the operative law; mirror the operative law into~\S3 with $r_c$.
    \item \textbf{Append} a permanent ``Chronology note'' linking VAM 0--4 to the Canon (\S0 Versioning $\rightarrow$ provenance).
\end{itemize}

\subsection{Citations to add in \S9 (BibTeX keys)}
\begin{itemize}
    \item \texttt{Kelvin1869} --- Circulation theorem.
    \item \texttt{Helmholtz1858} --- Vortex motion integrals.
    \item \texttt{Moffatt1969} --- Helicity/topological knottedness.\\
  (Keep existing entries; ensure all non-original laws are cited.)
\end{itemize}

\subsection{Generated indices for this batch (local paths)}
\begin{itemize}
    \item Outline (titles/sections): \texttt{vam\_corpus\_reports\_vam0\_4/outline.csv}
    \item Key equations (categorized): \texttt{vam\_corpus\_reports\_vam0\_4/equations\_categorized.csv}
    \item Equations shortlist: \texttt{vam\_corpus\_reports\_vam0\_4/equations\_shortlist.csv}
    \item Canon constants concordance: \texttt{vam\_corpus\_reports\_vam0\_4/canon\_concordance.csv}
    \item Postulates shortlist: \texttt{vam\_corpus\_reports\_vam0\_4/postulates\_shortlist.csv}
\end{itemize}

\textit{End of v0.3 draft delta.}



\section{VAM Canon v0.4 Delta: Derived Constants, Galactic Swirl Law, and Baryon Mass Map}
\label{sec:vam-canon-v0.4-delta}

\subsection{Canon Identities (to be promoted to \S3B)}
\begin{itemize}
    \item \textbf{Fine-structure constant from swirl speed:}
    \[
        \boxed{\alpha = \frac{2 C_e}{c}} \qquad \Longleftrightarrow \qquad \boxed{C_e = \frac{c\,\alpha}{2}}
    \]
    Dimensionality: velocity ratio $\rightarrow$ dimensionless (OK). Numerical check (Canon values): $\alpha=0.007297352557$.

    \item \textbf{Gravitational fine-structure constant:}
    \[
        \boxed{\alpha_g = \frac{C_e^{2} t_p^{2}}{r_c^{2}}}
    \]
    (dimensionless); $\ell_p \equiv c\,t_p$, $\ell_p^{2}=c^{2}t_p^{2}$.

    \item \textbf{Equivalents for $G$:}
    \[
        \boxed{G = \frac{\alpha_g c^{3} r_c}{C_e M_e} = \frac{C_e c\, \ell_p^{2}}{r_c M_e}}
    \]
    Using the VAM identity:
    \[
        \boxed{M_e = \frac{2 F_{\text{\ae}}^{\max} r_c}{c^{2}}}
    \]
    This equals the existing Canon coupling:
    \[
        \boxed{G_{\text{swirl}}=\frac{C_e c^{5} t_p^{2}}{2 F_{\text{\ae}}^{\max} r_c^{2}}}
    \]
    Numerical check (Canon values): $G=6.6743013\times10^{-11}\;\text{m}^3\,\text{kg}^{-1}\,\text{s}^{-2}$.
\end{itemize}

\subsection{Galactic Swirl Law (Disc Kinematics)}
A two-component velocity profile that captures solid-body rise and asymptotic flattening:
\[
    \boxed{v(r) = \frac{C_{\text{core}}}{\sqrt{1 + (r_c/r)^2}} + C_{\text{tail}}\,\big(1 - e^{-r/r_c}\big)}
\]
Limits: $v(0)=0$; $v(r\to\infty)=C_{\text{core}}+C_{\text{tail}}$. Small-$r$: core term $\sim (C_{\text{core}}/r_c)\,r$. Large-$r$: exponential approach governed by $r_c$.

\subsection{Baryon Mass Relations (VAM Knot Map)}
Let $M_u, M_d$ denote the effective VAM up/down knot masses. Then:
\[
    \boxed{M_p = \varphi^{-2}\,3^{-1/\varphi}\,(2 M_u + M_d)}\,,\qquad
    \boxed{M_n = \varphi^{-2}\,3^{-1/\varphi}\,(M_u + 2 M_d)}
\]

\subsection{Added/Derived Constants (Append to \S10.1)}
\begin{table}[h!]
    \centering
    \begin{tabular}{|l|l|l|l|}
        \hline
        \textbf{Symbol} & \textbf{Meaning} & \textbf{Value} & \textbf{Unit} \\
        \hline
        $M_e$ & Electron mass (derived in VAM) & $\dfrac{2 F_{\text{\ae}}^{\max} r_c}{c^2}$ $= 9.109383\times10^{-31}$ & kg \\
        $\ell_p$ & Planck length & $c t_p$ $= 1.616255\times10^{-35}$ & m \\
        $\alpha$ & Fine-structure (derived) & $2C_e/c$ $= 7.29735256\times10^{-3}$ & --- \\
        $\alpha_g$ & Gravitational fine-structure & $C_e^{2} t_p^{2}/r_c^{2}$ $= 1.75181\times10^{-45}$ & --- \\
        \hline
    \end{tabular}
    \caption{Added/derived constants for VAM Canon v0.4}
    \label{tab:added-derived-constants}
\end{table}

\subsection{Dimensional Validations}
\begin{itemize}
    \item $[\alpha]=[\alpha_g]=1$
    \item $[G]=\mathrm{L}^3\,\mathrm{M}^{-1}\,\mathrm{T}^{-2}$ from either boxed $G$ identity
    \item $[v(r)]=\mathrm{L}\,\mathrm{T}^{-1}$
\end{itemize}



\chapter*{VAM Canon — v0.5 Selections (ready-to-merge)}

\textit{Batch: VAM\_9--15 (Spacetime, Dark Sector \& Quantum Gravity)}

\section*{G) Boxed selections from VAM 9--15 --- to merge into Canon v0.5}

\subsection*{G.1 Effective metric / line element (axisymmetric swirl)}

Steady, incompressible, azimuthal drift $v_\theta(r)$ in cylindrical $(t,r,\theta,z)$:

\[
\boxed{
ds^2 = -\left(c^2 - v_\theta(r)^2\right) dt^2 + 2 v_\theta(r) r\, d\theta\, dt + dr^2 + r^2 d\theta^2 + dz^2
}
\]

In the co-rotating frame $\theta' = \theta - \int v_\theta(r)\,dt / r$, the cross term diagonalizes locally:

\[
\boxed{
ds^2 = -c^2\left(1 - \frac{v_\theta(r)^2}{c^2}\right) dt^2 + dr^2 + r^2 d\theta'^2 + dz^2
}
\]

This exposes the swirl-clock factor and matches Sec.~12.1/3.3 in the $v_\theta \ll c$ regime. Substitute your swirl law as needed, e.g. $v_\theta(r) = r\,\Omega_{\text{swirl}}(r)$ with $\Omega_{\text{swirl}}(r) = \frac{C_e}{r_c} e^{-r/r_c}$.

\textbf{BibTeX (analogue/PG background):} Unruh1981, Visser1998, Painleve1921, Gullstrand1922, Batchelor1967.

\subsection*{G.2 Swirl Hamiltonian density for Sec.~4 (dimensionally normalized)}

Take $\rho = \rho_{\ae}^{(\text{fluid})}$, $\vec{\omega} = \nabla \times \vec{v}$, $\lambda$ for incompressibility. A quadratic, Kelvin-compatible kernel:

\[
\boxed{
\mathcal{H}[\vec{v}] = \frac{1}{2} \rho \lVert \vec{v} \rVert^2 + \frac{1}{2} \rho \ell_\omega^2 \lVert \vec{\omega} \rVert^2 + \frac{1}{2} \rho \ell_\omega^4 \lVert \nabla \vec{\omega} \rVert^2 + \lambda (\nabla \cdot \vec{v})
}
\qquad \ell_\omega := r_c
\]

Units check: $[\rho \lVert \vec{v} \rVert^2] = \text{J}\,\text{m}^{-3}$; since $[\omega] = \text{s}^{-1}$, the coefficients $\rho, \ell_\omega^2$ and $\rho, \ell_\omega^4$ ensure the $|\omega|^2$ and $|\nabla \omega|^2$ terms also have energy-density units. In the $\ell_\omega \to 0$ limit this reduces to the bulk swirl energy.

\textit{(Optional minimal matter--swirl coupling, same section):}

\[
\boxed{
\mathcal{H}_\psi = \frac{\hbar^2}{2m} \left\lVert \left(\nabla - i\,\frac{m}{\hbar} \vec{A}_{\text{swirl}}\right) \psi \right\rVert^2 + U(|\psi|^2)
}
\qquad \vec{A}_{\text{swirl}} := \chi\,\vec{v}
\]

\subsection*{G.3 Dark-sector law beside $v(r)$ (Sec.~14.2)}

Radial Euler balance (steady, no radial flow) yields

\[
0 = -\frac{1}{\rho} \frac{dp_{\text{swirl}}}{dr} + \frac{v(r)^2}{r}
\qquad \Rightarrow \qquad
\boxed{
a_{\text{dark}}(r) \equiv \frac{1}{\rho} \frac{dp_{\text{swirl}}}{dr} = \frac{v(r)^2}{r}
}
\]

Equivalently as a pressure law paired with the swirl profile $v(r)$:

\[
\boxed{
\frac{dp_{\text{swirl}}}{dr} = \rho\,\frac{v(r)^2}{r}
}
\qquad \Longrightarrow \qquad
\boxed{
p_{\text{swirl}}(r) = p_0 + \rho\,v_0^2\,\ln\left(\frac{r}{r_0}\right)
}
\quad (\text{flat } v(r) \to v_0)
\]

\textit{Sign convention:} the inward centripetal requirement corresponds to an outward-rising pressure ($dp/dr > 0$) so that $-\nabla p/\rho$ supplies the inward acceleration.

\subsection*{G.4 Consistency vs Canon v0.1--v0.4}

\begin{itemize}
    \item Time-rate: metric's $g_{tt}$ gives $dt_{\text{local}}/dt_\infty = \sqrt{1- v_\theta^2/c^2}$, consistent with Sec.~12.1 choice $v_t = |\omega| r$ at $r = r_c$.
    \item Galactic law: use Sec.~14.2 $v(r)$ in G.3 to obtain explicit $p_{\text{swirl}}(r)$ in both core and tail limits.
    \item Dimensions: all boxed terms reduce to SI units with $\ell_\omega = r_c$ and $\rho = \rho_{\ae}^{(\text{fluid})}$.
\end{itemize}

\textbf{Ready to merge into Canon v0.5:} place G.1 in Sec.~3A/Sec.~6, G.2 in Sec.~4 (Hamiltonian), and G.3 alongside Sec.~14.2.

%===================== v0.6 Delta — Conclusions from VAM 16–20 =====================

    \section*{v0.6 Delta — Conclusions from VAM 16–20}
    \label{sec:v06-delta}

    \paragraph{Scope.}
    We consolidated the main outcomes of VAM-16–20 (Zero-Vorticity Line, photon/EM mapping, Kerr reinterpretation, vortex-string EFT, and Schr\"odinger hydrogen). Below are the boxed identities and consistency results that are ready for canonicalization (with dimensional and numerical checks). Items needing a policy decision are explicitly flagged.

%-------------------------------------------------------------
    \subsection*{C1) EM coupling emerges from core swirl pressure}
    \label{subsec:coulomb-from-swirl}

    Define the \emph{swirl Coulomb constant} via the pressure integral over a spherical surface:
    \[
        \boxed{ \Lambda \;\equiv\; \int_{S_r^2} p_{\text{swirl}}\,r^2\,d\Omega
        \;=\; 4\pi\,\rho_{\text{\ae}}^{(\text{mass})}\,C_e^2\,r_c^4 }
    \]
    Dimensions: $[\Lambda]=\mathrm{N\,m^2}=\mathrm{J\,m}$ (Coulomb constant units).
    \emph{Identification:}
    \[
        \boxed{ \Lambda \;=\; \frac{e^2}{4\pi\varepsilon_0} } \qquad \text{(EM coupling).}
    \]
    \textbf{Numerics (Canon values):} $\Lambda=2.30707733\times10^{-28}\,\mathrm{J\,m}$, matching $e^2/(4\pi\varepsilon_0)$ to $\lesssim10^{-7}$ relative.
    This promotes a tight constraint among $(\rho_{\text{\ae}}^{(\text{mass})},\,C_e,\,r_c)$ consistent with $\alpha = 2C_e/c$.
    \textbf{Status:} \emph{Promote} as Canon identity in \S3B and append $\Lambda$ to the constants table.

    \textit{Non-original comparison: Coulomb constant \cite{Jackson1999}.}

%-------------------------------------------------------------
    \subsection*{C2) Hydrogen Schr\"odinger equation with core softening}
    \label{subsec:hydrogen-soft-core}

    VAM replaces the Coulomb term by a swirl-induced softened potential
    \[
        \boxed{ V_{\text{VAM}}(r) \;=\; -\,\frac{\Lambda}{\sqrt{r^2+r_c^2}}
        \;\;\xrightarrow{r\gg r_c}\;\; -\frac{\Lambda}{r} }
    \]
    and the bound-state equation
    \[
        \boxed{ \left[-\frac{\hbar^2}{2\mu}\nabla^2 - \frac{\Lambda}{\sqrt{r^2+r_c^2}}\right]\psi \;=\; E\,\psi }
    \]
    recovers the textbook spectrum for $r\gg r_c$.
    \textbf{Bohr/Rydberg checks (H):}
    $a_0=\hbar^2/(\mu\Lambda)=5.29177262\times10^{-11}\,\mathrm{m}$,
    $E_{1}=\mu\Lambda^2/(2\hbar^2)=13.60569\,\mathrm{eV}$.
    Core softening gives $nS$ shifts $\sim \mathcal{O}\!\big((r_c/a_0)^2\big)\approx 7.1\times10^{-10}$ (H), and
    $\sim \mathcal{O}(2.4\times10^{-5})$ for muonic H (using $\mu\approx 186\,m_e$),
    i.e. a ground-state scale $\sim 6\times10^{-2}$\,eV --- a concrete experimental target.

    \textit{Non-original equations: Schr\"odinger hydrogen and Coulomb limit \cite{Schrodinger1926,Jackson1999}.}

%-------------------------------------------------------------
    \subsection*{C3) Frame-dragging / off-diagonal metric term as circulation}
    \label{subsec:metric-circulation}

    The PG-type analogue line element already in Canon (\S\,G.1) implies the mixed term
    \[
        \boxed{ g_{t\theta}^{\text{(VAM)}} \;=\; v_\theta(r)\,r \;=\; \frac{1}{2\pi}\,\Gamma_{\text{swirl}}(r) }
    \]
    with $\Gamma_{\text{swirl}}(r)=\oint v_\theta\,dl$ the azimuthal circulation at radius $r$.
    This is the precise VAM counterpart of GR's $g_{t\phi}$ frame-dragging structure for axisymmetric rotation, dovetailing with the Kerr reinterpretation draft.\footnote{In BL-like gauges one can factor $c^2$ to yield a dimensionless coefficient; the PG form used in Canon keeps $g_{t\theta}$ in velocity\,$\times$\,length units, matching the analogue-gravity construction.}
    \textbf{Status:} \emph{Promote} the boxed relation as a corollary to \S\,G.1.

    \textit{Non-original background: analogue/PG metrics and Kerr solution \cite{Painleve1921,Gullstrand1922,Unruh1981,Visser1998,Kerr1963}.}

%-------------------------------------------------------------
    \subsection*{C4) Hamiltonian/Lagrangian usage: $\rho^{(\text{fluid})}$ vs $\rho^{(\text{mass})}$}
    \label{subsec:rho-policy}

    Across VAM-16–20 two distinct densities appear:
    \begin{align*}
        &\text{Bulk swirl energetics (Canon \S4, G.2):}
        &&\frac{1}{2}\,\rho_{\text{\ae}}^{(\text{fluid})}\,\lVert \vec v\rVert^2
        \;+\;\frac{1}{2}\,\rho_{\text{\ae}}^{(\text{fluid})}\,r_c^{2}\lVert \vec\omega\rVert^2
        \;+\;\cdots \\
        &\text{EM coupling (\S\,\ref{subsec:coulomb-from-swirl}):}
        &&\Lambda = 4\pi\,\rho_{\text{\ae}}^{(\text{mass})}\,C_e^2\,r_c^4\,.
    \end{align*}
    \textbf{Conclusion (policy):} retain $\rho_{\text{\ae}}^{(\text{fluid})}$ in the \emph{Hamiltonian kernel} for continuum energetics (Canon \S4), and reserve $\rho_{\text{\ae}}^{(\text{mass})}$ for \emph{core/EM coupling} identities (\S\,C1). This resolves the apparent density ``swap'' without changing any numerics.

    \textit{Non-original context: continuum energy density forms \cite{LandauFluids}.}

%-------------------------------------------------------------
    \subsection*{C5) Time-rate law: confirm $r_c$ factor and note draft variance}
    \label{subsec:timerate-consistency}

    Some 16–20 drafts reused the historical form $dt_{\text{local}}/dt_\infty=\sqrt{1- \lVert\omega\rVert^2/C_e^2}$.
    \textbf{Canon-consistent law (dimensionally correct):}
    \[
        \boxed{ \frac{dt_{\text{local}}}{dt_\infty} \;=\; \sqrt{\,1- \frac{\lVert\omega\rVert^2\,r_c^2}{c^2}\,}
        \;=\; \sqrt{\,1- \frac{v_t^2}{c^2}\,},\;\;v_t:=\lVert\omega\rVert r_c }
    \]
    \emph{Action:} keep only the $r_c$-normalized law in Canon (\S\,12.1/\S\,3.3); mark the non-normalized variant as deprecated.

%-------------------------------------------------------------
    \subsection*{C6) ``Zero-vorticity line'' claim: status and correction path}
    \label{subsec:zero-vort-line}

    With the canonical profile $\Omega_{\text{swirl}}(r)=\tfrac{C_e}{r_c} e^{-r/r_c}$ and $v_\theta=r\Omega$, the axial vorticity is
    \[
        \omega_z(r) \;=\; \frac{1}{r}\frac{d}{dr}\!\big(r v_\theta\big) \;=\; 2\Omega(r)+ r\,\Omega'(r)\,,
    \]
    so $\omega_z(0)=2\,C_e/r_c\neq 0$.
    \textbf{Conclusion:} the \emph{``Zero-Vorticity Line''} is \emph{not} satisfied by the current core law.
    Two consistent options:
    (i) \emph{reinterpret} the phrase as a \emph{null pressure-gradient axis} (keeping $\omega_z(0)\neq 0$), or
    (ii) \emph{adopt} a modified core profile with $\Omega(r)\propto r$ as $r\to 0$ to enforce $\omega_z(0)=0$.
    \emph{Decision required} before promotion.

    \textit{Non-original identities: vorticity in cylindrical coordinates \cite{Batchelor1967}.}

%-------------------------------------------------------------
    \subsection*{C7) Vortex-string EFT mass functional (candidate form)}
    \label{subsec:vortex-string-mass}

    The VAM-20 EFT drafts propose a topological mass functional for a knotted core:
    \[
        \boxed{ m_{K}^{\text{sol}} \;=\; \mathcal{C}_0 \left(\sum_i V_i\right)\,
        \rho_{\text{\ae}}^{(\text{fluid})}\,\frac{C_e^2}{c^2}\;
        \Xi_{K}\!\Big(\mathrm{Tw},\mathrm{Wr},\mathrm{Lk};\,\varphi\Big) }
    \]
    where $\sum_i V_i$ is the effective core volume (possibly multi-tube), and $\Xi_K$ is a dimensionless topological factor (\emph{to be calibrated}, e.g.\ to the electron ring).
    \textbf{Status:} keep as \emph{research} (not yet Canon); compatible with Canon energetics (\S4) once $\mathcal{C}_0$ and $\Xi_K$ are fixed.

%-------------------------------------------------------------
    \subsection*{Summary of promotions and open items}
    \begin{itemize}
        \item \textbf{Promote now:} \S\,C1 ($\Lambda$ identity), \S\,C2 (soft-core hydrogen), \S\,C3 (circulation–$g_{t\theta}$ relation), and \S\,C5 (time-rate with $r_c$).
        \item \textbf{Append to constants table:} $\Lambda=4\pi\rho_{\text{\ae}}^{(\text{mass})}C_e^2 r_c^4$.
        \item \textbf{Keep research:} \S\,C6 (zero-vorticity line — needs definition/profile choice), \S\,C7 (vortex-string mass functional calibration).
    \end{itemize}

%--------------------------- BibTeX additions ---------------------------
% Add these to your .bib (non-original references cited above)
    \begin{thebibliography}{9}
        \bibitem{Jackson1999}
        J. D. Jackson, \emph{Classical Electrodynamics}, 3rd ed., Wiley, 1999.

        \bibitem{Schrodinger1926}
        E. Schr\"odinger, ``An Undulatory Theory of the Mechanics of Atoms and Molecules'',
        \emph{Phys.\ Rev.} \textbf{28}, 1049–1070 (1926). doi:10.1103/PhysRev.28.1049.

        \bibitem{Painleve1921}
        P. Painlev\'e, ``La m\'ecanique classique et la th\'eorie de la relativit\'e'',
        \emph{C. R. Acad. Sci. Paris} \textbf{173}, 677–680 (1921).

        \bibitem{Gullstrand1922}
        A. Gullstrand, ``Allgemeine L\"osung des statischen Eink\"orperproblems in der Einsteinschen Gravitationstheorie'',
        \emph{Ark. Mat. Astron. Fys.} \textbf{16}, 1–15 (1922).

        \bibitem{Unruh1981}
        W. G. Unruh, ``Experimental black-hole evaporation?'',
        \emph{Phys.\ Rev.\ Lett.} \textbf{46}, 1351–1353 (1981). doi:10.1103/PhysRevLett.46.1351.

        \bibitem{Visser1998}
        M. Visser, ``Acoustic black holes: horizons, ergospheres and Hawking radiation'',
        \emph{Class.\ Quantum Grav.} \textbf{15}, 1767–1791 (1998). doi:10.1088/0264-9381/15/6/024.

        \bibitem{Kerr1963}
        R. P. Kerr, ``Gravitational field of a spinning mass as an example of algebraically special metrics'',
        \emph{Phys.\ Rev.\ Lett.} \textbf{11}, 237–238 (1963). doi:10.1103/PhysRevLett.11.237.

        \bibitem{Batchelor1967}
        G. K. Batchelor, \emph{An Introduction to Fluid Dynamics}, Cambridge Univ. Press, 1967.

        \bibitem{LandauFluids}
        L. D. Landau and E. M. Lifshitz, \emph{Fluid Mechanics}, 2nd ed., Pergamon, 1987.
    \end{thebibliography}

%================== end v0.6 Delta — Conclusions from VAM 16–20 ==================


%===================== Add to §10.1 Constants Table =====================
% Append this row anywhere sensible in the existing table

    | $\Lambda$ | Swirl Coulomb constant (EM coupling) | $4\pi\,\rho_{\text{\ae}}^{(\text{mass})}\,C_e^2\,r_c^4$ $= 2.30707733\times 10^{-28}$ | J·m |

%===================== Add to §10.2 Boxed Canon Equations ===============
% Append these two boxed identities to the list in §10.2

    \[
        \boxed{\, \Lambda \;=\; \int_{S_r^2} p_{\text{swirl}}\,r^2\,d\Omega
        \;=\; 4\pi\,\rho_{\text{\ae}}^{(\text{mass})}\,C_e^2\,r_c^4
        \;=\; \frac{e^2}{4\pi\varepsilon_0} \,}
        \quad\text{[units: J·m]}
    \]

    \[
        \boxed{\,
        \left[-\frac{\hbar^2}{2\mu}\nabla^2 - \frac{\Lambda}{\sqrt{r^2+r_c^2}}\right]\psi = E\,\psi
        \;\xrightarrow{\,r\gg r_c\,}\;
        \left[-\frac{\hbar^2}{2\mu}\nabla^2 - \frac{\Lambda}{r}\right]\psi = E\,\psi
        \,}
    \]
%=======================================================================
%=========== Corollary to G.1 (place at end of the metric section) =====
    \paragraph*{Corollary (circulation–metric link).}
    With azimuthal drift $v_\theta(r)$, the PG-type analogue metric implies
    \[
        \boxed{\, g_{t\theta}^{\text{(VAM)}} \;=\; v_\theta(r)\,r \;=\; \frac{1}{2\pi}\,\Gamma_{\text{swirl}}(r) \,},
        \qquad \Gamma_{\text{swirl}}(r) := \oint v_\theta\,dl.
    \]
    This is the VAM counterpart of GR frame-dragging ($g_{t\phi}$) for axisymmetric rotation.
%=======================================================================




\end{document}
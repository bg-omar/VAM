% ===================== VAM Canon (v0.3 — Definitive Core) =====================
\documentclass[11pt,a4paper]{article}
\usepackage{amsmath,amssymb,bm}
\usepackage{siunitx}
\usepackage{physics}
\usepackage{geometry}
\usepackage{hyperref}
\geometry{margin=1in}
\sisetup{per-mode=symbol,round-mode=figures,round-precision=6}

% ---------- Macros ----------
\newcommand{\aeFluid}{\rho_{\text{\ae}}^{(\text{fluid})}}
\newcommand{\aeMass}{\rho_{\text{\ae}}^{(\text{mass})}}
\newcommand{\aeEnergy}{\rho_{\text{\ae}}^{(\text{energy})}}
\newcommand{\Ce}{C_e}
\newcommand{\rc}{r_c}
\newcommand{\tp}{t_p}
\newcommand{\aeforce}{F_{\text{\ae}}^{\max}}
\newcommand{\Lag}{\mathcal{L}}
\newcommand{\Ham}{\mathcal{H}}
\newcommand{\Om}{\Omega_{\text{swirl}}}
\newcommand{\phig}{\varphi}

\title{VAM Canon (v0.3 — Definitive Core)}
\author{Omar Iskandarani}
\date{2025--08--22}

\begin{document}
    \maketitle

    \begin{abstract}
        This Canon states the axioms, definitions, theorems, boxed master equations, and validation rules of the Vortex \AE{}ther Model (VAM). It consolidates v0.1--v0.2 (through \S12.6) and adds five rigorously promotable identities as canonical: (i) the swirl Coulomb constant $\Lambda$ and hydrogen soft-core, (ii) the circulation--metric corollary, (iii) the corrected time-rate law, (iv) the dimensionally normalized swirl Hamiltonian, and (v) the Euler corollary for the swirl pressure law. Non-canonical calibrations and research tracks remain out of this Core.
    \end{abstract}

    \tableofcontents

% ===================== 1) Postulates =====================
    \section{Core Postulates (Canonical)}
    \label{sec:postulates}
    \begin{enumerate}
        \item \textbf{Absolute time, Euclidean space ($\mathbb{R}^3$).} A universal clock defines a preferred foliation (A-time).
        \item \textbf{Incompressible, inviscid æther.} Ideal Euler dynamics; density $\aeFluid$ is constant at macroscales.
        \item \textbf{Particles = knotted vortex solitons.} Matter are closed, possibly linked/knotted filaments; bosons as unknots.
        \item \textbf{Gravity = structured swirl.} Macroscopic attraction arises from coherent vorticity and pressure gradients; Newton’s $G$ emerges via $G_{\text{swirl}}$.
        \item \textbf{Quantization from topology and circulation.} Discrete quantum numbers trace to linking/writhe/twist and circulation quantization.
        \item \textbf{Kelvin–Helmholtz invariants govern dynamics.} Circulation conservation and helicity underpin stability and reconnection phenomenology.
    \end{enumerate}

% ===================== 2) Master Equations =====================
    \section{Master Equations (Boxed, Definitive)}
    \label{sec:masters}

    \subsection{Master Energy and Mass Formula}
    Define the amplified swirl energy for a coherent VAM volume $V$:
    \[
        \boxed{\, E_{\text{VAM}}(V) = \frac{4}{\alpha\,\phig} \left( \frac{1}{2}\,\aeFluid\,\Ce^{2} \right) V \,}\quad [\text{J}]
    \]
    \[
        \boxed{\, M_{\text{VAM}}(V) = \frac{E_{\text{VAM}}(V)}{c^{2}} \,}\quad [\text{kg}]
    \]

    \subsection{Swirl Gravitational Coupling}
    \[
        \boxed{\, G_{\text{swirl}} = \frac{\Ce\,c^{5}\,\tp^{2}}{2\,\aeforce\,\rc^{2}} \,}\quad [\text{m}^3\,\text{kg}^{-1}\,\text{s}^{-2}]
    \]

    \subsection{Local Time-Rate (Swirl Clock) — \emph{operative}}
    \label{sec:time-rate}
    \[
        \boxed{\, \frac{dt_{\text{local}}}{dt_{\infty}} \;=\; \sqrt{\,1 - \frac{\lVert\vec{\omega}\rVert^{2}\,\rc^{2}}{c^{2}}\,}
        \;=\; \sqrt{\,1 - \frac{v_t^{2}}{c^{2}}\,},\qquad v_t:=\lVert\vec{\omega}\rVert\,\rc \,}
    \]
    \textit{Historical variant (deprecated, kept for traceability):} $dt_{\text{local}} = dt_{\infty}\sqrt{1 - \lVert\vec{\omega}\rVert^{2}/c^{2}}$.

    \subsection{Swirl Angular Frequency Profile}
    \[
        \boxed{\, \Om(r) = \frac{\Ce}{\rc}\, e^{-r/\rc} \,}\quad \Rightarrow\ \Om(0)=\Ce/\rc
    \]

    \subsection{Vorticity Potential (Canonical Form)}
    \[
        \boxed{\, \Phi(\vec r,\vec\omega) = \frac{\Ce^{2}}{2\,\aeforce}\,\vec\omega\cdot\vec r \,}
    \]
    (Use within the VAM Lagrangian; propagate units consistently.)

% ===================== 3) Foundational identities =====================
    \section{Foundational Identities (Conservation Laws)}
    \label{sec:foundational-identities}
    \textit{Assumptions: inviscid, barotropic æther; body forces derivable from a potential.}
    \begin{itemize}
        \item \textbf{Kelvin circulation:} $\dfrac{d\Gamma}{dt} = 0$ along a material loop. \cite{Kelvin1869,Helmholtz1858}
        \item \textbf{Vorticity transport:} $\dfrac{\partial\vec{\omega}}{\partial t} = \nabla \times (\vec{v} \times \vec{\omega})$.
        \item \textbf{Kinetic helicity density:} $h = \vec{v} \cdot \vec{\omega}$; \textbf{Helicity invariant:} $H = \int (\vec{v} \cdot \vec{\omega})\,dV$ (up to reconnections). \cite{Moffatt1969}
    \end{itemize}

% ===================== 4) Unified Lagrangian & Hamiltonian =====================
    \section{Unified VAM Lagrangian and Swirl Hamiltonian}
    \label{sec:lagrangian}
    Let $\vec v$ be the æther velocity, $\aeFluid$ constant (incompressible), $\vec\omega=\nabla\times\vec v$, and $p$ a Lagrange multiplier enforcing incompressibility.
    \[
        \boxed{\, \Lag_{\text{VAM}} =
            \frac{1}{2}\aeFluid\,\lVert\vec v\rVert^{2}
            - \aeFluid\,\Phi(\vec r,\vec\omega)
            + \lambda(\nabla\cdot\vec v)
            + \eta\,\mathcal{H}[\vec v]
            + \mathcal{L}_{\text{couple}}[\Gamma,\mathcal{K}] \,}
    \]
    with $\mathcal{H}[\vec v] = \int (\vec v\cdot\vec\omega)\,dV$ and $\mathcal{L}_{\text{couple}}$ encoding circulation/knot invariants.

    \paragraph{Canonical Swirl Hamiltonian (dimensionally normalized).}
    \[
        \boxed{\, \Ham[\vec v] =
            \frac{1}{2}\aeFluid\,\lVert\vec v\rVert^{2}
            + \frac{1}{2}\aeFluid\,\rc^{2}\,\lVert\vec\omega\rVert^{2}
            + \lambda\,(\nabla\cdot\vec v) \,}
    \]
    On-axis with $\omega=\Om(0)=\Ce/\rc$, the $\lVert\omega\rVert^2$ term reduces to $\tfrac12\,\aeFluid\,\Ce^2$, matching the bulk swirl energy density. \cite{Batchelor1967,LandauFluids}

% ===================== 5) Notation / Ontology =====================
    \section{Notation, Ontology, and Glossary}
    \begin{itemize}
        \item \textbf{A-time:} absolute æther time. \quad
        \textbf{C-time:} asymptotic observer time ($dt_{\infty}$).
        \item \textbf{Swirl Clock:} local rate set by $\lVert\vec\omega\rVert$ via \S\ref{sec:time-rate}.
        \item \textbf{Knot Taxonomy:} leptons $\leftrightarrow$ torus knots; quarks $\leftrightarrow$ chiral hyperbolic; bosons $\leftrightarrow$ unknots.
    \end{itemize}

% ===================== 6) Canonical Checks =====================
    \section{Canonical Checks (Required in Every Paper)}
    \begin{enumerate}
        \item Dimensional analysis on every new term/equation.
        \item Limiting behavior: $\lVert\omega\rVert\to 0$ recovers classical/EM; large averages reproduce Newtonian gravity with $G_{\text{swirl}}$.
        \item Numerical validation: provide prefactors using Canon constants; any new constants must be added to \S10.1.
        \item Explicit topology $\leftrightarrow$ quantum mapping (which invariants, how normalized).
        \item BibTeX for every non-original construct (see \S\ref{sec:bib}).
    \end{enumerate}

% ===================== 7) Persona / Session protocol (kept terse) =====================
    \section{Session Protocol (Abbrev.)}
    Start new task with this Canon, declare corrections that persist for the session, and record Canon deltas with version bump at end.

% ===================== 9) Citations skeleton =====================
    \section{Canon-Ready Citations (Skeleton)}
    \label{sec:bib}
    Provide BibTeX keys for all non-original laws used in the manuscript (see references at document end).

% ===================== 10) Appendix: Canon Tables =====================
    \section*{10) Appendix: Canon Tables}
    \subsection*{10.2 Boxed Canon Equations (paste-ready)}
    \begin{enumerate}
        \item \fbox{$E_{\text{VAM}} = \dfrac{4}{\alpha\phig}\left(\dfrac{1}{2}\aeFluid \Ce^2\right)V$}
        \item \fbox{$M_{\text{VAM}} = E_{\text{VAM}}/c^2$}
        \item \fbox{$G_{\text{swirl}} = \dfrac{\Ce c^5 \tp^2}{2\aeforce \rc^2}$}
        \item \fbox{$\dfrac{dt_{\text{local}}}{dt_{\infty}}=\sqrt{1-\lVert\omega\rVert^2 \rc^2/c^2}$}
        \item \fbox{$\Om(r) = \dfrac{\Ce}{\rc}e^{-r/\rc}$}
    \end{enumerate}

% ===================== 11) Change Log v0.1 =====================
    \section*{11) Change Log}
    \begin{itemize}
        \item \textbf{v0.1 (2025-08-22):} Initial Canon with core postulates, constants, boxed master equations, Lagrangian, and checks.
    \end{itemize}

% ===================== 12) v0.2 Deltas =====================
    \section*{12) v0.2 Delta — Corrections \& Additions (2025-08-22)}
    \subsection*{12.1 Dimensional correction to time-rate law}
    \textit{Operative form is \S\ref{sec:time-rate}; the historical variant is deprecated.}
    \subsection*{12.2 Canon tolerances \& symbol aliases}
    Numerical tolerances: relative $\le 10^{-6}$; absolute near zero $\le 10^{-12}$ (SI). Accepted aliases must be normalized into the Canon constants table.
    \subsection*{12.3 Validation protocol updates}
    \begin{enumerate}
        \item Ensure $\aeFluid \Phi$ has energy-density units (J\,m$^{-3}$) within \S\ref{sec:lagrangian}.
        \item When swirl/time enters, reduce via $v_t=|\omega|\rc$ unless a different scale is justified.
        \item Provide a short table with substituted Canon constants and 3--5 s.f. results.
    \end{enumerate}
    \subsection*{12.6 Change Log entry}
    \textit{As above; v0.2 content retained.}

% ===================== 13) v0.3 Canon Extensions (promoted) =====================
    \section{v0.3 Canon Extensions (Promoted to Core)}
    \label{sec:v03-extensions}

    \subsection{Swirl Coulomb Constant and Hydrogen Soft-Core}
    \label{sec:Lambda}
    \[
        \boxed{\, \Lambda \;\equiv\; \int_{S_r^2} p_{\text{swirl}}\,r^2\,d\Omega
        \;=\; 4\pi\,\aeMass\,\Ce^{2}\,\rc^{4}
            \;=\; \frac{e^{2}}{4\pi\varepsilon_0} \,}
        \qquad [\Lambda]=\text{J$\cdot$m}.
    \]
    \[
        \boxed{\, V_{\text{VAM}}(r) = -\frac{\Lambda}{\sqrt{r^2+\rc^2}}
        \;\xrightarrow{\,r\gg \rc\,}\;
        -\frac{\Lambda}{r} \,}
    \]
    Schrödinger hydrogen recovers Bohr/Rydberg limits with $e^2/(4\pi\varepsilon_0)\to\Lambda$. \cite{Jackson1999,Schrodinger1926}

    \subsection{Circulation–Metric Corollary (Frame-Dragging Analogue)}
    \label{sec:metric-corr}
    For axisymmetric swirl with azimuthal drift $v_\theta(r)$ in a PG-type analogue metric,
    \[
        \boxed{\, g^{(\mathrm{VAM})}_{t\theta} \;=\; r\,v_\theta(r) \;=\; \frac{1}{2\pi}\,\Gamma_{\text{swirl}}(r),\qquad
        \Gamma_{\text{swirl}}(r):=\oint v_\theta\,dl \,}
    \]
    linking the mixed metric term to Kelvin circulation. \cite{Painleve1921,Gullstrand1922,Unruh1981,Visser1998,Kerr1963}

    \subsection{Corrected Time-Rate Law (Consolidation)}
    \label{sec:timerate-consolidation}
    The operative law is \S\ref{sec:time-rate}; the non-normalized variant is deprecated.

    \subsection{Swirl Hamiltonian Density (Consolidation)}
    \label{sec:Hamiltonian-consolidation}
    The normalized Hamiltonian in \S\ref{sec:lagrangian} is part of the Canon. In the limit $\rc\to 0$ it reduces to the bulk swirl energy density. \cite{Batchelor1967,LandauFluids}

    \subsection{Swirl Pressure Law (Euler Corollary; Canonical)}
    \label{sec:darkpressure}
    For steady, purely azimuthal drift $v(r)$ with no radial flow, radial Euler balance yields
    \[
        \boxed{\, \frac{1}{\aeFluid}\,\frac{dp_{\text{swirl}}}{dr} \;=\; \frac{v(r)^{2}}{r} \,}
    \]
    and for an asymptotically flat curve $v(r)\to v_0$,
    \[
        p_{\text{swirl}}(r) = p_0 + \aeFluid\,v_0^2\,\ln\!\frac{r}{r_0}\,.
    \]
    This is an identity-level corollary of Euler’s equation (no empirical fit). \cite{Batchelor1967,LandauFluids}

% ===================== References =====================
    \begin{thebibliography}{99}

        \bibitem{Helmholtz1858}
        H. von Helmholtz, ``On Integrals of the Hydrodynamical Equations which Express Vortex-motion'',
        \emph{Philosophical Magazine} (1858).

        \bibitem{Kelvin1869}
        W. Thomson (Lord Kelvin), ``On Vortex Motion'',
        \emph{Trans. Roy. Soc. Edinburgh} (1869).

        \bibitem{Moffatt1969}
        H. K. Moffatt, ``The degree of knottedness of tangled vortex lines'',
        \emph{J. Fluid Mech.} \textbf{35}, 117–129 (1969).

        \bibitem{Batchelor1967}
        G. K. Batchelor, \emph{An Introduction to Fluid Dynamics}, Cambridge Univ. Press (1967).

        \bibitem{LandauFluids}
        L. D. Landau and E. M. Lifshitz, \emph{Fluid Mechanics} (2nd ed.), Pergamon (1987).

        \bibitem{Schrodinger1926}
        E. Schr\"odinger, ``An Undulatory Theory of the Mechanics of Atoms and Molecules'',
        \emph{Phys. Rev.} \textbf{28}, 1049–1070 (1926). \href{https://doi.org/10.1103/PhysRev.28.1049}{doi:10.1103/PhysRev.28.1049}.

        \bibitem{Jackson1999}
        J. D. Jackson, \emph{Classical Electrodynamics} (3rd ed.), Wiley (1999).

        \bibitem{Painleve1921}
        P. Painlev\'e, ``La m\'ecanique classique et la th\'eorie de la relativit\'e'',
        \emph{C. R. Acad. Sci. Paris} \textbf{173}, 677–680 (1921).

        \bibitem{Gullstrand1922}
        A. Gullstrand, ``Allgemeine L\"osung des statischen Eink\"orperproblems in der Einsteinschen Gravitationstheorie'',
        \emph{Ark. Mat. Astron. Fys.} \textbf{16}, 1–15 (1922).

        \bibitem{Unruh1981}
        W. G. Unruh, ``Experimental black-hole evaporation?'',
        \emph{Phys. Rev. Lett.} \textbf{46}, 1351–1353 (1981).
        \href{https://doi.org/10.1103/PhysRevLett.46.1351}{doi:10.1103/PhysRevLett.46.1351}.

        \bibitem{Visser1998}
        M. Visser, ``Acoustic black holes: horizons, ergospheres and Hawking radiation'',
        \emph{Class. Quantum Grav.} \textbf{15}, 1767–1791 (1998).
        \href{https://doi.org/10.1088/0264-9381/15/6/024}{doi:10.1088/0264-9381/15/6/024}.

        \bibitem{Kerr1963}
        R. P. Kerr, ``Gravitational field of a spinning mass as an example of algebraically special metrics'',
        \emph{Phys. Rev. Lett.} \textbf{11}, 237–238 (1963).
        \href{https://doi.org/10.1103/PhysRevLett.11.237}{doi:10.1103/PhysRevLett.11.237}.
    \end{thebibliography}

\end{document}
% ===================== End VAM Canon v0.3 =====================
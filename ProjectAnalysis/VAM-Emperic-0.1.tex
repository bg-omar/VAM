%! Author = omar.iskandarani
%! Date = 8/22/2025

% Preamble
\documentclass[11pt]{article}

% Packages
\usepackage{amsmath}

% Document
\begin{document}



    \subsection*{10.1 Constants Table (paste-ready)}

    \begin{table}[h!]
        \centering
        \begin{tabular}{|l|l|l|l|}
            \hline
            \textbf{Symbol} & \textbf{Meaning} & \textbf{Value} & \textbf{Unit} \\
            \hline
            $C_e$ & Vortex tangential velocity & $1.09384563\times 10^{6}$ & m\,s$^{-1}$ \\
            $r_c$ & Vortex-core radius & $1.40897017\times 10^{-15}$ & m \\
            $\rho_{\text{\ae}}^{(\text{fluid})}$ & Æther fluid density & $7.0\times 10^{-7}$ & kg\,m$^{-3}$ \\
            $\rho_{\text{\ae}}^{(\text{mass})}$ & Æther mass density & $3.8934358266918687\times 10^{18}$ & kg\,m$^{-3}$ \\
            $\rho_{\text{\ae}}^{(\text{energy})}$ & Æther energy density & $3.49924562\times 10^{35}$ & J\,m$^{-3}$ \\
            $F_{\text{\ae}}^{\max}$ & Max. Coulomb force & 29.053507 & N \\
            $F_{\text{gr}}^{\max}$ & Max. universal force & $3.02563\times 10^{43}$ & N \\
            $\alpha$ & Fine-structure constant & $7.2973525643\times 10^{-3}$ & --- \\
            $\varphi$ & Golden ratio & $1.61803398875$ & --- \\
            $c$ & Speed of light & 299792458 & m\,s$^{-1}$ \\
            $t_p$ & Planck time & $5.391247\times 10^{-44}$ & s \\
            \hline
        \end{tabular}
        \caption{Canonical constants for VAM (SI units unless stated).}
    \end{table}


    %-------------------------------------------------------------
    \subsection*{C2) Hydrogen Schr\"odinger equation with core softening}
    \label{subsec:hydrogen-soft-core}

    VAM replaces the Coulomb term by a swirl-induced softened potential
    \[
        \boxed{ V_{\text{VAM}}(r) \;=\; -\,\frac{\Lambda}{\sqrt{r^2+r_c^2}}
        \;\;\xrightarrow{r\gg r_c}\;\; -\frac{\Lambda}{r} }
    \]
    and the bound-state equation
    \[
        \boxed{ \left[-\frac{\hbar^2}{2\mu}\nabla^2 - \frac{\Lambda}{\sqrt{r^2+r_c^2}}\right]\psi \;=\; E\,\psi }
    \]
    recovers the textbook spectrum for $r\gg r_c$.
    \textbf{Bohr/Rydberg checks (H):}
    $a_0=\hbar^2/(\mu\Lambda)=5.29177262\times10^{-11}\,\mathrm{m}$,
    $E_{1}=\mu\Lambda^2/(2\hbar^2)=13.60569\,\mathrm{eV}$.
    Core softening gives $nS$ shifts $\sim \mathcal{O}\!\big((r_c/a_0)^2\big)\approx 7.1\times10^{-10}$ (H), and
    $\sim \mathcal{O}(2.4\times10^{-5})$ for muonic H (using $\mu\approx 186\,m_e$),
    i.e. a ground-state scale $\sim 6\times10^{-2}$\,eV --- a concrete experimental target.

    \textit{Non-original equations: Schr\"odinger hydrogen and Coulomb limit \cite{Schrodinger1926,Jackson1999}.}







    \section{v0.3 Draft Delta --- Core from VAM 0--4 (Einstein $\rightarrow$ Vortex Fluid)}
    \textbf{Status: DRAFT (pending promotion to sections 1--5 after review)}

    \subsection{Source batch (chronological TeX)}
    Parsed: \texttt{VAM\_0--4\_Einstein\_to\_Vortex\_Fluid} (TeX-first corpus)\\
    Artifacts indexed (TeX-aware): 2335 equation blocks; 268 constant definitions/assignments; 246 postulate-like sentences; structured outline per file.

    \subsection{Consolidated core postulates (canonical wording)}
    \begin{enumerate}
        \item \textbf{Absolute time, Euclidean space ($\mathbb{R}^3$).} A universal ``clock field'' defines a preferred foliation consistent with VAM's absolute æther time.
        \item \textbf{Incompressible, inviscid æther.} Background medium supports ideal Euler dynamics; density $\rho_{\text{\ae}}^{(\text{fluid})}$ is constant at macroscales.
        \item \textbf{Particles = knotted vortex solitons.} Matter is realized as closed, possibly linked/knotted filaments; bosons as unknotted excitations.
        \item \textbf{Gravity = structured swirl.} Macroscopic attraction emerges from coherent vorticity fields and pressure gradients; Newton's $G$ is recovered via $G_{\text{swirl}}$.
        \item \textbf{Quantization from topology and circulation.} Discrete quantum numbers trace to linking/writhe/twist and circulation quantization.
        \item \textbf{Kelvin--Helmholtz invariants govern dynamics.} Circulation conservation and helicity underpin stability, reconnection energetics, and decay.
    \end{enumerate}
    \textit{These six are promoted to Canon~\S1 after approval. Existing~\S1 will be rephrased to this exact minimal set.}

    \subsection{Canon conservation laws (add to \S3: ``Foundational identities'')}
    \begin{itemize}
        \item \textbf{Kelvin circulation (inviscid, barotropic):} $\frac{d\Gamma}{dt} = 0$ along a material loop.
        \item \textbf{Vorticity transport (Euler):} $\frac{\partial\vec{\omega}}{\partial t} = \nabla \times (\vec{v} \times \vec{\omega})$.
        \item \textbf{Kinetic helicity density:} $h = \vec{v} \cdot \vec{\omega}$; \textbf{Helicity invariant:} $H = \int (\vec{v} \cdot \vec{\omega})\,dV$ (up to reconnection events).
    \end{itemize}
    \textit{Rationale: These appear repeatedly across VAM 0--4 and are required to justify knot stability and reconnection phenomenology. They are background identities; use BibTeX keys in~\S9 (Helmholtz/Kelvin/Moffatt).}

    \subsection{Key equations shortlist (from VAM 0--4)}
    \begin{itemize}
        \item \textbf{Swirl profile:} $\Omega_{\text{swirl}}(r) = \frac{C_e}{r_c} \exp(-r/r_c)$ (consistent with Canon~\S3.4).
        \item \textbf{Time-rate (dimensionally corrected):} $dt_{\text{local}} = dt_{\infty} \sqrt{1 - |\omega|^2 r_c^2 / c^2} = dt_{\infty} \sqrt{1 - v_t^2/c^2}$.
        \item \textbf{Mass/Energy:} $E_{\text{VAM}} = \frac{4}{\alpha\varphi} \left(\frac{1}{2} \rho_{\text{\ae}}^{(\text{fluid})} C_e^2\right) V$, $M = E_{\text{VAM}}/c^2$.
        \item \textbf{G coupling:} $G_{\text{swirl}} = \frac{C_e c^5 t_p^2}{2 F_{\text{\ae}}^{\max} r_c^2}$.
        \item \textbf{Helicity/Lagrangian:} Canon~\S4 form with $H[\vec{v}] = \int (\vec{v} \cdot \vec{\omega})\,dV$ and incompressibility via $\lambda (\nabla\cdot\vec{v})$.
    \end{itemize}
    (Full ledger with file pointers is in the generated CSV: \texttt{equations\_shortlist.csv}.)

    \subsection{Canon constants concordance (snapshot)}
    The TeX sources define/mention aliases for: $C_e$, $r_c$, $\rho_{\text{\ae}}^{(\text{fluid|mass|energy})}$, $\alpha$, $c$, $t_p$, $\varphi$, $F_{\text{\ae}}^{\max}$, $F_{\text{gr}}^{\max}$.

    \textbf{Action:} Enforce the v0.2 alias table (Sec.~12.2). Manuscripts must include a normalized constants table per Canon~\S10.1.

    \subsection{Organization rule for VAM parts (Canon policy)}
    Each VAM ``part'' must answer in its abstract:
    \begin{enumerate}
        \item \textbf{Unique role:} What principle or equation does this part introduce that no other part covers?
        \item \textbf{Dependence:} Which Canon sections/parts are prerequisites?
        \item \textbf{Promotion path:} Which equations/postulates are candidates to move into Canon~\S\S1--5 after validation?
    \end{enumerate}

    \subsection{Promotion plan}
    \begin{itemize}
        \item \textbf{Promote} \S13.2 postulates to Canon~\S1 (replace/merge wording) after you approve.
        \item \textbf{Add} \S13.3 conservation laws as Canon~\S3A (``Foundational identities'').
        \item \textbf{Relabel} old~\S3.3 (time-rate) as ``historical'' and keep~\S12.1 as the operative law; mirror the operative law into~\S3 with $r_c$.
        \item \textbf{Append} a permanent ``Chronology note'' linking VAM 0--4 to the Canon (\S0 Versioning $\rightarrow$ provenance).
    \end{itemize}

    \subsection{Citations to add in \S9 (BibTeX keys)}
    \begin{itemize}
        \item \texttt{Kelvin1869} --- Circulation theorem.
        \item \texttt{Helmholtz1858} --- Vortex motion integrals.
        \item \texttt{Moffatt1969} --- Helicity/topological knottedness.\\
        (Keep existing entries; ensure all non-original laws are cited.)
    \end{itemize}

    \subsection{Generated indices for this batch (local paths)}
    \begin{itemize}
        \item Outline (titles/sections): \texttt{vam\_corpus\_reports\_vam0\_4/outline.csv}
        \item Key equations (categorized): \texttt{vam\_corpus\_reports\_vam0\_4/equations\_categorized.csv}
        \item Equations shortlist: \texttt{vam\_corpus\_reports\_vam0\_4/equations\_shortlist.csv}
        \item Canon constants concordance: \texttt{vam\_corpus\_reports\_vam0\_4/canon\_concordance.csv}
        \item Postulates shortlist: \texttt{vam\_corpus\_reports\_vam0\_4/postulates\_shortlist.csv}
    \end{itemize}

    \textit{End of v0.3 draft delta.}









    \section{VAM Canon v0.4 Delta: Derived Constants, Galactic Swirl Law, and Baryon Mass Map}
    \label{sec:vam-canon-v0.4-delta}

    \subsection{Canon Identities (to be promoted to \S3B)}
    \begin{itemize}
        \item \textbf{Fine-structure constant from swirl speed:}
        \[
            \boxed{\alpha = \frac{2 C_e}{c}} \qquad \Longleftrightarrow \qquad \boxed{C_e = \frac{c\,\alpha}{2}}
        \]
        Dimensionality: velocity ratio $\rightarrow$ dimensionless (OK). Numerical check (Canon values): $\alpha=0.007297352557$.

        \item \textbf{Gravitational fine-structure constant:}
        \[
            \boxed{\alpha_g = \frac{C_e^{2} t_p^{2}}{r_c^{2}}}
        \]
        (dimensionless); $\ell_p \equiv c\,t_p$, $\ell_p^{2}=c^{2}t_p^{2}$.

        \item \textbf{Equivalents for $G$:}
        \[
            \boxed{G = \frac{\alpha_g c^{3} r_c}{C_e M_e} = \frac{C_e c\, \ell_p^{2}}{r_c M_e}}
        \]
        Using the VAM identity:
        \[
            \boxed{M_e = \frac{2 F_{\text{\ae}}^{\max} r_c}{c^{2}}}
        \]
        This equals the existing Canon coupling:
        \[
            \boxed{G_{\text{swirl}}=\frac{C_e c^{5} t_p^{2}}{2 F_{\text{\ae}}^{\max} r_c^{2}}}
        \]
        Numerical check (Canon values): $G=6.6743013\times10^{-11}\;\text{m}^3\,\text{kg}^{-1}\,\text{s}^{-2}$.
    \end{itemize}

    \subsection{Galactic Swirl Law (Disc Kinematics)}
    A two-component velocity profile that captures solid-body rise and asymptotic flattening:
    \[
        \boxed{v(r) = \frac{C_{\text{core}}}{\sqrt{1 + (r_c/r)^2}} + C_{\text{tail}}\,\big(1 - e^{-r/r_c}\big)}
    \]
    Limits: $v(0)=0$; $v(r\to\infty)=C_{\text{core}}+C_{\text{tail}}$. Small-$r$: core term $\sim (C_{\text{core}}/r_c)\,r$. Large-$r$: exponential approach governed by $r_c$.

    \subsection{Baryon Mass Relations (VAM Knot Map)}
    Let $M_u, M_d$ denote the effective VAM up/down knot masses. Then:
    \[
        \boxed{M_p = \varphi^{-2}\,3^{-1/\varphi}\,(2 M_u + M_d)}\,,\qquad
        \boxed{M_n = \varphi^{-2}\,3^{-1/\varphi}\,(M_u + 2 M_d)}
    \]

    \subsection{Added/Derived Constants (Append to \S10.1)}
    \begin{table}[h!]
        \centering
        \begin{tabular}{|l|l|l|l|}
            \hline
            \textbf{Symbol} & \textbf{Meaning} & \textbf{Value} & \textbf{Unit} \\
            \hline
            $M_e$ & Electron mass (derived in VAM) & $\dfrac{2 F_{\text{\ae}}^{\max} r_c}{c^2}$ $= 9.109383\times10^{-31}$ & kg \\
            $\ell_p$ & Planck length & $c t_p$ $= 1.616255\times10^{-35}$ & m \\
            $\alpha$ & Fine-structure (derived) & $2C_e/c$ $= 7.29735256\times10^{-3}$ & --- \\
            $\alpha_g$ & Gravitational fine-structure & $C_e^{2} t_p^{2}/r_c^{2}$ $= 1.75181\times10^{-45}$ & --- \\
            \hline
        \end{tabular}
        \caption{Added/derived constants for VAM Canon v0.4}
        \label{tab:added-derived-constants}
    \end{table}

    \subsection{Dimensional Validations}
    \begin{itemize}
        \item $[\alpha]=[\alpha_g]=1$
        \item $[G]=\mathrm{L}^3\,\mathrm{M}^{-1}\,\mathrm{T}^{-2}$ from either boxed $G$ identity
        \item $[v(r)]=\mathrm{L}\,\mathrm{T}^{-1}$
    \end{itemize}

















    \chapter*{VAM Canon — v0.5 Selections (ready-to-merge)}

    \textit{Batch: VAM\_9--15 (Spacetime, Dark Sector \& Quantum Gravity)}

    \section*{G) Boxed selections from VAM 9--15 --- to merge into Canon v0.5}

    \subsection*{G.1 Effective metric / line element (axisymmetric swirl)}

    Steady, incompressible, azimuthal drift $v_\theta(r)$ in cylindrical $(t,r,\theta,z)$:

    \[
        \boxed{
            ds^2 = -\left(c^2 - v_\theta(r)^2\right) dt^2 + 2 v_\theta(r) r\, d\theta\, dt + dr^2 + r^2 d\theta^2 + dz^2
        }
    \]

    In the co-rotating frame $\theta' = \theta - \int v_\theta(r)\,dt / r$, the cross term diagonalizes locally:

    \[
        \boxed{
            ds^2 = -c^2\left(1 - \frac{v_\theta(r)^2}{c^2}\right) dt^2 + dr^2 + r^2 d\theta'^2 + dz^2
        }
    \]

    This exposes the swirl-clock factor and matches Sec.~12.1/3.3 in the $v_\theta \ll c$ regime. Substitute your swirl law as needed, e.g. $v_\theta(r) = r\,\Omega_{\text{swirl}}(r)$ with $\Omega_{\text{swirl}}(r) = \frac{C_e}{r_c} e^{-r/r_c}$.

    \textbf{BibTeX (analogue/PG background):} Unruh1981, Visser1998, Painleve1921, Gullstrand1922, Batchelor1967.

    \subsection*{G.2 Swirl Hamiltonian density for Sec.~4 (dimensionally normalized)}

    Take $\rho = \rho_{\ae}^{(\text{fluid})}$, $\vec{\omega} = \nabla \times \vec{v}$, $\lambda$ for incompressibility. A quadratic, Kelvin-compatible kernel:

    \[
        \boxed{
            \mathcal{H}[\vec{v}] = \frac{1}{2} \rho \lVert \vec{v} \rVert^2 + \frac{1}{2} \rho \ell_\omega^2 \lVert \vec{\omega} \rVert^2 + \frac{1}{2} \rho \ell_\omega^4 \lVert \nabla \vec{\omega} \rVert^2 + \lambda (\nabla \cdot \vec{v})
        }
        \qquad \ell_\omega := r_c
    \]

    Units check: $[\rho \lVert \vec{v} \rVert^2] = \text{J}\,\text{m}^{-3}$; since $[\omega] = \text{s}^{-1}$, the coefficients $\rho, \ell_\omega^2$ and $\rho, \ell_\omega^4$ ensure the $|\omega|^2$ and $|\nabla \omega|^2$ terms also have energy-density units. In the $\ell_\omega \to 0$ limit this reduces to the bulk swirl energy.

    \textit{(Optional minimal matter--swirl coupling, same section):}

    \[
        \boxed{
            \mathcal{H}_\psi = \frac{\hbar^2}{2m} \left\lVert \left(\nabla - i\,\frac{m}{\hbar} \vec{A}_{\text{swirl}}\right) \psi \right\rVert^2 + U(|\psi|^2)
        }
        \qquad \vec{A}_{\text{swirl}} := \chi\,\vec{v}
    \]

    \subsection*{G.3 Dark-sector law beside $v(r)$ (Sec.~14.2)}

    Radial Euler balance (steady, no radial flow) yields

    \[
        0 = -\frac{1}{\rho} \frac{dp_{\text{swirl}}}{dr} + \frac{v(r)^2}{r}
        \qquad \Rightarrow \qquad
        \boxed{
            a_{\text{dark}}(r) \equiv \frac{1}{\rho} \frac{dp_{\text{swirl}}}{dr} = \frac{v(r)^2}{r}
        }
    \]

    Equivalently as a pressure law paired with the swirl profile $v(r)$:

    \[
        \boxed{
            \frac{dp_{\text{swirl}}}{dr} = \rho\,\frac{v(r)^2}{r}
        }
        \qquad \Longrightarrow \qquad
        \boxed{
            p_{\text{swirl}}(r) = p_0 + \rho\,v_0^2\,\ln\left(\frac{r}{r_0}\right)
        }
        \quad (\text{flat } v(r) \to v_0)
    \]

    \textit{Sign convention:} the inward centripetal requirement corresponds to an outward-rising pressure ($dp/dr > 0$) so that $-\nabla p/\rho$ supplies the inward acceleration.

    \subsection*{G.4 Consistency vs Canon v0.1--v0.4}

    \begin{itemize}
        \item Time-rate: metric's $g_{tt}$ gives $dt_{\text{local}}/dt_\infty = \sqrt{1- v_\theta^2/c^2}$, consistent with Sec.~12.1 choice $v_t = |\omega| r$ at $r = r_c$.
        \item Galactic law: use Sec.~14.2 $v(r)$ in G.3 to obtain explicit $p_{\text{swirl}}(r)$ in both core and tail limits.
        \item Dimensions: all boxed terms reduce to SI units with $\ell_\omega = r_c$ and $\rho = \rho_{\ae}^{(\text{fluid})}$.
    \end{itemize}

    \textbf{Ready to merge into Canon v0.5:} place G.1 in Sec.~3A/Sec.~6, G.2 in Sec.~4 (Hamiltonian), and G.3 alongside Sec.~14.2.

%===================== v0.6 Delta — Conclusions from VAM 16–20 =====================

    \section*{v0.6 Delta — Conclusions from VAM 16–20}
    \label{sec:v06-delta}

    \paragraph{Scope.}
    We consolidated the main outcomes of VAM-16–20 (Zero-Vorticity Line, photon/EM mapping, Kerr reinterpretation, vortex-string EFT, and Schr\"odinger hydrogen). Below are the boxed identities and consistency results that are ready for canonicalization (with dimensional and numerical checks). Items needing a policy decision are explicitly flagged.

%-------------------------------------------------------------
    \subsection*{C1) EM coupling emerges from core swirl pressure}
    \label{subsec:coulomb-from-swirl}

    Define the \emph{swirl Coulomb constant} via the pressure integral over a spherical surface:
    \[
        \boxed{ \Lambda \;\equiv\; \int_{S_r^2} p_{\text{swirl}}\,r^2\,d\Omega
        \;=\; 4\pi\,\rho_{\text{\ae}}^{(\text{mass})}\,C_e^2\,r_c^4 }
    \]
    Dimensions: $[\Lambda]=\mathrm{N\,m^2}=\mathrm{J\,m}$ (Coulomb constant units).
    \emph{Identification:}
    \[
        \boxed{ \Lambda \;=\; \frac{e^2}{4\pi\varepsilon_0} } \qquad \text{(EM coupling).}
    \]
    \textbf{Numerics (Canon values):} $\Lambda=2.30707733\times10^{-28}\,\mathrm{J\,m}$, matching $e^2/(4\pi\varepsilon_0)$ to $\lesssim10^{-7}$ relative.
    This promotes a tight constraint among $(\rho_{\text{\ae}}^{(\text{mass})},\,C_e,\,r_c)$ consistent with $\alpha = 2C_e/c$.
    \textbf{Status:} \emph{Promote} as Canon identity in \S3B and append $\Lambda$ to the constants table.

    \textit{Non-original comparison: Coulomb constant \cite{Jackson1999}.}



%-------------------------------------------------------------
    \subsection*{C3) Frame-dragging / off-diagonal metric term as circulation}
    \label{subsec:metric-circulation}

    The PG-type analogue line element already in Canon (\S\,G.1) implies the mixed term
    \[
        \boxed{ g_{t\theta}^{\text{(VAM)}} \;=\; v_\theta(r)\,r \;=\; \frac{1}{2\pi}\,\Gamma_{\text{swirl}}(r) }
    \]
    with $\Gamma_{\text{swirl}}(r)=\oint v_\theta\,dl$ the azimuthal circulation at radius $r$.
    This is the precise VAM counterpart of GR's $g_{t\phi}$ frame-dragging structure for axisymmetric rotation, dovetailing with the Kerr reinterpretation draft.\footnote{In BL-like gauges one can factor $c^2$ to yield a dimensionless coefficient; the PG form used in Canon keeps $g_{t\theta}$ in velocity\,$\times$\,length units, matching the analogue-gravity construction.}
    \textbf{Status:} \emph{Promote} the boxed relation as a corollary to \S\,G.1.

    \textit{Non-original background: analogue/PG metrics and Kerr solution \cite{Painleve1921,Gullstrand1922,Unruh1981,Visser1998,Kerr1963}.}

%-------------------------------------------------------------
    \subsection*{C4) Hamiltonian/Lagrangian usage: $\rho^{(\text{fluid})}$ vs $\rho^{(\text{mass})}$}
    \label{subsec:rho-policy}

    Across VAM-16–20 two distinct densities appear:
    \begin{align*}
        &\text{Bulk swirl energetics (Canon \S4, G.2):}
        &&\frac{1}{2}\,\rho_{\text{\ae}}^{(\text{fluid})}\,\lVert \vec v\rVert^2
        \;+\;\frac{1}{2}\,\rho_{\text{\ae}}^{(\text{fluid})}\,r_c^{2}\lVert \vec\omega\rVert^2
        \;+\;\cdots \\
        &\text{EM coupling (\S\,\ref{subsec:coulomb-from-swirl}):}
        &&\Lambda = 4\pi\,\rho_{\text{\ae}}^{(\text{mass})}\,C_e^2\,r_c^4\,.
    \end{align*}
    \textbf{Conclusion (policy):} retain $\rho_{\text{\ae}}^{(\text{fluid})}$ in the \emph{Hamiltonian kernel} for continuum energetics (Canon \S4), and reserve $\rho_{\text{\ae}}^{(\text{mass})}$ for \emph{core/EM coupling} identities (\S\,C1). This resolves the apparent density ``swap'' without changing any numerics.

    \textit{Non-original context: continuum energy density forms \cite{LandauFluids}.}

%-------------------------------------------------------------
    \subsection*{C5) Time-rate law: confirm $r_c$ factor and note draft variance}
    \label{subsec:timerate-consistency}

    Some 16–20 drafts reused the historical form $dt_{\text{local}}/dt_\infty=\sqrt{1- \lVert\omega\rVert^2/C_e^2}$.
    \textbf{Canon-consistent law (dimensionally correct):}
    \[
        \boxed{ \frac{dt_{\text{local}}}{dt_\infty} \;=\; \sqrt{\,1- \frac{\lVert\omega\rVert^2\,r_c^2}{c^2}\,}
        \;=\; \sqrt{\,1- \frac{v_t^2}{c^2}\,},\;\;v_t:=\lVert\omega\rVert r_c }
    \]
    \emph{Action:} keep only the $r_c$-normalized law in Canon (\S\,12.1/\S\,3.3); mark the non-normalized variant as deprecated.

%-------------------------------------------------------------


%--------------------------- BibTeX additions ---------------------------
% Add these to your .bib (non-original references cited above)
    \begin{thebibliography}{9}
        \bibitem{Jackson1999}
        J. D. Jackson, \emph{Classical Electrodynamics}, 3rd ed., Wiley, 1999.

        \bibitem{Schrodinger1926}
        E. Schr\"odinger, ``An Undulatory Theory of the Mechanics of Atoms and Molecules'',
        \emph{Phys.\ Rev.} \textbf{28}, 1049–1070 (1926). doi:10.1103/PhysRev.28.1049.

        \bibitem{Painleve1921}
        P. Painlev\'e, ``La m\'ecanique classique et la th\'eorie de la relativit\'e'',
        \emph{C. R. Acad. Sci. Paris} \textbf{173}, 677–680 (1921).

        \bibitem{Gullstrand1922}
        A. Gullstrand, ``Allgemeine L\"osung des statischen Eink\"orperproblems in der Einsteinschen Gravitationstheorie'',
        \emph{Ark. Mat. Astron. Fys.} \textbf{16}, 1–15 (1922).

        \bibitem{Unruh1981}
        W. G. Unruh, ``Experimental black-hole evaporation?'',
        \emph{Phys.\ Rev.\ Lett.} \textbf{46}, 1351–1353 (1981). doi:10.1103/PhysRevLett.46.1351.

        \bibitem{Visser1998}
        M. Visser, ``Acoustic black holes: horizons, ergospheres and Hawking radiation'',
        \emph{Class.\ Quantum Grav.} \textbf{15}, 1767–1791 (1998). doi:10.1088/0264-9381/15/6/024.

        \bibitem{Kerr1963}
        R. P. Kerr, ``Gravitational field of a spinning mass as an example of algebraically special metrics'',
        \emph{Phys.\ Rev.\ Lett.} \textbf{11}, 237–238 (1963). doi:10.1103/PhysRevLett.11.237.

        \bibitem{Batchelor1967}
        G. K. Batchelor, \emph{An Introduction to Fluid Dynamics}, Cambridge Univ. Press, 1967.

        \bibitem{LandauFluids}
        L. D. Landau and E. M. Lifshitz, \emph{Fluid Mechanics}, 2nd ed., Pergamon, 1987.
    \end{thebibliography}

%================== end v0.6 Delta — Conclusions from VAM 16–20 ==================


%===================== Add to §10.1 Constants Table =====================
% Append this row anywhere sensible in the existing table

    | $\Lambda$ | Swirl Coulomb constant (EM coupling) | $4\pi\,\rho_{\text{\ae}}^{(\text{mass})}\,C_e^2\,r_c^4$ $= 2.30707733\times 10^{-28}$ | J·m |

%===================== Add to §10.2 Boxed Canon Equations ===============
% Append these two boxed identities to the list in §10.2

    \[
        \boxed{\, \Lambda \;=\; \int_{S_r^2} p_{\text{swirl}}\,r^2\,d\Omega
        \;=\; 4\pi\,\rho_{\text{\ae}}^{(\text{mass})}\,C_e^2\,r_c^4
        \;=\; \frac{e^2}{4\pi\varepsilon_0} \,}
        \quad\text{[units: J·m]}
    \]

    \[
        \boxed{\,
        \left[-\frac{\hbar^2}{2\mu}\nabla^2 - \frac{\Lambda}{\sqrt{r^2+r_c^2}}\right]\psi = E\,\psi
        \;\xrightarrow{\,r\gg r_c\,}\;
        \left[-\frac{\hbar^2}{2\mu}\nabla^2 - \frac{\Lambda}{r}\right]\psi = E\,\psi
        \,}
    \]
%=======================================================================
%=========== Corollary to G.1 (place at end of the metric section) =====
    \paragraph*{Corollary (circulation–metric link).}
    With azimuthal drift $v_\theta(r)$, the PG-type analogue metric implies
    \[
        \boxed{\, g_{t\theta}^{\text{(VAM)}} \;=\; v_\theta(r)\,r \;=\; \frac{1}{2\pi}\,\Gamma_{\text{swirl}}(r) \,},
        \qquad \Gamma_{\text{swirl}}(r) := \oint v_\theta\,dl.
    \]
    This is the VAM counterpart of GR frame-dragging ($g_{t\phi}$) for axisymmetric rotation.
%=======================================================================


    \usepackage{amsmath, amssymb}
    \usepackage{siunitx}
    \usepackage{hyperref}
    \usepackage{geometry}
    \usepackage{physics}
    \usepackage{bm}
    \usepackage{upgreek}
    \usepackage{graphicx}
    \geometry{margin=1in}
    \sisetup{per-mode=symbol,round-mode=figures,round-precision=6}

    \newcommand{\aeFluid}{\rho_{\text{\ae}}^{(\text{fluid})}}
    \newcommand{\aeMass}{\rho_{\text{\ae}}^{(\text{mass})}}
    \newcommand{\aeEnergy}{\rho_{\text{\ae}}^{(\text{energy})}}
    \newcommand{\Ce}{C_e}
    \newcommand{\rc}{r_c}
    \newcommand{\tp}{t_p}
    \newcommand{\aeforce}{F_{\text{\ae}}^{\max}}
    \newcommand{\Lag}{\mathcal{L}}
    \newcommand{\Ham}{\mathcal{H}}
    \newcommand{\Om}{\Omega_{\text{swirl}}}
    \newcommand{\Lam}{\Lambda_{\text{swirl}}}
    \newcommand{\phig}{\varphi}

% --- BibTeX block for journals that require it (covers all non-original formulas/ideas) ---
    \begin{filecontents*}{canon-extensions.bib}
        @article{Helmholtz1858,
        author  = {H. von Helmholtz},
        title   = {On Integrals of the Hydrodynamical Equations which Express Vortex-motion},
        journal = {Philosophical Magazine},
        year    = {1858}
        }
        @article{Kelvin1869,
        author  = {W. Thomson (Lord Kelvin)},
        title   = {On Vortex Motion},
        journal = {Transactions of the Royal Society of Edinburgh},
        year    = {1869}
        }
        @article{Moffatt1969,
        author  = {H. K. Moffatt},
        title   = {The degree of knottedness of tangled vortex lines},
        journal = {Journal of Fluid Mechanics},
        year    = {1969}
        }
        @book{Batchelor1967,
        author = {G. K. Batchelor},
        title  = {An Introduction to Fluid Dynamics},
        year   = {1967},
        publisher = {Cambridge University Press}
        }
        @book{LandauLifshitz1987,
        author = {L. D. Landau and E. M. Lifshitz},
        title  = {Fluid Mechanics (2nd ed.)},
        year   = {1987},
        publisher = {Pergamon}
        }
        @article{Schrodinger1926,
        author  = {E. Schr{\"o}dinger},
        title   = {An Undulatory Theory of the Mechanics of Atoms and Molecules},
        journal = {Physical Review},
        year    = {1926},
        doi     = {10.1103/PhysRev.28.1049}
        }
        @article{Unruh1981,
        author  = {W. G. Unruh},
        title   = {Experimental black-hole evaporation?},
        journal = {Physical Review Letters},
        year    = {1981},
        doi     = {10.1103/PhysRevLett.46.1351}
        }
        @article{Visser1998,
        author  = {M. Visser},
        title   = {Acoustic black holes: horizons, ergospheres and Hawking radiation},
        journal = {Classical and Quantum Gravity},
        year    = {1998},
        doi     = {10.1088/0264-9381/15/6/024}
        }
        @article{Painleve1921,
        author  = {P. Painlev{\'e}},
        title   = {La m{\'e}canique classique et la th{\'e}orie de la relativit{\'e}},
        journal = {C. R. Acad. Sci. Paris},
        year    = {1921}
        }
        @article{Gullstrand1922,
        author  = {A. Gullstrand},
        title   = {Allgemeine L{\"o}sung des statischen Eink{\"o}rperproblems in der Einsteinschen Gravitationstheorie},
        journal = {Arkiv f{\"o}r Matematik, Astronomi och Fysik},
        year    = {1922}
        }
        @book{Jackson1999,
        author = {J. D. Jackson},
        title  = {Classical Electrodynamics (3rd ed.)},
        year   = {1999},
        publisher = {Wiley}
        }
        @article{Kerr1963,
        author  = {R. P. Kerr},
        title   = {Gravitational field of a spinning mass as an example of algebraically special metrics},
        journal = {Physical Review Letters},
        year    = {1963},
        doi     = {10.1103/PhysRevLett.11.237}
        }
        @article{White1969,
        author  = {J. H. White},
        title   = {Self-linking and the Gauss integral in higher dimensions},
        journal = {American Journal of Mathematics},
        year    = {1969}
        }
        @article{Calugareanu1961,
        author  = {G. C\u{a}lug\u{a}reanu},
        title   = {Sur les classes d'isotopie des noeuds tridimensionnels et leurs invariants},
        journal = {Czechoslovak Mathematical Journal},
        year    = {1961}
        }
        @article{Wien1894,
        author  = {W. Wien},
        title   = {Ueber die Energieverteilung im Emissionsspectrum eines schwarzen K{\"o}rpers},
        journal = {Annalen der Physik},
        year    = {1894}
        }
        @article{Planck1901,
        author  = {M. Planck},
        title   = {On the Law of Distribution of Energy in the Normal Spectrum},
        journal = {Annalen der Physik},
        year    = {1901}
        }
    \end{filecontents*}

    \title{VAM Canon (v0.7-Extensions) \\ \large Canonical Enhancements from \emph{Before-VAM-and-Experiments.zip} and \emph{VAM-ONGOING-RESEARCHES.zip}}
    \author{Omar Iskandarani}
    \date{2025--08--22}


    \maketitle

    \begin{abstract}
        This document \emph{extends} the \textbf{VAM Canon (v0.1)} with items that are ready for canonicalization, distilled from the two corpora of uploaded work:
        \emph{Before-VAM-and-Experiments.zip} and \emph{VAM-ONGOING-RESEARCHES.zip}. We add: (i) foundational conservation laws (Kelvin, vorticity-transport, helicity) as \S3A, (ii) an effective metric and a circulation--metric link, (iii) a dimensionally normalized swirl Hamiltonian, (iv) a dark-sector pressure law aligned with flat rotation curves, (v) the \emph{swirl Coulomb constant} identity \(\Lam\) with hydrogen soft-core spectrum and numerical validation, and (vi) experimental validation protocols for \(\Ce=f\,\Delta x\) and for the swirl gravitational potential. Research-track notes from the blackbody/QED/knot-taxonomy files are included as non-canonical appendices.
    \end{abstract}

    \section*{Canon Delta Summary (Promote to Core)}
    \begin{enumerate}
        \item \textbf{Foundational identities (\S\ref{sec:foundational-identities}).} Kelvin circulation, vorticity transport, and helicity invariants \cite{Helmholtz1858,Kelvin1869,Moffatt1969,Batchelor1967,LandauLifshitz1987}.
        \item \textbf{Analogue/PG line element (\S\ref{sec:metric}).} Axisymmetric swirl metric with cross-term \(g_{t\theta}\) and \emph{corollary}: \(g^{(\mathrm{VAM})}_{t\theta}= r\,v_\theta(r) = \Gamma_{\text{swirl}}(r)/(2\pi)\) \cite{Unruh1981,Visser1998,Painleve1921,Gullstrand1922,Kerr1963}.
        \item \textbf{Swirl Hamiltonian (\S\ref{sec:hamiltonian}).} Kelvin-compatible, dimensionally normalized kernel with \(\ell_\omega=\rc\) and incompressibility constraint.
        \item \textbf{Dark-sector pressure law (\S\ref{sec:darkpressure}).} For steady azimuthal drift, \( \frac{1}{\rho}\,\dv{p_{\text{swirl}}}{r} = \frac{v(r)^2}{r}\) and \(p_{\text{swirl}}(r)=p_0+\rho v_0^2\ln(r/r_0)\) for flat \(v(r)\).
        \item \textbf{Swirl Coulomb constant and hydrogen (\S\ref{sec:lambda-hydrogen}).} \(\displaystyle \Lam = 4\pi \aeMass \Ce^2 \rc^4\) and the soft-core potential \(V(r)=-\Lam/\sqrt{r^2+\rc^2}\), recovering Bohr and Rydberg limits \cite{Jackson1999,Schrodinger1926}.
        \item \textbf{Experimental protocols (\S\ref{sec:experiments}).} Canon-ready protocols extracted from \texttt{appendix\_C} and \texttt{appendix\_D} files for validating \(\Ce\) and the swirl potential.
    \end{enumerate}

    \section{Foundational Identities (Add as Canon \S3A)}
    \label{sec:foundational-identities}
    Let \(\vb v\) be the æther velocity (\(\nabla\cdot \vb v=0\)), \(\bm{\omega}=\nabla\times \vb v\). For inviscid, barotropic flow \cite{Helmholtz1858,Kelvin1869,Batchelor1967,LandauLifshitz1987}:
    \begin{align}
        &\textbf{Kelvin circulation:} && \frac{d\Gamma}{dt}=0,\quad \Gamma=\oint_{\mathcal{C}(t)} \vb v\cdot d\vb \ell. \tag{F1} \label{eq:kelvin} \\
        &\textbf{Vorticity transport:} && \pdv{\bm{\omega}}{t} = \nabla\times(\vb v\times \bm{\omega}). \tag{F2} \label{eq:vorticity-transport}\\
        &\textbf{Helicity:} && h=\vb v\cdot \bm{\omega},\quad H=\int h\, dV\ \text{(invariant up to reconnections)}.~\cite{Moffatt1969} \tag{F3}\label{eq:helicity}
    \end{align}
    These underpin knotted-solition stability and reconnection energetics in VAM.

    \section{Axisymmetric Swirl Metric and Circulation Link}
    \label{sec:metric}
    In cylindrical \((t,r,\theta,z)\) with steady azimuthal drift \(v_\theta(r)\), adopt the Painlevé–Gullstrand analogue form \cite{Unruh1981,Visser1998,Painleve1921,Gullstrand1922}:
    \begin{align}
        ds^2 &= -\big(c^2 - v_\theta(r)^2\big)\,dt^2 + 2\,v_\theta(r)\,r\, d\theta\, dt + dr^2 + r^2 d\theta^2 + dz^2. \tag{M1}\label{eq:pg} \\
        \intertext{Co-rotating with \(\theta'=\theta-\!\int \! v_\theta(r)\,dt/r\) gives}
        ds^2 &= -c^2\!\left(1-\frac{v_\theta(r)^2}{c^2}\right)dt^2 + dr^2 + r^2 d\theta'^2 + dz^2, \tag{M2}
    \end{align}
    so the swirl-clock factor is \(dt_{\text{local}}/dt_\infty=\sqrt{1-v_\theta^2/c^2}\). \emph{Corollary (frame-dragging analogue):}
    \begin{equation}
        g^{(\mathrm{VAM})}_{t\theta} = r\,v_\theta(r) = \frac{1}{2\pi}\,\Gamma_{\text{swirl}}(r),\qquad \Gamma_{\text{swirl}}(r):=\oint v_\theta\, dl. \tag{M3}\label{eq:gtth}
    \end{equation}

    \section{Swirl Hamiltonian Density (Add to Canon \S4)}
    \label{sec:hamiltonian}
    With \(\rho=\aeFluid\), \(\bm{\omega}=\nabla\times \vb v\), and Lagrange multiplier \(\lambda\) for incompressibility, a Kelvin-compatible, dimensionally normalized kernel is
    \begin{equation}
        \Ham[\vb v] = \frac{1}{2}\rho\,\|\vb v\|^2 + \frac{1}{2}\rho\,\ell_\omega^2\,\|\bm{\omega}\|^2 + \frac{1}{2}\rho\,\ell_\omega^4\,\|\nabla \bm{\omega}\|^2 + \lambda (\nabla\cdot \vb v),\qquad \ell_\omega:=\rc. \tag{H1}\label{eq:Hamiltonian}
    \end{equation}
    All terms carry units of energy density (J\,m\(^{-3}\)). In the \(\ell_\omega\to 0\) limit this reduces to the bulk swirl energy used in Canon v0.1.

    \section{Dark-Sector Pressure Law (Place next to galactic \(v(r)\))}
    \label{sec:darkpressure}
    For steady, purely azimuthal drift \(v(r)\) and no radial flow, the radial Euler balance gives
    \begin{equation}
        0=-\frac{1}{\rho}\frac{dp_{\text{swirl}}}{dr}+\frac{v(r)^2}{r}
        \quad\Longrightarrow\quad
        \boxed{\ \frac{1}{\rho}\frac{dp_{\text{swirl}}}{dr}=\frac{v(r)^2}{r}\ } . \tag{D1}\label{eq:darklaw}
    \end{equation}
    For an asymptotically flat curve \(v(r)\to v_0\), integration yields
    \begin{equation}
        p_{\text{swirl}}(r)=p_0+\rho\,v_0^2\,\ln\!\frac{r}{r_0}. \tag{D2}
    \end{equation}
    Sign: outward-rising \(p\) produces inward acceleration \(-\nabla p/\rho\).

    \section{Swirl Coulomb Constant and Hydrogen Soft-Core}
    \label{sec:lambda-hydrogen}
    \subsection{Identity and dimensions}
    Define the \emph{swirl Coulomb constant} via the surface integral of swirl pressure over the sphere \(S_r^2\) (consistent with the experimental appendices and EM mapping notes):
    \begin{equation}
        \boxed{\ \Lam \equiv \int_{S_r^2} p_{\text{swirl}}\, r^2\, d\Omega \;=\; 4\pi\,\aeMass\,\Ce^2\,\rc^4\ } \quad [\Lam]=\mathrm{J\,m} = \mathrm{N\,m^2}. \tag{E1}\label{eq:LambdaDef}
    \end{equation}
    In VAM hydrogen, replace the Coulomb term by a softened potential
    \begin{equation}
        V_{\text{VAM}}(r) = -\frac{\Lam}{\sqrt{r^2+\rc^2}} \xrightarrow{r\gg \rc} -\frac{\Lam}{r}. \tag{E2}\label{eq:softcore}
    \end{equation}
    \subsection{Schr\"odinger equation and recovery of textbook limits \cite{Schrodinger1926,Jackson1999}}
    The bound-state equation
    \begin{equation}
        \left[-\frac{\hbar^2}{2\mu}\nabla^2 - \frac{\Lam}{\sqrt{r^2+\rc^2}}\right]\psi = E\,\psi \xrightarrow{r\gg \rc} \left[-\frac{\hbar^2}{2\mu}\nabla^2 - \frac{\Lam}{r}\right]\psi = E\,\psi . \tag{E3}
    \end{equation}
    Using \(\mu\approx m_e\), the Bohr radius and ground energy are recovered with \(\Lam\) in place of \(e^2/(4\pi\varepsilon_0)\):
    \begin{equation}
        a_0=\frac{\hbar^2}{\mu \Lam},\qquad E_1=\frac{\mu \Lam^2}{2\hbar^2}. \tag{E4}
    \end{equation}
    \paragraph{Numerical validation (Canon constants).}
    With \(\Ce=\num{1.09384563e6}\,\si{m/s}\), \(\rc=\num{1.40897017e-15}\,\si{m}\), \(\aeMass=\num{3.8934358266918687e18}\,\si{kg/m^3}\):
    \begin{align}
        \Lam &= 4\pi\,\aeMass\,\Ce^2\,\rc^4 = \num{2.3070773276484373e-28}\ \si{J.m}, \tag{E5}\\
        a_0 &= \num{5.2917726179579395e-11}\ \si{m},\qquad
        E_1 = \num{2.1798719391487416e-18}\ \si{J} = \num{13.605690489359251}\ \si{eV}. \tag{E6}
    \end{align}
    These match the textbook hydrogen values to within numerical tolerance, validating the identification of \(\Lam\).

    \section{Experimental Protocols (Canon-ready)}
    \label{sec:experiments}
    \subsection{Appendix C: Universality of \(\Ce=f\,\Delta x\) (metrology across platforms)}
    From \texttt{appendix\_C\_ExperimentalValidationOfVortexCoreTangientalVelocity.tex}: measure a natural frequency \(f\) and a spatial step \(\Delta x\) from standing/propagating modes; verify
    \begin{equation}
        \boxed{\ \Ce = f\,\Delta x \approx \num{1.09384563e6}\ \si{m/s}\ } . \tag{X1}
    \end{equation}
    Platforms: magnet/electret domains, laser interferometry on coil-bound modes, and acoustic analogues. Require ppm-level agreement; report mean and standard deviation across platforms.

    \subsection{Appendix D: Swirl gravitational potential}
    From \texttt{appendix\_D\_ExperimentalValidationOfGravitationalPotential.tex}: infer \(p_{\text{swirl}}(r)\) from centripetal balance (\S\ref{sec:darkpressure}) and compare predicted forces with measured thrust or buoyancy anomalies in shielded high-voltage/coil experiments (geometry: starship/Rodin coils). Ensure dimensional consistency and calibrate only via Canon constants.

    \section*{Policy Notes and Clarifications}
    \paragraph{Density usage.} Use \(\aeFluid\) in continuum energetics (\S\ref{sec:hamiltonian}); reserve \(\aeMass\) for core/EM coupling identities (\S\ref{sec:lambda-hydrogen}).
    \paragraph{Time-rate law.} Canon operative form (dimensionally correct): \( dt_{\text{local}}/dt_\infty = \sqrt{1-\|\bm{\omega}\|^2 \rc^2 / c^2} = \sqrt{1-v_t^2/c^2}\) with \(v_t:=\|\bm{\omega}\|\rc\).

    \appendix
    \section{Research Track (non-canonical yet)}
    \subsection{Blackbody via Swirl Temperature (from \texttt{BlackBody\_fromWein\_toNewEM.md})}
    \textit{Proposal.} Define a swirl temperature \(T_{\text{swirl}}\) via local vortex energy density and map Wien/Planck spectra by substituting \(\Lam\) in place of \(e^2/(4\pi\varepsilon_0)\). Requires a precise constitutive link between \(T\) and \(\|\bm{\omega}\|^2\); cite \cite{Wien1894,Planck1901}.

    \subsection{QED--VAM Mapping Notes (from \texttt{QED\_VAM\_RESEARCH\_NOTES.md})}
    \textit{Sketch.} Minimal coupling \(\nabla\!\to\!\nabla - i\frac{m}{\hbar}\bm{A}_{\text{swirl}}\) with \(\bm{A}_{\text{swirl}}=\chi\,\vb v\) inside \(\Ham\) (cf. \eqref{eq:Hamiltonian}); action \(S\) parallels circulation \(\Gamma\). Canonization deferred pending gauge-structure tests.

    \subsection{Knot Taxonomy Refinement}
    Use the Călugăreanu–White–Fuller relation \(Lk=Tw+Wr\) \cite{Calugareanu1961,White1969} to sharpen torus/hyperbolic assignments, and to parametrize chirality (matter/antimatter) via sign of \(Tw\).

    \section*{Numerical Snapshot of Canon Identities}
    \begin{align}
        \alpha &= \frac{2\Ce}{c} = \num{0.007297352557148052},\quad
        G_{\text{swirl}} = \frac{\Ce\,c^5 \tp^2}{2\aeforce\,\rc^2} = \num{6.674302004898925e-11}\ \si{m^3\,kg^{-1}\,s^{-2}},\\
        \Om(0) &= \frac{\Ce}{\rc} = \num{7.763440655383073e20}\ \si{s^{-1}},\quad
        \Lam = \num{2.3070773276484373e-28}\ \si{J.m}.
    \end{align}

    \vspace{1em}
    \noindent\textbf{Change Log for v0.7-Extensions (2025-08-22).} Added \S3A identities; \S\ref{sec:metric} metric and circulation corollary; \S\ref{sec:hamiltonian} Hamiltonian; \S\ref{sec:darkpressure} pressure law; \S\ref{sec:lambda-hydrogen} with numerical validation; \S\ref{sec:experiments} protocols; research appendices.




    \section{Clustered New Insights for VAM}

%-------------------------------
    \subsection{A. Knot-theoretic parametrization \& mass law (items 1,2,3,4,5,6,18)}
    \begin{enumerate}
        \item[\textbf{[E1]}] \textbf{Knot invariants as parameters:}\quad
        n=\gcd(p,q),\quad m=p\ \text{(or braid strand count)},\quad
        s=g=\frac{(p-1)(q-1)}{2}.
        \item[\textbf{[E2]}] \textbf{VAM mass law with torus invariants:}\quad
        \(
        M(p,q,V_i)=\Big(\frac{4}{\alpha}\Big)\,p^{-3/2}\,\phi^{-\frac{(p-1)(q-1)}{2}}\,[\gcd(p,q)]^{-1/\phi}\,
        \sum_i V_i\Big(\frac{1}{2\rho_{\ae}C_e^2}\Big).
        \)
        \item[\textbf{[E3]}] \textbf{General torus-link choices:}\quad
        \(m\equiv b(T)=\min(|p|,|q|),\quad
        s\equiv g(T)=\frac{(|p|-1)(|q|-1)-(n-1)}{2}.\)
        \item[\textbf{[E4]}] \textbf{Dimension correction (mass from energy density):}\quad
        \(M\propto\Big(\tfrac12\rho_{\ae}C_e^2\Big)\,\dfrac{\sum_i V_i}{c^2}.\)
        \item[\textbf{[E5]}] \textbf{Core volume via rope length:}\quad
        \(\displaystyle \sum_i V_i(T)=\pi r_c^3\,\mathcal{L}_{\mathrm{tot}}(T).\)
        \item[\textbf{[E6]}] \textbf{Invariant mass law (rope-length form):}\quad
        \(
        M(T(p,q))=\Big(\frac{4}{\alpha}\Big)b(T)^{-3/2}\,\phi^{-g(T)}\,n(T)^{-1/\phi}
        \Big(\frac{1}{2}\rho_{\ae}C_e^2\Big)\,
        \frac{\pi r_c^3\,\mathcal{L}_{\mathrm{tot}}(T)}{c^2}.
        \)
        \item[\textbf{[E18]}] \textbf{Compact mass expression:}\quad
        \(
        M=\frac{1}{\phi}\,\frac{4}{\alpha}\,\Big(\tfrac12\rho_{\ae}C_e^2\,V\Big).
        \)
    \end{enumerate}

    \paragraph{Explanation (Group A).}
    This family replaces heuristic “suppression constants” with knot invariants \((n,m,g)\) so the mass law is intrinsically topological.
    Rope length \(\mathcal{L}_{\mathrm{tot}}\) and core radius \(r_c\) provide a geometric scale for \(\sum_i V_i\).
    The factors \(c^{-2}\) and \(\rho_{\ae}C_e^2\) ensure dimensional consistency (energy-to-mass).
    [E2] and [E6] are functionally similar; [E6] is the invariant rope-length form, preferred for comparing different knots.

%-------------------------------
    \subsection{B. Helicity-based topological charge \& energy functionals (items 7,8,9,19)}
    \begin{enumerate}
        \item[\textbf{[E7]}] \textbf{Vortex helicity (topological charge):}\quad
        \(\displaystyle H_{\text{vortex}}=\frac{1}{(4\pi)^2}\int \vec v\cdot\vec\omega\,d^3x,\qquad H_{\text{vortex}}=nH_0.\)
        \item[\textbf{[E8]}] \textbf{Dimensionless normalization:}\quad
        \(\displaystyle Q_{\text{top}}=\frac{L}{(4\pi)^2\Gamma^2}\int \vec v\cdot\vec\omega\,d^3x.\)
        \item[\textbf{[E9]}] \textbf{Topological energy term and total energy:}\quad
        \(\mathcal{L}_{\text{top}}=\frac{C_e^2}{2}\rho_{\ae}\,\vec v\cdot\vec\omega,\quad
        \mathcal{E}_{\text{VAM}}=\int\!\big[\tfrac12\rho_{\ae}|\vec v|^2+\tfrac{C_e^2}{2}\rho_{\ae}\vec v\cdot\vec\omega+\Phi_{\text{swirl}}+P(\rho_{\ae})\big]\,d^3x.\)
        \item[\textbf{[E19]}] \textbf{Swirl potential (proposal):}\quad
        \(\displaystyle \Phi(r)=\frac{C_e^2}{2F_{\max}}\ \vec\omega(r)\cdot\vec r.\)
    \end{enumerate}

    \paragraph{Explanation (Group B).}
    [E7]–[E8] define (normalized) helicity as a measure of linkage/knottedness of vortex lines; [E8] makes \(Q_{\text{top}}\) comparable across configurations.
    [E9] couples topology directly to energy via \(\vec v\cdot\vec\omega\).
    [E19] is a modeling choice for \(\Phi\) that ties swirl fields to pressure/tension; it can be tuned for stability criteria.

%-------------------------------
    \subsection{C. Split helicity: writhe, twist \& time (items 10,11,12)}
    \begin{enumerate}
        \item[\textbf{[E10]}] \textbf{Split with flux \(C\):}\quad
        \(H=H_C+H_T,\quad H_C=C^2\,\mathrm{Wr},\quad H_T=C^2\,\mathrm{Tw},\quad H=C^2(\mathrm{Wr}+\mathrm{Tw}).\)
        \item[\textbf{[E11]}] \textbf{Writhe kernel (Gauss-like):}\quad
        \(\displaystyle \mathrm{Wr}=\frac{1}{4\pi}\int_C\!\!\int_C
        \frac{\big(\mathbf T(s)\times \mathbf T(s')\big)\cdot\big(\mathbf r(s)-\mathbf r(s')\big)}
        {|\mathbf r(s)-\mathbf r(s')|^3}\,ds\,ds'.\)
        \item[\textbf{[E12]}] \textbf{Time dilation with split helicity:}\quad
        \(\displaystyle dt=dt_\infty\,\sqrt{1-\frac{H_C+H_T}{H_{\max}}}.\)
    \end{enumerate}

    \paragraph{Explanation (Group C).}
    Total helicity decomposes into shape (\(\mathrm{Wr}\)) and torsion (\(\mathrm{Tw}\)) parts, weighted by flux \(C\).
    [E11] gives the geometric, reparameterization-invariant definition of writhe along the centerline.
    [E12] links topology to the local clock rate in VAM: greater knottedness/torsion increases time dilation.

%-------------------------------
    \subsection{D. ISF/wavefunction formulation \& Hopf map (items 13,14,15,16,17)}
    \begin{enumerate}
        \item[\textbf{[E13]}] \textbf{ISF-like VAM Lagrangian (two-component \(\psi\)):}\quad
        \(\displaystyle
        \mathcal{L}_\text{VAM}=\frac{i\hbar}{2}\!\left(\psi^\dagger \partial_t\psi-\psi\partial_t\psi^\dagger\right)
        -\frac{\hbar^2}{2m}|\nabla\psi|^2-\frac{\alpha}{8}\,|\nabla\vec s|^2.
        \)
        \item[\textbf{[E14]}] \textbf{Hopf spin from \(\psi=\begin{pmatrix}a+ib\\ c+id\end{pmatrix}\):}\quad
        \(\;s_1=a^2+b^2-c^2-d^2,\ \ s_2=2(bc-ad),\ \ s_3=2(ac+bd).\)
        \item[\textbf{[E15]}] \textbf{Euler–Lagrange with topological feedback:}\quad
        \(\displaystyle i\hbar\,\partial_t\psi=-\frac{\hbar^2}{2m}\nabla^2\psi+\frac{\alpha}{4}\,\frac{\delta}{\delta\psi^*}\,|\nabla\vec s|^2.\)
        \item[\textbf{[E16]}] \textbf{Madelung/incompressible ansatz:}\quad
        \(\psi=\sqrt{\rho_{\ae}}\,e^{i\theta},\quad \vec v=\nabla\theta,\quad |\psi|^2=\rho_{\ae}=\text{const}\ \Rightarrow\ \nabla\cdot\vec v=0.\)
        \item[\textbf{[E17]}] \textbf{``Quantum pressure'' analogue in VAM:}\quad
        \(\displaystyle Q_{\text{VAM}}=\frac{1}{2}\rho_{\ae}\,|\vec\omega|^2.\)
    \end{enumerate}

    \paragraph{Explanation (Group D).}
    This family casts vortex knots into a wavefunction framework: \(\psi\) evolves via [E15], while \(\vec s(\psi)\) (Hopf map) encodes topology.
    The Madelung ansatz [E16] connects phase to velocity and enforces incompressibility (ISF).
    [E17] replaces standard quantum pressure with a vorticity-based tension, consistent with VAM’s energy density.
    Together, this yields simulation-ready, topology-sensitive dynamics.

    \appendix
    \section{VAM Radiation Laws and Exotic Modes}
    \label{app:VAM-radiation}

    This appendix consolidates the Vortex \AE{}ther Model (VAM) derivations
    related to radiation laws, spectral behavior, and exotic non-EM wave modes.
    Equations are grouped by thematic category.

% ============================================================
    \subsection{Ætheric Energy–Temperature Coupling}

    The rotational kinetic energy density of the æther field is
    \[
        U_{\text{rot}} = \tfrac{1}{2} \rho_\text{\ae} |\vec{\omega}|^2.
    \]
    By analogy with kinetic theory, this rotational energy can be equated to
    thermal agitation in a coarse-graining cell of volume $V_{\text{cell}}$:
    \[
        k_B T \sim \tfrac{1}{2} \rho_\text{\ae} |\vec{\omega}|^2 V_{\text{cell}}.
    \]
    Thus, temperature $T$ is directly linked to local vorticity magnitude,
    providing the VAM foundation for blackbody radiation.

% ============================================================
    \subsection{Peak Wavelength Scaling in VAM}

    Radiation wavelength is inversely related to the vortex frequency:
    \[
        \lambda = \frac{c}{\nu}, \qquad \nu \sim |\vec{\omega}|.
    \]
    Substituting the vorticity–temperature correspondence gives
    \[
        \lambda_{\text{peak}} \sim
        \frac{c}{\sqrt{ \dfrac{2 k_B T}{\rho_\text{\ae} V_{\text{cell}}} }},
    \]
    which simplifies to
    \[
        \lambda_{\text{peak}} =
        \left( \frac{c \sqrt{\rho_\text{\ae} V_{\text{cell}}}}{\sqrt{2 k_B}} \right) T^{-1/2}.
    \]
    Hence, VAM predicts a $\lambda_{\text{peak}} \propto T^{-1/2}$ scaling,
    slower than Wien’s $\lambda_{\text{max}} \propto 1/T$ law.

% ============================================================
    \subsection{Quantized Vortex Photon Spectrum}

    A photon in VAM is modeled as a rotating vortex torus of radius $r_n$
    and circulation $\Gamma_n$, yielding an energy
    \[
        E_n = \tfrac{1}{2} \rho_\text{\ae} \frac{\Gamma_n^2}{r_n}.
    \]
    With quantization ansatz
    \[
        \Gamma_n \sim \frac{n h}{M_e}, \quad r_n \sim \frac{r_c}{n},
    \]
    the energy scales as
    \[
        E_n \sim n^3 \cdot \text{const}.
    \]
    Meanwhile, the mode frequency is
    \[
        \nu_n = \frac{C_e}{2\pi r_n} \sim n \nu_0,
        \qquad \nu_0 = \frac{C_e}{2\pi r_c}.
    \]
    Thus,
    \[
        E_n \sim h_{\text{eff}}(n)\, \nu_n,
        \qquad h_{\text{eff}} \sim n^2 h,
    \]
    indicating an $n$-dependent effective Planck constant
    and a cubic scaling of vortex photon energies.

% ============================================================
    \subsection{Spectral Energy Density in VAM}

    The density of vortex photon modes is suppressed at high frequency due to
    instability:
    \[
        g(\nu) \sim \nu^2 e^{-\alpha \nu/\nu_c}, \qquad
        \nu_c = \frac{C_e}{2\pi r_c}.
    \]
    The spectral energy density is therefore
    \[
        u(\nu, T) = g(\nu)\,\frac{E(\nu)}{e^{E(\nu)/k_B T} - 1}.
    \]
    If $E(\nu) \sim \nu^3$, one obtains
    \[
        u(\nu, T) =
        A \nu^2 e^{-\alpha \nu/\nu_c} \cdot
        \frac{\nu^3}{e^{B \nu^3/T} - 1},
    \]
    a VAM-modified blackbody spectrum that resolves the ultraviolet catastrophe
    via exponential suppression.

% ============================================================
    \subsection{Exotic Ætheric Radiation Modes}

    \paragraph{Torsional shocks.}
    Rapid torque collapse produces a vorticity impulse:
    \[
        \frac{\partial}{\partial t} \big( \nabla \cdot \vec{L}_\text{\ae} \big) \gg 0,
        \qquad
        \frac{d\Gamma}{dt} = \oint_{\partial S} \vec{v}\cdot d\vec{\ell}
        \to \text{singular impulse}.
    \]
    The governing transport law is
    \[
        \rho_\text{\ae}\!\left(
                             \frac{\partial \vec{\omega}}{\partial t}
                             + (\vec{v}\cdot \nabla)\vec{\omega}
        \right) = \nabla \times
        \big( \vec{f}_{\text{topo}} + \vec{f}_{\text{shear}} \big).
    \]

    \paragraph{\AE{}ther solitons.}
    Localized vortex packets arise from a nonlinear
    Klein–Gordon equation:
    \[
        \left( \tfrac{\partial^2 \psi}{\partial t^2}
            - C_e^2 \nabla^2 \psi \right) + \beta \psi^3 = 0,
    \]
    with stable solution
    \[
        \psi(x,t) = A\,\text{sech}\!\left(\tfrac{x - v t}{\Delta}\right).
    \]

    \paragraph{Knot collapse flashes (Æ-gamma).}
    If stored knot energy exceeds the Planck threshold,
    \[
        E_\text{stored} \gtrsim E_{\text{Planck}}
        \quad \Rightarrow \quad
        \delta t \approx t_p, \quad \delta E \approx E_p,
    \]
    a violent burst occurs, possibly generating matter and spacetime
    perturbations.


\end{document}

\documentclass[a4paper,11pt]{article}
\usepackage[utf8]{inputenc}
\usepackage{amsmath,graphicx,cite}
\usepackage[left=2.5cm,right=2.5cm,top=2.5cm,bottom=2.5cm]{geometry}
\title{Swirl Bit Taxonomy: Encoding Quantum States in Topological Knots}
\author{ScholarGPT \\ scholargpt@openai.com}
\date{}

\begin{document}
\maketitle

\begin{abstract}
This paper introduces a taxonomy of topologically encoded quantum bits—``swirl bits''—within the Vortex Æther Model (VAM). We map particle-like vortex knots (e.g., trefoil, figure-eight, Hopf, Solomon) to computational basis states with intrinsic chirality, genus, and helicity. These characteristics stabilize quantum information against decoherence. This taxonomy guides the selection of knot types for logic operations and memory encoding in a quantum fluidic substrate.
\end{abstract}

\section*{1. Knot Selection for Logical Encoding}
Each knot type offers distinct advantages for memory stability and operational control. Trefoils encode chiral binary states (0/1), while Solomon links serve as entangled memory blocks. The genus \( g = \frac{(p-1)(q-1)}{2} \) sets the topological suppression index for swirl decay \cite{Kauffman1991}.

\section*{2. Classification Table}
We organize swirl bits by (p, q)-torus knot, chirality, genus, and topological volume. Chiral knots (e.g., 3$_1$, 6$_1$) serve as qubit primitives, while higher-genus knots (e.g., 7$_4$, 8$_8$) enable multi-valued or error-protected computation \cite{Hoste1998}.

\section*{3. Particle Equivalents}
Knot types map to particle analogs: trefoil = electron, figure-eight = neutral boson, Hopf link = polariton, Solomon = e⁻/e⁺ pair \cite{Iskandarani2025}. Each serves as a candidate for vortex logic in condensed systems \cite{Hall2016}.

\section*{Conclusion}
Swirl bit classification enables a structured, physically realizable topological quantum computer. With precise vortex manipulation tools, knot-based logic becomes testable and tunable.

\bibliographystyle{unsrt}
\bibliography{swirl_bits}
\end{document}

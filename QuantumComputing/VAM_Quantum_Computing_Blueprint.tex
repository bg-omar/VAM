
\documentclass[a4paper,11pt]{article}
\usepackage[utf8]{inputenc}
\usepackage{amsmath,graphicx,cite}
\usepackage[left=2.5cm,right=2.5cm,top=2.5cm,bottom=2.5cm]{geometry}
\title{VAM Quantum Computing Blueprint: A Topological Framework}
\author{ScholarGPT \\ scholargpt@openai.com}
\date{}

\begin{document}
\maketitle

\begin{abstract}
The Vortex Æther Model (VAM) offers a topologically protected platform for quantum computation, leveraging knot-based vortex structures to encode stable, low-decoherence states. Unlike traditional qubits susceptible to phase noise, VAM ``swirlbits'' exploit vortex helicity, genus, and linking number to define logical states. We propose an architecture for swirl-clock-based timing, vortex-knot memory, and topological entangling gates, grounded in the physical mechanisms of æther drag and Kelvin wave coupling.
\end{abstract}

\section*{1. Swirlbits as Topological Qubits}
Knots such as trefoils, Hopf links, and Solomon configurations naturally encode binary and multistate logic. Their genus and helicity phase spaces offer a rich basis for resilient state encoding \cite{Kauffman1991,Vilenkin1994}. A trefoil vortex, for example, maintains chirality and twist phase even under fluidic perturbation, thereby offering a natural memory unit.

\section*{2. Entanglement via Vortex Linking}
Whereas quantum entanglement is usually treated algebraically, VAM proposes a physical realization: linked knots (e.g., Solomon or Hopf links) serve as entangled pairs with shared circulation \cite{Iskandarani2025,Hall2016}. Annihilation events may be interpreted as unlinking bifurcations---VAM's version of measurement collapse.

\section*{3. Internal Clocking and Gates}
Swirl Clocks govern internal timing without external RF pulses. Gate operations are implemented via swirl-knot reconnection and Kelvin-mode interference \cite{Anderson2001,Berloff2014}. The phase evolution \( S(t) = \Omega t \) with \( \Omega = C_e / r_c \) determines operational fidelity.

\section*{Conclusion}
VAM computing offers a quantum logic landscape grounded not in abstraction but in topological fluid dynamics. With increasing experimental vortex control in BECs and optical fluids, practical implementations appear within reach.

\bibliographystyle{unsrt}
\bibliography{vam_quantum}
\end{document}

\chapter*{Vortex Æther Models and Quantum Spin Analogs}
\section*{Vortices as Atomic Models and Quantization of Angular Momentum}
The idea of modeling particles as\textit{vortices}in a fluidic “æther” dates back to Lord Kelvin’s 19th-century vortex atom hypothesis\href{https://www.tandfonline.com/doi/full/10.1080/23746149.2020.1843535#:~:text=Full%20article%3A%20Optical%20vortex%20knots,Although}{tandfonline.com}. In the mid-20th century, researchers like O.C. Hilgenberg and Carl F. Krafft revived this concept by developing detailed vortex models of atomic structure. Hilgenberg’s 1938 and 1959 works formulated a\textit{“vortex atom model”}with a full quantum numbering system for the elements\href{https://www.bibliotecapleyades.net/sociopolitica/reichblacksun/chapter13.htm#:~:text=In%201938%20Hilgenberg%20followed%20this,System%20of%20the%20Chemical%20Elements}{bibliotecapleyades.net}\href{https://www.padrak.com/ine/RS_REF9.html#:~:text=Matter,out%20energy%20and%20ether%2Fspace%20at}{padrak.com}. In these models,\textit{quantized}atomic properties emerge naturally from the dynamics of rotating vortex rings. For example, Krafft argued that the\textit{quantization of energy}follows as alogical consequenceof a system of vortices that can only exchange energy/æther in discrete rotational modes (i.e. at\textit{limited rates}determined by ring rotation)\href{https://www.padrak.com/ine/RS_REF9.html#:~:text=Matter,out%20energy%20and%20ether%2Fspace%20at}{padrak.com}. In other words, a vortex can only spin in certain stable modes, much like an electron’s orbitals or spin states in quantum mechanics. This allowed the vortex theory to reproduce quantum phenomena (spectral lines, periodic table structure, etc.) without point particles. Notably, Krafft showed that assuming energy quanta (integral multiples of a unit quantum) is\textit{“just as compatible with the vortex theory as with the nuclear theory”}\href{https://www.scribd.com/document/310197123/Ether-and-Mater-by-Carl-Krafft-pdf#:~:text=only%20assumption%20which%20needs%20to,made%20in%20the%20derivation%20of}{scribd.com}. Quantized angular momentum in particular arises from the discrete geometry of vortex loops – only certain\textit{topological configurations}or rotation speeds yield stable states, analogous to quantum angular momentum eigenstates.

\section*{Static Vorticity Fields and Spin‐½ Behavior}
A striking success of vortex models is their ability to mimic electronspinand its associated angular momentum. In the standard quantum view, an electron’s spin $S$ is an\textit{intrinsic}angular momentum with fixed magnitude $\sqrt{s(s+1)}\hbar$ (with $s=\tfrac{1}{2}$ for an electron) and two allowed $S_z$ projections ($+\hbar/2$ or $-\hbar/2$). Classically, one might expect a spinning object to have a continuum of orientations, but an electron yields only two outcomes for $S_z$. Vortex aether models reproduce this by treating the electron as a\textit{circulating vortex loop}that can be oriented in one of two stable states (often interpreted as clockwise vs. counter-clockwise circulation). In Krafft’s vortex-atom description, there was\textit{“no inherent difficulty”}explaining fine and hyperfine spectral structure without invoking mysterious point-particle spin\href{https://www.scribd.com/document/310197123/Ether-and-Mater-by-Carl-Krafft-pdf#:~:text=structure%2C%20the%20nuclear%20theory%20assumes,that%20the%20electrons%20can}{scribd.com}\href{https://www.scribd.com/document/310197123/Ether-and-Mater-by-Carl-Krafft-pdf#:~:text=There%20are%20no%20such%20inherent,difficulties%20in%20accounting%20for}{scribd.com}. Instead, the atom’s\textit{vortex sub-structures}(like smaller linked rings) could adopt\textit{several discrete orientations of stable equilibrium}– much like a gear that slots into only certain positions\href{https://www.scribd.com/document/310197123/Ether-and-Mater-by-Carl-Krafft-pdf#:~:text=There%20are%20no%20such%20inherent,difficulties%20in%20accounting%20for}{scribd.com}. These discrete orientations of the vortex correspond to different electromagnetic configurations, producing the same spectral splitting that, in conventional theory, is attributed to electron and nuclear spin up vs. down. In effect, the vortex model replaces quantum spin $½$ with two possiblecirculation statesof a rotating fluid structure. The expectation value ⟨$S_z$⟩ in such a model would be determined by the net circulation orientation along the chosen axis. For a randomly oriented ensemble of vortex-electrons, a Stern–Gerlach-type measurement (aligning an external field along $z$) would bifurcate them into two groups of opposite circulation, analogous to spin-$↑$ vs. spin-$↓$ outcomes. This mirrors how a quantum spin-$½$ yields $\pm\tfrac{\hbar}{2}$ for $S_z$. Indeed, modern revisitations of the electron as a circulating lightlike wave or vortex loop achieve the same property: the electron’s internal circulation must undergo a $720^\circ$ rotation to return to its initial state (hence spin-$½$) and naturally gives two spin orientations. For example, Williamson & van der Mark’s toroidal photon model of the electron and Hestenes’ Zitterbewegung interpretation both identify\textit{intrinsic spin}with anorbital angular momentum of an internal circulatory motion\href{https://diracfortherestofus.wordpress.com/2018/06/05/12-zitterwebegung/#:~:text=direction%20of%20the%20electron%20spin,intrinsic%20magnetic%20moment%20of%20the}{diracfortherestofus.wordpress.com}. In Hestenes’ analysis of the Dirac electron, the electron is envisaged as a tiny circulating charge current (of radius on the order of the Compton wavelength) looping at light-speed. This internal vortex-like motion carries angular momentum $L=\tfrac{\hbar}{2}$ and its associated current reproduces the electron’s magnetic dipole moment\href{https://diracfortherestofus.wordpress.com/2018/06/05/12-zitterwebegung/#:~:text=direction%20of%20the%20electron%20spin,intrinsic%20magnetic%20moment%20of%20the}{diracfortherestofus.wordpress.com}. As Hestenes puts it,\textit{“the intrinsic spin of the electron may be looked upon as the orbital angular momentum of this [internal] motion. The current produced by the Zitterbewegung gives rise to the intrinsic magnetic moment.”}\href{https://diracfortherestofus.wordpress.com/2018/06/05/12-zitterwebegung/#:~:text=direction%20of%20the%20electron%20spin,intrinsic%20magnetic%20moment%20of%20the}{diracfortherestofus.wordpress.com}. Thus, a\textit{static}but circulating field configuration can account for both the magnitude of spin angular momentum and the\textit{g-factor}(~2) of the electron’s magnetic moment – key quantum properties emerging from a swirling aether flow.

\section*{Trefoil Knots, Helicity, and Quantized Invariants}
Vortex æther models often postulate that different elementary particles correspond to differentknot or link topologiesin a conserved vorticity field. For instance, one proposal identifies the electron with the simplest closed vortex loop and the proton with a more complex trefoil knot configuration\href{https://physicsdetective.com/how-pair-production-works/#:~:text=,Image%E2%80%9C}{physicsdetective.com}. These static knotted vorticity fields carry robust topological invariants like\textit{circulation}and\textit{helicity}that correspond to conserved quantum numbers.Circulation(the line integral of velocity around the vortex core) is quantized in units of $h/m$ in superfluid helium and Bose–Einstein condensates – a direct analog of quantized angular momentum\href{https://link.aps.org/doi/10.1103/PhysRevResearch.3.033009#:~:text=Superfluid%20vortices%20are%20quantum%20excitations,Based%20on%20various}{link.aps.org}. A\textit{quantum vortex}in a superfluid is literally a topological defect carrying a fixed amount of angular momentum (an integer multiple of $\hbar$)\href{https://link.aps.org/doi/10.1103/PhysRevResearch.3.033009#:~:text=Superfluid%20vortices%20are%20quantum%20excitations,Based%20on%20various}{link.aps.org}. If we imagine an electron as an indivisible vortex loop in a subquantum fluid, its circulation could be the source of the electron’s fixed spin angular momentum. Indeed, an\textit{isolated}ideal vortex conserves its circulation indefinitely (Kelvin’s theorem), which parallels the conservation of a particle’s spin in the absence of torques.Helicity– defined in fluid dynamics as the volume integral of $\mathbf{v}\cdot\boldsymbol{\omega}$ (velocity·vorticity) – measures the knottedness/linking of vortex lines and is an\textit{invariant}of ideal flow. A nontrivial knotted vortex (like a trefoil) possesses a quantized helicity reflecting its topological linkage. This has no continuous analog; helicity can only change via reconnections that rearrange topology\href{https://www.pnas.org/doi/10.1073/pnas.1407232111#:~:text=,This%20process%20is}{pnas.org}. Such topological quantization is evocative of quantum angular momentum\textit{spectra}, where only discrete values are allowed. In fact, in field theory one finds that a gauge field configuration’s winding or linking number is often quantized and can correspond to angular momentum or charge. For a vortex-aether electron, one can interpret its helicity (or another topological invariant) as corresponding to the electron’s spin state. A right-handed knot vs. a left-handed knot might represent spin $+\tfrac{1}{2}$ vs. $-\tfrac{1}{2}$, for example. Crucially, theexpectation value⟨$S_z$⟩ in a quantum state corresponds, in the vortex picture, to the\textit{ensemble average}of the fluid’s oriented circulation. The\textit{Stokes parameters}used in classical optics to describe polarization have an analogous interpretation – they are in fact the expectation values of spin operators for a photon\href{https://www.frontiersin.org/journals/physics/articles/10.3389/fphy.2023.1225334/full#:~:text=description%20of%20polarisation%20states%20by,parameters%20are%20order%20parameters%20to}{frontiersin.org}. By a similar token, one can envision an electron’s spin-up vs. spin-down populations as two polarization states of the vortex structure, whose average defines a spin vector direction. The vortex model thus provides a\textit{mechanistic}underpinning for spin expectation values: they emerge from thegeometry and dynamics of the swirl.

\section*{Angular Momentum Conservation via Vorticity Conservation}
One of the appealing features of vortex models is that they embed quantum angular momentum conservation in the geometry of an incompressible flow. In quantum mechanics, the total angular momentum (including spin) of an isolated system is rigorously conserved. Analogously, an ideal fluid with no external torque rigorously conserves its total circulation and helicity. Researchers have shown that even in\textit{quantum fluids}(e.g. spinful plasmas), one can define a combined“generalized vorticity”that includes the spin contribution, and this obeys a conservation law analogous to the classical Kelvin circulation theorem\href{https://link.aps.org/accepted/10.1103/PhysRevLett.107.195003#:~:text=It%20is%20shown%20that%20a,a%20well%20known%20and%20highly}{link.aps.org}\href{https://link.aps.org/accepted/10.1103/PhysRevLett.107.195003#:~:text=In%20this%20paper%20we%20demonstrate,macroscopic%20spin%20vector%20field%20S}{link.aps.org}. In a sense, the fluid’s vorticity flux plays the role of angular momentum, and its conservation ensures the\textit{constancy}of spin. This means that if an electron is a stable knotted vortex, it cannot\textit{lose}its spin – the circulation that embodies its $½\hbar$ is an inviolable constant of motion (unless the vortex interacts or breaks). The Vortex Aether Model (VAM) thereby offers a pictorial way to understandspin quantization and conservation. A static trefoil-knot vortex, for instance, has a fixed linking number and associated angular momentum that\textit{must}be carried with it, just as an electron in quantum theory always carries spin-$½$. Furthermore, when interactions occur, the vortex model can mimic quantum outcomes. A measurement of spin along some axis (say $z$) corresponds to coupling the vortex to an external field (analogous to Stern–Gerlach magnets or photon absorption). The vortex will tend to align one way or the opposite in the field, effectively “choosing” one of its two circulation orientations. This reproduces thequantized spin projectionoutcomes while respecting angular momentum conservation (the external field/vessel gains the compensating angular momentum).

In summary,Vortex Æther Models (VAM)show through explicit derivations that many quantum mechanical quantities – especially those related to angular momentum and spin – can be captured by static vorticity configurations. By treating fundamental particles as stable, knotted or looped vortices in a fluidic medium, these models recover key quantum features:\textit{quantized angular momentum spectra}(via discrete vortex modes)\href{https://www.padrak.com/ine/RS_REF9.html#:~:text=Matter,out%20energy%20and%20ether%2Fspace%20at}{padrak.com}, fixed spin-$½$ values (via two stable circulation states)\href{https://www.scribd.com/document/310197123/Ether-and-Mater-by-Carl-Krafft-pdf#:~:text=structure%2C%20the%20nuclear%20theory%20assumes,that%20the%20electrons%20can}{scribd.com}\href{https://www.scribd.com/document/310197123/Ether-and-Mater-by-Carl-Krafft-pdf#:~:text=There%20are%20no%20such%20inherent,difficulties%20in%20accounting%20for}{scribd.com}, correct magnetic moments and $g$-factors (via circulating charge flows)\href{https://diracfortherestofus.wordpress.com/2018/06/05/12-zitterwebegung/#:~:text=direction%20of%20the%20electron%20spin,intrinsic%20magnetic%20moment%20of%20the}{diracfortherestofus.wordpress.com}, and strict conservation of spin (via helicity/circulation conservation). The correspondence between a vortex’s circulation or helicity and a particle’s spin illustrates howquantum spin expectation valueslike ⟨$S_z$⟩ may\textit{emerge from circulation and swirl dynamics}. As one source succinctly put it, the electron’s spin is nothing but\textit{“the orbital angular momentum of [an internal] motion”}of the vortex, with the fluid’s swirl acting as a tangible model for the abstract quantum spin\href{https://diracfortherestofus.wordpress.com/2018/06/05/12-zitterwebegung/#:~:text=direction%20of%20the%20electron%20spin,intrinsic%20magnetic%20moment%20of%20the}{diracfortherestofus.wordpress.com}. Thus, static knotted vorticity fields not only\textit{qualitatively}resemble quantum particles, but in many formulations they yield the\textit{quantitative}values (½ħ spin, discrete levels, etc.) associated with quantum mechanical angular momentum. This convergence of fluid dynamics and quantum theory is a cornerstone of VAM research, suggesting that what we call “spin” may be viewed as ahidden vorticityresiding in the fabric of space itself\href{https://www.padrak.com/ine/RS_REF9.html#:~:text=Matter,out%20energy%20and%20ether%2Fspace%20at}{padrak.com}\href{https://www.scribd.com/document/310197123/Ether-and-Mater-by-Carl-Krafft-pdf#:~:text=There%20are%20no%20such%20inherent,difficulties%20in%20accounting%20for}{scribd.com}.

Sources:Quantum-vortex atomic models by Hilgenberg and Krafft\href{https://www.padrak.com/ine/RS_REF9.html#:~:text=Matter,out%20energy%20and%20ether%2Fspace%20at}{padrak.com}\href{https://www.scribd.com/document/310197123/Ether-and-Mater-by-Carl-Krafft-pdf#:~:text=There%20are%20no%20such%20inherent,difficulties%20in%20accounting%20for}{scribd.com}; superfluid quantization of circulation\href{https://link.aps.org/doi/10.1103/PhysRevResearch.3.033009#:~:text=Superfluid%20vortices%20are%20quantum%20excitations,Based%20on%20various}{link.aps.org}; modern vortex-knot theories\href{https://physicsdetective.com/how-pair-production-works/#:~:text=,Image%E2%80%9C}{physicsdetective.com}; and analyses linking spin-$½$ to circulatory motion\href{https://diracfortherestofus.wordpress.com/2018/06/05/12-zitterwebegung/#:~:text=direction%20of%20the%20electron%20spin,intrinsic%20magnetic%20moment%20of%20the}{diracfortherestofus.wordpress.com}, among others.






Citations
\href{https://www.tandfonline.com/doi/full/10.1080/23746149.2020.1843535#:~:text=Full%20article%3A%20Optical%20vortex%20knots,Although}{Full article: Optical vortex knots and links via holographic metasurfaces


https://www.tandfonline.com/doi/full/10.1080/23746149.2020.1843535
}\href{https://www.bibliotecapleyades.net/sociopolitica/reichblacksun/chapter13.htm#:~:text=In%201938%20Hilgenberg%20followed%20this,System%20of%20the%20Chemical%20Elements}{Reich of The Black Sun - 13


https://www.bibliotecapleyades.net/sociopolitica/reichblacksun/chapter13.htm
}\href{https://www.padrak.com/ine/RS_REF9.html#:~:text=Matter,out%20energy%20and%20ether%2Fspace%20at}{RS Electrogravitic References: Part 9 of 19


https://www.padrak.com/ine/RS_REF9.html
}\href{https://www.scribd.com/document/310197123/Ether-and-Mater-by-Carl-Krafft-pdf#:~:text=only%20assumption%20which%20needs%20to,made%20in%20the%20derivation%20of}{

Ether and Mater by Carl Krafft PDF | PDF | Luminiferous Aether | Special Relativity


https://www.scribd.com/document/310197123/Ether-and-Mater-by-Carl-Krafft-pdf
}\href{https://www.scribd.com/document/310197123/Ether-and-Mater-by-Carl-Krafft-pdf#:~:text=structure%2C%20the%20nuclear%20theory%20assumes,that%20the%20electrons%20can}{

Ether and Mater by Carl Krafft PDF | PDF | Luminiferous Aether | Special Relativity


https://www.scribd.com/document/310197123/Ether-and-Mater-by-Carl-Krafft-pdf
}\href{https://www.scribd.com/document/310197123/Ether-and-Mater-by-Carl-Krafft-pdf#:~:text=There%20are%20no%20such%20inherent,difficulties%20in%20accounting%20for}{

Ether and Mater by Carl Krafft PDF | PDF | Luminiferous Aether | Special Relativity


https://www.scribd.com/document/310197123/Ether-and-Mater-by-Carl-Krafft-pdf
}\href{https://diracfortherestofus.wordpress.com/2018/06/05/12-zitterwebegung/#:~:text=direction%20of%20the%20electron%20spin,intrinsic%20magnetic%20moment%20of%20the}{

12) Trembling in spacetime – Dirac for the rest of us


https://diracfortherestofus.wordpress.com/2018/06/05/12-zitterwebegung/
}\href{https://physicsdetective.com/how-pair-production-works/#:~:text=,Image%E2%80%9C}{

How pair production works – THE PHYSICS DETECTIVE


https://physicsdetective.com/how-pair-production-works/
}\href{https://link.aps.org/doi/10.1103/PhysRevResearch.3.033009#:~:text=Superfluid%20vortices%20are%20quantum%20excitations,Based%20on%20various}{

Classification of magnetic vortices by angular momentum conservation | Phys. Rev. Research


https://link.aps.org/doi/10.1103/PhysRevResearch.3.033009
}\href{https://www.pnas.org/doi/10.1073/pnas.1407232111#:~:text=,This%20process%20is}{

Helicity conservation by flow across scales in reconnecting vortex ...


https://www.pnas.org/doi/10.1073/pnas.1407232111
}\href{https://www.frontiersin.org/journals/physics/articles/10.3389/fphy.2023.1225334/full#:~:text=description%20of%20polarisation%20states%20by,parameters%20are%20order%20parameters%20to}{

Frontiers | Quantum field theory for coherent photons: isomorphism between Stokes parameters and spin expectation values


https://www.frontiersin.org/journals/physics/articles/10.3389/fphy.2023.1225334/full
}\href{https://link.aps.org/accepted/10.1103/PhysRevLett.107.195003#:~:text=It%20is%20shown%20that%20a,a%20well%20known%20and%20highly}{




https://link.aps.org/accepted/10.1103/PhysRevLett.107.195003
}\href{https://link.aps.org/accepted/10.1103/PhysRevLett.107.195003#:~:text=In%20this%20paper%20we%20demonstrate,macroscopic%20spin%20vector%20field%20S}{




https://link.aps.org/accepted/10.1103/PhysRevLett.107.195003
}

All Sources
\href{https://www.tandfonline.com/doi/full/10.1080/23746149.2020.1843535#:~:text=Full%20article%3A%20Optical%20vortex%20knots,Although}{tandfonline
}\href{https://www.bibliotecapleyades.net/sociopolitica/reichblacksun/chapter13.htm#:~:text=In%201938%20Hilgenberg%20followed%20this,System%20of%20the%20Chemical%20Elements}{bibliotecapleyades
}\href{https://www.padrak.com/ine/RS_REF9.html#:~:text=Matter,out%20energy%20and%20ether%2Fspace%20at}{padrak
}\href{https://www.scribd.com/document/310197123/Ether-and-Mater-by-Carl-Krafft-pdf#:~:text=only%20assumption%20which%20needs%20to,made%20in%20the%20derivation%20of}{

scribd
}\href{https://diracfortherestofus.wordpress.com/2018/06/05/12-zitterwebegung/#:~:text=direction%20of%20the%20electron%20spin,intrinsic%20magnetic%20moment%20of%20the}{

diracfor...wordpress
}\href{https://physicsdetective.com/how-pair-production-works/#:~:text=,Image%E2%80%9C}{

physicsdetective
}\href{https://link.aps.org/doi/10.1103/PhysRevResearch.3.033009#:~:text=Superfluid%20vortices%20are%20quantum%20excitations,Based%20on%20various}{

link.aps
}\href{https://www.pnas.org/doi/10.1073/pnas.1407232111#:~:text=,This%20process%20is}{

pnas
}\href{https://www.frontiersin.org/journals/physics/articles/10.3389/fphy.2023.1225334/full#:~:text=description%20of%20polarisation%20states%20by,parameters%20are%20order%20parameters%20to}{

frontiersin
}




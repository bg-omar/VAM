 \section*{Introduction}

    In popular imagination, it is often said that Einstein \grqq abolished the æther.\textquotedblright However, this statement is a severe simplification of his actual stance~\cite{einstein1920aether}. In reality, Einstein distinguished between different concepts of æther, and while in his early work (1905) he omitted the luminiferous æther, in later lectures he returned to a more subtly defined concept of æther as a physical medium carrying field properties. This text explores what Einstein actually wrote, what he meant, and how this aligns with modern models such as the Vortex Æther Model (VAM), a contemporary physical framework that reintroduces the æther as a central player in physics.

    Imagine a universe where swirling vortices, rather than invisible forces, govern the cosmos. This is the essence of the Vortex Æther Model (VAM): a framework in which gravitation, inertia, time, and even quantum behavior emerge from structured motion in an incompressible, non-viscous fluid — the Æther~\cite{iskandarani2024vam1, iskandarani2024vam2}. Unlike conventional field theories that abstract away the medium, VAM embraces it as fundamental. In this view, matter is topology, time is rotation, and space is filled not with void, but with conserved vorticity.

    This analysis is not merely a historical correction, but also a bridge between Einstein's original insights~\cite{einstein1920aether} and contemporary models that aim to reformulate gravity, quantum behavior, and time perception based on fluid dynamics~\cite{iskandarani2014vam}. By carefully considering the context of Einstein's quotes, it becomes clear that his philosophical and physical stance toward æther was more complex and nuanced than is often assumed.


    \section{The Misconception: Einstein \("\)Abolished the Æther\("\)}

    In 1905, Einstein introduced the Special Theory of Relativity. In it, the concept of the luminiferous æther was absent. Many took this as a rejection of the æther concept altogether~\cite{einstein1920aether}. But this was not Einstein's intention. He wrote:

    \begin{quote}
        \("\)The introduction of a \('\)light-bearer\('\) (luminiferous æther) proves to be superfluous.\("\)
    \end{quote}

    This is not the same as saying the æther does not exist. Rather, it implies that a mechanical carrier for light waves is not necessary to explain electromagnetic phenomena. But this does not rule out the possibility that space itself possesses properties that interact with matter, energy, and time. It is this subtle transition from medium to field substrate that is often misunderstood.

    \section{The Return of the Æther Concept (1920)}

    In his 1920 lecture in Leiden, Einstein explicitly stated:

    \begin{quote}
        \("\)According to the general theory of relativity, space is endowed with physical qualities; in this sense, therefore, there exists an æther. According to the general theory of relativity, space without æther is unthinkable.\("\)~\cite{einstein1920aether}
    \end{quote}

    Here Einstein describes an æther that does not have material properties such as velocity or location, but that does carry the qualities of space itself — such as curvature, field strength, and gravitational content. He emphasized that this æther has no separate existence outside spacetime but is inextricably connected to the geometric and energetic structure thereof.

    This statement represents a clear break from the dogma that Einstein was anti-æther. Rather, it shows that he revised his views in line with the development of General Relativity. His æther is not a fluid in the classical sense, but a physical substrate that influences forces and time.

     In the remainder of this article, we revisit Einstein\rqs s æther-related statements and interpret them in light of modern topological fluid models, culminating in the correspondence with VAM.
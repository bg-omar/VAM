
    \section{Historical Continuity}

    A careful reading of Einstein’s work reveals that he:
    \begin{itemize}
        \item Did \emph{not} reject æther, but \emph{redefined} it,
        \item Sought a \textbf{medium} carrying the properties of spacetime,
        \item And ultimately aimed to \textbf{unify} what VAM now combines: gravity, time perception, and field interaction through vorticity.
    \end{itemize}

    Einstein implicitly acknowledged that space is not an empty backdrop, but an active physical domain. In this context, VAM is a logical continuation — a model that not only accepts this active structure but also reconstructs it mathematically and physically using conserved vortex fields, knot structures, and energetic boundary conditions~\cite{iskandarani2024vam1, iskandarani2024vam2}
.

    It is important to note that many modern models, such as emergent gravity and superfluid vacuum theory, show parallels with VAM. But while those models often remain abstract or only partially consistent, VAM is explicit, consistent, experimentally oriented, and mathematically verifiable from hydrodynamic first principles.

    \section*{Conclusion: Æther Reclaimed}

    Modern science is gradually returning to ideas that were ignored for a century — not because they were wrong, but because the time was not yet right. Einstein foresaw this. And the Vortex Æther Model is not a regression to outdated views, but a progression toward a coherent, experimentally testable worldview where æther, vorticity, and reality are once again interconnected.

    It is time to stop viewing Einstein as the man who excluded the æther and instead recognize him as the thinker who transformed the concept into something that is once again relevant today. In that spirit, VAM is both a tribute and a continuation of an intellectual journey that is far from over.

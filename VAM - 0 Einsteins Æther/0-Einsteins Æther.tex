%! Author = Omar Iskandarani
%! Title = Einstein and the Æther
%! Date = May 23, 2025
%! Affiliation = Independent Researcher, Groningen, The Netherlands
%! License = CC-BY 4.0
%! ORCID = 0009-0006-1686-3961

\documentclass[a4paper,12pt]{revtex4}
\usepackage[a4paper, margin=2cm]{geometry}
\usepackage[none]{hyphenat}
\usepackage{amsmath}
\usepackage{graphicx}
\usepackage{hyperref}
\usepackage{array}
\usepackage{booktabs}
\usepackage{amssymb}
\usepackage{float}
\usepackage{physics}
\usepackage{tikz}
\usepackage[utf8]{inputenc}
\usepackage[font=footnotesize]{caption}
\usetikzlibrary{arrows.meta}
\usepackage{pgfplots}
\pgfplotsset{compat=1.18}
\usetikzlibrary{arrows.meta}
\sloppy

\begin{document}
    \date{\today}
    \author{Omar Iskandarani}
    \title{Einstein and the Æther}
    \email{info@omariskandarani.com}
    \affiliation{Independent Researcher, Groningen, The Netherlands}
    \thanks{ORCID: \href{https://orcid.org/0009-0006-1686-3961}{0009-0006-1686-3961}}
    \keywords{æther, Einstein, Vortex Æther Model, vorticity, time dilation, topological matter, fluid dynamics, Helmholtz, Kelvin, Maxwell}

    \begin{abstract}
    This paper re-examines the æther concept through the lens of Einstein’s evolving views and contrasts them with a contemporary model: the Vortex Æther Model (VAM). Contrary to popular belief, Einstein did not abolish the æther but redefined it as a non-mechanical, structured medium carrying the physical qualities of space. We trace his development from the 1905 rejection of the luminiferous æther to his 1920 assertion that “space without æther is unthinkable.” The paper then connects this philosophical and physical substrate to VAM, which models gravity, inertia, and time as emergent from vorticity in an incompressible superfluid æther. Using topological vortex structures and conserved circulation, VAM formalizes Einstein’s late æther view into a mathematically rigorous and experimentally testable framework.
    \end{abstract}

    \maketitle

     \section*{Introduction}

    In popular imagination, it is often said that Einstein “abolished the æther.” However, this statement is a severe simplification of his actual stance~\cite{einstein1920aether}. In reality, Einstein distinguished between different concepts of æther, and while in his early work (1905) he omitted the luminiferous æther, in later lectures he returned to a more subtly defined concept of æther as a physical medium carrying field properties. This text explores what Einstein actually wrote, what he meant, and how this aligns with modern models such as the Vortex Æther Model (VAM), a contemporary physical framework that reintroduces the æther as a central player in physics.

    Imagine a universe where swirling vortices, rather than invisible forces, govern the cosmos. This is the essence of the Vortex Æther Model (VAM): a framework in which gravitation, inertia, time, and even quantum behavior emerge from structured motion in an incompressible, non-viscous fluid — the Æther~\cite{iskandarani2024vam1, iskandarani2024vam2}. Unlike conventional field theories that abstract away the medium, VAM embraces it as fundamental. In this view, matter is topology, time is rotation, and space is filled not with void, but with conserved vorticity.

    This analysis is not merely a historical correction, but also a bridge between Einstein's original insights~\cite{einstein1920aether} and contemporary models that aim to reformulate gravity, quantum behavior, and time perception based on fluid dynamics~\cite{iskandarani2014vam}. By carefully considering the context of Einstein's quotes, it becomes clear that his philosophical and physical stance toward æther was more complex and nuanced than is often assumed.


    \section{The Misconception: Einstein \("\)Abolished the Æther\("\)}

    In 1905, Einstein introduced the Special Theory of Relativity. In it, the concept of the luminiferous æther was absent. Many took this as a rejection of the æther concept altogether~\cite{einstein1920aether}. But this was not Einstein's intention. He wrote:

    \begin{quote}
        \("\)The introduction of a \('\)light-bearer\('\) (luminiferous æther) proves to be superfluous.\("\)
    \end{quote}

    This is not the same as saying the æther does not exist. Rather, it implies that a mechanical carrier for light waves is not necessary to explain electromagnetic phenomena. But this does not rule out the possibility that space itself possesses properties that interact with matter, energy, and time. It is this subtle transition from medium to field substrate that is often misunderstood.

    \section{The Return of the Æther Concept (1920)}

    In his 1920 lecture in Leiden, Einstein explicitly stated:

    \begin{quote}
        \("\)According to the general theory of relativity, space is endowed with physical qualities; in this sense, therefore, there exists an æther. According to the general theory of relativity, space without æther is unthinkable.\("\)~\cite{einstein1920aether}
    \end{quote}

    Here Einstein describes an æther that does not have material properties such as velocity or location, but that does carry the qualities of space itself — such as curvature, field strength, and gravitational content. He emphasized that this æther has no separate existence outside spacetime but is inextricably connected to the geometric and energetic structure thereof.

    This statement represents a clear break from the dogma that Einstein was anti-æther. Rather, it shows that he revised his views in line with the development of General Relativity. His æther is not a fluid in the classical sense, but a physical substrate that influences forces and time.

     In the remainder of this article, we revisit Einstein’s æther-related statements and interpret them in light of modern topological fluid models, culminating in the correspondence with VAM.
    \section{Æther as Carrier of Field Quality}

    Einstein emphasized that his æther:
    \begin{itemize}
        \item Was not composed of particles,
        \item Had no absolute rest state,
        \item But \emph{was} responsible for physical phenomena such as gravity, field interactions, and temporal evolution.
    \end{itemize}

    This explicitly deviates from the 19th-century mechanical æther model and posits instead a structured, field-like medium closely resembling modern notions of a vacuum with properties. In fact, we can say that Einstein's \("\)æther\("\) functions as a form of intrinsic space dynamics, where the metric tensor and curvature tensor are merely mathematical projections of a deeper, possibly vorticial, reality.

    In this interpretation, æther can be considered a fundamental background from which not only gravity and electromagnetism emerge, but also temporal progression, inertia, and quantum fluctuations. This opens the door to contemporary models that describe space as a dynamic, structured fluid configuration.

    \section{Connection to the Vortex Æther Model (VAM)}

    In VAM — developed since 2012 by O. Iskandarani — the æther is modeled as a non-viscous, incompressible superfluid in which vorticity, as a fundamental quantity, determines local time, gravity, and mass. As in Einstein's later work, the æther is regarded as fundamental and determinative of physical interactions.

    Additionally, VAM includes:
    \begin{itemize}
        \item Topological structures (such as knots and trefoils) as elementary particles,
        \item Local time dilation resulting from core vorticity,
        \item A new system of fundamental constants, with $C_e$ (vortex boundary velocity) and $F^{\text{\ae}}_{\text{max}}$ (maximum force).
    \end{itemize}

    \subsection*{Example Formula}

    The gravitational constant in VAM is derived as:
    \begin{equation}
        G_\text{swirl} = \frac{C_e c^5 t_p^2}{2 F^{\text{\ae}}_{\text{max}} r_c^2}
    \end{equation}

        This formula—derived from rotational conservation in VAM’s Swirl Clock formalism—connects the gravitational constant to Planck-scale rotational energy, showing that gravity emerges from bounded vorticity fields~\cite{iskandarani2024vam2}. It reflects an æther with dynamic properties — just as Einstein envisioned in 1920. VAM also employs vorticial circulation, local pressure gradients, and conservation of absolute vorticity to explain interactions between knots and fields, aligning strongly with Einstein’s quest for a unified field theory.

    By combining Einstein’s æther concept with modern hydrodynamics and topological stability, VAM becomes a powerful framework for reinterpreting fundamental laws. The model offers experimental testability via superfluid analogies, electric vortex interference, and gravitational simulations in laboratory conditions.
    
    \section{Historical Continuity}

    A careful reading of Einstein's work reveals that he:
    \begin{itemize}
        \item Did \emph{not} reject æther, but \emph{redefined} it,
        \item Sought a \textbf{medium} carrying the properties of spacetime,
        \item And ultimately aimed to \textbf{unify} what VAM now combines: gravity, time perception, and field interaction through vorticity.
    \end{itemize}

    Einstein implicitly acknowledged that space is not an empty backdrop, but an active physical domain. In this context, VAM is a logical continuation — a model that not only accepts this active structure but also reconstructs it mathematically and physically using conserved vortex fields, knot structures, and energetic boundary conditions~\cite{iskandarani2024vam1, iskandarani2024vam2}
.

    It is important to note that many modern models, such as emergent gravity and superfluid vacuum theory, show parallels with VAM. But while those models often remain abstract or only partially consistent, VAM is explicit, consistent, experimentally oriented, and mathematically verifiable from hydrodynamic first principles.

    \section*{Conclusion: Æther Reclaimed}

    Modern science is gradually returning to ideas that were ignored for a century — not because they were wrong, but because the time was not yet right. Einstein foresaw this. And the Vortex Æther Model is not a regression to outdated views, but a progression toward a coherent, experimentally testable worldview where æther, vorticity, and reality are once again interconnected.

    It is time to stop viewing Einstein as the man who excluded the æther and instead recognize him as the thinker who transformed the concept into something that is once again relevant today. In that spirit, VAM is both a tribute and a continuation of an intellectual journey that is far from over.


    \appendix \label{sec:Appendix}
        \input{appendix_1_einstein_on_the_æther}
        
\section*{Appendix II: Lord Kelvin and the Knot-Æther Critique}

In the late 19th century, William Thomson (Lord Kelvin) proposed that atoms might be stable vortex knots in an invisible æther — a topological interpretation of matter. Yet he himself raised the most pointed critique:

\begin{quote}
\grqq I am afraid of the smoke and complication, of all the varieties of knots and links, if they are to explain the variety of elements.\textquotedblright \\
— Lord Kelvin, 1890
\end{quote}

Kelvin feared that the near-infinite number of possible knots and links in three-dimensional space would not correspond to the relatively small number of stable chemical elements~\cite{thomson1890knots,tait1877knots}. Without a natural principle of selection, the theory risked degeneracy: the proliferation of mathematically possible but physically irrelevant structures.

\subsection*{Historical Context}

In the second half of the 19th century, the vortex atom theory was developed, primarily by William Thomson (Lord Kelvin) and Peter Guthrie Tait~\cite{thomson1890knots,tait1877knots}. In this framework, atoms were envisioned as stable knots or vortex rings in an ideal, invisible fluid — the so-called luminiferous æther. The idea was that both the discrete nature of atomic species and their remarkable stability could be explained through topological invariants from knot theory.

\grqq Helmholtz's 1858 paper introduced the conservation of vorticity in ideal fluids, laying the mathematical foundation upon which Kelvin and Tait constructed the vortex atom theory~\cite{helmholtz1858vortices}. This conservation principle is central to both classical vortex stability and the topological persistence employed in VAM.\textquotedblright

Kelvin's model was deeply influenced by the work of Helmholtz (1858) on vortex conservation in ideal fluids. He imagined that different types of knotted or linked vortices might correspond to different elements.

\subsection*{Kelvin's Principal Objection}

Despite its elegance, Kelvin identified a critical flaw:

\begin{quote}
 \grqq I am afraid of the smoke and complication, of all the varieties of knots and links, if they are to explain the variety of elements.\textquotedblright \\
 — William Thomson (Lord Kelvin), Baltimore Lectures, 1890
\end{quote}

The mathematical space of knots is vast, and Kelvin recognized the absence of a physical filter. He was acutely aware that the theory, though geometrically rich, lacked a way to explain *why only some knots should be stable atoms*. It had no built-in energetic, dynamic, or entropic selection rule.

\subsection*{Experimental Shortcomings}

Kelvin also noted the absence of empirical correspondence between specific knot types and actual elements. Without experimental access to the supposed vortex knots — their formation, stability, or interaction — the theory remained speculative.

Nonetheless, the idea lived on, inspiring both topological mathematics and future models of discrete matter arising from continuous media.

\subsection*{Comparison to the Modern Particle Zoo}

Kelvin's critique is echoed in modern particle physics. The Standard Model contains a large number of particles, generations, couplings, and constants — many set only by experimental input, not derivable from deeper principles.

The degeneracy Kelvin foresaw reappears: a theory with many admissible but unexplained types of particles. The need for a *selection mechanism* remains urgent.

\subsection*{The VAM Response}

The Vortex Æther Model (VAM) revives the topological atom intuition but answers Kelvin's critique with concrete physical principles:

\begin{itemize}
  \item Thermodynamic constraints (via Clausius entropy) limit allowable knot growth~\cite{clausius1865entropy}.
  \item Quantized circulation excludes unstable, high-energy configurations.
  \item Absolute vorticity conservation enforces topological stability.
  \item Vortex reconnection thresholds act as evolutionary boundaries.
\end{itemize}

As a result, VAM predicts only a finite, physically meaningful spectrum of topological matter structures — in line with observed baryons and leptons.

\subsection*{Concluding Reflection}

Kelvin's objection was not to knots themselves, but to their uncontrolled proliferation. VAM reclaims his vision, but grounds it in hydrodynamic logic, energy bounds, and field evolution:

\begin{quote}
\grqq Knots without constraints become chaos. Knots with physics become atoms.\textquotedblright — O. Iskandarani
\end{quote}


        \section*{Appendix III: James Clerk Maxwell on the Æther and the Vortex Atom Theory}

James Clerk Maxwell (1831–1879), one of the foundational figures of modern physics, held deep and evolving views on the concept of the æther. While best known for formulating the electromagnetic field equations, Maxwell also contributed to the theoretical underpinnings of the æther and engaged directly with the emerging vortex atom theories of his time.

\subsection*{Maxwell's View on the Æther}
Maxwell firmly believed that the æther was a physically real, omnipresent medium necessary for the transmission of electromagnetic waves~\cite{maxwell1878britannica}:

\begin{quote}
\("\)There can be no doubt that the interplanetary and interstellar spaces are not empty, but are occupied by a material substance\ldots which is certainly the largest and probably the most uniform body of which we have any knowledge.\("\)
\end{quote}

To Maxwell, the electromagnetic field was not abstract, but a manifestation of real stresses and strains in the æther~\cite{maxwell1878britannica}. He imagined it as an elastic medium capable of supporting tension (electric fields), rotation (magnetic lines), and vibrational energy (light).

\subsection*{Maxwell and the Vortex Atom Theory}

Maxwell was intrigued by Lord Kelvin’s proposal that atoms could be modeled as stable vortex knots in the æther — the so-called vortex atom theory~\cite{maxwell1875molecules}. In his 1875 lecture \("\)Molecules,\("\) he expressed qualified enthusiasm:

\begin{quote}
\("\)The vortex theory of atoms, first proposed by Helmholtz and developed by Sir William Thomson\ldots has made it conceivable that the properties of matter may depend solely on motion in a medium, and not on anything in the nature of the atom itself.\("\)
\\— James Clerk Maxwell, 1875, \("\)Molecules\("\)
\end{quote}

This radical idea — that all matter could emerge from organized motion in a universal fluid — deeply appealed to Maxwell’s mechanical sensibilities. However, he also expressed caution:

\begin{quote}
\("\)The difficulty is that we know so little about fluid motion, and the equations are so intractable, that no one has yet been able to deduce the properties of any known substance from such a theory.\("\)
\end{quote}

In short, the theory was conceptually beautiful but lacked mathematical tractability and predictive power. Maxwell understood the elegance of vortex-based models but noted that fluid dynamics was still too undeveloped to make the theory physically useful~\cite{maxwell1875molecules}.

\subsection*{Legacy and Connection to VAM}

Maxwell's æther was a mechanical medium filled with stresses, pressures, and circulations — not unlike the vortex fields described in the Vortex \AE ther Model (VAM). His aspirations for a unified field theory based on æther mechanics resonate strongly with VAM’s goals:

\begin{itemize}
  \item Both view the vacuum as structured and dynamic.
  \item Both describe matter as emergent from motion in the medium.
  \item Both seek to replace ad hoc constants with field-based origins.
\end{itemize}

Maxwell anticipated that future physicists might unlock the mathematics of vortex-structured æther. VAM — using conservation of vorticity, topological invariants, and pressure-induced time dilation — picks up where Maxwell’s generation left off.

\subsection*{Reflection}

Maxwell’s words remind us that the æther was never fully dismissed on scientific grounds, but rather due to limitations in modeling and experiment. With modern tools, those limitations are no longer insurmountable.

\begin{quote}
\("\)A field is not a ghost. It is the visible strain of the invisible æther.\("\) — paraphrasing Maxwell
\end{quote}


    \bibliographystyle{unsrt}
    \bibliography{0-einsteins-æther}

\end{document}
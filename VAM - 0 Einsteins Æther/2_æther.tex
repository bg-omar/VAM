\section{Æther as Carrier of Field Quality}

    Einstein emphasized that his æther:
    \begin{itemize}
        \item Was not composed of particles,
        \item Had no absolute rest state,
        \item But \emph{was} responsible for physical phenomena such as gravity, field interactions, and temporal evolution.
    \end{itemize}

    This explicitly deviates from the 19th-century mechanical æther model and posits instead a structured, field-like medium closely resembling modern notions of a vacuum with properties. In fact, we can say that Einstein's \("\)æther\("\) functions as a form of intrinsic space dynamics, where the metric tensor and curvature tensor are merely mathematical projections of a deeper, possibly vorticial, reality.

    In this interpretation, æther can be considered a fundamental background from which not only gravity and electromagnetism emerge, but also temporal progression, inertia, and quantum fluctuations. This opens the door to contemporary models that describe space as a dynamic, structured fluid configuration.

    \section{Connection to the Vortex Æther Model (VAM)}

    In VAM — developed since 2012 by O. Iskandarani — the æther is modeled as a non-viscous, incompressible superfluid in which vorticity, as a fundamental quantity, determines local time, gravity, and mass. As in Einstein's later work, the æther is regarded as fundamental and determinative of physical interactions.

    Additionally, VAM includes:
    \begin{itemize}
        \item Topological structures (such as knots and trefoils) as elementary particles,
        \item Local time dilation resulting from core vorticity,
        \item A new system of fundamental constants, with $C_e$ (vortex boundary velocity) and $F^{\text{\ae}}_{\text{max}}$ (maximum force).
    \end{itemize}

    \subsection*{Example Formula}

    The gravitational constant in VAM is derived as:
    \begin{equation}
        G_\text{swirl} = \frac{C_e c^5 t_p^2}{2 F^{\text{\ae}}_{\text{max}} r_c^2}
    \end{equation}

        This formula—derived from rotational conservation in VAM’s Swirl Clock formalism—connects the gravitational constant to Planck-scale rotational energy, showing that gravity emerges from bounded vorticity fields~\cite{iskandarani2024vam2}. It reflects an æther with dynamic properties — just as Einstein envisioned in 1920. VAM also employs vorticial circulation, local pressure gradients, and conservation of absolute vorticity to explain interactions between knots and fields, aligning strongly with Einstein’s quest for a unified field theory.

    By combining Einstein’s æther concept with modern hydrodynamics and topological stability, VAM becomes a powerful framework for reinterpreting fundamental laws. The model offers experimental testability via superfluid analogies, electric vortex interference, and gravitational simulations in laboratory conditions.
\section{Tijddilatatie door relatieve beweging}

Beschouw eerst de tijddilatatie voor een deeltje dat met hoge snelheid beweegt ten opzichte van het æther-rustframe. Empirisch weten we dat een klok die met snelheid $v$ beweegt, tijd ervaart die langzamer is met de Lorentz-factor $\gamma = 1/\sqrt{1 - v^2/c^2}$. In dit model leiden we hetzelfde effect af door de invloed van absolute æther-beweging op de rotatie van de wervelkern te analyseren.

\subsection*{(a) Kinematische afleiding}

Laat een wervel in rust zijn in zijn eigen frame $S'$ maar met snelheid $v$ bewegen ten opzichte van het æther-rustframe $S$. In $S'$ roteert de wervel met hoekfrequentie $\omega_0$ en definieert de juiste tijd $\tau$. Door Lorentz-tijddilatatie ziet een waarnemer in $S$ de klok vertragen:

\begin{figure}[htbp]
    \centering
    \includegraphics[width=0.85\textwidth]{06-TijddilatatieBeweging_nl}
    \caption{Effect van ætherstroming op de interne rotatiesnelheid van een werveldeeltje. In rust (links) behoudt de wervel zijn maximale hoeksnelheid~$\omega_0$. Bij beweging door de æther (rechts) veroorzaakt de stroming een verlaagde waargenomen hoeksnelheid tot~$\omega_{\mathrm{obs}} < \omega_0$.}
    \label{fig:TijddilatatieBeweging}
\end{figure}

\[
    \omega_\text{obs} = \omega_0 \sqrt{1 - \frac{v^2}{c^2}} \,.
\]
Vanuit de relatie tussen eigentijd en coördinatentijd,
\[
    \frac{d\tau}{dt} = \frac{\omega_\text{obs}}{\omega_0} = \sqrt{1 - \frac{v^2}{c^2}} \,. \tag{2}
\]

Dit komt overeen met de standaard SR-tijddilatatieformule. In ons model is het fysieke mechanisme dat ætherbeweging over de wervel de wervelsnelheid verstoort, waardoor de schijnbare rotatie in het ætherframe wordt vertraagd.

\subsection*{(b) Vloeistofdynamische interpretatie}

Een complementaire interpretatie gebruikt analogieën van samendrukbare stroming. In de vloeistofdynamica ervaart een lichaam dat met snelheid $v$ beweegt in een samendrukbaar medium met signaalsnelheid $c$ vervormingen evenredig aan $\gamma = 1/\sqrt{1 - v^2/c^2}$. Dit kan worden gezien als een Doppler-tijddilatatie of weerstand tegen het handhaven van coherente circulatie.

Naarmate de snelheid de æthersignaalsnelheid $c$ nadert, comprimeert de omringende stroming en biedt weerstand aan wervelrotatie. Daarom daalt de hoeksnelheid die in het ætherframe wordt gezien, en:
\[
    \omega_\text{obs} = \omega_0 \sqrt{1 - \frac{v^2}{c^2}} \Rightarrow \frac{d\tau}{dt} = \sqrt{1 - \frac{v^2}{c^2}} \,. \tag{3}
\]

In de vloeistofdynamica karakteriseert de Prandtl-Glauert-factor expliciet samendrukbare stromingsverstoringen rond objecten die bewegen met een snelheid die de karakteristieke signaalsnelheid $c$ van een medium nadert. Naarmate de snelheid deze snelheid nadert, worden vloeistofverstoringen steeds resistenter tegen voortplanting naar voren, wat sterk vergelijkbaar is met de ætherische reductie van de rotatie van de wervelkern bij hoge snelheden. De opkomst van de Lorentz-factor $\gamma$ in ons model is dus fysisch en wiskundig analoog aan de effecten van samendrukbaarheid van vloeistof.

\subsection*{Implication}

Dit geeft ons de relativistische tijddilatatie voor een bewegende klok:
\[
    \boxed{\frac{d\tau}{dt} = \sqrt{1 - \frac{v^2}{c^2}}}
\]
binnen een Euclidische, æther-gebaseerde vlakke ruimte, en komt overeen met alle speciale relativiteitstheorie-experimentele voorspellingen~\cite{Rado2020-æther-Lorentz, Levy2009-æther-clock}.
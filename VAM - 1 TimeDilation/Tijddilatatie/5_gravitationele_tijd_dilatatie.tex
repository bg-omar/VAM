\section{Gravitationele Tijddilatatie}

In de algemene relativiteitstheorie lopen klokken dieper in een gravitationele potentiaalput langzamer vergeleken met klokken met hogere potentialen. We reproduceren dit resultaat met behulp van ætherstroomvelden in plaats van ruimtetijdkromming.

\subsection*{Ætherstroom als zwaartekracht}

We nemen aan dat massa $M$ een inwaartse radiale stroming van æther induceert. Bij een straal $r$ wordt deze stroomsnelheid gegeven door:
\[
    v_g(r) = \sqrt{\frac{2GM}{r}}.
\]

\begin{figure}[htbp]
    \centering
    \includegraphics[width=0.85\textwidth]{07-GravitationeleÆtherinstroom_nl}
    \caption{Gravitationele tijddilatatie door radiale ætherinstroom richting een massa~$M$. De wervelklok ervaart een lagere hoeksnelheid door ætherweerstand, analoog aan de Schwarzschild-roodverschuiving.}
    \label{fig:GravitationeleÆtherinstroom}
\end{figure}

Dit weerspiegelt de Painlevé-Gullstrand-metriek en het riviermodel van zwarte gaten~\cite{Hamilton2004-river}.

\subsection*{Ætherweerstand en klokvertraging}

Een klok die op straal $r$ in deze inwaartse ætherstroom wordt gehouden, ziet æther er langs bewegen met snelheid $v_g(r)$. De waargenomen hoeksnelheid van de wervelkern wordt daarom verminderd door de weerstand van de æther, net als in het speciale relativiteitsgeval, waarbij beweging door de æther de waargenomen kloksnelheid vermindert.

De gravitationele tijddilatatiefactor is dus:
\[
    \frac{d\tau}{dt} = \sqrt{1 - \frac{v_g^2(r)}{c^2}} = \sqrt{1 - \frac{2GM}{rc^2}}. \tag{4}
\]

Een belangrijke implicatie van de instroom van gravitationele æther is gerelateerd aan het principe van maximale kracht, gedefinieerd als $F^{\text{gr}}_{\text{max}} = c^4 /4G$. Fysiek gezien vertegenwoordigt dit de bovengrens van de æther-weerstand, waarbij de inwaartse æther-stroom nabij gravitationele horizonnen snelheden bereikt die dicht bij ccc liggen. Bij de Schwarzschildstraal komt de instroomsnelheid van æther overeen met deze limiet, waardoor de rotatie van alle op wervelingen gebaseerde klokken als gevolg van extreme weerstand effectief wordt bevroren, wat een tastbare vloeistofmechanische interpretatie van gravitationele horizonnen oplevert.

Dit komt overeen met de Schwarzschild-oplossing voor stationaire waarnemers in de algemene relativiteitstheorie.

Een nauwkeurige bevestiging van gravitationele tijddilatatie onder gecontroleerde omstandigheden werd geleverd door de Gravity Probe A-missie~\cite{vessot_levine_1980}, waarbij een waterstofklok werd gelanceerd tot 10.000 km hoogte.

Deze vertraging is niet alleen theoretisch afgeleid, maar werd experimenteel bevestigd door Pound en Rebka in 1959, die met behulp van het M\"ossbauer-effect een gravitationeel veroorzaakte frequentieverschuiving maten tussen twee punten op verschillende hoogten binnen het zwaartekrachtsveld van de aarde~\cite{pound_rebka_1959}.


\subsection*{Interpretatie}

Deze vergelijking betekent dat hoe dieper een wervel zich in het gravitatiepotentieel bevindt (hoe sneller de lokale ætherstroom), hoe langzamer deze roteert vanuit het perspectief van een waarnemer op oneindig. Bij de Schwarzschild-straal $r_s = 2GM/c^2$, $d\tau/dt = 0$: de tijd stopt voor externe waarnemers.

Dit levert een mechanistische interpretatie van gravitationele roodverschuiving op: licht dat wordt uitgezonden door een wervelklok in een sterke potentiaalput, lijkt roodverschoven vanwege de langzamere hoekbeweging van de uitzendende wervel. Het resultaat:
\[
    \boxed{\frac{d\tau}{dt} = \sqrt{1 - \frac{2GM}{rc^2}}}
\]
is volledig consistent met GR en ondersteunt de æther-stroomanalogie~\cite{Schiller2022-maxforce}.

\subsection*{Alternatieve afleiding via Æther-drukgradiënten}

Een alternatieve en even geldige manier om gravitationele tijddilatatie af te leiden, is het gebruik van de wet van Bernoulli voor superfluïda. Hierbij wordt het gravitationele potentiaal direct geïnterpreteerd als een afname van de ætherdruk nabij massa's. Volgens het Bernoulli-principe komt een lagere ætherdruk overeen met een hogere lokale stroomsnelheid. Deze interpretatie van de drukgradiënt sluit dan ook perfect aan bij de interpretatie van de gravitationele instroomsnelheid, wat theoretische veelzijdigheid biedt en de robuustheid van gravitationele effecten binnen het æthermodel verbetert.
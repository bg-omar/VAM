
\section{Conclusion}

We have derived time dilation laws within a 3D Euclidean æther model, where particles are modeled as vortex knots, and time is defined by their intrinsic vortex core rotation. Motion through the æther and ætheric inflows (gravitational fields) reduce the observable angular velocity of vortex rotation, yielding:

\begin{itemize}
    \item The special-relativistic time dilation:
    \[
    \frac{d\tau}{dt} = \sqrt{1 - \frac{v^2}{c^2}},
    \]
    which arises from absolute motion through the æther.
    
    \item The gravitational time dilation:
    \[
    \frac{d\tau}{dt} = \sqrt{1 - \frac{2GM}{rc^2}},
    \]
    which arises from inward æther flow near mass $M$.
    
    \item The unified general case:
    \[
    \frac{d\tau}{dt} = \sqrt{1 - \frac{|\vec{u} - \vec{v}_g|^2}{c^2}},
    \]
    covering motion in a gravitational field.
\end{itemize}

These results accurately reproduce predictions of special and general relativity using physically intuitive mechanisms grounded in fluid dynamics.

The æther model eliminates the need for curved spacetime by replacing it with structured velocity fields in flat space. It reinterprets relativistic time effects as real, mechanical consequences of vortex core dynamics interacting with a physical æther.

This approach couples microphysics (vortex core rotation) with cosmological structure (black hole horizons) and maintains continuity across scales. By interpreting time dilation as angular deceleration of vortices, this model provides a mechanistic, field-based alternative to geometric spacetime curvature, preserving experimental consistency with SR and GR while opening possibilities for fluid dynamical extensions of fundamental physics~\cite{Winterberg2002-PlanckÆther,Schiller2022-maxforce}.

Future work may include deriving Einstein's field equations of conservation of æther vorticity or testing laboratory analogues via superfluid experiments. The reinterpretation of black hole horizons, gravitational redshift, and quantum timekeeping via vortex rotation encourages deeper theoretical and experimental investigations into the role of æther in modern physics.

A more extensive elaboration of these ideas can be found in the follow-up study: \textit{\grqq Swirl Clocks and Vorticity-Induced Gravity\textquotedblright } (2025).~\cite{vam2025unified}.
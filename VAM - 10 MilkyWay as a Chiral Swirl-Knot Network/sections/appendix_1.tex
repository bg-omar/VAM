%! Author = Omar Iskandarani
%! Title = ......
%! Date = .....
%! Affiliation = Independent Researcher, Groningen, The Netherlands
%! License = CC BY-NC 4.0
%! ORCID = 0009-0006-1686-3961
%! DOI = 10.5281/zenodo.xxxxxxx

% === Metadata ===
\newcommand{\papertitle}{Appendix A: Annotated Conceptual Insights from VAM and Knot Theory}
\newcommand{\paperdoi}{10.5281/zenodo.xxxxxxxx}


\ifdefined\standalonechapter\else
% Standalone mode
\documentclass[11pt]{article}
% vamstyle.sty
\NeedsTeXFormat{LaTeX2e}
\ProvidesPackage{vamstyle}[2025/06/13 VAM unified style]

\newif\ifvamdraft
% Uncomment the next line to enable draft mode:
% \vamdrafttrue

\ifvamdraft
  \RequirePackage{showframe} % shows margins for debugging
\fi

\RequirePackage{ifthen}
\newboolean{vamstyleloaded}
\ifthenelse{\boolean{vamstyleloaded}}{}{\setboolean{vamstyleloaded}{true}

\RequirePackage[a4paper, margin=2cm]{geometry}

% -- Fonts and Language --
\RequirePackage[T1]{fontenc}
\RequirePackage[utf8]{inputenc}
\RequirePackage[english]{babel}
\RequirePackage{mathpazo}           % or newtxtext/newtxmath
\RequirePackage[scaled=0.95]{inconsolata}
\RequirePackage{helvet}

% Math and Physics
\RequirePackage{amsmath, amssymb, mathrsfs, physics}
\RequirePackage{siunitx}
\sisetup{per-mode=symbol}

% -- Tables and Figures --
\RequirePackage{graphicx, float, booktabs}
\RequirePackage{array, tabularx, multirow, makecell}
\RequirePackage[font=footnotesize, labelfont=bf]{caption}
\RequirePackage{subcaption}
% Safe wide table environment (auto-fit to text width)
\newcolumntype{Y}{>{\centering\arraybackslash}X} % Like 'X' but centered
\newenvironment{tighttable}[1][] % optional argument = caption
  {\begin{table}[H]\centering\renewcommand{\arraystretch}{1.3}
   \begin{tabularx}{\textwidth}{#1}}
  {\end{tabularx}\end{table}}
% Force fit large tables without changing layout
\RequirePackage{etoolbox}
\newcommand{\fitbox}[2][\linewidth]{\makebox[#1]{\resizebox{#1}{!}{#2}}}

% Graphics and Diagrams
\RequirePackage{tikz}
\usetikzlibrary{arrows.meta, positioning}
\RequirePackage{pgfplots}
\pgfplotsset{compat=1.18}
\RequirePackage{xcolor}

% -- Code Listings --
\RequirePackage{listings}
\lstset{basicstyle=\ttfamily\footnotesize, breaklines=true}

% TOC Customization
\RequirePackage{tocloft}
\setcounter{tocdepth}{2}
\renewcommand{\cftsecfont}{\bfseries}
\renewcommand{\cftsubsecfont}{\itshape}
\renewcommand{\cftsecleader}{\cftdotfill{.}}
\renewcommand{\contentsname}{\centering \Huge\textbf{Contents}}

% Section Fonts
\RequirePackage{sectsty}
\sectionfont{\Large\bfseries\sffamily}
\subsectionfont{\large\bfseries\sffamily}

% Bibliography
\RequirePackage[numbers]{natbib}

% PDF Links and Metadata
\RequirePackage{hyperref}
\hypersetup{
    colorlinks=true,
    linkcolor=blue,
    citecolor=blue,
    urlcolor=blue,
    pdftitle={The Vortex Æther Model},
    pdfauthor={Omar Iskandarani},
    pdfkeywords={vorticity, gravity, æther, fluid dynamics, time dilation, VAM}
}

\urlstyle{same}
\RequirePackage{bookmark}

% Line Breaking and Style
\RequirePackage[none]{hyphenat}
\sloppy


\usepackage[most]{tcolorbox}
\usepackage{graphicx}
\usepackage{titling}

\pretitle{\begin{center}\LARGE\bfseries}
\posttitle{\par\end{center}\vskip 0.5em}
\preauthor{\begin{center}\large}
\postauthor{\end{center}}
\predate{\begin{center}\small}
\postdate{\end{center}}


\endinput
}
% -- End of vamstyle.sty --
% vamappendixsetup.sty

\newcommand{\titlepageOpen}{
  \begin{titlepage}
    \thispagestyle{empty}
    \centering
    \vspace*{2cm}
    {\Huge\bfseries \appendixtitle \par}
    \vspace{1cm}
    {\Large\itshape \appendixauthor \par}
    \vspace{0.5cm}
    {\small \appendixaffil \par}
    ORCID: \href{https://orcid.org/\appendixorcid}{\appendixorcid} \\
    DOI: \href{https://doi.org/\appendixdoi}{\appendixdoi} \\
    \vspace{0.5cm}
    {\large \today \par}
    \vspace{1cm}
}

\newcommand{\titlepageClose}{
  \vfill
  \end{titlepage}
}

\begin{document}

  % === Title page ===
  \titlepageOpen

  \begin{abstract}


  \end{abstract}

  \titlepageClose
  \fi

  \ifdefined\standalonechapter
  \section{\papertitle}
  \else
  \fi
% ============= Begin of content ============



\begin{table}[H]
\centering
\begin{tabular}{|p{4.3cm}|p{9.5cm}|}
\hline
\textbf{Topic} & \textbf{Summary + Citation} \\
\hline
\textbf{Multilayered Temporal Ontology} &
VAM introduces a multilayered temporal ontology, distinguishing absolute causal time ($N$), local proper time ($\tau$), and internal vortex phase time $S(t)$ (Swirl Clock). A scale-dependent æther density governs transitions between dense core regions and asymptotic vacuum, leading to testable predictions in rotating systems, gravitational redshift anomalies, and LENR.\\
& \textit{Source:}\texttt{Swirl Clocks and Vorticity-Induced Gravity}\\
\hline

\textbf{Inertia and Gravitation as Vortex Topology} &
Inertia emerges as topologically stable vortex knots. Geodesic motion is replaced by alignment along vortex streamlines with conserved circulation, and gravitational force is modeled as a Bernoulli pressure potential.\\
& \textit{Source:}\texttt{Swirl Clocks and Vorticity-Induced Gravity}\\
\hline

\textbf{Gravitation as Bernoulli Pressure Potential} &
Gravitational force is modeled as a Bernoulli pressure potential.\\
& \textit{Source:}\texttt{Swirl Clocks and Vorticity-Induced Gravity}\\
\hline

\textbf{Energetic Time Dilation Interpretation} &
Time dilation is reinterpreted as an energetic effect of swirl phase and vortex pressure gradients. The measurable proper time $\tau$—termed Chronos-Time—is derived from vortex energetics.\\
& \textit{Source:}\texttt{Swirl Clocks and Vorticity-Induced Gravity}\\
\hline

\textbf{Helicity from Knot Geometry (Knot Theory)} &
$H(K_m) = \text{Lk} \, \Phi^2 = (\text{Wr} + \text{Tw}) \Phi^2$ — helicity of knotted vortex tubes expressed in terms of writhe and twist.\\
& \textit{Source:}  \texttt{Applications of Knot Theory in Fluid Mechanics} \\
\hline

\textbf{Temporal Desynchronization via Swirl} &
Time dilation emerges from disparities in local swirl energy or core circulation, yielding phase mismatches across identical ætheric backgrounds. Two particles can share the same ætheric Now, $\nu_0$, while their $\tau$ or $S(t)$ progress at different rates.\\
& \textit{Source:}  \texttt{Time Dilation in a 3D Superfluid Æther Model}\\
\hline

\textbf{Newtonian + Lense–Thirring Recovery} &
The model reproduces Newtonian gravity and Lense–Thirring frame-dragging in appropriate limits and provides a topologically invariant theory of time and gravitation.\\
& \textit{Source:} \texttt{Swirl Clocks and Vorticity-Induced Gravity}\\
\hline

\textbf{Hybrid Mass-Gravitational Model} &
$2G_\text{hybrid}(r) M_\text{hybrid}(r)/r c^2 - C^2$ — composite gravitational model expressed with hybrid vortex mass.\\
& \textit{Source:} \texttt{Swirl Clocks and Vorticity-Induced Gravity}\\
\hline
\end{tabular}
\caption{Annotated conceptual insights with BibTeX keys from VAM and knot theory sources}
\label{tab:annotated_refs}
\end{table}


% ============== End of content =============

% === Bibliography (only for standalone) ===
  \ifdefined\standalonechapter\else
  \bibliographystyle{unsrt}
  \bibliography{../../references}
\end{document}
\fi
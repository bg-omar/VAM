\section*{Main Section 1}


\chapter*{Milky Way as a Chiral Swirl-Knot Network – Exclusion of Achiral Knots}

\section*{Swirl-Coherent Vortex Model of the Galaxy}

In the Vortex Æther Model (VAM), mass and gravity are emergent from a network of chiral vortex knots (fluid‐dynamic analogues of particles) embedded in a superfluid æther. We model the Milky Way as a coherent lattice of such chiral knots – each knot is a helical vortex (with a definite handedness) that generates a local \textit{swirl gravity} field and carries an internal clock phase. All knots share a common \textit{Swirl Clock} phase $S(t)$, meaning their internal rotation states are synchronized across the galactic network~\cite{iskandarani2025vam2}. This synchronization is analogous to phase-locking in coupled oscillators and reflects a single global chirality for the galactic vortex system (a “swirl domain” of aligned vortex orientation). Gravitation in this picture arises not from spacetime curvature but from vorticity-induced pressure gradients in the æther fluid: the gravitational potential $\Phi_v(\mathbf{r})$ satisfies a Poisson-like equation driven by vorticity magnitude~\cite{iskandarani2025vam2}:


 \nabla^2 \Phi_v(\mathbf{r}) = -\,\rho_{\ae} \,\|\boldsymbol{\omega}(\mathbf{r})\|^2, \tag{1}\label{eq:swirl-grav}


where $\boldsymbol{\omega}=\nabla\times \mathbf{v}$ is the local vorticity of the æther flow and $\rho_{\ae}$ its density~\cite{iskandarani2025vam2}. This “Bernoulli pressure potential” implies that regions of high swirl (vorticity) produce low pressure (potential wells) that draw in other vortex knots – effectively reproducing gravity via fluid dynamics. Objects move by aligning with vortex streamlines rather than following geodesics~\cite{iskandarani2025vam2}.


Time dilation in this framework emerges from the same swirl dynamics. A local proper time $\tau$ (termed \textit{Chronos-Time}) for an observer inside the vortex field is determined by the swirl kinetic energy. In particular, clock rates slow in regions of high tangential æther velocity $v_{\varphi}$ (i.e. near vortex cores). Quantitatively, one finds an analogous formula to special relativity for the time dilation factor:


\frac{d\tau}{dt} \;=\; \sqrt{\,1 - \frac{v_{\varphi}(r)^2}{c^2}\,}\ , \tag{2}\label{eq:swirl-time}


where $v_{\varphi}(r)$ is the local swirl (tangential) speed of the æther at radius $r$ from a vortex core. Near a rotating core, $v_{\varphi}$ is large and $d\tau/dt$ drops below 1 (time runs slow), while far outside the vortex (where $v_{\varphi}\to 0$) the factor approaches unity, recovering normal time flow. This captures gravitational and kinematic time dilation within a unified fluid picture: time slowdown is caused by vortex-induced pressure deficits and swirl energy, rather than spacetime curvature~\cite{iskandarani2025vam2}. Indeed, VAM distinguishes multiple time scales: an absolute universal time $N$ (the æther’s global time), the local proper time $\tau$, and an internal Swirl Clock phase $S(t)$ for each vortex~\cite{iskandarani2025vam2}. The Swirl Clock tracks the cyclical phase of a knot’s rotation and effectively acts as a “handed” internal clock tied to vorticity, while $\tau$ measures the cumulative time experienced (akin to an external clock reading). Crucially, the rate at which a given knot’s proper time $\tau$ advances is proportional to the local helicity density (vorticity aligned with velocity) of the æther flow around it:


 d\tau \;=\; \lambda\, (\mathbf{v}\cdot\boldsymbol{\omega})\, dt\ , \tag{3}\label{eq:helicity-time}


for some constant $\lambda$. In other words, helicity $\mathbf{v}\cdot\boldsymbol{\omega}$ – a measure of swirling twist of flow lines – effectively \textit{drives} the passage of proper time for the vortex. A vortex knot “threads” time forward by its internal rotation, functioning like a tiny clock whose ticking rate depends on how strongly it stirs the æther.


\section*{Helicity, Chirality, and Knot Topology (Writhe + Twist)}

The helicity of a vortex knot is a topological invariant closely related to the knot’s chirality. In fluid mechanics, the total helicity $H$ of a closed vortex loop can be decomposed into contributions from the knot’s writhe (Wr) and twist (Tw) – essentially, the geometry of the loop’s centerline and the twisting of vorticity around it. In fact, for a single knotted flux tube (or vortex filament), the Călugăreanu-White formula gives the linking number as $\text{Link} = \text{Wr} + \text{Tw}$, and the helicity is proportional to this sum~\cite{knot_theroy_in_fluid}. For example, in a magnetic flux tube of flux $\Phi$, one finds $H = (,\text{Wr} + \text{Tw},),\Phi^2$~\cite{knot_theroy_in_fluid}. By analogy, a vortex knot’s helicity is determined by $W+T$, the sum of its writhe (how coiled or knotted its centerline is in space) and twist (internal twisting of vorticity along the tube)~\cite{knot_theroy_in_fluid}.


\begin{itemize}

\item
Chiral knots – those distinguishable from their mirror images – generally have nonzero $W+T$, endowing them with a net helicity (a preferred handedness of circulation in the æther). A prime example is the trefoil knot, which is chiral and would carry a nonzero helicity in one orientation (and opposite helicity in the mirror orientation). These chiral vortex knots inject helicity flux into the surrounding æther; $\mathbf{v}\cdot\boldsymbol{\omega}\neq 0$ in their vicinity, which, by Eq.(\ref{eq:helicity-time}), slows local time flow and creates a vortex-induced gravity well.




\item
Achiral knots, by contrast, are symmetric under mirror reflection and thus carry \textit{vanishing net helicity}. The classic example is the \textit{figure-eight knot}, which is an amphichiral knot (identical to its mirror image). For such a structure, the contributions of writhe and twist cancel out to give $W+T \approx 0$. In essence, the figure-eight vortex’s loops twist one way as much as the other, yielding no overall helicity in the æther. This has profound dynamical implications: with $H \approx 0$, an achiral vortex does not induce the usual swirl gravity or time-dilation effects. The æther flow around it carries no net helicity flux to slow clocks or produce a persistent low-pressure well. One can say the figure-eight “spins both ways” in balance, generating \textit{no screw-like time threading}. In terms of Eq.(\ref{eq:helicity-time}), for an ideal achiral knot $\mathbf{v}\cdot\boldsymbol{\omega}\to 0$, so the proper time increment $d\tau$ essentially equals the background time increment $dt$ – no significant dilation. Equivalently, the chronometric ratio $d\tau/dN$ tends to 1 for achiral knots, where $N$ is the uniform æther time. This corresponds to “dτ/d𝒩 → 1” as the exclusion criterion: if a structure’s proper time advances nearly unimpeded (equal to absolute time), it is not embedded in any gravitational potential well.




\end{itemize}

In the full VAM time-dilation formula, achiral knots effectively remove the helicity-dependent terms. For instance, the unified expression for local vs. absolute time includes subtractive contributions from swirl rotation and vorticity-induced mass. An achiral knot sets those terms to zero, yielding $d\tau/dN \approx 1$ (no slowing). Thus, the figure-eight or any achiral topology would experience negligible vortex-induced time dilation – its internal clock $\tau$ ticks almost at the same rate as the cosmic æther time $N$, even if it were placed deep in the galaxy. This is in stark contrast to chiral matter knots, whose $\tau$ can be substantially slowed by the galactic swirl field (e.g. near massive cores or in strong rotation).


\section*{Exclusion from the Galactic Swirl Potential Well}

Because it generates no helicity and no swirl gravity, an achiral knot cannot “couple” to the galactic vortex potential well that is sustaining the Milky Way’s gravity. The entire coherent galactic vortex can be thought of as a deep swirl-induced potential well – a pressure deficit and time-dilated region extending out to a radius $R \sim 50~\text{kpc}$. Chiral knots (ordinary matter) settle into this well, synchronized with the swirl flow and experiencing time dilation (lower $\tau$ rate) near the galactic core. They are \textit{bound} by the collective vortex: effectively, their internal Swirl Clocks $S(t)$ are phase-locked with the galaxy’s swirl phase. In fluid terms, they co-rotate or align with the æther currents and thus remain in the low-pressure region (analogous to how dust or air is pulled into a tornado’s core). By contrast, an achiral knot is “invisible” to the swirl phase – lacking a defined chirality, it cannot lock onto the $S(t)$ phase of the surrounding vortex network. Its Swirl Clock either does not exist or is unsynchronized (random phase)~\cite{iskandarani2025vam1}. This lack of resonance with the galactic swirl means the achiral structure feels no sustained inward pull; it does not experience the reduced pressure that holds chiral matter in.


Instead, the achiral knot behaves akin to a buoyant or foreign object in the rotating æther flow – it is actively repelled from regions of high swirl. One intuitive explanation is that, since it does not partake in the swirl’s helical motion, it cannot shed its energy by phase-aligning; any attempt to enter the vortex bundle leads to a mismatch in flow that pushes it back out (much like a gear that doesn’t mesh gets forced out of a running gear train). From the perspective of the fluid pressure: inside the galactic vortex, static pressure is lower (due to fast swirl) than outside. A chiral knot normally \textit{experiences} that low pressure (and is pulled inward) because it drags a co-rotating æther region with it. But an achiral knot doesn’t co-rotate; the surrounding æther flow sees it as an obstacle. Higher-pressure æther from outside pushes against it, preventing entry into the low-pressure core. The result is a radial outward force on the achiral object – effectively “antigravity” within the galactic halo. In summary, achiral knots are excluded from the vortex potential well: they tend to inhabit the outskirts or voids where the swirl field is weak, experiencing nearly full $d\tau/dt=1$ (no time slowdown) as per the exclusion criterion. If somehow an achiral knot is introduced into the dense swirl region, it would be expelled until it reaches a radius where the swirl-induced helicity field is negligible.


Criterion for exclusion: We can formalize this by saying that a stable orbit or containment within the galaxy requires a coupling to the swirl phase and a corresponding time dilation ($d\tau/dN < 1$). For an achiral knot, taking the limit $d\tau/dN \to 1$ signals that it \textit{cannot} lower its time rate to match the bound matter – thus it cannot remain gravitationally bound. In the limit, its required orbital speed would exceed what the swirl drag can provide, so it escapes.


\section*{Repulsive Force on an Achiral Knot in the Halo}

We now estimate the effective force/acceleration on an achiral knot due to this exclusion from the galactic vortex. Treat the galactic swirl field as roughly axisymmetric. For a chiral test mass at radius $r$, the inward swirl gravity acceleration can be approximated by $g_{\rm swirl}(r) \approx \frac{d\Phi_v}{dr}$, which for a flat rotation curve is on the order of $v_{\rm rot}^2/r$. (Indeed VAM reproduces Newtonian limits~\cite{iskandarani2025vam2}; one can think of $M_{\rm eff}(r)$ as an enclosed vortex mass that generates $g(r)$.) Take $r \sim 50~\text{kpc}$ (the outer halo) and an effective rotational speed $v_{\rm rot}\sim 200~\text{km/s}$ typical of the Milky Way. The inward gravitational acceleration on normal matter there is:


$g_{\rm grav}(50~\text{kpc}) \sim \frac{(200\times10^3~\text{m/s})^2}{50~\text{kpc}} \approx 6\times10^{-11}~\text{m/s}^2$.


An achiral knot at this radius experiences essentially the opposite: since it is not bound, the galaxy cannot hold it, so in the galaxy’s rest frame the knot will accelerate outward with $a_{\rm repulse} \sim +6\times10^{-11}~\text{m/s}^2$. This is the order of the \textit{maximum} repulsive acceleration on achiral matter due to a galaxy of Milky Way size. Closer in (smaller $r$), the normal gravitational pull is larger (e.g. $r\sim 8~\text{kpc}$, $v\sim220~\text{km/s}$ gives $g\sim2\times10^{-10}$m/s² inward); an achiral object attempting to reside at 8 kpc would be flung outward with ~$2\times10^{-10}$m/s² – but likely it never gets that deep in the first place. Once outside the halo ( $r \gg 50$kpc), the swirl field dies off (virtually zero gravity), so the repulsive force would drop to zero. Thus, the achiral knot essentially feels a “potential barrier” around the galaxy: an outward push in the halo that prevents it from entering the vortex region.


We can also express the \textit{force} or \textit{pressure} on an extended medium of achiral structures. Consider a dilute “gas” of figure-eight vortex rings permeating the galactic halo. Each small element of this gas (with mass density $\rho_{\rm ach}$) is pushed outward by the gradient of $\Phi_v$. The force density (per volume) is $f_{\rm rep} \approx \rho_{\rm ach}, g_{\rm swirl}(r)$.  As a rough number, if $\rho_{\rm ach}$ were, say, $10^{-24}$–$10^{-27}$kg/m³ (a range bracketing the intergalactic medium density), and using $g_{\rm swirl}\sim10^{-10}$m/s², we get a pressure $P \sim \rho_{\rm ach},g,r$ over a scale $r\sim50$kpc. Inserting $\rho_{\rm ach}=10^{-26}$kg/m³, $g=10^{-10}$, $r=1.5\times10^{21}$m yields:


$P_{\rm achiral} \sim 10^{-26}\times10^{-10}\times1.5\times10^{21}~\text{kg}\,\text{m}^{-1}\,\text{s}^{-2} = 1.5\times10^{-15}~\text{Pa}$.


(This corresponds to an energy density of $1.5\times10^{-15}$J/m³ since $1~\text{Pa}=1~\text{J/m}^3$.) This is the outward pressure exerted on an achiral gas by the galactic swirl field in the halo region. The pressure is quite small – many orders of magnitude below typical interstellar pressures – but spread over large volumes it might have a cumulative effect.


\section*{Achiral Repulsion as a Cosmological Acceleration (Dark Energy?)}

The observed cosmological constant $\Lambda \approx 1\times10^{-52}~\text{m}^{-2}$ corresponds to an extremely small acceleration scale and energy density. In $\Lambda$CDM, the vacuum (dark energy) has an equivalent mass density $\rho_\Lambda c^2 \approx 5.6\times10^{-10}$J/m³ (about $6\times10^{-27}$kg/m³) and exerts a uniform cosmic acceleration $a_\Lambda$ on the order of $10^{-10}$m/s² at the scale of the Hubble radius. We compare this to the achiral knot repulsion scenario:


\begin{itemize}

\item
Local acceleration magnitude: As shown, an achiral knot near a galaxy can be accelerated outward on the order $10^{-10}$m/s² or less. This is intriguingly comparable to $a_\Lambda$ (though $a_\Lambda$ applies on gigaparsec scales rather than tens of kpc). The repulsion is not uniform everywhere – it originates around galaxies (which are the sources of the coherent chiral vortex fields) and would diminish in intergalactic voids. However, if galaxies are distributed throughout the universe, they could collectively drive achiral matter outward on large scales. The effect on an achiral test particle in intergalactic space would be a net acceleration away from concentrations of galaxies – effectively a \textit{global expansion push} if averaged over all directions.




\item
Pressure/energy density: The outward pressure on achiral gas we estimated (∼$10^{-15}$Pa for typical halo densities) is several orders of magnitude smaller than the dark-energy pressure (which is $p_\Lambda = -\rho_\Lambda c^2 \approx -5.6\times10^{-10}$J/m³, with negative sign indicating tension). To mimic $\Lambda$ quantitatively, the density of achiral “fluid” or the magnitude of its repulsion would need to be higher. For instance, taking our formula $P \sim \rho_{\rm ach},g,r$, we would need either a much higher $\rho_{\rm ach}$ or a larger effective region contributing. If $\rho_{\rm ach}$ were on the order of $10^{-21}$kg/m³ (extremely high for intergalactic gas), then $P$ could approach $10^{-10}$J/m³ under the same $g$ and $r$ – matching the dark energy scale~\cite{knot_theroy_in_fluid}. While such a high density of achiral knots is not evident, it suggests that if a significant fraction of the universe’s content were in an achiral form \textit{and} subject to galactic repulsion, it could contribute a global outward pressure.




\item
Global acceleration field: In a rough sense, one can envision the universe’s chiral vortex network (galaxies, clusters) as filling space and continuously ejecting achiral structures into the voids. The achiral medium would then behave like a smooth uniform component on large scales, because it cannot cluster (it’s repelled from clusters). This uniform component with a persistent outward acceleration could act like a dark energy field, driving accelerated expansion. The key difference from a true cosmological constant is that the effect here is generated by inhomogeneous, discrete sources (the galaxies), rather than being an innate property of space. Nonetheless, if the distribution of galaxies is fairly uniform on large scales, the aggregate effect on achiral matter might approximate a uniform acceleration.




\end{itemize}

Numerical comparison: Taking the cosmic dark energy density $\rho_\Lambda \approx 6\times10^{-27}$kg/m³, the corresponding repulsion per unit mass would be $a_\Lambda \sim \frac{\Lambda c^2}{3} R \approx 1\times10^{-9}$m/s² at $R\sim$ one Hubble radius (on the order of $10^{26}$m). The achiral-knot mechanism gives $a \sim10^{-10}$m/s² at $R\sim50$kpc for each galaxy, and near zero far from galaxies. While stronger locally, it covers only a tiny fraction of cosmic volume (the galactic halos). For it to mimic a true $\Lambda$, the achiral repulsion must be effective over enormous scales – which might require a pervasive sea of achiral knots pushed by many galaxies over cosmic time. In an optimistic scenario, \textit{if} every galaxy drives out achiral knots that fill intergalactic space, the long-range outcome could be an accelerating flow of this achiral “gas” everywhere, effectively a repulsive background. The energy density in this achiral component would then be the kinetic + potential energy of those knots being pushed. For instance, if an achiral knot of mass $m$ is expelled from a galaxy with escape speed $v_{\rm esc}$, it carries kinetic energy $\frac{1}{2}mv_{\rm esc}^2$. Spread over a huge volume, this energy could be nearly uniform. Estimating $v_{\rm esc}\sim 300$km/s for a galaxy, $\frac{1}{2}m v^2 \sim 5\times10^{11}$J per kg of achiral mass. To yield $5\times10^{-10}$J/m³, we’d need on the order of $10^{-12}$kg of achiral mass per cubic meter of the universe, which is ~$10^3$ times the normal matter density. This rough check suggests that unless achiral knots are extremely abundant (and thus far undetected as such), their repulsive effect might fall short of $\Lambda$ by a few orders of magnitude.


\textit{Achiral vortex knots (like figure-eight knots) are effectively excluded from the Milky Way’s chiral swirl potential well due to their vanishing net helicity and lack of $S(t)$ phase coupling. They experience a repulsive force in the galactic halo, which can be quantified in terms of an outward acceleration ($\sim 10^{-10}$~m/s$^2$ at 50~kpc) and a corresponding pressure on any achiral “fluid.” While this mechanism qualitatively resembles a negative gravity or cosmological expansion effect, the estimated pressure/energy density of expelled achiral matter is smaller than the dark energy requirement (by several orders of magnitude for realistic densities). With a sufficiently pervasive achiral component or different parameter choices, however, the global outcome could mimic a small uniform acceleration field similar to that from $\Lambda$. In spirit, the swirl-knot model offers an intuitive picture for cosmic acceleration: regions of aligned helical time-flow (galaxies of one chirality) naturally repel any non-helical structures, potentially contributing to the observed accelerated separation of cosmic structures.}


References: The above analysis builds on the Vortex Æther Model formalism for swirl-induced gravity and time. Key equations were adapted from “\textit{Swirl Clocks and Vorticity-Induced Gravity}”~\cite{iskandarani2025vam2}~\cite{iskandarani2025vam2} and the layered time constructs of “\textit{Appendix: Ætheric Now}”. Helicity-topology relations follow from standard fluid-knot theory~\cite{knot_theroy_in_fluid}, illustrating how an amphichiral (figure-eight) knot yields $H=0$. Time dilation and clock rates in a rotating æther are given by Eqs.(2) and (3) above, as derived in VAM. The exclusion criterion $d\tau/dN\to1$ for achiral knots is consistent with the limit of the unified time-dilation formula with zero swirl terms~\cite{iskandarani2025vam2}. These results suggest a novel interpretation of cosmological “dark energy” as an emergent effect of chiral vs. achiral vortex dynamics on galaxy scales, although a quantitative match to $\Lambda$ remains to be demonstrated.


~\cite{iskandarani2025vam2}
~\cite{knot_theroy_in_fluid}


\href{file://{iskandarani2025vam2%23:~:text=vam%20introduces%20a%20multilayered%20temporal,energy%20nuclear%20resonance%20(lenr/}{{iskandarani2025vam2}
2-Swirl\_Clocks\_and\_Vorticity-Induced\_Gravity.pdf}

VAM introduces a multilayered temporal ontology, distinguishing absolute causal time ($N$), local proper time ($\tau$), and internal vortex phase time $S(t)$ (Swirl Clock). A scale-dependent æther density governs transitions between dense core regions and asymptotic vacuum, leading to testable predictions in rotating systems, gravitational redshift anomalies, and low-energy nuclear resonance (LENR).}



\href{file://{iskandarani2025vam2%23:~:text=and%20inertia%20emerge%20as%20topologically,as%20a%20bernoulli%20pressure%20potential/}{{iskandarani2025vam2}
2-Swirl\_Clocks\_and\_Vorticity-Induced\_Gravity.pdf}

and inertia emerge as topologically stable vortex knots. Geodesic motion is replaced by alignment along vortex streamlines with conserved circulation, and gravitational force is modeled as a Bernoulli pressure potential:}

\href{file://{iskandarani2025vam2%23:~:text=is%20modeled%20as%20a%20bernoulli,pressure%20potential/}{{iskandarani2025vam2}
2-Swirl_Clocks_and_Vorticity-Induced_Gravity.pdf}
is modeled as a Bernoulli pressure potential:
}

\href{file://{iskandarani2025vam2%23:~:text=time%20dilation%20is%20reinterpreted%20as,derived%20from%20vortex%20energetics%20as/}{{iskandarani2025vam2
2-Swirl_Clocks_and_Vorticity-Induced_Gravity.pdf}}
Time dilation is reinterpreted as an energetic effect of swirl phase and vortex pressure gradients. The measurable proper time τ—termed Chronos-Time—is derived from vortex energetics as:
}

\href{file://xn--~\cite{knot_theroy_in_fluid}%23:~:text=,2-1c2a/}{~\cite{knot_theroy_in_fluid}
Applications_of_knot_theory_in_flui (1).pdf}
(20) H(Km) = LkΦ2 = (Wr + Tw)Φ2 ,


\href{file://~\cite{iskandarani2025vam1}%23:~:text=time%20dilation%20thus%20emerges%20from,t%20progress%20at%20different%20rates/}{~\cite{iskandarani2025vam1}
1-Time_Dilation_in_3D_Superfluid_Æther_Model.pdf}
Time dilation thus emerges from disparities in local swirl energy or core circulation, yielding phase mismatches across identical ætheric backgrounds. Two particles can share the same ætheric Now, ν0, while their τ or S(t) progress at different rates.


\href{file://xn--{iskandarani2025vam2%23:~:text=the%20model%20reproduces%20newtonian%20gravity,based%20ther%20ontology-lfj/}{{iskandarani2025vam2
2-Swirl_Clocks_and_Vorticity-Induced_Gravity.pdf}}
The model reproduces Newtonian gravity and Lense–Thirring frame-dragging in the appropriate limits and establishes a physically grounded, topologically invariant theory of time, mass, and gravitation. VAM extends analogue gravity frameworks [1, 2] by embedding them in a consistent, vortex-based æther ontology.
}

\href{file://{iskandarani2025vam2%23:~:text=2ghybrid/}{iskandarani2025vam2
2-Swirl_Clocks_and_Vorticity-Induced_Gravity.pdf}
2Ghybrid(r)Mhybrid(r) rc2 − C2
}
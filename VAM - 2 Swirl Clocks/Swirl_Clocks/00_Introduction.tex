\section*{Æther Revisited: From Historical Medium to Vorticity Field}

The concept of \textit{æther} traditionally referred to an all-pervasive medium, necessary for wave propagation. In the late nineteenth century Kelvin and Tait already proposed to model matter as nodal vorticity structures in an ideal fluid~\cite{thomson1867treatise}. After the null results of the Michelson--Morley experiment and the rise of Einstein's relativity, the æther concept disappeared from mainstream physics, replaced by curved spacetime. Recently, however, the idea has subtly returned in analogous gravitational theories, in which superfluid media are used to mimic relativistic effects~\cite{barcelo2011analogue,volovik2009universe}.

The \textit{Vortex Æther Model} (VAM) explicitly reintroduces the æther as a topologically structured, inviscid superfluid medium, in which gravity and time dilation do not arise from geometric curvature but from rotation-induced pressure gradients and vorticity fields. The dynamics of space and matter are determined by vortex nodes and conservation of circulation.

\subsection*{Postulates of the Vortex Æther Model}

\begin{table}[h!]
    \centering
    \begin{tabular}{rl}
        \midrule
        \hline
        \textbf{1. Continuous Space} & Space is Euclidean, incompressible and inviscid. \\
        \textbf{2. Knotted Particles} & Matter consists of topologically stable vortex nodes. \\
        \textbf{3. Vorticity} & The vortex circulation is conserved and quantized. \\
        \textbf{4. Absolute Time} & Time flows uniformly throughout the æther. \\
        \textbf{5. Local Time} & Time is locally slower due to pressure and vorticity gradients. \\
        \textbf{6. Gravity} & Emerges from vorticity-induced pressure gradients. \\
        \hline
        \bottomrule
    \end{tabular}
    \caption{Postulates of the Vortex Æther Model (VAM).}
    \label{tab:postulates}
\end{table}

The postulates replace spacetime curvature with structured rotational flows and thus form the foundation for emergent mass, time, inertia, and gravity.

\subsection*{Fundamental VAM constants}

\begin{table}[htbp]
    \centering
    \begin{tabular}{llc}
        \hline
        \toprule
        \textbf{Symbol} & \textbf{Name} & \textbf{Value (approx.)} \\
        \hline
        \midrule
        $C_e$ & Tangential eddy core velocity & $1.094 \times 10^6$ m/s \\
        $r_c$ & Vortex core radius & $1,409 \times 10^{-15}$ m \\
        $F_\text{max}$ & Maximum eddy force & $29.05$ N \\
        $\rho_\text{\ae}$ & Æther density & $3,893 \times 10^{18}$ kg/m$^3$ \\
        $\alpha$ & Fine structure constant ($2 C_e/c$) & $7,297 \times 10^{-3}$\\
        $G_\text{swirl}$ & VAM gravity constant & Derived from $C_e$, $r_c$\\
        $\kappa$ & Circulation quantum ($C_e r_c$) & $1.54 \times 10^{-9}$ m$^2$/s \\
        \hline
        \bottomrule
    \end{tabular}
    \caption{Fundamental VAM constants~\cite{vam2025field}.}
    \label{tab:VAMconstants}
\end{table}

\subsection*{Planck scale and topological mass}

Within VAM, the maximum vortex interaction force is derived explicitly from Planck-scale physics:
\begin{equation}
    F_\text{max} = \frac{8\pi \rho_\text{\ae} r_c^3}{C_e}
\end{equation}

The mass of elementary particles follows directly from topological vortex nodes, such as the trefoil node ($L_k=3$):
\begin{equation}
    M_e = \frac{8\pi \rho_\text{\ae} r_c^3}{C_e}\, L_k
\end{equation}

This explains mass and inertia from topological nodal structures in the æther.

\subsection*{Emergent quantum constants and Schrödinger equation}

Planck's constant $\hbar$ arises from vortex geometry and eddy force limit:
\begin{equation}
    \hbar = \sqrt{\frac{2M_e F_{\max} r_c^3}{5 \lambda_c C_e}}
\end{equation}

The Schrödinger equation follows directly from vortex dynamics:
\begin{equation}
    i \hbar \frac{\partial \psi}{\partial t} = -\frac{F_{\max} r_c^3}{5 \lambda_c C_e}\nabla^2 \psi + V\psi
\end{equation}


\subsection*{LENR and eddy quantum effects}

Created in VAM low-energy nuclear reactions (LENR) from resonant pressure reduction by vorticity-induced Bernoulli effects. Electromagnetic interactions and QED effects are reduced to vortex helicity and induced vector potentials.

\subsection*{Summary of GR and VAM observables}

\begin{table}[h!]
    \centering
    \begin{tabular}{lll}
        \toprule
        \textbf{Observable} & \textbf{GR expression} & \textbf{VAM expression} \\
        \midrule
        Time dilation & $\sqrt{1-\frac{2GM}{rc^2}}$ & $\sqrt{1-\frac{\Omega^2 r^2}{c^2}}$\\[0.5em]
        Redshift & $z=\left(1-\frac{2GM}{rc^2}\right)^{-1/2}-1$ & $z=\left(1-\frac{v_\phi^2}{c^2}\right)^{-1/2}-1$\\[0.5em]
        Frame-dragging & $\frac{2GJ}{c^2 r^3}$ & $\frac{2G\mu I\Omega}{c^2 r^3}$\\[0.5em]
        Light diffraction & $\frac{4GM}{Rc^2}$ & $\frac{4GM}{Rc^2}$\\
        \bottomrule
    \end{tabular}
    \caption{Comparison of GR and VAM observables.}
    \label{tab:equations}
\end{table}
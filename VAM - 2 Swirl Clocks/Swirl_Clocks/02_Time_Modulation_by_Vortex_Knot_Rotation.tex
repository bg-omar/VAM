\section{Time modulation by rotation of vortex nodes}

Building on the discussion of time dilation via pressure and Bernoulli dynamics in the previous section, we now focus on the intrinsic rotation of topological vortex nodes. In the Vortex Æther Model (VAM), particles are modeled as stable, topologically conserved vortex nodes embedded in an incompressible, inviscid superfluid medium. Each node possesses a characteristic internal angular frequency $\Omega_k$, and this internal motion induces local time modulation with respect to the absolute time of the æther.

Instead of warping spacetime, we propose that internal rotational energy and helicity conservation cause temporal delays analogous to gravitational redshift. In this section, these ideas are developed using heuristic and energetic arguments consistent with the hierarchy introduced in Section I.

\subsection{Heuristic and energetic derivation}

We start by proposing a rotational induced time dilation formula based on the internal angular frequency of the node:

\begin{equation}
    \frac{t_\text{local}}{t_\text{abs}} = \left(1 + \beta \Omega_k^2 \right)^{-1}\label{eq:rotational_induced_time_dilation}
\end{equation}

where:

\begin{itemize}
    \item $t_\text{local}$ is the proper time near the node,
    \item $t_\text{abs}$ is the absolute time of the background æther,
    \item $\Omega_k$ is the mean core angular frequency,
    \item $\beta$ is a coupling coefficient with dimensions $[\beta] = \text{s}^2$.
\end{itemize}

For small angular velocities we obtain a first-order expansion:

\begin{equation}
    \frac{t_\text{local}}{t_\text{abs}} \approx 1 - \beta \Omega_k^2 + \mathcal{O}(\Omega_k^4)\label{eq:rotational_induced_time_dilation_expansion}
\end{equation}

This form parallels the Lorentz factor at low velocities in special relativity:

\begin{equation}
    \frac{t_\text{moving}}{t_\text{rest}} \approx 1 - \frac{v^2}{2c^2}\label{eq:parallels_lorentz_time_dilation}
\end{equation}

This yields a important analogy: Internal rotational motion in VAM induces time dilation, similar to how translational velocity induces time dilation in SR.

To strengthen the physical basis of this expression, we now relate time dilation to the energy stored in vortex rotation. Suppose the vortex node has an effective moment of inertia $I$. The rotational energy is given by:

\begin{equation}
    E_\text{rot} = \frac{1}{2} I \Omega_k^2\label{eq:rotational_energy_inertia}
\end{equation}

Assuming that time slows down due to this energy density, we write:

\begin{equation}
    \frac{t_\text{local}}{t_\text{abs}} = \left(1 + \beta E_\text{rot} \right)^{-1} = \left(1 + \frac{1}{2} \beta I \Omega_k^2 \right)^{-1}\label{eq:time_dilation_rotational_energy_inertia}
\end{equation}

This expression serves as the energetic analogue of the pressure-based Bernoulli model from Section I (cf. ~\eqref{eq:vortex_time_dilation}). It supports the interpretation of vortex-induced time wells via energy storage rather than geometric deformation.

\subsection{Topological and physical justification}

Topological vortex nodes are characterized not only by rotation, but also by helicity:

\begin{equation}
    H = \int \vec{v} \cdot \vec{\omega} \, d^3x \label{eq:helicity_rotation}
\end{equation}

Helicity is a conserved quantity in ideal (invisible, incompressible) fluids, which encodes the connection and rotation of vortex lines. The rotation frequency $\Omega_k$ becomes a topologically meaningful indicator of the identity and dynamic state of the node.

Higher $\Omega_k$ values indicate more rotational energy and deeper pressure wells, leading to transient delays that resemble gravitational redshift, but without spacetime curvature.

Each particle is a topological vortex knot:
\begin{itemize}
    \item Charge $\leftrightarrow$ rotation or chirality of the knot
    \item Mass $\leftrightarrow$ integrated vorticity energy
    \item Spin $\leftrightarrow$ knot helix:
\end{itemize}
Stability $\leftrightarrow$ knot type (Hopf connections, Trefoil, etc.) and energy minimization in the vortex core

This model:

\begin{itemize}
    \item Attributes time modulation to conserved, intrinsic rotational energy,
    \item Requires no external frames of reference (absolute æther time is universal),
    \item Preserves temporal isotropy outside the vortex core,
    \item Provides a natural replacement for the spacetime curvature of GR. \end{itemize}

Therefore, this vortex-energetic time dilation principle provides a powerful alternative to relativistic time modulation by anchoring all temporal effects in rotational energetics and topological invariants.

In the next section, we will show how these ideas reproduce metric-like behavior for rotating observers, including a direct fluid-mechanical analogue to the Kerr metric of general relativity.
\section{Entropy and Quantum Effects in the Vortex Æther Model}

The Vortex Æther Model (VAM) provides a mechanistic basis for both thermodynamic and quantum mechanical phenomena, not through postulates about abstract state spaces, but via the dynamics of knots and vortices in a superfluid æther. Two central concepts—entropy and quantization—are derived in VAM from vorticity distribution and knot topology, respectively.

\subsection{Entropy as vorticity distribution}

In thermodynamics, entropy $S$ is a measure of the internal energy distribution or disorder. In VAM, entropy does not arise as a statistical phenomenon, but from spatial variations in vorticity. For a vortex configuration $V$ the entropy is given by:

\begin{equation}
    S \propto \int_V \|\vec{\omega}\|^2 \, dV,
\end{equation}

where $\vec{\omega} = \nabla \times \vec{v}$ is the local vorticity. This means:

\begin{itemize}
    \item \textbf{More rotation = more entropy}: Regions with strong swirl contribute to increased entropy.
    \item \textbf{Thermodynamic behavior arises from vortex expansion}: With the addition of energy (heat), the vortex boundary expands, the swirl decreases and $S$ increases—analogy with gas expansion.
\end{itemize}

This interpretation connects Clausius' heat theory with æther mechanics: heat is equivalent to increased swirl spreading.

\subsection{Quantum behavior from knotted vortex structures}

Quantum phenomena such as discrete energy levels, spin, and wave-particle duality originate in VAM from topologically conserved vortex knots:

\begin{itemize}
    \item \textbf{Circulation quantization:}
    \begin{equation}
        \Gamma = \oint \vec{v} \cdot d\vec{l} = n \cdot \kappa,
    \end{equation}
    where $\kappa = h/m$ and $n \in \mathbb{Z}$ is the winding number.
    \item \textbf{Integers arise from knot topology:} The helical structure of a vortex knot (such as a trefoil) provides discrete states with certain linking numbers $L_k$.
    \item \textbf{Helicity as a spin analogue:}
    \begin{equation}
        H = \int \vec{v} \cdot \vec{\omega} \, dV,
    \end{equation}
    where $H$ is invariant under ideal flow, just as spin is conserved in quantum mechanics.
\end{itemize}

\subsection{VAM interpretation of quantization and duality}

Instead of abstract Hilbert spaces, VAM considers a particle as a stable node in the æther field. This vortex configuration has:

\begin{itemize}
    \item A \textbf{core} (nodal body) with quantum jumps (resonances).
    \item An \textbf{outer field} that acts as a wave (like the Schrödinger wave).

    \item A \textbf{helicity} that behaves as internal degrees of freedom (e.g. spin).
\end{itemize}

The wave-particle dualism thus arises from the fact that knots are both localized (core) and spread out (field).

\subsection{Summary}

VAM thus provides a coherent, fluid-mechanical origin for both:

\begin{enumerate}
    \item \textbf{Thermodynamics:} Entropy arises from swirl distribution.
    \item \textbf{Quantum mechanics:} Quantization and duality are emergent properties of knotted vortex topologies.
\end{enumerate}

This approach shows that quantum and thermodynamic phenomena are not fundamentally different, but arise from the same vortex mechanism at different scales.

\section*{Entropy as a Vorticity-Weighted Invariant}\label{sec:entropy_vorticity}

In the Vortex Æther Model (VAM), we reinterpret classical entropy as a conserved scalar related to the internal vorticity structure of knotted field regions. The classical thermodynamic differential form:
\begin{equation}
    dS = \frac{\delta Q}{T},
\end{equation}
acquires a new form when heat exchange is replaced by rotational stress input into vortex knots:
\begin{equation}
    dS = \frac{\delta \Pi_\text{rot}}{\mathcal{T}_\omega},
\end{equation}
where:
\begin{itemize}
    \item $\delta \Pi_\text{rot}$ is the differential rotational energy input to the vortex core,
    \item $\mathcal{T}_\omega$ is the effective swirl-defined temperature field,
    \item $\omega = \nabla \times \vec{v}$ is the local vorticity.
\end{itemize}
This connects thermodynamic irreversibility directly to vorticity injection and local time dilation.

\subsection*{VAM Pressure Gradients and Entropy Flow}

In VAM, pressure gradients are induced by angular momentum conservation in the æther. The classical Euler equation for incompressible inviscid flow:
\begin{equation}
    \nabla P = -\rho_\text{\ae} (\vec{v} \cdot \nabla) \vec{v},
\end{equation}
is used to express entropy production through vorticity current divergence:
\begin{equation}
    \frac{dS}{dt} = \int_V \frac{\nabla \cdot \vec{J}_\text{vortex}}{T_\omega} \, dV,
\end{equation}
where $\vec{J}_\text{vortex}$ is the swirl energy flux density. This forms the entropy production analogue of Fourier's heat conduction law within the vortex medium.

\subsection*{Thermal Expansion of Vortex Knots}

Inspired by Clausius' treatment of thermal expansion, we define a vorticity-based expansion law for knotted vortex structures:
\begin{equation}
    \Delta V_\text{knot} = \alpha_\omega V_0 \Delta T_\omega,
\end{equation}
with:
\begin{equation}
    \alpha_\omega = \frac{1}{r_c} \frac{d r_k}{d T_\omega} \sim \frac{C_e^2}{r_c k_B T_\omega},
\end{equation}
where $r_k$ is the effective knot radius, $r_c$ is the core radius, $C_e$ the core swirl velocity, and $k_B$ the Boltzmann constant. Knot inflation in VAM thus follows from ætheric heating.

\subsection*{Clausius Inequality and Helicity Dissipation}

The Clausius inequality:
\begin{equation}
    \oint \frac{\delta Q}{T} \leq 0,
\end{equation}
is reinterpreted in VAM as a constraint on helicity-induced vorticity flow:
\begin{equation}
    \oint \frac{\vec{v} \cdot d\vec{\omega}}{\mathcal{T}_\omega} \leq 0,
\end{equation}
which implies that net swirl energy circulation around closed loops is dissipative unless compensated by external ætheric drive. This underpins the irreversibility of vortex-knot interactions.

\subsection*{Carnot Efficiency in Swirl Fields}

Classical Carnot engine efficiency:
\begin{equation}
    \eta = 1 - \frac{T_C}{T_H},
\end{equation}
can be reformulated in VAM via vorticity amplitudes:
\begin{equation}
    \eta_\text{VAM} = 1 - \frac{\Omega_C^2}{\Omega_H^2},
\end{equation}
where $\Omega_H$ and $\Omega_C$ are internal angular velocities of vortex knots in high and low swirl zones. This formulation links macroscopic energy conversion directly to microscopic vorticity gradients.


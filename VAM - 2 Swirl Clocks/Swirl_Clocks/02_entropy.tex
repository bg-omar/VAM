\section{Entropy and Quantum Effects in the Vortex Æther Model}

The Vortex Æther Model (VAM) provides a mechanistic basis for both thermodynamic and quantum mechanical phenomena, not through postulates about abstract state spaces, but via the dynamics of knots and vortices in a superfluid æther. Two central concepts—entropy and quantization—are derived in VAM from vorticity distribution and knot topology, respectively.

\subsection{Entropy as vorticity distribution}

In thermodynamics, entropy $S$ is a measure of the internal energy distribution or disorder. In VAM, entropy does not arise as a statistical phenomenon, but from spatial variations in vorticity. For a vortex configuration $V$ the entropy is given by:

\begin{equation}
    S \propto \int_V \|\vec{\omega}\|^2 \, dV,
\end{equation}

where $\vec{\omega} = \nabla \times \vec{v}$ is the local vorticity. This means:

\begin{itemize}
    \item \textbf{More rotation = more entropy}: Regions with strong swirl contribute to increased entropy.
    \item \textbf{Thermodynamic behavior arises from vortex expansion}: With the addition of energy (heat), the vortex boundary expands, the swirl decreases and $S$ increases—analogy with gas expansion.
\end{itemize}

This interpretation connects Clausius' heat theory with æther mechanics: heat is equivalent to increased swirl spreading.

\subsection{Quantum behavior from knotted vortex structures}

Quantum phenomena such as discrete energy levels, spin, and wave-particle duality originate in VAM from topologically conserved vortex knots:

\begin{itemize}
    \item \textbf{Circulation quantization:}
    \begin{equation}
        \Gamma = \oint \vec{v} \cdot d\vec{l} = n \cdot \kappa,
    \end{equation}
    where $\kappa = h/m$ and $n \in \mathbb{Z}$ is the winding number.
    \item \textbf{Integers arise from knot topology:} The helical structure of a vortex knot (such as a trefoil) provides discrete states with certain linking numbers $L_k$.
    \item \textbf{Helicity as a spin analogue:}
    \begin{equation}
        H = \int \vec{v} \cdot \vec{\omega} \, dV,
    \end{equation}
    where $H$ is invariant under ideal flow, just as spin is conserved in quantum mechanics.
\end{itemize}

\subsection{VAM interpretation of quantization and duality}

Instead of abstract Hilbert spaces, VAM considers a particle as a stable node in the æther field. This vortex configuration has:

\begin{itemize}
    \item A \textbf{core} (nodal body) with quantum jumps (resonances).
    \item An \textbf{outer field} that acts as a wave (like the Schrödinger wave).

    \item A \textbf{helicity} that behaves as internal degrees of freedom (e.g. spin).
\end{itemize}

The wave-particle dualism thus arises from the fact that knots are both localized (core) and spread out (field).

\subsection{Summary}

VAM thus provides a coherent, fluid-mechanical origin for both:

\begin{enumerate}
    \item \textbf{Thermodynamics:} Entropy arises from swirl distribution.
    \item \textbf{Quantum mechanics:} Quantization and duality are emergent properties of knotted vortex topologies.
\end{enumerate}

This approach shows that quantum and thermodynamic phenomena are not fundamentally different, but arise from the same vortex mechanism at different scales.
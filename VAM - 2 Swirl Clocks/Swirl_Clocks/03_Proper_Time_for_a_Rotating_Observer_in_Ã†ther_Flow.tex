\section{Proper time for a rotating observer in æther flow}

Having established time dilation in the Vortex Æther Model (VAM) by means of pressure, angular velocity, and rotational energy, we now extend our formalism to rotating observers. This section shows that fluid dynamical time modulation in VAM can reproduce expressions that are structurally similar to those derived from general relativity (GR), in particular in axially symmetric rotating spacetimes such as the Kerr geometry. However, VAM achieves this without invoking spacetime curvature. Instead, time modulation is determined entirely by kinetic variables in the æther field.

\subsection{GR-proper time in rotating frames}

In general relativity, the proper time \(d\tau\) for an observer with angular velocity \(\Omega_\text{eff}\) in a stationary, axially symmetric spacetime is given by:

\begin{equation}
 \left( \frac{d\tau}{dt} \right)^2_\text{GR} = -\left[ g_{tt} + 2g_{t\varphi} \Omega_\text{eff} + g_{\varphi\varphi} \Omega_\text{eff}^2 \right]
 \label{eq:GR_proper_time}
\end{equation}

where \(g_{\mu\nu}\) are components of the spacetime metric (e.g. in Boyer-Lindquist coordinates for Kerr spacetime). This formulation takes into account both gravitational redshift and rotational effects (frame-dragging).

\subsection{Æther-based analogy: Velocity-derived time modulation}

In VAM, spacetime is not curved. Instead, observers are in a dynamically structured æther whose local flow velocities determine the time dilation. Let the radial and tangential components of the æther velocity be:

\begin{itemize}
 \item \(v_r\): radial velocity,
 \item \(v_\varphi = r\Omega_k\): tangential velocity due to local vortex rotation,
 \item \(\Omega_k = \frac{\kappa}{2\pi r^2}\): local angular velocity (with \(\kappa\) as circulation).
\end{itemize}

We postulate a correspondence between GR metric components and other velocity terms:

\begin{equation}
 \begin{aligned}
  g_{tt} &\rightarrow -\left(1 - \frac{v_r^2}{c^2}\right), \\
  g_{t\varphi} &\rightarrow -\frac{v_r v_\varphi}{c^2}, \\
  g_{\varphi\varphi} &\rightarrow -\frac{v_\varphi^2}{c^2 r^2}
 \end{aligned}
 \label{eq:VAM_metric_terms}
\end{equation}

Substituting this into the GR expression for the appropriate tense, we obtain the VAM-based analogue:

\begin{equation}
 \left( \frac{d\tau}{dt} \right)^2_\text{\ae} = 1 - \frac{v_r^2}{c^2} - \frac{2v_r v_\varphi}{c^2} - \frac{v_\varphi^2}{c^2}
 \label{eq:VAM_proper_time}
\end{equation}

Combining the terms:

\begin{equation}
 \left( \frac{d\tau}{dt} \right)^2_\text{\ae} = 1 - \frac{1}{c^2}(v_r + v_\varphi)^2
 \label{eq:VAM_proper_time_combined}
\end{equation}

This formulation reproduces gravitational and frame-dragging time effects purely from æther dynamics: $\langle \omega^2 \rangle$ plays the role of gravitational redshift and circulation $\kappa$ encodes rotational drag. This approach is consistent with recent fluid dynamic interpretations of gravity and time \cite{barcelo2011analogue}, \cite{fedi2017gravity}.
This model currently assumes irrotational flow outside nodes and neglects viscosity, turbulence and quantum compressibility. Future extensions may include quantized circulation spectra or boundary effects in confined æther systems.

\begin{equation}
 \boxed{\left( \frac{d\tau}{dt} \right)^2_\text{\ae} = 1 - \frac{1}{c^2}(v_r + r\Omega_k)^2}
 \label{eq:VAM_proper_time_final}
\end{equation}

\subsection{Physical interpretation and model consistency}

This result in the box mirrors the GR expression for rotating observers, but stems strictly from classical fluid dynamics. It shows that as the local æther velocity approaches the speed of light – due to radial inflow or rotational motion – the proper time slows down. This implies the existence of "time wells" where the kinetic energy density dominates.

Key observations:

\begin{itemize}
 \item In the absence of radial flow (\(v_r = 0\)), time delay arises entirely from vortex rotation.
 \item When both \(v_r\) and \(\Omega_k\) are present, the cumulative velocity decreases the local time velocity.
 \item This expression agrees with the energetic model of Section II if we interpret \(v_r + r\Omega_k\) as a contribution to the local energy density.
\end{itemize}

In the VAM framework, the structure of the observer's proper time thus arises from ætheric flow fields. This confirms that GR-like temporal behavior can arise in a flat, Euclidean 3D space with absolute time, entirely determined by structured vorticity and circulation.

In the next section we investigate how VAM extends this correspondence to gravitational potentials and frame-dragging effects via circulation and vorticity intensity, thus providing an analogy for the Kerr time redshift formula.
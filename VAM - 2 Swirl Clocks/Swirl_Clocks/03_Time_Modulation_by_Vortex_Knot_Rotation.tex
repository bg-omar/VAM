\section{Time Modulation by Rotation of Vortex Nodes}

Building on the discussion of time dilation via pressure and Bernoulli dynamics in the previous section, we now focus on the intrinsic rotation of topological vortex nodes. In the Vortex Æther Model (VAM), particles are modeled as stable, topologically conserved vortex nodes embedded in an incompressible, inviscid superfluid medium. Each node possesses a characteristic internal angular frequency $\Omega_k$, and this internal motion induces local time modulation with respect to the absolute time $\mathcal{N}$ of the æther.

Instead of warping spacetime, we propose that internal rotational energy and helicity conservation cause temporal delays analogous to gravitational redshift. In this section, these ideas are developed using heuristic and energetic arguments consistent with the hierarchy introduced in Section I.

\subsection{Heuristic and Energetic Derivation}

We start by proposing a rotationally induced time dilation formula based on the internal angular frequency of the vortex node:

\begin{equation}
    \frac{d\tau}{d\mathcal{N}} = \left(1 + \beta \Omega_k^2 \right)^{-1}
    \quad \text{(Chronos-Time slowdown due to internal vortex rotation)}
    \label{eq:chronos_swirl_slowdown}
\end{equation}

where:

\begin{itemize}
    \item $\tau$ is the local Chronos-Time (proper time experienced by the vortex structure),
    \item $\mathcal{N}$ is the absolute Aithēr-Time (universal causal background),
    \item $\Omega_k$ is the mean core angular frequency,
    \item $\beta$ is a coupling coefficient with units $[\beta] = \mathrm{s}^2$.
\end{itemize}

For small angular velocities we obtain a first-order expansion:

\begin{equation}
    \frac{d\tau}{d\mathcal{N}} \approx 1 - \beta \Omega_k^2 + \mathcal{O}(\Omega_k^4)
\end{equation}

This mirrors the low-velocity expansion of the Lorentz factor in special relativity:
\begin{equation}
    \frac{t_\text{moving}}{t_\text{rest}} \approx 1 - \frac{v^2}{2c^2}
\end{equation}

We observe that internal rotation in VAM induces time dilation just as relative motion does in SR—yet from internal dynamics, not frame-relative velocity.

To support this with physical grounding, we connect time dilation to rotational energy. Suppose the vortex node has an effective moment of inertia $I$. The rotational energy becomes:

\begin{equation}
    E_\text{rot} = \frac{1}{2} I \Omega_k^2
\end{equation}

This leads to the energetic expression:
\begin{equation}
    \frac{d\tau}{d\mathcal{N}} = \left(1 + \beta E_\text{rot} \right)^{-1}
    = \left(1 + \frac{1}{2} \beta I \Omega_k^2 \right)^{-1}
\end{equation}

This equation parallels the pressure-induced time modulation derived from Bernoulli dynamics earlier in the paper and supports the concept of rotational time wells induced by internal energy storage.

\begin{tcolorbox}[colback=gray!5, colframe=black!70, sharp corners=southwest, title=Temporal Mapping for Vortex Nodes]
In the Vortex Æther Model:
\begin{itemize}
    \item \( \mathcal{N} \) — \textbf{Aithēr-Time:} Universal causal flow (background time),
    \item \( \tau \) — \textbf{Chronos-Time:} Local inertial time along vortex evolution,
    \item \( S(t) \) — \textbf{Swirl Clock:} Phase memory due to internal angular rotation.
\end{itemize}
Equation~\eqref{eq:chronos_swirl_slowdown} captures how increasing swirl leads to slower proper time relative to the ætheric background.
\end{tcolorbox}

\subsection{Topological and Physical Justification}

Topological vortex nodes are characterized not only by rotation, but also by helicity:
\begin{equation}
    H = \int \vec{v} \cdot \vec{\omega} \, d^3x
\end{equation}

Helicity is a conserved quantity in ideal fluids and encodes the topological linkage and twist of vortex lines. Thus, the rotation frequency $\Omega_k$ becomes a signature of the knot’s identity and energy state.

Higher $\Omega_k$ values produce stronger swirl wells and deeper pressure minima, resulting in longer internal durations per unit $\mathcal{N}$. This time dilation is interpreted as a reduction in Chronos-Time, not as a change in spacetime geometry.

Each particle is a topological vortex knot, where:
\begin{itemize}
    \item \textbf{Charge} maps to chirality and rotational direction,
    \item \textbf{Mass} maps to total vortex energy (and inertia),
    \item \textbf{Spin} maps to knot helicity and winding structure.
\end{itemize}

Knot type (e.g. Hopf, Trefoil) determines its stability and energetic minimum.

This model:

\begin{itemize}
    \item Attributes temporal modulation to conserved rotational energy,
    \item Requires no relativistic reference frames,
    \item Embeds all time shifts within the æther’s causal substrate $\mathcal{N}$,
    \item Provides a direct fluid-mechanical analogue to gravitational redshift.
\end{itemize}

\textbf{In summary:} Vortex-induced time dilation in VAM is governed by the equation
\[
    \frac{d\tau}{d\mathcal{N}} = \left(1 + \beta \Omega_k^2 \right)^{-1}
\]
showing that Chronos-Time slows as internal vortex angular velocity increases — a purely mechanical, topologically-grounded origin of time dilation, replacing the abstract spacetime curvature of general relativity.

\section{VAM Vorticity Scattering Framework (inspired by elastic theory)}

\subsection{Governing equations of VAM Vorticity dynamics}

\subsubsection*{Vorticity transport equation (linearized form)}

In the Vortex Æther Model (VAM), the dynamics of the vorticity field \(\vec{\omega} = \nabla \times \vec{v}\) is governed by the Euler equation and the associated vorticity form:

\[
    \frac{\partial \omega_i}{\partial t} + v_j \partial_j \omega_i = \omega_j \partial_j v_i
\]

This nonlinear structure implies vortex deformation by stretching and advection. For small perturbations \(\delta\omega\) near a background vortex node field \(\omega^{(0)}\) linearization yields:

\[
    \frac{\partial (\delta \omega_i)}{\partial t} + v_j^{(0)} \partial_j (\delta \omega_i) \approx \omega_j^{(0)} \partial_j (\delta v_i)
\]

Define the linear response operator of VAM \(\mathcal{L}_{ij}\):

\[
    \mathcal{L}_{ij} \, \delta v_j(\vec{r}) = \delta F_i^\text{vortex}(\vec{r})
\]

\subsubsection*{Green Tensor Vorticity Equation}

\[
    \mathcal{L}_{ij} \, \mathcal{G}_{jk}(\vec{r}, \vec{r}') = -\delta_{ik} \, \delta(\vec{r} - \vec{r}')
\]

The induced velocity field \(v_i\) of a source vortex force \(F_k(\vec{r}')\) is then:

\[
    v_i(\vec{r}) = \int \mathcal{G}_{ik}(\vec{r}, \vec{r}') \, F_k^\text{vortex}(\vec{r}') \, d^3 r'
\]

\subsection{Vortex filament interaction}
Interactions arise from exchange of vortex force or Reconnections between vortex filaments:
\begin{itemize}
    \item Attractive when filaments reinforce the circulation (parallel)
    \item Repulsive when filaments cancel each other out (antiparallel)
    \item Interaction strength:
\end{itemize}
\begin{equation}
    \vec{F}_\text{int} = \beta \cdot \kappa_1 \kappa_2 \cdot \frac{\vec{r}_{12} \times (\vec{v}_1 - \vec{v}_2)}{|\vec{r}_{12}|^3}\label{eq:interaction_strength}
\end{equation}
Where \(\kappa_i\) are the circulations of filaments and \(\vec{r}_{12}\) is the vector between them.

\subsection{Thermodynamic \& quantum behavior of vorticity fluctuations}
\begin{itemize}
    \item Entropy \(\leftrightarrow\) volume of vortex expansion or knot deformation
    \item Quantum transitions \(\leftrightarrow\) topological reconnection events
    \item Zero-point motion \(\leftrightarrow\) background quantum turbulence of the Æther:
\end{itemize}

\subsubsection*{Quantum vorticity background}
\begin{equation}
    \langle \omega^2 \rangle \sim \frac{\hbar}{\rho_\text{æ} \xi^4}\label{eq:quantum_vorticity_background}
\end{equation}
Where \(\xi\) is the coherence length between vortex filaments.

\subsection{VAM scattering theory for vortex nodes}

\subsubsection*{Born approximation for vortex perturbations}

Suppose that an incident vortex potential \(\Phi^{(0)}(\vec{r})\) encounters a vortex node at \(\vec{r}_k\). The scattered vorticity field becomes:

\[
    \Phi(\vec{r}) = \Phi^{(0)}(\vec{r}) + \int \mathcal{G}_{ij}(\vec{r}, \vec{r}') \, \delta \mathcal{V}_{jk}(\vec{r}') \, v_k^{(0)}(\vec{r}') \, d^3r'
\]

Here \(\delta \mathcal{V}_{jk}\) represents a vorticity polarization tensor associated with the node – a VAM analogue of elastic moduli perturbation.

\subsection{Æther stress tensor and energy flux}

\subsubsection*{VAM stress tensor}

\[
    \mathcal{T}_{ij} = \rho_\text{\ae} \, v_i v_j - \frac{1}{2} \delta_{ij} \rho_\text{\ae} v^2
\]

\subsubsection*{Æther Vorticity Force Density}

\[
    f_i^\text{vortex} = \partial_j \mathcal{T}_{ij}
\]

\subsubsection*{Vorticity Energy Flux}

\[
    \vec{S}_\omega = - \mathcal{T} \cdot \vec{v}
\]

This vector captures the energy transfer via vortex node interactions and defines Scattering of "cross sections" via the divergence \(\nabla \cdot \vec{S}_\omega\).

\subsection{Time dilation and nodal scattering}

\subsubsection*{Time dilation due to nodal rotation}

Let the incident vortex field cause a local time delay due to the rotational energy of a node:

\[
    \frac{t_\text{local}}{t_{\infty}} = \left(1 + \frac{1}{2} \beta I \Omega_k^2 \right)^{-1}
\]

In the Born approximation, the change in proper time near a node under external vortex flow is:

\subsubsection*{Scattered correction due to external field}

\begin{gather*}
    \delta \left( \frac{t_\text{local}}{t_{\infty}} \right) \approx - \frac{1}{2} \beta I \Omega_k \, \delta \Omega_k\\
    \delta \Omega_k \sim \int \chi(\vec{r}_k - \vec{r}') \cdot \vec{\omega}^{(0)}(\vec{r}') \, d^3r'\\
\end{gather*}


Here \(\chi\) is the topological eddy sensitivity core.
\subsection{Summary of VAM-inspired scattering structures}

\begin{table}[htbp]
    \centering
    \begin{tabular}{lll}
        \toprule
    \textbf{Concept} & \textbf{Elastic theory} & \textbf{VAM analogue} \\
    \midrule
    Medium property & \( c_{ijkl} \) & \( \rho_\text{\ae},\, \Omega_k,\, \kappa \) \\
    Wavefield & \( u_i \) (displacement) & \( v_i \) (æther velocity) \\
    Source & \( f_i \) (body force) & \( F_i^\text{vortex} \) (vorticity forcing) \\
    Green function & \( G_{ij}(\vec{r}, \vec{r}') \) & \( \mathcal{G}_{ij}(\vec{r}, \vec{r}') \) \\
    Stress tensor & \( \tau_{ij} \) & \( \mathcal{T}_{ij} \) \\
    Energy flux & \( J_{P,i} = -\tau_{ij} \dot{u}_j \) & \( S_{\omega,i} = -\mathcal{T}_{ij} v_j \) \\
    Time dilation mechanism & \( g_{\mu\nu} \) (GR metric) & \( \Omega_k,\, \kappa,\, \langle \omega^2 \rangle \) \\
    \bottomrule
    \end{tabular}
    \caption{Conceptual correspondence between classical elasticity and Vortex Æther Model (VAM).}
    \label{tab:elastic-vam-analogy}
\end{table}

This scattering framework generalizes classical elastic analogs to a topologically and energetically motivated Ætheric formalism. It allows the calculation of field modifications, time dilation effects, and energy flux due to stable, interacting vortices in the Vortex Æther Model (VAM).
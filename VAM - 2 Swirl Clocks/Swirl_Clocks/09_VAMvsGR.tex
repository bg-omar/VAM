\section{VAM versus GR: Corresponding Predictions}
\label{sec:vam-versus-gr-corresponding-predictions}

Although the Vortex Æther Model uses a fundamentally different ontology than the curvature-based structure of general relativity, it leads in many cases to similar expressions for physically observable phenomena. In this section we show how VAM reproduces the classical predictions of GR — but with alternative underlying mechanisms.

\subsection*{VAM Orbital Precession (GR Equivalent)}

In general relativity, perihelion precession of a rotating body is attributed to spacetime curvature. In the Vortex Æther Model (VAM), this effect is replaced by the cumulative influence of a vortex-induced vorticity field within a rotating Æther medium.

The equivalent VAM formulation mirrors the GR forecast, but is based on vorticity-induced pressure gradients and circulation:

\begin{equation}
    \Delta\phi_\text{VAM} =
    \frac{6\pi G M}{a(1 - e^2) c^2}
\end{equation}

whereby:
\begin{itemize}
    \item \( M \): mass of the central vortex attractor,
    \item \( a \): semi-major axis of the orbit,
    \item \( e \): eccentricity of the orbit,
    \item \( G \): gravitational constant (reduced from VAM coupling),
    \item \( c \): speed of light.
\end{itemize}
Although formally identical to the GR expression, in VAM this arises from the variation in local circulation and angular momentum flux within the surrounding Æther, which modulates the effective potential and gives rise to precessional motion.

\subsection*{VAM Light Deflection by Ætheric Circulation}

In general relativity, light deflection by massive bodies is caused by spacetime curvature. In the Vortex Æther Model, light (considered as a perturbation or mode in the Æther) deflects due to circulation-induced pressure gradients and anisotropic refractive index fields near rotating vortex attractors.

The equivalent VAM deflection angle for a light beam passing a spherical vortex mass is given by:

\begin{equation}
    \delta_\text{VAM} =
    \frac{4 G M}{R c^2}
\end{equation}

where:
\begin{itemize}
    \item \( M \): effective mass of the rotating vortex node,
    \item \( R \): closest approach (impact parameter),
    \item \( G \): vortex coupling constant (restoration of Newtonian \( G \) under macroscopic limits),
    \item \( c \): speed of light.
\end{itemize}

In VAM this is due to the interaction between the propagation velocity of light and the surrounding rotational field. The light wavefront is locally compressed or refracted by tangential æther flow gradients, resulting in an observable angular deflection.

\subsection*{Overview of the observable correspondence between VAM and GR}

\begin{table}[ht]
    \centering
    \caption{Comparison of GR and VAM for gravity-related observables}
    \label{tab:VAM-GR}
    \begin{tabular}{|l|c|l|}
        \hline
        \textbf{Observable} & \textbf{Theory} & \textbf{Expression} \\
        \hline
        Time dilation & GR & $ \frac{d\tau}{dt} = \sqrt{1 - \frac{2GM}{rc^2}} $ \\
        & VAM & $ \frac{d\tau}{dt} = \sqrt{1 - \frac{\Omega^2 r^2}{c^2}} $ \\
        \hline
        Redshift & GR & $ z = \left(1 - \frac{2GM}{rc^2} \right)^{-1/2} - 1 $ \\
        & VAM & $ z = \left(1 - \frac{v_\phi^2}{c^2} \right)^{-1/2} - 1 $ \\
        \hline
        Frame drag & GR & $ \omega_\text{LT} = \frac{2GJ}{c^2 r^3} $ \\
        & VAM & $ \omega_\text{drag} = \frac{2G \mu I \Omega}{c^2 r^3} $ \\
        \hline
        Precession & GR/VAM & $ \Delta\phi = \frac{6\pi GM}{a(1 - e^2)c^2} $ \\
        \hline
        Light diffraction & GR/VAM & $ \delta = \frac{4GM}{Rc^2} $ \\
        \hline
        Gravitational potential & GR & $ \Phi = -\frac{GM}{r} $ \\
        & VAM & $ \Phi = -\frac{1}{2} \vec{\omega} \cdot \vec{v} $ \\
        \hline
        Gravitational constant & VAM & $ G = \frac{C_e c^5 t_p^2}{2 F_{\max} r_c^2} $ \\
        \hline
    \end{tabular}
\end{table}


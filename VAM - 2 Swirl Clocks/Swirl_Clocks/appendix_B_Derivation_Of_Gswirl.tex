%! Author = Omar Iskandarani
%! Date = 2/20/2025

\section{Appendix B: Derivation of Fundamental Constants from Vortex Dynamics}


\paragraph*{Introduction}
This document aims to provide a comprehensive and rigorous derivation of the fine-structure constant $\alpha$ grounded in classical physical principles.
The derivation integrates the electron's classical radius and its Compton angular frequency to elucidate the relationship between these fundamental constants and the tangential velocity $C_e$.
This velocity arises naturally when the electron is conceptualized as a vortex-like structure, offering a geometrically intuitive interpretation of the fine-structure constant.
By extending classical formulations, the discussion highlights the profound interplay between quantum phenomena and vortex dynamics.


\paragraph*{The Fine-Structure Constant:}
$\alpha$ serves as a dimensionless measure of electromagnetic interaction strength~\cite{maxwell1861}.
It is mathematically expressed as:

\begin{equation*}
    \alpha = \frac{e^2}{4\pi \varepsilon_0 \hbar c},
\end{equation*}
where $e$ is the elementary charge, $\varepsilon_0$ is the vacuum permittivity, $\hbar$ is the reduced Planck constant, and $c$ is the speed of light \cite{dirac1930quantum}.

\subsection*{Relevant Definitions and Formulas}
\paragraph*{The Classical Electron Radius}
$R_e$ represents the scale at which classical electrostatic energy equals the electron's rest energy. It is defined as:
\begin{equation*}
    R_e = \frac{e^2}{4\pi \varepsilon_0 m_e c^2},
\end{equation*}
where $m_e$ is the electron mass~\cite{helmholtz1858}.

\paragraph*{The Compton Angular Frequency}
$\omega_c$ corresponds to the intrinsic rotational frequency of the electron when treated as a quantum oscillator:
\begin{equation*}
    \omega_c = \frac{m_e c^2}{\hbar}.
\end{equation*}
This frequency is pivotal in characterizing the electron's interaction with electromagnetic waves~\cite{kelvin1867}.

\subsubsection*{Half the Classical Electron Radius}
We assume an electron to be a vortex, its particle form is a folded vortex tube shaped as a torus, hence both the Ring radius R and Core radius r are defined as half the classical electron radius  $r_c$ :
\begin{equation*}
    r_c = \frac{1}{2} R_e.
\end{equation*}
This simplification aligns with established models of vortex structures in fluid dynamics~\cite{kleckner2013}.

\subsubsection*{Definition of Tangential Velocity $C_e$}
To conceptualize the electron as a vortex ring, we associate its tangential velocity $C_e$ with its rotational dynamics:
\begin{equation*}
    C_e = \omega_c r_c.
\end{equation*}
Substituting $\omega_c = \frac{m_e c^2}{\hbar}$ and $r_c = \frac{1}{2} R_e$, we find:
\begin{equation*}
    C_e = \left( \frac{m_e c^2}{\hbar} \right) \left( \frac{1}{2} \frac{e^2}{4\pi \varepsilon_0 m_e c^2} \right).
\end{equation*}
Simplifying by canceling $m_e c^2$ yields:
\begin{equation*}
    C_e = \frac{1}{2} \frac{e^2}{4\pi \varepsilon_0 \hbar}.\label{eq:C_e-from-compton}
\end{equation*}
This result directly links $C_e$ to the fine-structure constant \cite{vinen2024}.

\subsubsection*{Physical Interpretation}
The tangential velocity $C_e$ embodies the rotational speed at the electron's vortex boundary. Its value, approximately:
\begin{equation*}
    C_e \approx 1.0938 \times 10^6 \ \text{m/s},
\end{equation*}
is consistent with the experimentally observed fine-structure constant $\alpha \approx 1/137$~\cite{ricca1998}.

\subsubsection*{Conclusion}
The derivation presented elucidates the fine-structure constant $\alpha$ using fundamental classical principles, including the electron's classical radius, Compton angular frequency, and vortex tangential velocity. The result:
\begin{equation*}
    \alpha = \frac{2 C_e}{c},
\end{equation*}
reveals a profound geometric and physical connection underpinning electromagnetic interactions.
This perspective enriches our understanding of $\alpha$ and highlights the deep ties between classical mechanics and quantum electrodynamics.


\subsection{Derivation of the VAM Gravitational Constant \(  G_\text{swirl} \)}

\subsubsection*{Introduction}
In the Vortex Æther Model (VAM), gravitational interactions arise from vorticity dynamics in a superfluidic Æther medium rather than from mass-induced spacetime curvature. This leads to a modification of the gravitational constant, which we denote as \(  G_\text{swirl} \), as a function of vortex field parameters.

To derive \(  G_\text{swirl} \), we assume that gravitational effects emerge from vortex-induced energy density rather than mass-energy tensor formulations. The fundamental relation between vorticity, circulation, and energy density will be used to establish an equivalent gravitational constant in VAM \cite{onsager_superfluid,barcelo_superfluid,moffatt_helicity}.

\subsubsection*{Vortex-Induced Energy Density}
In classical fluid dynamics, vorticity is defined as the curl of velocity:
\begin{equation*}
    \vec{\omega} = \nabla \times \vec{v}
\end{equation*}
where \( \vec{\omega} \) represents the vorticity field.

The corresponding vorticity energy density is given by:
\begin{equation*}
    U_\text{vortex} = \frac{1}{2} \rho_\text{\ae} |\vec{\omega}|^2
\end{equation*}
where:
\begin{itemize}
    \item \( \rho_\text{\ae} \) is the density of the Æther medium,
    \item \( |\vec{\omega}|^2 \) is the squared vorticity magnitude.
\end{itemize}

Since vorticity magnitude scales with core tangential velocity as:
\begin{equation*}
    |\vec{\omega}|^2 \sim \frac{C_e^2}{r_c^2}
\end{equation*}
we obtain an approximate energy density for the vortex field:
\begin{equation*}
    U_\text{vortex} \approx \frac{C_e^2}{2 r_c^2}.
\end{equation*}

\subsubsection*{Gravitational Constant from Vorticity}
In standard General Relativity (GR), Newton's gravitational constant \( G \) appears in:
\begin{equation*}
    F = \frac{GMm}{r^2}.
\end{equation*}

In VAM, we assume that the gravitational constant \(  G_\text{swirl} \) is defined in terms of \textbf{vorticity energy density} rather than mass-energy.

Since gravitational force scales with \textbf{energy density per unit mass}, we set:
\begin{equation*}
    G_\text{swirl} \sim \frac{U_\text{vortex} c^n}{F_{\max}},
\end{equation*}
where:
\begin{itemize}
    \item \( c^n \) represents relativistic corrections,
    \item \( F_{\max} \) is the maximum force in VAM, set to approximately \textbf{29 N} \cite{schiller_max_force}.
\end{itemize}

Substituting \( U_\text{vortex} \):
\begin{equation*}
    G_\text{swirl} \sim \frac{\left( \frac{C_e^2}{2 r_c^2} \right) c^n}{F_{\max}}.
\end{equation*}

The choice of \( n \) depends on whether we use \textbf{Planck length} (\( l_p^2 \)) or \textbf{Planck time} (\( t_p^2 \)).

\subsubsection*{Two Possible Forms of \(  G_\text{swirl} \)}
\subsubsection*{Form 1: Using Planck Length}
The Planck length is defined as:
\begin{equation*}
    l_p^2 = \frac{\hbar G}{c^3}.
\end{equation*}

Using \( c^3 l_p^2 \) as the relativistic correction factor, we obtain:
\begin{equation*}
    G_\text{swirl} = \frac{C_e c^3 l_p^2}{2 F_{\max} r_c^2}.
\end{equation*}

\subsubsection*{Form 2: Using Planck Time}
The Planck time is given by:
\begin{equation*}
    t_p^2 = \frac{\hbar G}{c^5}.
\end{equation*}

Using \( c^5 t_p^2 \) as the relativistic correction factor, we obtain:
\begin{equation*}
    G_\text{swirl} = \frac{C_e c^5 t_p^2}{2 F_{\max} r_c^2}.
\end{equation*}

\subsubsection*{Physical Interpretation of \(  G_\text{swirl} \)}
These formulations of the gravitational constant in VAM highlight a fundamental difference from GR:
\begin{itemize}
    \item Gravity is not driven by mass-energy, but by \textbf{vortex energy density} \cite{barcelo_superfluid}.
    \item \(  G_\text{swirl} \) scales with the core vortex velocity \( C_e \), linking gravity directly to vorticity \cite{moffatt_helicity}.
    \item The \textbf{maximum force \( F_{\max} \)} (≈29 N) acts as a natural cutoff, limiting the strength of gravitational interactions \cite{schiller_max_force}.
\end{itemize}

\subsubsection*{Conclusion}
We have derived two equivalent formulations of \(  G_\text{swirl} \) using \textbf{Planck scale physics} and \textbf{vortex energy density principles} in VAM. The final expressions:
\begin{equation*}
    G_\text{swirl} = \frac{C_e c^3 l_p^2}{2 F_{\max} r_c^2}
\end{equation*}
and
\begin{equation*}
    G_\text{swirl} = \frac{C_e c^5 t_p^2}{2 F_{\max} r_c^2}
\end{equation*}
demonstrate that gravitational interactions in VAM are governed by vorticity rather than mass-induced curvature.



\subsection{Vorticity-Based Reformulation of General Relativity Laws in a 3D Absolute Time Framework}

\subsubsection*{Vorticity as the Fundamental Gravitational Interaction}

General Relativity (GR) describes gravity as a result of \textbf{spacetime curvature}, governed by Einstein's field equations:

\begin{equation*}
    G_{\mu\nu} = \frac{8\pi G}{c^4} T_{\mu\nu}.
\end{equation*}

However, in the Vortex Æther Model (VAM), \textbf{gravity is not caused by curvature} but by \textbf{vorticity-induced pressure gradients in an inviscid Æther}. Instead of using a \textbf{4D metric tensor}, we define gravity as a 3D vorticity field \( \omega \), where mass acts as a localized vortex concentration.

\subsubsection*{Replacing Einstein's Equations with a 3D Vorticity Field Equation}

We replace the Einstein curvature equations with a \textbf{3D vorticity-Poisson equation}, where the gravitational potential \( \Phi_v \) is related to vorticity magnitude:

\begin{equation*}
    \nabla^2 \Phi_v = - \rho_\text{Æ} |\omega|^2.
\end{equation*}

Here:
- \( \Phi_v \) is the \textbf{vorticity-induced gravitational potential}.
- \( \rho_\text{Æ} \) is the \textbf{local Æther density}.
- \( |\omega|^2 = (\nabla \times v)^2 \) is the \textbf{vorticity magnitude}.

Instead of \textbf{mass-energy tensor components}, gravity is determined by the \textbf{local vorticity density}.

---

\subsubsection*{Motion in VAM: Replacing Geodesics with Vortex Streamlines}

In GR, test particles follow \textbf{geodesics in curved spacetime}. In VAM, particles follow \textbf{vortex streamlines}, governed by the \textbf{vorticity transport equation}:

\begin{equation*}
    \frac{D\omega}{Dt} = (\omega \cdot \nabla) v - (\nabla \cdot v) \omega.
\end{equation*}

This replaces \textbf{spacetime curvature} with \textbf{fluid-dynamic vorticity transport} \cite{lamb_hydrodynamics,feynman_superfluid}.

---

\subsubsection*{Frame-Dragging as a 3D Vortex Effect}

In General Relativity, frame-dragging is described using the \textbf{Kerr metric}, which predicts that spinning masses drag spacetime along with them. In VAM, this effect is caused by \textbf{vortex interactions in the Æther}.

We replace the Kerr metric with the \textbf{vorticity-induced rotational velocity field}:

\begin{equation*}
    \Omega_\text{vortex} = \frac{\Gamma}{2\pi r^2},
\end{equation*}

where:
- \( \Gamma = \oint v \cdot dl \) is the \textbf{circulation of the vortex}.
- \( r \) is the \textbf{radial distance from the vortex core}.

This ensures that frame-dragging emerges \textbf{naturally} from vorticity rather than requiring \textbf{spacetime warping}.

---

\subsubsection*{Replacing Gravitational Time Dilation with Vorticity Effects}

In GR, time dilation is caused by \textbf{spacetime curvature}, leading to:

\begin{equation*}
    dt_\text{GR} = dt \sqrt{1 - \frac{2GM}{rc^2}}.
\end{equation*}

In VAM, time dilation is caused by \textbf{vorticity-induced energy gradients}, leading to:

\begin{equation*}
    dt_\text{VAM} = \frac{dt}{\sqrt{1 - \frac{C_e^2}{c^2} e^{-r/r_c} - \frac{\Omega^2}{c^2} e^{-r/r_c}}},
\end{equation*}

where:
- \( C_e \) is the \textbf{vortex core tangential velocity}.
- \( \Omega \) is the \textbf{local vorticity angular velocity}.

This formula eliminates the need for \textbf{mass-based time dilation} and instead relies purely on \textbf{fluid dynamic principles}.

---

\subsubsection*{Summary: A 3D Vorticity-Based Alternative to General Relativity}

The Vortex Æther Model replaces the \textbf{4D spacetime formalism of General Relativity} with a \textbf{3D vorticity-driven description}:

\begin{itemize}
    \item \textbf{Gravity} is not caused by \textbf{curved spacetime} but by \textbf{vorticity-induced pressure gradients}.
    \item \textbf{Geodesic motion} is replaced by \textbf{vortex streamlines} in an inviscid Æther.
    \item \textbf{Frame-dragging} is explained through \textbf{circulation velocity in vorticity fields}, rather than through Kerr spacetime.
    \item \textbf{Time dilation} arises from \textbf{energy gradients in vortex structures}, not from mass-induced curvature.
\end{itemize}

%! Author = Omar Iskandarani
%! Title = Appendix: Derivation of Fundamental Constants from Vortex Dynamics
%! Date = 2/20/2025
%! Affiliation = Independent Researcher, Groningen, The Netherlands
%! License = CC-BY 4.0
%! ORCID = 0009-0006-1686-3961
%! DOI = 10.5281/zenodo.15692509

% === Metadata ===
%\newcommand{\appendixtitle}{\textbf{Appendix: Derivation of Fundamental Constants from Vortex Dynamics}}
%\newcommand{\paperdoi}{10.5281/zenodo.15692509}
%
%
%\ifdefined\standalonechapter\else
%% Standalone mode
%\documentclass[12pt]{article}
%% vamstyle.sty
\NeedsTeXFormat{LaTeX2e}
\ProvidesPackage{vamstyle}[2025/07/01 VAM unified style]

% === Constants ===
\newcommand{\hbarVal}{\ensuremath{1.054571817 \times 10^{-34}}} % J\cdot s
\newcommand{\meVal}{\ensuremath{9.10938356 \times 10^{-31}}} % kg
\newcommand{\cVal}{\ensuremath{2.99792458 \times 10^{8}}} % m/s
\newcommand{\alphaVal}{\ensuremath{1 / 137.035999084}} % unitless
\newcommand{\alphaGVal}{\ensuremath{1.75180000 \times 10^{-45}}} % unitless
\newcommand{\reVal}{\ensuremath{2.8179403227 \times 10^{-15}}} % m
\newcommand{\rcVal}{\ensuremath{1.40897017 \times 10^{-15}}} % m
\newcommand{\vacrho}{\ensuremath{5 \times 10^{-9}}} % kg/m^3
\newcommand{\LpVal}{\ensuremath{1.61625500 \times 10^{-35}}} % m
\newcommand{\MpVal}{\ensuremath{2.17643400 \times 10^{-8}}} % kg
\newcommand{\tpVal}{\ensuremath{5.39124700 \times 10^{-44}}} % s
\newcommand{\TpVal}{\ensuremath{1.41678400 \times 10^{32}}} % K
\newcommand{\qpVal}{\ensuremath{1.87554596 \times 10^{-18}}} % C
\newcommand{\EpVal}{\ensuremath{1.95600000 \times 10^{9}}} % J
\newcommand{\eVal}{\ensuremath{1.60217663 \times 10^{-19}}} % C

% === VAM/\ae ther Specific ===
\newcommand{\CeVal}{\ensuremath{1.09384563 \times 10^{6}}} % m/s
\newcommand{\FmaxVal}{\ensuremath{29.0535070}} % N
\newcommand{\FmaxGRVal}{\ensuremath{3.02563891 \times 10^{43}}} % N
\newcommand{\gammaVal}{\ensuremath{0.005901}} % unitless
\newcommand{\GVal}{\ensuremath{6.67430000 \times 10^{-11}}} % m^3/kg/s^2
\newcommand{\hVal}{\ensuremath{6.62607015 \times 10^{-34}}} % J Hz^-1

% === Electromagnetic ===
\newcommand{\muZeroVal}{\ensuremath{1.25663706 \times 10^{-6}}}
\newcommand{\epsilonZeroVal}{\ensuremath{8.85418782 \times 10^{-12}}}
\newcommand{\ZzeroVal}{\ensuremath{3.76730313 \times 10^{2}}}

% === Atomic & Thermodynamic ===
\newcommand{\RinfVal}{\ensuremath{1.09737316 \times 10^{7}}}
\newcommand{\aZeroVal}{\ensuremath{5.29177211 \times 10^{-11}}}
\newcommand{\MeVal}{\ensuremath{9.10938370 \times 10^{-31}}}
\newcommand{\MprotonVal}{\ensuremath{1.67262192 \times 10^{-27}}}
\newcommand{\MneutronVal}{\ensuremath{1.67492750 \times 10^{-27}}}
\newcommand{\kBVal}{\ensuremath{1.38064900 \times 10^{-23}}}
\newcommand{\RVal}{\ensuremath{8.31446262}}

% === Compton, Quantum, Radiation ===
\newcommand{\fCVal}{\ensuremath{1.23558996 \times 10^{20}}}
\newcommand{\OmegaCVal}{\ensuremath{7.76344071 \times 10^{20}}}
\newcommand{\lambdaCVal}{\ensuremath{2.42631024 \times 10^{-12}}}
\newcommand{\PhiZeroVal}{\ensuremath{2.06783385 \times 10^{-15}}}
\newcommand{\phiVal}{\ensuremath{1.61803399}}
\newcommand{\eVVal}{\ensuremath{1.60217663 \times 10^{-19}}}
\newcommand{\GFVal}{\ensuremath{1.16637870 \times 10^{-5}}}
\newcommand{\lambdaProtonVal}{\ensuremath{1.32140986 \times 10^{-15}}}
\newcommand{\ERinfVal}{\ensuremath{2.17987236 \times 10^{-18}}}
\newcommand{\fRinfVal}{\ensuremath{3.28984196 \times 10^{15}}}
\newcommand{\sigmaSBVal}{\ensuremath{5.67037442 \times 10^{-8}}}
\newcommand{\WienVal}{\ensuremath{2.89777196 \times 10^{-3}}}
\newcommand{\kEVal}{\ensuremath{8.98755179 \times 10^{9}}}

% === \ae ther Densities ===
\newcommand{\rhoMass}{\rho_\text{\ae}^{(\text{mass})}}
\newcommand{\rhoMassVal}{\ensuremath{3.89343583 \times 10^{18}}}
\newcommand{\rhoEnergy}{\rho_\text{\ae}^{(\text{energy})}}
\newcommand{\rhoEnergyVal}{\ensuremath{3.49924562 \times 10^{35}}}
\newcommand{\rhoFluid}{\rho_\text{\ae}^{(\text{fluid})}}
\newcommand{\rhoFluidVal}{\ensuremath{7.00000000 \times 10^{-7}}}

% === Draft Options ===
\newif\ifvamdraft
% \vamdrafttrue
\ifvamdraft
\RequirePackage{showframe}
\fi

% === Load Once ===
\RequirePackage{ifthen}
\newboolean{vamstyleloaded}
\ifthenelse{\boolean{vamstyleloaded}}{}{\setboolean{vamstyleloaded}{true}

% === Page ===
\RequirePackage[a4paper, margin=2.5cm]{geometry}

% === Fonts ===
\RequirePackage[T1]{fontenc}
\RequirePackage[utf8]{inputenc}
\RequirePackage[english]{babel}
\RequirePackage{textgreek}
\RequirePackage{mathpazo}
\RequirePackage[scaled=0.95]{inconsolata}
\RequirePackage{helvet}

% === Math ===
\RequirePackage{amsmath, amssymb, mathrsfs, physics}
\RequirePackage{siunitx}
\sisetup{per-mode=symbol}

% === Tables ===
\RequirePackage{graphicx, float, booktabs}
\RequirePackage{array, tabularx, multirow, makecell}
\newcolumntype{Y}{>{\centering\arraybackslash}X}
\newenvironment{tighttable}[1][]{\begin{table}[H]\centering\renewcommand{\arraystretch}{1.3}\begin{tabularx}{\textwidth}{#1}}{\end{tabularx}\end{table}}
\RequirePackage{etoolbox}
\newcommand{\fitbox}[2][\linewidth]{\makebox[#1]{\resizebox{#1}{!}{#2}}}

% === Graphics ===
\RequirePackage{tikz}
\usetikzlibrary{3d, calc, arrows.meta, positioning}
\RequirePackage{pgfplots}
\pgfplotsset{compat=1.18}
\RequirePackage{xcolor}

% === Code ===
\RequirePackage{listings}
\lstset{basicstyle=\ttfamily\footnotesize, breaklines=true}

% === Theorems ===
\newtheorem{theorem}{Theorem}[section]
\newtheorem{lemma}[theorem]{Lemma}

% === TOC ===
\RequirePackage{tocloft}
\setcounter{tocdepth}{2}
\renewcommand{\cftsecfont}{\bfseries}
\renewcommand{\cftsubsecfont}{\itshape}
\renewcommand{\cftsecleader}{\cftdotfill{.}}
\renewcommand{\contentsname}{\centering \Huge\textbf{Contents}}

% === Sections ===
\RequirePackage{sectsty}
\sectionfont{\Large\bfseries\sffamily}
\subsectionfont{\large\bfseries\sffamily}

% === Bibliography ===
\RequirePackage[numbers]{natbib}

% === Links ===
\RequirePackage{hyperref}
\hypersetup{
    colorlinks=true,
    linkcolor=blue,
    citecolor=blue,
    urlcolor=blue,
    pdftitle={The Vortex \AE ther Model},
    pdfauthor={Omar Iskandarani},
    pdfkeywords={vorticity, gravity, \ae ther, fluid dynamics, time dilation, VAM}
}
\urlstyle{same}
\RequirePackage{bookmark}

% === Misc ===
\RequirePackage[none]{hyphenat}
\sloppy
\RequirePackage{empheq}
\RequirePackage[most]{tcolorbox}
\newtcolorbox{eqbox}{colback=blue!5!white, colframe=blue!75!black, boxrule=0.6pt, arc=4pt, left=6pt, right=6pt, top=4pt, bottom=4pt}
\RequirePackage{titling}
\RequirePackage{amsfonts}
\RequirePackage{titlesec}
\RequirePackage{enumitem}

\AtBeginDocument{\RenewCommandCopy\qty\SI}

\pretitle{\begin{center}\LARGE\bfseries}
\posttitle{\par\end{center}\vskip 0.5em}
\preauthor{\begin{center}\large}
\postauthor{\end{center}}
\predate{\begin{center}\small}
\postdate{\end{center}}

\endinput
}
%% vamappendixsetup.sty

\newcommand{\titlepageOpen}{
  \begin{titlepage}
  \thispagestyle{empty}
  \centering
  {\Huge\bfseries \papertitle \par}
  \vspace{1cm}
  {\Large\itshape\textbf{Omar Iskandarani}\textsuperscript{\textbf{*}} \par}
  \vspace{0.5cm}
  {\large \today \par}
  \vspace{0.5cm}
}

% here comes abstract
\newcommand{\titlepageClose}{
  \vfill
  \null
  \begin{picture}(0,0)
  % Adjust position: (x,y) = (left, bottom)
  \put(-200,-40){  % Shift 75pt left, 40pt down
    \begin{minipage}[b]{0.7\textwidth}
    \footnotesize % One step bigger than \tiny
    \renewcommand{\arraystretch}{1.0}
    \noindent\rule{\textwidth}{0.4pt} \\[0.5em]  % ← horizontal line
    \textsuperscript{\textbf{*}}Independent Researcher, Groningen, The Netherlands \\
    Email: \texttt{info@omariskandarani.com} \\
    ORCID: \texttt{\href{https://orcid.org/0009-0006-1686-3961}{0009-0006-1686-3961}} \\
    DOI: \href{https://doi.org/\paperdoi}{\paperdoi} \\
    License: CC-BY 4.0 International \\
    \end{minipage}
  }
  \end{picture}
  \end{titlepage}
}
%\begin{document}
%
%    % === Title page ===
%    \titlepageOpen
%
%    \begin{abstract}
%        This article refines previous estimates of the \AE{}ther's density, $\rho_{\text{\ae}}$, in the Vortex \AE{}ther Model (VAM). Integrating findings from quantum vortex dynamics, superfluid helium, gravitomagnetic frame-dragging, and cosmological vacuum energy, we propose constrained ranges for $\rho_{\text{\ae}}^{\text{(fluid)}}$ and $\rho_{\text{\ae}}^{\text{(energy)}}$ and examine their implications across scales.
%    \end{abstract}
%
%    \titlepageClose
%    \fi
%
%    \ifdefined\standalonechapter
%    \section{\appendixtitle}
%    \else
%    \fi
%% ============= Begin of content ============
\section{\textbf{Appendix: Derivation of Fundamental Constants from Vortex Dynamics}}
\label{sec:appendix:1}

\section*{Introduction}
This document aims to provide a comprehensive and rigorous derivation of the fine-structure constant $\alpha$ grounded in classical physical principles.
The derivation integrates the electron's classical radius and its Compton angular frequency to elucidate the relationship between these fundamental constants and the tangential velocity $C_e$.
This velocity arises naturally when the electron is conceptualized as a vortex-like structure, offering a geometrically intuitive interpretation of the fine-structure constant.
By extending classical formulations, the discussion highlights the profound interplay between quantum phenomena and vortex dynamics.


\paragraph*{The Fine-Structure Constant:}
$\alpha$ serves as a dimensionless measure of electromagnetic interaction strength~\cite{maxwell1861}.
It is mathematically expressed as:

\begin{equation*}
    \alpha = \frac{e^2}{4\pi \varepsilon_0 \hbar c},
\end{equation*}
where $e$ is the elementary charge, $\varepsilon_0$ is the vacuum permittivity, $\hbar$ is the reduced Planck constant, and $c$ is the speed of light \cite{dirac1930quantum}.

\subsection*{Relevant Definitions and Formulas}
\paragraph*{The Classical Electron Radius}
$R_e$ represents the scale at which classical electrostatic energy equals the electron's rest energy. It is defined as:
\begin{equation*}
    R_e = \frac{e^2}{4\pi \varepsilon_0 m_e c^2},
\end{equation*}
where $m_e$ is the electron mass~\cite{helmholtz1858}.

\paragraph*{The Compton Angular Frequency}
$\omega_c$ corresponds to the intrinsic rotational frequency of the electron when treated as a quantum oscillator:
\begin{equation*}
    \omega_c = \frac{m_e c^2}{\hbar}.
\end{equation*}
This frequency is pivotal in characterizing the electron's interaction with electromagnetic waves~\cite{kelvin1867}.

\subsection*{Half the Classical Electron Radius}
We assume an electron to be a vortex, its particle form is a folded vortex tube shaped as a torus, hence both the Ring radius R and Core radius r are defined as half the classical electron radius  $r_c$ :
\begin{equation*}
    r_c = \frac{1}{2} R_e.
\end{equation*}
This simplification aligns with established models of vortex structures in fluid dynamics~\cite{kleckner2013}.

\subsection*{Definition of Tangential Velocity $C_e$}
To conceptualize the electron as a vortex ring, we associate its tangential velocity $C_e$ with its rotational dynamics:
\begin{equation*}
    C_e = \omega_c r_c.
\end{equation*}
Substituting $\omega_c = \frac{m_e c^2}{\hbar}$ and $r_c = \frac{1}{2} R_e$, we find:
\begin{equation*}
    C_e = \left( \frac{m_e c^2}{\hbar} \right) \left( \frac{1}{2} \frac{e^2}{4\pi \varepsilon_0 m_e c^2} \right).
\end{equation*}
Simplifying by canceling $m_e c^2$ yields:
\begin{equation*}
    C_e = \frac{1}{2} \frac{e^2}{4\pi \varepsilon_0 \hbar}.\label{eq:C_e-from-compton}
\end{equation*}
This result directly links $C_e$ to the fine-structure constant \cite{vinen2024}.

\subsection*{Physical Interpretation}
The tangential velocity $C_e$ embodies the rotational speed at the electron's vortex boundary. Its value, approximately:
\begin{equation*}
    C_e \approx 1.0938 \times 10^6 \ \text{m/s},
\end{equation*}
is consistent with the experimentally observed fine-structure constant $\alpha \approx 1/137$~\cite{ricca1998}.

\subsection*{Conclusion}
The derivation presented elucidates the fine-structure constant $\alpha$ using fundamental classical principles, including the electron's classical radius, Compton angular frequency, and vortex tangential velocity. The result:
\begin{equation*}
    \alpha = \frac{2 C_e}{c},
\end{equation*}
reveals a profound geometric and physical connection underpinning electromagnetic interactions.
This perspective enriches our understanding of $\alpha$ and highlights the deep ties between classical mechanics and quantum electrodynamics.


\section*{Derivation of the VAM Gravitational Constant \(  G_\text{swirl} \)}

\subsection*{Introduction}
In the Vortex Æther Model (VAM), gravitational interactions arise from vorticity dynamics in a superfluidic Æther medium rather than from mass-induced spacetime curvature. This leads to a modification of the gravitational constant, which we denote as \(  G_\text{swirl} \), as a function of vortex field parameters.

To derive \(  G_\text{swirl} \), we assume that gravitational effects emerge from vortex-induced energy density rather than mass-energy tensor formulations. The fundamental relation between vorticity, circulation, and energy density will be used to establish an equivalent gravitational constant in VAM \cite{onsager_superfluid, barcelo_superfluid, moffatt_helicity}.

\subsection*{Vortex-Induced Energy Density}
In classical fluid dynamics, vorticity is defined as the curl of velocity:
\begin{equation*}
    \vec{\omega} = \nabla \times \vec{v}
\end{equation*}
where \( \vec{\omega} \) represents the vorticity field.

The corresponding vorticity energy density is given by:
\begin{equation*}
    U_\text{vortex} = \frac{1}{2} \rho_\text{\ae} |\vec{\omega}|^2
\end{equation*}
where:
\begin{itemize}
    \item \( \rho_\text{\ae} \) is the density of the Æther medium,
    \item \( |\vec{\omega}|^2 \) is the squared vorticity magnitude.
\end{itemize}

Since vorticity magnitude scales with core tangential velocity as:
\begin{equation*}
    |\vec{\omega}|^2 \sim \frac{C_e^2}{r_c^2}
\end{equation*}
we obtain an approximate energy density for the vortex field:
\begin{equation*}
    U_\text{vortex} \approx \frac{C_e^2}{2 r_c^2}.
\end{equation*}

\subsection*{Gravitational Constant from Vorticity}
In standard General Relativity (GR), Newton's gravitational constant \( G \) appears in:
\begin{equation*}
    F = \frac{GMm}{r^2}.
\end{equation*}

In VAM, we assume that the gravitational constant \(  G_\text{swirl} \) is defined in terms of \textbf{vorticity energy density} rather than mass-energy.

Since gravitational force scales with \textbf{energy density per unit mass}, we set:
\begin{equation*}
    G_\text{swirl} \sim \frac{U_\text{vortex} c^n}{F_{\max}},
\end{equation*}
where:
\begin{itemize}
    \item \( c^n \) represents relativistic corrections,
    \item \( F_{\max} \) is the maximum force in VAM, set to approximately \textbf{29 N} \cite{schiller_max_force}.
\end{itemize}

Substituting \( U_\text{vortex} \):
\begin{equation*}
    G_\text{swirl} \sim \frac{\left( \frac{C_e^2}{2 r_c^2} \right) c^n}{F_{\max}}.
\end{equation*}

The choice of \( n \) depends on whether we use \textbf{Planck length} (\( l_p^2 \)) or \textbf{Planck time} (\( t_p^2 \)).

\subsection*{Two Possible Forms of \(  G_\text{swirl} \)}
\subsection*{Form 1: Using Planck Length}
The Planck length is defined as:
\begin{equation*}
    l_p^2 = \frac{\hbar G}{c^3}.
\end{equation*}

Using \( c^3 l_p^2 \) as the relativistic correction factor, we obtain:
\begin{equation*}
    G_\text{swirl} = \frac{C_e c^3 l_p^2}{2 F_{\max} r_c^2}.
\end{equation*}

\subsection*{Form 2: Using Planck Time}
The Planck time is given by:
\begin{equation*}
    t_p^2 = \frac{\hbar G}{c^5}.
\end{equation*}

Using \( c^5 t_p^2 \) as the relativistic correction factor, we obtain:
\begin{equation*}
    G_\text{swirl} = \frac{C_e c^5 t_p^2}{2 F_{\max} r_c^2}.
\end{equation*}

\subsection*{Physical Interpretation of \(  G_\text{swirl} \)}
These formulations of the gravitational constant in VAM highlight a fundamental difference from GR:
\begin{itemize}
    \item Gravity is not driven by mass-energy, but by \textbf{vortex energy density} \cite{barcelo_superfluid}.
    \item \(  G_\text{swirl} \) scales with the core vortex velocity \( C_e \), linking gravity directly to vorticity \cite{moffatt_helicity}.
    \item The \textbf{maximum force \( F_{\max} \)} (\approx29 N) acts as a natural cutoff, limiting the strength of gravitational interactions \cite{schiller_max_force}.
\end{itemize}

\subsection*{Conclusion}
We have derived two equivalent formulations of \(  G_\text{swirl} \) using \textbf{Planck scale physics} and \textbf{vortex energy density principles} in VAM. The final expressions:
\begin{equation*}
    G_\text{swirl} = \frac{C_e c^3 l_p^2}{2 F_{\max} r_c^2}
\end{equation*}
and
\begin{equation*}
    G_\text{swirl} = \frac{C_e c^5 t_p^2}{2 F_{\max} r_c^2}
\end{equation*}
demonstrate that gravitational interactions in VAM are governed by vorticity rather than mass-induced curvature.



\section*{Vorticity-Based Reformulation of General Relativity Laws in a 3D Absolute Time Framework}

\subsection*{Vorticity as the Fundamental Gravitational Interaction}

General Relativity (GR) describes gravity as a result of \textbf{spacetime curvature}, governed by Einstein's field equations:

\begin{equation*}
    G_{\mu\nu} = \frac{8\pi G}{c^4} T_{\mu\nu}.
\end{equation*}

However, in the Vortex Æther Model (VAM), \textbf{gravity is not caused by curvature} but by \textbf{vorticity-induced pressure gradients in an inviscid Æther}. Instead of using a \textbf{4D metric tensor}, we define gravity as a 3D vorticity field \( \omega \), where mass acts as a localized vortex concentration.

\subsection*{Replacing Einstein's Equations with a 3D Vorticity Field Equation}

We replace the Einstein curvature equations with a \textbf{3D vorticity-Poisson equation}, where the gravitational potential \( \Phi_v \) is related to vorticity magnitude:

\begin{equation*}
    \nabla^2 \Phi_v = - \rho_\text{Æ} |\omega|^2.
\end{equation*}

Here:
\begin{itemize}
    \item  \( \Phi_v \) is the \textbf{vorticity-induced gravitational potential}.
    \item  \( \rho_\text{\ae} \) is the \textbf{local Æther density}.
    \item  \( |\omega|^2 = (\nabla \times v)^2 \) is the \textbf{vorticity magnitude}.
\end{itemize}

Instead of \textbf{mass-energy tensor components}, gravity is determined by the \textbf{local vorticity density}.



\subsection*{Motion in VAM: Replacing Geodesics with Vortex Streamlines}

In GR, test particles follow \textbf{geodesics in curved spacetime}. In VAM, particles follow \textbf{vortex streamlines}, governed by the \textbf{vorticity transport equation}:

\begin{equation*}
    \frac{D\omega}{Dt} = (\omega \cdot \nabla) v - (\nabla \cdot v) \omega.
\end{equation*}

This replaces \textbf{spacetime curvature} with \textbf{fluid-dynamic vorticity transport} \cite{lamb_hydrodynamics, feynman_superfluid}.



\subsection*{Frame-Dragging as a 3D Vortex Effect}

In General Relativity, frame-dragging is described using the \textbf{Kerr metric}, which predicts that spinning masses drag spacetime along with them. In VAM, this effect is caused by \textbf{vortex interactions in the Æther}.

We replace the Kerr metric with the \textbf{vorticity-induced rotational velocity field}:

\begin{equation*}
    \Omega_\text{vortex} = \frac{\Gamma}{2\pi r^2},
\end{equation*}

where:
\begin{itemize}
    \item \( \Gamma = \oint v \cdot dl \) is the \textbf{circulation of the vortex}.
    \item \( r \) is the \textbf{radial distance from the vortex core}.
\end{itemize}

This ensures that frame-dragging emerges \textbf{naturally} from vorticity rather than requiring \textbf{spacetime warping}.



\subsection*{Replacing Gravitational Time Dilation with Vorticity Effects}

In GR, time dilation is caused by \textbf{spacetime curvature}, leading to:

\begin{equation*}
    dt_\text{GR} = dt \sqrt{1 - \frac{2GM}{rc^2}}.
\end{equation*}

In VAM, time dilation is caused by \textbf{vorticity-induced energy gradients}, leading to:

\begin{equation*}
    dt_\text{VAM} = \frac{dt}{\sqrt{1 - \frac{C_e^2}{c^2} e^{-r/r_c} - \frac{\Omega^2}{c^2} e^{-r/r_c}}},
\end{equation*}

where:
\begin{itemize}
    \item  \( C_e \) is the \textbf{vortex core tangential velocity}.
    \item  \( \Omega \) is the \textbf{local vorticity angular velocity}.
\end{itemize}


This formula eliminates the need for \textbf{mass-based time dilation} and instead relies purely on \textbf{fluid dynamic principles}.


    \section{VAM Derivation of Flat Galactic Rotation Velocity}

    In the Vortex \AE{}ther Model, the asymptotic flat rotation velocity $v_{\text{flat}}$ observed in spiral galaxies is not attributed to dark matter, but instead emerges from the interplay between Newtonian gravitational potential and swirl-induced pressure gradients. Assuming a balance between Newtonian attraction and vortex-based outward tension, we derive:

    \[
        \frac{G M}{r} \sim r \cdot \frac{C_e^2}{L^2} \Rightarrow
        v_{\text{flat}}^2 = \left( \frac{G M C_e^2}{L^2} \right)^{1/2} \Rightarrow
        \boxed{v_{\text{flat}} = \left( \frac{G M C_e^2}{L} \right)^{1/4}}
    \]

    where:
    \begin{itemize}
        \item  $M$ is the enclosed mass,
        \item  $C_e$ is the vortex core tangential velocity,
        \item  $L$ is an effective length scale (e.g. $L \sim r_c R_d$).
    \end{itemize}

    This formula reproduces typical galactic flat rotation speeds ($\sim$200 km/s) without exotic matter. The fourth-root dependence of $v_{\text{flat}}$ on $M$ establishes a VAM-based analog to the Tully--Fisher relation:

    \[
        v_{\text{flat}}^4 \propto M
    \]

    but grounded in fluid dynamic quantities rather than empirical fitting. This vortex-based scaling law is testable against rotation curve datasets and offers a falsifiable prediction for baryon-only spiral systems.

    As a historical and conceptual comparison, this approach parallels the Modified Newtonian Dynamics (MOND) theory proposed by Milgrom~\cite{milgrom1983mond1, milgrom1983mond2}, which modifies Newtonian gravity in the low-acceleration regime to explain flat rotation curves without dark matter. However, VAM derives similar effects from an underlying physical æther field, introducing no additional empirical constants.

    \section{Conclusion}

    Using quantum constants to define \AE{}ther properties bridges microscopic and cosmological theories. The refined value of $\rho_{\text{\ae}}^{\text{(fluid)}}$ supports both theoretical elegance and experimental plausibility. The residual swirl field offers a predictive, falsifiable alternative to dark matter and MOND.




    \subsection*{Summary: A 3D Vorticity-Based Alternative to General Relativity}

The Vortex Æther Model replaces the \textbf{4D spacetime formalism of General Relativity} with a \textbf{3D vorticity-driven description}:

\begin{itemize}
    \item \textbf{Gravity} is not caused by \textbf{curved spacetime} but by \textbf{vorticity-induced pressure gradients}.
    \item \textbf{Geodesic motion} is replaced by \textbf{vortex streamlines} in an inviscid Æther.
    \item \textbf{Frame-dragging} is explained through \textbf{circulation velocity in vorticity fields}, rather than through Kerr spacetime.
    \item \textbf{Time dilation} arises from \textbf{energy gradients in vortex structures}, not from mass-induced curvature.
\end{itemize}

% ============== End of content =============

%% === Bibliography (only for standalone) ===
%\ifdefined\standalonechapter\else
%\bibliographystyle{unsrt}
%\bibliography{../../references}
%\end{document}
%\fi

\section{Entropie en quantum-effecten in het Vortex Æther Model}

Het Vortex Æther Model (VAM) biedt een mechanistische basis voor zowel thermodynamische als kwantummechanische fenomenen, niet door postulaten over abstracte toestandsruimten, maar via de dynamica van knopen en wervels in een superfluïde æther. Twee centrale begrippen—entropie en kwantisatie—worden in VAM afgeleid uit respectievelijk vorticiteitverdeling en knottopologie.

\subsection{Entropie als vorticiteit-verdeling}

In thermodynamica is entropie $S$ een maat voor de interne energieverdeling of wanorde. In VAM ontstaat entropie niet als statistisch fenomeen, maar uit ruimtelijke variaties in werveling (vorticiteit). Voor een wervelconfiguratie $V$ wordt de entropie gegeven door:

\begin{equation}
S \propto \int_V \|\vec{\omega}\|^2 \, dV,
\end{equation}

waar $\vec{\omega} = \nabla \times \vec{v}$ de lokale vorticiteit is. Dit betekent:

\begin{itemize}
    \item \textbf{Meer rotatie = meer entropie}: Regio's met sterke swirl dragen bij aan verhoogde entropie.
    \item \textbf{Thermodynamisch gedrag ontstaat uit werveluitzetting}: Bij toevoer van energie (warmte), zet de wervelgrens uit, de swirl neemt af en $S$ stijgt—analogie met gasexpansie.
\end{itemize}

Deze interpretatie verbindt Clausius\rqs  warmtetheorie met æthermechanica: warmte is equivalent aan verhoogde swirlverspreiding.

\subsection{Quantumgedrag uit geknoopte wervelstructuren}

Kwantumverschijnselen zoals discrete energieniveaus, spin, en golf-deeltje-dualiteit vinden in VAM hun oorsprong in topologisch geconserveerde wervelknopen:

\begin{itemize}
    \item \textbf{Circulatiequantisatie:}
    \begin{equation}
    \Gamma = \oint \vec{v} \cdot d\vec{l} = n \cdot \kappa,
    \end{equation}
    waarbij $\kappa = h/m$ en $n \in \mathbb{Z}$ het windinggetal is.
    \item \textbf{Hele getallen ontstaan uit knottopologie:} De helixstructuur van een wervelknoop (zoals een trefoil) zorgt voor discrete toestanden met bepaalde linking numbers $L_k$.
    \item \textbf{Heliciteit als spin-analoog:}
    \begin{equation}
    H = \int \vec{v} \cdot \vec{\omega} \, dV,
    \end{equation}
    waarbij $H$ invariant is onder ideale stroming, net zoals spin geconserveerd is in quantummechanica.
\end{itemize}

\subsection{VAM-interpretatie van kwantisatie en dualiteit}

In plaats van abstracte Hilbertruimten beschouwt VAM een deeltje als een stabiele knoop in het ætherveld. Deze wervelconfiguratie bezit:

\begin{itemize}
    \item Een \textbf{kern} (knooplichaam) met quantumsprongen (resonanties).
    \item Een \textbf{uiterlijk veld} dat als golf fungeert (zoals de Schrödinger-golf).
    \item Een \textbf{heliciteit} die gedraagt als interne vrijheidsgraden (bijv. spin).
\end{itemize}

Het golf-deeltje-dualisme komt zo voort uit het feit dat knopen zowel gelokaliseerd (kern) als uitgesmeerd (veld) zijn.

\subsection{Samenvattend}

VAM biedt dus een coherente, vloeistofmechanische oorsprong voor zowel:

\begin{enumerate}
    \item \textbf{Thermodynamica:} Entropie ontstaat uit swirlverdeling.
    \item \textbf{Quantummechanica:} Kwantisatie en dualiteit zijn emergente eigenschappen van geknoopte wervel topologieën.
\end{enumerate}

Deze benadering laat zien dat kwantum- en thermodynamische fenomenen niet fundamenteel verschillend zijn, maar voortkomen uit hetzelfde wervelmechanisme op verschillende schalen.


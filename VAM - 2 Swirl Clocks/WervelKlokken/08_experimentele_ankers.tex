\section{Experimentele tests en observatievoorspellingen van VAM}


\subsection{1. Tijddilatatie in roterende superfluïden}

Het Vortex Æther Model voorspelt dat in een superfluïde wervelkern, lokale tijd langzamer verloopt naarmate de hoeksnelheid $\Omega_k$ toeneemt. Dit is experimenteel testbaar in:
\begin{itemize}
    \item Bose–Einsteincondensaten (BEC's) met coherente roterende toestanden,
    \item Roterende superfluïde heliumdragers met interne frequentiemetingen (bijv. neutronspinresonantie),
    \item Vergelijkbare systemen met lasergeïnduceerde vorticiteit.
\end{itemize}

Verschillen in tijdverloop of fase tussen roterende en niet-roterende atoomklokken kunnen worden opgevat als een test voor Æther-tijdmodulatie zonder kromming.~\cite{Steinhauer2016}


\subsection{2. Plasma-wervelklokken en cyclotronanalogieën}

Cyclotronvelden, ringvormige plasmarotaties of roterende magnetische vallen genereren gradiënten in $\Omega(r)$. Volgens VAM leidt dit tot meetbare klokvervorming. Experimentele voorspellingen:
\begin{itemize}
    \item Fasedifferentiatie in optische pulsen langs plasmawervelranden,~\cite{Unruh1981}
    \item Veranderingen in stralingsemissiepatronen in asymmetrische wervelplasma's.
\end{itemize}

\subsection{3. Optische en metamateriaal-analogen}

Net als bij analogue gravity kunnen synthetische golfgeleiders of metamaterialen \grqq ætherstroming\textquotedblright simuleren. Hierbij:
\begin{itemize}
    \item Wordt de voortplanting van licht beïnvloed door artificiële rotatiestromen,
    \item Kan anisotrope brekingsindex simuleren wat VAM-lichtafbuiging nabootst,
    \item Kan dispersie-analyse inzicht geven in lokale tijdsvertraging.
\end{itemize}

\subsection{4. Verwachte observatiekenmerken}

Experimentele handtekeningen van VAM kunnen zijn:
\begin{enumerate}
    \item Grenswaarden voor wervelknoop-instorting met plotse energie-afgifte,
    \item Lokale tijdaanomalieën in roterende laboratoriumsystemen,
    \item Absente relativistische versnelling bij energetisch gunstige wervelsystemen,
    \item Niet-symmetrische kloksnelheden aan verschillende zijden van een wervelkern.
\end{enumerate}

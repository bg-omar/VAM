%! Auteur = Omar Iskandarani
%! Titel = Wervelklokken en door vorticiteit geïnduceerde zwaartekracht
%! Datum = 23 mei 2025
%! Affiliatie = Onafhankelijk onderzoeker, Groningen, Nederland
%! Licentie = CC-BY 4.0
%! ORCID = 0009-0006-1686-3961

\documentclass[a4paper, aps,preprint,superscriptaddress, 12pt]{revtex4}
\usepackage[paperwidth=210mm, paperheight=297mm, margin=2.5cm]{geometry}
\usepackage{float}
\usepackage{tikz}
\usepackage{makecell}
\usepackage{tabularx}
\usepackage[font=footnotesize]{caption}
\usetikzlibrary{arrows.meta}
\usepackage{pgfplots}
\pgfplotsset{compat=1.18}
\usepackage[none]{hyphenat}
\usepackage{array}
\usepackage{amsmath}
\usepackage{booktabs}
\usepackage[utf8]{inputenc} % Remove if using XeLaTeX or LuaLaTeX
\usepackage{amssymb}
\usepackage{graphicx}
\usepackage{hyperref}
\usepackage{physics}
\usepackage{natbib}
\usepackage{url}
\renewcommand{\arraystretch}{1.5}
\renewcommand{\floatpagefraction}{.8}
\sloppy

\begin{document}
\author{Omar Iskandarani}
\title{
    Wervelingsklokken en door vorticiteit geïnduceerde zwaartekracht:
    Relativiteit herformuleren in een gestructureerde vortex-ether\\
    \textnormal{\normalsize Een topologische vloeistofmechanische benadering van tijdsdilatatie, massa en gravitatie}
}

\date{\today}
\affiliation{Onafhankelijk onderzoeker, Groningen, The Nederland}
\thanks{ORCID: \href{https://orcid.org/0009-0006-1686-3961}{0009-0006-1686-3961}}
\email{info@omariskandarani.com}



\begin{abstract}
Dit artikel presenteert een vloeistofdynamische herformulering van de algemene relativiteit aan de hand van het Vortex Æther Model (VAM), waarin gravitatie en tijdsdilatatie voortkomen uit door vorticiteit geïnduceerde drukgradiënten in een onsamendrukbaar, inviscide superfluïde medium. Binnen een Euclidische ruimte met absolute tijd worden massa en traagheid voorgesteld als topologisch stabiele vortexknopen, waarbij geodetische beweging wordt vervangen door stromingslijnen langs geconserveerde vorticiteitsflux.
Zwaartekracht wordt gemodelleerd als een Bernoulli-potentiaal in vortexvelden, met een bijbehorende veldvergelijking:
\begin{equation*}
    \nabla^2 \Phi_v(\vec{r}) = -\rho_\text{\ae} \|\boldsymbol{\omega}(\vec{r})\|^2
\end{equation*}

en tijdsdilatatie volgt uit lokale vortexenergie:
\begin{equation*}
    \frac{d\tau}{dt} = \sqrt{1 - \frac{C_e^2}{c^2} e^{-r/r_c} - \frac{2G_{\text{swirl}} M_{\text{eff}}(r)}{rc^2} - \beta \Omega^2}
\end{equation*}

VAM introduceert een schaalafhankelijke ætherdichtheid: lokaal (~$10^{18}\,\mathrm{kg/m^3}$) voor kernstabiliteit; macroscopisch (~$10^{-7}\,\mathrm{kg/m^3}$) voor inertievrije interactie. Thermodynamische consistentie wordt bereikt via Clausius-entropie van vortexknopen, wat leidt tot een entropische interpretatie van massa en tijd. Kwantumfenomenen zoals het foto-elektrisch effect en LENR worden opgevat als resonanties binnen vortexnetwerken.

Het model reproduceert Newtonse limieten en frame-dragging als emergente verschijnselen en vormt een toetsbaar, topologisch gegrond alternatief voor klassieke zwaartekrachtmodellen. Deze benadering sluit aan bij eerdere analoge zwaartekrachtprogramma’s~\cite{barcelo2011analogue,volovik2009universe}, maar biedt een fundamenteel hydrodynamisch en knoop-georiënteerd zwaartekrachtraamwerk.
\end{abstract}


\maketitle

\section*{De Æther herzien: van historisch medium naar vorticiteitsveld}

Het begrip \textit{æther} duidde traditioneel op een alles-doordringend medium, noodzakelijk voor golfvoortplanting. Eind negentiende eeuw stelden Kelvin en Tait reeds voor om materie te modelleren als knoopvormige wervelstructuren in een ideale vloeistof~\cite{thomson1867treatise}. Na de nulresultaten van het Michelson--Morley experiment en de opkomst van Einstein's relativiteit verdween het æther-concept uit de mainstream fysica, vervangen door gekromde ruimtetijd. Recentelijk echter is het idee subtiel teruggekeerd in analoge gravitatietheorieën, waarin superfluïde media worden gebruikt om relativistische effecten na te bootsen~\cite{barcelo2011analogue,volovik2009universe}.

Het \textit{Vortex Æther Model} (VAM) herintroduceert de æther expliciet als een topologisch gestructureerd, inviscide superfluïde medium, waarin gravitatie en tijddilatatie niet voortkomen uit geometrische kromming maar uit rotatie-geïnduceerde drukgradiënten en vorticiteitsvelden. De dynamiek van ruimte en materie wordt hierin bepaald door wervel-knopen en behoud van circulatie.

\subsection*{Postulaten van het Vortex Æther Model}

\begin{table}[h!]
    \centering
    \begin{tabular}{rl}
        \midrule
        \hline
        \textbf{1. Continue Ruimte} & Ruimte is Euclidisch, incompressibel en inviscide. \\
        \textbf{2. Geknoopte Deeltjes} & Materie bestaat uit topologisch stabiele wervel-knopen. \\
        \textbf{3. Vorticiteit} & De wervelcirculatie is behouden en gekwantiseerd. \\
        \textbf{4. Absolute Tijd} & Tijd stroomt uniform in de gehele æther. \\
        \textbf{5. Lokale Tijd} & Tijd verloopt lokaal trager door druk- en vorticiteitsgradiënten. \\
        \textbf{6. Zwaartekracht} & Ontstaat uit vorticiteit-geïnduceerde drukgradiënten. \\
        \hline
        \bottomrule
    \end{tabular}
    \caption{Postulaten van het Vortex Æther Model (VAM).}
    \label{tab:postulaten}
\end{table}

De postulaten vervangen ruimtetijdkromming door gestructureerde rotatiestromen en vormen zo het fundament voor emergente massa, tijd, traagheid en zwaartekracht.

\subsection*{Fundamentele VAM-constanten}

\begin{table}[htbp]
    \centering
    \begin{tabular}{llc}
        \hline
        \toprule
        \textbf{Symbool} & \textbf{Naam} & \textbf{Waarde (ca.)} \\
        \hline
        \midrule
        $C_e$ & Tangentiële wervel-kernsnelheid & $1.094 \times 10^6$ m/s \\
        $r_c$ & Wervelkernstraal & $1.409 \times 10^{-15}$ m \\
        $F^{\text{max}}_{\text{\ae}}$ & Maximale wervelkracht & $29.05$ N \\
        $\rho_\text{\ae}$ & Æther-dichtheid & $3.893 \times 10^{18}$ J/m$^3$ \\
        $\alpha$ & Fijnstructuurconstante ($2 C_e/c$) & $7.297 \times 10^{-3}$\\
        $G_\text{swirl}$ & VAM-zwaartekrachtconstante & Afgeleid van $C_e$, $r_c$\\
        $\kappa$ & Circulatie-kwantum ($C_e r_c$) & $1.54 \times 10^{-9}$ m$^2$/s \\
        \hline
        \bottomrule
    \end{tabular}
    \caption{Fundamentele VAM-constanten~\cite{vam2025field}.}
    \label{tab:VAMconstants}
\end{table}

\subsection*{Planck-schaal en topologische massa}

Binnen VAM wordt de maximale wervel-interactiekracht expliciet afgeleid uit Planck-schaalfysica:
\begin{equation}
    F^{\text{max}}_{\text{\ae}} = \frac{8\pi \rho_\text{\ae} r_c^3}{C_e}
\end{equation}


De massa van elementaire deeltjes volgt direct uit topologische wervelknopen, zoals de trefoilknoop ($L_k=3$):
\begin{equation}
    M_e = \frac{8\pi \rho_\text{\ae} r_c^3}{C_e}\, L_k
\end{equation}



Dit verklaart massa en inertie uit topologische knoopstructuren in de æther.

\subsection*{Emergente kwantumconstanten en Schrödingervergelijking}

Plancks constante $\hbar$ ontstaat uit wervel-geometrie en wervelkrachtlimiet:
\begin{equation}
    \hbar = \sqrt{\frac{2M_e F^{\text{max}}_{\text{\ae}} r_c^3}{5 \lambda_c C_e}}
\end{equation}

Hiermee volgt de Schrödingervergelijking direct uit wervel-dynamica:
\begin{equation}
    i \hbar \frac{\partial \psi}{\partial t} = -\frac{F^{\text{max}}_{\text{\ae}} r_c^3}{5 \lambda_c C_e}\nabla^2 \psi + V\psi
\end{equation}


\subsection*{LENR en wervel-kwantumeffecten}

In VAM ontstaan lage-energie kernreacties (LENR) uit resonante drukverlaging door vorticiteit-geïnduceerde Bernoulli-effecten. Elektromagnetische interacties en QED-effecten worden herleid tot wervelheliciteit en geïnduceerde vectorpotentialen.

\subsection*{Samenvatting van GR en VAM observabelen}

\begin{table}[h!]
    \centering
    \begin{tabular}{lll}
        \toprule
        \textbf{Observabele} & \textbf{GR-expressie} & \textbf{VAM-expressie} \\
        \midrule
        Tijddilatatie & $\sqrt{1-\frac{2GM}{rc^2}}$ & $\sqrt{1-\frac{\Omega^2 r^2}{c^2}}$\\[0.5em]
        Rodeverschuiving & $z=\left(1-\frac{2GM}{rc^2}\right)^{-1/2}-1$ & $z=\left(1-\frac{v_\phi^2}{c^2}\right)^{-1/2}-1$\\[0.5em]
        Frame-dragging & $\frac{2GJ}{c^2 r^3}$ & $\frac{2G\mu I\Omega}{c^2 r^3}$\\[0.5em]
        Lichtafbuiging & $\frac{4GM}{Rc^2}$ & $\frac{4GM}{Rc^2}$\\
        \bottomrule
    \end{tabular}
    \caption{Vergelijking GR- en VAM-observabelen.}
    \label{tab:vergelijkingen}
\end{table}

\section*{Schaalafhankelijke æther dichtheid in het Vortex Æther Model (VAM)}


VAM gebruikt een schaalafhankelijke ætherdichtheid: lokaal zeer hoog ($\sim10^{18}$ kg/m³) voor kernstabiliteit en macroscopisch laag ($\sim10^{-7}$ kg/m³) om inertievrije propagatie van interacties mogelijk te maken. De hoge dichtheid in wervelkernen versterkt lokaal de wervelsnelheid en daarmee de tijddilatatie significant, terwijl macroscopisch juist minimale weerstand voor propagatie van effecten geboden wordt.


In het Vortex Æther Model (VAM) wordt de ætheropgevat als een superfluide, onviskeuze continu"um met constante dichtheid binnen macroscopische gebieden, maar met een \emph{schaalafhankelijke structuur} rondom vortexknopen. Deze structuur vereist een hoge lokale dichtheid nabij de kern voor stabiliteit, en een ijle ætherop grote schaal om vrije propagatie van signalen (zoals licht) mogelijk te maken.

\subsection*{1. Kernregime}

De dichtheid in de kern benadert:
\begin{equation}
    \rho_\text{\ae}(r \to 0) \sim \SI{3.89e18}{kg/m^3},
\end{equation}
vereist om topologische stabiliteit van de vortexkern te garanderen. Deze waarde volgt uit energetische argumenten:
\begin{equation}
    E_{\text{vortex}} = \frac{1}{2} \rho_\text{\ae} \Omega^2 r_c^5 \quad\Rightarrow\quad \rho_\text{\ae} \sim \frac{2 E}{\Omega^2 r_c^5},
\end{equation}
waarbij \( \Omega = \frac{C_e}{r_c} \) de kernrotatie is, met \( C_e \approx \SI{1.094e6}{m/s} \) en \( r_c \approx \SI{1.409e-15}{m} \).

\subsection*{2. Overgangsregime}

Voor afstanden groter dan de kern, maar kleiner dan macroschaal, geldt een exponenti"ele afname:
\begin{equation}
    \rho_\text{\ae}(r) = \rho_\text{far} + (\rho_\text{core} - \rho_\text{far}) e^{-r/r_*},
\end{equation}
met \( r_* \sim \SI{1e-12}{m} \) de karakteristieke overgangsschaal. Deze waarde wordt gemotiveerd door het bereik van vortexinvloeden (zoals in EM-interacties).

\subsection*{3. Macroscopisch Regime}

Voor \( r \gg r_* \) bereikt \( \rho_\text{\ae} \) asymptotisch een constante waarde:
\begin{equation}
    \rho_\text{far} \sim \SI{1e-7}{kg/m^3},
\end{equation}
waardoor vrije voortplanting van signalen zonder merkbare inertie optreedt. Dit simuleert een vacu"umachtig gedrag.


\begin{figure}[htbp]
    \centering
    \includegraphics[width=0.85\textwidth]{00_scaleDependentÆtherDensity_nl}
    \caption{De ætherdichtheid neemt exponentieel af vanaf de vortexkern en benadert asymptotisch een constante waarde op macroschaal.}
    \label{fig:vortexfields2}
\end{figure}

\begin{table}[h!]
    \centering
    \begin{tabular}{|c|c|c|l|}
        \hline
        Regime & Afstand $r$ & $\rho_\text{\ae}(r)$ & Fysische interpretatie \\
        \hline
        Kern & $r < 10^{-14}$ m & $\sim 10^{18}$ kg/m$^3$ & Vortexstabiliteit \& inertie \\
        Overgang & $10^{-14} - 10^{-11}$ m & Exponentieel dalend & Swirl-uitdoving \& massa-interactie \\
        Macroscopisch & $r > 10^{-11}$ m & $\sim 10^{-7}$ kg/m$^3$ & Vrije ætherzonder massaweerstand \\
        \hline
    \end{tabular}
    \caption{Gedrag van de ætherdichtheid op verschillende schalen.}
\end{table}


\section{Tijddilatatie vanuit wervel dynamiek}

We beschouwen een onzichtbare, rotatievrije superfluïde æther met stabiele topologische wervelknopen. Absolute tijd $t_\text{abs}$ stroomt met een constante snelheid, terwijl lokale klokken mogelijk een lagere snelheid ervaren als gevolg van drukgradiënten en knoopenergetica. Het Vortex Æther Model veronderstelt dat de snelheid waarmee tijd in het lokale frame (dichtbij de knoop) stroomt, afhangt van de interne hoekfrequentie $\Omega_k$. In deze sectie leiden we tijddilatatie-analogen af, geïnspireerd door de voorspellingen van de algemene relativiteitstheorie (GR), uitsluitend gebaseerd op druk- en vorticiteitsgradiënten in de vloeistof.

\begin{figure}[htbp]
\centering
\includegraphics[width=0.85\textwidth]{01-streamlinesDiPole_nl}
\caption{Snelheid stroomlijnt, vorticiteit, druk en lokale tijdsnelheid $\tau$ voor een gesimuleerd wervelpaar. Het drukminimum en de tijdvertraging komen duidelijk overeen met de gebieden met hoge vorticiteit. Dit illustreert direct de centrale bewering van het æther model: tijddilatatie volgt uit wervelenergetica en drukvermindering.}
\label{fig:vortexfields}
\end{figure}

In het Vortex Æther Model (VAM) ontstaat tijddilatatie niet vanuit de kromming van ruimtetijd, maar vanuit lokale wervel dynamica. Elk materiedeeltje is in VAM een wervel-knoopstructuur waarvan de interne rotatie (\textit{swirl}) de lokale klokfrequentie beïnvloedt.

De fundamentele koppeling tussen lokale wervel-snelheid en de lokale tijdsmeting volgt uit de Bernoulli-achtige relatie voor drukverlaging in stromingsvelden. De lokale klokfrequentie is gerelateerd aan de wervel-tangentiële snelheid $v_{\phi}(r)$ via de formule:
\begin{equation}\label{eq:vortex_tijddilatatie}
    \frac{d\tau}{dt} = \sqrt{1 - \frac{v_{\phi}^2(r)}{c^2}}
\end{equation}

Hierbij is $v_{\phi}(r)$ de tangentiële snelheid van het æther medium op afstand $r$ tot het centrum van de wervel, en $c$ de lichtsnelheid. Dit is een directe analogie met de speciale relativistische snelheidsafhankelijke tijddilatatie, echter zonder ruimtetijdkromming en louter veroorzaakt door lokale rotatie van het æther medium.

Om het externe gedrag van tijdsdilatatie, zoals voorspeld door het heuristische vortex-geïnduceerde model, te visualiseren, breiden we het radiale domein uit tot macroscopische femtometerschalen. Dit onthult het asymptotische gedrag van herstel van de tijdssnelheid in het verre veld, wat de overeenkomst bevestigt met bekende gravitationele tijdsdilatatie-vervalprofielen.

\begin{figure}[H]
    \centering
    \includegraphics[width=0.7\textwidth]{06-HeuristicTimeDilation4_nl}
    \caption{
        Ingezoomd radiaal profiel van vortex-geïnduceerde tijddilatatie nabij de kern.
        Deze heuristische grafiek illustreert hoe de genormaliseerde lokale kloksnelheid
        $\frac{d\tau}{dt}$ snel toeneemt met de afstand $r$ tot de kern,
        en asymptotisch één nadert. Dit visualiseert direct het effect van
        de tangentiële vortexsnelheid $v_\varphi(r) \sim \kappa / r$ op de lokale tijdsverloop,
        zoals voorspeld door vergelijking~\eqref{eq:vortex_tijd_expliciet}.
    }
    \label{fig:HeuristicTimeDilation}
\end{figure}

\subsection{Afleiding vanuit wervel hydrodynamica}

De afleiding volgt uit het Bernoulli-principe voor een ideale vloeistofstroming, gegeven door:
\begin{equation}\label{eq:Bernoulli}
    P + \frac{1}{2}\rho_\text{\ae} v^2 = \text{constant}
\end{equation}

Met wervel-stroming geïntroduceerd via vorticiteit $\vec{\omega} = \nabla \times \vec{v}$, definieert de lokale drukverlaging ten opzichte van de verre omgeving een lokale tijdvertraging. De lokale wervelsnelheid is gegeven door:
\begin{equation}\label{eq:tangentiele_snelheid}
    v_{\phi}(r) = \frac{\Gamma}{2\pi r} = \frac{\kappa}{r}
\end{equation}

waarbij $\Gamma$ de circulatieconstante is, en $\kappa$ het circulatiekwantum. Substitutie van \eqref{eq:tangentiele_snelheid} in \eqref{eq:vortex_tijddilatatie} geeft expliciet:
\begin{equation}\label{eq:vortex_tijd_expliciet}
    \frac{d\tau}{dt} = \sqrt{1 - \frac{\kappa^2}{c^2 r^2}}
\end{equation}

Hiermee is de tijddilatatie expliciet uitgedrukt in fundamentele wervel-parameters.

\begin{figure}[H]
  \centering
  \includegraphics[width=0.7\textwidth]{02-RadialProfileOfLocalTimeDilation_Radial_LocalTime_Dilation_nl}
  \caption{Radiaal profiel van tijddilatatie als gevolg van wervelrotatie \( v_\varphi(r) = \kappa / r \). De lokale klokfrequentie neemt af met \(1/r^2\) en nadert asymptotisch 1 op grote afstand.}
  \label{fig:radiale_tijddilatatie}
\end{figure}


\subsection{Vergelijking met algemene relativiteit}

Ter vergelijking, in algemene relativiteit (GR) ontstaat gravitationele tijddilatatie uit ruimtetijdkromming, uitgedrukt door de Schwarzschildmetriek~\cite{schutz2009first}:
\begin{equation}\label{eq:GRtijd}
    \frac{d\tau}{dt} = \sqrt{1 - \frac{2GM}{rc^2}}
\end{equation}


De overeenkomsten en verschillen zijn direct zichtbaar: GR's gravitationele tijddilatatie is gerelateerd aan massa $M$ en gravitatieconstante $G$, terwijl VAM tijddilatatie puur hydrodynamisch is en direct verbonden met de lokale rotatiesnelheid van het æther medium via wervel-circulatie $\kappa$.

\begin{figure}[ht!]
    \centering
    \includegraphics[width=0.7\linewidth]{02-RadialProfileOfLocalTimeDilation_Vortex-Induced_Time_Well_nl}
    \caption{Vergelijking tussen VAM- (vortex dynamiek) en GR-tijddilatatie, als functie van afstand tot wervelkern en Schwarzschildradius.}
    \label{fig:vergelijking_VAMGR}
\end{figure}

In Figuur~\ref{fig:vergelijking_VAMGR} zien we dat de VAM-tijddilatatie functioneel vergelijkbaar is met GR-prediction bij voldoende afstand. Bij afnemende afstand (nabij wervelkern of Schwarzschildradius) ontstaan verschillen door wervel-specifieke effecten en topologische knoopstructuren.

Samenvattend vervangt het VAM ruimtetijdkromming door werveldynamica, met behoud van meetbare tijddilatatie-effecten die overeenstemmen met gevestigde experimentele resultaten zoals Hafele–Keating~\cite{hafele1972around}, maar vanuit een fundamenteel andere fysische verklaring.


Ter illustratie vergelijken we in Figuur~\ref{fig:vergelijkingVAMGR} VAM en GR expliciet voor een neutronenster met $M = 2\,M_\odot$ en radius $R = 10\,\text{km}$. De verschillen worden duidelijk nabij de oppervlakte van het object, waar wervel-specifieke effecten optreden.

\begin{figure}[ht!]
    \centering
    \includegraphics[width=0.7\linewidth]{04-RotationalVsHeuristicTimeDilation_nl}
    \caption{Verschil tussen VAM en GR-tijddilatatie voor een neutronenster ($2\,M_\odot$, $R=10$ km).}
    \label{fig:vergelijkingVAMGR}
\end{figure}


\subsection{Praktische implicaties en experimentele toetsbaarheid}

Een praktische implicatie van wervel-geïnduceerde tijddilatatie is dat klokken dicht bij intense wervelvelden meetbaar trager zouden lopen. Dit kan theoretisch getoetst worden met ultra-precieze atoomklokken in laboratorium wervelexperimenten, of indirect via astrofysische observaties van pulsars en neutronensterren. Het Hafele–Keating experiment biedt een directe analogie voor tijddilatatie door beweging en hoogteverschillen, die in VAM overeenkomt met lokale wervelvariaties~\cite{hafele1972around}.

\begin{figure}[ht!]
    \centering
    \includegraphics[width=0.7\linewidth]{05-LogarithmicDecayLocalTime_nl}
    \caption{Uitgebreid radiaal profiel van tijddilatatie met $\Omega_k \propto 1/r^2$, dat de diepe tijdsput-karakteristieken van wervelvelden op grote straal toont.}
    \label{fig:NewGraph}
\end{figure}

\section{Entropie en quantum-effecten in het Vortex Æther Model}

Het Vortex Æther Model (VAM) biedt een mechanistische basis voor zowel thermodynamische als kwantummechanische fenomenen, niet door postulaten over abstracte toestandsruimten, maar via de dynamica van knopen en wervels in een superfluïde æther. Twee centrale begrippen—entropie en kwantisatie—worden in VAM afgeleid uit respectievelijk vorticiteitverdeling en knottopologie.

\subsection{Entropie als vorticiteit-verdeling}

In thermodynamica is entropie $S$ een maat voor de interne energieverdeling of wanorde. In VAM ontstaat entropie niet als statistisch fenomeen, maar uit ruimtelijke variaties in werveling (vorticiteit). Voor een wervelconfiguratie $V$ wordt de entropie gegeven door:

\begin{equation}
S \propto \int_V \|\vec{\omega}\|^2 \, dV,
\end{equation}

waar $\vec{\omega} = \nabla \times \vec{v}$ de lokale vorticiteit is. Dit betekent:

\begin{itemize}
    \item \textbf{Meer rotatie = meer entropie}: Regio's met sterke swirl dragen bij aan verhoogde entropie.
    \item \textbf{Thermodynamisch gedrag ontstaat uit werveluitzetting}: Bij toevoer van energie (warmte), zet de wervelgrens uit, de swirl neemt af en $S$ stijgt—analogie met gasexpansie.
\end{itemize}

Deze interpretatie verbindt Clausius’ warmtetheorie met æthermechanica: warmte is equivalent aan verhoogde swirlverspreiding.

\subsection{Quantumgedrag uit knotted wervelstructuren}

Kwantumverschijnselen zoals discrete energieniveaus, spin, en golf-deeltje-dualiteit vinden in VAM hun oorsprong in topologisch geconserveerde wervelknopen:

\begin{itemize}
    \item \textbf{Circulatiequantisatie:}
    \begin{equation}
    \Gamma = \oint \vec{v} \cdot d\vec{l} = n \cdot \kappa,
    \end{equation}
    waarbij $\kappa = h/m$ en $n \in \mathbb{Z}$ het windinggetal is.

    \item \textbf{Hele getallen ontstaan uit knottopologie:} De helixstructuur van een wervelknoop (zoals een trefoil) zorgt voor discrete toestanden met bepaalde linking numbers $L_k$.

    \item \textbf{Heliciteit als spin-analoog:}
    \begin{equation}
    H = \int \vec{v} \cdot \vec{\omega} \, dV,
    \end{equation}
    waarbij $H$ invariant is onder ideale stroming, net zoals spin geconserveerd is in quantummechanica.
\end{itemize}

\subsection{VAM-interpretatie van kwantisatie en dualiteit}

In plaats van abstracte Hilbertruimten beschouwt VAM een deeltje als een stabiele knoop in het ætherveld. Deze vortexconfiguratie bezit:

\begin{itemize}
    \item Een \textbf{kern} (knooplichaam) met quantumsprongen (resonanties).
    \item Een \textbf{uiterlijk veld} dat als golf fungeert (zoals de Schrödinger-golf).
    \item Een \textbf{heliciteit} die gedraagt als interne vrijheidsgraden (bijv. spin).
\end{itemize}

Het golf-deeltje-dualisme komt zo voort uit het feit dat knopen zowel gelokaliseerd (kern) als uitgesmeerd (veld) zijn.

\subsection{Samenvattend}

VAM biedt dus een coherente, vloeistofmechanische oorsprong voor zowel:

\begin{enumerate}
    \item \textbf{Thermodynamica:} Entropie ontstaat uit swirlverdeling.
    \item \textbf{Quantummechanica:} Kwantisatie en dualiteit zijn emergente eigenschappen van knotted vortex topologieën.
\end{enumerate}

Deze benadering laat zien dat kwantum- en thermodynamische fenomenen niet fundamenteel verschillend zijn, maar voortkomen uit hetzelfde wervelmechanisme op verschillende schalen.


\section{Tijdsmodulatie door rotatie van wervelknopen}

Voortbouwend op de behandeling van tijdsdilatatie via druk en Bernoulli-dynamica in de vorige sectie, richten we ons nu op de intrinsieke rotatie van topologische wervelknopen. In het Vortex Æther Model (VAM) worden deeltjes gemodelleerd als stabiele, topologisch behouden wervelknopen ingebed in een onsamendrukbaar, niet-viskeus superfluïde medium. Elke knoop bezit een karakteristieke interne hoekfrequentie $\Omega_k$, en deze interne beweging induceert lokale tijdmodulatie ten opzichte van de absolute tijd van de æther.

In plaats van het krommen van de ruimtetijd, stellen we voor dat interne rotatie-energie en heliciditeitsbehoud temporele vertragingen veroorzaken die analoog zijn aan gravitationele roodverschuiving. In deze sectie worden deze ideeën uitgewerkt met behulp van heuristische en energetische argumenten die consistent zijn met de hiërarchie die in Sectie I is geïntroduceerd.

\subsection{Heuristische en energetische afleiding}

We beginnen met het voorstellen van een rotatiegeïnduceerde tijdsdilatatieformule op basis van de interne hoekfrequentie van de knoop:

\begin{equation}
\frac{t_{\text{local}}}{t_{\text{abs}}} = \left(1 + \beta \Omega_k^2 \right)^{-1}\label{eq:rotational_induced_time_dilation}
\end{equation}

waarbij:

\begin{itemize}
\item $t_{\text{local}}$ de eigentijd nabij de knoop is,
\item $t_{\text{abs}}$ de absolute tijd van de achtergrond-æther is,
\item $\Omega_k$ de gemiddelde kernhoekfrequentie is frequentie,
\item $\beta$ is een koppelingscoëfficiënt met dimensies $[\beta] = \text{s}^2$.
\end{itemize}

Voor kleine hoeksnelheden verkrijgen we een eerste-orde-expansie:

\begin{equation}
\frac{t_{\text{local}}}{t_{\text{abs}}} \approx 1 - \beta \Omega_k^2 + \mathcal{O}(\Omega_k^4)\label{eq:rotational_induced_time_dilation_expansion}
\end{equation}

Deze vorm loopt parallel met de Lorentzfactor bij lage snelheden in de speciale relativiteitstheorie:

\begin{equation}
\frac{t_{\text{moving}}}{t_{\text{rest}}} \approx 1 - \frac{v^2}{2c^2}\label{eq:parallels_lorentz_time_dilation}
\end{equation}

Dit levert een belangrijke analogie op: Interne rotatiebeweging in VAM induceert tijdvertraging, vergelijkbaar met hoe translationele snelheid tijddilatatie induceert in SR.

Om de fysische basis van deze uitdrukking te versterken, relateren we tijddilatatie nu aan de energie die is opgeslagen in wervelrotatie. Stel dat de wervelknoop een effectief traagheidsmoment $I$ heeft. De rotatie-energie wordt gegeven door:

\begin{equation}
E_{\text{rot}} = \frac{1}{2} I \Omega_k^2\label{eq:rotational_energy_inertia}
\end{equation}

Aannemende dat de tijd vertraagt door deze energiedichtheid, schrijven we:

\begin{equation}
\frac{t_{\text{local}}}{t_{\text{abs}}} = \left(1 + \beta E_{\text{rot}} \right)^{-1} = \left(1 + \frac{1}{2} \beta I \Omega_k^2 \right)^{-1}\label{eq:time_dilation_rotational_energy_inertia}
\end{equation}

Deze uitdrukking dient als de energetische analoog van het op druk gebaseerde Bernoulli-model uit Sectie I (zie vgl. ~\eqref{eq:localtime_vortex}). Het ondersteunt de interpretatie van vortex-geïnduceerde tijdsputten via energieopslag in plaats van geometrische deformatie.

\subsection{Topologische en fysische rechtvaardiging}

Topologische vortexknopen worden niet alleen gekenmerkt door rotatie, maar ook door heliciteit:

\begin{equation}
H = \int \vec{v} \cdot \vec{\omega} \, d^3x \label{eq:helicity_rotation}
\end{equation}

Heliciteit is een behouden grootheid in ideale (onzichtbare, onsamendrukbare) vloeistoffen, die de verbinding en draaiing van vortexlijnen codeert. De rotatiefrequentie $\Omega_k$ wordt een topologisch betekenisvolle indicator van de identiteit en dynamische toestand van de knoop.

Hogere Omega_k-waarden duiden op meer rotatie-energie en diepere drukputten, wat leidt tot tijdelijke vertragingen die lijken op gravitationele roodverschuiving, maar zonder dat er sprake is van ruimtetijdkromming.

Elk deeltje is een topologische vortexknoop:
\begin{itemize}
\item Lading $\leftrightarrow$ draaiing of chiraliteit van de knoop
\item Massa $\leftrightarrow$ geïntegreerde vorticiteitsenergie
\item Spin $\leftrightarrow$ knoophelix:
\end{itemize}
Stabiliteit $\leftrightarrow$ knooptype (Hopf-verbindingen, Trefoil, enz.) en energieminimalisatie in de vortexkern

Dit model:

\begin{itemize}
\item Schrijft tijdmodulatie toe aan behouden, intrinsieke rotatie-energie,
\item Vereist geen externe referentiekaders (absolute æthertijd is universeel),
\item Behoudt temporele isotropie buiten de vortexkern,
\item Biedt een natuurlijke vervanging voor de ruimtetijdkromming van GR. \end{itemize}

Daarom biedt dit vortex-energetische tijdsdilatatieprincipe een krachtig alternatief voor relativistische tijdmodulatie door alle temporele effecten te verankeren in rotatie-energetica en topologische invarianten.

In de volgende sectie zullen we laten zien hoe deze ideeën metriekachtig gedrag reproduceren voor roterende waarnemers, inclusief een directe vloeistofmechanische analoog aan de Kerr-metriek van de algemene relativiteitstheorie.
\section{Eigen tijd voor een roterende waarnemer in ætherstroming}

Nu we tijdsdilatatie hebben vastgesteld in het Vortex Æther Model (VAM) door middel van druk, hoeksnelheid en rotatie-energie, breiden we ons formalisme nu uit naar roterende waarnemers. Deze sectie toont aan dat vloeistofdynamische tijdmodulatie in VAM uitdrukkingen kan reproduceren die structureel vergelijkbaar zijn met die afgeleid uit de algemene relativiteitstheorie (GR), met name in axiaal symmetrische roterende ruimtetijden zoals de Kerr-geometrie. VAM bereikt dit echter zonder ruimtetijdkromming aan te roepen. In plaats daarvan wordt tijdmodulatie volledig bepaald door kinetische variabelen in het ætherveld.

\subsection{GR-propertijd in roterende frames}

In de algemene relativiteitstheorie wordt de eigentijd \(d\tau\) voor een waarnemer met hoeksnelheid \(\Omega_{\text{eff}}\) in een stationaire, axiaal symmetrische ruimtetijd gegeven door:

\begin{equation}
\left( \frac{d\tau}{dt} \right)^2_{\text{GR}} = -\left[ g_{tt} + 2g_{t\varphi} \Omega_{\text{eff}} + g_{\varphi\varphi} \Omega_{\text{eff}}^2 \right]
\label{eq:GR_proper_time}
\end{equation}

waarbij \(g_{\mu\nu}\) componenten zijn van de ruimtetijdmetriek (bijv. in Boyer-Lindquist coördinaten voor Kerr-ruimtetijd). Deze formulering houdt rekening met zowel gravitationele roodverschuiving als rotatie-effecten (frame-dragging).

\subsection{Æther-gebaseerde analogie: Snelheidsafgeleide tijdmodulatie}

In VAM is de ruimtetijd niet gekromd. Waarnemers bevinden zich in plaats daarvan in een dynamisch gestructureerde æther waarvan de lokale stroomsnelheden de tijddilatatie bepalen. Laat de radiale en tangentiële componenten van de æthersnelheid zijn:

\begin{itemize}
\item \(v_r\): radiale snelheid,
\item \(v_\varphi = r\Omega_k\): tangentiële snelheid als gevolg van lokale wervelrotatie,
\item \(\Omega_k = \frac{\kappa}{2\pi r^2}\): lokale hoeksnelheid (met \(\kappa\) als circulatie).
\end{itemize}

We postuleren een correspondentie tussen GR-metrische componenten en andere snelheidstermen:

\begin{equation}
\begin{aligned}
g_{tt} &\rightarrow -\left(1 - \frac{v_r^2}{c^2}\right), \\
g_{t\varphi} &\rightarrow -\frac{v_r v_\varphi}{c^2}, \\
g_{\varphi\varphi} &\rightarrow -\frac{v_\varphi^2}{c^2 r^2}
\end{aligned}
\label{eq:VAM_metric_terms}
\end{equation}

Door deze in de GR-expressie voor de juiste tijd te substitueren, verkrijgen we de VAM-gebaseerde analoog:

\begin{equation}
\left( \frac{d\tau}{dt} \right)^2_{\text{\ae}} = 1 - \frac{v_r^2}{c^2} - \frac{2v_r v_\varphi}{c^2} - \frac{v_\varphi^2}{c^2}
\label{eq:VAM_proper_time}
\end{equation}

De termen combineren:

\begin{equation}
\left( \frac{d\tau}{dt} \right)^2_{\text{\ae}} = 1 - \frac{1}{c^2}(v_r + v_\varphi)^2
\label{eq:VAM_proper_time_combined}
\end{equation}

Deze formulering reproduceert gravitationele en frame-dragging tijdseffecten puur uit de ætherdynamica: $\langle \omega^2 \rangle$ speelt de rol van gravitationele roodverschuiving en circulatie $\kappa$ codeert rotatieweerstand. Deze benadering sluit aan bij recente vloeistofdynamische interpretaties van zwaartekracht en tijd \cite{barcelo2011analogue}, \cite{fedi2017gravity}.
Dit model gaat momenteel uit van rotatievrije stroming buiten knopen en verwaarloost viscositeit, turbulentie en kwantumcompressibiliteit. Toekomstige uitbreidingen kunnen gekwantiseerde circulatiespectra of grenseffecten in beperkte æthersystemen omvatten.

\begin{equation}
\boxed{\left( \frac{d\tau}{dt} \right)^2_{\text{\ae}} = 1 - \frac{1}{c^2}(v_r + r\Omega_k)^2}
\label{eq:VAM_proper_time_final}
\end{equation}

\subsection{Fysische interpretatie en modelconsistentie}

Dit resultaat in het kader weerspiegelt de GR-uitdrukking voor roterende waarnemers, maar komt strikt voort uit de klassieke vloeistofdynamica. Het laat zien dat naarmate de lokale æthersnelheid de lichtsnelheid nadert – door radiale instroom of rotatiebeweging – de eigentijd vertraagt. Dit impliceert het bestaan van "tijdputten" waar de kinetische energiedichtheid domineert.

Belangrijkste observaties:

\begin{itemize}
\item Bij afwezigheid van radiale stroming (\(v_r = 0\)) ontstaat tijdvertraging volledig door wervelrotatie.
\item Wanneer zowel \(v_r\) als \(\Omega_k\) aanwezig zijn, verlaagt de cumulatieve snelheid de lokale tijdssnelheid.
\item Deze uitdrukking komt overeen met het energetische model van Sectie II als we \(v_r + r\Omega_k\) interpreteren als een bijdrage aan de lokale energiedichtheid.
\end{itemize}

In het VAM-kader komt de structuur van de eigentijd van de waarnemer dus voort uit ætherische stromingsvelden. Dit bevestigt dat GR-achtig temporeel gedrag kan ontstaan in een vlakke, Euclidische 3D-ruimte met absolute tijd, volledig bepaald door gestructureerde vorticiteit en circulatie.

In het volgende gedeelte onderzoeken we hoe VAM deze overeenkomst uitbreidt naar gravitatiepotentialen en frame-dragging-effecten via circulatie en vorticiteitsintensiteit, en zo een analogie vormt voor de Kerr-tijd-roodverschuivingsformule.
\section{Kerr-achtige tijdsaanpassing op basis van vorticiteit en circulatie}

Om de analogie tussen de algemene relativiteitstheorie (GR) en het wervel-æthermodel (VAM) te voltooien, leiden we nu een tijdmodulatieformule af die de roodverschuiving en frame-dragging-structuur in de Kerr-oplossing weerspiegelt. In GR beschrijft de Kerr-metriek de ruimtetijdgeometrie rond een roterende massa en voorspelt zowel gravitationele tijddilatatie als frame-dragging als gevolg van impulsmoment. VAM legt vergelijkbare verschijnselen vast via de dynamiek van gestructureerde vorticiteit en circulatie in de æther, zonder dat ruimtetijdkromming nodig is.

\subsection{Algemene relativistische Kerr-roodverschuivingsstructuur}

In de GR-Kerr-metriek wordt de eigentijd $d\tau$ voor een waarnemer nabij een roterende massa beïnvloed door zowel massa-energie als impulsmoment. Een vereenvoudigde benadering voor de tijddilatatiefactor nabij een roterend lichaam is:
\begin{equation}
t_\text{adjusted} = \Delta t \cdot \sqrt{1 - \frac{2GM}{rc^2} - \frac{J^2}{r^3c^2}}
\label{eq:Kerr_time_dilation}
\end{equation}
waarbij:
\begin{itemize}
\item $M$: massa van het roterende lichaam,
\item $J$: impulsmoment,
\item $r$: radiale afstand tot de bron,
\item $G$: gravitatieconstante van Newton,
\item $c$: lichtsnelheid.
\end{itemize}

De eerste term correspondeert met gravitationele roodverschuiving ten opzichte van de massa, terwijl de tweede rekening houdt met rotatie-effecten (frame-dragging).

\subsection{Æther analoog via vorticiteit en circulatie}

In VAM drukken we gravitatieachtige invloeden uit via vorticiteitsintensiteit $\langle \omega^2 \rangle$ en totale circulatie $\kappa$. Deze worden geïnterpreteerd als:
\begin{itemize}
\item $\langle \omega^2 \rangle$: gemiddelde kwadratische vorticiteit over een gebied,
\item $\kappa$: behouden circulatie, coderend voor impulsmoment.
\end{itemize}

We definiëren de æthergebaseerde analoog door de volgende vervangingen uit te voeren:
\begin{equation}
\begin{aligned}
\frac{2GM}{rc^2} &\rightarrow \frac{\gamma \langle \omega^2 \rangle}{rc^2}, \\
\frac{J^2}{r^3c^2} &\rightarrow \frac{\kappa^2}{r^3c^2}
\end{aligned}
\label{eq:Kerr_replacements}
\end{equation}

Hier is $\gamma$ een koppelingsconstante die de vorticiteit relateert aan de effectieve zwaartekracht (analoog aan $G$). De op æther gebaseerde eigentijd wordt dan:

\begin{equation}
\boxed{t_\text{adjusted} = \Delta t \cdot \sqrt{1 - \frac{\gamma \langle \omega^2 \rangle}{rc^2} - \frac{\kappa^2}{r^3c^2}}}
\label{eq:Kerr_time_dilation_ae}
\end{equation}

Dit weerspiegelt de Kerr-roodverschuiving en frame-dragging-structuur met behulp van vloeistofdynamische variabelen. In deze afbeelding:
\begin{itemize}
\item $\langle \omega^2 \rangle$ speelt de rol van energiedichtheid die gravitationele roodverschuiving produceert,
\item $\kappa$ vertegenwoordigt impulsmoment dat tijdelijke frame-dragging genereert,
\item De vergelijking reduceert tot een vlakke æthertijd ($t_\text{aangepast} \to \Delta t$) wanneer beide termen verdwijnen.
\end{itemize}


\subsection*{Hybride VAM Frame-Dragging Hoeksnelheid}


In het Vortex Æther Model (VAM) wordt de frame-dragging hoeksnelheid, geïnduceerd door een roterend wervelgebonden object, analoog gedefinieerd aan het Lense-Thirring effect in de algemene relativiteitstheorie, maar met een schaalafhankelijke Koppeling:

\begin{equation}
    \omega_\text{drag}^\text{VAM}(r) =
    \frac{4 G m}{5 c^2 r} \cdot \mu(r) \cdot \Omega(r)
\end{equation}

Hierbij is \( G \) de gravitatieconstante, \( c \) de lichtsnelheid, \( m \) de massa van het object, \( r \) de karakteristieke straal en \( \Omega(r) \) de hoeksnelheid.

De hybride koppelingsfactor \( \mu(r) \) interpoleert tussen wervelgedrag op kwantumschaal en klassieke macroscopische rotatie:

\begin{equation}
    \mu(r) =
    \begin{cases}
        \displaystyle \frac{r_c C_e}{r^2}, & \text{if } r < r_\ast \quad \text{(kwantum- of wervelkernregime)} \\
        1, & \text{if } r \geq r_\ast \quad \text{(macroscopisch regime)}
    \end{cases}
\end{equation}

waarbij:
\begin{itemize}
    \item \( r_c \) de straal van de wervelkern is,
    \item \( C_e \) de tangentiële snelheid van de wervelkern is,
    \item \( r_\ast \sim 10^{-3} \, \text{m} \) de overgangsradius tussen microscopische en macroscopische regimes is.
\end{itemize}

Deze formulering zorgt voor continuïteit met GR-voorspellingen voor hemellichamen, terwijl VAM-specifieke voorspellingen voor elementaire deeltjes en subatomaire wervelstructuren mogelijk worden.


\subsection*{VAM Gravitationele Roodverschuiving vanuit Kernrotatie}

In het Vortex Æther Model (VAM) ontstaat gravitationele roodverschuiving door de lokale rotatiesnelheid \( v_\phi \) aan de buitengrens van een wervelknoop. Uitgaande van geen ruimtetijdkromming en absolute tijd, wordt de effectieve gravitationele roodverschuiving gegeven door:

\begin{equation}
    z_\text{VAM} =
    \left( 1 - \frac{v_\phi^2}{c^2} \right)^{-\frac{1}{2}} - 1
\end{equation}

waarbij:
\begin{itemize}
    \item \( v_\phi = \Omega(r) \cdot r \) de tangentiële snelheid is ten gevolge van lokale rotatie,
    \item \( \Omega(r) \) de hoeksnelheid is bij de meetstraal \( r \),
    \item \( c \) de lichtsnelheid in vacuüm is.
\end{itemize}

Deze uitdrukking weerspiegelt de verandering van de tijdsperceptie veroorzaakt door lokale rotatie-energie, waarbij de op kromming gebaseerde gravitatiepotentiaal \( \Phi \) van de algemene relativiteitstheorie wordt vervangen door een snelheidsveldterm. Deze wordt equivalent aan de GR Schwarzschild-roodverschuiving voor lage \( v_\phi \) en divergeert als \( v_\phi \rightarrow c \), wat een natuurlijke grens vormt voor de evolutie van het lokale frame:

\begin{equation}
    \lim_{v_\phi \to c} z_\text{VAM} \to \infty
\end{equation}

\subsection*{VAM Lokale Tijddilatatie Modellen}

In het Vortex Æther Model (VAM) wordt lokale tijddilatatie geïnterpreteerd als de modulatie van absolute tijd door interne werveldynamica, niet door ruimtetijdkromming. Afhankelijk van de systeemschaal worden twee fysisch gefundeerde formuleringen gebruikt:

\paragraph{1. Tijddilatatie op basis van snelheidsvelden}

Dit model relateert de lokale tijdstroom aan de tangentiële snelheid van de roterende ætherische structuur (vortexknoop, planeet of ster):

\begin{equation}
    \frac{d\tau}{dt} =
    \sqrt{1 - \frac{v_\phi^2}{c^2}} =
    \sqrt{1 - \frac{\Omega^2 r^2}{c^2}}
\end{equation}

waarbij:
\begin{itemize}
    \item \( v_\phi = \Omega \cdot r \) de tangentiële snelheid is,
    \item \( \Omega \) de hoeksnelheid bij straal \( r \) is,
    \item \( c \) de lichtsnelheid is.
\end{itemize}

\paragraph{2. Tijddilatatie op basis van rotatie-energie}

Op grote schaal of met hoge rotatietraagheid ontstaat tijddilatatie door opgeslagen rotatie-energie, wat leidt tot:

\begin{equation}
    \frac{d\tau}{dt} =
    \left(1 + \frac{1}{2} \cdot \beta \cdot I \cdot \Omega^2 \right)^{-1}
\end{equation}

met:
\begin{itemize}
    \item \( I = \frac{2}{5} m r^2 \): traagheidsmoment voor een uniforme bol,
    \item \( \beta = \frac{r_c^2}{C_e^2} \): koppelingsconstante van wervel-kerndynamica,
    \item \( m \) is de massa van het object. \end{itemize}

\paragraph{Interpretatie}

Deze modellen impliceren dat de tijd vertraagt in gebieden met een hoge lokale rotatie-energie of vorticiteit, in overeenstemming met gravitationele tijddilatatie-effecten in GR. In VAM ontstaan deze effecten echter uitsluitend door de interne dynamiek van de ætherstroming, onder vlakke 3D Euclidische meetkunde en absolute tijd.


\subsection{Modelaannames en reikwijdte}

Dit resultaat is afhankelijk van verschillende aannames:
\begin{itemize}
\item De stroming is rotatievrij buiten de wervelkernen,
\item Viscositeit en turbulentie worden verwaarloosd,
\item Samendrukbaarheid wordt genegeerd (ideale onsamendrukbare superfluïde),
\item Vorticiteitsvelden zijn voldoende glad om $\langle \omega^2 \rangle$ te definiëren.
\end{itemize}

Deze omstandigheden weerspiegelen de aannames van analoge modellen van ideale vloeistof-GR. De formulering overbrugt de macroscopische stromingsdynamica van de æther met effectieve geometrische voorspellingen, wat de mogelijkheid versterkt om gekromde ruimtetijd te vervangen door gestructureerde vorticiteitsvelden.

Zie Appendix 7~\ref{sec:appendix_7} voor gedetailleerde afleidingen van kruisenergie- en wervelinteractie-energetica.

In toekomstig werk kunnen correcties voor randvoorwaarden, gekwantiseerde vorticiteitsspectra en compressibele effecten worden toegevoegd om de analogie te verfijnen. Vervolgens vatten we samen hoe deze vloeistofgebaseerde tijddilatatiemechanismen zich verenigen binnen het VAM-kader en identificeren we hun experimentele implicaties.
\section{Unified Framework en Synthese van Tijdsdilatatie in VAM}

Deze sectie verenigt de tijdsdilatatiemechanismen die in het artikel worden besproken onder het Vortex-Æther Model (VAM). In plaats van te vertrouwen op ruimtetijdkromming, schrijft VAM temporele effecten toe aan klassieke vloeistofdynamica, rotatie-energie en topologische vorticiteit.

\subsection{Hiërarchische Structuur van Tijdsdilatatiemechanismen}

Elk deel van dit werk draagt ​​een afzonderlijk maar onderling gerelateerd mechanisme voor tijdsdilatatie bij:

\begin{enumerate}
\item \textbf{Bernoulli-Geïnduceerde Tijdsdepletie:} Tijd vertraagt ​​in de buurt van gebieden met lage druk als gevolg van vortex-geïnduceerde kinetische snelheidsvelden. Dit resulteert in een speciale relativistische tijdsdilatatievorm wanneer \( \rho_{\text{\ae}} / p_0 \sim 1/c^2 \).
\item \textbf{Heuristisch model voor hoekfrequentie:} Een kwadratische afhankelijkheid van de tijdsnelheid van de lokale knoophoekfrequentie \( \Omega_k^2 \), die de Lorentz-factorexpansie voor kleine snelheden nabootst.
\item \textbf{Energetische formulering via rotatietraagheid:}
\[
\boxed{\frac{t_{\text{local}}}{t_{\text{abs}}} = \left(1 + \frac{1}{2} \beta I \Omega_k^2 \right)^{-1}}
\]
koppelt tijdmodulatie direct aan de rotatie-energie van vortexknopen. \item \textbf{Eigen tijdstroom gebaseerd op snelheidsveld:}
\[
\boxed{\left( \frac{d\tau}{dt} \right)^2 = 1 - \frac{1}{c^2}(v_r + r\Omega_k)^2}
\]
\item \textbf{Kerr-achtige roodverschuiving en frame-drag:}
\[
\boxed{t_{\text{aangepast}} = \Delta t \cdot \sqrt{1 - \frac{\gamma \langle \omega^2 \rangle}{rc^2} - \frac{\kappa^2}{r^3c^2}}}
\]
\end{enumerate}

Deze vijf expressies vormen een zelfconsistente ladder, gaande van heuristisch tot rigoureus, en vormen een Robuuste vervanging voor algemeen relativistische tijdsdilatatie, volledig gebaseerd op klassieke veldvariabelen.

\subsection{Fysische unificatie: Tijd als een van vorticiteit afgeleide waarneembare variabele}

In alle formuleringen komt een terugkerend thema naar voren: \textit{tijdmodulatie in VAM is altijd reduceerbaar tot lokale kinetische of rotatie-energiedichtheid binnen de ether}. Of deze nu gecodeerd is in druk (Bernoulli), hoekfrequentie (\( \Omega_k \)) of veldcirculatie (\( \kappa \)), de modulatie van tijd is niet geometrisch maar energetisch en topologisch.

\begin{itemize}
\item Lokale tijdputten ontstaan ​​door hoge vorticiteit en circulatie.
\item Frame-onafhankelijkheid: Absolute tijd bestaat; alleen lokale snelheden worden beïnvloed.
\item Geen noodzaak voor tensorgeometrie: Alle tijdseffecten ontstaan ​​door scalaire of vectorvelden.
Topologisch behoud: Vortexknopen behouden heliciteit en circulatie, wat zorgt voor temporele consistentie.

Deze unificatie versterkt de conceptuele kern van VAM: ruimtetijdkromming is een opkomende illusie die wordt veroorzaakt door gestructureerde vorticiteit in een absolute, superfluïde ether.

Experimentele implicaties en vooruitzichten

Elke hier geïntroduceerde tijdsdilatatieformule kan in principe worden getest in analoge laboratoriumsystemen:

Roterende superfluïde druppels (bijv. helium-II, BEC's)
Elektrohydrodynamische lifters en plasmavortexsystemen
Magnetofluïdische en optische analogen

Toekomstig werk omvat:
Totstandkoming van items
Het afleiden van dynamische vergelijkingen voor temporele feedback in systemen met meerdere knopen. \item Het meten van werveling-geïnduceerde klokdrift in roterende superfluïda.
\item Het toepassen van het model op astrofysische observaties (bijv. precessie van neutronensterren, frame dragging, tijdsvertraging).
\end{itemize}

\subsection{Uitdagingen, beperkingen en paden naar bredere relevantie}

\textbf{Fundamentele aannames:} De herintroductie van een ether met absolute tijd vormt een uitdaging voor een eeuw relativistische fysica.

\textbf{Experimentele validatie:} Er is nog geen direct empirisch bewijs dat de voorgestelde ether of specifieke dilatatiemechanismen ondersteunt.

\textbf{Ontvangst in de mainstream natuurkunde:} Hoewel nichegemeenschappen zich kunnen inzetten, kan de mainstream natuurkunde weerstand bieden vanwege afwijkingen van gevestigde kaders.

\subsection{Versterking van wetenschappelijke nauwkeurigheid en bredere aantrekkingskracht}

\begin{itemize}
\item \textbf{Stel testbare voorspellingen voor:} vooral waar VAM afwijkt van GR.
\item \textbf{Integreer met gevestigde theorieën:} toon grensgevallen die overeenkomen met GR/QM. \item \textbf{Historische bezwaren aanpakken:} herdefinieer æther duidelijk met moderne beperkingen.
\item \textbf{Peer Review en samenwerking:} nodigen uit tot kritiek van specialisten.
\item \textbf{Helderheid en toegankelijkheid:} vereenvoudigen de conceptuele presentatie zonder in te boeten aan nauwkeurigheid.
\end{itemize}

\subsection{Afsluitend perspectief}

Het Vortex Æther Model (VAM) biedt een gedurfde herinterpretatie van gravitationele tijdsdilatatie als gevolg van vorticiteitsgestuurde energetica in een absoluut, superfluïde medium. Door een hiërarchie van afleidingen – die Bernoulli-stromingen, vortexrotatie, energiedichtheid en circulatie omvatten – biedt het een coherent alternatief voor relativistische, op kromming gebaseerde beschrijvingen. Hoewel VAM afwijkt van conventionele theorieën, rechtvaardigen de interne logica en conceptuele helderheid ervan verder onderzoek. Voortdurende verfijning, integratie en empirische testen zullen bepalen welke rol de technologie zal spelen bij het verder verdiepen van ons begrip van de zwaartekracht, de tijd en de structuur van het heelal.
\section{Toepassingen van VAM op kwantum- en kernprocessen}
\label{sec:LENR_QED}
\subsection*{LENR via resonantietunneling}

Zwaartekrachtsverval door vorticiteit verlaagt tijdelijk de Coulomb-barrière:

\begin{equation}
    V_{\text{Coulomb}} = \frac{Z_1 Z_2 e^2}{4\pi \varepsilon_0 r}, \quad \Delta P = \frac{1}{2} \rho_{\text{\ae}} r_c^2 (\Omega_1^2 + \Omega_2^2)
\end{equation}

Resonantie treedt op wanneer:

\begin{equation}
    \Delta P \geq \frac{Z_1 Z_2 e^2}{4\pi \varepsilon_0 r_t^2}
\end{equation}

\subsection*{Resonante Ætherische tunneling en LENR in VAM}

In het Vortex Æther Model (VAM) worden laagenergetische kernreacties (LENR) geherinterpreteerd als resonante tunnelinggebeurtenissen die worden gemedieerd door gestructureerde wervelinteracties in de Æther. In tegenstelling tot conventionele kwantumtunneling, die afhankelijk is van deeltjesgolffuncties die een statische Coulomb-potentiaalbarrière penetreren, stelt VAM dat lokale drukminima – voortkomend uit wervel-geïnduceerde Bernoulli-tekorten – de barrière tijdelijk kunnen verminderen of volledig kunnen elimineren~\cite{Barcelo2011,Volovik2003}.

De klassieke Coulomb-afstoting tussen twee kernen van ladingen \( Z_1 e \) en \( Z_2 e \) wordt gegeven door:
\begin{equation}
    V_{\text{Coulomb}}(r) = \frac{Z_1 Z_2 e^2}{4\pi \varepsilon_0 r}
\end{equation}

In VAM genereren twee roterende wervelknopen in de nabijheid van \( r \sim 2r_c \) een door werveling geïnduceerde drukval~\cite{Saffman1992} via:
\begin{equation}
    \Delta P = \frac{1}{2} \rho_{\text{\ae}} r_c^2 (\Omega_1^2 + \Omega_2^2)
\end{equation}

Deze drukval wijzigt de effectieve interactiepotentiaal:
\begin{equation}
    V_{\text{eff}}(r) = V_{\text{Coulomb}}(r) - \Phi_\omega(r)
\end{equation}
waarbij de wervelpotentiaal \( \Phi_\omega(r) \) wordt gedefinieerd door:
\begin{equation}
    \Phi_\omega(r) = \gamma \int \frac{|\vec{\omega}(r')|^2}{|\vec{r} - \vec{r}'|} \, d^3r',
    \quad \text{met} \quad
    \gamma = G \rho_{\text{\ae}}^2
\end{equation}

Resonante tunneling treedt op wanneer het gecombineerde effect van \( \Delta P \) en \( \Phi_\omega \) de Coulombbarrière bij een kritische scheiding \( r_t \) neutraliseert:
\begin{equation}
    \frac{1}{2} \rho_{\text{\ae}} r_c^2 (\Omega_1^2 + \Omega_2^2) \geq \frac{Z_1 Z_2 e^2}{4\pi \varepsilon_0 r_t^2}
\end{equation}

De resulterende conditie maakt overgangen mogelijk, zelfs bij thermische of subthermische kinetische energieën, waardoor LENR-processen kunnen plaatsvinden zonder de barrière daadwerkelijk te hoeven overwinnen. In plaats daarvan wordt deze dynamisch uitgewist via wervelresonantie – een mechanisme dat consistent is met sommige empirische observaties~\cite{Storms2021}. De tunneling is dus een manifestatie van Ætherische fase-uitlijning en drukgemedieerde coherentie in beperkte wervelconfiguraties.



\subsection*{VAM Quantum Electrodynamics (QED) Lagrangian}

In het Vortex Æther Model (VAM) ontstaat de interactie tussen wervelknopen en elektromagnetische velden uit hun helicoïdale structuur en de daarmee gepaarde geïnduceerde vectorpotentialen. De standaard Lagrangiaan van de kwantumelektrodynamica (QED) wordt in dit model vervangen door:

\begin{equation}
    \mathcal{L}_{\text{VAM-QED}} =
    \bar{\psi} \left[ i \gamma^\mu \partial_\mu
                   - \gamma^\mu \left( \frac{C_e^2 r_c}{\lambda_c} \right) A_\mu
                   - \left( \frac{8\pi \rho_{\text{\ae}} r_c^3 Lk}{C_e} \right) \right] \psi
    - \frac{1}{4} F_{\mu\nu} F^{\mu\nu}
\end{equation}

In deze formulering:

\begin{itemize}
    \item Ontstaat de massa als gevolg van topologisch verbonden wervelkernen, waarbij de heliciteit van de wervelstructuur de rol van massa speelt~\cite{Volovik2003}.
    \item Komt de ijkkoppeling voort uit æthercirculatie en het daaruit voortvloeiende vectorpotentiaal.
    \item Blijft de elektromagnetische veldtensor \( F_{\mu\nu} \) ongewijzigd, die de rotatie van de æther (de krulcomponent) beschrijft in de omringende superfluïde.
\end{itemize}

Deze alternatieve Lagrangiaan koppelt dus wervelstructuren direct aan veldinteracties, waarbij de gebruikelijke constanten \( m \) (massa) en \( q \) (lading) worden vervangen door emergente termen die voortkomen uit de geometrie, rotatiesnelheid en topologie van het æthermedium.

Door afleiding van de Euler–Lagrangevergelijking voor het spinorveld \( \psi \), vinden we:

\begin{equation}
    \boxed{ \left( i \gamma^\mu \partial_\mu - \gamma^\mu q_{\text{vortex}} A_\mu - M_{\text{vortex}} \right)\psi = 0 }
\end{equation}

Deze vergelijking is structureel identiek aan de Dirac-vergelijking, maar met fysische parameters die voortkomen uit wervelmechanica in plaats van als fundamenteel gegeven. Daarmee levert VAM een alternatief voor de oorsprong van massa en lading~\cite{Barcelo2011,Volovik2003}.

%! Auteur = dhr
%! Datum = 29-3-2025

\section{VAM Wervelverstrooiingsraamwerk (geïnspireerd door elastische theorie)}

\subsection{Bepalende vergelijkingen van VAM Vorticiteitsdynamiek}

\subsubsection*{Vorticiteitstransportvergelijking (gelineariseerde vorm)}

In het Vortex Æther Model (VAM) wordt de dynamiek van het vorticiteitsveld \(\vec{\omega} = \nabla \times \vec{v}\) bepaald door de Euler-vergelijking en de bijbehorende vorticiteitsvorm:

\[
\frac{\partial \omega_i}{\partial t} + v_j \partial_j \omega_i = \omega_j \partial_j v_i
\]

Deze niet-lineaire structuur impliceert wervelvervorming door uitrekking en advectie. Voor kleine verstoringen \(\delta\omega\) nabij een achtergrondwervelknoopveld \(\omega^{(0)}\) geeft linearisatie:

\[
\frac{\partial (\delta \omega_i)}{\partial t} + v_j^{(0)} \partial_j (\delta \omega_i) \approx \omega_j^{(0)} \partial_j (\delta v_i)
\]

Definieer de lineaire responsoperator van VAM \(\mathcal{L}_{ij}\):

\[
\mathcal{L}_{ij} \, \delta v_j(\vec{r}) = \delta F_i^{\text{wervel}}(\vec{r})
\]

\subsubsection*{Vorticiteit Groene Tensor Vergelijking}

\[
\mathcal{L}_{ij} \, \mathcal{G}_{jk}(\vec{r}, \vec{r}') = -\delta_{ik} \, \delta(\vec{r} - \vec{r}')
\]

Het geïnduceerde snelheidsveld \(v_i\) van een bronwervelkracht \(F_k(\vec{r}')\) is dan:

\[
v_i(\vec{r}) = \int \mathcal{G}_{ik}(\vec{r}, \vec{r}') \, F_k^{\text{wervel}}(\vec{r}') \, d^3 r'
\]

\subsection{Wisselwerking werveldraad}
Interacties ontstaan door uitwisseling van wervelkracht of Herverbindingen tussen wervelfilamenten:
\begin{itemize}
\item Aantrekkelijk als draden de circulatie versterken (parallel)
\item Afstotend als draden elkaar opheffen (antiparallel)
\item Interactiesterkte:
\end{itemize}
\begin{equation}
\vec{F}_{\text{int}} = \beta \cdot \kappa_1 \kappa_2 \cdot \frac{\vec{r}_{12} \times (\vec{v}_1 - \vec{v}_2)}{|\vec{r}_{12}|^3}\label{eq:interaction_strength}
\end{equation}
Waar \(\kappa_i\) de circulaties van filamenten zijn en \(\vec{r}_{12}\) de vector ertussen.

\subsection{Thermodynamisch & kwantumgedrag van vorticiteitsfluctuaties}
\begin{itemize}
\item Entropie \(\leftrightarrow\) volume van werveluitbreiding of knoopvervorming
\item Kwantumovergangen \(\leftrightarrow\) topologische herverbindingsgebeurtenissen
\item Nulpuntbeweging \(\leftrightarrow\) achtergrondkwantumturbulentie van de Æther:
\end{itemize}

\subsubsection*{Achtergrond kwantumvorticiteit}
\begin{equation}
\langle \omega^2 \rangle \sim \frac{\hbar}{\rho_\text{æ} \xi^4}\label{eq:quantum_vorticity_background}
\end{equation}
Waarbij \(\xi\) de coherentielengte tussen wervelfilamenten is.

\subsection{VAM-verstrooiingstheorie voor wervelknopen}

\subsubsection*{Born-benadering voor wervelstoringen}

Veronderstel dat een invallende wervelpotentiaal \(\Phi^{(0)}(\vec{r})\) een wervelknoop tegenkomt op \(\vec{r}_k\). Het verstrooide vorticiteitsveld wordt:

\[
\Phi(\vec{r}) = \Phi^{(0)}(\vec{r}) + \int \mathcal{G}_{ij}(\vec{r}, \vec{r}') \, \delta \mathcal{V}_{jk}(\vec{r}') \, v_k^{(0)}(\vec{r}') \, d^3r'
\]

Hier vertegenwoordigt \(\delta \mathcal{V}_{jk}\) een vorticiteitspolarisatietensor geassocieerd met de knoop – een VAM-analoog aan elastische moduliperturbatie.

\subsection{Ætherspanningstensor en energieflux}

\subsubsection*{VAM-spanningstensor}

\[
\mathcal{T}_{ij} = \rho_{\text{\ae}} \, v_i v_j - \frac{1}{2} \delta_{ij} \rho_{\text{\ae}} v^2
\]

\subsubsection*{Æther Vorticiteit Krachtdichtheid}

\[
f_i^{\text{wervel}} = \partial_j \mathcal{T}_{ij}
\]

\subsubsection*{Vorticiteit Energieflux}

\[
\vec{S}_\omega = - \mathcal{T} \cdot \vec{v}
\]

Deze vector legt de energieoverdracht vast via wervelknoopinteracties en definieert Verstrooiing van "dwarsdoorsneden" via de divergentie \(\nabla \cdot \vec{S}_\omega\).

\subsection{Tijdsdilatatie en knoopverstrooiing}

\subsubsection*{Tijdsdilatatie door knooprotatie}

Laat het invallende wervelveld een lokale tijdvertraging veroorzaken als gevolg van de rotatie-energie van een knoop:

\[
\frac{t_{\text{local}}}{t_{\infty}} = \left(1 + \frac{1}{2} \beta I \Omega_k^2 \right)^{-1}
\]

In de Born-benadering is de verandering in eigentijd nabij een knoop onder externe wervelstroom:

\subsubsection*{Verstrooide correctie door extern veld}

\begin{gather*}
\delta \left( \frac{t_{\text{local}}}{t_{\infty}} \right) \approx - \frac{1}{2} \beta I \Omega_k \, \delta \Omega_k\\
\delta \Omega_k \sim \int \chi(\vec{r}_k - \vec{r}') \cdot \vec{\omega}^{(0)}(\vec{r}') \, d^3r'\\
\end{gather*}

Hier is \(\chi\) de topologische wervelgevoeligheidskern.

\subsection{Samenvatting van VAM-geïnspireerde verstrooiingsconstructies}

\begin{table}[htbp]
\centering
\begin{tabular}{lll}
\toprule
\textbf{Concept} & \textbf{Elastische theorie} & \textbf{VAM-analoog} \\
\midrule
Mediumeigenschap & \( c_{ijkl} \) & \( \rho_{\text{\ae}},\, \Omega_k,\, \kappa \) \\
Golfveld & \( u_i \) (verplaatsing) & \( v_i \) (æthersnelheid) \\
Bron & \( f_i \) (lichaamskracht) & \( F_i^{\text{vortex}} \) (vorticiteitsforcering) \\
Groene functie & \( G_{ij}(\vec{r}, \vec{r}') \) & \( \mathcal{G}_{ij}(\vec{r}, \vec{r}') \) \\
Spanningstensor & \( \tau_{ij} \) & \( \mathcal{T}_{ij} \) \\
Energieflux & \( J_{P,i} = -\tau_{ij} \dot{u}_j \) & \( S_{\omega,i} = -\mathcal{T}_{ij} v_j \) \\
Tijddilatatiemechanisme & \( g_{\mu\nu} \) (GR metrisch) & \( \Omega_k,\, \kappa,\, \langle \omega^2 \rangle \) \\
\bottomrule
\end{tabular}
\caption{Conceptuele overeenkomst tussen klassieke elasticiteit en Vortex Æther Model (VAM).}
\label{tab:elastic-vam-analogy}
\end{table}

Dit verstrooiingsraamwerk generaliseert klassieke elastische analogen naar een topologisch en energetisch gemotiveerd Ætherisch formalisme. Het maakt de berekening mogelijk van veldmodificaties, tijddilatatie-effecten en energieflux als gevolg van stabiele, interacterende wervelknopen in het Vortex Æther Model (VAM).
\section{Experimentele tests en observatievoorspellingen van VAM}


\subsection{1. Tijddilatatie in roterende superfluïden}

Het Vortex Æther Model voorspelt dat in een superfluïde wervelkern, lokale tijd langzamer verloopt naarmate de hoeksnelheid $\Omega_k$ toeneemt. Dit is experimenteel testbaar in:
\begin{itemize}
    \item Bose–Einsteincondensaten (BEC's) met coherente roterende toestanden,
    \item Roterende superfluïde heliumdragers met interne frequentiemetingen (bijv. neutronspinresonantie),
    \item Vergelijkbare systemen met lasergeïnduceerde vorticiteit.
\end{itemize}

Verschillen in tijdverloop of fase tussen roterende en niet-roterende atoomklokken kunnen worden opgevat als een test voor Æther-tijdmodulatie zonder kromming.~\cite{Steinhauer2016}


\subsection{2. Plasma-wervelklokken en cyclotronanalogieën}

Cyclotronvelden, ringvormige plasmarotaties of roterende magnetische vallen genereren gradiënten in $\Omega(r)$. Volgens VAM leidt dit tot meetbare klokvervorming. Experimentele voorspellingen:
\begin{itemize}
    \item Fasedifferentiatie in optische pulsen langs plasmawervelranden,~\cite{Unruh1981}
    \item Veranderingen in stralingsemissiepatronen in asymmetrische wervelplasma's.
\end{itemize}

\subsection{3. Optische en metamateriaal-analogen}

Net als bij analogue gravity kunnen synthetische golfgeleiders of metamaterialen “ætherstroming” simuleren. Hierbij:
\begin{itemize}
    \item Wordt de voortplanting van licht beïnvloed door artificiële rotatiestromen,
    \item Kan anisotrope brekingsindex simuleren wat VAM-lichtafbuiging nabootst,
    \item Kan dispersie-analyse inzicht geven in lokale tijdsvertraging.
\end{itemize}

\subsection{4. Verwachte observatiekenmerken}

Experimentele handtekeningen van VAM kunnen zijn:
\begin{enumerate}
    \item Grenswaarden voor wervelknoop-instorting met plotse energie-afgifte,
    \item Lokale tijdaanomalieën in roterende laboratoriumsystemen,
    \item Absente relativistische versnelling bij energetisch gunstige wervelsystemen,
    \item Niet-symmetrische kloksnelheden aan verschillende zijden van een wervelkern.
\end{enumerate}


\section{VAM versus GR – Overeenkomstige Voorspellingen}

Hoewel het Vortex Æther Model een fundamenteel andere ontologie hanteert dan de kromming-gebaseerde structuur van algemene relativiteit, leidt het in vele gevallen tot vergelijkbare uitdrukkingen voor fysisch waarneembare fenomenen. In deze sectie tonen we hoe VAM de klassieke voorspellingen van GR reproduceert — maar met alternatieve onderliggende mechanismen.


\subsection*{VAM-orbitaalprecessie (GR-equivalent)}


In de algemene relativiteitstheorie wordt de periheliumprecessie van een draaiend lichaam toegeschreven aan ruimtetijdkromming. In het Vortex Æther Model (VAM) wordt dit effect vervangen door de cumulatieve invloed van een werveling-geïnduceerd vorticiteitsveld binnen een roterend Æthermedium.

De equivalente VAM-formulering weerspiegelt de GR-voorspelling, maar is gebaseerd op door vorticiteit geïnduceerde drukgradiënten en circulatie:

\begin{equation}
    \Delta\phi_{\text{VAM}} =
    \frac{6\pi G M}{a(1 - e^2) c^2}
\end{equation}

waarbij:
\begin{itemize}
    \item \( M \): massa van de centrale wervel-attractor,
    \item \( a \): halve lange as van de baan,
    \item \( e \): excentriciteit van de baan,
    \item \( G \): gravitatieconstante (herleid uit VAM-koppeling),
    \item \( c \): lichtsnelheid.
\end{itemize}
Hoewel formeel identiek aan de GR-uitdrukking, ontstaat dit in VAM door de variatie in lokale circulatie en impulsmomentflux binnen de omringende Æther, waardoor het effectieve potentiaal wordt gemoduleerd en precessiebeweging ontstaat.

\subsection*{VAM-lichtafbuiging door Ætherische circulatie}


In de algemene relativiteitstheorie wordt lichtafbuiging door massieve lichamen veroorzaakt door ruimtetijdkromming. In het Vortex Æther Model buigt licht (beschouwd als een verstoring of modus in de Æther) af als gevolg van door circulatie geïnduceerde drukgradiënten en anisotrope brekingsindexvelden in de buurt van roterende wervel-aantrekkers.

De equivalente VAM-afbuigingshoek voor een lichtstraal die langs een sferische wervelmassa strijkt, wordt gegeven door:

\begin{equation}
    \delta_{\text{VAM}} =
    \frac{4 G M}{R c^2}
\end{equation}

waarbij:
\begin{itemize}
    \item \( M \): effectieve massa van de roterende wervelknoop,
    \item \( R \): dichtstbijzijnde nadering (impactparameter),
    \item \( G \): wervelkoppelingsconstante (herstel van Newtoniaanse \( G \) onder macroscopische grenzen),
    \item \( c \): lichtsnelheid.
\end{itemize}

In VAM is dit het gevolg van de interactie tussen de voortplantingssnelheid van het licht en het omringende rotatieveld. Het lichtgolffront wordt lokaal samengedrukt of gebroken door tangentiële ætherstroomgradiënten, wat leidt tot een waarneembare hoekafbuiging.

\subsection*{Overzicht van de waarneembare correspondentie tussen VAM en GR}

\begin{table}[ht]
    \centering
    \caption{Vergelijking van GR en VAM voor gravitatiegerelateerde observabelen}
    \label{tab:VAM-GR}
    \begin{tabularx}{0.8\textwidth}{|>{\raggedright\arraybackslash}X|c|>{\centering\arraybackslash}X|}

                \hline
                \textbf{Waarneembaar} & \textbf{Theorie} & \textbf{Uitdrukking} \\
                \hline

                \multirow{2}{=}{Tijdsdilatatie}
                & GR & \( \displaystyle \frac{d\tau}{dt} = \sqrt{1 - \frac{2GM}{rc^2}} \) \\
                & VAM & \( \displaystyle \frac{d\tau}{dt} = \sqrt{1 - \frac{\Omega^2 r^2}{c^2}} \) \\

                \hline
                \multirow{2}{=}{Roodverschuiving}
                & GR & \( \displaystyle z = \left(1 - \frac{2GM}{rc^2} \right)^{-1/2} - 1 \) \\
                & VAM & \( \displaystyle z = \left(1 - \frac{v_\phi^2}{c^2} \right)^{-1/2} - 1 \) \\

                \hline
                \multirow{2}{=}{Frame slepen}
                & GR & \( \displaystyle \omega_{\text{LT}} = \frac{2GJ}{c^2 r^3} \) \\
                & VAM & \( \displaystyle \omega_{\text{drag}} = \frac{2G \mu I \Omega}{c^2 r^3} \) \\

                \hline
                Precessie & GR/VAM & \( \displaystyle \Delta\phi = \frac{6\pi GM}{a(1 - e^2)c^2} \) \\
                \hline
                Lichtafbuiging & GR/VAM & \( \displaystyle \delta = \frac{4GM}{Rc^2} \) \\
                \hline

                \multirow{2}{=}{Zwaartekracht\-potentiaal}
                & GR & \( \displaystyle \Phi = -\frac{GM}{r} \) \\
                & VAM & \( \displaystyle \Phi = -\frac{1}{2} \vec{\omega} \cdot \vec{v} \) \\
                \hline

                Zwaartekracht\-constante & VAM & \( \displaystyle G = \frac{C_e c^5 t_p^2}{2 F_{\max} r_c^2} \) \\
                \hline


    \end{tabularx}
\end{table}


\newline
\bibliography{2-Wervelklokken_en_vorticiteit_geïnduceerde_zwaartekracht}
\bibliographystyle{unsrt}

\appendix \label{sec:appendix}
    %! Author = MissAliceWonderland
%! Date = 5/4/2025

\section{Afleiding van de tijdsdilatatieformule binnen VAM}

Binnen het Vortex Æther Model (VAM) ontstaat tijdsdilatatie niet uit ruimtetijdkromming, maar uit lokale energetische eigenschappen van het ætherveld, zoals rotatie (vorticiteit), drukgradiënten en topologische eigenschappen van wervelstructuren. De lokale klokfrequentie van een wervel—geassocieerd met een elementair deeltje of een macroscopisch object—is afhankelijk van zowel de interne kernrotatie als externe omgevingsinvloeden zoals zwaartekrachtsvelden en frame-dragging.

De tijdsdilatatiefactor $\frac{d\tau}{dt}$ wordt in VAM uitgedrukt als een samengestelde correctie op de universele tijd $t$, waarin de lokale "eigenklok" $\tau$ trager tikt onder invloed van:

1. Vervorming van ætherstroom rond een wervelkern;
2. Externe gravitationele vorticiteit veroorzaakt door massa;
3. Roterende achtergrondvelden.

We leiden de volgende formule af:

\begin{equation}
\frac{d\tau}{dt} = \sqrt{1 - \frac{C_e^2}{c^2} e^{-r/r_c} - \frac{2G_{\text{swirl}} M_{\text{eff}}(r)}{r c^2} - \beta \Omega^2}
\end{equation}

Elke term vertegenwoordigt een fysisch mechanisme:

\begin{itemize}
  \item \textbf{Term 1: Kernrotatie (lokale swirl)}
  \[
  \frac{C_e^2}{c^2} e^{-r/r_c}
  \]
  Deze term is afgeleid uit de intrinsieke hoeksnelheid $\Omega_\text{core}$ van de wervelkern. De tangentiële snelheid $C_e$ is de maximale swirl op de kernrand, en $r_c$ is de straal van de wervelkern. De exponentiële factor $e^{-r/r_c}$ geeft de afname van invloed weer op afstand $r$ buiten de kern. Deze term representeert de tijdvertraging als gevolg van lokale ætherrotatie.

  \item \textbf{Term 2: Zwaartekrachtsveld (vorticiteit-geïnduceerde potentiaal)}
  \[
  \frac{2 G_{\text{swirl}} M_{\text{eff}}(r)}{r c^2}
  \]
  Deze term bootst de klassieke gravitationele roodverschuiving na, maar met een alternatieve zwaartekrachtsconstante $G_{\text{swirl}}$ die volgt uit ætherparameters zoals dichtheid en swirlkracht. De effectieve massa $M_{\text{eff}}(r)$ kan hier worden opgevat als de æther-vortexenergie binnen straal $r$, i.p.v. conventionele massa. Deze term komt voort uit het drukdeficit door externe swirl en vervangt Newtonse zwaartekracht.

  \item \textbf{Term 3: Macroscopische rotatie (frame-dragging)}
  \[
  \beta \Omega^2
  \]
  Deze term representeert frame-dragging-effecten binnen een draaiende wervelconfiguratie (vergelijkbaar met het Kerr-metriek-effect in GR). De factor $\Omega$ is de rotatiesnelheid van het macroscopisch object (bijv. planeet of neutronenster), en $\beta$ is een koppelingsconstante die afhangt van ætherparameters. Deze term veroorzaakt extra vertraging van lokale tijd door circulatie van het omringende ætherveld.

\end{itemize}

De bovenstaande vergelijking is analoog aan relativistische formules, maar wortelt in vloeistofmechanische oorsprong. Experimenteel kunnen componenten van deze formule worden teruggevonden in tijdsdilatatie van GPS-klokken (zwaartekracht), Lense-Thirring-effecten (rotatie), en hypothetische laboratoriummetingen van kernrotaties op quantum- of wervelschaal.


    %! Author = Omar Iskandarani
%! Date = 5/4/2025

\section{Afleiding van het vorticiteit-gebaseerde gravitationele veld}\label{sec:appendix_2}

In het Vortex Æther Model (VAM) wordt de æther gemodelleerd als een stationaire, onsamendrukbare, inviscide vloeistof met constante massadichtheid~$\rho$. De dynamica van zo'n medium wordt beschreven door de stationaire Eulervergelijking:

\begin{equation}
(\vec{v} \cdot \nabla)\vec{v} = -\frac{1}{\rho} \nabla p,
\end{equation}

waarbij $\vec{v}$ het snelheidsveld is en $p$ de druk. Om deze uitdrukking te herschrijven gebruiken we een vectoridentiteit:

\begin{equation}
(\vec{v} \cdot \nabla)\vec{v} = \nabla\left(\frac{1}{2}v^2\right) - \vec{v} \times (\nabla \times \vec{v}) = \nabla\left(\frac{1}{2}v^2\right) - \vec{v} \times \vec{\omega},
\end{equation}

waar $\vec{\omega} = \nabla \times \vec{v}$ de lokale vorticiteit is. Substitutie levert:

\begin{equation}
\nabla\left(\frac{1}{2}v^2\right) - \vec{v} \times \vec{\omega} = -\frac{1}{\rho} \nabla p.
\end{equation}

We nemen nu het scalair product met $\vec{v}$ aan beide zijden:

\begin{equation}
\vec{v} \cdot \nabla\left(\frac{1}{2}v^2 + \frac{p}{\rho}\right) = 0.
\end{equation}

Deze vergelijking toont aan dat de grootheid

\begin{equation}
B = \frac{1}{2}v^2 + \frac{p}{\rho}
\end{equation}

constant is langs stroomlijnen, een bekende vorm van de Bernoulli-vergelijking. In gebieden met hoge vorticiteit (zoals in wervelkernen), is $v$ groot en dus $p$ relatief laag. Dit resulteert in een drukgradiënt die zich gedraagt als een aantrekkende kracht—een zwaartekrachtanalogie binnen het VAM-kader.

We definiëren daarom een vorticiteit-geïnduceerde potentiaal $\Phi_v$ zodanig dat:

\begin{equation}
\vec{F}_g = -\nabla \Phi_v,
\end{equation}

waarbij de potentiaal wordt gegeven door:

\begin{equation}
\Phi_v(\vec{r}) = \gamma \int \frac{\|\vec{\omega}(\vec{r}')\|^2}{\|\vec{r} - \vec{r}'\|} \, d^3r',
\end{equation}

met $\gamma$ de vorticiteit-gravitatiekoppeling. Dit leidt tot de Poisson-achtige vergelijking:

\begin{equation}
\nabla^2 \Phi_v(\vec{r}) = -\rho \|\vec{\omega}(\vec{r})\|^2,
\end{equation}

waarbij de rol van massadichtheid (zoals in Newtoniaanse gravitatietheorie) is vervangen door vorticiteitintensiteit. Dit bevestigt de kernhypothese van het VAM: zwaartekracht is geen gevolg van ruimtetijdkromming, maar een emergent fenomeen voortkomend uit drukverschillen veroorzaakt door wervelstroming.
    %! Author = Omar Iskandarani
%! Date = 5/4/2025

\section{Newtonse limiet en validatie van tijddilatatie}\label{sec:appendix_3}

Om de fysische geldigheid van het Vortex Æther Model (VAM) te bevestigen, analyseren we de limiet $r \gg r_c$, waarin het zwaartekrachtsveld zwak is en de vorticiteit zich ver weg van de bron bevindt. We tonen dat in deze limiet de vorticiteitspotentiaal $\Phi_v$ en de tijddilatatieformule van VAM overgaan in de klassieke Newtonse en relativistische vormen.

\subsection{Vorticiteitspotentiaal op grote afstand}

De vorticiteit-geïnduceerde potentiaal is in VAM gedefinieerd als:

\begin{equation}
\Phi_v(\vec{r}) = \gamma \int \frac{\|\vec{\omega}(\vec{r}')\|^2}{\|\vec{r} - \vec{r}'\|} \, d^3r',
\end{equation}

waar $\gamma = G \rho_\text{æ}^2$ de vorticiteit-gravitatiekoppeling is. Voor een sterk gelokaliseerde wervel (kernstraal $r_c \ll r$), kunnen we buiten de kern de integratie benaderen als afkomstig van een effectieve puntmassa:

\begin{equation}
\Phi_v(r) \to -\frac{G M_\text{eff}}{r},
\end{equation}

waar $M_\text{eff} = \int \rho_\text{æ} \|\vec{\omega}(\vec{r}')\|^2 d^3r' / \rho_\text{æ}$ fungeert als equivalente massa via wervelenergie. Deze benadering reproduceert exact de Newtonse zwaartekrachtswet.

\subsection{Tijddilatatie in de zwakveldgrens}

Voor $r \gg r_c$ geldt $e^{-r/r_c} \to 0$ en $\Omega^2 \approx 0$ voor niet-roterende objecten. De tijddilatatieformule reduceert dan tot:

\begin{equation}
\frac{d\tau}{dt} \approx \sqrt{1 - \frac{2 G_\text{swirl} M_\text{eff}}{r c^2}}.
\end{equation}

Indien we $G_\text{swirl} \approx G$ aannemen (in de macroscopische limiet), komt deze exact overeen met de eerste-orde benadering van de Schwarzschild-oplossing in algemene relativiteit:

\begin{equation}
\frac{d\tau}{dt}_\text{GR} \approx \sqrt{1 - \frac{2GM}{rc^2}}.
\end{equation}

Hiermee toont VAM dus consistente overgang naar GR in zwakke velden.

\subsection{Voorbeeld: de Aarde als wervelmassa}

Beschouw de Aarde als een wervelmassa met massa $M = 5.97 \times 10^{24}$ kg en straal $R = 6.371 \times 10^6$ m. De Newtonse zwaartekrachtsversnelling aan het oppervlak is:

\begin{equation}
g = \frac{G M}{R^2} \approx \frac{6.674 \times 10^{-11} \cdot 5.97 \times 10^{24}}{(6.371 \times 10^6)^2} \approx 9.8 \, \text{m/s}^2.
\end{equation}

In het VAM wordt deze versnelling opgevat als de gradiënt van de vorticiteitspotentiaal:

\begin{equation}
g = -\frac{d\Phi_v}{dr} \approx \frac{G M_\text{eff}}{R^2}.
\end{equation}

Zolang $M_\text{eff} \approx M$ reproduceert het VAM exact de bekende gravitatieversnelling op Aarde, inclusief de correcte roodverschuiving van tijd bij klokken op verschillende hoogtes (zoals waargenomen in GPS-systemen).

\section{Validatie met het Hafele–Keating-klokexperiment}

Een empirische toets voor tijddilatatie is het beroemde Hafele–Keating-experiment (1971), waarin atoomklokken in vliegtuigen de aarde omcirkelden in oostelijke en westelijke richting. De resultaten toonden significante tijdsverschillen vergeleken met klokken op aarde, consistent met voorspellingen van zowel speciale als algemene relativiteit. In het Vortex Æther Model (VAM) worden deze verschillen gereproduceerd door variaties in lokale ætherrotatie en drukvelden.

\subsection{Samenvatting van het experiment}

In het experiment werden vier cesiumklokken aan boord van commerciële vliegtuigen geplaatst die de aarde omcirkelen in twee richtingen:

\begin{itemize}
    \item \textbf{Oostwaarts} (met de rotatie van de aarde): verhoogde snelheid $\Rightarrow$ kinetische tijddilatatie.
    \item \textbf{Westwaarts} (tegen de rotatie in): verlaagde snelheid $\Rightarrow$ minder kinetische vertraging.
\end{itemize}

Daarnaast bevonden de vliegtuigen zich op grotere hoogte, wat leidde tot een lagere zwaartekrachtsversnelling en dus een gravitationele \emph{versnelling} van de klokfrequentie (blauwverschuiving).

De gemeten afwijkingen bedroegen:

\begin{itemize}
    \item Oostwaarts: $\Delta\tau \approx -59$ ns (vertraging)
    \item Westwaarts: $\Delta\tau \approx +273$ ns (versnelling)
\end{itemize}

\subsection{Interpretatie binnen het Vortex Æther Model}

In VAM worden beide effecten gereproduceerd via de tijddilatatieformule:

\begin{equation}
\frac{d\tau}{dt} = \sqrt{1 - \frac{C_e^2}{c^2} e^{-r/r_c} - \frac{2G_\text{swirl} M_\text{eff}(r)}{rc^2} - \beta \Omega^2}
\end{equation}

\begin{itemize}
    \item De \textbf{zwaartekrachtterm} $- \frac{2G_\text{swirl} M_\text{eff}(r)}{rc^2}$ wordt kleiner op grotere hoogte $\Rightarrow$ $\tau$ versnelt (klok tikt sneller).
    \item De \textbf{rotatieterm} $-\beta \Omega^2$ groeit met toenemende tangentiële snelheid van het vliegtuig $\Rightarrow$ $\tau$ vertraagt (klok tikt trager).
\end{itemize}

Voor oostwaarts bewegende klokken versterken beide effecten elkaar: lagere potentiaal en hogere snelheid vertragen de klok. Voor westwaarts bewegende klokken compenseren ze elkaar deels, wat resulteert in een nettoversnelling van tijd.

\subsection{Numerieke overeenstemming}

Gebruikmakend van realistische waarden voor $r_c$, $C_e$, en $\beta$ afgeleid uit ætherdichtheid en kernstructuur (zie Tabel~\ref{tab:constants}), kan het VAM binnen de meetnauwkeurigheid van het experiment reproduceerbare afwijkingen voorspellen van dezelfde grootteorde als gemeten. Hiermee toont het model niet alleen conceptuele overeenstemming met GR, maar ook experimentele compatibiliteit.

\begin{table}[h!]
\centering
\caption{Typische parameters in het VAM-model}
\label{tab:constants}
\begin{tabular}{lll}
\toprule
Symbool & Betekenis & Waarde \\
\midrule
$C_e$ & Tangentiële snelheid kern & $\sim 1.09 \times 10^6$ m/s \\
$r_c$ & Wervelkernstraal & $\sim 1.4 \times 10^{-15}$ m \\
$\beta$ & Tijddilatatiekoppeling & $\sim 1.66 \times 10^{-42}$ s$^2$ \\
$G_\text{swirl}$ & VAM-gravitatieconstante & $\sim G$ (macro) \\
\bottomrule
\end{tabular}
\end{table}
    %! Author = MissAliceWonderland
%! Date = 5/4/2025

\section{Dynamica van vortexcirculatie en kwantisatie}

Een centrale bouwsteen van het Vortex Æther Model (VAM) is de dynamica van circulerende stroming rond een vortexkern. De hoeveelheid rotatie in een gesloten lus rondom de vortex wordt beschreven via de circulatie \( \Gamma \), een fundamentele grootheid in klassieke en topologische vloeistofdynamica.

\subsection{Kelvin's circulatietheorema}

Volgens Kelvin’s circulatietheorema blijft de circulatie \( \Gamma \) behouden in een ideale, inviscide vloeistof bij afwezigheid van externe krachten:

\begin{equation}
\Gamma = \oint_{\mathcal{C}(t)} \vec{v} \cdot d\vec{l} = \text{const.}
\end{equation}

Hier is \( \mathcal{C}(t) \) een gesloten lus die meebeweegt met het fluïde. In het geval van een superfluïde æther betekent dit dat vortexstructuren stabiel en topologisch beschermd zijn — ze kunnen niet eenvoudig vervormen of verdwijnen zonder verbreking van conservatie.

\subsection{Circulatie rond de vortexkern}

Voor een stationaire vortexconfiguratie met kernstraal \( r_c \) en maximale tangentiële snelheid \( C_e \), volgt uit symmetrie:

\begin{equation}
\Gamma = \oint \vec{v} \cdot d\vec{l} = 2\pi r_c C_e.
\end{equation}

Deze uitdrukking beschrijft de totale rotatie van het ætherveld rond een enkel vortexdeeltje, zoals een elektron.

\subsection{Kwantisering van circulatie}

In superfluïda zoals helium II is waargenomen dat circulatie slechts in discrete eenheden voorkomt. Dit principe wordt overgenomen in VAM door te stellen dat circulatie kwantiseert in gehele veelvouden van een basiseenheid \( \kappa \):

\begin{equation}
\Gamma_n = n \cdot \kappa, \quad n \in \mathbb{Z},
\end{equation}

waarbij

\begin{equation}
\kappa = C_e r_c
\end{equation}

de elementaire circulatieconstante is. Deze waarde is analoog aan \( h/m \) in de context van kwantumvloeistoffen en wordt in VAM gekoppeld aan vortexkernparameters.

\subsection{Fysische interpretatie}

\begin{itemize}
    \item De circulatie \( \Gamma \) bepaalt de rotatie-inhoud van een vortexknoop en is gekoppeld aan de massa en inertie van het corresponderende deeltje.
    \item De constante \( \kappa \) bepaalt de \("\)spin\("\)-eenheid of vortex-heliciteit van een elementair vortexdeeltje.
    \item De vortexcirculatie is een conserved quantity en leidt tot intrinsiek stabiele en discrete toestanden — een directe analogie met quantisatie in deeltjesfysica.
\end{itemize}

Hiermee biedt VAM een formeel raamwerk waarin klassieke stromingswetten — via Kelvin en Euler — overgaan in topologisch gekwantiseerde veldstructuren die fundamentele deeltjes beschrijven.

    %! Author = MissAliceWonderland
%! Date = 5/4/2025

\section{Tijdsdilatatie uit vortexenergie en drukgradiënten}

In het Vortex Æther Model (VAM) wordt tijdsdilatatie opgevat als een energetisch fenomeen dat voortkomt uit de rotatie-energie van lokale æthervortices. In plaats van af te hangen van ruimtetijdkromming zoals in de algemene relativiteitstheorie, is de klokfrequentie in VAM gekoppeld aan de vortexkinetiek in het omringende æther.

\subsection{Formule: klokvertraging door rotatie-energie}

De eigenfrequentie van een vortex-gebaseerde klok is afhankelijk van de totale energie opgeslagen in lokale kernrotatie. Voor een klok met moment van traagheid $I$ en hoeksnelheid $\Omega$ geldt:

\begin{equation}
\frac{d\tau}{dt} = \left(1 + \frac{1}{2} \beta I \Omega^2 \right)^{-1},
\end{equation}

waar $\beta$ een tijd-dilatatiekoppeling is afgeleid uit ætherparameters (bijv. $r_c$, $C_e$). Deze formule impliceert:

\begin{itemize}
    \item Hoe groter de lokale rotatie-energie, hoe sterker de klokvertraging.
    \item Voor zwakke rotatie ($\Omega \to 0$) geldt $\tau \approx t$ (geen dilatatie).
\end{itemize}

Deze uitdrukking is analoog aan relativistische dilatatieformules, maar wortelt in vortexmechanica.

\subsection{Alternatieve afleiding via drukverschil (Bernoulli-benadering)}

Dezelfde effect kan worden afgeleid via Bernoulli’s wet in een stationaire stroming:

\begin{equation}
\frac{1}{2} \rho v^2 + p = \text{const.}
\end{equation}

Rond een roterende vortex geldt:

\[
v = \Omega r, \quad \Rightarrow \quad \Delta p = -\frac{1}{2} \rho (\Omega r)^2
\]

Dit leidt tot een lokaal drukdeficit rond de vortexas. In het VAM wordt verondersteld dat de klokfrequentie $\nu$ stijgt bij hogere druk (hogere ætherdichtheid), en daalt bij lage druk. De klokvertraging volgt dan via enthalpie:

\begin{equation}
\frac{d\tau}{dt} \sim \frac{H_\text{ref}}{H_\text{loc}} \approx \frac{1}{1 + \frac{\Delta p}{\rho}},
\end{equation}

wat voor kleine $\Delta p$ ook leidt tot een benadering van de vorm:

\begin{equation}
\frac{d\tau}{dt} \approx \left(1 + \frac{1}{2} \beta I \Omega^2 \right)^{-1}.
\end{equation}

\subsection{Fysische interpretatie}

\begin{itemize}
    \item \textbf{Mechanisch}: Tijdsdilatatie is een maat voor de energie opgeslagen in kernrotatie; sneller draaiende knopen vertragen de lokale klok.
    \item \textbf{Hydrodynamisch}: Drukverlaging door swirl vertraagt tijd — conform Bernoulli.
    \item \textbf{Thermodynamisch}: Entropiestijging in werveluitzetting correleert met tijdvertraging.
\end{itemize}

Hiermee toont VAM dat tijdsdilatatie een emergent verschijnsel is van vortexenergie en stromingsdruk, en reproduceert het klassieke relativistische gedrag vanuit vloeistofdynamische principes.


    %! Author = Omar Iskandarani
%! Date = 5/4/2025
\section{Parameterafstemming en limietgedrag}\label{sec:appendix_6}

Om de vergelijkingen van het Vortex Æther Model (VAM) in overeenstemming te brengen met klassieke zwaartekracht, moeten de modelparameters zodanig afgesteld worden dat ze bekende fysische constanten reproduceren in de juiste limieten. In deze sectie leiden we de effectieve gravitatieconstante $G_\text{swirl}$ af en analyseren we het gedrag van het zwaartekrachtsveld voor $r \to \infty$.

\subsection{Afleiding van $G_\text{swirl}$ uit wervelparameters}

De VAM-potentiaal is gegeven door:

\begin{equation}
\Phi_v(\vec{r}) = G_\text{swirl} \int \frac{\|\vec{\omega}(\vec{r}')\|^2}{\|\vec{r} - \vec{r}'\|} \, d^3r',
\end{equation}

waarbij $G_\text{swirl}$ moet voldoen aan een dimensionele en fysisch consistente relatie met fundamentele wervelparameters. In termen van:

\begin{itemize}
  \item $C_e$: tangentiële snelheid aan de wervelkern,
  \item $r_c$: wervelkernstraal,
  \item $t_p$: Planck-tijd,
  \item $F^{\text{max}}_{\text{\ae}}$: maximale kracht in ætherinteracties,
\end{itemize}

leiden we af:

\begin{equation}
G_\text{swirl} = \frac{C_e c^5 t_p^2}{2 F^{\text{max}}_{\text{\ae}} r_c^2}.
\end{equation}

Deze expressie volgt uit dimensie-analyse en matching van de VAM-veldvergelijkingen met de Newtonse limiet (zie ook [Iskandarani, 2025]).

\subsection{Limiet $r \to \infty$: klassieke zwaartekracht}

Voor grote afstanden buiten een compacte wervelconfiguratie geldt:

\begin{equation}
\Phi_v(r) = G_\text{swirl} \int \frac{\|\vec{\omega}(\vec{r}')\|^2}{|\vec{r} - \vec{r}'|} d^3r' \approx \frac{G_\text{swirl}}{r} \int \|\vec{\omega}(\vec{r}')\|^2 d^3r'.
\end{equation}

Definieer de \textbf{effectieve massa} van het wervelobject als:

\begin{equation}
M_\text{eff} = \frac{1}{\rho_\text{æ}} \int \rho_\text{æ} \|\vec{\omega}(\vec{r}')\|^2 d^3r' = \int \|\vec{\omega}(\vec{r}')\|^2 d^3r'.
\end{equation}

Daarmee wordt:

\begin{equation}
\Phi_v(r) \to -\frac{G_\text{swirl} M_\text{eff}}{r},
\end{equation}

wat identiek is aan de Newtonse potentiaal mits $M_\text{eff} \approx M_\text{grav}$ en $G_\text{swirl} \approx G$.

\subsection{Relatie tussen $M_\text{eff}$ en geobserveerde massa}

De effectieve massa $M_\text{eff}$ is geen directe massa-inhoud zoals in klassieke fysica, maar weerspiegelt de geïntegreerde vorticiteitenergie in de æther:

\begin{equation}
  M_\text{eff} \propto \int \frac{1}{2} \rho_\text{æ} \|\vec{v}(\vec{r})\|^2 d^3r.
\end{equation}

In VAM wordt deze massa geassocieerd met een topologisch stabiele wervelknoop (zoals een trefoil voor het elektron) en dus kwantitatief:

\begin{equation}
M_\text{eff} = \alpha \cdot \rho_\text{æ} C_e r_c^3 \cdot L_k,
\end{equation}

waarbij $L_k$ de linking number is van de knoop en $\alpha$ een vormfactor. Door afstemming van $C_e$, $r_c$ en $\rho_\text{æ}$ op bekende massa's (bijv. van het elektron of de aarde), kan VAM de klassieke massa exact reproduceren:

\begin{equation}
M_\text{eff} \overset{!}{=} M_\text{obs}.
\end{equation}

\subsection{Conclusie}

Door parameterafstemming voldoet $G_\text{swirl}$ aan klassieke limieten en levert VAM een zwaartekrachtsveld dat bij grote afstanden overeenkomt met Newtonse gravitatie. De effectieve massa $M_\text{eff}$ fungeert als bronterm, analoog aan de rol van $M$ in Newton en GR.
    \section{Grondslagen van snelheidsvelden en energieën in een wervelsysteem.}

\subsection{Inleiding}
Werveldynamica is een kerncomponent van veel vloeistof- en plasmasystemen, waaronder
tornado-achtige stromingen, geknoopte wervels in klassieke of superfluïde turbulentie en diverse
complexe topologische vloeistofsystemen. Een beter begrip van de energiebalansen
die met deze stromingen gepaard gaan, kan licht werpen op processen zoals wervelstabiliteit, herverbinding
en globale stromingsorganisatie. We beginnen met een motivatie voor hoe snelheidsvelden kunnen worden
ontbonden om de totale energie (d.w.z. zelf- plus kruisenergie) vast te leggen, en hoe
deze aanpak helpt bij het volgen van stromingen in zowel 2D als 3D.

\subsection{Fundamenten: Snelheidsvelden en totale (zelf- + dwars-)energie}
\label{sec:foundations}
In een onsamendrukbare vloeistof wordt het snelheidsveld $\mathbf{u}(\mathbf{x}, t)$ doorgaans
bepaald door de Navier-Stokes- of Euler-vergelijkingen. Voor niet-viskeuze analyses luiden de Euler-vergelijkingen voor onsamendrukbare stroming als volgt:
\begin{equation}
\frac{\partial \mathbf{u}}{\partial t} + (\mathbf{u} \cdot \nabla)\mathbf{u} = -\frac{1}{\rho}\nabla p,
\quad \nabla \cdot \mathbf{u} = 0.\label{eq:appendix:Euler}
\end{equation}
We beschouwen ook de vorticiteit $\boldsymbol{\omega} = \nabla \times \mathbf{u}$,
die kan worden gebruikt om wervelstructuren te karakteriseren.

Om de totale kinetische energie te begrijpen, kunnen we deze als volgt opsplitsen:
\begin{equation}
E_{\text{totaal}} \;=\; E_{\text{zelf}} \;+\; E_{\text{kruis}}.\label{eq:appendix:totale-energie}
\end{equation}
Hier is $E_{\text{zelf}}$ het deel van de energie dat elk wervel- of deelstroomelement onafhankelijk bijdraagt (bijvoorbeeld door lokale wervelbewegingen), terwijl
$E_{\text{kruis}}$ de bijdragen codeert die voortkomen uit de interactie van verschillende
wervelelementen. In een multi-wervelscenario helpt een dergelijke decompositie om de
directe interactie tussen twee (of meer) wervelfilamenten of -lagen te isoleren.

\subsection{Overwegingen met betrekking tot impuls en eigen energie}
\label{sec:impuls}
Een startpunt is om te onthouden dat voor een enkele circulatiewervel $\Gamma$, met een
azimutaal symmetrische kern, de geïnduceerde snelheid soms wordt benaderd door
klassieke resultaten zoals
\begin{equation}
V \;=\; \frac{\Gamma}{4 \pi R}
\bigl(\ln \tfrac{8 R}{a} - \beta \bigr),\label{eq:appendix:velocity}
\end{equation}
waarbij $R$ de straal van de hoofdwervellus is, $a \ll R$ een maat is voor de kerndikte,
en $\beta$ afhangt van de details van het kernmodel \cite{Saffman1992}. De
\emph{zelfenergie} die aan die wervel is gekoppeld, $E_{\text{self}}$, kan in een
vergelijkbare vorm worden gegoten die afhankelijk is van $\ln(R/a)$, wat illustreert hoe de energieën van dunnekernwervelen
schalen met de geometrie.

In meer algemene vloeistof- of wervelroostermodellen kunnen we $E_{\text{self}}$ volgen als de
som van de individuele kernenergieën. Bovendien wijzigt de aanwezigheid van meerdere filamenten
de totale energie door de kruistermen van de snelheidsvelden (de kruisenergie). Deze
kruisenergie is vaak de drijvende kracht achter belangrijke fenomenen zoals het samensmelten van wervels of de `terugslag'-effecten
in golf-vortexinteracties.

\subsection{Definiëren en volgen van kruisenergie}
\label{sec:cross}
Wanneer meerdere wervelingen (of gedeeltelijke snelheidsverdelingen) naast elkaar bestaan, kan het totale snelheidsveld $\mathbf{u}$ worden gesuperponeerd:
\begin{equation}
\mathbf{u} \;=\; \mathbf{u}_1 \;+\;\mathbf{u}_2,\label{eq:appendix:superpose}
\end{equation}
waarbij $\mathbf{u}_1$ en $\mathbf{u}_2$ afkomstig zijn van verschillende subsystemen. In dat
scenario is de kinetische energie voor een vloeistofvolume $V$
\begin{align}
E_{\text{total}} &= \frac{\rho}{2} \int_V \mathbf{u}^2 \,dV
= \frac{\rho}{2} \int_V \bigl(\mathbf{u}_1 + \mathbf{u}_2 \bigr)^2\, dV \\
&= \frac{\rho}{2} \int_V \mathbf{u}_1^2 \,dV \;+\;\frac{\rho}{2} \int_V \mathbf{u}_2^2 \,dV
\;+\;\rho \int_V \mathbf{u}_1 \cdot \mathbf{u}_2 \, dV,
\end{align}
onthulling van een interactie of \emph{kruisenergie} term
\begin{equation}
E_{\text{cross}} \;=\; \rho \int_V \mathbf{u}_1 \cdot \mathbf{u}_2 \, dV.
\label{eq:cross-term}
\end{equation}
Veel van de interessante natuurkunde komt voort uit \eqref{eq:cross-term}, omdat deze
groeit of krimpt afhankelijk van de geometrie van de wervels en de afstand ertussen.
De dynamische evolutie ervan kan bijvoorbeeld leiden tot samensmelting of terugvering. Een belangrijk punt is dat
de eigensnelheid van elke wervel de onderlinge snelheden aanzienlijk kan beïnvloeden en zo
nettokrachten of koppel kan creëren.
\subsection{Toepassingen op heliciteit en topologische stromingen}
\label{sec:helicity}
Een verwant concept is heliciteit, waarmee de topologische complexiteit (knopen of
verbindingen) van wervelbuizen wordt gemeten. Klassiek wordt heliciteit $H$ gegeven door
\begin{equation}
H \;=\; \int_V \mathbf{u} \cdot \boldsymbol{\omega}\, dV,\label{eq:appendix:helicity}
\end{equation}
die constant kan blijven of gedeeltelijk verloren kan gaan tijdens herverbindingsgebeurtenissen. In bepaalde
dissipatieve stromingen kunnen de kruisenergietermen in \eqref{eq:cross-term} de effectieve snelheid van heliciteitsverandering beïnvloeden. Het begrijpen van $E_{\text{cross}}$ is belangrijk
voor het analyseren van herverbindingspaden in klassieke of superfluïde turbulentie.

\subsection{Afleidingsschema voor kruisenergie}
\label{sec:derivation}
Ten slotte geven we een beknopt schema voor het afleiden van de uitdrukking voor kruisenergie. Beginnend met het totale snelheidsveld $\mathbf{u} = \sum_{n=1}^N \mathbf{u}_n$
voor $N$ wervel- of partiële snelheidsvelden is de totale kinetische energie:
\begin{equation}
E_{\text{totaal}}
= \frac{\rho}{2} \int_V \left(\sum_{n=1}^N \mathbf{u}_n \right)^2 dV
= \frac{\rho}{2} \sum_{n=1}^N \int_V \mathbf{u}_n^2 \, dV
\;+\;\rho \sum_{n<m} \int_V \mathbf{u}_n \cdot \mathbf{u}_m \, dV.\label{eq:appendix:total-energy-derivation}
\end{equation}
Men verkrijgt $N$ zelfenergietermen plus paarsgewijze kruisenergie-integralen.
De kruisenergie voor een paar $(i,j)$ is:
\begin{equation}
E_{\text{cross}}^{(ij)} \;=\; \rho \int_V \mathbf{u}_i \cdot \mathbf{u}_j \, dV.\label{eq:appendix:cross-energy-derivation}
\end{equation}
In de praktijk kan elke $\mathbf{u}_n$ worden weergegeven door bekende oplossingen van de
Stokes- of potentiaalstroomvergelijkingen, of door benaderde oplossingen voor wervellussen. Vervolgens verkrijgt men, analytisch of numeriek, benaderde kruisenergieën
die gebruikt kunnen worden in gereduceerde modellen die de evolutie van multi-vortexsystemen beschrijven.

\subsection*{Conclusion}
We hebben onderzocht hoe de totale kinetische energie van vloeistoffen in de aanwezigheid van meerdere
vortices kan worden opgesplitst in termen van zelf- en kruisenergie. Deze bijdragen van kruisenergieën
zijn cruciaal voor het begrijpen van het samensmelten van wervels, het ontwarren van geknoopte wervels, of wervel-golfinteracties in klassieke, superfluïde en plasmastromen. Daarnaast hebben we een systematische afleiding van kruisenergie geschetst en
belangrijke aspecten benadrukt bij de bespreking van impuls en heliciteit. Toekomstige richtingen
zijn onder meer het verfijnen van deze uitdrukkingen voor axiaal symmetrische of geknoopte wervels en
het integreren ervan in grootschalige modellen of computationele kaders.\label{appendix:7}
    \section{Integration of Clausius' heat theory into VAM}\label{sec:appendix:8}

The integration of Clausius' mechanical heat theory into the Vortex Æther Model (VAM) extends the scope of the framework to thermodynamics,
enabling a unified interpretation of energy, entropy, and quantum behavior based on structured vorticity in a viscous, superfluid-like æther
medium \cite{clausius1865mechanical, maxwell1865electromagnetic, helmholtz1858integrals}.

\subsection{Thermodynamic Basics in VAM}

The classical first law of thermodynamics is expressed as follows:
\begin{equation}
    \Delta U = Q - W,\label{eq:first_law_thermodynamics}
\end{equation}
where $\Delta U$ is the change in internal energy, $Q$ is the added heat, and $W$ is the work done by the system \cite{clausius1865mechanical}. Within VAM this becomes:
\begin{equation}
    \Delta U = \Delta \left( \frac{1}{2} \rho_\text{\ae} \int v^2 \, dV + \int P \, dV \right),\label{eq:first_law_vam}
\end{equation}
with $\rho_\text{\ae}$ the æther density, $v$ the local velocity and $P$ the pressure within equilibrium vortex domains \cite{vam2025unified}.

\subsection{Entropy and structured vorticity}

VAM states that entropy is a function of vorticity intensity:
\begin{equation}
    S \propto \int \omega^2 \, dV,\label{eq:entropy_vorticity}
\end{equation}
where $\omega = \nabla \times v$ \cite{kelvin1867vortex}. Entropy thus becomes a measure of the topological complexity and energy dispersion encoded in the vortex network.

\subsection{Thermal response of vortex nodes}

Stable vortex nodes embedded in equilibrium pressure surfaces behave analogously to thermodynamic systems:
\begin{itemize}
    \item \textbf{Heating ($Q > 0$)} expands the node, decreases the core pressure, and increases the entropy. \item \textbf{Cooling ($Q < 0$)} causes a contraction of the node, concentrating energy and stabilizing the vorticity.
\end{itemize}
This provides a fluid mechanics analogy for gas laws under energetic input.

\subsection{Photoelectric analogy in VAM}

Instead of invoking quantized photons, VAM interprets the photoelectric effect via vortex dynamics. A vortex must absorb enough energy to destabilize and eject its structure:
\begin{equation}
    W = \frac{1}{2} \rho_\text{\ae} \int v^2 \, dV + P_\text{eq} V_\text{eq},\label{eq:photoelectric_work}
\end{equation}
where $W$ is the threshold for disintegration work. If an incident wave further modulates the internal vortex energy, ejection occurs \cite{vam2025unified}.

The critical force for vortex ejection is:
\begin{equation}
    F^{\text{max}}_{\text{\ae}} = \rho_\text{\ae} C_e^2 \pi r_c^2,\label{eq:critical_force}
\end{equation}
where $C_e$ is the edge velocity of the vortex and $r_c$ is the core radius. This provides a natural frequency limit below which no interaction occurs, comparable to the threshold frequency in quantum photoelectricity \cite{einstein1905photoelectric}.

\subsection*{Conclusion and integration}

This thermodynamic extension of VAM enriches the model by integrating classical heat and entropy principles into fluid dynamics. It not only bridges the gap between vortex physics and Clausius laws, but also provides a field-based reinterpretation of light-matter interactions, unifying mechanical and electromagnetic thermodynamics without discrete particle assumptions.\label{appendix:8}
    \input{WervelKlokken/appendix_09_TopologischeLading}\label{appendix:9}
    %! Author = Omar Iskandarani
%! Date = 5/23/2025

\section{Gesplitste Helicity in het Vortex Æther Model}

\subsection{Motivatie en Context}

In de klassieke stromingsleer beschrijft heliciteit de topologische complexiteit van wervelstructuren. Binnen het Vortex Æther Model (VAM), waarin materie wordt opgevat als knopen in een superfluïde Æther, is heliciteit essentieel voor de stabiliteit, energiedistributie en tijdsdilatatie.

Op basis van het werk van Tao et al.~\cite{Tao2021} splitsen we de totale heliciteit $H$ van een vortexbuis in twee componenten:
\begin{equation}
    H = H_C + H_T,
\end{equation}
waarbij:
\begin{itemize}
    \item $H_C$: de \textbf{centerline-heliciteit}, gekoppeld aan de geometrische vorm van de vortexas;
    \item $H_T$: de \textbf{twist-heliciteit}, bepaald door de rotatie van vortexlijnen rond deze as.
\end{itemize}

\subsection{Formulering van de Helicitycomponenten}

Voor een vortexbuis met vorticiteitsflux $C$ langs zijn centrale as, geldt:
\begin{align}
    H_C &= C^2 \cdot \text{Wr}, \\
    H_T &= C^2 \cdot \text{Tw}, \\
    H &= C^2 (\text{Wr} + \text{Tw}),
\end{align}
waarbij:
\begin{itemize}
    \item $\text{Wr}$: de \textbf{writhe}, een maat voor de globale kromming en zelfkoppeling van de vortexas;
    \item $\text{Tw}$: de \textbf{twist}, een maat voor de interne torsie van vortexlijnen om de as.
\end{itemize}

De writhe wordt berekend als:
\begin{equation}
    \text{Wr} = \frac{1}{4\pi} \int_C \int_C \frac{\left(\vec{T}(s) \times \vec{T}(s')\right) \cdot \left(\vec{r}(s) - \vec{r}(s')\right)}{|\vec{r}(s) - \vec{r}(s')|^3} \, ds \, ds',
\end{equation}
met $\vec{T}(s)$ de raakvector van de curve $C$.

\subsection{Toepassing in VAM-tijdsdilatatie}

De gesplitste heliciteit beïnvloedt de lokale klokfrequentie van een vortexdeeltje. We stellen voor:
\begin{equation}
    dt = dt_\infty \sqrt{1 - \frac{H_C + H_T}{H_{\text{max}}}} = dt_\infty \sqrt{1 - \frac{C^2 (\text{Wr} + \text{Tw})}{H_{\text{max}}}}.
\end{equation}

Deze formulering generaliseert de eerdere energiegebaseerde tijdsdilatatieformule, door topologische informatie expliciet te koppelen aan de tijdsverloop.\label{appendix:10}
    %! Author = Omar Iskandarani
%! Date = 5/23/2025

\section{VAM Lagrangian Based on Incompressible Schrödinger Flow}

\subsection{Complex Vortex Waves in Æther}

We model a vortex particle as a normalized two-fold complex wavefunction:
\[
    \psi(\vec{r}, t) = \begin{pmatrix} a + ib \\ c + id \end{pmatrix}, \quad |\psi|^2 = 1,
\]
from which the spin vector $\vec{s} = (s_1, s_2, s_3)$ and vortex field $\vec{\omega}$ are defined via a Hopf mapping.

\subsection{Lagrangian with Landau–Lifshitz-like term}

We define the VAM wavefunction Lagrangian as:
\begin{equation}
    \mathcal{L}_\text{VAM}[\psi] =
    \frac{i\hbar}{2} \left( \psi^\dagger \partial_t \psi - \psi \partial_t \psi^\dagger \right)
    - \frac{\hbar^2}{2m} |\nabla \psi|^2
    - \frac{\alpha}{8} |\nabla \vec{s}|^2,
\end{equation}
where:
\begin{itemize}
    \item $\hbar$ is replaced by a VAM-conformal quantization constant,
    \item $\alpha$ is a dimensionless vortex coupling constant,
    \item $\vec{s}$ is the Hopf spin vector, calculated from $\psi$ via:
    \[
        s_1 = a^2 + b^2 - c^2 - d^2, \quad
        s_2 = 2(bc - ad), \quad
        s_3 = 2(ac + bd).
    \]
\end{itemize}

\subsection{Derivation of the VAM field equation}

Variation with respect to $\psi^*$ yields the modified ISF equation:
\[
    i\hbar \frac{\partial \psi}{\partial t} =
    - \frac{\hbar^2}{2m} \nabla^2 \psi
    + \frac{\alpha}{4} \frac{\delta}{\delta \psi^*} |\nabla \vec{s}|^2.
\]

The derived Euler-Lagrange equation contains topological feedback of the nodal structure on the time evolution of the wave.

\subsection{Physical Interpretation}

This formulation allows us to:
\begin{enumerate}
    \item Describe quantum superposition of vortex particles; \item Derive VAM time delay from the helicity of $\vec{s}$;
    \item Coupling stability of vortex nodes to an effective potential \( V(\vec{s}) \sim |\nabla \vec{s}|^2 \);
    \item Simulate evolution without using classical Navier–Stokes dissipation.
\end{enumerate}\label{appendix:11}
\end{document}
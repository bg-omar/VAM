\section{Experimental tests and observational predictions of VAM}

\subsection{1. Time dilation in rotating superfluids}

The Vortex Æther Model predicts that in a superfluid vortex core, local time slows down as the angular velocity $\Omega_k$ increases. This is experimentally testable in:
\begin{itemize}
    \item Bose–Einstein condensates (BECs) with coherent rotating states,
    \item Rotating superfluid helium carriers with internal frequency measurements (e.g. neutron spin resonance),
    \item Similar systems with laser-induced vorticity.
\end{itemize}

Differences in time course or phase between rotating and non-rotating atomic clocks can be taken as a test for Æther time modulation without curvature.~\cite{Steinhauer2016}

\subsection{2. Plasma vortex clocks and cyclotron analogies}

Cyclotron fields, annular plasma rotations or rotating magnetic traps generate gradients in $\Omega(r)$. According to VAM this leads to measurable clock distortion. Experimental predictions:
\begin{itemize}
    \item Phase differentiation in optical pulses along plasma vortex edges,~\cite{Unruh1981}
    \item Changes in radiative emission patterns in asymmetric vortex plasmas.
\end{itemize}

\subsection{3. Optical and metamaterial analogs}

As with analogue gravity, synthetic waveguides or metamaterials can simulate “æther flow”. Here:
\begin{itemize}
    \item Light propagation is affected by artificial rotational flows,
    \item Can simulate anisotropic refractive index which mimics VAM light deflection,
    \item Can dispersion analysis provide insight into local time delay.
\end{itemize}

\subsection{4. Expected observational features}

Experimental signatures of VAM can be:
\begin{enumerate}
    \item Boundary values ​​for vortex node collapse with sudden energy release,
    \item Local time anomalies in rotating laboratory systems,
    \item Absent relativistic acceleration in energetically favorable vortex systems,
    \item Non-symmetric clock rates on different sides of a vortex core.
\end{enumerate}
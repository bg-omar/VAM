\section{Foundations of Velocity Fields and Energies in a Vortex System.}

\subsection{Introduction}
Vortex dynamics are a core component of many fluid and plasma systems, including
tornado-like flows, knotted vortices in classical or superfluid turbulence, and various
complex topological fluid systems. A deeper understanding of the energy budgets
associated with these flows can shed light on processes like vortex stability, reconnection,
and global flow organization. We begin by motivating how velocity fields can be
decomposed so as to capture the total energy (i.e.\ self- plus cross-energy), and how
this approach helps track flows in both 2D and 3D.

\subsection{Foundations: Velocity Fields and Total (Self + Cross) Energy}
\label{sec:foundations}
In an incompressible fluid, the velocity field $\mathbf{u}(\mathbf{x}, t)$ is typically
governed by the Navier--Stokes or Euler equations. For inviscid analyses, the Euler
equations for incompressible flow read
\begin{equation}
  \frac{\partial \mathbf{u}}{\partial t} + (\mathbf{u} \cdot \nabla)\mathbf{u} = -\frac{1}{\rho}\nabla p,
  \quad \nabla \cdot \mathbf{u} = 0.\label{eq:appendix:Euler}
\end{equation}
We also consider the vorticity $\boldsymbol{\omega} = \nabla \times \mathbf{u}$,
which can be used to characterize vortex structures.

To understand the \emph{total} kinetic energy, we can split it as follows:
\begin{equation}
  E_{\text{total}} \;=\; E_{\text{self}} \;+\; E_{\text{cross}}.\label{eq:appendix:total-energy}
\end{equation}
Here, $E_{\text{self}}$ is that portion of energy which each vortex or partial flow
element contributes independently (for instance, from local swirling motions), while
$E_{\text{cross}}$ encodes the contributions that arise from the interaction of different
vortical elements. In a multi-vortex scenario, such a decomposition helps isolate the
direct interaction between two (or more) vortex filaments or sheets.

\subsection{Momentum and Self-Energy Considerations}
\label{sec:momentum}
A starting point is to recall that for a single vortex of circulation $\Gamma$, with an
azimuthally symmetric core, the induced velocity is sometimes approximated by
classical results such as
\begin{equation}
   V \;=\; \frac{\Gamma}{4 \pi R}
   \bigl(\ln \tfrac{8 R}{a} - \beta \bigr),\label{eq:appendix:velocity}
\end{equation}
where $R$ is the main vortex loop radius, $a \ll R$ is a measure of core thickness,
and $\beta$ depends on details of the core model \cite{Saffman1992}. The
\emph{self-energy} associated with that vortex, $E_{\text{self}}$, can be cast in a
similar form that depends on $\ln(R/a)$, exemplifying how thin-core vortices'
energies scale with geometry.

In more general fluid or vortex-lattice models, we can track $E_{\text{self}}$ as the
sum of individual core energies. Further, the presence of multiple filaments modifies
the total energy by cross-terms of the velocity fields (the cross-energy). This
cross-energy often drives key phenomena such as vortex merging or the `recoil'
effects in wave--vortex interactions.

\subsection{Defining and Tracking Cross-Energy}
\label{sec:cross}
When multiple vortices (or partial velocity distributions) co-exist, the total velocity
field $\mathbf{u}$ can be superposed:
\begin{equation}
   \mathbf{u} \;=\; \mathbf{u}_1 \;+\;\mathbf{u}_2,\label{eq:appendix:superpose}
\end{equation}
where $\mathbf{u}_1$ and $\mathbf{u}_2$ come from distinct sub-systems. In that
scenario, the kinetic energy for a fluid volume $V$ is
\begin{align}
   E_{\text{total}} &= \frac{\rho}{2} \int_V \mathbf{u}^2 \,dV
   = \frac{\rho}{2} \int_V \bigl(\mathbf{u}_1 + \mathbf{u}_2 \bigr)^2\, dV \\
   &= \frac{\rho}{2} \int_V \mathbf{u}_1^2 \,dV \;+\;\frac{\rho}{2} \int_V \mathbf{u}_2^2 \, dV
   \;+\;\rho \int_V \mathbf{u}_1 \cdot \mathbf{u}_2 \, dV,
\end{align}
revealing an interaction or \emph{cross-energy} term
\begin{equation}
   E_{\text{cross}} \;=\; \rho \int_V \mathbf{u}_1 \cdot \mathbf{u}_2 \, dV.
   \label{eq:cross-term}
\end{equation}
Much of the interesting physics arises from \eqref{eq:cross-term}, because it
grows or shrinks depending on the vortex geometry and distance between them.
Its dynamical evolution can lead to, e.g., merging or rebound. A main point is that
each vortex's self-velocity can significantly affect the mutual velocities and thus
create net forces or torque.

\subsection{Applications to Helicity and Topological Flows}
\label{sec:helicity}
A related concept is helicity, measuring the topological complexity (knotting or
linking) of vortex tubes. Classically, helicity $H$ is given by
\begin{equation}
   H \;=\; \int_V \mathbf{u} \cdot \boldsymbol{\omega}\, dV,\label{eq:appendix:helicity}
\end{equation}
which can remain constant or be partially lost during reconnection events. In certain
dissipative flows, the cross-energy terms in \eqref{eq:cross-term} can influence
the effective rate of helicity change. Understanding $E_{\text{cross}}$ is important
for analyzing reconnection pathways in classical or superfluid turbulence.

\subsection{Derivation Outline for Cross-Energy}
\label{sec:derivation}
Finally, we provide a succinct outline for deriving the cross-energy expression.
Starting with the total velocity field $\mathbf{u} = \sum_{n=1}^N \mathbf{u}_n$
for $N$ vortex or partial velocity fields, the total kinetic energy is:
\begin{equation}
   E_{\text{total}}
   = \frac{\rho}{2} \int_V \left(\sum_{n=1}^N \mathbf{u}_n \right)^2 dV
   = \frac{\rho}{2} \sum_{n=1}^N \int_V \mathbf{u}_n^2 \, dV
      \;+\;\rho \sum_{n<m} \int_V \mathbf{u}_n \cdot \mathbf{u}_m \, dV.\label{eq:appendix:total-energy-derivation}
\end{equation}
One obtains $N$ self-energy terms plus pairwise cross-energy integrals.
The cross-energy for a pair $(i,j)$ is:
\begin{equation}
   E_{\text{cross}}^{(ij)} \;=\; \rho \int_V \mathbf{u}_i \cdot \mathbf{u}_j \, dV.\label{eq:appendix:cross-energy-derivation}
\end{equation}
In practice, each $\mathbf{u}_n$ may be represented by known solutions of the
Stokes or potential flow equations, or from approximate solutions for vortex loops.
Then, either analytically or numerically, one obtains approximate cross-energies
that can be used in reduced models describing the evolution of multi-vortex
systems.

\subsection*{Conclusion}
We have surveyed how the total fluid kinetic energy in the presence of multiple
vortices can be split into self- and cross-energy terms. These cross-energy
contributions are crucial for understanding vortex merging, knotted vortex
untangling, or vortex–wave interactions in classical, superfluid, and plasma
flows. In addition, we have sketched a systematic derivation of cross-energy and
highlighted key aspects in discussing momentum and helicity. Future directions
include refining these expressions for axisymmetric or knotted vortices and
integrating them into large-scale models or computational frameworks.
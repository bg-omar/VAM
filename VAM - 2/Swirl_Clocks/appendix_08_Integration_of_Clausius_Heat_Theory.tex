

\section{Integration of Clausius' Heat Theory into the Vortex \AE ther Model (VAM)}

The integration of Clausius' Mechanical Theory of Heat into the Vortex \AE ther Model (VAM) extends the framework's reach into thermodynamics,
allowing a unified interpretation of energy, entropy, and quantum behavior based on structured vorticity in an inviscid superfluid-like \ae ther
medium \cite{clausius1865mechanical, maxwell1865electromagnetic, helmholtz1858integrals}.

\subsection{Thermodynamic First Principles in VAM}

The classical first law of thermodynamics is expressed as:
\begin{equation}
\Delta U = Q - W,\label{eq:first_law_thermodynamics}
\end{equation}
where $\Delta U$ is the change in internal energy, $Q$ is heat added, and $W$ is work done by the system \cite{clausius1865mechanical}. Within VAM, this becomes:
\begin{equation}
\Delta U = \Delta \left( \frac{1}{2} \rho_{\text{\ae}} \int v^2 \, dV + \int P \, dV \right),\label{eq:first_law_vam}
\end{equation}
with $\rho_{\text{\ae}}$ the æther density, $v$ the local velocity, and $P$ the pressure within equilibrium vortex domains \cite{vam2025unified}.

\subsection{Entropy and Structured Vorticity}

VAM posits that entropy is a function of vorticity intensity:
\begin{equation}
S \propto \int \omega^2 \, dV,\label{eq:entropy_vorticity}
\end{equation}
where $\omega = \nabla \times v$ \cite{kelvin1867vortex}. Thus, entropy becomes a measure of topological complexity and energy dispersion encoded in the vortex network.

\subsection{Thermal Response of Vortex Knots}

Stable vortex knots embedded in equilibrium pressure surfaces behave analogously to thermodynamic systems:
\begin{itemize}
\item \textbf{Heating ($Q > 0$)} expands the knot, lowers core pressure, and increases entropy.
\item \textbf{Cooling ($Q < 0$)} contracts the knot, concentrating energy and stabilizing vorticity.
\end{itemize}
This provides a fluid-mechanical analog to gas laws under energetic input.

\subsection{Photoelectric Analogy in VAM}

Rather than invoking quantized photons, VAM interprets the photoelectric effect through vortex dynamics. A vortex must absorb enough energy to destabilize and eject its structure:
\begin{equation}
W = \frac{1}{2} \rho_{\text{\ae}} \int v^2 \, dV + P_{\text{eq}} V_{\text{eq}},\label{eq:photoelectric_work}
\end{equation}
where $W$ is the disintegration work threshold. If an incident wave modulates internal vortex energy beyond this, ejection occurs \cite{vam2025unified}.

The critical force for vortex ejection is:
\begin{equation}
F_{\max} = \rho_{\text{\ae}} C_e^2 \pi r_c^2,\label{eq:critical_force}
\end{equation}
with $C_e$ the vortex's edge velocity and $r_c$ its core radius. This yields a natural frequency cutoff below which no interaction occurs, akin to the threshold frequency in quantum photoelectricity \cite{einstein1905photoelectric}.

\subsection*{Conclusion and Integration}

This thermodynamic extension of VAM enriches the model by embedding classical heat and entropy principles within fluid-dynamic structures. It not only bridges vortex physics with Clausius' laws but also offers a field-based reinterpretation of light-matter interactions, unifying mechanical and electromagnetic thermodynamics without discrete particle assumptions.


\section{Tijdsdilatatie vanuit vortex dynamiek}

In het Vortex Æther Model (VAM) ontstaat tijdsdilatatie niet vanuit de kromming van ruimtetijd, maar vanuit lokale vortex dynamica. Elk materiedeeltje is in VAM een vortex-knoopstructuur waarvan de interne rotatie (\textit{swirl}) de lokale klokfrequentie beïnvloedt.

De fundamentele koppeling tussen lokale vortex-snelheid en de lokale tijdsmeting volgt uit de Bernoulli-achtige relatie voor drukverlaging in stromingsvelden. De lokale klokfrequentie is gerelateerd aan de vortex-tangentiële snelheid $v_{\phi}(r)$ via de formule:
\begin{equation}\label{eq:vortex_tijdsdilatatie}
    \frac{d\tau}{dt} = \sqrt{1 - \frac{v_{\phi}^2(r)}{c^2}}
\end{equation}

Hierbij is $v_{\phi}(r)$ de tangentiële snelheid van het æthermedium op afstand $r$ tot het centrum van de vortex, en $c$ de lichtsnelheid. Dit is een directe analogie met de speciale relativistische snelheidsafhankelijke tijddilatatie, echter zonder ruimtetijdkromming en louter veroorzaakt door lokale rotatie van het æthermedium.

\subsection{Afleiding vanuit vortex hydrodynamica}

De afleiding volgt uit het Bernoulli-principe voor een ideale vloeistofstroming, gegeven door:
\begin{equation}\label{eq:Bernoulli}
    P + \frac{1}{2}\rho_{\ae} v^2 = \text{constant}
\end{equation}

Met vortex-stroming geïntroduceerd via vorticiteit $\vec{\omega} = \nabla \times \vec{v}$, definieert de lokale drukverlaging ten opzichte van de verre omgeving een lokale tijdvertraging. De lokale vortexsnelheid is gegeven door:
\begin{equation}\label{eq:tangentiele_snelheid}
    v_{\phi}(r) = \frac{\Gamma}{2\pi r} = \frac{\kappa}{r}
\end{equation}

waarbij $\Gamma$ de circulatieconstante is, en $\kappa$ het circulatiekwantum. Substitutie van \eqref{eq:tangentiele_snelheid} in \eqref{eq:vortex_tijdsdilatatie} geeft expliciet:
\begin{equation}\label{eq:vortex_tijd_expliciet}
    \frac{d\tau}{dt} = \sqrt{1 - \frac{\kappa^2}{c^2 r^2}}
\end{equation}

Hiermee is de tijdsdilatatie expliciet uitgedrukt in fundamentele vortex-parameters.

\subsection{Vergelijking met algemene relativiteit}

Ter vergelijking, in algemene relativiteit (GR) ontstaat gravitationele tijddilatatie uit ruimtetijdkromming, uitgedrukt door de Schwarzschildmetriek~\cite{schutz2009first}:
\begin{equation}\label{eq:GRtijd}
    \frac{d\tau}{dt} = \sqrt{1 - \frac{2GM}{rc^2}}
\end{equation}

De overeenkomsten en verschillen zijn direct zichtbaar: GR's gravitationele tijddilatatie is gerelateerd aan massa $M$ en gravitatieconstante $G$, terwijl VAM tijdsdilatatie puur hydrodynamisch is en direct verbonden met de lokale rotatiesnelheid van het æthermedium via vortex-circulatie $\kappa$.

\begin{figure}[ht!]
    \centering
    % Plaats hier een eigen grafiek of illustratie.
    \includegraphics[width=0.7\linewidth]{tijdsdilatatie_VAM_GR.png}
    \caption{Vergelijking tussen VAM- (vortex dynamiek) en GR-tijdsdilatatie, als functie van afstand tot vortexkern en Schwarzschildradius.}
    \label{fig:vergelijkingVAMGR}
\end{figure}

In Figuur~\ref{fig:vergelijkingVAMGR} zien we dat de VAM-tijdsdilatatie functioneel vergelijkbaar is met GR-prediction bij voldoende afstand. Bij afnemende afstand (nabij vortexkern of Schwarzschildradius) ontstaan verschillen door vortex-specifieke effecten en topologische knoopstructuren.

Samenvattend vervangt het VAM ruimtetijdkromming door werveldynamica, met behoud van meetbare tijddilatatie-effecten die overeenstemmen met gevestigde experimentele resultaten zoals Hafele–Keating~\cite{hafele1972around}, maar vanuit een fundamenteel andere fysische verklaring.


\section{Tijdsmodulatie door rotatie van wervelknopen}

Voortbouwend op de behandeling van tijdsdilatatie via druk en Bernoulli-dynamica in de vorige sectie, richten we ons nu op de intrinsieke rotatie van topologische wervelknopen. In het Vortex Æther Model (VAM) worden deeltjes gemodelleerd als stabiele, topologisch behouden wervelknopen ingebed in een onsamendrukbaar, niet-viskeus superfluïde medium. Elke knoop bezit een karakteristieke interne hoekfrequentie $\Omega_k$, en deze interne beweging induceert lokale tijdmodulatie ten opzichte van de absolute tijd van de æther.

In plaats van het krommen van de ruimtetijd, stellen we voor dat interne rotatie-energie en heliciditeitsbehoud temporele vertragingen veroorzaken die analoog zijn aan gravitationele roodverschuiving. In deze sectie worden deze ideeën uitgewerkt met behulp van heuristische en energetische argumenten die consistent zijn met de hiërarchie die in Sectie I is geïntroduceerd.

\subsection{Heuristische en energetische afleiding}

We beginnen met het voorstellen van een rotatiegeïnduceerde tijdsdilatatieformule op basis van de interne hoekfrequentie van de knoop:

\begin{equation}
\frac{t_{\text{local}}}{t_{\text{abs}}} = \left(1 + \beta \Omega_k^2 \right)^{-1}\label{eq:rotational_induced_time_dilation}
\end{equation}

waarbij:

\begin{itemize}
\item $t_{\text{local}}$ de eigentijd nabij de knoop is,
\item $t_{\text{abs}}$ de absolute tijd van de achtergrond-æther is,
\item $\Omega_k$ de gemiddelde kernhoekfrequentie is frequentie,
\item $\beta$ is een koppelingscoëfficiënt met dimensies $[\beta] = \text{s}^2$.
\end{itemize}

Voor kleine hoeksnelheden verkrijgen we een eerste-orde-expansie:

\begin{equation}
\frac{t_{\text{local}}}{t_{\text{abs}}} \approx 1 - \beta \Omega_k^2 + \mathcal{O}(\Omega_k^4)\label{eq:rotational_induced_time_dilation_expansion}
\end{equation}

Deze vorm loopt parallel met de Lorentzfactor bij lage snelheden in de speciale relativiteitstheorie:

\begin{equation}
\frac{t_{\text{moving}}}{t_{\text{rest}}} \approx 1 - \frac{v^2}{2c^2}\label{eq:parallels_lorentz_time_dilation}
\end{equation}

Dit levert een belangrijke analogie op: Interne rotatiebeweging in VAM induceert tijdvertraging, vergelijkbaar met hoe translationele snelheid tijddilatatie induceert in SR.

Om de fysische basis van deze uitdrukking te versterken, relateren we tijddilatatie nu aan de energie die is opgeslagen in wervelrotatie. Stel dat de wervelknoop een effectief traagheidsmoment $I$ heeft. De rotatie-energie wordt gegeven door:

\begin{equation}
E_{\text{rot}} = \frac{1}{2} I \Omega_k^2\label{eq:rotational_energy_inertia}
\end{equation}

Aannemende dat de tijd vertraagt door deze energiedichtheid, schrijven we:

\begin{equation}
\frac{t_{\text{local}}}{t_{\text{abs}}} = \left(1 + \beta E_{\text{rot}} \right)^{-1} = \left(1 + \frac{1}{2} \beta I \Omega_k^2 \right)^{-1}\label{eq:time_dilation_rotational_energy_inertia}
\end{equation}

Deze uitdrukking dient als de energetische analoog van het op druk gebaseerde Bernoulli-model uit Sectie I (zie vgl. ~\eqref{eq:localtime_vortex}). Het ondersteunt de interpretatie van vortex-geïnduceerde tijdsputten via energieopslag in plaats van geometrische deformatie.

\subsection{Topologische en fysische rechtvaardiging}

Topologische vortexknopen worden niet alleen gekenmerkt door rotatie, maar ook door heliciteit:

\begin{equation}
H = \int \vec{v} \cdot \vec{\omega} \, d^3x \label{eq:helicity_rotation}
\end{equation}

Heliciteit is een behouden grootheid in ideale (onzichtbare, onsamendrukbare) vloeistoffen, die de verbinding en draaiing van vortexlijnen codeert. De rotatiefrequentie $\Omega_k$ wordt een topologisch betekenisvolle indicator van de identiteit en dynamische toestand van de knoop.

Hogere Omega_k-waarden duiden op meer rotatie-energie en diepere drukputten, wat leidt tot tijdelijke vertragingen die lijken op gravitationele roodverschuiving, maar zonder dat er sprake is van ruimtetijdkromming.

Elk deeltje is een topologische vortexknoop:
\begin{itemize}
\item Lading $\leftrightarrow$ draaiing of chiraliteit van de knoop
\item Massa $\leftrightarrow$ geïntegreerde vorticiteitsenergie
\item Spin $\leftrightarrow$ knoophelix:
\end{itemize}
Stabiliteit $\leftrightarrow$ knooptype (Hopf-verbindingen, Trefoil, enz.) en energieminimalisatie in de vortexkern

Dit model:

\begin{itemize}
\item Schrijft tijdmodulatie toe aan behouden, intrinsieke rotatie-energie,
\item Vereist geen externe referentiekaders (absolute æthertijd is universeel),
\item Behoudt temporele isotropie buiten de vortexkern,
\item Biedt een natuurlijke vervanging voor de ruimtetijdkromming van GR. \end{itemize}

Daarom biedt dit vortex-energetische tijdsdilatatieprincipe een krachtig alternatief voor relativistische tijdmodulatie door alle temporele effecten te verankeren in rotatie-energetica en topologische invarianten.

In de volgende sectie zullen we laten zien hoe deze ideeën metriekachtig gedrag reproduceren voor roterende waarnemers, inclusief een directe vloeistofmechanische analoog aan de Kerr-metriek van de algemene relativiteitstheorie.
\section{Eigen tijd voor een roterende waarnemer in ætherstroming}

Nu we tijdsdilatatie hebben vastgesteld in het Vortex æther Model (VAM) door middel van druk, hoeksnelheid en rotatie-energie, breiden we ons formalisme nu uit naar roterende waarnemers. Deze sectie toont aan dat vloeistofdynamische tijdmodulatie in VAM uitdrukkingen kan reproduceren die structureel vergelijkbaar zijn met die afgeleid uit de algemene relativiteitstheorie (GR), met name in axiaal symmetrische roterende ruimtetijden zoals de Kerr-geometrie. VAM bereikt dit echter zonder ruimtetijdkromming aan te roepen. In plaats daarvan wordt tijdmodulatie volledig bepaald door kinetische variabelen in het ætherveld.

\subsection{GR-propertijd in roterende frames}

In de algemene relativiteitstheorie wordt de eigentijd \(d\tau\) voor een waarnemer met hoeksnelheid \(\Omega_{\text{eff}}\) in een stationaire, axiaal symmetrische ruimtetijd gegeven door:

\begin{equation}
\left( \frac{d\tau}{dt} \right)^2_{\text{GR}} = -\left[ g_{tt} + 2g_{t\varphi} \Omega_{\text{eff}} + g_{\varphi\varphi} \Omega_{\text{eff}}^2 \right]
\label{eq:GR_proper_time}
\end{equation}

waarbij \(g_{\mu\nu}\) componenten zijn van de ruimtetijdmetriek (bijv. in Boyer-Lindquist coördinaten voor Kerr-ruimtetijd). Deze formulering houdt rekening met zowel gravitationele roodverschuiving als rotatie-effecten (frame-dragging).

\subsection{Æther-gebaseerde analogie: Snelheidsafgeleide tijdmodulatie}

In VAM is de ruimtetijd niet gekromd. Waarnemers bevinden zich in plaats daarvan in een dynamisch gestructureerde æther waarvan de lokale stroomsnelheden de tijddilatatie bepalen. Laat de radiale en tangentiële componenten van de æthersnelheid zijn:

\begin{itemize}
\item \(v_r\): radiale snelheid,
\item \(v_\varphi = r\Omega_k\): tangentiële snelheid als gevolg van lokale wervelrotatie,
\item \(\Omega_k = \frac{\kappa}{2\pi r^2}\): lokale hoeksnelheid (met \(\kappa\) als circulatie).
\end{itemize}

We postuleren een correspondentie tussen GR-metrische componenten en andere snelheidstermen:

\begin{equation}
\begin{aligned}
g_{tt} &\rightarrow -\left(1 - \frac{v_r^2}{c^2}\right), \\
g_{t\varphi} &\rightarrow -\frac{v_r v_\varphi}{c^2}, \\
g_{\varphi\varphi} &\rightarrow -\frac{v_\varphi^2}{c^2 r^2}
\end{aligned}
\label{eq:VAM_metric_terms}
\end{equation}

Door deze in de GR-expressie voor de juiste tijd te substitueren, verkrijgen we de VAM-gebaseerde analoog:

\begin{equation}
\left( \frac{d\tau}{dt} \right)^2_{\text{\ae}} = 1 - \frac{v_r^2}{c^2} - \frac{2v_r v_\varphi}{c^2} - \frac{v_\varphi^2}{c^2}
\label{eq:VAM_proper_time}
\end{equation}

De termen combineren:

\begin{equation}
\left( \frac{d\tau}{dt} \right)^2_{\text{\ae}} = 1 - \frac{1}{c^2}(v_r + v_\varphi)^2
\label{eq:VAM_proper_time_combined}
\end{equation}

Deze formulering reproduceert gravitationele en frame-dragging tijdseffecten puur uit de etherdynamica: $\langle \omega^2 \rangle$ speelt de rol van gravitationele roodverschuiving en circulatie $\kappa$ codeert rotatieweerstand. Deze benadering sluit aan bij recente vloeistofdynamische interpretaties van zwaartekracht en tijd \cite{barcelo2011analogue}, \cite{fedi2017gravity}.
Dit model gaat momenteel uit van rotatievrije stroming buiten knopen en verwaarloost viscositeit, turbulentie en kwantumcompressibiliteit. Toekomstige uitbreidingen kunnen gekwantiseerde circulatiespectra of grenseffecten in beperkte ethersystemen omvatten.

\begin{equation}
\boxed{\left( \frac{d\tau}{dt} \right)^2_{\text{\ae}} = 1 - \frac{1}{c^2}(v_r + r\Omega_k)^2}
\label{eq:VAM_proper_time_final}
\end{equation}

\subsection{Fysische interpretatie en modelconsistentie}

Dit resultaat in het kader weerspiegelt de GR-uitdrukking voor roterende waarnemers, maar komt strikt voort uit de klassieke vloeistofdynamica. Het laat zien dat naarmate de lokale ethersnelheid de lichtsnelheid nadert – door radiale instroom of rotatiebeweging – de eigentijd vertraagt. Dit impliceert het bestaan ​​van "tijdputten" waar de kinetische energiedichtheid domineert.

Belangrijkste observaties:

\begin{itemize}
\item Bij afwezigheid van radiale stroming (\(v_r = 0\)) ontstaat tijdvertraging volledig door vortexrotatie.
\item Wanneer zowel \(v_r\) als \(\Omega_k\) aanwezig zijn, verlaagt de cumulatieve snelheid de lokale tijdssnelheid.
\item Deze uitdrukking komt overeen met het energetische model van Sectie II als we \(v_r + r\Omega_k\) interpreteren als een bijdrage aan de lokale energiedichtheid.
\end{itemize}

In het VAM-kader komt de structuur van de eigentijd van de waarnemer dus voort uit aetherische stromingsvelden. Dit bevestigt dat GR-achtig temporeel gedrag kan ontstaan ​​in een vlakke, Euclidische 3D-ruimte met absolute tijd, volledig bepaald door gestructureerde vorticiteit en circulatie.

In het volgende gedeelte onderzoeken we hoe VAM deze overeenkomst uitbreidt naar gravitatiepotentialen en frame-dragging-effecten via circulatie en vorticiteitsintensiteit, en zo een analogie vormt voor de Kerr-tijd-roodverschuivingsformule.
\section{Kerr-achtige tijdsaanpassing op basis van vorticiteit en circulatie}

Om de analogie tussen de algemene relativiteitstheorie (GR) en het vortex-ethermodel (VAM) te voltooien, leiden we nu een tijdmodulatieformule af die de roodverschuiving en frame-dragging-structuur in de Kerr-oplossing weerspiegelt. In GR beschrijft de Kerr-metriek de ruimtetijdgeometrie rond een roterende massa en voorspelt zowel gravitationele tijdsdilatatie als frame-dragging als gevolg van impulsmoment. VAM legt vergelijkbare verschijnselen vast via de dynamiek van gestructureerde vorticiteit en circulatie in de ether, zonder dat ruimtetijdkromming nodig is.

\subsection{Algemene relativistische Kerr-roodverschuivingsstructuur}

In de GR-Kerr-metriek wordt de eigentijd $d\tau$ voor een waarnemer nabij een roterende massa beïnvloed door zowel massa-energie als impulsmoment. Een vereenvoudigde benadering voor de tijddilatatiefactor nabij een roterend lichaam is:
\begin{equation}
t_{\text{adjusted}} = \Delta t \cdot \sqrt{1 - \frac{2GM}{rc^2} - \frac{J^2}{r^3c^2}}
\label{eq:Kerr_time_dilation}
\end{equation}
waarbij:
\begin{itemize}
\item $M$: massa van het roterende lichaam,
\item $J$: impulsmoment,
\item $r$: radiale afstand tot de bron,
\item $G$: gravitatieconstante van Newton,
\item $c$: lichtsnelheid.
\end{itemize}

De eerste term correspondeert met gravitationele roodverschuiving ten opzichte van de massa, terwijl de tweede rekening houdt met rotatie-effecten (frame-dragging).

\subsection{Æther analoog via vorticiteit en circulatie}

In VAM drukken we gravitatieachtige invloeden uit via vorticiteitsintensiteit $\langle \omega^2 \rangle$ en totale circulatie $\kappa$. Deze worden geïnterpreteerd als:
\begin{itemize}
\item $\langle \omega^2 \rangle$: gemiddelde kwadratische vorticiteit over een gebied,
\item $\kappa$: behouden circulatie, coderend voor impulsmoment.
\end{itemize}

We definiëren de ethergebaseerde analoog door de volgende vervangingen uit te voeren:
\begin{equation}
\begin{aligned}
\frac{2GM}{rc^2} &\rightarrow \frac{\gamma \langle \omega^2 \rangle}{rc^2}, \\
\frac{J^2}{r^3c^2} &\rightarrow \frac{\kappa^2}{r^3c^2}
\end{aligned}
\label{eq:Kerr_replacements}
\end{equation}

Hier is $\gamma$ een koppelingsconstante die de vorticiteit relateert aan de effectieve zwaartekracht (analoog aan $G$). De op ether gebaseerde eigentijd wordt dan:

\begin{equation}
\boxed{t_{\text{adjusted}} = \Delta t \cdot \sqrt{1 - \frac{\gamma \langle \omega^2 \rangle}{rc^2} - \frac{\kappa^2}{r^3c^2}}}
\label{eq:Kerr_time_dilation_ae}
\end{equation}

Dit weerspiegelt de Kerr-roodverschuiving en frame-dragging-structuur met behulp van vloeistofdynamische variabelen. In deze afbeelding:
\begin{itemize}
\item $\langle \omega^2 \rangle$ speelt de rol van energiedichtheid die gravitationele roodverschuiving produceert,
\item $\kappa$ vertegenwoordigt impulsmoment dat tijdelijke frame-dragging genereert,
\item De vergelijking reduceert tot een vlakke æthertijd ($t_{\text{aangepast}} \tot \Delta t$) wanneer beide termen verdwijnen.
\end{itemize}

\subsection{Modelaannames en reikwijdte}

Dit resultaat is afhankelijk van verschillende aannames:
\begin{itemize}
\item De stroming is rotatievrij buiten de vortexkernen,
\item Viscositeit en turbulentie worden verwaarloosd,
\item Samendrukbaarheid wordt genegeerd (ideale onsamendrukbare superfluïde),
\item Vorticiteitsvelden zijn voldoende glad om $\langle \omega^2 \rangle$ te definiëren.
\end{itemize}

Deze omstandigheden weerspiegelen de aannames van analoge modellen van ideale vloeistof-GR. De formulering overbrugt de macroscopische stromingsdynamica van de ether met effectieve geometrische voorspellingen, wat de mogelijkheid versterkt om gekromde ruimtetijd te vervangen door gestructureerde vorticiteitsvelden.

Zie Appendix~\ref{appendix:1} voor gedetailleerde afleidingen van kruisenergie- en vortexinteractie-energetica.

In toekomstig werk kunnen correcties voor randvoorwaarden, gekwantiseerde vorticiteitsspectra en compressibele effecten worden toegevoegd om de analogie te verfijnen. Vervolgens vatten we samen hoe deze vloeistofgebaseerde tijddilatatiemechanismen zich verenigen binnen het VAM-kader en identificeren we hun experimentele implicaties.
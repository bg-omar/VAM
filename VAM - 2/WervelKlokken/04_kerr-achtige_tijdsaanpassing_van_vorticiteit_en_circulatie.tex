\section{Kerr-achtige tijdsaanpassing op basis van vorticiteit en circulatie}

Om de analogie tussen de algemene relativiteitstheorie (GR) en het wervel-æthermodel (VAM) te voltooien, leiden we nu een tijdmodulatieformule af die de roodverschuiving en frame-dragging-structuur in de Kerr-oplossing weerspiegelt. In GR beschrijft de Kerr-metriek de ruimtetijdgeometrie rond een roterende massa en voorspelt zowel gravitationele tijdsdilatatie als frame-dragging als gevolg van impulsmoment. VAM legt vergelijkbare verschijnselen vast via de dynamiek van gestructureerde vorticiteit en circulatie in de æther, zonder dat ruimtetijdkromming nodig is.

\subsection{Algemene relativistische Kerr-roodverschuivingsstructuur}

In de GR-Kerr-metriek wordt de eigentijd $d\tau$ voor een waarnemer nabij een roterende massa beïnvloed door zowel massa-energie als impulsmoment. Een vereenvoudigde benadering voor de tijddilatatiefactor nabij een roterend lichaam is:
\begin{equation}
t_{\text{adjusted}} = \Delta t \cdot \sqrt{1 - \frac{2GM}{rc^2} - \frac{J^2}{r^3c^2}}
\label{eq:Kerr_time_dilation}
\end{equation}
waarbij:
\begin{itemize}
\item $M$: massa van het roterende lichaam,
\item $J$: impulsmoment,
\item $r$: radiale afstand tot de bron,
\item $G$: gravitatieconstante van Newton,
\item $c$: lichtsnelheid.
\end{itemize}

De eerste term correspondeert met gravitationele roodverschuiving ten opzichte van de massa, terwijl de tweede rekening houdt met rotatie-effecten (frame-dragging).

\subsection{Æther analoog via vorticiteit en circulatie}

In VAM drukken we gravitatieachtige invloeden uit via vorticiteitsintensiteit $\langle \omega^2 \rangle$ en totale circulatie $\kappa$. Deze worden geïnterpreteerd als:
\begin{itemize}
\item $\langle \omega^2 \rangle$: gemiddelde kwadratische vorticiteit over een gebied,
\item $\kappa$: behouden circulatie, coderend voor impulsmoment.
\end{itemize}

We definiëren de æthergebaseerde analoog door de volgende vervangingen uit te voeren:
\begin{equation}
\begin{aligned}
\frac{2GM}{rc^2} &\rightarrow \frac{\gamma \langle \omega^2 \rangle}{rc^2}, \\
\frac{J^2}{r^3c^2} &\rightarrow \frac{\kappa^2}{r^3c^2}
\end{aligned}
\label{eq:Kerr_replacements}
\end{equation}

Hier is $\gamma$ een koppelingsconstante die de vorticiteit relateert aan de effectieve zwaartekracht (analoog aan $G$). De op æther gebaseerde eigentijd wordt dan:

\begin{equation}
\boxed{t_{\text{adjusted}} = \Delta t \cdot \sqrt{1 - \frac{\gamma \langle \omega^2 \rangle}{rc^2} - \frac{\kappa^2}{r^3c^2}}}
\label{eq:Kerr_time_dilation_ae}
\end{equation}

Dit weerspiegelt de Kerr-roodverschuiving en frame-dragging-structuur met behulp van vloeistofdynamische variabelen. In deze afbeelding:
\begin{itemize}
\item $\langle \omega^2 \rangle$ speelt de rol van energiedichtheid die gravitationele roodverschuiving produceert,
\item $\kappa$ vertegenwoordigt impulsmoment dat tijdelijke frame-dragging genereert,
\item De vergelijking reduceert tot een vlakke æthertijd ($t_{\text{aangepast}} \tot \Delta t$) wanneer beide termen verdwijnen.
\end{itemize}


\subsection*{Hybride VAM Frame-Dragging Hoeksnelheid}


In het Vortex Æther Model (VAM) wordt de frame-dragging hoeksnelheid, geïnduceerd door een roterend wervelgebonden object, analoog gedefinieerd aan het Lense-Thirring effect in de algemene relativiteitstheorie, maar met een schaalafhankelijke Koppeling:

\begin{equation}
    \omega_{\text{drag}}^{\text{VAM}}(r) =
    \frac{4 G m}{5 c^2 r} \cdot \mu(r) \cdot \Omega(r)
\end{equation}

Hierbij is \( G \) de gravitatieconstante, \( c \) de lichtsnelheid, \( m \) de massa van het object, \( r \) de karakteristieke straal en \( \Omega(r) \) de hoeksnelheid.

De hybride koppelingsfactor \( \mu(r) \) interpoleert tussen wervelgedrag op kwantumschaal en klassieke macroscopische rotatie:

\begin{equation}
    \mu(r) =
    \begin{cases}
        \displaystyle \frac{r_c C_e}{r^2}, & \text{if } r < r_\ast \quad \text{(kwantum- of vortexkernregime)} \\
        1, & \text{if } r \geq r_\ast \quad \text{(macroscopisch regime)}
    \end{cases}
\end{equation}

waarbij:
\begin{itemize}
    \item \( r_c \) de straal van de vortexkern is,
    \item \( C_e \) de tangentiële snelheid van de vortexkern is,
    \item \( r_\ast \sim 10^{-3} \, \text{m} \) de overgangsradius tussen microscopische en macroscopische regimes is.
\end{itemize}

Deze formulering zorgt voor continuïteit met GR-voorspellingen voor hemellichamen, terwijl VAM-specifieke voorspellingen voor elementaire deeltjes en subatomaire vortexstructuren mogelijk worden.


\subsection*{VAM Gravitationele Roodverschuiving vanuit Kernrotatie}

In het Vortex Æther Model (VAM) ontstaat gravitationele roodverschuiving door de lokale rotatiesnelheid \( v_\phi \) aan de buitengrens van een vortexknoop. Uitgaande van geen ruimtetijdkromming en absolute tijd, wordt de effectieve gravitationele roodverschuiving gegeven door:

\begin{equation}
    z_{\text{VAM}} =
    \left( 1 - \frac{v_\phi^2}{c^2} \right)^{-\frac{1}{2}} - 1
\end{equation}

waarbij:
\begin{itemize}
    \item \( v_\phi = \Omega(r) \cdot r \) de tangentiële snelheid is ten gevolge van lokale rotatie,
    \item \( \Omega(r) \) de hoeksnelheid is bij de meetstraal \( r \),
    \item \( c \) de lichtsnelheid in vacuüm is.
\end{itemize}

Deze uitdrukking weerspiegelt de verandering van de tijdsperceptie veroorzaakt door lokale rotatie-energie, waarbij de op kromming gebaseerde gravitatiepotentiaal \( \Phi \) van de algemene relativiteitstheorie wordt vervangen door een snelheidsveldterm. Deze wordt equivalent aan de GR Schwarzschild-roodverschuiving voor lage \( v_\phi \) en divergeert als \( v_\phi \rightarrow c \), wat een natuurlijke grens vormt voor de evolutie van het lokale frame:

\begin{equation}
    \lim_{v_\phi \to c} z_{\text{VAM}} \to \infty
\end{equation}



\subsection*{VAM Lokale Tijdsdilatatie Modellen}

In het Vortex Æther Model (VAM) wordt lokale tijdsdilatatie geïnterpreteerd als de modulatie van absolute tijd door interne vortexdynamica, niet door ruimtetijdkromming. Afhankelijk van de systeemschaal worden twee fysisch gefundeerde formuleringen gebruikt:

\paragraph{1. Tijddilatatie op basis van snelheidsvelden}

Dit model relateert de lokale tijdstroom aan de tangentiële snelheid van de roterende ætherische structuur (vortexknoop, planeet of ster):

\begin{equation}
    \frac{d\tau}{dt} =
    \sqrt{1 - \frac{v_\phi^2}{c^2}} =
    \sqrt{1 - \frac{\Omega^2 r^2}{c^2}}
\end{equation}

waarbij:
\begin{itemize}
    \item \( v_\phi = \Omega \cdot r \) de tangentiële snelheid is,
    \item \( \Omega \) de hoeksnelheid bij straal \( r \) is,
    \item \( c \) de lichtsnelheid is.
\end{itemize}

\paragraph{2. Tijddilatatie op basis van rotatie-energie}

Op grote schaal of met hoge rotatietraagheid ontstaat tijddilatatie door opgeslagen rotatie-energie, wat leidt tot:

\begin{equation}
    \frac{d\tau}{dt} =
    \left(1 + \frac{1}{2} \cdot \beta \cdot I \cdot \Omega^2 \right)^{-1}
\end{equation}

met:
\begin{itemize}
    \item \( I = \frac{2}{5} m r^2 \): traagheidsmoment voor een uniforme bol,
    \item \( \beta = \frac{r_c^2}{C_e^2} \): koppelingsconstante van vortex-kerndynamica,
    \item \( m \) is de massa van het object. \end{itemize}

\paragraph{Interpretatie}

Deze modellen impliceren dat de tijd vertraagt in gebieden met een hoge lokale rotatie-energie of vorticiteit, in overeenstemming met gravitationele tijddilatatie-effecten in GR. In VAM ontstaan deze effecten echter uitsluitend door de interne dynamiek van de ætherstroming, onder vlakke 3D Euclidische meetkunde en absolute tijd.


\subsection{Modelaannames en reikwijdte}

Dit resultaat is afhankelijk van verschillende aannames:
\begin{itemize}
\item De stroming is rotatievrij buiten de vortexkernen,
\item Viscositeit en turbulentie worden verwaarloosd,
\item Samendrukbaarheid wordt genegeerd (ideale onsamendrukbare superfluïde),
\item Vorticiteitsvelden zijn voldoende glad om $\langle \omega^2 \rangle$ te definiëren.
\end{itemize}

Deze omstandigheden weerspiegelen de aannames van analoge modellen van ideale vloeistof-GR. De formulering overbrugt de macroscopische stromingsdynamica van de æther met effectieve geometrische voorspellingen, wat de mogelijkheid versterkt om gekromde ruimtetijd te vervangen door gestructureerde vorticiteitsvelden.

Zie Appendix~\ref{appendix:1} voor gedetailleerde afleidingen van kruisenergie- en vortexinteractie-energetica.

In toekomstig werk kunnen correcties voor randvoorwaarden, gekwantiseerde vorticiteitsspectra en compressibele effecten worden toegevoegd om de analogie te verfijnen. Vervolgens vatten we samen hoe deze vloeistofgebaseerde tijddilatatiemechanismen zich verenigen binnen het VAM-kader en identificeren we hun experimentele implicaties.
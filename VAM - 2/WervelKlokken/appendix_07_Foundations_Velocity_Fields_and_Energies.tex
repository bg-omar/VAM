\section{Grondslagen van snelheidsvelden en energieën in een wervelsysteem.}

\subsection{Inleiding}
Werveldynamica is een kerncomponent van veel vloeistof- en plasmasystemen, waaronder
tornado-achtige stromingen, geknoopte wervels in klassieke of superfluïde turbulentie en diverse
complexe topologische vloeistofsystemen. Een beter begrip van de energiebalansen
die met deze stromingen gepaard gaan, kan licht werpen op processen zoals wervelstabiliteit, herverbinding
en globale stromingsorganisatie. We beginnen met een motivatie voor hoe snelheidsvelden kunnen worden
ontbonden om de totale energie (d.w.z. zelf- plus kruisenergie) vast te leggen, en hoe
deze aanpak helpt bij het volgen van stromingen in zowel 2D als 3D.

\subsection{Fundamenten: Snelheidsvelden en totale (zelf- + dwars-)energie}
\label{sec:foundations}
In een onsamendrukbare vloeistof wordt het snelheidsveld $\mathbf{u}(\mathbf{x}, t)$ doorgaans
bepaald door de Navier-Stokes- of Euler-vergelijkingen. Voor niet-viskeuze analyses luiden de Euler-vergelijkingen voor onsamendrukbare stroming als volgt:
\begin{equation}
\frac{\partial \mathbf{u}}{\partial t} + (\mathbf{u} \cdot \nabla)\mathbf{u} = -\frac{1}{\rho}\nabla p,
\quad \nabla \cdot \mathbf{u} = 0.\label{eq:appendix:Euler}
\end{equation}
We beschouwen ook de vorticiteit $\boldsymbol{\omega} = \nabla \times \mathbf{u}$,
die kan worden gebruikt om wervelstructuren te karakteriseren.

Om de totale kinetische energie te begrijpen, kunnen we deze als volgt opsplitsen:
\begin{equation}
E_{\text{totaal}} \;=\; E_{\text{zelf}} \;+\; E_{\text{kruis}}.\label{eq:appendix:totale-energie}
\end{equation}
Hier is $E_{\text{zelf}}$ het deel van de energie dat elk wervel- of deelstroomelement onafhankelijk bijdraagt (bijvoorbeeld door lokale wervelbewegingen), terwijl
$E_{\text{kruis}}$ de bijdragen codeert die voortkomen uit de interactie van verschillende
wervelelementen. In een multi-wervelscenario helpt een dergelijke decompositie om de
directe interactie tussen twee (of meer) wervelfilamenten of -lagen te isoleren.

\subsection{Overwegingen met betrekking tot impuls en eigen energie}
\label{sec:impuls}
Een startpunt is om te onthouden dat voor een enkele circulatiewervel $\Gamma$, met een
azimutaal symmetrische kern, de geïnduceerde snelheid soms wordt benaderd door
klassieke resultaten zoals
\begin{equation}
V \;=\; \frac{\Gamma}{4 \pi R}
\bigl(\ln \tfrac{8 R}{a} - \beta \bigr),\label{eq:appendix:velocity}
\end{equation}
waarbij $R$ de straal van de hoofdwervellus is, $a \ll R$ een maat is voor de kerndikte,
en $\beta$ afhangt van de details van het kernmodel \cite{Saffman1992}. De
\emph{zelfenergie} die aan die wervel is gekoppeld, $E_{\text{self}}$, kan in een
vergelijkbare vorm worden gegoten die afhankelijk is van $\ln(R/a)$, wat illustreert hoe de energieën van dunnekernwervelen
schalen met de geometrie.

In meer algemene vloeistof- of wervelroostermodellen kunnen we $E_{\text{self}}$ volgen als de
som van de individuele kernenergieën. Bovendien wijzigt de aanwezigheid van meerdere filamenten
de totale energie door de kruistermen van de snelheidsvelden (de kruisenergie). Deze
kruisenergie is vaak de drijvende kracht achter belangrijke fenomenen zoals het samensmelten van wervels of de `terugslag'-effecten
in golf-vortexinteracties.

\subsection{Definiëren en volgen van kruisenergie}
\label{sec:cross}
Wanneer meerdere wervelingen (of gedeeltelijke snelheidsverdelingen) naast elkaar bestaan, kan het totale snelheidsveld $\mathbf{u}$ worden gesuperponeerd:
\begin{equation}
\mathbf{u} \;=\; \mathbf{u}_1 \;+\;\mathbf{u}_2,\label{eq:appendix:superpose}
\end{equation}
waarbij $\mathbf{u}_1$ en $\mathbf{u}_2$ afkomstig zijn van verschillende subsystemen. In dat
scenario is de kinetische energie voor een vloeistofvolume $V$
\begin{align}
E_{\text{total}} &= \frac{\rho}{2} \int_V \mathbf{u}^2 \,dV
= \frac{\rho}{2} \int_V \bigl(\mathbf{u}_1 + \mathbf{u}_2 \bigr)^2\, dV \\
&= \frac{\rho}{2} \int_V \mathbf{u}_1^2 \,dV \;+\;\frac{\rho}{2} \int_V \mathbf{u}_2^2 \,dV
\;+\;\rho \int_V \mathbf{u}_1 \cdot \mathbf{u}_2 \, dV,
\end{align}
onthulling van een interactie of \emph{kruisenergie} term
\begin{equation}
E_{\text{cross}} \;=\; \rho \int_V \mathbf{u}_1 \cdot \mathbf{u}_2 \, dV.
\label{eq:cross-term}
\end{equation}
Veel van de interessante natuurkunde komt voort uit \eqref{eq:cross-term}, omdat deze
groeit of krimpt afhankelijk van de geometrie van de wervels en de afstand ertussen.
De dynamische evolutie ervan kan bijvoorbeeld leiden tot samensmelting of terugvering. Een belangrijk punt is dat
de eigensnelheid van elke wervel de onderlinge snelheden aanzienlijk kan beïnvloeden en zo
nettokrachten of koppel kan creëren.
\subsection{Toepassingen op heliciteit en topologische stromingen}
\label{sec:helicity}
Een verwant concept is heliciteit, waarmee de topologische complexiteit (knopen of
verbindingen) van wervelbuizen wordt gemeten. Klassiek wordt heliciteit $H$ gegeven door
\begin{equation}
H \;=\; \int_V \mathbf{u} \cdot \boldsymbol{\omega}\, dV,\label{eq:appendix:helicity}
\end{equation}
die constant kan blijven of gedeeltelijk verloren kan gaan tijdens herverbindingsgebeurtenissen. In bepaalde
dissipatieve stromingen kunnen de kruisenergietermen in \eqref{eq:cross-term} de effectieve snelheid van heliciteitsverandering beïnvloeden. Het begrijpen van $E_{\text{cross}}$ is belangrijk
voor het analyseren van herverbindingspaden in klassieke of superfluïde turbulentie.

\subsection{Afleidingsschema voor kruisenergie}
\label{sec:derivation}
Ten slotte geven we een beknopt schema voor het afleiden van de uitdrukking voor kruisenergie. Beginnend met het totale snelheidsveld $\mathbf{u} = \sum_{n=1}^N \mathbf{u}_n$
voor $N$ wervel- of partiële snelheidsvelden is de totale kinetische energie:
\begin{equation}
E_{\text{totaal}}
= \frac{\rho}{2} \int_V \left(\sum_{n=1}^N \mathbf{u}_n \right)^2 dV
= \frac{\rho}{2} \sum_{n=1}^N \int_V \mathbf{u}_n^2 \, dV
\;+\;\rho \sum_{n<m} \int_V \mathbf{u}_n \cdot \mathbf{u}_m \, dV.\label{eq:appendix:total-energy-derivation}
\end{equation}
Men verkrijgt $N$ zelfenergietermen plus paarsgewijze kruisenergie-integralen.
De kruisenergie voor een paar $(i,j)$ is:
\begin{equation}
E_{\text{cross}}^{(ij)} \;=\; \rho \int_V \mathbf{u}_i \cdot \mathbf{u}_j \, dV.\label{eq:appendix:cross-energy-derivation}
\end{equation}
In de praktijk kan elke $\mathbf{u}_n$ worden weergegeven door bekende oplossingen van de
Stokes- of potentiaalstroomvergelijkingen, of door benaderde oplossingen voor wervellussen. Vervolgens verkrijgt men, analytisch of numeriek, benaderde kruisenergieën
die gebruikt kunnen worden in gereduceerde modellen die de evolutie van multi-vortexsystemen beschrijven.

\subsection*{Conclusion}
We hebben onderzocht hoe de totale kinetische energie van vloeistoffen in de aanwezigheid van meerdere
vortices kan worden opgesplitst in termen van zelf- en kruisenergie. Deze bijdragen van kruisenergieën
zijn cruciaal voor het begrijpen van het samensmelten van wervels, het ontwarren van geknoopte wervels, of wervel-golfinteracties in klassieke, superfluïde en plasmastromen. Daarnaast hebben we een systematische afleiding van kruisenergie geschetst en
belangrijke aspecten benadrukt bij de bespreking van impuls en heliciteit. Toekomstige richtingen
zijn onder meer het verfijnen van deze uitdrukkingen voor axiaal symmetrische of geknoopte wervels en
het integreren ervan in grootschalige modellen of computationele kaders.
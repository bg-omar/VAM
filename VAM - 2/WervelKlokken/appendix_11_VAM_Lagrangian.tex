%! Author = Omar Iskandarani
%! Date = 5/23/2025

\section{VAM-Lagrangiaan Gebaseerd op Incompressibele Schrödinger Flow}

\subsection{Complexe Vortexgolven in Æther}

We modelleren een vortexdeeltje als een genormaliseerde tweevoudige complexe golffunctie:
\[
    \psi(\vec{r}, t) = \begin{pmatrix} a + ib \\ c + id \end{pmatrix}, \quad |\psi|^2 = 1,
\]
waaruit de spinvector $\vec{s} = (s_1, s_2, s_3)$ en wervelveld $\vec{\omega}$ worden gedefinieerd via een Hopf-mapping.

\subsection{Lagrangiaan met Landau–Lifshitz-achtige term}

We definiëren de VAM-golffunctie-Lagrangiaan als:
\begin{equation}
    \mathcal{L}_\text{VAM}[\psi] =
    \frac{i\hbar}{2} \left( \psi^\dagger \partial_t \psi - \psi \partial_t \psi^\dagger \right)
    - \frac{\hbar^2}{2m} |\nabla \psi|^2
    - \frac{\alpha}{8} |\nabla \vec{s}|^2,
\end{equation}
waar:
\begin{itemize}
    \item $\hbar$ wordt vervangen door een VAM-conforme kwantisatieconstante,
    \item $\alpha$ is een dimensieloze vortexkoppelingsconstante,
    \item $\vec{s}$ de Hopf-spinvector is, berekend uit $\psi$ via:
    \[
        s_1 = a^2 + b^2 - c^2 - d^2, \quad
        s_2 = 2(bc - ad), \quad
        s_3 = 2(ac + bd).
    \]
\end{itemize}

\subsection{Afleiding van de VAM-veldvergelijking}

Via variatie ten opzichte van $\psi^*$ verkrijgen we de gemodificeerde ISF-vergelijking:
\[
    i\hbar \frac{\partial \psi}{\partial t} =
    - \frac{\hbar^2}{2m} \nabla^2 \psi
    + \frac{\alpha}{4} \frac{\delta}{\delta \psi^*} |\nabla \vec{s}|^2.
\]

De afgeleide Euler-Lagrange-vergelijking bevat topologische terugkoppeling van de knoopstructuur op de tijdsontwikkeling van de golf.

\subsection{Fysische Interpretatie}

Deze formulering maakt het mogelijk om:
\begin{enumerate}
    \item Quantumsuperpositie van vortexdeeltjes te beschrijven;
    \item VAM-tijdsvertraging af te leiden uit de heliciteit van $\vec{s}$;
    \item Stabiliteit van vortexknopen te koppelen aan een effectieve potentiaal \( V(\vec{s}) \sim |\nabla \vec{s}|^2 \);
    \item Evolutie te simuleren zonder gebruik van klassieke Navier–Stokes-dissipatie.
\end{enumerate}
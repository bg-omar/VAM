
\section{Vorticiteit en zwaartekrachtsveld}
In VAM wordt de zwaartekrachtsversnelling \( \vec{g}_v \) afgeleid uit de drukgradiënt in een roterend æther:
\begin{equation}
    \nabla^2 \Phi_v = -\rho \norm{\vec{\omega}}^2
\end{equation}
met \( \vec{\omega} = \nabla \times \vec{v} \) de lokale wervelvector en \( \Phi_v \) het potentiaalveld. Deze vergelijking is analoog aan Poisson's vergelijking in Newtonse zwaartekracht.





\section{Tijdsdilatatie uit kernrotatie}
Voor een lokaal draaiend object geldt:
\begin{equation}
    \frac{d\tau}{dt} = \left(1 + \frac{1}{2} \beta I \Omega^2 \right)^{-1},
\end{equation}
waar \( I \) het traagheidsmoment is, \( \Omega \) de hoeksnelheid en \( \beta = \frac{r_c^2}{C_e^2} \) een koppelingsconstante.








\section{Roodverschuiving uit kernrotatie}
\begin{equation}
    z = \left(1 - \frac{v_\phi^2}{c^2} \right)^{-1/2} - 1, \quad v_\phi = \Omega r
\end{equation}
Deze benadering reproduceert de GR-roodverschuiving in de limiet \( v_\phi \ll c \).





\section{Frame-dragging en precessie}
VAM-model voor frame-dragging:
\begin{equation}
    \omega_{\text{drag}}^{\text{VAM}} = \frac{4 G m}{5 c^2 r} \cdot \mu(r) \cdot \Omega(r)
\end{equation}
met schaalafhankelijke interpolatiefactor:
\begin{equation}
    \mu(r) =
    \begin{cases}
        \displaystyle \frac{r_c C_e}{r^2}, & r < r_* \\
        1, & r \geq r_*
    \end{cases}
\end{equation}





\section{Vergelijking met GR-observabelen}
\begin{center}
    \begin{tabular}{|c|c|c|}
        \hline
        \textbf{Waarneming} & GR & VAM \\
        \hline
        Tijdsdilatatie & $\sqrt{1 - \frac{2GM}{rc^2}}$ & $\sqrt{1 - \frac{\Omega^2 r^2}{c^2}}$ \\
        Roodverschuiving & idem & $\left(1 - \frac{v_\phi^2}{c^2} \right)^{-1/2} - 1$ \\
        Frame-dragging & $\omega_{LT} = \frac{2GJ}{c^2 r^3}$ & $\frac{2G \mu I \Omega}{c^2 r^3}$ \\
        Precessie & $\Delta\phi = \frac{6\pi GM}{a(1 - e^2)c^2}$ & idem \\
        Lichtafbuiging & $\delta = \frac{4GM}{Rc^2}$ & idem \\
        Gravitatiepotentiaal & $\Phi = -\frac{GM}{r}$ & $\Phi = -\frac{1}{2} \vec{\omega} \cdot \vec{v}$ \\
        \hline
    \end{tabular}
\end{center}



\subsection*{VAM-orbitaalprecessie (GR-equivalent)}

    In de algemene relativiteitstheorie wordt de periheliumprecessie van een draaiend lichaam toegeschreven aan ruimtetijdkromming. In het Vortex-Æthermodel (VAM) wordt dit effect vervangen door de cumulatieve invloed van een werveling-geïnduceerd vorticiteitsveld binnen een roterend Æthermedium.

    De equivalente VAM-formulering weerspiegelt de GR-voorspelling, maar is gebaseerd op door vorticiteit geïnduceerde drukgradiënten en circulatie:

    \begin{equation}
        \Delta\phi_{\text{VAM}} =
        \frac{6\pi G M}{a(1 - e^2) c^2}
    \end{equation}

    waarbij:
    \begin{itemize}
        \item \( M \): massa van de centrale vortex-attractor,
        \item \( a \): halve lange as van de baan,
        \item \( e \): excentriciteit van de baan,
        \item \( G \): gravitatieconstante (herleid uit VAM-koppeling),
        \item \( c \): lichtsnelheid.
    \end{itemize}
    Hoewel formeel identiek aan de GR-uitdrukking, ontstaat dit in VAM door de variatie in lokale circulatie en impulsmomentflux binnen de omringende Æther, waardoor het effectieve potentiaal wordt gemoduleerd en precessiebeweging ontstaat.

\subsection*{VAM-lichtafbuiging door Ætherische circulatie}

    In de algemene relativiteitstheorie wordt lichtafbuiging door massieve lichamen veroorzaakt door ruimtetijdkromming. In het Vortex-Æthermodel buigt licht (beschouwd als een verstoring of modus in de Æther) af als gevolg van door circulatie geïnduceerde drukgradiënten en anisotrope brekingsindexvelden in de buurt van roterende vortex-aantrekkers.

    De equivalente VAM-afbuigingshoek voor een lichtstraal die langs een sferische wervelmassa strijkt, wordt gegeven door:

    \begin{equation}
        \delta_{\text{VAM}} =
        \frac{4 G M}{R c^2}
    \end{equation}

    waarbij:
    \begin{itemize}
        \item \( M \): effectieve massa van de roterende wervelknoop,
        \item \( R \): dichtstbijzijnde nadering (impactparameter),
        \item \( G \): wervelkoppelingsconstante (herstel van Newtoniaanse \( G \) onder macroscopische grenzen),
        \item \( c \): lichtsnelheid.
    \end{itemize}

    In VAM is dit het gevolg van de interactie tussen de voortplantingssnelheid van het licht en het omringende rotatieveld. Het lichtgolffront wordt lokaal samengedrukt of gebroken door tangentiële etherstroomgradiënten, wat leidt tot een waarneembare hoekafbuiging.
\subsection*{Overzicht van de waarneembare correspondentie tussen VAM en GR}

    \begin{tabular}{|c|c|l|}
        \hline
        \textbf{Waarneembaar} & \textbf{Theorie} & \textbf{Uitdrukking} \\
        \hline

        \multirow{2}{*}{Tijdsdilatatie}
        & GR & \( \displaystyle \frac{d\tau}{dt} = \sqrt{1 - \frac{2GM}{rc^2}} \) \\
        & VAM & \( \displaystyle \sqrt{1 - \frac{\Omega^2 r^2}{c^2}} \) \\

        \hline
        \multirow{2}{*}{Roodverschuiving}
        & GR & \( \displaystyle z = \left(1 - \frac{2GM}{rc^2} \right)^{-1/2} - 1 \) \\
        & VAM & \( \displaystyle z = \left(1 - \frac{v_\phi^2}{c^2} \right)^{-1/2} - 1 \) \\

        \hline
        \multirow{2}{*}{Frame slepen}
        & GR & \( \displaystyle \omega_{\text{LT}} = \frac{2GJ}{c^2 r^3} \) \\
        & VAM & \( \displaystyle \frac{2G \mu I \Omega}{c^2 r^3} \) \\

        \hline
        \multirow{2}{*}{Precessie}
        & GR/VAM & \( \displaystyle \Delta\phi = \frac{6\pi GM}{a(1 - e^2)c^2} \) \\

        \hline
        \multirow{2}{*}{Lichtafbuiging}
        & GR/VAM & \( \displaystyle \delta = \frac{4GM}{Rc^2} \) \\

        \hline
        \multirow{2}{*}{Zwaartekrachtpotentiaal}
        & GR & \( \Phi = -\frac{GM}{r} \) \\
        & VAM & \( \Phi = -\frac{1}{2} \vec{\omega} \cdot \vec{v} \) \\

        \hline
        Zwaartekrachtconstante
        & VAM & \( \displaystyle G = \frac{C_e c^5 t_p^2}{2 F_{\max} r_c^2} \) \\
        \hline
    \end{tabular}

\documentclass{article}
\usepackage{amsmath}
\usepackage{graphicx}
\usepackage{geometry}
\geometry{margin=1in}
\title{Appendix: Nuclear Activation via Swirl Resonance in the Vortex Æther Model}
\author{Vortex Æther Dynamics}
\date{}
\begin{document}
\maketitle
\section*{1. Overview}
Recent experimental work on low-energy nuclear reactions (LENR), especially using reactions like $^{11}\mathrm{B}(d,n\gamma)^{12}\mathrm{C}$, reveals nuclear states that can be selectively activated using monoenergetic gamma beams. Within the Vortex Æther Model (VAM), these results are interpreted as \textit{swirl resonance phenomena}, wherein specific angular frequency components of the injected field couple with vortex knot eigenmodes.
\section*{2. Swirl Resonance Yield}
We model the activation yield $Y_{\mathrm{VAM}}$ as a spectral overlap:
\[
Y_{\mathrm{VAM}} = \int_0^\infty \rho_{\mathrm{beam}}(\omega) \cdot \sigma_{\mathrm{knot}}(\omega) \, d\omega
\]
Here:
\begin{itemize}
  \item $\rho_{\mathrm{beam}}(\omega)$ is the angular frequency spectrum of the injected beam (gamma or ion-induced swirl),
  \item $\sigma_{\mathrm{knot}}(\omega)$ is the knot's absorption cross-section, modeled by:
  \[
  \sigma_{\mathrm{knot}}(\omega) = \sum_n \frac{B_n \Gamma_n^2}{(\omega - \omega_n)^2 + \Gamma_n^2}
  \]
  where $\omega_n$ is the $n^\text{th}$ knot mode, $\Gamma_n$ is its linewidth, and $B_n$ is the coupling strength.
\end{itemize}
\section*{3. Core-Shell Vortex Structure}
Different gamma energies interact with different radial layers of the knot:
\begin{itemize}
  \item 4.438 MeV photons $\rightarrow$ outer sheath Compton-like swirl scattering,
  \item 15.1 MeV photons $\rightarrow$ core-pair production and knot annihilation.
\end{itemize}
The knot cross-section becomes:
\[
\sigma(\omega) = \sum_i \sigma_i(\omega) \cdot \Theta(r_i - r)
\]
where each shell $r_i$ absorbs distinct $\omega$ bands.
\section*{4. Swirl Rigidity and $Z_{\mathrm{eff}}^{(\mathrm{VAM})}$}
The effective nuclear impedance in VAM becomes:
\[
Z_{\mathrm{eff}}^{(\mathrm{VAM})} = \frac{P_{\mathrm{core}} \cdot r_c}{C_e \hbar}
\]
mapping absorption behavior to Æther pressure properties.
\section*{5. Delayed Neutron Decay as Topological Swirl Collapse}
The classic 6-group delayed neutron model maps to sequential vorticity leakage from nested shells:
\[
\omega(t) = \sum_{i=1}^6 \omega_{0i} e^{-t/\tau_i}, \quad \tau_i = \frac{r_i}{C_e}
\]
Each decay constant corresponds to a specific radius $r_i$ and swirl lifetime.
\section*{6. Experimental Confirmation}
The presence of discrete gamma thresholds, delayed neutron curves, and resonance-specific yields all confirm the VAM prediction that:
\begin{itemize}
  \item Knot excitation is frequency-selective.
  \item Fusion activation is not thermal but \textbf{topological and swirl-driven}.
  \item External fields must match the vortex eigenfrequency to unlock nuclear reactions.
\end{itemize}
\end{document}
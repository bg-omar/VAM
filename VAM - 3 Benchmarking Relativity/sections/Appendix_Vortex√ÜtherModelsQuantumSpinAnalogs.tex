\chapter*{Vortex Æther Models and Quantum Spin Analogs}


\section*{Vortices as Atomic Models and Quantization of Angular Momentum}

The idea of modeling particles as \textit{vortices} in a fluidic "\ae{}ther" dates back to Lord Kelvin's 19th-century vortex atom hypothesis \cite{Dennis2020}. In the mid-20th century, researchers like O.C. Hilgenberg and Carl F. Krafft revived this concept by developing detailed vortex models of atomic structure \cite{Hilgenberg1938, KrafftEtherMatter}. Hilgenberg's 1938 and 1959 works formulated a \textit{vortex atom model} with a full quantum numbering system for the elements \cite{ReichBlackSun}. In these models, \textit{quantized} atomic properties emerge naturally from the dynamics of rotating vortex rings. Krafft argued that the \textit{quantization of energy} follows logically from a system of vortices that can only exchange energy or \ae{}ther in discrete rotational modes, determined by ring rotation \cite{PadrakINE9}. These vortex rings can only spin in certain stable modes, much like the discrete orbital and spin states in quantum mechanics.


\section*{Static Vorticity Fields and Spin-\textonehalf{} Behavior}

A striking success of vortex models is their ability to reproduce the properties of electron spin and its associated angular momentum. In the standard quantum view, an electron's spin $S$ is an \textit{intrinsic} angular momentum with magnitude $\sqrt{s(s+1)}\hbar$ (with $s=\tfrac{1}{2}$) and two $S_z$ projections ($\pm \hbar/2$). Classical models struggle to explain this discreteness, but vortex \ae{}ther models offer a clear analogy: the electron is modeled as a \textit{circulating vortex loop} with two possible stable orientations—clockwise or counterclockwise circulation. This directly mirrors spin-$\uparrow$ vs. spin-$\downarrow$ states \cite{KrafftEtherMatter}.


Modern vortex models, such as the toroidal photon model of Williamson and van der Mark or Hestenes' Zitterbewegung interpretation, equate intrinsic spin with internal circulatory motion of the vortex structure \cite{HestenesZitterbewegung}. In this view, the electron's spin-$\tfrac{1}{2}$ arises from a circulating current of radius on the order of the Compton wavelength, looping at or near light speed. This internal motion carries angular momentum $L=\tfrac{\hbar}{2}$ and reproduces the correct magnetic dipole moment.


\section*{Trefoil Knots, Helicity, and Quantized Invariants}

Vortex \ae{}ther models often map different elementary particles to different topological knot configurations in a conserved vorticity field. For example, the electron might be modeled as a closed vortex ring, while the proton is modeled as a trefoil knot \cite{PhysicsDetectivePair}. These configurations carry topological invariants like \textit{circulation} and \textit{helicity}, which are conserved quantities in ideal fluid flows.


Circulation, the line integral of velocity around a vortex loop, is quantized in superfluids in units of $h/m$ \cite{PRR2021}. A vortex loop corresponding to a particle like an electron may thus have a fixed circulation corresponding to its intrinsic spin. Helicity, defined as $\int \vec{v} \cdot \vec{\omega}, dV$, measures the linking and twisting of vortex lines and is a conserved quantity in ideal flows \cite{PNAS2014}. The handedness of a knot (left or right trefoil) may directly map to spin-$\uparrow$ or spin-$\downarrow$.


\section*{Angular Momentum Conservation via Vorticity Conservation}

Quantum angular momentum conservation (including spin) is mirrored in vortex \ae{}ther models by conservation of circulation and helicity. In ideal incompressible fluids, Kelvin's circulation theorem ensures the persistence of circulation unless acted upon externally. In the quantum-fluid limit, generalized vorticity (including spin density) obeys a conservation law \cite{PRL2011}.


A static knotted vortex structure like a trefoil possesses a fixed linking number, serving as a topological anchor for its angular momentum. Measurement interactions (such as spin projection measurements) correspond to aligning external fields with the vortex's circulation axis. The vortex then settles into one of two possible alignments—reproducing the spin projection quantization of $\pm \hbar/2$.


\section*{Magnetic Moment and the Anomalous \texorpdfstring{$g$}{g}-Factor in VAM}

In the Vortex \AE{}ther Model, the magnetic moment $\mu$ of an electron arises from its internal vortex circulation:

\begin{equation}

\mu = \frac{1}{2} e C_e r_c

\end{equation}

This produces a gyromagnetic ratio of $g=1$ when compared with angular momentum $L = M_e r_c C_e$. However, including relativistic internal motion (Zitterbewegung) with radius $r_{\text{zbw}} = \hbar / (M_e c)$ leads to:

\begin{equation}

\mu = \frac{e \hbar}{2 M_e}, \quad L = \frac{\hbar}{2} \Rightarrow g = 2

\end{equation}

To account for the measured anomaly, VAM introduces a correction from swirl-field feedback:

\begin{equation}

\mu = \frac{e \hbar}{2 M_e} \left(1 + \frac{\alpha}{2\pi} \right) \Rightarrow g = 2 + \frac{\alpha}{\pi}

\end{equation}

which matches the leading-order QED prediction \cite{Schwinger1948}. Thus, the anomaly arises from self-interaction between the vortex and the structured \ae{}ther swirl.

\section*{Spin Precession and Vortex Alignment in External Fields}
In standard quantum mechanics, spin precession arises when a magnetic moment $\vec{\mu}$ interacts with an external magnetic field $\vec{B}$, yielding a torque $\vec{\tau} = \vec{\mu} \times \vec{B}$. The spin vector precesses at the Larmor frequency:
\begin{equation}
\omega_L = \frac{g e B}{2 M_e}
\end{equation}
In the Vortex \AE{}ther Model, this behavior emerges naturally as the torque on a rotating vortex ring due to an imposed \ae{}theric vorticity gradient $\nabla \times \vec{v}\text{ext}$. The circulation axis of the vortex aligns with $\vec{B}$ through an induced swirl coupling:
\begin{equation}
\frac{d \vec{L}}{dt} = \vec{r} c \times \vec{F}\textit{{\text{swirl}} = \vec{\mu}}{\text{vortex}} \times \vec{B}\text{eff}
\end{equation}
where $\vec{B}{\text{eff}}$ is the magnetic field analog induced by external vorticity in the \ae{}ther. The Larmor frequency thus corresponds to the rate of precession of the vortex axis in the local swirl potential. This explains gyromagnetic ratios and spin alignment phenomena using vortex-fluid coupling.

\section*{Electron Spin Coupling and Measurement in VAM}
When a knotted vortex (representing an electron) is subject to an external field gradient, such as in a Stern–Gerlach experiment, the two circulation states (clockwise and counter-clockwise) experience differential swirl coupling. The result is a bifurcation of trajectories:
\begin{equation}
\Delta E = - \vec{\mu} \cdot \vec{B} \Rightarrow \pm \mu B_z = \pm \frac{e \hbar B}{2 M_e}
\end{equation}
This energy splitting arises from the torque-induced reconfiguration of the vortex's orientation within the structured field. Measurement collapses the ensemble of vortex orientations into one of two possible swirl alignments, mimicking spin projection quantization:
\begin{equation}
\langle S_z \rangle = \frac{\hbar}{2} \left( P_{\uparrow} - P_{\downarrow} \right)
\end{equation}
Thus, in VAM, measurement outcomes correspond to stable equilibrium orientations of the circulation axis within an external swirl field, recovering the probabilistic structure of quantum spin statistics.

\subsection*{Numerical Estimate of the Larmor Frequency}
Using standard VAM constants, we numerically evaluate the Larmor precession of an electron in a 1 Tesla magnetic field:
\begin{align*}
    e &= 1.602 \times 10^{-19}~\text{C} \\
    M_e &= 9.109 \times 10^{-31}~\text{kg} \\
    g &= 2.002319 \quad \text{(electron $g$-factor)} \\
    B &= 1~\text{T}
\end{align*}
Substituting into the Larmor formula:
\begin{equation}
    \omega_L = \frac{g e B}{2 M_e} \approx 1.76 \times 10^{11}~\text{rad/s}, \quad
    f_L = \frac{\omega_L}{2\pi} \approx 28.0~\text{GHz}
\end{equation}
This matches experimental electron spin resonance (ESR) values and confirms that VAM predicts correct precession dynamics through ætheric swirl torque.

\section*{Conclusion}

Vortex \AE{}ther Models (VAM) offer a geometric and fluid-dynamical underpinning for quantum mechanical spin. Through vorticity conservation, circulation quantization, and topological stability, VAM recovers key quantum features:

\begin{itemize}

\item Discrete angular momentum levels

\item Spin-$\tfrac{1}{2}$ behavior from circulation states

\item Conservation laws mapped to vortex invariants

\item Magnetic moment and $g$-factor from internal motion and swirl feedback

\end{itemize}

Spin, in this model, is not an abstract quantum label but a tangible expression of knotted fluid motion.



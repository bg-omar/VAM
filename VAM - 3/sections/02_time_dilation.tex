\section{Gravitational Time Dilation (Static Field)}

Gravitational time dilation in General Relativity (GR), under the Schwarzschild solution for a static spherical mass, is given by:
\[
    \frac{d\tau}{dt}_\text{GR} = \sqrt{1 - \frac{2GM}{rc^2}},
\]
where $\tau$ is proper time and $t$ is coordinate time at radial distance $r$ from mass $M$. For weak fields, the fractional slowdown is approximately $\frac{GM}{rc^2}$~\cite{will2014confrontation}.

\subsection*{VAM Interpretation}
In the Vortex Æther Model (VAM), gravitational time dilation arises from the rotational kinetic energy of a vortex in the æther medium. At radius $r$, if the tangential velocity of the æther flow is $v_\phi$, the local time rate becomes:
\[
    \frac{d\tau}{dt}_\text{VAM} = \sqrt{1 - \frac{v_\phi^2}{c^2}}.
\]
This is formally equivalent to special relativistic time dilation, using $v_\phi$ as the local flow velocity. VAM posits that for massive objects, $v_\phi^2 \approx 2GM/r$ (approximately the escape velocity squared), thus reproducing the first-order GR result~\cite{iskandarani2025VAM2}.

\begin{table}[h]
    \centering
    \footnotesize
    \caption{Gravitational Time Dilation at the Surface: GR vs VAM vs Observation}
    \begin{tabular}{|l|c|c|c|c|}
        \hline
        \textbf{Object} & \textbf{GR: $\frac{d\tau}{dt}$} & \textbf{VAM: $\frac{d\tau}{dt}$} & \textbf{Observed Effect} & \textbf{Rel. Error (VAM)} \\
        \hline
        Earth & 0.9999999993 & 0.9999999993 ($v_\phi\approx 11.2$ km/s) & $+45~\mu\text{s}/\text{day}$ (GPS)~\cite{ashby2003relativity} & $\sim$0\% \\
        Sun & 0.9999979 & 0.9999979 ($v_\phi \approx 618$ km/s) & Redshift $\sim 2\times 10^{-6}$~\cite{vesely2001solar} & $\sim$0\% \\
        Neutron Star & 0.875 & 0.875 ($v_\phi \approx 0.65c$) & X-ray redshift $z\sim 0.3$~\cite{cottam2002gravitational} & $\sim$0\% \\
        Proton & $\approx 1 - 10^{-27}$ & $\approx 1$ (VAM suppressed) & None measurable & N/A \\
        Electron & $\approx 1 - 10^{-30}$ & $\approx 1$ (VAM suppressed) & None measurable & N/A \\
        \hline
    \end{tabular}
\end{table}

\subsection*{Observational Agreement}
Gravitational redshift was confirmed by the Pound–Rebka experiment, showing $\Delta\nu/\nu = 2.5\times 10^{-15}$ over a 22.5 m height~\cite{pound1960apparent}. Modern atomic clock experiments (e.g., GPS satellites and Hafele–Keating) verify GR and SR combined dilation to precision better than $10^{-14}$~\cite{ashby2003relativity}.

\subsection*{Rotational Energy Formulation in VAM}
VAM optionally describes time dilation via stored rotational energy:
\[
    \frac{d\tau}{dt} = \left(1 + \frac{1}{2}\beta I \Omega^2\right)^{-1},
\]
where $I$ is the moment of inertia, $\Omega$ is angular velocity, and $\beta$ is a coupling parameter. For macroscopic bodies, tuning $\beta$ such that:
\[
    \frac{1}{2} \beta I \Omega^2 \approx \frac{GM}{Rc^2}
\]
ensures agreement with GR~\cite{iskandarani2025VAM2}.

\subsection*{Suppression at Quantum Scales}
To explain negligible gravity for elementary particles, VAM introduces a scale-dependent suppression factor $\mu(r)$, effective below $r^* \sim 10^{-3}$ m. This prevents excessive gravity from quantum-scale vortices while preserving agreement with Newtonian/GR gravity down to millimeter tests~\cite{adelberger2003tests}.

\subsection*{Conclusion}
VAM matches GR's gravitational time dilation in weak and strong fields by assigning appropriate ætheric swirl velocities. Deviations are avoided by tuning $\beta$ and applying scale suppression $\mu(r)$, making VAM experimentally indistinguishable from GR for time dilation.
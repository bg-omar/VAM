\section{Gravitational Waves and Binary Inspiral Decay}

One of the most stringent tests of General Relativity (GR) is the observation of gravitational waves, particularly through the orbital decay of binary pulsars. The first such indirect detection came from the Hulse--Taylor binary pulsar (PSR B1913+16).

\subsection*{GR Prediction}

According to GR, two orbiting masses emit energy via gravitational radiation. For PSR B1913+16, with orbital period $P_b = 7.75$ hours and eccentricity $e = 0.617$, the predicted orbital period derivative due to gravitational wave emission is:
\begin{equation}
    \frac{dP_b}{dt}_{\text{GR}} = -2.4025\times10^{-12} \ \text{s/s}
\end{equation}

The observed decay, corrected for galactic acceleration, is:
\begin{equation}
    \frac{dP_b}{dt}_{\text{obs}} = -2.4056(\pm 0.0051)\times10^{-12} \ \text{s/s}
\end{equation}

This agreement within 0.13\% is a hallmark success of GR~\cite{weisberg2016}. Direct detections by LIGO/Virgo~\cite{abbott2016} have further confirmed gravitational wave theory.

\subsection*{VAM Outlook}

The Vortex \AE ther Model (VAM) in its current form describes gravity via stationary aether vortices in an incompressible, inviscid medium. In such a medium, there is no mechanism for radiation from orbiting bodies. VAM would thus predict:
\begin{equation}
    \frac{dP_b}{dt}_{\text{VAM}} \approx 0
\end{equation}

This is in stark contrast with observations. Table~\ref{tab:gw_comparison} summarizes the discrepancy.

\begin{table}[h!]
    \centering
    \caption{Binary Inspiral Decay Predictions and Observations}
    \label{tab:gw_comparison}
    \begin{tabular}{lccc}
        \toprule
        System & $\frac{dP}{dt}_{\text{GR}}$ (s/s) & $\frac{dP}{dt}_{\text{VAM}}$ & $\frac{dP}{dt}_{\text{Obs}}$ (s/s) \\
        \midrule
        PSR B1913+16 & $-2.4025\times10^{-12}$ & $\sim 0$ & $-2.4056(51)\times10^{-12}$ \\
        PSR J0737--3039A/B & $-1.252\times10^{-12}$ & $\sim 0$ & $-1.252(17)\times10^{-12}$ \\
        GW150914 (BH merger) & $\sim 3M_\odot c^2$ radiated & No GW & Direct detection (LIGO) \\
        \bottomrule
    \end{tabular}
\end{table}

\subsection*{Possible Extensions to VAM}

To address this shortcoming, VAM must introduce a radiation mechanism. Several possibilities are:

\paragraph{1. Compressible Aether} If the \ae ther is compressible, then orbiting masses could emit longitudinal or transverse waves. By tuning the \ae ther's compressibility such that wave speed is $c$, one could emulate GR's gravitational waves.

\paragraph{2. Vortex Shedding and Turbulence} Orbiting vortices might induce a cascade or shed smaller vortices, analogous to vortex street formation. If this couples to another field or if minimal viscosity exists, energy could be radiated.

\paragraph{3. Thermodynamic Coupling} The VAM formalism introduces entropy fields; these could support excitations. Merging vortex knots could radiate in this field, analogous to massless spin-2 graviton-like excitations.

\subsection*{Suggested Remedy}

A viable extension to VAM would introduce a dynamical perturbation field $\psi$ such that:
\begin{equation}
    \nabla^2 \psi - \frac{1}{c^2} \frac{\partial^2 \psi}{\partial t^2} = S(t)
\end{equation}
where $S(t)$ is sourced by the time-varying quadrupole moment of the mass-vortex system.

This would allow VAM to:
\begin{itemize}
    \item Match $\frac{dP_b}{dt}$ in pulsars.
    \item Emit waveform structures compatible with LIGO/Virgo detections.
    \item Preserve the Newtonian and GR limits in the far-field.
\end{itemize}

\subsection*{Conclusion}

As it stands, VAM fails to reproduce gravitational wave emission and orbital decay. Extending it to include compressibility or dynamic field equations is essential. Until then, GR remains the only model consistent with pulsar timing and direct gravitational wave detection.


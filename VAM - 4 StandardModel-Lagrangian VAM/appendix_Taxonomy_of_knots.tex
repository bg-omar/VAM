%! Author = Omar Iskandarani
%! Date = 2025-06-13

% === Metadata ===
\newcommand{\papertitle}{Appendix: VAM Knot Taxonomy: A Layered Topological Structure of Matter}
\newcommand{\paperauthor}{Omar Iskandarani}
\newcommand{\paperaffil}{Independent Researcher, Groningen, The Netherlands}
\newcommand{\paperdoi}{10.5281/zenodo.xxxxxxxx}
\newcommand{\paperorcid}{0009-0006-1686-3961}

\ifdefined\standalonechapter\else
  % Standalone mode
  \documentclass[12pt]{article}
  % vamstyle.sty
\NeedsTeXFormat{LaTeX2e}
\ProvidesPackage{vamstyle}[2025/07/01 VAM unified style]

% === Constants ===
\newcommand{\hbarVal}{\ensuremath{1.054571817 \times 10^{-34}}} % J\cdot s
\newcommand{\meVal}{\ensuremath{9.10938356 \times 10^{-31}}} % kg
\newcommand{\cVal}{\ensuremath{2.99792458 \times 10^{8}}} % m/s
\newcommand{\alphaVal}{\ensuremath{1 / 137.035999084}} % unitless
\newcommand{\alphaGVal}{\ensuremath{1.75180000 \times 10^{-45}}} % unitless
\newcommand{\reVal}{\ensuremath{2.8179403227 \times 10^{-15}}} % m
\newcommand{\rcVal}{\ensuremath{1.40897017 \times 10^{-15}}} % m
\newcommand{\vacrho}{\ensuremath{5 \times 10^{-9}}} % kg/m^3
\newcommand{\LpVal}{\ensuremath{1.61625500 \times 10^{-35}}} % m
\newcommand{\MpVal}{\ensuremath{2.17643400 \times 10^{-8}}} % kg
\newcommand{\tpVal}{\ensuremath{5.39124700 \times 10^{-44}}} % s
\newcommand{\TpVal}{\ensuremath{1.41678400 \times 10^{32}}} % K
\newcommand{\qpVal}{\ensuremath{1.87554596 \times 10^{-18}}} % C
\newcommand{\EpVal}{\ensuremath{1.95600000 \times 10^{9}}} % J
\newcommand{\eVal}{\ensuremath{1.60217663 \times 10^{-19}}} % C

% === VAM/\ae ther Specific ===
\newcommand{\CeVal}{\ensuremath{1.09384563 \times 10^{6}}} % m/s
\newcommand{\FmaxVal}{\ensuremath{29.0535070}} % N
\newcommand{\FmaxGRVal}{\ensuremath{3.02563891 \times 10^{43}}} % N
\newcommand{\gammaVal}{\ensuremath{0.005901}} % unitless
\newcommand{\GVal}{\ensuremath{6.67430000 \times 10^{-11}}} % m^3/kg/s^2
\newcommand{\hVal}{\ensuremath{6.62607015 \times 10^{-34}}} % J Hz^-1

% === Electromagnetic ===
\newcommand{\muZeroVal}{\ensuremath{1.25663706 \times 10^{-6}}}
\newcommand{\epsilonZeroVal}{\ensuremath{8.85418782 \times 10^{-12}}}
\newcommand{\ZzeroVal}{\ensuremath{3.76730313 \times 10^{2}}}

% === Atomic & Thermodynamic ===
\newcommand{\RinfVal}{\ensuremath{1.09737316 \times 10^{7}}}
\newcommand{\aZeroVal}{\ensuremath{5.29177211 \times 10^{-11}}}
\newcommand{\MeVal}{\ensuremath{9.10938370 \times 10^{-31}}}
\newcommand{\MprotonVal}{\ensuremath{1.67262192 \times 10^{-27}}}
\newcommand{\MneutronVal}{\ensuremath{1.67492750 \times 10^{-27}}}
\newcommand{\kBVal}{\ensuremath{1.38064900 \times 10^{-23}}}
\newcommand{\RVal}{\ensuremath{8.31446262}}

% === Compton, Quantum, Radiation ===
\newcommand{\fCVal}{\ensuremath{1.23558996 \times 10^{20}}}
\newcommand{\OmegaCVal}{\ensuremath{7.76344071 \times 10^{20}}}
\newcommand{\lambdaCVal}{\ensuremath{2.42631024 \times 10^{-12}}}
\newcommand{\PhiZeroVal}{\ensuremath{2.06783385 \times 10^{-15}}}
\newcommand{\phiVal}{\ensuremath{1.61803399}}
\newcommand{\eVVal}{\ensuremath{1.60217663 \times 10^{-19}}}
\newcommand{\GFVal}{\ensuremath{1.16637870 \times 10^{-5}}}
\newcommand{\lambdaProtonVal}{\ensuremath{1.32140986 \times 10^{-15}}}
\newcommand{\ERinfVal}{\ensuremath{2.17987236 \times 10^{-18}}}
\newcommand{\fRinfVal}{\ensuremath{3.28984196 \times 10^{15}}}
\newcommand{\sigmaSBVal}{\ensuremath{5.67037442 \times 10^{-8}}}
\newcommand{\WienVal}{\ensuremath{2.89777196 \times 10^{-3}}}
\newcommand{\kEVal}{\ensuremath{8.98755179 \times 10^{9}}}

% === \ae ther Densities ===
\newcommand{\rhoMass}{\rho_\text{\ae}^{(\text{mass})}}
\newcommand{\rhoMassVal}{\ensuremath{3.89343583 \times 10^{18}}}
\newcommand{\rhoEnergy}{\rho_\text{\ae}^{(\text{energy})}}
\newcommand{\rhoEnergyVal}{\ensuremath{3.49924562 \times 10^{35}}}
\newcommand{\rhoFluid}{\rho_\text{\ae}^{(\text{fluid})}}
\newcommand{\rhoFluidVal}{\ensuremath{7.00000000 \times 10^{-7}}}

% === Draft Options ===
\newif\ifvamdraft
% \vamdrafttrue
\ifvamdraft
\RequirePackage{showframe}
\fi

% === Load Once ===
\RequirePackage{ifthen}
\newboolean{vamstyleloaded}
\ifthenelse{\boolean{vamstyleloaded}}{}{\setboolean{vamstyleloaded}{true}

% === Page ===
\RequirePackage[a4paper, margin=2.5cm]{geometry}

% === Fonts ===
\RequirePackage[T1]{fontenc}
\RequirePackage[utf8]{inputenc}
\RequirePackage[english]{babel}
\RequirePackage{textgreek}
\RequirePackage{mathpazo}
\RequirePackage[scaled=0.95]{inconsolata}
\RequirePackage{helvet}

% === Math ===
\RequirePackage{amsmath, amssymb, mathrsfs, physics}
\RequirePackage{siunitx}
\sisetup{per-mode=symbol}

% === Tables ===
\RequirePackage{graphicx, float, booktabs}
\RequirePackage{array, tabularx, multirow, makecell}
\newcolumntype{Y}{>{\centering\arraybackslash}X}
\newenvironment{tighttable}[1][]{\begin{table}[H]\centering\renewcommand{\arraystretch}{1.3}\begin{tabularx}{\textwidth}{#1}}{\end{tabularx}\end{table}}
\RequirePackage{etoolbox}
\newcommand{\fitbox}[2][\linewidth]{\makebox[#1]{\resizebox{#1}{!}{#2}}}

% === Graphics ===
\RequirePackage{tikz}
\usetikzlibrary{3d, calc, arrows.meta, positioning}
\RequirePackage{pgfplots}
\pgfplotsset{compat=1.18}
\RequirePackage{xcolor}

% === Code ===
\RequirePackage{listings}
\lstset{basicstyle=\ttfamily\footnotesize, breaklines=true}

% === Theorems ===
\newtheorem{theorem}{Theorem}[section]
\newtheorem{lemma}[theorem]{Lemma}

% === TOC ===
\RequirePackage{tocloft}
\setcounter{tocdepth}{2}
\renewcommand{\cftsecfont}{\bfseries}
\renewcommand{\cftsubsecfont}{\itshape}
\renewcommand{\cftsecleader}{\cftdotfill{.}}
\renewcommand{\contentsname}{\centering \Huge\textbf{Contents}}

% === Sections ===
\RequirePackage{sectsty}
\sectionfont{\Large\bfseries\sffamily}
\subsectionfont{\large\bfseries\sffamily}

% === Bibliography ===
\RequirePackage[numbers]{natbib}

% === Links ===
\RequirePackage{hyperref}
\hypersetup{
    colorlinks=true,
    linkcolor=blue,
    citecolor=blue,
    urlcolor=blue,
    pdftitle={The Vortex \AE ther Model},
    pdfauthor={Omar Iskandarani},
    pdfkeywords={vorticity, gravity, \ae ther, fluid dynamics, time dilation, VAM}
}
\urlstyle{same}
\RequirePackage{bookmark}

% === Misc ===
\RequirePackage[none]{hyphenat}
\sloppy
\RequirePackage{empheq}
\RequirePackage[most]{tcolorbox}
\newtcolorbox{eqbox}{colback=blue!5!white, colframe=blue!75!black, boxrule=0.6pt, arc=4pt, left=6pt, right=6pt, top=4pt, bottom=4pt}
\RequirePackage{titling}
\RequirePackage{amsfonts}
\RequirePackage{titlesec}
\RequirePackage{enumitem}

\AtBeginDocument{\RenewCommandCopy\qty\SI}

\pretitle{\begin{center}\LARGE\bfseries}
\posttitle{\par\end{center}\vskip 0.5em}
\preauthor{\begin{center}\large}
\postauthor{\end{center}}
\predate{\begin{center}\small}
\postdate{\end{center}}

\endinput
}
  \usepackage{ifthen} % we can use it safely now
  \usepackage{import}
  \usepackage{subfiles}
  \usepackage{hyperref}
  \usepackage{graphicx}
  \usepackage{amsmath, amssymb, physics}
  \usepackage{siunitx}
  \usepackage{tikz}
  \usetikzlibrary{arrows.meta, positioning}
  \usepackage{booktabs}
  \usepackage{caption}
  \usepackage{array, tabularx}
  \usepackage{listings}
  \usepackage{bookmark}
  \usepackage{newtxtext,newtxmath}
  \usepackage[scaled=0.95]{inconsolata}
  \usepackage{mathrsfs}
  % vamappendixsetup.sty

\newcommand{\titlepageOpen}{
  \begin{titlepage}
  \thispagestyle{empty}
  \centering
  {\Huge\bfseries \papertitle \par}
  \vspace{1cm}
  {\Large\itshape\textbf{Omar Iskandarani}\textsuperscript{\textbf{*}} \par}
  \vspace{0.5cm}
  {\large \today \par}
  \vspace{0.5cm}
}

% here comes abstract
\newcommand{\titlepageClose}{
  \vfill
  \null
  \begin{picture}(0,0)
  % Adjust position: (x,y) = (left, bottom)
  \put(-200,-40){  % Shift 75pt left, 40pt down
    \begin{minipage}[b]{0.7\textwidth}
    \footnotesize % One step bigger than \tiny
    \renewcommand{\arraystretch}{1.0}
    \noindent\rule{\textwidth}{0.4pt} \\[0.5em]  % ← horizontal line
    \textsuperscript{\textbf{*}}Independent Researcher, Groningen, The Netherlands \\
    Email: \texttt{info@omariskandarani.com} \\
    ORCID: \texttt{\href{https://orcid.org/0009-0006-1686-3961}{0009-0006-1686-3961}} \\
    DOI: \href{https://doi.org/\paperdoi}{\paperdoi} \\
    License: CC-BY 4.0 International \\
    \end{minipage}
  }
  \end{picture}
  \end{titlepage}
}
  \begin{document}

  % === Title page ===
  \titlepageOpen

  \begin{abstract}
        This document presents a comprehensive topological classification of matter, energy, and interaction types within the Vortex Æther Model (VAM), a fluid-dynamic framework wherein all particles arise from structured vortex knots in an incompressible, inviscid æther. The taxonomy organizes elementary and composite particles according to knot topology (torus, hyperbolic, cable, satellite), chirality, and internal curvature tension. A foundational distinction is established between chiral and achiral knots: chiral knots couple to gravitational swirl fields and are classified as matter (or antimatter under reversed chirality), while achiral hyperbolic knots are expelled due to their misalignment energy, and trivial knots such as unknots and Hopf links passively follow swirl lines without gravitational coupling. A formal classifier equation is introduced to predict gravitational response from knot properties, and a hierarchical framework is built connecting fundamental knot types to leptons, quarks, bosons, hadrons, atoms, and molecules. The taxonomy also delineates dark energy and dark matter in terms of excluded topologies and residual swirl fields, respectively. This knot-based ontology aims to unify particle physics and gravitation through topological fluid dynamics, offering a deterministic and geometric alternative to quantum field theory and spacetime curvature.


        \begin{figure}[H]
            \centering
            \footnotesize
            \scalebox{0.75}{
                \begin{tikzpicture}[
                  box/.style = {draw, rounded corners, minimum width=2.0cm, minimum height=0.8cm, font=\small, align=center},
                  arrow/.style = {->, thick},   node distance=1.5cm and 1.5cm
                ]
                    % Inputs
                    \node[box] (topology) {Knot Topology};
                    \node[box, right=of topology] (chirality) {Chirality};
                    \node[box, right=of chirality] (tension) {Tension};

                    % Swirl coupling
                    \node[box, below=of chirality, minimum width=5.5cm] (coupling) {Swirl Coupling Condition};

                    % Gravitational response
                    \node[box, below=of coupling] (grav) {Gravitational Response};

                    % Gravitational classes
                    \node[box, below left=2cm and 2.0cm of grav] (matter) {Chiral\\Leptons \& Quarks};
                    \node[box, below=2cm of grav] (boson) {Achiral + No Tension\\→ Bosons};
                    \node[box, below right=2cm and 2.2cm of grav] (dark) {Achiral + Tension\\→ Dark Energy Knots};

                    % Subclasses
                    \node[box, below=of matter, xshift=-1.1cm] (leptons) {Leptons\\(e⁻ = T(2,3))};
                    \node[box, below=of matter, xshift=+1.0cm] (quarks) {Quarks\\(6₂, 7₄, 8₁₉)};

                    \node[box, below=of boson, xshift=-2.1cm] (photon) {Photon\\(unknot)};
                    \node[box, below=of boson] (gluon) {Gluon\\(Hopf link)};
                    \node[box, below=of boson, xshift=+2.3cm] (zboson) {Z⁰\\(neutral loop)};

                    \node[box, below=of dark] (darkex) {Example: \(4_{1}, 8_{17}\)\\ (achiral hyperbolic)};

                    % Arrows to coupling
                    \draw[arrow] (topology.south) -- ++(0,-0.5) -| (coupling.north west);
                    \draw[arrow] (chirality.south) -- (coupling.north);
                    \draw[arrow] (tension.south) -- ++(0,-0.5) -| (coupling.north east);

                    % Arrows down flow
                    \draw[arrow] (coupling.south) -- (grav.north);
                    \draw[arrow] (grav.south) -- (matter.north);
                    \draw[arrow] (grav.south) -- (boson.north);
                    \draw[arrow] (grav.south) -- (dark.north);

                    % Particle branches
                    \draw[arrow] (matter.south) -- (leptons.north);
                    \draw[arrow] (matter.south) -- (quarks.north);

                    \draw[arrow] (boson.south) -- (photon.north);
                    \draw[arrow] (boson.south) -- (gluon.north);
                    \draw[arrow] (boson.south) -- (zboson.north);

                    \draw[arrow] (dark.south) -- (darkex.north);
                \end{tikzpicture}
            }
            \caption{Knot Classification by Swirl Coupling.
                The flowchart visualizes how knot topology, chirality, and curvature tension determine gravitational behavior, and how this leads to specific particle subclasses:
                    \\ \textbf{Chiral knots} align with swirl fields and form matter: \textbf{leptons} (torus knots) and \textbf{quarks} (hyperbolic knots).
                    \\ \textbf{Achiral knots with tension} are expelled, forming \textbf{dark energy} candidates.
                    \\ \textbf{Achiral, tensionless} structures like unknots and Hopf links are \textbf{bosons}, passively guided by swirl tubes.
            }
        \end{figure}

  \end{abstract}

  \titlepageClose
\fi


% ============= Begin of content ============
\section*{\papertitle}



\section*{I. Overview}

\subsection*{Foundational Postulate: Chirality and Swirl Gravity Response}
In the Vortex Æther Model (VAM), the response of a knot to swirl-induced gravitation depends not just on chirality, but also on internal topological structure:

\begin{itemize}
    \item \textbf{Achiral hyperbolic knots} (with mass and internal tension) are \textbf{expelled} from vortex tubes due to their inability to align with the swirl field.
    \item \textbf{Unknots and Hopf links}, being topologically trivial or minimally linked and without curvature tension, are \textbf{not expelled}, but instead \textbf{passively follow} the structured æther swirl paths.
\end{itemize}

This distinction is critical: while both are achiral, only the structured knots with misalignment energy are repelled by the gravitational swirl gradient.

In the Vortex Æther Model (VAM), all physical matter arises from stable, chiral vortex knots in an incompressible, inviscid fluid-like æther. These vortex knots are classified by their topological features: torus knots, hyperbolic knots, cable knots, and satellite knots. The chirality (↺ ccw = matter, ↻ cw = antimatter) determines gravitational interaction, while knot complexity governs mass and stability.

\subsection*{Axioms of the VAM Knot Taxonomy}
\begin{enumerate}
    \item All physical entities are structured as vortex knots in an inviscid, incompressible æther.
    \item Gravitational interaction arises from chirality-swirl coupling: only chiral knots couple to swirl fields.
    \item Helicity encodes mass-energy; more complex knots store more curvature energy.
    \item Achiral knots with internal tension resist swirl alignment and are expelled.
    \item Unknotted or tensionless forms (bosons) follow swirl field lines passively.
\end{enumerate}

\begin{center}
    \fbox{
        \parbox{0.95\textwidth}{
            \textbf{Hyperbolic Mass Wells —} Chiral hyperbolic vortex knots generate deep ætheric swirl wells due to their internal curvature and topological linking. These defects concentrate rotational energy and induce strong pressure gradients in the surrounding æther field. As a result, they act as gravitational mass sources within the Vortex Æther Model, mimicking the mass-energy tensor of General Relativity through structured vorticity rather than spacetime curvature.
        }
    }
\end{center}

\section*{II. Taxonomic Layers}

\subsection*{Fundamental Knot Species}
\begin{center}
    \footnotesize
    \begin{tabular}{|l|l|l|l|l|l|}
        \hline
        \textbf{Knot Type} & \textbf{Example} & \textbf{Chirality} & \textbf{Geometry} & \textbf{VAM Role} & \textbf{Gravity Reactive?} \\
        \hline
        Torus Knot & \( T(2,3), T(2,5) \) & Chiral & Toroidal & Leptons (e.g., \( e^-, \mu^- \)) & Yes \\
        Hyperbolic Knot & \( 6_2, 7_4 \) & Chiral & Hyperbolic & Quarks (u, d, s...) & Yes \\
        Achiral Hyperbolic & \( 8_{17} \) & None & Hyperbolic & Dark Energy knots & No — expelled \\
        Unknot / Hopf Link & \( \varnothing, \text{Link} \) & None & Trivial & Bosons (γ, g, \( Z^0 \)) & No — passive \\
        \hline
    \end{tabular}
\end{center}

\subsection*{II. Composite Knots and Cables}
\begin{center}
    \footnotesize
    \begin{tabular}{|l|p{8cm}|l|}
        \hline
        \textbf{Structure} & \textbf{Description} & \textbf{VAM Interpretation} \\
        \hline
        Cable Knot \( C(p,q)(T(2,3)) \) & Thread wound on trefoil core & Baryons (p, n) \\
        Satellite Knot & Composite of multiple knots in thick torus & Hadrons, mesons \\
        Knot Sum \( K_1 \# K_2 \) & Topological addition of two knots & Multi-core particles \\
        \hline
    \end{tabular}
\end{center}

\section*{III. Chemical and Physical Emergence}

\subsection*{Leptonic Layer (Torus Knot Dominated)}
\begin{itemize}
    \item Standalone leptons (e.g., \( e^- = T(2,3) \))
    \item Outer electron orbitals in atoms
    \item Basis of chemical behavior in nonmetals
\end{itemize}

\subsection*{Hadronic Layer (Cable and Satellite Knots)}
\begin{itemize}
    \item Protons = cable of trefoil, e.g., \( C(2,1)(T(2,3)) \)
    \item Neutrons = composite cable-satellite configuration
    \item Hadrons as vortex composites with stable embedding
\end{itemize}

\subsection*{Atomic Layer (Knot Couplings)}
\begin{itemize}
    \item Hydrogen = proton + electron knot coupling
    \item Atoms = quark core + lepton orbital system
    \item Periodic table classes emerge from electron topology
\end{itemize}

\subsection*{Molecular Layer (Topological Bonding)}
\begin{itemize}
    \item Molecules = stable linkage of electron vortices
    \item Covalent bonds = shared torus knot interactions
    \item Ionic bonds = asymmetric vortex attraction/repulsion
\end{itemize}

\section*{IV. Exotic Layers}

\subsection*{Dark Energy Layer}
\begin{itemize}
    \item Achiral hyperbolic knots that do not couple to swirl fields
    \item Expelled from gravitational tubes — repelled by structured vorticity
\end{itemize}

\subsection*{Dark Matter Layer}
\begin{itemize}
    \item Residual galactic-scale swirl fields (net helicity)
    \item Not knots themselves, but fluid field gradients
\end{itemize}

\subsection*{Bosonic Swirl Followers}
\begin{itemize}
    \item Unknots and Hopf links do not gravitate
    \item Passively follow structured æther vortex tubes (swirl gravity channels)
    \item Include photons, gluons, and neutral weak bosons
\end{itemize}

\section*{Chirality and Time}
\begin{itemize}
    \item Matter = ccw knots (↺)
    \item Antimatter = cw knots (↻)
\end{itemize}
Gravitational interaction emerges from swirl coupling:
\[
F_g \propto \vec{\omega}_\text{local} \cdot \vec{\omega}_\text{swirl}
\]

\section*{VI. Summary Diagram (to be rendered)}
Tree showing levels:
\begin{itemize}
    \item Knot Species → Particle Type → Atom → Molecule
    \item With chirality, helicity, and knot geometry labeled
\end{itemize}

\section*{VII. Taxonomy Equation for Gravitational Behavior}

To formalize the gravitational response of vortex knots, we define a classifier function \( \mathcal{G} \):

Let:
\begin{itemize}
    \item \( \chi \in \{-1, 0, +1\} \): chirality
    \item \( H \geq 0 \): helicity
    \item \( \tau \in \{0,1\} \): structural tension
    \item \( \mathcal{G} \in \{-1, 0, +1\} \): gravitational response
\end{itemize}

\[
\boxed{\mathcal{G} = \operatorname{sign}(\chi \cdot H) + \delta_{\chi, 0} \cdot \left[ -\tau + (1 - \tau) \right]}
\]

Where:
\[ \operatorname{sign}(x) = \begin{cases}  +1 & x > 0 \\
                                            0 & x = 0 \\
                                           -1 & x < 0
                            \end{cases}, \quad \delta_{\chi,0} = \begin{cases} 1 & \chi = 0 \\
                                                                               0 & \text{otherwise}
                                                                 \end{cases}
\]

\subsection*{Interpretation}

\begin{center}
\begin{tabular}{|c|c|c|c|l|}
\hline
\( \chi \) & \( H \) & \( \tau \) & \( \mathcal{G} \) & Interpretation \\
\hline
±1 & >0 & 1 & ±1 & Gravity-reactive matter or antimatter \\
0 & >0 & 1 & −1 & Expelled achiral hyperbolic knot \\
0 & ~0 & 0 & 0 & Passively guided (unknot, Hopf link) \\
\hline
\end{tabular}
\end{center}



\section*{IX. Topological Reinterpretation of Standard Model Forces}

The Vortex Æther Model (VAM) replaces conventional gauge-field descriptions of the fundamental forces with dynamical responses of vortex knots to structured æther flows. The same parameters that govern gravitational behavior — chirality \( \chi \), helicity \( H \), and structural tension \( \tau \) — also explain the emergence of the strong and weak nuclear interactions.

\subsection*{A. Gravity as Swirl Coupling}

Gravity in VAM emerges from alignment between the knot’s local vorticity and the background swirl field:

\[
F_g \propto \vec{\omega}_{\text{local}} \cdot \vec{\omega}_{\text{swirl}}
\]

Only chiral knots couple positively or negatively to swirl gradients. Achiral knots are expelled (if structured), or guided (if tensionless), but do not gravitationally attract.

\subsection*{B. Strong Force as Topological Confinement}

The strong interaction is interpreted as a consequence of topological entanglement:

\begin{itemize}
    \item \textbf{Quarks} are modeled as chiral hyperbolic knots (e.g., \( 6_2, 7_4 \)).
    \item These knots cannot be isolated without violating fluid continuity — leading to \emph{topological confinement}.
    \item \textbf{Gluons} are modeled as Hopf-linked swirl pulses that mediate reconfiguration of swirl tension.
\end{itemize}

Thus, confinement is not a field-mediated interaction but a geometric property of vortex interlinkage:
\[
\text{Confinement} = \text{Topological Inseparability of Chiral Hyperbolic Knots}
\]

\subsection*{C. Weak Force as Chirality Reversal and Knot Decay}

The weak interaction arises from chirality transitions and knot reconnections:

\begin{itemize}
    \item \textbf{Beta decay} is interpreted as a transition between knot classes, such as torus → unknot or torus → link + neutrino.
    \item \textbf{W⁺, W⁻, Z⁰} bosons are modeled as high-tension, guided bosonic structures that carry localized swirl energy.
    \item \textbf{Neutrinos} are modeled as topologically neutral Hopf-linked loops — achiral and nearly swirl-invisible.
\end{itemize}

The weak force thus measures a system’s capacity to transition between chirality classes through reconnection.

\subsection*{D. Summary Table}

\begin{center}
\begin{tabular}{|c|c|c|c|}
\hline
\textbf{Force} & \textbf{VAM Mechanism} & \textbf{Topological Interpretation} & \textbf{Example} \\
\hline
Gravity & Swirl coupling & Chirality alignment with swirl field & \( e^- \rightarrow M_{\text{eff}}(r) \) \\
Strong & Confinement & Hyperbolic knot entanglement & \( u, d \) inside proton \\
Weak & Reconnection & Chirality/knot-class decay & \( n \rightarrow p + e^- + \bar{\nu}_e \) \\
\hline
\end{tabular}
\end{center}

\subsection*{E. Suggested Diagrams}

\begin{itemize}
    \item \textbf{Strong force:} Show two hyperbolic knots with interlocked loop regions representing confinement.
    \item \textbf{Weak force:} Illustrate a trefoil reconnecting into an unknot + twist-loop (neutrino).
\end{itemize}

\bigskip

These reinterpretations support the hypothesis that all Standard Model interactions arise from a unified, vorticity-based ontology within a topological superfluid æther.







  \subsection*{Helicity Interference Suppression Term from Vortex Knot Packing}

In the Vortex Æther Model (VAM), mass arises from swirl energy stored in knotted structures within the incompressible æther. For composite particles composed of multiple vortex cores (e.g., protons, nuclei), we must account for mutual interference between their individual swirl fields.

We define a suppression factor \( \xi(n) \) that reduces the effective inertial mass based on the number of interacting cores \( n \):

\begin{equation}
\boxed{
\xi(n) = 1 - \beta \cdot \log(n)
}
\qquad \text{with } \beta \approx 0.06
\end{equation}

This form reflects the fact that helicity interference grows sublinearly with knot number, due to angular misalignment and partial swirl overlap in tightly packed vortex systems.

\paragraph{Derivation:}
The total helicity of a multi-core knot system can be written as:

\begin{equation}
\mathcal{H}_{\text{total}} = \sum_i \mathcal{H}_i + \sum_{i \neq j} \int_V \vec{v}_i \cdot (\nabla \times \vec{v}_j) \, dV
\end{equation}

The first term is the sum of self-helicities of the individual cores; the second term includes all cross-helicity contributions. Due to vortex misalignment and destructive interference in the composite system, these cross-terms are generally negative and grow roughly as:

\[
\sum_{i \neq j} \mathcal{H}_{ij} \sim -\log(n)
\]

This leads to an effective helicity scaling of:

\[
\mathcal{H}_{\text{effective}} \sim n - \log(n)
\quad \Rightarrow \quad
\xi(n) = \frac{\mathcal{H}_{\text{effective}}}{n} = 1 - \beta \log(n)
\]

with \( \beta \) capturing the average angular interference per added core.

\paragraph{Refined Mass Formula:}

\begin{equation}
\boxed{
M = \left( \frac{1}{\varphi} \right) \cdot \left( \frac{4}{\alpha} \right) \cdot \underbrace{\left(1 - \beta \log(n)\right)}_{\text{helicity interference}} \cdot \left( \frac{1}{2} \rho_\text{\ae} C_e^2 V \right)
}
\end{equation}

\paragraph{Physical Interpretation:}
\begin{itemize}
  \item \( \frac{1}{\varphi} \): Topological packing constraint — fewer tight configurations per volume.
  \item \( \frac{4}{\alpha} \): Swirl–electromagnetic amplification factor.
  \item \( \xi(n) \): Reduces net mass as vortex cores increase, due to interference.
  \item \( \rho_\text{\ae} C_e^2 V \): Raw vortex energy from fluid swirl.
\end{itemize}

\noindent
This correction accounts for the ~15\% discrepancy observed between hydrogen and helium–beryllium mass predictions, and reveals a fundamental geometrical and topological origin for inertial mass suppression in composite structures.

% ============== End of content =============

\ifdefined\standalonechapter\else
    \bibliographystyle{unsrt}
    \bibliography{../../references}
    \end{document}
\fi
\section{Introduction}\label{sec:inleiding}

Despite the empirical success of the Standard Model (SM) of particle physics and General Relativity (GR), fundamental questions remain unresolved: What is the physical origin of mass? Why do gauge interactions exhibit their particular symmetries? What gives rise to natural constants such as $\hbar$, $e$, or $\alpha$ beyond dimensional convenience?

Mainstream physics relies heavily on abstract mathematical formalisms—such as symmetry groups, Lagrangian terms, and quantum operators—that, while predictive, often obscure the underlying physical ontology. This paper proposes an alternative: the \emph{Vortex Æther Model} (VAM), a mechanistic, fluid-dynamic framework in which spacetime and all physical phenomena emerge from structured motion in a compressible, superfluid-like æther.

In VAM, elementary particles are not point-like fields but stable, knotted vortex structures embedded in the æther. Observable properties such as mass, charge, spin, and flavor are reinterpreted as topological and dynamical characteristics—circulation strength, core radius, swirl helicity—of these vortex knots. Gauge and Higgs interactions are expressed as manifestations of fluid tension, reconnection, and swirl transfer.

Crucially, this is not merely a reformulation of mathematical symbols. The goal of VAM is to provide an \emph{ontological replacement} for conventional quantum field theory: a physically intuitive, testable substrate from which all constants and couplings emerge. Within this framework, the Standard Model is reconstructed from five physically meaningful ætheric quantities: swirl velocity $C_e$, core radius $r_c$, æther density $\rho_\text{\ae}$\footnote{The VAM framework distinguishes between three æther densities depending on context: fluid, energy, and mass-equivalent. See Table~\ref{tab:ae_densities_foot} for a breakdown of these definitions. A mismatch in interpretation leads to inconsistencies in field derivations.}, maximum force $F^{\text{max}}_{\text{\ae}}$, and circulation $\Gamma$.

This paper presents a full reformulation of the Standard Model Lagrangian using these VAM-derived units and fields. Each term acquires a mechanical and geometric interpretation, leading to a unified description where quantum phenomena, gauge structures, and mass generation are consequences of vortex dynamics in an inviscid æther. A full field-theoretic derivation of the model dynamics is presented in Appendix~\ref{sec:EL-derivation}.

Historically, this effort revives foundational ideas from Kelvin's vortex-atom hypothesis and Maxwell's æther mechanics, updating them within a modern context informed by quantum fluids, superfluid analogs of gravity, and topological field theory. See, for example, Volovik's emergent gravity framework in helium II~\cite{Volovik2003UniverseInHelium}, Barceló et al.'s review of analog spacetime geometries~\cite{Barcelo2005AnalogueGravityReview}, and Kleckner and Irvine's experimental realization of knotted vortices~\cite{Kleckner2013KnottedVortices}. While this paper is designed to be standalone, these works contextualize the broader landscape of fluid-based physical models.

By grounding the abstract structures of modern physics in vortex geometry, VAM aims to bridge the gap between formal theory and intuitive physical mechanisms—offering not only reinterpretation, but a re-foundation of particle physics itself.

This work builds on a series of earlier papers developing the Vortex Æther Model (VAM). In \cite{iskandarani2025timedilation}, proper time was defined through internal angular motion of vortex cores, introducing the concept of \("\)swirl clocks\("\) as the microscopic origin of time dilation. This was extended in \cite{iskandarani2025swirlgravity}, which proposed that gradients in swirl clocks — arising from non-uniform vorticity — mimic gravitational curvature, including analogs to event horizons. The present work synthesizes these concepts into a variational field-theoretic framework, reformulating the Standard Model Lagrangian in terms of helicity, core structure, and topological æther dynamics.

\subsection*{Postulates of the Vortex Æther Model}
\begin{table}[h!]
    \centering
    \begin{tabular}{rl}
        \hline
        \textbf{1. Continuous Space} & Space is Euclidean, incompressible and inviscid. \\
        \textbf{2. Knotted Particles} & Matter consists of topologically stable vortex nodes. \\
        \textbf{3. Vorticity} & The vortex circulation is conserved and quantized. \\
        \textbf{4. Absolute Time} & Time flows uniformly throughout the æther. \\
        \textbf{5. Local Time} & Time is locally slower due to pressure and vorticity gradients. \\
        \textbf{6. Gravity} & Emerges from vorticity-induced pressure gradients. \\
        \hline
        \bottomrule
    \end{tabular}
    \caption{Postulates of the Vortex Æther Model (VAM).}
    \label{tab:postulates}
\end{table}

The postulates replace spacetime curvature with structured rotational flows and thus form the foundation for emergent mass, time, inertia, and gravity.

\section*{Terminology and Classical Correspondence}

We introduce several novel constructs to describe the vortex-based field framework. For clarity, Table~\ref{tab:vam_definitions} provides precise definitions and links to standard physics concepts.

\begin{table}[H]
    \centering
    \scriptsize
    \renewcommand{\arraystretch}{1.3}
    \begin{tabular}{|l|l|l|}
        \hline
        \textbf{Term} & \textbf{Definition in VAM} & \textbf{Analogy in Established Theory} \\
        \hline
        \makecell[l]{Swirl Clock} &
        \makecell[l]{Proper time defined by internal angular frequency \\ $\omega_0$ of a vortex core} &
        \makecell[l]{Atomic clock (GR); spin-precession \\ in gyroscopes} \\
        \hline
        \makecell[l]{Swirl Lagrangian} &
        \makecell[l]{Field Lagrangian including helicity term \\ $\lambda (\mathbf{v} \cdot \boldsymbol{\omega})$} &
        \makecell[l]{Chern–Simons terms; \\ topological terms in QFT} \\
        \hline
        \makecell[l]{Helicity Time} &
        \makecell[l]{Clock rate modulated by helicity density: \\ $d\tau \propto \mathbf{v} \cdot \boldsymbol{\omega}$} &
        \makecell[l]{Phase evolution in rotating frames; \\ action-angle methods} \\
        \hline
        \makecell[l]{Core Radius $r_c$} &
        \makecell[l]{Characteristic radius of maximal vorticity \\ and core energy density} &
        \makecell[l]{Healing length in BECs; \\ flux tube radius in QCD} \\
        \hline
        \makecell[l]{Swirl Speed $C_e$} &
        \makecell[l]{Tangential speed of æther flow \\ at core radius} &
        \makecell[l]{Sound speed in superfluids; \\ Lorentz frame velocity} \\
        \hline
        \makecell[l]{Swirl Horizon} &
        \makecell[l]{Boundary beyond which $\omega_{\text{obs}} \to 0$ \\ and clocks stall} &
        \makecell[l]{GR event horizon; \\ ergosphere boundary (Kerr geometry)} \\
        \hline
    \end{tabular}
    \caption{Key theoretical constructs in the Vortex Æther Model (VAM), mapped to classical and quantum analogs for interpretability.}
    \label{tab:vam_definitions}
\end{table}


These constructs provide an intuitive bridge between fluid mechanics, quantum field theory, and emergent spacetime phenomena, facilitating reinterpretation of the Standard Model Lagrangian in a vortex-based æther framework.
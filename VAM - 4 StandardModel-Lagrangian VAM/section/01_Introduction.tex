\section{Introduction}\label{sec:inleiding}

Despite the empirical success of the Standard Model (SM) of particle physics and General Relativity (GR), fundamental questions remain unresolved: What is the physical origin of mass? Why do gauge interactions exhibit their particular symmetries? What gives rise to natural constants such as $\hbar$, $e$, or $\alpha$ beyond dimensional convenience? And what, ultimately, is the physical nature of time?

Mainstream physics relies heavily on abstract mathematical formalisms—symmetry groups, Lagrangian operators, curved spacetimes—that, while predictive, often obscure the underlying ontology. This paper proposes an alternative: the \emph{Vortex Æther Model} (VAM), a fluid-mechanical framework in which all physical phenomena emerge from structured vorticity and pressure gradients within an incompressible, inviscid æther medium. Unlike GR, which interprets mass and gravity through geometry, VAM models them as dynamical properties of knotted vortex flows.

In this picture, elementary particles are not point-like excitations, but topologically stable vortex knots embedded in the æther. Observable properties—mass, charge, spin, and flavor—emerge from circulation, helicity, and core geometry. Gauge interactions arise from fluid tension and reconnection; symmetry breaking becomes a topological bifurcation. Crucially, time itself becomes layered: not a single scalar parameter, but a family of time modes shaped by internal rotation, circulation loops, and swirl phase gradients.

The full VAM temporal ontology, introduced in \cite{iskandarani2025swirlgravity}, distinguishes six fundamental time modes:
\begin{itemize}
    \item $\mathcal{N}$ — Aithēr-Time: global causal substrate of the æther;
    \item $\nu_0$ — Now-Point: localized absolute simultaneity;
    \item $\tau$ — Chronos-Time: proper time of observers embedded in the æther;
    \item $S(t)$ — Swirl Clock: phase accumulation inside vortex knots;
    \item $T_v$ — Vortex Proper Time: loop-integrated time along circulation paths;
    \item $\bar{t}$ — External Clock Time: far-field coordinate time in laboratory instruments;
    \item $\kappa$ — Kairos Moment: topological or energetic bifurcation in vortex evolution.
\end{itemize}

Each of these time modes plays a role in how fields evolve, interact, and synchronize. In particular, the Swirl Clock $S(t)$ governs internal quantum phase evolution, while Chronos-Time $\tau$ tracks inertial dynamics. The appearance of mass, redshift, and even tunneling transitions (LENR) in VAM follows from modulations between these time layers.

This paper presents a full reformulation of the Standard Model Lagrangian using VAM field variables—swirl velocity $C_e$, core radius $r_c$, æther density $\rho_\text{\ae}$\footnote{VAM distinguishes between $\rho_\text{\ae}^{(\text{fluid})}$, $\rho_\text{\ae}^{(\text{energy})}$, and $\rho_\text{\ae}^{(\text{mass})}$. See Table~\ref{tab:ae_densities_foot}.}, maximum force $F^{\text{max}}_{\text{\ae}}$, and quantized circulation $\Gamma$—where every term is given a mechanical, topological, and temporal interpretation.

The goal of this reformulation is not symbolic substitution, but ontological grounding. Each coupling, interaction, and symmetry-breaking term is recast as a consequence of vortex topology evolving over the time modes of the æther. The resulting Lagrangian unifies quantum behavior, gauge fields, and mass generation as emergent properties of structured vorticity in a flat 3D fluid medium governed by absolute time.

This work synthesizes and extends prior developments of the VAM framework. In \cite{iskandarani2025timedilation}, proper time was derived from angular momentum density within rotating vortex cores, yielding quantized time rates via Swirl Clocks. This was extended in \cite{iskandarani2025swirlgravity} to show that gravitational analogs—such as redshift and horizon effects—can be reproduced from swirl gradients in a flat æther. The present paper integrates these concepts into a coherent variational field theory, reconstructing the Standard Model Lagrangian through the lens of æther dynamics, helicity conservation, and temporal stratification.

\subsection*{Postulates of the Vortex Æther Model}

\begin{table}[h!]
    \centering
    \begin{tabular}{rl}
        \hline
        \textbf{1. Continuous Space} & Space is Euclidean, incompressible and inviscid. \\
        \textbf{2. Knotted Particles} & Matter consists of topologically stable vortex nodes. \\
        \textbf{3. Vorticity} & Vortex circulation is conserved and quantized. \\
        \textbf{4. Absolute Time} & Time $\mathcal{N}$ flows uniformly across the æther. \\
        \textbf{5. Local Time Modes} & $\tau$, $S(t)$, $T_v$ slow relative to $\mathcal{N}$ near vortex structures. \\
        \textbf{6. Gravity} & Emerges from vorticity-induced pressure gradients. \\
        \hline
    \end{tabular}
    \caption{Postulates of the Vortex Æther Model (VAM).}
    \label{tab:postulates}
\end{table}

\noindent These postulates replace spacetime curvature with structured rotation and circulation, forming the physical substrate for the emergence of mass, time, gauge interaction, and gravitation.

\section*{Terminology and Classical Correspondence}

We introduce several novel constructs to describe the vortex-based field framework, each grounded in the layered temporal and topological ontology of the Vortex Æther Model (VAM). For clarity, Table~\ref{tab:vam_definitions} defines key quantities and maps them to their closest analogs in conventional physics. Notably, several of these constructs—such as the \emph{Swirl Clock}, \emph{Helicity Time}, and \emph{Swirl Horizon}—manifest distinct temporal modes in VAM’s time stratification.

\begin{table}[H]
    \centering
    \scriptsize
    \renewcommand{\arraystretch}{1.4}
    \begin{tabular}{|l|l|l|}
        \hline
        \textbf{Term} & \textbf{Definition in VAM} & \textbf{Analogy in Established Theory} \\
        \hline
        \makecell[l]{Swirl Clock $S(t)$} &
        \makecell[l]{Phase-based time mode defined by angular frequency $\omega_0$ \\ of a vortex core; stores internal rotational memory.} &
        \makecell[l]{Atomic clock (GR); spin-precession \\ in gyroscopes} \\
        \hline
        \makecell[l]{Swirl Lagrangian} &
        \makecell[l]{Field Lagrangian including topological helicity term \\ $\lambda (\mathbf{v} \cdot \boldsymbol{\omega})$; evolves over $S(t)$ and $T_v$.} &
        \makecell[l]{Chern–Simons terms; \\ topological terms in QFT} \\
        \hline
        \makecell[l]{Helicity Time} &
        \makecell[l]{Clock rate modulated by helicity density: \\ $d\tau \propto \mathbf{v} \cdot \boldsymbol{\omega}$; affects $\tau$.} &
        \makecell[l]{Phase evolution in rotating frames; \\ action-angle formalism} \\
        \hline
        \makecell[l]{Core Radius $r_c$} &
        \makecell[l]{Characteristic radius of maximal vorticity and \\ exponential decay scale for pressure and energy.} &
        \makecell[l]{Healing length in BECs; \\ flux tube radius in QCD} \\
        \hline
        \makecell[l]{Swirl Speed $C_e$} &
        \makecell[l]{Maximal tangential speed of æther flow at core radius; \\ appears in all mass and time dilation formulas.} &
        \makecell[l]{Sound speed in superfluids; \\ Lorentz frame velocity} \\
        \hline
        \makecell[l]{Swirl Horizon} &
        \makecell[l]{Boundary where observer swirl frequency $\omega_{\text{obs}} \to 0$; \\ vortex clocks stall ($d\tau/d\mathcal{N} \to 0$).} &
        \makecell[l]{GR event horizon; \\ ergosphere boundary (Kerr)} \\
        \hline
        \makecell[l]{Aithēr-Time $\mathcal{N}$} &
        \makecell[l]{Absolute causal background time of the æther; \\ universal evolution parameter for field action.} &
        \makecell[l]{Newtonian universal time; \\ background foliation time} \\
        \hline
        \makecell[l]{Vortex Proper Time $T_v$} &
        \makecell[l]{Loop-integrated circulation-based time: \\ $T_v = \oint \frac{dl}{v_\varphi(r)}$; governs vortex energy.} &
        \makecell[l]{GR proper time on a closed path; \\ orbital period in $\tau$} \\
        \hline
        \makecell[l]{Kairos Moment $\kappa$} &
        \makecell[l]{Irreversible topological bifurcation in vortex structure; \\ signals causal branch or LENR onset.} &
        \makecell[l]{Quantum transition; \\ symmetry breaking point} \\
        \hline
    \end{tabular}
    \caption{Key theoretical constructs in the Vortex Æther Model (VAM), mapped to classical and quantum analogs. Several terms represent distinct modes in VAM's temporal ontology.}
    \label{tab:vam_definitions}
\end{table}

These constructs provide an intuitive bridge between fluid mechanics, quantum field theory, and emergent spacetime phenomena. In the VAM framework, every interaction term in the Lagrangian evolves along one or more of the time modes listed above, and each conserved quantity—mass, charge, spin—emerges from circulation, helicity, and energy in the ætheric medium.


\section{Natural Æther Constants and Dimensional Reformulation}

The Vortex Æther Model (VAM) proposes a fundamental shift in how physical quantities are derived and interpreted. Rather than relying on constants introduced purely for dimensional consistency (as in Planck units), VAM defines a small set of physically grounded parameters that emerge from the topological and fluid-dynamical behavior of a compressible æther medium. These parameters—accessible through theoretical analysis and analog systems—serve as the natural units for describing mass, energy, charge, and time.

The five core æther parameters are:

\begin{itemize}
    \item \textbf{Swirl Velocity \( C_e \)}: The tangential velocity of vortex flow, typically around \( 10^6 \, \text{m/s} \), inferred from simulations of stable quantized vortices in Bose–Einstein condensates (BECs) \cite{Pethick2008BEC,Kleckner2013KnottedVortices}.
    \item \textbf{Core Radius \( r_c \)}: The minimal confinement radius of stable topological knots, matched to the proton charge radius (\( \sim 1.4 \times 10^{-15} \, \text{m} \)).
    \item \textbf{Æther Density \( \rho_\text{\ae} \)\footnote{The VAM framework distinguishes between three æther densities depending on context: fluid, energy, and mass-equivalent. See Table~\ref{tab:ae_densities_foot} for a breakdown of these definitions. A mismatch in interpretation leads to inconsistencies in field derivations.}}: This quantity appears in three distinct physical roles (see Table~\ref{tab:ae_densities_foot}).
    \item \textbf{Circulation Quantum \( \kappa \)}: Defined analogously to superfluid systems, where circulation is quantized as \( \kappa = h/m \) \cite{Donnelly1991QuantizedVortices}.
    \item \textbf{Maximum Force \( F^{\text{max}}_{\text{\ae}} \)}: The maximum stress transmissible through a coherent vortex core, tied to \( \rho_\text{\ae}^{\text{(energy)}} \), \( C_e \), and \( r_c \).
\end{itemize}

\vspace{0.5em}

Together, these quantities form a physically grounded unit system based on fluid geometry, not abstract spacetime constants. Table~\ref{tab:VAM_constants} summarizes how they reconstruct core constants of the Standard Model.
\begin{table}[H]
\centering
\footnotesize
\renewcommand{\arraystretch}{1.3}
\begin{tabular}{|c|c|c|p{7.5cm}|}
\hline
\textbf{Symbol} & \textbf{Name} & \textbf{Units} & \textbf{Physical Role} \\
\hline
\( \rho_\text{\ae}^{\text{fluid}} \) & Fluid Density & \( \mathrm{kg/m}^3 \) & Governs inertial dynamics and kinetic energy of vortices. Used in \( \frac{1}{2} \rho v^2 \). Approx. \( 7 \times 10^{-7} \, \mathrm{kg/m^3} \). \\
\hline
\( \rho_\text{\ae}^{\text{energy}} \) & Energy Density & \( \mathrm{J/m}^3 \) & Represents internal energy stored in the æther field. Estimated from Planck-tension bounds: \( \sim 3 \times 10^{35} \, \mathrm{J/m^3} \). \\
\hline
\( \rho_\text{\ae}^{\text{mass}} \) & Mass-Equivalent Density & \( \mathrm{kg/m}^3 \) & Derived via \( \rho_\text{\ae}^{\text{energy}} / c^2 \). Used in gravitational coupling. Approx. \( 3 \times 10^{18} \, \mathrm{kg/m^3} \). \\
\hline
\end{tabular}
\caption{Distinct æther densities used in VAM, depending on context and physical domain.}
\label{tab:ae_densities_foot}
\end{table}

\begin{table}[H]
    \centering
    \footnotesize
    \renewcommand{\arraystretch}{1.3}
    \begin{tabular}{|c|c|l|}
        \hline
        \textbf{Symbol} & \textbf{Expression} & \textbf{Interpretation} \\
        \hline
        \( \hbar_{\text{VAM}} \) & \( m_e C_e r_c \) & Angular impulse from swirl core (Planck analog) \\
        \hline
        \( c \) & \( \sqrt{\frac{2 F^{\text{max}}_{\text{\ae}} r_c}{m_e}} \) & Signal speed as elastic wave limit of æther \\
        \hline
        \( \alpha \) & \( \frac{2 C_e}{c} \) & Fine-structure constant from swirl-to-light ratio \\
        \hline
        \( e^2 \) & \( 8\pi m_e C_e^2 r_c \) & Electromagnetic charge as swirl tension through core \\
        \hline
        \( \Gamma \) & \( 2\pi r_c C_e \) & Circulation per core — quantized as \( h/m \) \\
        \hline
        \( v \) & \( \sqrt{\frac{F^{\text{max}}_{\text{\ae}} r_c^3}{C_e^2}} \) & Higgs-like amplitude from ætheric elasticity \\
        \hline
    \end{tabular}
    \caption{Derived constants and coupling strengths in VAM from vortex structure.}
    \label{tab:VAM_constants}
\end{table}

As one illustrative result, the rest mass of a particle becomes:
\[
M = \frac{\rho_\text{\ae}^{\text{fluid}} \Gamma^2}{L_k \pi r_c C_e^2}
\]
with \( L_k \) as the linking number of the knot. The full derivation appears in Appendix~\ref{sec:derivation-of-the-kinetic-energy-of-a-circular-vortex-loop}.

Thus, VAM replaces dimensionally convenient but ontologically opaque constants with experimentally accessible and fluid-dynamically derived quantities.
\section{Running Coupling Constants from Æther Density}

In conventional quantum field theory, coupling constants such as the fine-structure constant \( \alpha \) are not truly constant: they evolve with energy scale due to vacuum polarization effects. This scale dependence is governed by the renormalization group (RG) flow and often expressed as:
\begin{equation}
\alpha(k^2) = \frac{\alpha_0}{1 - \Pi(k^2)},
\end{equation}
where \( \Pi(k^2) \) encodes virtual particle contributions to vacuum screening at energy scale \( k \).

\vspace{0.5em}

In the Vortex Æther Model (VAM), we reinterpret this phenomenon from first principles: rather than arising from quantum vacuum fluctuations, RG-like behavior is attributed to \textbf{variations in the local structure of the æther medium} — specifically its density, compressibility, and vorticity field.

We therefore propose a spatially varying fine-structure constant, derived not from perturbative diagrams, but from fluid mechanics:
\begin{equation}
\alpha(\vec{x}) = \frac{e^2}{4\pi \varepsilon_0(\vec{x}) \hbar c_{\text{eff}}(\vec{x})} = \alpha_0 \cdot f\left( \rho_\text{\ae}(\vec{x}), \, |\vec{\omega}(\vec{x})| \right),
\end{equation}
where \( f \) is a function of the local æther density and vorticity magnitude. Here:

\begin{itemize}
    \item \( \rho_\text{\ae}(\vec{x}) \) refers to the \textbf{local fluid or energy density} of the æther,
    \item \( \vec{\omega} = \nabla \times \vec{v} \) is the \textbf{local vorticity field},
    \item \( c_{\text{eff}} \) is the \textbf{local signal speed}, analogous to a renormalized light speed.
\end{itemize}

This formulation introduces a direct mechanical analog to the renormalization of coupling constants in QED — but replaces abstract Feynman diagram sums with \textbf{fluid strain and energy flow} in a structured vortex medium.

\vspace{0.5em}

\textbf{Key relations governing the spatial variation include:}
\begin{equation}
    c_{\text{eff}}(\vec{x}) \propto \sqrt{ \frac{B(\vec{x})}{\rho_\text{\ae}(\vec{x})} }, \qquad
    \varepsilon_0(\vec{x}) \sim \frac{1}{\rho_\text{\ae}(\vec{x}) C_e^2},
\end{equation}
where:
- \( B(\vec{x}) \) is the \textbf{local bulk modulus} of the æther (its resistance to compression),
- \( \rho_\text{\ae}(\vec{x}) \) determines both inertial and field strength response,
- \( C_e \) remains the swirl velocity constant used for defining kinetic and electromagnetic scaling.

\vspace{0.5em}

\textbf{Physical interpretation:}
\begin{itemize}
    \item In regions of high vorticity or density (e.g., near a massive vortex core), the æther becomes more strained and stiffer, altering the effective permittivity \( \varepsilon_0 \) and wave propagation speed \( c_{\text{eff}} \).
    \item These local changes deform the apparent strength of interactions — much like vacuum polarization in standard QFT — but are sourced from \textbf{fluid mechanics} rather than quantum fluctuations.
    \item In this way, VAM replicates the core behavior of running couplings, but with a deterministic substrate.
\end{itemize}

\vspace{0.5em}

\textbf{Experimental Implications:}
This prediction suggests that \textbf{fundamental constants may vary measurably across spacetime} in regions of extreme vorticity, gravitational compression, or swirl torsion. This could be tested through:
\begin{itemize}
    \item High-precision atomic clocks near rotating bodies,
    \item Astrophysical spectroscopy across varying galactic curvature,
    \item Cosmological comparisons of α in high-redshift quasars \cite{shapiro2004variation, uzan2011varying},
    \item Analog vortex simulations in BECs with variable density profiles.
\end{itemize}

This reinterpretation unifies the idea of renormalization with mechanical properties of the vacuum itself, extending the VAM framework into the regime of coupling evolution without relying on perturbative QFT assumptions — and placing it in conceptual alignment with entropic gravity theories such as Verlinde's \cite{verlinde2016emergent}.



\section{Natural Æther Constants and Dimensional Reformulation}

The Vortex Æther Model (VAM) proposes a fundamental shift in how physical quantities are derived and interpreted. Rather than relying on constants introduced purely for dimensional consistency (as in Planck units), VAM defines a minimal set of physically grounded parameters that emerge from the topological and fluid-dynamical behavior of a compressible æther medium. These parameters—accessible through theoretical analysis and analog systems—serve as the natural units for describing mass, energy, charge, and time.

The five core æther parameters are:

\begin{itemize}
    \item \textbf{Swirl Velocity \( C_e \)}: The tangential velocity of stable vortex flow, typically around \( 10^6 \, \text{m/s} \), inferred from simulations of quantized vortices in Bose–Einstein condensates (BECs)~\cite{Pethick2008BEC, Kleckner2013KnottedVortices}.

    \item \textbf{Core Radius \( r_c \)}: The confinement radius of stable topological knots, matched to the proton charge radius (\( \sim 1.4 \times 10^{-15} \, \text{m} \)).

    \item \textbf{Æther Density \( \rho_\text{\ae} \)\footnote{VAM distinguishes between three æther densities depending on context: fluid density \( \rho_\text{\ae}^{\text{fluid}} \), energy density \( \rho_\text{\ae}^{\text{energy}} \), and mass-equivalent density \( \rho_\text{\ae}^{\text{mass}} \). See Table~\ref{tab:ae_densities_foot}. A mismatch in interpretation leads to inconsistency in field derivations.}}: Governs force, inertia, and topological energy storage.

    \item \textbf{Circulation Quantum \( \kappa \)}: Analogous to superfluid systems, defined via quantized loop integral \( \kappa = h / m \)~\cite{Donnelly1991QuantizedVortices}.

    \item \textbf{Maximum Force \( F^{\text{max}}_{\text{\ae}} \)}: The peak stress transmissible through a coherent vortex core; a derived quantity from \( \rho_\text{\ae}^{\text{(energy)}}, C_e, r_c \).
\end{itemize}

\vspace{0.5em}

Together, these quantities form a physically motivated alternative to Planck-scale dimensional primitives. They offer a complete unit system based on vortex structure, replacing geometric postulates with fluid dynamics. Table~\ref{tab:ae_densities_foot} details the distinctions among the density modes used in VAM.

\begin{table}[H]
\centering
\footnotesize
\renewcommand{\arraystretch}{1.3}
\begin{tabular}{|c|c|c|p{7.5cm}|}
\hline
\textbf{Symbol} & \textbf{Name} & \textbf{Units} & \textbf{Physical Role} \\
\hline
\( \rho_\text{\ae}^{\text{fluid}} \) & Fluid Density & \( \mathrm{kg/m^3} \) & Governs inertial motion of the æther. Appears in Bernoulli-type terms \( \frac{1}{2} \rho v^2 \). Estimated as \( \sim 7 \times 10^{-7} \, \mathrm{kg/m^3} \). \\
\hline
\( \rho_\text{\ae}^{\text{energy}} \) & Energy Density & \( \mathrm{J/m^3} \) & Energy stored in core-swirl regions. Estimated from maximum tension as \( \sim 3 \times 10^{35} \, \mathrm{J/m^3} \). \\
\hline
\( \rho_\text{\ae}^{\text{mass}} \) & Mass-Equivalent Density & \( \mathrm{kg/m^3} \) & Defined by \( \rho^{\text{energy}} / c^2 \); enters gravitational and inertial derivations. Approx.\ \( 3 \times 10^{18} \, \mathrm{kg/m^3} \). \\
\hline
\end{tabular}
\caption{Distinct æther densities used in VAM depending on physical context.}
\label{tab:ae_densities_foot}
\end{table}

\begin{table}[H]
    \centering
    \footnotesize
    \renewcommand{\arraystretch}{1.3}
    \begin{tabular}{|c|c|l|}
        \hline
        \textbf{Symbol} & \textbf{Expression} & \textbf{Interpretation} \\
        \hline
        \( \hbar_{\text{VAM}} \) & \( m_e C_e r_c \) & Angular impulse from core swirl (Planck analog) \\
        \hline
        \( c \) & \( \sqrt{\frac{2 F^{\text{max}}_{\text{\ae}} r_c}{m_e}} \) & Effective wave speed in æther; maximum signal velocity \\
        \hline
        \( \alpha \) & \( \frac{2 C_e}{c} \) & Fine-structure constant from swirl-to-light ratio \\
        \hline
        \( e^2 \) & \( 8\pi m_e C_e^2 r_c \) & Electric charge as swirl pressure across vortex boundary \\
        \hline
        \( \Gamma \) & \( 2\pi r_c C_e \) & Total vortex circulation (quantized in superfluids) \\
        \hline
        \( v \) & \( \sqrt{\frac{F^{\text{max}}_{\text{\ae}} r_c^3}{C_e^2}} \) & Higgs-like amplitude from ætheric elasticity \\
        \hline
    \end{tabular}
    \caption{Reconstruction of known constants from æther-based vortex parameters.}
    \label{tab:VAM_constants}
\end{table}

\noindent As an illustrative result, the rest mass of a vortex-knot particle is given by:
\[
M = \frac{\rho_\text{\ae}^{\text{fluid}} \Gamma^2}{L_k \pi r_c C_e^2}
\]
where \( L_k \) is the topological linking number of the knot. This expression arises from energetic analysis of closed vortex loops and is derived in Appendix~\ref{sec:derivation-of-the-kinetic-energy-of-a-circular-vortex-loop}.

\vspace{0.5em}

Thus, the Vortex Æther Model replaces dimensionally convenient but ontologically opaque constants with experimentally meaningful quantities derived from the geometry and energetics of fluid structures. The result is a unified physical interpretation of mass, charge, and coupling strengths—all emerging from coherent dynamics in a topologically structured æther.

\section*{Natural Unit Reformulation: \( C_e = 1, \, r_c = 1 \)}

To simplify the dimensional structure of VAM, we introduce a natural unit system by setting the two most intrinsic geometric quantities to unity:

\[
C_e = 1, \qquad r_c = 1
\]

This system treats the swirl velocity and vortex core radius as base units of speed and length, respectively. All other quantities are then rendered dimensionless or scaled relative to these units. The æther's characteristic energy, tension, and inertia become natural geometric outputs of the knotted structure.

\subsection*{Normalized Quantities}

\begin{table}[H]
    \centering
    \footnotesize
    \renewcommand{\arraystretch}{1.3}
    \begin{tabular}{|c|l|l|}
        \hline
        \textbf{Quantity} & \textbf{Natural Unit Expression} & \textbf{Interpretation} \\
        \hline
        \( \Gamma \) & \( 2\pi \) & Unit vortex circulation quantum \\
        \hline
        \( \alpha \) & \( 2 / c \) & Swirl-to-light ratio (dimensionless) \\
        \hline
        \( \hbar_{\text{VAM}} \) & \( m_e \) & Angular impulse equals rest mass (since \( C_e r_c = 1 \)) \\
        \hline
        \( e^2 \) & \( 8\pi m_e \) & Charge-energy coupling proportional to mass \\
        \hline
        \( F^{\text{max}}_{\text{\ae}} \) & \( \pi \rho_\text{\ae} \) & Max ætheric stress (now purely density-scaled) \\
        \hline
        \( v \) & \( \sqrt{F^{\text{max}}_{\text{\ae}}} \) & Vacuum amplitude as square root of ætheric tension \\
        \hline
        \( t_p \) & \( 1/c \) & Ætheric Planck time as inverse signal speed \\
        \hline
    \end{tabular}
    \caption{Natural unit forms of VAM-derived quantities when \( C_e = r_c = 1 \).}
    \label{tab:natural_units_vam}
\end{table}

\noindent In this normalized system, physical constants take on clear geometric interpretations:

- \textbf{Mass} is dimensionless and equals the angular impulse.
- \textbf{Charge} becomes a dimensionless swirl-energy flux.
- \textbf{Time} is measured in vortex rotations per \( 1/c \).

The normalized mass expression becomes:

\[
M = \frac{\rho_\text{\ae} \, \Gamma^2}{L_k \pi}
\quad \Rightarrow \quad
M = \frac{4\pi \rho_\text{\ae}}{L_k}
\quad \text{(since } \Gamma = 2\pi)
\]

This compact form shows that mass scales directly with æther density and inversely with knot complexity (via the linking number \( L_k \)).

\subsection*{Advantages of the Natural Unit Form}

- \textbf{Simplifies analytical derivations} in Lagrangians and conservation laws.
- \textbf{Makes topological scaling explicit}, e.g., how mass changes with \( L_k \) or how helicity enters time dilation.
- \textbf{Eliminates Planck-scale opacity}: everything derives from vortex properties without black-box constants.

This formulation may be especially useful in symbolic computation, numerical simulations, or extending VAM to cosmological scales, where \( C_e \) and \( r_c \) can serve as natural units of large-scale structure or vacuum flow.
\subsection{Running Coupling Constants from Æther Density}

In conventional quantum field theory, coupling constants such as the fine-structure constant \( \alpha \) are not truly constant: they evolve with energy scale due to vacuum polarization effects. This scale dependence is governed by the renormalization group (RG) flow, typically expressed as:
\begin{equation}
\alpha(k^2) = \frac{\alpha_0}{1 - \Pi(k^2)},
\end{equation}
where \( \Pi(k^2) \) encodes the contribution of virtual particle loops to vacuum screening, and \( \alpha_0 \) is the asymptotic low-energy value.

\vspace{0.5em}

In the Vortex Æther Model (VAM), this phenomenon is reinterpreted from first principles. Rather than arising from quantum fluctuations in field modes, the running of coupling constants is attributed to \emph{variations in the local structure of the æther medium} — specifically its density, compressibility, and vorticity distribution.

We propose a spatially varying fine-structure constant defined by fluid-mechanical response:
\begin{equation}
\alpha(\vec{x}) = \frac{e^2}{4\pi \varepsilon_0(\vec{x}) \hbar \, c_{\text{eff}}(\vec{x})} = \alpha_0 \cdot f\left( \rho_\text{\ae}(\vec{x}), \, |\vec{\omega}(\vec{x})| \right),
\end{equation}
where:
\begin{itemize}
    \item \( \rho_\text{\ae}(\vec{x}) \): local æther density (fluid or energy),
    \item \( \vec{\omega} = \nabla \times \vec{v} \): local vorticity,
    \item \( c_{\text{eff}}(\vec{x}) \): local effective signal speed in the æther.
\end{itemize}

This formulation introduces a deterministic analog to renormalization: interaction strengths depend not on loop corrections but on the mechanical properties of the medium. The variation arises from gradients in flow structure, replacing renormalization group β-functions with hydrodynamic strain and density profiles.

\vspace{0.5em}

The effective light-speed and permittivity are governed by:
\begin{equation}
    c_{\text{eff}}(\vec{x}) \propto \sqrt{ \frac{B(\vec{x})}{\rho_\text{\ae}(\vec{x})} }, \qquad
    \varepsilon_0(\vec{x}) \sim \frac{1}{\rho_\text{\ae}(\vec{x}) C_e^2},
\end{equation}
where:
- \( B(\vec{x}) \) is the bulk modulus of the æther — its resistance to compression,
- \( \rho_\text{\ae}(\vec{x}) \) determines both inertial and field strength response,
- \( C_e \) is the swirl velocity scale (constant in the far field).

\vspace{0.5em}

\paragraph{Physical Interpretation.}
\begin{itemize}
    \item In regions of high vorticity or density — such as near vortex knots, gravitating bodies, or boundaries of topological defects — the æther stiffens, leading to increased bulk modulus and altered signal propagation speed.
    \item This modifies both \( \varepsilon_0 \) and \( c_{\text{eff}} \), resulting in a local shift in the effective fine-structure constant \( \alpha(\vec{x}) \).
    \item Hence, what is traditionally attributed to “vacuum polarization” becomes a function of vortex-induced curvature in the flow field.
\end{itemize}

\vspace{0.5em}

\paragraph{Experimental Implications.}
This model predicts that \emph{fundamental constants may vary measurably across spacetime}, especially in regions of high swirl, strain, or gravitational density. Potential testbeds include:
\begin{itemize}
    \item High-precision atomic clocks near rotating masses or in fluidic gyroscopes;
    \item Spectral analysis of quasars and interstellar media in high-redshift regions~\cite{shapiro2004variation, uzan2011varying};
    \item Analog experiments in BECs or superfluid helium with spatially varying density and vorticity;
    \item Laboratory systems with tunable swirl fields (e.g., rotating plasmas or optical vortices).
\end{itemize}

\vspace{0.5em}

This reinterpretation places VAM in conceptual alignment with emergent gravity approaches such as Verlinde’s entropic framework~\cite{verlinde2016emergent}, while providing a concrete mechanical basis for renormalization — rooted not in formal regularization, but in measurable ætheric response.



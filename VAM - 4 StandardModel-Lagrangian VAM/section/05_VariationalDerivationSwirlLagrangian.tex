\section{Variational Derivation of the Swirl Lagrangian}

To rigorously support the Vortex Æther Model (VAM), we derive the swirl Lagrangian using a variational principle analogous to classical field theory. This establishes a formal path from æther vortex dynamics to field-theoretic particle analogs.

\subsection{Field Structure and Helmholtz Decomposition}

The æther velocity field $\mathbf{v}(\mathbf{x}, t)$ is decomposed via Helmholtz's theorem:

\begin{equation}
\mathbf{v} = \nabla \theta + \mathbf{A},
\end{equation}

where $\theta$ is a scalar potential (irrotational component), and $\mathbf{A}$ is the divergence-free vector potential representing swirl, with $\nabla \cdot \mathbf{A} = 0$. The vorticity field is:

\begin{equation}
\boldsymbol{\omega} = \nabla \times \mathbf{v} = \nabla \times \mathbf{A}.
\end{equation}

\subsection{Action Functional and Swirl Gauge Field}

We define the action $S$ as:

\begin{equation}
S[\theta, \mathbf{A}] = \int d^4x \, \mathcal{L}_{\text{VAM}},
\end{equation}

where the Lagrangian density is:

\begin{equation}
\mathcal{L}_{\text{VAM}} = \frac{1}{2} \rho (\nabla \theta + \mathbf{A})^2 - \lambda (|\phi|^2 - F^{\text{max}}_{\text{\ae}}^2)^2 - \frac{1}{4} S_{\mu\nu} S^{\mu\nu} + \left( \frac{\rho_{\text{æ}} r_c^2}{C_e} \right) (\mathbf{v} \cdot \boldsymbol{\omega}).
\end{equation}

In this form:
\begin{itemize}
    \item The second term is a self-generated core potential representing stress from radial æther compression, replacing $\rho \Phi$.
    \item $S_{\mu\nu} = \partial_\mu W_\nu - \partial_\nu W_\mu$ is the swirl field strength tensor, with $W_\mu = (\phi, \mathbf{A})$.
    \item The final term is a helicity-density-based coupling, with $\rho_{\text{æ}}$ the æther density, $r_c$ the vortex core radius, and $C_e$ the swirl velocity (effective light speed).
\end{itemize}

\subsection{Euler-Lagrange Equations and Continuity}

Varying the action with respect to $\theta$ recovers the continuity equation:

\begin{equation}
\partial_t \rho + \nabla \cdot (\rho \mathbf{v}) = 0.
\end{equation}

Variation with respect to $\mathbf{A}$ gives a generalized swirl equation of motion:

\begin{equation}
\rho \mathbf{v} - \nabla \cdot \left( \frac{\partial \mathcal{L}_{\text{swirl}}}{\partial (\nabla \mathbf{A})} \right) + \left( \frac{\rho_{\text{æ}} r_c^2}{C_e} \right) \boldsymbol{\omega} = 0.
\end{equation}

This coupling of vorticity to mass-like topological terms gives rise to effective inertial behavior.

\subsection{Mass from Topology and Helicity}

The helicity density $h = \mathbf{v} \cdot \boldsymbol{\omega}$ is interpreted as a local "spin clock rate" of vortex knots. Integrated over a topologically linked region, it yields:

\begin{equation}
m_{\text{eff}} \sim \left( \frac{\rho_{\text{æ}} r_c^2}{C_e} \right) \int_V \mathbf{v} \cdot \boldsymbol{\omega} \, d^3x.
\end{equation}

This expression ties particle mass directly to topological properties such as twist, writhe, and linking number of the vortex core.

\subsection{Outlook: Quantization Path}

The swirl gauge field admits canonical quantization via:

\begin{align}
\Pi^\mu &= \frac{\partial \mathcal{L}}{\partial (\partial_0 W_\mu)}, \\
[W_\mu(\mathbf{x}), \Pi^\nu(\mathbf{x}')] &= i \delta^\nu_\mu \delta^3(\mathbf{x} - \mathbf{x}'),
\end{align}

and path integral representation:

\begin{equation}
Z = \int \mathcal{D}[W_\mu] \exp\left(i \int d^4x \, \mathcal{L}_{\text{VAM}}\right).
\end{equation}

This establishes a formal pathway to embedding the Vortex Æther Model in a quantum field-theoretic setting, while preserving its topological and hydrodynamic origins.

\section{Canonical Commutators and Swirl Quantization}

To formulate a consistent quantum field theory from the Vortex Æther Model (VAM), it is essential to specify canonical commutation relations between fundamental fluid observables. In standard quantum field theory, canonical quantization imposes:
\begin{equation}
[\phi(x), \pi(y)] = i \delta(x - y),
\end{equation}
where $\phi$ is a field and $\pi$ its conjugate momentum.

We propose that a similar structure exists in the VAM, where the swirl potential $\theta(\vec{x})$ and the æther density $\rho_{\ae}(\vec{x})$ form a canonical pair:
\begin{equation}
[\theta(\vec{x}), \rho_{\ae}(\vec{y})] = i \delta^3(\vec{x} - \vec{y}),
\end{equation}
implying an uncertainty relation between vortex phase and æther mass density, akin to the number-phase relation in Bose fluids.

Alternatively, one may define canonical brackets between the velocity and vorticity fields:
\begin{equation}
[v_i(\vec{x}), \omega_j(\vec{y})] \sim i \epsilon_{ijk} \partial_k \delta^3(\vec{x} - \vec{y}),
\end{equation}
consistent with the Lie algebra structure of vector fields under the Helmholtz decomposition.

This structure leads to a Hamiltonian formalism for VAM fluid dynamics:
\begin{equation}
\mathcal{H}[\theta, \rho_{\ae}] = \int d^3x \left[ \frac{1}{2} \rho_{\ae}(\vec{x})\, |\nabla \theta(\vec{x})|^2 + V(\rho_{\ae}) \right],
\end{equation}
where $V(\rho_{\ae})$ represents the potential energy density of the æther medium, potentially including self-interaction or compressibility terms.

The formal identification of conjugate variables and commutators in the VAM allows quantization of vortex excitations through standard Fock space methods, in close analogy with the quantized phonon and roton spectra of superfluid helium systems \cite{fetter1971nonuniform,stone2000superfluidity,verlinde2021qft}.



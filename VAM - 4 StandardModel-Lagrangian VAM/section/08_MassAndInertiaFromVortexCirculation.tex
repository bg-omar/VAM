\section{Mass and Inertia from Vortex Circulation}

In the Vortex Æther Model (VAM), mass is not a fundamental attribute but emerges from fluid motion—specifically the swirl dynamics and circulation of knotted vortex structures. This section derives the mass-energy relation, effective inertial mass, and corresponding Lagrangian term based purely on ætheric fluid mechanics.

\subsection{Emergent Relativistic Limit from Æther Dynamics}

In the Vortex Æther Model (VAM), the relativistic energy relation \( E = mc^2 \) arises not as an axiom, but as a natural consequence of the æther’s fluid dynamics. The key is the propagation speed of perturbations—both scalar and vectorial—within the medium.

\paragraph{Speed of Sound Analogy.}
In compressible fluids, the maximum propagation speed of pressure or scalar waves is given by:
\[
    c_s = \sqrt{\frac{\partial p}{\partial \rho}}.
\]
In the æther, this corresponds to the speed of longitudinal strain propagation. For small perturbations near the equilibrium density \( \rho_0 \), we can write:
\[
    c^2 = \left.\frac{d^2 V}{d\rho^2} \right|_{\rho_0} \cdot \frac{1}{\rho_0},
\]
where \( V(\rho) \) is the æther potential. This defines \( c \) as the \textbf{maximum signal speed}, similar to light speed in relativistic spacetime.

\paragraph{Limiting Velocity for Vortex Motion.}
Swirl propagation is limited by the maximum tangential velocity \( C_e \), tied to vortex stability:
\[
    \Gamma = 2\pi r_c C_e.
\]
However, long-range signal transmission (e.g., interactions between vortices) is constrained by the bulk medium. Thus, \( c \) acts as the \textbf{emergent limiting velocity} for field propagation and topological interactions.

\paragraph{Lorentz Invariance as an Emergent Symmetry.}
As shown in analog gravity systems \cite{barcelo2011}, effective Lorentz symmetry can emerge in low-energy excitations of superfluid systems. Similarly, the VAM supports Lorentz invariance as an emergent property of linearized vortex perturbations, especially in the deep infrared regime.

\paragraph{Matching with Observed Constants.}
To align with observed particle properties, the VAM allows:
\[
    \hbar_{\text{VAM}} = 2 m C_e a_0, \qquad E = m c^2, \qquad \text{and} \qquad \Gamma = \frac{h}{m}.
\]
These expressions relate observable constants to ætheric dynamics. Importantly, constants such as \( \hbar \), \( c \), and \( e \) are \textbf{inserted as axioms}, but \textbf{emerge from circulation, wave speed, and topological parameters} in the æther framework.


\subsection{Kinetic Energy of a Vortex Knot}

The kinetic energy of a localized vortex knot in an incompressible æther is given by:
\begin{equation}
    \mathcal{L}_\text{kin} = \frac{1}{2} \rho_\text{\ae} |\vec{v}|^2,
\end{equation}
where $\vec{v}$ is the swirl velocity and $\rho_\text{\ae}$ the local æther density. For a stable vortex knot, the core swirl velocity saturates at a characteristic value \( C_e \), yielding:
\[
    \mathcal{L}_\text{kin} \approx \frac{1}{2} \rho_\text{\ae} C_e^2.
\]
Assuming a knot core with radius \( r_c \), the total kinetic energy becomes:
\[
    E_\text{kin} \approx \frac{1}{2} \rho_\text{\ae} C_e^2 \cdot \frac{4}{3} \pi r_c^3.
\]
This naturally defines an effective inertial mass:

\[
    m_\text{eff} = \rho_\text{\ae} \cdot \frac{4}{3} \pi r_c^3,
\]
associated with the fluid's resistance to swirl acceleration. The local kinetic energy is:
\[
    E_{\text{kin}} = \frac{1}{2} m_\text{eff} C_e^2.
\]
Note that this expression describes mechanical energy from internal circulation. In the VAM framework, the total rest energy of the vortex object later aligns with \( E = m c^2 \), where \( c \) is the emergent relativistic limit derived from æther dynamics.

\subsection{Circulation and Geometric Mass Emergence}

In vortex mechanics, circulation is conserved and fundamental. It is defined as:
\begin{equation}
    \Gamma = \oint_{\partial S} \vec{v} \cdot d\vec{\ell} = 2\pi r_c C_e.
\end{equation}
This implies that changes in core radius \( r_c \) require reciprocal changes in swirl velocity \( C_e \), enforcing inertial resistance.

We compute the kinetic energy:
\begin{align}
    E &= \frac{1}{2} \rho_\text{\ae} \left( \frac{\Gamma}{2\pi r_c} \right)^2 \cdot \frac{4}{3} \pi r_c^3
    = \frac{\rho_\text{\ae} \Gamma^2}{6\pi r_c}.
\end{align}
Comparing with \( E = m c^2 \), we extract the effective inertial mass:
\begin{equation}
    m_\text{eff} = \frac{\rho_\text{\ae} \Gamma^2}{6\pi r_c c^2}.
\end{equation}
This shows that mass is an emergent quantity—arising from vortex geometry and æther circulation, not inserted as a primitive parameter.

Although \( C_e \) governs the local swirl velocity within the vortex core, the inertial energy scale aligns with the broader æther dynamics, where \( c \) defines the maximum speed of long-range signal propagation (e.g., strain waves).

Thus, the relation \( E = m c^2 \) in VAM arises not from postulated spacetime symmetry, but from bulk æther behavior near equilibrium density. It provides a natural bridge between microscopic vortex circulation and macroscopic relativistic dynamics.


\subsection{Lagrangian Mass Term in VAM}

Given the above, the corresponding mass term for a fermion field $\psi_f$ becomes:
\begin{equation}
    \mathcal{L}_\text{mass} = \hbar_{\text{VAM}} \cdot \bar{\psi}_f \psi_f,
\end{equation}
with
\begin{equation}
    \boxed{
        \hbar_{\text{VAM}} = 2 m_f C_e a_0
    }
\end{equation}
Here, \( a_0 \) is the Bohr ground-state radius, and the factor of 2 accounts for the angular momentum structure of vortex-bound states, possibly representing a double-cover topology or dual-swirl configuration.

This identification ensures consistency with:
\[
    h = 4\pi m_e C_e a_0 \quad \Rightarrow \quad \hbar = 2 m_e C_e a_0,
\]
recovering the known Planck scale from æther dynamics.

This mass term replaces the abstract Yukawa interaction with a fluid-mechanical origin, grounded in vortex inertia and quantized swirl structure.

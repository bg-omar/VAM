\section{Mapping \texorpdfstring{$SU(3)_C \times SU(2)_L \times U(1)_Y$}{SU(3) x SU(2) x U(1)} to VAM Swirl Groups}

The Standard Model Lagrangian is governed by the gauge group:
\[
    SU(3)_C \times SU(2)_L \times U(1)_Y
\]
which encodes the strong interaction (QCD), the weak interaction, and electromagnetism via their corresponding gauge fields. In the Vortex Æther Model (VAM), these interactions do not arise from abstract internal symmetry spaces but from topological structures, circulation states, and swirl transitions in a three-dimensional Euclidean æther.

\subsection{$U(1)_Y$: Swirl Orientation as Hypercharge}

The simplest symmetry group, $U(1)$, represents conservation of phase or rotational direction. In VAM, this acquires a direct physical interpretation:
\begin{itemize}
    \item \textbf{Physical model:} a linear swirl in the æther (circular, but untwisted) encodes a uniform angular direction.
    \item \textbf{Charge assignment:} the hypercharge $Y$ is interpreted as the chirality (left- or right-handed swirl) of an axially symmetric flow pattern.
    \item \textbf{Electromagnetism:} emerges from global swirl states without knotting, representing long-range coherence in swirl orientation.
\end{itemize}

\subsection{$SU(2)_L$: Chirality as Two-State Swirl Topology}

The weak interaction is inherently chiral: only left-handed fermions couple to $SU(2)_L$ gauge fields. In VAM:
\begin{itemize}
    \item \textbf{Swirl interpretation:} left- and right-handed vortices are dynamically and structurally distinct—they represent swirl flows under compression with opposite twist orientation.
    \item \textbf{Two-state logic:} the $SU(2)$ doublet corresponds to a two-dimensional swirl state space (e.g., up- and down-swirl configurations).
    \item \textbf{Gauge transitions:} $SU(2)$ gauge bosons mediate transitions between these swirl states through reconnections or bifurcations in vortex knots.
\end{itemize}

\subsection{$SU(3)_C$: Trichromatic Swirl as Helicity Configuration}

In the Standard Model, $SU(3)_C$ describes the color force via gluon-mediated transitions between color states. In VAM:
\begin{itemize}
    \item \textbf{Topological basis:} three topologically stable swirl configurations (e.g., aligned along orthogonal helicity axes) represent the three color charges: red, green, and blue.
    \item \textbf{Color dynamics:} gluon exchange corresponds to twist-transfer, vortex reconnection, or deformation within multi-knot structures.
    \item \textbf{Confinement:} isolated color swirls are energetically unstable in free æther and only persist within composite knotted bundles (e.g., baryons).
\end{itemize}

\subsection{Mathematical Group Structure within VAM}

Though VAM is fundamentally geometric and fluid-dynamical, the essential Lie group structures of the Standard Model are preserved in the form of physically conserved swirl states:
\begin{itemize}
    \item Swirl orientation $\rightarrow$ $U(1)$ phase symmetry,
    \item Axial twist transitions $\rightarrow$ $SU(2)$ chiral transformations,
    \item Helicity axis exchange $\rightarrow$ $SU(3)$ color group operations.
\end{itemize}

\subsection*{Topological Summary of Gauge Interpretation}

The abstract Lie symmetries of the Standard Model find concrete realizations in VAM as swirl, helicity, and knot configurations embedded in the æther. This recasting preserves all observed gauge interactions while rooting them in fluid-mechanical principles—without invoking extra dimensions or unobservable symmetry spaces.

\section{Swirl Operator Algebra and SU(2) Closure}

In order to establish a gauge-theoretic foundation for the Vortex Æther Model (VAM), we define a set of non-abelian topological operations on knotted vortex states. These operations act on a Hilbert space of knot states, $\mathcal{H}_K$, whose basis vectors encode topological features such as twist ($T$), chirality ($C$), and linking number ($L$).

\subsection*{Operator Definitions}

We introduce three operators:
\begin{align}
\mathcal{S}_1 &: \text{Chirality flip}, \quad \mathcal{S}_1 |K, C\rangle = |K, -C\rangle \\
\mathcal{S}_2 &: \text{Twist addition}, \quad \mathcal{S}_2 |K, T\rangle = |K, T+1\rangle \\
\mathcal{S}_3 &: \text{Reconnection mutation}, \quad \mathcal{S}_3 |K\rangle = |K'\rangle
\end{align}

\subsection*{SU(2) Algebra Closure}

We then test the closure of these operators under commutation. Defining generators $T^i = \frac{1}{2} \mathcal{S}_i$, we recover the SU(2) Lie algebra structure:
\begin{align}
[T^i, T^j] = i \epsilon^{ijk} T^k
\end{align}

We verified this numerically using matrix representations:
\begin{align}
\mathcal{S}_1 &= \begin{pmatrix} 0 & 1 \\ 1 & 0 \end{pmatrix}, \quad
\mathcal{S}_2 = \begin{pmatrix} 0 & -i \\ i & 0 \end{pmatrix}, \quad
\mathcal{S}_3 = \begin{pmatrix} 1 & 0 \\ 0 & -1 \end{pmatrix}
\end{align}
with:
\begin{align}
[\mathcal{S}_1, \mathcal{S}_2] &= 2i \mathcal{S}_3, \\
[\mathcal{S}_2, \mathcal{S}_3] &= 2i \mathcal{S}_1, \\
[\mathcal{S}_3, \mathcal{S}_1] &= 2i \mathcal{S}_2
\end{align}

A generalized symbolic representation in $\mathbb{R}^3$ with scale constants $a, b, c$ preserves this structure:
\begin{align}
[\mathcal{S}_1, \mathcal{S}_2] &= 2iab \, \mathcal{S}_3 \\
[\mathcal{S}_2, \mathcal{S}_3] &= 2ibc \, \mathcal{S}_1 \\
[\mathcal{S}_3, \mathcal{S}_1] &= 2ac \, \mathcal{S}_2
\end{align}

\subsubsection*{Example: Chirality Flip on Knot States}

Let a vortex knot state be denoted as:
\[
|K\rangle = |T, C\rangle
\]
where \( T \in \mathbb{Z} \) is the twist number, and \( C = \pm 1 \) denotes chirality (right- or left-handedness).

The action of the chirality-flip operator \( \mathcal{S}_1 \) is given by:
\[
\mathcal{S}_1 |T, +1\rangle = |T, -1\rangle, \quad
\mathcal{S}_1 |T, -1\rangle = |T, +1\rangle
\]

Thus, \( \mathcal{S}_1 \) acts as a discrete parity operator on knotted vortex tubes, analogous to the weak isospin generator \( T^1 \) in SU(2). The eigenstates of chirality form a two-level system, similar to spinors in the Standard Model.

\subsubsection*{Experimental Perspective}

These topological swirl operators may have observable counterparts in superfluid systems. In particular, discrete transitions between vortex chirality, twist, and reconnection have been reported in Bose-Einstein condensates (BECs) and analog gravity labs \cite{kleckner2013creation,ray2015observation}.

\subsection{Swirl Field Resonance Spectrum and Bound Knot States}

In the Vortex Æther Model (VAM), composite particles (e.g., baryons, mesons) are modeled as knotted vortex configurations linked via swirl field tubes. These connecting swirl regions can support quantized standing waves, giving rise to a discrete \textit{resonance spectrum} analogous to atomic or molecular energy levels. This spectrum plays a key role in determining the stability, oscillation behavior, and decay channels of vortex-bound states.

\subsubsection*{Wave Equation for Swirl Modes}

We consider the simplest model of the inter-knot swirl field as a one-dimensional scalar field \( \phi(x,t) \) connecting two fixed knotted cores separated by distance \( L \). The field obeys the linear wave equation:
\begin{equation}
\frac{\partial^2 \phi}{\partial t^2} - c_s^2 \frac{\partial^2 \phi}{\partial x^2} = 0,
\label{eq:swirl_wave}
\end{equation}
where \( c_s \) is the swirl mode propagation speed in the æther, determined by local circulation density.

\subsubsection*{Boundary Conditions and Standing Waves}

We impose Dirichlet boundary conditions at the knot positions:
\begin{equation}
\phi(0,t) = \phi(L,t) = 0,
\end{equation}
modeling the knots as fixed topological nodes. The general solution becomes a standing wave:
\begin{equation}
\phi_n(x,t) = A_n \sin\left( \frac{n\pi x}{L} \right) e^{i \omega_n t}, \quad n \in \mathbb{Z}^+.
\end{equation}

This leads to the quantized resonance frequencies:
\begin{equation}
\omega_n = \frac{n\pi c_s}{L}, \quad n = 1,2,3,...
\label{eq:resonance}
\end{equation}

\subsubsection*{Interpretation in the Vortex Æther Model}

Each \( \omega_n \) corresponds to a distinct swirl excitation mode that mediates the interaction between the knotted cores. This resonance condition underlies several key physical effects:
\begin{itemize}
    \item \textbf{Bound states:} Knots form stable molecular states when coupled via resonant swirl modes.
    \item \textbf{Quantized energy:} These resonances represent discrete energy levels, potentially explaining mass splittings and flavor mixing in composite states.
    \item \textbf{Decay and transitions:} De-excitation occurs via swirl-mode emission (analog of gluon or photon), obeying conservation of circulation.
    \item \textbf{Confinement:} Disallowed \( \omega_n \) modes lead to energetically unstable configurations — offering a mechanism for topological confinement.
\end{itemize}

\subsubsection*{Mapping to Particle-like Excitations}

The spectrum \( \omega_n \) serves as a classification scheme for composite particles in the VAM. Below is a tentative mapping:

\begin{table}[H]
    \centering
    \footnotesize
    \renewcommand{\arraystretch}{1.4}
    \begin{tabular}{|c|c|c|c|}
        \hline
        \textbf{Mode} \( n \) & \textbf{Swirl Frequency} \( \omega_n \) & \textbf{Knot Class} & \textbf{Interpretation} \\
        \hline
        1 & \( \frac{\pi c_s}{L} \) & Hopfion doublet & Ground-state vortex molecule \\
        2 & \( \frac{2\pi c_s}{L} \) & Trefoil triplet & Excited baryonic analog \\
        3 & \( \frac{3\pi c_s}{L} \) & Triskelion braid & Higher twist fermionic bound state \\
        \hline
    \end{tabular}
    \caption{Sample resonance modes and corresponding vortex-knot states in VAM.}
\end{table}

This formulation echoes the quantized bound-state spectra seen in black hole binaries coupled to light fields \cite{baumann2023black}, suggesting a broader universality in emergent, field-mediated compositeness.



\section{Extension to SU(3): Triskelion and Braid Operator Algebra}

To capture the full non-abelian gauge structure of the Standard Model within the Vortex Æther Model (VAM), we extend the SU(2) swirl operator algebra to SU(3) using braid-based topological operations on vortex bundles.

\subsection*{Triskelion States and Braid Operators}

These vortex bundles are visualized as three interlinked flux tubes, each representing a ‘color', whose topology determines the chromodynamic state.
Let each \("\)color\("\) in quantum chromodynamics correspond to one vortex strand in a triple-knot configuration—denoted a \textit{triskelion} state:
\[
|K\rangle = |R, G, B\rangle
\]
We define braid-like swirl operators \( \mathcal{B}_1, \mathcal{B}_2, \mathcal{B}_3 \), each acting locally on a pair of vortex colors. Their action mimics gluon exchange via reconnection and twist of the bundle.

\subsection*{Braid Group Algebra}

The operators obey modified braid group relations:
\begin{align}
\mathcal{B}_i \mathcal{B}_{i+1} \mathcal{B}_i &= \mathcal{B}_{i+1} \mathcal{B}_i \mathcal{B}_{i+1}, \\
\mathcal{B}_i \mathcal{B}_j &= \mathcal{B}_j \mathcal{B}_i \quad \text{for } |i-j| > 1
\end{align}

Linear combinations of these braids generate an algebra:
\begin{align}
[T^a, T^b] = i f^{abc} T^c
\end{align}
where \( T^a \sim \mathcal{B}_a \) are the topological gluon modes, and \( f^{abc} \) are the SU(3) structure constants \cite{witten1989quantum}.

\subsection*{Topological Interpretation of Color Charge}

\begin{itemize}
    \item \textbf{Color charge} is the topological identity of each vortex in the triskelion.
    \item \textbf{Gluons} correspond to triskelion-preserving reconnection modes \( \mathcal{B}_a \).
    \item \textbf{Confinement} emerges from the topological stability of linked triskelion bundles — a single vortex cannot be detached without violating circulation conservation \cite{kauffman1991knots,faddeev1997knots}.
\end{itemize}

This construction provides a fluid-dynamical representation of SU(3), with color interactions arising from internal braid dynamics. The VAM thus naturally embeds the full SU(3)$\times$SU(2)$\times$U(1) structure within a topological framework.

\subsection{Swirl Operators and Topological Transitions in the Vortex Æther Model}

In the Vortex Æther Model (VAM), particle properties emerge from the topological structure and dynamics of knotted vortex tubes. To capture internal transformations such as chirality changes, angular momentum variations, and topology shifts, we introduce three discrete operators acting on knot states:
\begin{itemize}
    \item $\mathcal{S}_1$: Chirality flip (left $\leftrightarrow$ right)
    \item $\mathcal{S}_2$: Twist addition (increasing internal winding)
    \item $\mathcal{S}_3$: Reconnection mutation (topological class change)
\end{itemize}

These operators act on a topological state space $\mathcal{H}_K$, where each knot state is defined by quantities such as chirality $C$, twist $T$, linking number $Lk$, and topological class $Q$. Their algebra is non-abelian:
\[
[\mathcal{S}_i, \mathcal{S}_j] \neq 0
\]
which permits a correspondence with non-abelian gauge groups like SU(2). The physical interpretation of these operators as analogs to quantum field transformations is summarized below.

\begin{table}[H]
    \centering
    \footnotesize
    \renewcommand{\arraystretch}{1.4}
    \begin{tabular}{|l|l|l|l|l|}
        \hline
        \textbf{Swirl Operator} & \textbf{Swirl Action} & \textbf{Affected Invariant} & \textbf{Eigenvalue Change} & \textbf{QFT Analog} \\
        \hline
        $\mathcal{S}_1$ & Chirality Flip & Chirality $C$, Helicity $H$ & $C \rightarrow -C$, $H \rightarrow -H$ & Chiral projection: $\psi_L \leftrightarrow \psi_R$ \\
        \hline
        $\mathcal{S}_2$ & Twist Addition & Twist $T$, Writhe $Wr$, Spin $s$ & $T \rightarrow T + 1$, $s \rightarrow s + \hbar$ & Spin raising operator \\
        \hline
        $\mathcal{S}_3$ & Reconnection Mutation & Knot Type, $Lk$, Topological Class $Q$ & $|K\rangle \rightarrow |K'\rangle$, $Lk \rightarrow Lk \pm 1$ & Flavor/decay/topology shift \\
        \hline
    \end{tabular}
    \caption{Algebraic and physical interpretation of swirl operators acting on vortex knot states.}
\end{table}

These transformations serve as the basis for constructing topological analogs of SU(2) and SU(3) gauge field algebras, with $\mathcal{S}_1, \mathcal{S}_2, \mathcal{S}_3$ forming a closed non-commutative set analogous to the SU(2) Lie algebra.

\subsection{Toward SU(3): Braid Operators and Topological Color Charge}

To extend the topological formalism of VAM to the gauge algebra of the strong interaction, we introduce braid operators $\mathcal{B}_a$ acting on bundles of vortex tubes. These operators correspond to the eight gluon generators of SU(3)$_C$, which mediate interactions between color charges in quantum chromodynamics.

In the VAM context, we model composite particles (e.g. hadrons) as tightly bound vortex triplets — analogous to Y-shaped triskelion knots or braided filament networks. The braid operators $\mathcal{B}_1, \mathcal{B}_2, ..., \mathcal{B}_8$ act on these networks to permute, twist, or reconnect their strands in a non-abelian fashion.

The braid operators $\mathcal{B}_a$ satisfy the Artin braid group relations:
\[
\begin{aligned}
\mathcal{B}_i \mathcal{B}_{i+1} \mathcal{B}_i &= \mathcal{B}_{i+1} \mathcal{B}_i \mathcal{B}_{i+1}, \\
\mathcal{B}_i \mathcal{B}_j &= \mathcal{B}_j \mathcal{B}_i \quad \text{for } |i - j| > 1,
\end{aligned}
\]
and we postulate that their algebra closes under SU(3) commutation relations:
\[
[\mathcal{B}_a, \mathcal{B}_b] = i f^{abc} \mathcal{B}_c,
\]
where $f^{abc}$ are the SU(3) structure constants.

\begin{table}[H]
    \centering
    \footnotesize
    \renewcommand{\arraystretch}{1.4}
    \begin{tabular}{|l|l|l|l|}
        \hline
        \textbf{Braid Operator} & \textbf{Topological Action} & \textbf{QCD Analog} & \textbf{Physical Interpretation} \\
        \hline
        $\mathcal{B}_1$ & Swap two adjacent strands in a triplet bundle & Gluon exchange (red $\leftrightarrow$ green) & Induces color rotation in vortex filaments \\
        \hline
        $\mathcal{B}_2$ & Twist a filament across two others & 3-body gluon vertex & Encodes phase shifts or internal energy exchanges \\
        \hline
        $\mathcal{B}_3$ through $\mathcal{B}_8$ (composite modes)& Composite reconnections and multi-twist interactions & Remaining SU(3) generators & Mix topological braid classes; mediate color confinement \\
        \hline
    \end{tabular}
    \caption{Topological braid operators $\mathcal{B}_a$ as analogs to SU(3)$_C$ gluon generators acting on vortex bundles. These include braid generators that induce transformations across all three filament channels, consistent with the 8-dimensional adjoint rep of SU(3).}
\end{table}

The color charge of a vortex triplet is defined by its braid class (e.g., symmetric, asymmetric, twisted), and confinement emerges from the non-trivial topological energy required to separate such bundles. This framework aligns with observations from knot theory, braid group algebra, and the structure of hadrons in QCD.

\subsection{Gravitational Molecules and Swirl-Bound Topological States}

Recent theoretical work in relativistic gravity has introduced the idea of \textit{gravitational molecules}—quasi-stable bound states of black hole binaries mediated by scalar or vector fields \cite{baumann2023black}. These structures arise from resonant couplings between massive cores and bound field modes, forming effective multi-body interactions even in the absence of direct contact.

We propose a topologically fluid analog within the Vortex Æther Model (VAM): namely, that \textit{vortex knots}—topological excitations of the æther—can form metastable bound states via the exchange of swirl field modes. These ``vortex molecules'' represent emergent structures with quantized energy levels and long-lived resonances.

\subsubsection*{Analogy with Gravitoelectromagnetism}

In the gravitoelectromagnetic (GEM) framework, weak-field gravity resembles electromagnetism through a vector potential $A^\mu = (\phi, \vec{A})$, producing gravitoelectric and gravitomagnetic fields \cite{mashhoon2001gravito}. VAM models this geometrically:
\begin{itemize}
    \item The \textbf{swirl vector potential} corresponds to $\vec{A}$—representing the directionality of vortex flows,
    \item The \textbf{helical energy density} plays the role of $\phi$—modulating local flow inertia.
\end{itemize}

Swirl excitations obey a wave equation analogous to Maxwell's equations in curved space:
\begin{equation}
\partial_\mu \partial^\mu \vec{v}_\text{swirl} = J^\mu_{\text{topo}},
\end{equation}
where $J^\mu_{\text{topo}}$ is the topological current associated with reconnections or circulation defects.

\section*{Gauge Symmetry from Vortex Phase Redundancy}

In quantum field theory, gauge invariance is a cornerstone of modern particle physics. The $U(1)$ gauge symmetry underlying electromagnetism allows for a local phase transformation:
\begin{equation}
\psi(x) \rightarrow e^{i\alpha(x)} \psi(x), \quad A_\mu(x) \rightarrow A_\mu(x) + \partial_\mu \alpha(x),
\end{equation}
which leaves the Lagrangian invariant.

In the Vortex Æther Model (VAM), we propose that a similar symmetry emerges from the local phase freedom of the swirl potential $\theta(\vec{x})$, which defines the flow field:
\begin{equation}
\vec{v} = \nabla \theta(\vec{x}).
\end{equation}

This formulation is invariant under the local transformation:
\begin{equation}
\theta(\vec{x}) \rightarrow \theta(\vec{x}) + \alpha(\vec{x}) \quad \Rightarrow \quad \vec{v} \rightarrow \vec{v} + \nabla \alpha(\vec{x}),
\end{equation}
which is structurally identical to a $U(1)$ gauge transformation. To maintain invariance under this transformation, we define a swirl gauge field $\vec{A}_v$:
\begin{equation}
\vec{A}_v \rightarrow \vec{A}_v + \nabla \alpha(\vec{x}).
\end{equation}

In this picture, the swirl velocity $\vec{v}$ is no longer a physical observable by itself, but only gauge-invariant quantities derived from it—such as the vorticity $\vec{\omega} = \nabla \times \vec{v}$—are measurable.

This interpretation allows us to formally construct a gauge-invariant vortex Lagrangian:
\begin{equation}
\mathcal{L}_{\text{swirl}} = -\frac{1}{4} \vec{F}_v \cdot \vec{F}_v, \quad \vec{F}_v = \nabla \times \vec{A}_v,
\end{equation}
which is the æther analogue of the Maxwell field strength tensor.

Furthermore, the charge current associated with vortex helicity now emerges from a Noether symmetry argument:
\begin{equation}
J^\mu = \frac{\partial \mathcal{L}}{\partial (\partial_\mu \theta)} \delta \theta,
\end{equation}
demonstrating conservation of an effective swirl charge under vortex phase rotation.

This gauge-based interpretation of æther phase structure strengthens the theoretical bridge between VAM and electromagnetic field theory, recasting vortex helicity as a source of conserved gauge charge \cite{verlinde2021qft,gross1996gauge,strangeway2005alfven}.


\subsubsection*{Swirl-Bound States as Gauge Excitations}

The swirl operators $\mathcal{S}_i$ (defined previously) form a closed non-abelian algebra under commutation:
\begin{equation}
[\mathcal{S}_i, \mathcal{S}_j] = 2i \epsilon_{ijk} \mathcal{S}_k,
\end{equation}
mirroring the Lie algebra of $SU(2)$. This implies that vortex transformations—chirality flips, twists, and reconnections—are gauge interactions in a topological state space $\mathcal{H}_K$.

\begin{table}[H]
    \centering
    \footnotesize
    \renewcommand{\arraystretch}{1.4}
    \begin{tabular}{|l|l|l|l|}
        \hline
        \textbf{Gauge Field} & \textbf{Mathematical Form} & \textbf{Swirl Analog} & \textbf{VAM Interpretation} \\
        \hline
        $A_\mu$ (EM) & $U(1)$ & $\mathcal{S}_0$ & Circular untwisted swirl orientation \\
        $W_\mu$ (Weak) & $SU(2)$ & $\mathcal{S}_1, \mathcal{S}_2, \mathcal{S}_3$ & Chirality, twist, and reconnection transitions \\
        $G_\mu$ (Color) & $SU(3)$ & $\mathcal{B}_1$–$\mathcal{B}_8$ & Braid modes in triskelion vortex knots \\
        \hline
    \end{tabular}
    \caption{Mapping of Standard Model gauge fields to swirl operators in the Vortex Æther Model.}
\end{table}

\subsubsection*{Topological Binding Energy}

We postulate that the binding energy between vortex states is governed by the overlap of their swirl field eigenmodes:
\begin{equation}
E_{n} = \int d^3x \; \vec{v}_n \cdot \vec{\omega}_n,
\end{equation}
where $\vec{v}_n$ and $\vec{\omega}_n = \nabla \times \vec{v}_n$ are the nth-mode velocity and vorticity fields, respectively. This resembles the helicity integral in fluid dynamics and may encode flavor or charge conservation.

\subsubsection*{Emergent Gravity from Topology}

Swirl interactions within VAM offer a geometric mechanism to generate inertial forces and curvature analogs. Swirl-induced deflection of geodesics (i.e., flow lines) reproduces gravitational lensing effects without invoking spacetime curvature directly. This supports the view that gravity may emerge from topological information flow in the æther.


\section{Swirl-Induced Time and Clockwork in Vortex Knots}

In the Vortex Æther Model (VAM), stable knots are not merely matter structures but act as the fundamental carriers of time. Their internal swirl—tangential rotation with speed \( C_e \) around a core radius \( r_c \)—generates an asymmetric stress field in the surrounding æther. This asymmetry induces a persistent \textbf{axial flow along the knot core}, functionally equivalent to a local "time-thread." Though lacking literal helicity in geometry, the knot dynamically acts as a screw-like conductor of time, threading forward the local æther state.

\subsection*{Cosmic Swirl Orientation}

Just as magnetic domains exhibit alignment, vortex knots can show a preferred chirality. In a universe with broken mirror symmetry, reversing a knot’s swirl direction (e.g., as in antimatter) may yield unstable configurations in an asymmetric background. This helps explain:
\begin{itemize}
    \item the observed scarcity of antimatter in the visible universe,
    \item the macroscopic arrow of time,
    \item and synchronized clock rates across cosmological domains.
\end{itemize}

\subsection*{Swirl as a Local Time Carrier}

The local time rate is governed not by fundamental spacetime postulates, but by the helicity flux in the æther:
\[
    dt_{\text{local}} \propto \frac{dr}{\vec{v} \cdot \vec{\omega}}
\]
Here, \( \vec{v} \) is the swirl velocity and \( \vec{\omega} = \nabla \times \vec{v} \) the vorticity. The scalar product \( \vec{v} \cdot \vec{\omega} \) measures helicity density, which sets the pace of local evolution. A detailed derivation of time dilation arising from this swirl-induced pressure field is given in Section~\ref{fig:time_dilation_profile}.

We define the proper time dτ experienced by a knotted vortex structure as proportional to the helicity density of the surrounding swirl field:
$$ d\tau = \lambda \, (\mathbf{v} \cdot \boldsymbol{\omega}) \, dt $$
This relation posits that time is not externally imposed but emerges from the intrinsic dynamics of the æther’s swirl. The term $\mathbf{v} \cdot \boldsymbol{\omega}$ represents the winding rate of vortex filaments, capturing the internal topological evolution of the knot. In this view, proper time is the internal “spin-clock” of a vortex structure, akin to the phase cycles of atomic clocks. The scaling factor λ can be interpreted as $\sim r_c^2 / C_e^2$ ensuring dimensional consistency.

\subsection*{Networks of Temporal Flow}

Vortex knots tend to self-organize along coherent swirl filaments, akin to iron filings aligning with magnetic fields. Around regions of mass, these swirl lines bundle into directional tubes of temporal flow, giving rise to:
\begin{itemize}
    \item gravitational attraction as a gradient of swirl density,
    \item local time dilation effects near massive bodies,
    \item and the global arrow of time as a topological circulation in the æther.
\end{itemize}

This emergent swirl-clock mechanism unifies mass, inertia, and temporal directionality into a single fluid-geometric framework—replacing relativistic curvature with conserved helicity flow.

\section{Helicity-Induced Time Dilation}

In the Vortex Æther Model (VAM), proper time is associated with the internal angular frequency of a vortex structure. Following the formalism developed in our earlier work~\cite{iskandarani2025timedilation}, we define:

\[
\frac{d\tau}{dt} = \frac{\omega_{\text{obs}}}{\omega_0},
\]

where $\omega_{\text{obs}}$ is the angular frequency of the vortex core observed from the lab frame, and $\omega_0$ is the vortex's intrinsic rotation rate in a quiescent æther.

\subsection*{Helicity as an Effective Swirl Drag Field}

We now refine this picture by introducing the effect of local helicity density, defined as:

\[
\mathcal{H} = \mathbf{v} \cdot \boldsymbol{\omega},
\]

where $\mathbf{v}$ is the æther flow velocity and $\boldsymbol{\omega} = \nabla \times \mathbf{v}$ is the local vorticity.

Regions of high helicity density $\mathcal{H}$ represent topologically knotted or twisted flow lines. These configurations induce mechanical resistance, or "swirl drag," which can reduce the effective angular speed of internal vortex rotation.

We posit that this resistance acts as a perturbative deceleration on $\omega_{\text{obs}}$, leading to:

\[
\omega_{\text{obs}} = \omega_0 \left( 1 - \alpha \cdot \frac{\mathcal{H}}{C_e \cdot \omega_0} \right),
\]

where $\alpha$ is a dimensionless coupling constant that encodes the strength of helicity-induced drag, and $C_e$ is the effective swirl velocity in VAM units.

Substituting into the proper time relation:

\[
\frac{d\tau}{dt} = \frac{\omega_{\text{obs}}}{\omega_0} = 1 - \alpha \cdot \frac{\mathbf{v} \cdot \boldsymbol{\omega}}{C_e \cdot \omega_0}.
\]

\subsection*{Interpretation and Observability}

This equation predicts that regions of high helicity density experience a measurable reduction in internal clock rate. Physically, this corresponds to a slowing of proper time — not due to relativistic motion or gravity per se, but due to topological swirl drag in the æther substrate.

Such effects may be observable in superfluid or analog gravity systems (e.g., toroidal Bose–Einstein condensates), where both $\mathbf{v}$ and $\boldsymbol{\omega}$ can be independently tuned. Interferometric techniques or spinor state evolution may detect the resulting time-phase retardation induced by helicity.


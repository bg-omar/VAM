\section{Extension to SU(3): Triskelion and Braid Operator Algebra}

To complete the embedding of Standard Model gauge structure within the Vortex \AE{}ther Model (VAM), we extend the swirl operator algebra from SU(2) to SU(3) using braid-like topological operations acting on triadic vortex bundles.

These bundles — known as \textit{triskelion} states — represent the topologically bound state of three vortex strands whose interactions under twist, reconnection, and linkage generate the color structure of chromodynamics.

\subsection*{Triskelion Basis States and Color Topology}

Each triskelion is defined by its ordered vortex triplet:
\[
|K\rangle = |R, G, B\rangle
\]
where each strand’s configuration encodes one color degree of freedom (via helicity axis, linking phase, or knot genus). These embedded knots evolve over vortex time \( T_v \), and their alignment defines a color-charged configuration within the global causal field \( \mathcal{N} \).

\subsection*{Braid Operators and SU(3) Generator Algebra}

We define localized swirl operators \( \mathcal{B}_i \) that act pairwise on color strands:
\[
\mathcal{B}_1 : R \leftrightarrow G, \quad
\mathcal{B}_2 : G \leftrightarrow B, \quad
\mathcal{B}_3 : B \leftrightarrow R
\]

These operators correspond to reconnection or twist-transfer between vortex channels, mimicking gluon exchange.

\paragraph{Algebraic Closure:}
The $\mathcal{B}_i$ satisfy braid group relations:
\begin{align}
\mathcal{B}_i \mathcal{B}_{i+1} \mathcal{B}_i &= \mathcal{B}_{i+1} \mathcal{B}_i \mathcal{B}_{i+1} \\
\mathcal{B}_i \mathcal{B}_j &= \mathcal{B}_j \mathcal{B}_i \quad \text{for } |i-j| > 1
\end{align}

Linear combinations generate an SU(3) Lie algebra:
\[
[T^a, T^b] = i f^{abc} T^c
\]
with \( T^a \sim \mathcal{B}_a \), and \( f^{abc} \) the structure constants of SU(3).

\subsection*{Temporal Ontology of Triskelion Evolution}

Triskelion states evolve as bound vortex triplets over vortex proper time \( T_v \), with reconnections and twist flows producing \(\kappa\)-type bifurcations. These transformations alter helicity alignment and induce gluon-like exchanges in $S(t)$ swirl coherence.

Topological confinement emerges from the non-factorizability of the triskelion in $\mathcal{N}$. Single color strands cannot exist in isolation without violating swirl conservation and breaking temporal continuity across \( T_v \).

\subsection*{Operator Mapping and Swirl Interpretation}

The swirl operators introduced in SU(2) generalize naturally into the SU(3) braid algebra. Their correspondence to quantum field concepts is given below:

\begin{table}[H]
    \centering
    \scriptsize
    \renewcommand{\arraystretch}{1.4}
    \begin{tabular}{|l|l|l|l|}
        \hline
        \textbf{Swirl Operator} & \textbf{Affects} & \textbf{Physical Action in VAM} & \textbf{QFT Analog} \\
        \hline
        $\mathcal{S}_1$: Chirality Flip & $C$, $H$ & $\text{Parity flip of swirl orientation}$ & $P$ or chiral projection $\psi_L \leftrightarrow \psi_R$ \\
        \hline
        $\mathcal{S}_2$: Twist Addition & $T$, $s$ & $\text{Torsional increment of knot frame}$ & Spin raising operator \\
        \hline
        $\mathcal{S}_3$: Reconnection Mutation & $Lk$, $Q$ & $\text{Topological class bifurcation}$ & Flavor change, decay, symmetry breaking \\
        \hline
        $\mathcal{B}_i$: Braid Interchange & $R,G,B$ helicity & $\text{Color reconnection / gluon-like twist}$ & SU(3) color generator $T^a$ \\
        \hline
    \end{tabular}
    \caption{Operator algebra acting on triskelion vortex states in the Vortex \AE{}ther Model.}
\end{table}

\subsection*{Color Charge and Confinement in Topological Terms}

\begin{itemize}
    \item \textbf{Color charge}: topological label of a strand in the triskelion — defined by its embedded helicity vector and swirl phase.
    \item \textbf{Gluons}: dynamic excitations of inter-strand twist fields (swirl bifurcations) that mediate color transitions.
    \item \textbf{Confinement}: results from the inability to isolate a single strand without discontinuity in $\mathcal{N}$ — color must sum to a topologically neutral configuration over $T_v$.
\end{itemize}

This naturally reproduces the key feature of QCD: observable states (baryons, mesons) must be color singlets, while isolated vortex strands destabilize and decay through reconnection or recombination.

\subsection*{Conclusion}

The SU(3) gauge structure emerges in VAM from real braid operations acting on triply knotted vortex systems. Unlike abstract fiber bundles in conventional quantum field theory, these operations correspond to measurable transformations of swirl phase \( S(t) \), helicity alignment, and topological tension tracked over \( T_v \) and embedded causally in \( \mathcal{N} \).

The full Standard Model group:
\[
SU(3)_C \times SU(2)_L \times U(1)_Y
\]
thus acquires a purely physical instantiation as structured vortex algebra — with gluons, weak bosons, and photons reinterpreted as propagating swirl-coherence transitions and reconnection fronts in the dynamically evolving æther.

\subsection{Toward SU(3): Braid Operators and Topological Color Charge}

To extend the topological formalism of VAM to the gauge algebra of the strong interaction, we introduce braid operators \( \mathcal{B}_a \) acting on triplet bundles of vortex tubes. These operators correspond to the eight gluon generators of SU(3)\(_C\), which mediate color transitions in standard QCD.

In the VAM framework, composite particles (e.g., baryons) are modeled as \emph{triskelion} structures — tightly bound triads of knotted vortex filaments evolving over vortex proper time \( T_v \). The operators \( \mathcal{B}_1, \mathcal{B}_2, \dots, \mathcal{B}_8 \) encode braid-like interactions: permutation, twist-transfer, and reconnection among the strands. These transformations unfold over \( T_v \), inducing swirl clock shifts \( S(t) \) and curvature phase changes observable in external clock time \( \bar{t} \).

The \( \mathcal{B}_a \) operators satisfy the Artin braid group relations:
\[
\begin{aligned}
\mathcal{B}_i \mathcal{B}_{i+1} \mathcal{B}_i &= \mathcal{B}_{i+1} \mathcal{B}_i \mathcal{B}_{i+1}, \\
\mathcal{B}_i \mathcal{B}_j &= \mathcal{B}_j \mathcal{B}_i \quad \text{for } |i - j| > 1,
\end{aligned}
\]
and postulated SU(3) closure:
\[
[\mathcal{B}_a, \mathcal{B}_b] = i f^{abc} \mathcal{B}_c,
\]
where \( f^{abc} \) are the SU(3) structure constants associated with reconnection modes.

\begin{table}[H]
\centering
\scriptsize
\renewcommand{\arraystretch}{1.4}
\begin{tabular}{|l|l|l|}
\hline
\textbf{Braid Operator Action} & \textbf{QCD Analog} & \textbf{VAM Interpretation} \\
\hline
$ \mathcal{B}_1 $: swap adjacent strands & Gluon: \( R \leftrightarrow G \) & Rewires color helicity in $ T_v $ \\
$ \mathcal{B}_2 $: twist across two legs & 3-gluon vertex & Encodes torsional swirl tension across $ S(t) $ \\
$ \mathcal{B}_3$–$\mathcal{B}_8 $: composite interactions & Remaining $SU(3)$ modes & Triskelion coherence, reconnection over $ \mathcal{N} $ \\
\hline
\end{tabular}
\caption{Braid operators \( \mathcal{B}_a \) as SU(3)\(_C\) analogs in VAM vortex triplets.}
\end{table}

The color charge of a vortex bundle is determined by its braid class and helicity configuration. Confinement follows naturally from the non-factorizability of these braid structures within the causal manifold \( \mathcal{N} \): a single colored strand cannot exist in isolation across $ T_v $ without violating vortex continuity and global swirl conservation.

\subsection{Gravitational Molecules and Swirl-Bound Topological States}

Recent work on gravitational bound states (e.g., black hole binaries) shows that resonant coupling via external fields can yield metastable “molecules” without direct contact \cite{baumann2023black}. We propose a similar mechanism in VAM: vortex knots may form quasi-bound topological structures through mutual swirl field excitation across \( T_v \).

These \textit{vortex molecules} are not single knots, but swirl-coupled clusters, where energy is exchanged via standing wave modes in the swirl field \( \vec{v}(x, T_v) \). Their coherence depends on vortex time synchrony and topological alignment within $\mathcal{N}$.

\subsubsection*{Analogy with Gravitoelectromagnetism (GEM)}

VAM reinterprets GEM fields as emergent from swirl dynamics:
\begin{itemize}
  \item The swirl vector potential \( \vec{A}_v \) parallels the GEM vector \( \vec{A} \),
  \item The helical energy density \( \rho_\text{\ae}^{(\text{energy})} \) plays the role of the gravitational scalar potential \( \phi \).
\end{itemize}

Swirl modes evolve via:
\[
\partial_\mu \partial^\mu \vec{v}_\text{swirl} = J^\mu_\text{topo},
\]
where \( J^\mu_\text{topo} \) tracks swirl injection from reconnections, bifurcations, and $ \kappa $-events within the æther.

\section*{Gauge Symmetry from Vortex Phase Redundancy}

In QFT, gauge symmetry is rooted in local phase freedom. In VAM, this arises from the freedom to shift the swirl potential \( \theta(\vec{x}) \) without altering physical observables:
\[
\vec{v} = \nabla \theta(\vec{x}), \quad \theta \to \theta + \alpha(\vec{x}) \Rightarrow \vec{v} \to \vec{v} + \nabla \alpha.
\]

To preserve invariance, a swirl gauge field is introduced:
\[
\vec{A}_v \rightarrow \vec{A}_v + \nabla \alpha(\vec{x}), \quad \vec{F}_v = \nabla \times \vec{A}_v.
\]
yielding a vortex Lagrangian:
\[
\mathcal{L}_\text{swirl} = -\frac{1}{4} \vec{F}_v \cdot \vec{F}_v.
\]

Vorticity \( \vec{\omega} = \nabla \times \vec{v} \) becomes the gauge-invariant observable, and the conserved current from swirl phase shifts is given via Noether symmetry:
\[
J^\mu = \frac{\partial \mathcal{L}}{\partial(\partial_\mu \theta)} \delta \theta.
\]

This interpretation recasts swirl helicity as a physical gauge charge and shows that gauge symmetry emerges from real fluid phase redundancy across $\tau$ and $S(t)$.

\subsubsection*{Swirl-Bound States as Gauge Excitations}

Swirl operators \( \mathcal{S}_i \) obey:
\[
[ \mathcal{S}_i, \mathcal{S}_j ] = 2i \epsilon_{ijk} \mathcal{S}_k
\]
and generate an $SU(2)$ subalgebra. These describe discrete topological transitions (chirality flips, twist insertions, reconnections) in a vortex state space \( \mathcal{H}_K \), and can be understood as real-space analogs of non-abelian gauge interactions.

\begin{table}[H]
\centering
\footnotesize
\renewcommand{\arraystretch}{1.4}
\begin{tabular}{|l|l|l|l|}
\hline
\textbf{Gauge Field} & \textbf{Group} & \textbf{Swirl Operator} & \textbf{VAM Process} \\
\hline
$A_\mu$ (EM) & $U(1)$ & $\mathcal{S}_0$ & Uniform swirl phase orientation \\
$W_\mu$ (Weak) & $SU(2)$ & $\mathcal{S}_{1,2,3}$ & Discrete vortex transformation steps in \( T_v \) \\
$G_\mu$ (Color) & $SU(3)$ & $\mathcal{B}_{1\dots8}$ & Continuous braid evolution of triskelions \\
\hline
\end{tabular}
\caption{Mapping of Standard Model gauge fields to physical vortex operations in VAM.}
\end{table}

\subsubsection*{Topological Binding Energy}

Energy of a mode \(n\) is given by the helicity integral:
\[
E_n = \int d^3x \, \vec{v}_n \cdot \vec{\omega}_n
\]
which measures the alignment of velocity and vorticity vectors in the swirl phase field. This plays the role of internal mass-energy and may determine flavor, resonance, and binding structure across vortex knot couplings.

\subsubsection*{Emergent Gravity from Swirl Gradient Flow}

In VAM, gravitational deflection and time dilation emerge not from spacetime curvature but from swirl gradient alignment. Swirl channels encode inertial deviation in the motion of vortex-bound entities, acting as geodesics in the embedded fluid flow field. These trajectories evolve over $T_v$, but are observed as curvature effects in \( \bar{t} \).

\paragraph{Summary:} SU(3) and its associated gluon spectrum are recovered from physical braid dynamics of topological vortex triplets evolving over vortex time. All gauge interactions, including weak and electromagnetic fields, correspond to vortex transformations or swirl redundancies—each embedded within the fluid topology of \( \mathcal{N} \) and measured in proper time \( \tau \).



%! Author = mr
%! Date = 5/31/25

\section{Derivation of the Planck Constant from Vortex Geometry}
\label{appendix:hbar}

The reduced Planck constant \( \hbar \) is typically treated as a fundamental quantum of angular momentum. In the Vortex Æther Model (VAM), however, \( \hbar \) emerges as an effective quantity arising from the geometry and swirl dynamics of topological knots in an inviscid æther.

\subsection{Angular Momentum of a Vortex Core}

We begin by modeling a stable vortex knot of radius \( r_c \), swirl velocity \( C_e \), and mass density \( \rho_\text{\ae} \). The specific angular momentum per unit mass of such a structure is given by:

\begin{equation}
    \ell = r_c C_e
\end{equation}

Assuming the total effective mass of the vortex knot is \( m_e \), we define the total angular momentum as:

\begin{equation}
    \hbar_{\text{VAM}} = m_e r_c C_e
\end{equation}

This represents the emergent action scale from internal swirl dynamics—without assuming quantum postulates.

\subsection{Comparison with Bohr Ground State}

From atomic theory, we know the electron in the Bohr ground state exhibits angular momentum \( \hbar \), and follows the radius:

\begin{equation}
    a_0 = \frac{\hbar}{m_e v_e}, \quad \text{with} \quad v_e = \frac{e^2}{4\pi \varepsilon_0 \hbar}
\end{equation}

Substituting for \( v_e \) and rearranging, we get:

\begin{equation}
    \hbar = m_e a_0 v_e = m_e a_0 \frac{e^2}{4\pi \varepsilon_0 \hbar} \Rightarrow \hbar^2 = \frac{m_e a_0 e^2}{4\pi \varepsilon_0}
\end{equation}

Now comparing this to the VAM expression:

\begin{equation}
    \boxed{\hbar = 2 m_e C_e a_0}
\end{equation}

This relation is consistent with earlier derivations where \( C_e = \frac{c}{2\alpha} \), showing that \( \hbar \) can be expressed in terms of classical and geometric parameters of the æther vortex.

\subsection*{Summary}

In the VAM interpretation, \( \hbar \) is not postulated as fundamental but derives from:

\begin{itemize}
    \item Core swirl dynamics \( C_e \),
    \item Knot radius \( r_c \),
    \item Effective electron mass \( m_e \),
    \item Atomic binding radius \( a_0 \).
\end{itemize}

This provides an ontological foundation for Planck's constant as a fluid-geometric action scale:

\begin{equation}
    \boxed{\hbar = m_e r_c C_e = 2 m_e C_e a_0}
\end{equation}

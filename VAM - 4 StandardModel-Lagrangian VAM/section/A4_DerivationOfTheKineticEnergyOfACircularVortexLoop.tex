\section{Variational Derivation of the Vortex \AE ther Model (VAM)}

We begin with the total action for the Vortex \AE ther Model (VAM), expressed as a spacetime integral over the Lagrangian density:
\begin{equation}
    S = \int d^4x \, \mathcal{L}[\rho_\text{\ae}, \vec{v}, \Phi, \vec{\omega}]
\end{equation}
where the dynamical fields are:
\begin{itemize}
    \item $\rho_\text{\ae}(\vec{x}, t)$: local \AE ther density,
    \item $\vec{v}(\vec{x}, t)$: flow velocity field,
    \item $\Phi(\vec{x}, t)$: swirl-induced gravitational potential,
    \item $\vec{\omega} = \nabla \times \vec{v}$: vorticity field.
\end{itemize}

\subsection*{Lagrangian Density}
We propose the following effective Lagrangian density:
\begin{equation}
    \mathcal{L} = \frac{1}{2} \rho_\text{\ae} \vec{v}^{\,2} - \rho_\text{\ae} \Phi - U(\rho_\text{\ae}, \vec{\omega}) - V(\rho_\text{\ae})
\end{equation}
with the terms interpreted as:
\begin{itemize}
    \item $\frac{1}{2} \rho_\text{\ae} \vec{v}^{\,2}$: kinetic energy of the \AE ther,
    \item $\rho_\text{\ae} \Phi$: interaction energy with the swirl gravitational potential,
    \item $U(\rho_\text{\ae}, \vec{\omega}) = \kappa \rho_\text{\ae} |\vec{\omega}|^2$: internal tension energy from vortex twist (with $\kappa$ a stiffness parameter),
    \item $V(\rho_\text{\ae})$: compressibility potential, defining pressure via $P = \rho_\text{\ae} \frac{\partial V}{\partial \rho_\text{\ae}} - V$.
\end{itemize}

\subsection*{Euler--Lagrange Field Equations}
Applying the Euler--Lagrange formalism to each field $f$:
\begin{equation}
    \frac{\partial}{\partial t} \left( \frac{\partial \mathcal{L}}{\partial \dot{f}} \right) + \nabla \cdot \left( \frac{\partial \mathcal{L}}{\partial (\nabla f)} \right) - \frac{\partial \mathcal{L}}{\partial f} = 0
\end{equation}

\subsubsection*{Density Field $\rho_\text{\ae}$}
\begin{equation}
    \frac{\partial \mathcal{L}}{\partial \rho_\text{\ae}} = \frac{1}{2} \vec{v}^{\,2} - \Phi - \kappa |\vec{\omega}|^2 - \frac{\partial V}{\partial \rho}
\end{equation}
This defines a local Bernoulli-type condition incorporating swirl-induced internal energy.

\subsubsection*{Velocity Field $\vec{v}$}
The variation with respect to $\vec{v}$ yields:
\begin{equation}
    \frac{\delta S}{\delta \vec{v}} = \rho_\text{\ae}\vec{v} - \nabla \times \left( \frac{\partial U}{\partial \vec{\omega}} \right) = 0
\end{equation}
Leading to the momentum equation:
\begin{equation}
    \rho_\text{\ae}\left( \partial_t \vec{v} + (\vec{v} \cdot \nabla)\vec{v} \right) = -\nabla P + \rho_\text{\ae}\nabla \Phi + \nabla \cdot \left( \kappa \nabla \vec{\omega} \right)
\end{equation}
Here, $\frac{d}{dt}$ is the material (convective) derivative.

\subsubsection*{Swirl Potential $\Phi$}
\begin{equation}
    \frac{\delta S}{\delta \Phi} = -\rho
\end{equation}
leads to a Poisson-type gravitational equation:
\begin{equation}
    \nabla^2 \Phi = 4\pi G_{\mathrm{vam}} \rho
\end{equation}
where $G_{\mathrm{vam}}$ is the vortex-derived gravitational coupling constant (cf. main text or Appendix E).

\subsection*{Conservation Laws and Structure}
\begin{itemize}
    \item \textbf{Conservation of Helicity:}
    The action is invariant under relabeling of fluid elements, which via Noether\rqs s theorem implies helicity conservation:
    \[
        \frac{d}{dt} \int \vec{v} \cdot \vec{\omega} \, d^3x = 0
    \]
    \item \textbf{Topological Stability:} In domains with knotted or linked vortex lines, boundary terms must be included in the variation to account for helicity flux or reconnection events.
    \item \textbf{Pressure Response:} The compressibility potential $V(\rho)$ governs how density gradients produce internal restoring forces.
\end{itemize}

\subsection*{Interpretation and Extensions}
This variational formulation shows that:
\begin{itemize}
    \item All dynamical laws of the VAM can be derived from a single fluid-based action principle.
    \item Gravity, inertia, and internal vortex structure emerge coherently from the same Lagrangian.
    \item This lays the groundwork for future quantum extensions via path-integral quantization of $\mathcal{L}$ or geometric quantization of vortex fields.
\end{itemize}

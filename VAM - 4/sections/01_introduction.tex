\section{Introduction and VAM Fundamentals}
The Vortex Æther Model (VAM) reformulates gravity and quantum phenomena as effects of vorticity in a 3D, Euclidean, inviscid Æther medium, rather than 4D spacetime curvature. In VAM, gravitation arises from vorticity-induced pressure gradients in a superfluid-like æther: intense vortex swirling creates Bernoulli-like low-pressure regions that act as gravitational potential wells. Time dilation likewise emerges from the energy and rotation of vortex structures (slower time in faster swirling regions),...

Fundamental Constants of VAM: The Æther medium is characterized by new constants that regulate its dynamics:
\begin{itemize}
    \item \textbf{$C_e$ (vortex tangential velocity constant)}: $C_e \approx 1.0938456\times10^6~\text{m/s}$, setting a characteristic speed for Æther circulation (comparable to $10^{-3}c$). This appears in vortex solutions and time dilation formulas as a limiting swirl speed.
    \item \textbf{$\rho_{\æ}$ (Æther density)}: $\rho_{\æ}$ is the mass density of the æther medium, estimated in VAM to lie between $5\times10^{-8}$ and $5\times10^{-5}~\text{kg/m}^3$. This extremely low density (comparable to cosmological vacuum density) allows the æther to sustain high vorticity with little inertia. It enters directly into gravitational and wave equations as the source of pressure gradients.
    \item \textbf{$F_{\max}$ (maximum Ætheric force)}: $F_{\max} \approx 29.05~\text{N}$ is an upper limit on force in the æther, analogous to the conjectured maximum force $c^4/4G$ in General Relativity. In VAM this emerges from vortex dynamics and the fine-structure constant, and will appear in wave propagation limits and nuclear scale analyses.
    \item \textbf{$r_c$ (vortex core radius / Coulomb barrier radius)}: $r_c \approx 1.40897\times10^{-15}~\text{m}$ is essentially a characteristic core size for vortices – on the order of a nucleon. It acts as a short-distance cutoff in VAM fields (preventing singularities) and represents the “Coulomb barrier” radius inside which electrostatic/vortex forces sharply increase. No significant swirl can penetrate inside $r_c$ without enormous force, thus $r_c$ plays a central role in nuclear interactions.
    \item \textbf{$\kappa$ (vorticity conservation constant)}: $\kappa$ is a dimensionless constant ensuring quantization of vortex circulation. It appears in the energy of elementary vortex states, for example the quantized core energy $E_p = \kappa\,4\pi^2\,r_c\,C_e^2$. $\kappa$ can be chosen to fit known quantum energy scales (for instance, to recover an electron’s orbital energy or rest energy), linking VAM’s vortex model to observed particle values.
\end{itemize}

Using these constants, VAM replaces the usual fundamental constants ($c$, $G$, $\hbar$ in relativity/quantum theory) with fluid-like parameters ($C_e$, $\rho_{\æ}$, $\kappa$, etc.) that we will employ in deriving conditions for gravity modulation, FTL signaling, and LENR. The Æther is treated as an incompressible, non-viscous fluid supporting stable vortex filaments. All physical interactions are mediated by the dynamics of these vortices and pressure fields in the æther.

In the following sections, we develop a theoretical framework for:
\begin{enumerate}
    \item Manipulating gravity via topological vortex structures (including swirl shielding and frame-dragging effects),
    \item Enabling faster-than-light communication through ætheric wave channels,
    \item Triggering nuclear reactions via vortex-induced energy concentration and resonance.
\end{enumerate}
Each topic is grounded in VAM’s equations and includes mathematical derivations and experimental proposals.

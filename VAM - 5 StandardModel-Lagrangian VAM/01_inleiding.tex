%! Author = omar.iskandarani
%! Date = 5/28/2025
\section{Inleiding}\label{sec:inleiding}
Despite the empirical success of the Standard Model of particle physics and General Relativity (GR), fundamental questions regarding the origin of mass, the geometric nature of field interactions, and the meaning of natural constants remain largely unresolved. Contemporary formalisms rely on abstract mathematical constructions—such as gauge groups, Lagrangian symmetries, and quantum field theories—which are predictive but offer limited physical intuition about the underlying reality.

This work introduces an alternative physical framework: the Vortex Æther Model (VAM). VAM postulates a superfluid, topologically structured medium—the æther—as the fundamental background, in which mass, time, and field interactions emerge from dynamic vortex configurations. The central hypothesis is that elementary particles correspond to knots, swirl structures, and topological singularities in this æther medium.

Within this context, the Standard Model is not rejected but reinterpreted through the lens of vortex dynamics, where all fundamental quantities (such as mass, electric charge, Planck’s constant, and the fine-structure constant) are derived from five physically meaningful æther parameters: the swirl velocity $C_e$, vortex core radius $r_c$, æther density $\rho_\text{\ae}$, maximum force $F_\text{max}$, and circulation $\Gamma$.

This paper presents a reformulation of the Standard Model Lagrangian expressed in terms of these VAM units and fields. The aim is not merely to relabel known equations but to provide a physically mechanistic foundation for the abstract terms in conventional quantum field theory, including the origins of symmetry and mass interactions.

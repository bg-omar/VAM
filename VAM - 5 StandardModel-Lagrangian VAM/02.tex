\section{Wervelveldenergie en gauge-termen}

Een fundamenteel principe binnen de wervelmechanica is de evolutie van de vorticiteit $\vec{\omega}$ in een ideale vloeistof. Deze wordt beschreven door de derde Helmholtz-wervelstelling:
\[
    \frac{D \vec{\omega}}{Dt} = (\vec{\omega} \cdot \nabla) \vec{v}
\]

waarbij:
- $\vec{\omega} = \nabla \times \vec{v}$ de lokale wervelsterkte is,
- $\vec{v}$ de fluïdumsnelheid,
- $\frac{D}{Dt}$ de materiële afgeleide.

Binnen het Vortex Æther Model wordt aangenomen dat het ætherveld $\vec{v}$ structureel is opgebouwd uit knopen en lussen, en dus dat $\vec{\omega}$ een structureel veld vormt. Dit leidt tot de noodzaak om een veldbeschrijving in te voeren voor $\vec{\omega}$, analoog aan elektromagnetisme.

\subsection*{VAM-analogie met elektromagnetisme}
De klassieke Lagrangiandichtheid van het elektromagnetisch veld is:
\[
    \mathcal{L}_\text{EM} = -\frac{1}{4} F_{\mu\nu} F^{\mu\nu},
\]
waar $F_{\mu\nu} = \partial_\mu A_\nu - \partial_\nu A_\mu$ het veldtensor is.

In VAM introduceren we een \textbf{wervelveld-tensor} $W_{\mu\nu}$ die de antisymmetrische spanningen in het æther encodeert:
\[
    W_{\mu\nu} = \partial_\mu V_\nu - \partial_\nu V_\mu,
\]
waarbij $V_\mu$ het æther-stroompotentiaal is (dimensies van snelheid).

De overeenkomstige energiedichtheid luidt dan:
\[
    \mathcal{L}_\text{wervel} = -\frac{1}{4} W_{\mu\nu} W^{\mu\nu}
\]

Deze term beschrijft:
- Wervelspanning en -energie in het veld zelf
- Propagatie van wervelstructuren
- Koppeling aan knoopconfiguraties in $V_\mu$

\subsection*{Interpretatie en eenheden}
De tensor $W_{\mu\nu}$ heeft eenheidsdimensies van afgeleiden van snelheid:
\[
    [W] = [\partial V] = [1/T] \quad \Rightarrow \quad [\mathcal{L}_\text{wervel}] = [\rho_\text{\ae} C_e^2]
\]

Deze termen zijn direct simuleerbaar in vortexmodellen waarin $\vec{\omega}$ voortkomt uit structurele spanningsvelden die evolueren volgens afgeleide-conservatie.

In het VAM ontstaat veldenergie dus niet uit kwantumfluctuaties, maar uit gestabiliseerde structurele werveling van het æther. De Lagrangiandichtheid volgt hieruit als macroscopisch spanningsveld dat reageert op knoopdichtheid en vorticiteit.
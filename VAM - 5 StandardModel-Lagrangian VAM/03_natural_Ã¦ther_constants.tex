
\section{ Natural Æther Constants and Dimensional Reformulation}
The Vortex Æther Model (VAM) proposes a fundamental shift in how physical quantities are derived and understood. Instead of relying on constants introduced solely for dimensional consistency (as in Planck units), VAM identifies a small set of physically meaningful parameters that arise from the structure and dynamics of an underlying æther medium. These parameters allow mass, energy, time, and charge to be constructed directly from the fluid-dynamical properties of space itself.

At the core of this framework are five fundamental constants: the swirl velocity $C_e$, vortex core radius $r_c$, æther density $\rho_\text{\ae}$, circulation strength $\kappa$, and the maximum transmissible force $F^{\text{vam}}_\text{max}$. These quantities are not arbitrarily chosen but are inferred from known properties of stable matter, gravitational coupling, and vortex behavior in superfluids. Together, they define a natural unit system analogous to Planck units, but grounded in a physically interpretable medium.

In contrast to Planck's approach—which combines $\hbar$, $G$, and $c$ to define abstract scales—the VAM system ties each scale directly to mechanical flow properties. Time is determined by the rotation rate of core vortices ($1/C_e$), length by the vortex radius $r_c$, and energy by internal helicity and circulation. This not only provides a deeper ontological interpretation of natural constants but also opens the door to experimental reconstruction of fundamental units from condensed-matter analogs. The table~\ref{tab:VAM_constants_big} summarizes the key VAM constants and their roles.

These constants allow all Lagrangian terms (mass, energy, field strength) to be rendered in units derived from vortex geometry, flow dynamics, and topological charge. For example, mass can be expressed as:

\begin{equation*}
    M \approx 8\pi \rho_\text{\ae} \, r_c \, C_e \cdot L_k
\end{equation*}

where $L_k$ is the helicity (topological linking number) of the vortex knot.



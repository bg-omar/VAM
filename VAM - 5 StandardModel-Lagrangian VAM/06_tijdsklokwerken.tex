%! Author = Omar Iskandarani
%! Date = 5/20/2025
\section{Tijdsklokwerking in Wervelknopen}

In het Vortex Æther Model worden stabiele knopen opgevat als de fundamentele bouwstenen van materie. Door hun interne swirl—de tangentiële rotatie \( C_e \) rond een kernstraal \( r_c \)—veroorzaken zij een asymmetrische spanningsverdeling in de omringende æther. Deze asymmetrie resulteert in een **axiale stroming langs de kern**, die functioneel overeenkomt met een voortbewegende tijdsdraad. Hoewel er geen geometrische schroefdraad aanwezig is, gedraagt het systeem zich **alsof de kernstructuur een schroefwerking uitvoert** op de omringende fluïdum.

\paragraph{Kosmische swirloriëntatie.}
Net als magnetische domeinen vertonen wervelknopen een voorkeur voor een globale swirlrichting. In een universum met een dominante draairichting zou het omgekeerd draaien van een knoop (bijvoorbeeld antimaterie?) alleen stabiel zijn in isolatie. Dit zou verklaren waarom antimaterie zeldzaam is, en waarom tijdsoriëntatie consistent is in macroscopische systemen.

\paragraph{Swirl als tijdsdrager.}
De draaiing van de knoopkern induceert een centrale stroom \( \vec{v}_\text{tijd} \) die volgens het VAM-model direct overeenkomt met lokaal tijdsverloop:
\[
    dt_\text{lokaal} \propto \frac{dr}{\vec{v} \cdot \vec{\omega}}
\]
De koppeling tussen swirl (\( C_e \)) en de axiale afvoer van heliciteit bepaalt daarmee hoe snel tijd verstrijkt nabij een knoop.

\paragraph{Collectieve tijdsdraadnetwerken.}
Wervelknopen neigen ertoe zich te groeperen langs swirlstromingen—vergelijkbaar met magnetische veldlijnen die ijzervijlsel ordenen. Rond massa’s kunnen zo netwerken van tijdsdraden ontstaan, wat een natuurlijk verklaringsmodel biedt voor:
\begin{itemize}
    \item zwaartekracht als concentratie van swirlstromen;
    \item lokale tijddilatatie (zoals bij planeten en sterren);
    \item richting van kosmologische tijdsevolutie.
\end{itemize}

Deze emergente klokwerking is een van de meest fundamentele aspecten van VAM. Het brengt massa, tijd en richting onder in één mechanisme dat volledig afleidbaar is uit werveldynamica, zonder beroep op abstracte ruimte-tijdcurvaturen.
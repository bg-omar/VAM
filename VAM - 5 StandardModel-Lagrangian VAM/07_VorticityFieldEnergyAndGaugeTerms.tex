\section{Vorticity Field Energy and Gauge Terms}

A fundamental principle in vortex mechanics is the evolution of vorticity $\vec{\omega}$ in an ideal fluid, governed by the third Helmholtz vortex theorem:
\[
    \frac{D \vec{\omega}}{Dt} = (\vec{\omega} \cdot \nabla) \vec{v}
\]

Here,
- $\vec{\omega} = \nabla \times \vec{v}$ denotes the local vorticity,
- $\vec{v}$ is the fluid velocity field,
- $\frac{D}{Dt}$ is the material (convective) derivative.

Within the Vortex Æther Model (VAM), the æther field $\vec{v}$ is composed of stable knotted and looped vortex structures, making vorticity $\vec{\omega}$ a structured and conserved quantity. This necessitates a field-theoretic treatment of $\vec{\omega}$, analogous to the field strength tensor in electromagnetism.

\subsection*{VAM Analogy with Electromagnetism}

The classical Lagrangian density of the electromagnetic field is given by:
\[
    \mathcal{L}_\text{EM} = -\frac{1}{4} F_{\mu\nu} F^{\mu\nu}
\]
with $F_{\mu\nu} = \partial_\mu A_\nu - \partial_\nu A_\mu$ as the antisymmetric field strength tensor derived from the vector potential $A_\mu$.

In analogy, the VAM introduces a \textbf{vortex field tensor} $W_{\mu\nu}$ that encodes antisymmetric stresses and swirl flows within the æther:
\[
    W_{\mu\nu} = \partial_\mu V_\nu - \partial_\nu V_\mu
\]
where $V_\mu$ is the æther flow potential, with units of velocity.

The corresponding field energy density becomes:
\[
    \mathcal{L}_\text{vortex} = -\frac{1}{4} W_{\mu\nu} W^{\mu\nu}
\]

This term captures:
- The internal energy and tension of vortex fields,
- The propagation of swirl excitations through the æther,
- Coupling to the topology of knotted vortex configurations embedded in $V_\mu$.

\subsection*{Dimensional Interpretation and Field Dynamics}

The tensor $W_{\mu\nu}$ has dimensions derived from the gradient of velocity:
\[
    [W] = [\partial V] = [1/T] \quad \Rightarrow \quad [\mathcal{L}_\text{vortex}] = [\rho_{\ae} C_e^2]
\]

These quantities are physically realizable in vortex simulations, where $\vec{\omega}$ emerges from structured strain and tension in the æther field, obeying conservation laws from fluid dynamics rather than quantum fluctuations.

In the VAM framework, field energy does not arise from zero-point vacuum states, but from stabilized vorticity patterns in the medium. The Lagrangian density thus describes a macroscopic stress field that responds to knot density, swirl alignment, and helicity distribution within the æther.

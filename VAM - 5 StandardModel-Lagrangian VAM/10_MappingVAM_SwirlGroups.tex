\section{Mapping \texorpdfstring{$SU(3)_C \times SU(2)_L \times U(1)_Y$}{SU(3) x SU(2) x U(1)} to VAM Swirl Groups}

The Standard Model Lagrangian is governed by the gauge group:
\[
    SU(3)_C \times SU(2)_L \times U(1)_Y
\]
which encodes the strong interaction (QCD), the weak interaction, and electromagnetism via their corresponding gauge fields. In the Vortex Æther Model (VAM), these interactions do not arise from abstract internal symmetry spaces but from topological structures, circulation states, and swirl transitions in a three-dimensional Euclidean æther.

\subsection{$U(1)_Y$: Swirl Orientation as Hypercharge}

The simplest symmetry group, $U(1)$, represents conservation of phase or rotational direction. In VAM, this acquires a direct physical interpretation:
\begin{itemize}
    \item \textbf{Physical model:} a linear swirl in the æther (circular, but untwisted) encodes a uniform angular direction.
    \item \textbf{Charge assignment:} the hypercharge $Y$ is interpreted as the chirality (left- or right-handed swirl) of an axially symmetric flow pattern.
    \item \textbf{Electromagnetism:} emerges from global swirl states without knotting, representing long-range coherence in swirl orientation.
\end{itemize}

\subsection{$SU(2)_L$: Chirality as Two-State Swirl Topology}

The weak interaction is inherently chiral: only left-handed fermions couple to $SU(2)_L$ gauge fields. In VAM:
\begin{itemize}
    \item \textbf{Swirl interpretation:} left- and right-handed vortices are dynamically and structurally distinct—they represent swirl flows under compression with opposite twist orientation.
    \item \textbf{Two-state logic:} the $SU(2)$ doublet corresponds to a two-dimensional swirl state space (e.g., up- and down-swirl configurations).
    \item \textbf{Gauge transitions:} $SU(2)$ gauge bosons mediate transitions between these swirl states through reconnections or bifurcations in vortex knots.
\end{itemize}

\subsection{$SU(3)_C$: Trichromatic Swirl as Helicity Configuration}

In the Standard Model, $SU(3)_C$ describes the color force via gluon-mediated transitions between color states. In VAM:
\begin{itemize}
    \item \textbf{Topological basis:} three topologically stable swirl configurations (e.g., aligned along orthogonal helicity axes) represent the three color charges: red, green, and blue.
    \item \textbf{Color dynamics:} gluon exchange corresponds to twist-transfer, vortex reconnection, or deformation within multi-knot structures.
    \item \textbf{Confinement:} isolated color swirls are energetically unstable in free æther and only persist within composite knotted bundles (e.g., baryons).
\end{itemize}

\subsection{Mathematical Group Structure within VAM}

Though VAM is fundamentally geometric and fluid-dynamical, the essential Lie group structures of the Standard Model are preserved in the form of physically conserved swirl states:
\begin{itemize}
    \item Swirl orientation $\rightarrow$ $U(1)$ phase symmetry,
    \item Axial twist transitions $\rightarrow$ $SU(2)$ chiral transformations,
    \item Helicity axis exchange $\rightarrow$ $SU(3)$ color group operations.
\end{itemize}

\subsection*{Topological Summary of Gauge Interpretation}

The abstract Lie symmetries of the Standard Model find concrete realizations in VAM as swirl, helicity, and knot configurations embedded in the æther. This recasting preserves all observed gauge interactions while rooting them in fluid-mechanical principles—without invoking extra dimensions or unobservable symmetry spaces.

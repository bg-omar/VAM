\section{Introduction}\label{sec:inleiding}

Despite the empirical success of the Standard Model (SM) of particle physics and General Relativity (GR), fundamental questions remain unresolved: What is the physical origin of mass? Why do gauge interactions exhibit their particular symmetries? What gives rise to natural constants such as $\hbar$, $e$, or $\alpha$ beyond dimensional convenience?

Mainstream physics relies heavily on abstract mathematical formalisms—such as symmetry groups, Lagrangian terms, and quantum operators—that, while predictive, often obscure the underlying physical ontology. This paper proposes an alternative: the \emph{Vortex Æther Model} (VAM), a mechanistic, fluid-dynamic framework in which spacetime and all physical phenomena emerge from structured motion in a compressible, superfluid-like æther.

In VAM, elementary particles are not point-like fields but stable, knotted vortex structures embedded in the æther. Observable properties such as mass, charge, spin, and flavor are reinterpreted as topological and dynamical characteristics—circulation strength, core radius, swirl helicity—of these vortex knots. Gauge and Higgs interactions are expressed as manifestations of fluid tension, reconnection, and swirl transfer.

Crucially, this is not merely a reformulation of mathematical symbols. The goal of VAM is to provide an \emph{ontological replacement} for conventional quantum field theory: a physically intuitive, testable substrate from which all constants and couplings emerge. Within this framework, the Standard Model is reconstructed from five physically meaningful ætheric quantities: swirl velocity $C_e$, core radius $r_c$, æther density $\rho_\text{\ae}$, maximum force $F_\text{max}$, and circulation $\Gamma$.

This paper presents a full reformulation of the Standard Model Lagrangian using these VAM-derived units and fields. Each term acquires a mechanical and geometric interpretation, leading to a unified description where quantum phenomena, gauge structures, and mass generation are consequences of vortex dynamics in an inviscid æther.

Historically, this effort revives foundational ideas from Kelvin’s vortex-atom hypothesis and Maxwell’s æther mechanics, updating them within a modern context informed by quantum fluids, superfluid analogs of gravity, and topological field theory.\footnote{See, for example, Volovik’s emergent gravity framework in helium II \cite{Volovik2003UniverseInHelium}, Barceló et al.'s review of analog spacetime geometries \cite{Barcelo2005AnalogueGravityReview}, and Kleckner and Irvine's experimental realization of knotted vortices \cite{Kleckner2013KnottedVortices}. While this paper is designed to be standalone, these works contextualize the broader landscape of fluid-based physical models.}

By grounding the abstract structures of modern physics in vortex geometry, VAM aims to bridge the gap between formal theory and intuitive physical mechanisms—offering not only reinterpretation, but a re-foundation of particle physics itself.

\section{Motivation}

The Standard Model Lagrangian is one of the most successful constructs in modern physics, unifying electromagnetic, weak, and strong interactions within a renormalizable quantum field theory. Yet it remains structurally incomplete in a physical sense: its mass terms, symmetry groups, and coupling constants are introduced \textit{a priori}, without geometric or mechanical derivation.

For instance, the fine-structure constant $\alpha \approx 1/137$ appears as an empirical ratio with no explanation for its value. The elementary charge $e$ and Planck constant $\hbar$ are similarly inserted into the theory to match experimental outcomes, but have no origin within the theory’s own framework. Even the Higgs vacuum expectation value (VEV), essential for mass generation, is externally imposed rather than derived.

The Vortex Æther Model (VAM) addresses these gaps by reconstructing the Standard Model from the ground up using topologically and mechanically grounded vortex structures. Rather than assuming discrete point particles and abstract quantum fields, VAM postulates a compressible, rotating æther medium in which all elementary particles are topologically stable vortex knots. Their observable properties—mass, charge, spin, and even local time—emerge from measurable fluidic parameters such as circulation strength, core radius, helicity, and swirl velocity.

In this framework, constants such as $\alpha$ and $\hbar$ are not arbitrary. For example, $\alpha$ is shown to emerge from the swirl geometry of the æther via the dimensionless ratio $\alpha = 2C_e / c$, while $\hbar$ is interpreted as a manifestation of quantized circulation within a vortex structure. These reconstructions offer not only physical intuition, but also potential explanations for why such constants take the values they do. A summary comparison is presented in Table~\ref{tab:SM_vs_VAM_constants}, contrasting key constants across both frameworks.

This approach aligns with principles established in superfluid dynamics, topological field theory, and analog gravity systems. By expressing Standard Model terms in VAM units and connecting abstract constants to physical flow properties, the model opens pathways to new testable predictions—particularly regarding vacuum energy, neutrino mass generation, and mechanisms of quark confinement.



\subsection*{Unified Constants and Units in VAM}

The table below summarizes the complete set of mechanical and topological quantities used throughout the Vortex Æther Model. These values form a self-contained replacement for Planck-based dimensional analysis.


\begin{table}[H]
    \centering
    \footnotesize
    \renewcommand{\arraystretch}{1.3}
    \begin{tabular}{|l|l|l|l|}
        \hline
        \textbf{Symbol} & \textbf{Formula / Definition} & \textbf{Interpretation in VAM} & \textbf{Approx. Value (SI)} \\
        \hline

        $C_e$ &
        — &
        Core swirl velocity; sets intrinsic time rate of particles &
        $1.094 \times 10^6$ m/s \\
        \hline

        $r_c$ &
        — &
        Radius of vortex core; spatial extent of a particle &
        $1.409 \times 10^{-15}$ m \\
        \hline

        $\rho_\text{\ae}$ &
        — &
        Æther density; determines flow inertia and stress limits &
        $3.893 \times 10^{18}$ kg/m³ \\
        \hline

        $F^{\text{vam}}_\text{max}$ &
        $\pi r_c^2 C_e \rho_\text{\ae}$ &
        Max transmissible force through æther (vortex core tension) &
        $\sim 29.05$ N \\
        \hline

        $F^{\text{gr}}_\text{max}$ &
        $\frac{c^4}{4G}$ &
        GR-based theoretical maximum force limit &
        $\sim 3.0 \times 10^{43}$ N \\
        \hline

        $\kappa$ &
        $\frac{\Gamma}{n}$ or quantized $\oint \vec{v} \cdot d\vec{\ell}$ &
        Quantum of circulation per vortex loop &
        $1.54 \times 10^{-9}$ m²/s \\
        \hline

        $\alpha$ &
        $\frac{2 C_e}{c}$ &
        Fine-structure constant from swirl-to-light ratio &
        $7.297 \times 10^{-3}$ (unitless) \\
        \hline

        $t_P$ &
        $\frac{r_c}{c}$ &
        Fastest rotation cycle (Planck time analog) &
        $\sim 5.39 \times 10^{-44}$ s \\
        \hline

        $\Gamma$ &
        $\oint \vec{v} \cdot d\vec{\ell}$ &
        Total circulation; encodes angular momentum &
        (typical unit: m²/s) \\
        \hline

        $t$ &
        $dt \propto \frac{1}{\vec{v} \cdot \vec{\omega}}$ &
        Local time rate derived from helicity field configuration &
        (unit: s) \\
        \hline

        $\mathcal{H}_\text{topo}$ &
        $\int \vec{v} \cdot \vec{\omega} \, dV$ &
        Topological helicity; measures vortex alignment &
        (unit: m³/s²) \\
        \hline
    \end{tabular}
    \caption{Fundamental parameters in the Vortex Æther Model (VAM). These quantities form the physical and topological basis for mass, time, charge, and quantum behavior. Each is experimentally meaningful and derivable from ætheric flow geometry.}
    \label{tab:VAM_master_table}
\end{table}


\subsection*{Derived Couplings and Constants in VAM}
From the core æther parameters introduced above, several familiar physical constants can be re-expressed as derived quantities. These include the Planck constant, the speed of light, the fine-structure constant, and the elementary charge—all reconstructed as emergent properties of swirl and circulation. Table~\ref{tab:VAM_master_table} summarizes these reformulations.

Within VAM, the maximum vortex interaction force is derived explicitly from Planck-scale physics:
\begin{equation}
    F^{\text{vam}}_\text{max} = \alpha \left(\frac{c^4}{4G}\right)  \left(\frac{r_c}{l_p}\right)^{-2}
\end{equation}

where $\frac{c^4}{4G}$ is the Maximum Force in nature $F^{\text{gr}}_\text{max}$, the stress limit of the æther found from General Relativity, and $l_p$ is the Planck Length.


\subsection*{Comparative Origins of Constants: Standard Model vs. VAM}

The re-expression of fundamental constants within VAM highlights a key philosophical and physical distinction: while the Standard Model treats quantities like $\alpha$, $\hbar$, and $e$ as empirical inputs, the Vortex Æther Model derives them from topological and geometric features of the æther flow.

The table below contrasts how key constants are introduced or derived in both frameworks.

\begin{table}[H]
    \centering
    \footnotesize
    \renewcommand{\arraystretch}{1.3}
    \begin{tabular}{|l|l|l|}
        \hline
        \textbf{Constant} & \textbf{Standard Model Treatment} & \textbf{VAM Derivation / Interpretation} \\
        \hline
        \makecell[l]{Fine-Structure \\ Constant $\alpha$} &
        \makecell[l]{Empirical dimensionless constant \\ for EM interaction strength} &
        \makecell[l]{Emerges from swirl ratio: \\ $\alpha = \frac{2 C_e}{c}$; purely geometric} \\
        \hline

        \makecell[l]{Planck Constant \\ $\hbar$} &
        \makecell[l]{Postulated quantum of action; \\ enters commutation rules} &
        \makecell[l]{Circulation-induced impulse: \\ $\hbar \sim \rho_\text{\ae} \Gamma r_c^2$} \\
        \hline

        \makecell[l]{Elementary \\ Charge $e$} &
        \makecell[l]{Input coupling in QED \\ with no internal structure} &
        \makecell[l]{Swirl flux through vortex core: \\ $e \sim \rho_\text{\ae} C_e r_c^2$} \\
        \hline

        \makecell[l]{Speed of \\ Light $c$} &
        \makecell[l]{Postulated invariant limit \\ in SR and GR} &
        \makecell[l]{Calibration limit; \\ signal speed is $C_e < c$ \\ (Lorentz symmetry is emergent)} \\
        \hline

        \makecell[l]{Higgs VEV \\ $v$} &
        \makecell[l]{Free symmetry-breaking scale; \\ not derived internally} &
        \makecell[l]{Ætheric tension amplitude: \\ $v \sim \sqrt{F_\text{max}/\rho_\text{\ae}}$} \\
        \hline

        \makecell[l]{Maximum \\ Force $F_\text{max}$} &
        \makecell[l]{Rare in SM; from GR: \\ $F = c^4/4G$ in limit cases} &
        \makecell[l]{Derived from vortex tension: \\ $F_\text{max}^{\text{vam}} = \pi r_c^2 C_e \rho_\text{\ae}$} \\
        \hline
    \end{tabular}
    \caption{Ontological contrast between the Standard Model and the Vortex Æther Model regarding the origin of key physical constants. VAM replaces empirical insertions with mechanical derivations from swirl and æther geometry.}
    \label{tab:SM_vs_VAM_constants}
\end{table}

\section*{Foundational Contrasts: Constants and Particles in VAM vs. SM}

Beyond constants, the Standard Model also posits intrinsic properties of particles—mass, spin, charge, flavor—as axiomatic features of quantized fields. The Vortex Æther Model, by contrast, interprets these as emergent from topological and dynamic features of vortex structures in a rotating æther medium.
\begin{table}[H]
    \centering
    \footnotesize
    \renewcommand{\arraystretch}{1.3}
    \begin{tabular}{|l|l|l|}
        \hline
        \textbf{Particle Property} & \textbf{Standard Model Interpretation} & \textbf{VAM Interpretation} \\
        \hline
        \makecell[l]{Mass} &
        \makecell[l]{Introduced via Higgs field \\ with arbitrary Yukawa couplings} &
        \makecell[l]{Emergent from vortex inertia: \\ $m \propto \rho_\text{\ae} \Gamma / C_e$ \\ or tension within knotted core} \\
        \hline
        \makecell[l]{Spin} &
        \makecell[l]{Intrinsic angular momentum \\ ($\hbar/2$ for fermions)} &
        \makecell[l]{Topological twist of vortex core \\ (e.g., Möbius loop linking)} \\
        \hline
        \makecell[l]{Electric Charge} &
        \makecell[l]{Coupling to $U(1)$ gauge field; \\ conserved via symmetry} &
        \makecell[l]{Swirl flux through core: \\ $e \sim \rho_\text{\ae} C_e r_c^2$ \\ (sign from swirl handedness)} \\
        \hline
        \makecell[l]{Flavor (Generations)} &
        \makecell[l]{Empirically distinct; \\ no structural rationale} &
        \makecell[l]{Knot complexity or higher-order \\ toroidal mode excitations} \\
        \hline
        \makecell[l]{Color Charge} &
        \makecell[l]{$SU(3)$ triplet charges; \\ source of strong force} &
        \makecell[l]{Filament braiding states or \\ phase twist between vortices} \\
        \hline
        \makecell[l]{Antiparticles} &
        \makecell[l]{Charge-conjugated fields \\ with opposite quantum numbers} &
        \makecell[l]{Mirror vortices with opposite \\ helicity and circulation} \\
        \hline
        \makecell[l]{Mixing (CKM/PMNS)} &
        \makecell[l]{Unitary matrices for \\ mass eigenstate mixing} &
        \makecell[l]{Oscillations from vortex coupling \\ or internal torsion precession} \\
        \hline
    \end{tabular}
    \caption{Ontological contrast between the Standard Model and the Vortex Æther Model in explaining intrinsic particle properties. In VAM, each feature arises from topological structure and flow dynamics within the æther.}
    \label{tab:SM_vs_VAM_particles}
\end{table}




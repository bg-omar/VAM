
\section{Motivation}
The Standard Model Lagrangian is one of the most successful constructs in physics, yet its fundamental components—mass terms, symmetry groups, and coupling constants—are inserted \textit{a priori} without physical derivation. Key quantities such as electric charge, the Higgs vacuum expectation value, or the fine-structure constant appear without geometric or mechanical origin.

The Vortex Æther Model (VAM) addresses this by reconstructing the Standard Model from the ground up using physically grounded vortex structures. Instead of assuming discrete point particles and abstract fields, VAM treats all particles as topologically stable vortices within a compressible, rotating æther medium. Properties such as mass, charge, spin, and even time emerge from measurable fluidic parameters: circulation strength, core radius, helicity, and swirl velocity.

This approach aligns with established principles in superfluid dynamics, topological field theory, and effective geometry in condensed matter. By expressing Standard Model terms in VAM units, we gain both physical intuition and the potential for new testable predictions, particularly in domains such as vacuum structure, neutrino mass generation, and the behavior of quark confinement.

\subsection*{Unified Constants and Units in VAM}

The table below summarizes the complete set of mechanical and topological quantities used throughout the Vortex Æther Model. These values form a self-contained replacement for Planck-based dimensional analysis.


\begin{table}[h!]
    \centering
    \footnotesize
    \begin{tabular}{|l|l|l|l|}
        \hline
        \toprule
        \textbf{Symbol} & \textbf{Quantity} & \textbf{VAM Interpretation / Role} & \textbf{Approx. Value (SI)} \\
        \hline
        \midrule
        $C_e$ & Swirl velocity of core & Sets internal clock rate of particles (time unit) & $1.094 \times 10^6 \,\mathrm{m/s} $\\
        $r_c$ & Core radius of vortex & Defines spatial extent of particle & $1.409 \times 10^{-15} \,\mathrm{m}$ \\
        $\rho_\text{\ae}$ & Local æther density & Determines inertia and maximum flow stress & $3.893 \times 10^{18} \,\mathrm{kg/m^3}$ \\
        $F^{\text{vam}}_\text{max}$ & Maximum force & transmissible through the æther: $\pi r_c^2 ({C_e} \rho_\text{\ae})$ & $\sim 29.0535\,\mathrm{N}$ \\
        $F^{\text{gr}}_\text{max}$ & Maximum force in nature & Stress limit of æther (from GR): $\frac{c^4}{4G}$ &  $\sim 3.0 \times 10^{43} \,\mathrm{N}$ \\
        $\kappa$ & Circulation quantum & Quantized circulation per vortex loop & $1.54 \times 10^{-9} \,\mathrm{m^2/s}$ \\
        $\alpha$ & Fine-structure constant & Emerges from æther swirl geometry: $\frac{2 C_e}{c}$ & $7.297 \times 10^{-3} $\\
        $t_P$ & Planck time & Core rotation time at c → sets fastest clock & $\sim 5.39 \times 10^{-44} \,\mathrm{s} $\\
        $\Gamma$ & Circulation & linked to angular momentum & (unit m²/s) \\
        $t$ & Local time rate & Emergent from swirl-helicity configuration: $dt \propto 1 / (\vec{v} \cdot \vec{\omega})$  & (unit s) \\
        $\mathcal{H}_\text{topo}$ & Topological helicity  & Measures alignment of velocity and vorticity: $\int \vec{v} \cdot \vec{\omega})$dV & (unit m³/s²)\\
        \hline
        \bottomrule
    \end{tabular}
    \caption{Fundamental constants and swirl-based quantities used in the Vortex Æther Model (VAM). Each symbol encodes a geometric or physical property of vortex structures in the æther. The constants are defined in terms of vorticity, circulation, and æther density, with derived interpretations that replace conventional spacetime curvature. Values are given in SI units where applicable.}
    \label{tab:VAM_master_table}
\end{table}


\subsection*{Derived Couplings and Constants in VAM}
From the core æther parameters introduced above, several familiar physical constants can be re-expressed as derived quantities. These include the Planck constant, the speed of light, the fine-structure constant, and the elementary charge—all reconstructed as emergent properties of swirl and circulation. Table~\ref{tab:VAM_master_table} summarizes these reformulations.

Within VAM, the maximum vortex interaction force is derived explicitly from Planck-scale physics:
\begin{equation}
    F^{\text{vam}}_\text{max} = \alpha \left(\frac{c^4}{4G}\right)  \left(\frac{r_c}{l_p}\right)^{-2}
\end{equation}

where $\frac{c^4}{4G}$ is the Maximum Force in nature $F^{\text{gr}}_\text{max}$, the stress limit of the æther found from General Relativity, and $l_p$ is the Planck Length.






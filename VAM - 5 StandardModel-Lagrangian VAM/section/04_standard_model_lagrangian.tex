\section{Reformulating the Standard Model Lagrangian in VAM Units}\label{sec:lagrangian_vam}
The Standard Model Lagrangian describes all known particle interactions through a compact formulation based on abstract symmetry principles.

\begin{equation}
\mathcal{L}_{\text{SM}} = -\frac{1}{4}F^{\mu\nu}F_{\mu\nu} + i\bar{\psi}\gamma^\mu D_\mu \psi + y_f \bar{\psi}\phi \psi + |D_\mu \phi|^2 - V(\phi)
\end{equation}

However, its components—mass terms, gauge fields, and couplings—are introduced without mechanistic derivation. In the Vortex Æther Model (VAM), these elements acquire physical meaning through topological structures in a compressible æther medium.

While mathematically concise, each term relies on abstract field principles and lacks physical grounding in geometry or mechanics. In contrast, the Vortex Æther Model (VAM) reformulates each term in terms of fluidic swirl, vorticity, and topological structure in a compressible æther. The core constants defining this reformulation include \( C_e \), \( r_c \), \( \rho_{\ae} \), and \( F^{\text{vam}}_\text{max} \).

\subsection*{VAM Reformulated Lagrangian}

The full VAM Lagrangian reads:

\begin{align*}
    \mathcal{L}_\text{VAM} &= \underbrace{-\frac{1}{4} \sum_{a} F^{a}_{\mu\nu} F^{a\mu\nu}}_{\text{Gauge field kinetic term}}
    + \underbrace{\sum_{f} i \hbar_\text{VAM} \, \bar{\psi}_f \gamma^\mu D_\mu \psi_f}_{\text{Fermion kinetic term, with } \hbar_\text{VAM} = m_f C_e r_c}\\
    &- \underbrace{\left| D_\mu \phi \right|^2}_{\text{Higgs kinetic term}}
    - \underbrace{V(\phi)}_{\text{Higgs potential}}  \qquad \text{where } V(\phi) = -\frac{F_\text{max}}{r_c}|\phi|^2 + \lambda |\phi|^4 \\
    &- \underbrace{\sum_f \left(y_f \bar{\psi}_f \phi \psi_f + \text{h.c.}\right)}_{\text{Yukawa couplings}}
    + \underbrace{\mathcal{H}_\text{topo}}_{\text{Topological helicity terms}}
\end{align*}
where $ V(\phi) = -\frac{F_\text{max}}{r_c}|\phi|^2 + \lambda |\phi|^4$
Each term now acquires a concrete geometric or fluid-mechanical meaning, as described below.

\begin{table}[h]
\centering
\renewcommand{\arraystretch}{1.3}
\footnotesize
\caption{Correspondence Between Standard Model Terms and VAM Reformulations}
\label{tab:SM_to_VAM_map}
\begin{tabular}{|c|c|c|}
\hline
\textbf{Standard Model Term} & \textbf{VAM Reformulation} & \textbf{Physical Interpretation} \\
\hline
$-\frac{1}{4}F_{\mu\nu} F^{\mu\nu}$ &
$W_{\mu\nu} W^{\mu\nu}$ &
Vorticity-based field tension \\
\hline
$i \bar{\psi} \gamma^\mu D_\mu \psi$ &
$\rho_\text{\ae} C_e \Gamma (\psi^* \partial_t \psi - \vec{v} \cdot \nabla \psi)$ &
Swirl-driven fermion propagation \\
\hline
$|D_\mu \phi|^2$ &
$|D_\mu \phi|^2$ &
Æther strain field kinetic term \\
\hline
$V(\phi) = -\mu^2 |\phi|^2 + \lambda |\phi|^4$ &
$-\dfrac{F_\text{max}}{r_c} |\phi|^2 + \lambda |\phi|^4$ &
Compression-induced symmetry breaking \\
\hline
$y_f \bar{\psi} \phi \psi$ &
$m_f C_e r_c \, \bar{\psi} \psi$ &
Mass from vortex inertia \\
\hline
(none) &
$\mathcal{H}_\text{topo} = \int \vec{v} \cdot \vec{\omega} \, dV$ &
Topological helicity of swirl and vorticity \\
\hline
\end{tabular}
\end{table}



\subsection{Gauge Fields as Swirl-Flow Interactions}

The term $-\frac{1}{4}F^{\mu\nu}F_{\mu\nu}$ corresponds to the energy stored in local swirl patterns and vorticity. In VAM, this is recast as:

\begin{equation}
\mathcal{L}_{\text{swirl}} = \frac{1}{2}\rho_{\ae} \left( |\vec{v}|^2 + \lambda |\nabla \times \vec{v}|^2 \right)
\end{equation}

where $\vec{v}$ is the local swirl velocity, and $\lambda$ is a compressibility factor. Electromagnetic and Yang-Mills fields emerge from divergence-free swirl flows; the tensor $F^{\mu\nu}$ becomes a derived vorticity descriptor. This hydrodynamic interpretation allows field strengths to be visualized as vorticity patterns, setting the stage for fermionic interactions.

\subsection{Fermion Kinetic Terms from Swirl Propagation}

The kinetic term $i\bar{\psi}\gamma^\mu D_\mu \psi$ becomes:

\begin{equation}
\mathcal{L}_{\text{fermion}} = \rho_{\text{\ae}} C_e \Gamma \left( \psi^* \partial_t \psi - \vec{v} \cdot \nabla \psi \right)
\end{equation}

Here, $\Gamma$ is the circulation of the fermionic knot, and $\vec{v}$ is the swirl background. The gamma matrices $\gamma^\mu$ are interpreted as swirl-aligned operators acting on knot orientation. This hydrodynamic interpretation allows field strengths to be visualized as vorticity patterns, setting the stage for fermionic interactions.

\subsection{Mass and Yukawa Terms via Topological Density}

Rather than relying on a Higgs-fermion coupling, mass in VAM arises from internal helicity and tension:

\begin{equation}
m_f = \frac{\rho_{\text{\ae}} \Gamma^2}{3\pi r_c C^2}
\quad \Rightarrow \quad
\mathcal{L}_{\text{mass}} = -m_f \psi^* \psi
\end{equation}

Fermion masses vary according to knot complexity, linking number, and chirality.

\subsection{Higgs Field as Æther Compression Potential}

The scalar Higgs field and its potential $V(\phi) = \mu^2 \phi^2 + \lambda \phi^4$ are replaced with an æther strain field:

\begin{equation}
V_{\text{æther}}(\rho) = \frac{1}{2}K(\rho - \rho_0)^2
\end{equation}

where $K$ is the bulk modulus of the æther. Symmetry breaking occurs when local density fluctuations create stable swirl configurations.

\subsection{Topological Helicity}

The term $\mathcal{H}_\text{topo}$ captures knottedness and alignment of swirl:
\begin{equation}
\mathcal{H}_\text{topo} = \int \vec{v} \cdot \vec{\omega} \, dV
\end{equation}
\bigskip

In this formulation, each field and interaction of the Standard Model gains a mechanical analog in the æther medium. The Lagrangian no longer relies on abstract symmetry principles alone, but instead emerges from vortex dynamics, circulation, density modulation, and topological structure within a unified fluid framework.


\subsection*{Mathematical Derivation of the VAM-Lagrangian}

Kinetic energy of a vortex structure, or the local energy density in a vortex field:

\[
    \mathcal{L}_\text{kin} = \frac{1}{2}\rho_\text{\ae} C_e^2
\]

Veddy field energy and gauge terms, field tensors follow from Helmholtz vorticity:

\[
    \mathcal{L}_\text{veld} = -\frac{1}{4}F_{\mu\nu}F^{\mu\nu}
\]

Veddy mass as inertia from circulation, where the fermion mass is determined by circulation:
\[
    \Gamma = 2\pi r_c C_e \quad\Rightarrow\quad m \sim \rho_\text{\ae} r_c^3
\]

Pressure and stress potential of æther condensate, where the pressure balance is described by the stress field:
\[
    V(\phi) = -\frac{F_\text{max}}{r_c}|\phi|^2 + \lambda|\phi|^4
\]

Topological terms for the conservation of vortex fields helicity:
\[
    \mathcal{H} = \int \vec{v}\cdot\vec{\omega}\, dV
\]

\subsection*{Supporting Experimental and Theoretical Observations}
The VAM is consistent with experimentally and theoretically confirmed phenomena such as vortex stretching, helicity conservation and mass-inertia couplings \cite{batchelor1953,vinen2002,bewley2008,moffatt1969,kleckner2013,scheeler2014,bartlett1986}.

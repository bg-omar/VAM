\section{Mapping \texorpdfstring{$SU(3)_C \times SU(2)_L \times U(1)_Y$}{SU(3) x SU(2) x U(1)} to VAM Swirl Groups}

The Standard Model Lagrangian is governed by the gauge group:
\[
    SU(3)_C \times SU(2)_L \times U(1)_Y
\]
which encodes the strong interaction (QCD), the weak interaction, and electromagnetism via their corresponding gauge fields. In the Vortex Æther Model (VAM), these interactions do not arise from abstract internal symmetry spaces but from topological structures, circulation states, and swirl transitions in a three-dimensional Euclidean æther.

\subsection{$U(1)_Y$: Swirl Orientation as Hypercharge}

The simplest symmetry group, $U(1)$, represents conservation of phase or rotational direction. In VAM, this acquires a direct physical interpretation:
\begin{itemize}
    \item \textbf{Physical model:} a linear swirl in the æther (circular, but untwisted) encodes a uniform angular direction.
    \item \textbf{Charge assignment:} the hypercharge $Y$ is interpreted as the chirality (left- or right-handed swirl) of an axially symmetric flow pattern.
    \item \textbf{Electromagnetism:} emerges from global swirl states without knotting, representing long-range coherence in swirl orientation.
\end{itemize}

\subsection{$SU(2)_L$: Chirality as Two-State Swirl Topology}

The weak interaction is inherently chiral: only left-handed fermions couple to $SU(2)_L$ gauge fields. In VAM:
\begin{itemize}
    \item \textbf{Swirl interpretation:} left- and right-handed vortices are dynamically and structurally distinct—they represent swirl flows under compression with opposite twist orientation.
    \item \textbf{Two-state logic:} the $SU(2)$ doublet corresponds to a two-dimensional swirl state space (e.g., up- and down-swirl configurations).
    \item \textbf{Gauge transitions:} $SU(2)$ gauge bosons mediate transitions between these swirl states through reconnections or bifurcations in vortex knots.
\end{itemize}

\subsection{$SU(3)_C$: Trichromatic Swirl as Helicity Configuration}

In the Standard Model, $SU(3)_C$ describes the color force via gluon-mediated transitions between color states. In VAM:
\begin{itemize}
    \item \textbf{Topological basis:} three topologically stable swirl configurations (e.g., aligned along orthogonal helicity axes) represent the three color charges: red, green, and blue.
    \item \textbf{Color dynamics:} gluon exchange corresponds to twist-transfer, vortex reconnection, or deformation within multi-knot structures.
    \item \textbf{Confinement:} isolated color swirls are energetically unstable in free æther and only persist within composite knotted bundles (e.g., baryons).
\end{itemize}

\subsection{Mathematical Group Structure within VAM}

Though VAM is fundamentally geometric and fluid-dynamical, the essential Lie group structures of the Standard Model are preserved in the form of physically conserved swirl states:
\begin{itemize}
    \item Swirl orientation $\rightarrow$ $U(1)$ phase symmetry,
    \item Axial twist transitions $\rightarrow$ $SU(2)$ chiral transformations,
    \item Helicity axis exchange $\rightarrow$ $SU(3)$ color group operations.
\end{itemize}

\subsection*{Topological Summary of Gauge Interpretation}

The abstract Lie symmetries of the Standard Model find concrete realizations in VAM as swirl, helicity, and knot configurations embedded in the æther. This recasting preserves all observed gauge interactions while rooting them in fluid-mechanical principles—without invoking extra dimensions or unobservable symmetry spaces.

\section{Swirl Operator Algebra and SU(2) Closure}

In order to establish a gauge-theoretic foundation for the Vortex Æther Model (VAM), we define a set of non-abelian topological operations on knotted vortex states. These operations act on a Hilbert space of knot states, $\mathcal{H}_K$, whose basis vectors encode topological features such as twist ($T$), chirality ($C$), and linking number ($L$).

\subsection*{Operator Definitions}

We introduce three operators:
\begin{align}
\mathcal{S}_1 &: \text{Chirality flip}, \quad \mathcal{S}_1 |K, C\rangle = |K, -C\rangle \\
\mathcal{S}_2 &: \text{Twist addition}, \quad \mathcal{S}_2 |K, T\rangle = |K, T+1\rangle \\
\mathcal{S}_3 &: \text{Reconnection mutation}, \quad \mathcal{S}_3 |K\rangle = |K'\rangle
\end{align}

\subsection*{SU(2) Algebra Closure}

We then test the closure of these operators under commutation. Defining generators $T^i = \frac{1}{2} \mathcal{S}_i$, we recover the SU(2) Lie algebra structure:
\begin{align}
[T^i, T^j] = i \epsilon^{ijk} T^k
\end{align}

We verified this numerically using matrix representations:
\begin{align}
\mathcal{S}_1 &= \begin{pmatrix} 0 & 1 \\ 1 & 0 \end{pmatrix}, \quad
\mathcal{S}_2 = \begin{pmatrix} 0 & -i \\ i & 0 \end{pmatrix}, \quad
\mathcal{S}_3 = \begin{pmatrix} 1 & 0 \\ 0 & -1 \end{pmatrix}
\end{align}
with:
\begin{align}
[\mathcal{S}_1, \mathcal{S}_2] &= 2i \mathcal{S}_3, \\
[\mathcal{S}_2, \mathcal{S}_3] &= 2i \mathcal{S}_1, \\
[\mathcal{S}_3, \mathcal{S}_1] &= 2i \mathcal{S}_2
\end{align}

A generalized symbolic representation in $\mathbb{R}^3$ with scale constants $a, b, c$ preserves this structure:
\begin{align}
[\mathcal{S}_1, \mathcal{S}_2] &= 2iab \, \mathcal{S}_3 \\
[\mathcal{S}_2, \mathcal{S}_3] &= 2ibc \, \mathcal{S}_1 \\
[\mathcal{S}_3, \mathcal{S}_1] &= 2ac \, \mathcal{S}_2
\end{align}

\subsubsection*{Example: Chirality Flip on Knot States}

Let a vortex knot state be denoted as:
\[
|K\rangle = |T, C\rangle
\]
where \( T \in \mathbb{Z} \) is the twist number, and \( C = \pm 1 \) denotes chirality (right- or left-handedness).

The action of the chirality-flip operator \( \mathcal{S}_1 \) is given by:
\[
\mathcal{S}_1 |T, +1\rangle = |T, -1\rangle, \quad
\mathcal{S}_1 |T, -1\rangle = |T, +1\rangle
\]

Thus, \( \mathcal{S}_1 \) acts as a discrete parity operator on knotted vortex tubes, analogous to the weak isospin generator \( T^1 \) in SU(2). The eigenstates of chirality form a two-level system, similar to spinors in the Standard Model.

\subsubsection*{Experimental Perspective}

These topological swirl operators may have observable counterparts in superfluid systems. In particular, discrete transitions between vortex chirality, twist, and reconnection have been reported in Bose-Einstein condensates (BECs) and analog gravity labs \cite{kleckner2013creation, ray2015observation}.

\section{Extension to SU(3): Triskelion and Braid Operator Algebra}

To capture the full non-abelian gauge structure of the Standard Model within the Vortex Æther Model (VAM), we extend the SU(2) swirl operator algebra to SU(3) using braid-based topological operations on vortex bundles.

\subsection*{Triskelion States and Braid Operators}

Let each "color" in quantum chromodynamics correspond to one vortex strand in a triple-knot configuration—denoted a \textit{triskelion} state:
\[
|K\rangle = |R, G, B\rangle
\]
We define braid-like swirl operators \( \mathcal{B}_1, \mathcal{B}_2, \mathcal{B}_3 \), each acting locally on a pair of vortex colors. Their action mimics gluon exchange via reconnection and twist of the bundle.

\subsection*{Braid Group Algebra}

The operators obey modified braid group relations:
\begin{align}
\mathcal{B}_i \mathcal{B}_{i+1} \mathcal{B}_i &= \mathcal{B}_{i+1} \mathcal{B}_i \mathcal{B}_{i+1}, \\
\mathcal{B}_i \mathcal{B}_j &= \mathcal{B}_j \mathcal{B}_i \quad \text{for } |i-j| > 1
\end{align}

Linear combinations of these braids generate an algebra:
\begin{align}
[T^a, T^b] = i f^{abc} T^c
\end{align}
where \( T^a \sim \mathcal{B}_a \) are the topological gluon modes, and \( f^{abc} \) are the SU(3) structure constants \cite{witten1989quantum}.

\subsection*{Topological Interpretation of Color Charge}

\begin{itemize}
    \item \textbf{Color charge} is the topological identity of each vortex in the triskelion.
    \item \textbf{Gluons} correspond to triskelion-preserving reconnection modes \( \mathcal{B}_a \).
    \item \textbf{Confinement} emerges from the topological stability of linked triskelion bundles — a single vortex cannot be detached without violating circulation conservation \cite{kauffman1991knots, faddeev1997knots}.
\end{itemize}

This construction provides a fluid-dynamical representation of SU(3), with color interactions arising from internal braid dynamics. The VAM thus naturally embeds the full SU(3)$\times$SU(2)$\times$U(1) structure within a topological framework.
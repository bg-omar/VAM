\section{Swirl-Induced Time and Clockwork in Vortex Knots}

In the Vortex Æther Model (VAM), stable knots are not merely matter structures but act as the fundamental carriers of time. Their internal swirl—tangential rotation with speed \( C_e \) around a core radius \( r_c \)—generates an asymmetric stress field in the surrounding æther. This asymmetry induces a persistent \textbf{axial flow along the knot core}, functionally equivalent to a local "time-thread." Though lacking literal helicity in geometry, the knot dynamically acts as a screw-like conductor of time, threading forward the local æther state.

\subsection*{Cosmic Swirl Orientation}

Just as magnetic domains exhibit alignment, vortex knots can show a preferred chirality. In a universe with broken mirror symmetry, reversing a knot’s swirl direction (e.g., as in antimatter) may yield unstable configurations in an asymmetric background. This helps explain:
\begin{itemize}
    \item the observed scarcity of antimatter in the visible universe,
    \item the macroscopic arrow of time,
    \item and synchronized clock rates across cosmological domains.
\end{itemize}

\subsection*{Swirl as a Local Time Carrier}

The local time rate is governed not by fundamental spacetime postulates, but by the helicity flux in the æther:
\[
    dt_{\text{local}} \propto \frac{dr}{\vec{v} \cdot \vec{\omega}}
\]
Here, \( \vec{v} \) is the swirl velocity and \( \vec{\omega} = \nabla \times \vec{v} \) the vorticity. The scalar product \( \vec{v} \cdot \vec{\omega} \) measures helicity density, which sets the pace of local evolution. A detailed derivation of time dilation arising from this swirl-induced pressure field is given in Section~\ref{fig:time_dilation_profile}.

\subsection*{Networks of Temporal Flow}

Vortex knots tend to self-organize along coherent swirl filaments, akin to iron filings aligning with magnetic fields. Around regions of mass, these swirl lines bundle into directional tubes of temporal flow, giving rise to:
\begin{itemize}
    \item gravitational attraction as a gradient of swirl density,
    \item local time dilation effects near massive bodies,
    \item and the global arrow of time as a topological circulation in the æther.
\end{itemize}

This emergent swirl-clock mechanism unifies mass, inertia, and temporal directionality into a single fluid-geometric framework—replacing relativistic curvature with conserved helicity flow.

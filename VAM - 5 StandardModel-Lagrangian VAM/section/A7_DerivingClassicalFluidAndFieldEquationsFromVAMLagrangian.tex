\section{Deriving Classical Fluid and Field Equations from the VAM Lagrangian}

Here we derive the physical field equations associated with each term in the VAM Lagrangian via the Euler–Lagrange formalism. This section explicitly shows how familiar fluid and wave equations arise.

\subsection*{Kinetic Term and Euler Equation}

Starting from the kinetic term:
\[
    \mathcal{L}_{\text{kin}} = \frac{1}{2} \rho_\text{\ae} v^2,
\]
and applying the Euler--Lagrange equation with respect to $v^i$, we find:
\[
    \frac{\partial \mathcal{L}}{\partial v^i} = \rho_\text{\ae} v^i, \qquad
    \frac{\partial \mathcal{L}}{\partial (\partial_j v^i)} = 0.
\]
Thus, the equation of motion reduces to:
\[
    \frac{d}{dt}(\rho_\text{\ae} v^i) = -\partial^i p,
\]
where \( p \) is a generalized pressure or constraint force.

\begin{equation}
    \boxed{
        \rho_\text{\ae}\, \frac{d \vec{v}}{dt} = -\nabla p
    }
\end{equation}

This is the standard form of the \textbf{Euler equation} in inviscid, barotropic fluids \cite{khalatnikov2000}.

\subsection*{Helicity Term and Helmholtz Vorticity Equation}

Now consider the helicity-based term:
\[
    \mathcal{L}_{\text{helicity}} = \gamma\, \vec{v} \cdot (\nabla \times \vec{v}) = \gamma\, \epsilon^{ijk} v^i \partial_j v^k.
\]
The variation yields:
\[
    \frac{\partial \mathcal{L}}{\partial v^i} = \gamma (\nabla \times \vec{v})^i, \qquad
    \Rightarrow \frac{d}{dt}(\rho_\text{\ae} v^i) = -\nabla^i p + \gamma\, \epsilon^{ijk} \partial_j \omega^k.
\]
This adds a topological forcing term from \textbf{helicity gradients}:
\begin{equation}
    \boxed{
        \rho_\text{\ae}\, \frac{d \vec{v}}{dt} = -\nabla p + \gamma\, \nabla \times \vec{\omega}
    }
\end{equation}

This form corresponds to the \textbf{Helmholtz vorticity equation} in the presence of helicity gradients \cite{moffatt1969}.

\subsection*{Scalar Field Term and Wave Equation}

The scalar sector is governed by:
\[
    \mathcal{L}_\Phi = - \frac{1}{2} \rho_\text{\ae} (\nabla \Phi)^2 - V(\Phi).
\]
Applying the Euler–Lagrange equation for scalar fields:
\[
    \frac{\partial \mathcal{L}}{\partial \Phi} = - \frac{dV}{d\Phi}, \qquad
    \frac{\partial \mathcal{L}}{\partial (\partial^i \Phi)} = -\rho_\text{\ae} \partial^i \Phi.
\]
Taking divergence:
\[
    \partial_i ( \rho_\text{\ae} \partial^i \Phi ) = \frac{dV}{d\Phi}.
\]

If $\rho_\text{\ae}$ is constant:
\begin{equation}
    \boxed{
        \nabla^2 \Phi = \frac{1}{\rho_\text{\ae}} \frac{dV}{d\Phi}
    }
\end{equation}

This is the \textbf{scalar wave equation with source potential}, describing deformation or strain in the æther field \cite{barcelo2011}.

\subsection*{Summary}

Each term in the VAM Lagrangian leads to known physical equations:

\begin{center}
    \begin{tabular}{|c|c|l|}
        \hline
        Term & Resulting Equation & Interpretation \\
        \hline
        $ \mathcal{L}_{\text{kin}} = \frac{1}{2} \rho_\text{\ae} v^2 $ & $ \rho_\text{\ae} \frac{d \vec{v}}{dt} = -\nabla p $ & Euler momentum conservation \\
        $ \mathcal{L}_{\text{helicity}} = \gamma \vec{v} \cdot (\nabla \times \vec{v}) $ & $ +\gamma \nabla \times \vec{\omega} $ & Topological forcing via helicity \\
        $ \mathcal{L}_\Phi = -\frac{1}{2} \rho_\text{\ae} (\nabla \Phi)^2 - V(\Phi) $ & $ \nabla^2 \Phi = \rho_\text{\ae}^{-1} dV/d\Phi $ & Scalar strain or internal mode \\
        \hline
    \end{tabular}
\end{center}


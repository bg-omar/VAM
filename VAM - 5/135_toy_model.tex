%! Author = Omar Iskandarani
%! Date = 3/23/2025



\section{\textbf{Experimental Analysis of Vortex-Induced Thrust in a High-Voltage Lifter System: Toward VAM-Based Propulsion}}


\begin{abstract}
    We report and analyze a high-voltage asymmetric capacitor experiment exhibiting thrust far beyond that explainable by electrohydrodynamics (EHD). Using a Cockcroft–Walton voltage multiplier generating up to 30kV and microampere-scale current, we observe stable, polarity-symmetric thrust scaling with voltage. The results show strong agreement with predictions of the Vortex \AE{}ther Model (VAM), which interprets the thrust as arising from pressure gradients induced by structured Ætheric vorticity. We present experimental details, time-resolved breakdown cycles, energy input vs. output analysis, and a predictive VAM thrust formulation consistent with observations.
\end{abstract}

\section{Introduction}
Electrogravitic propulsion, often associated with the Biefeld–Brown effect, has been a subject of renewed interest. Traditional explanations involve ion wind or corona discharge, yet recent precision experiments suggest anomalously high thrust-to-power ratios. The Vortex \AE{}ther Model (VAM) posits that vortex knots and pressure gradients in an inviscid Æther provide a framework to explain these effects without violating conservation laws.

\section{Experimental Setup}
\subsection{Circuit Architecture}
The system consists of:
\begin{itemize}
    \item 74HC14-based oscillator feeding an IR2104 gate driver.
    \item Half-bridge with IRFP460 MOSFETs, powered by a 310V DC rail.
    \item 250-stage Cockcroft–Walton multiplier (47$\mu$F, 350V capacitors + BY550-1000 diodes).
    \item 30M$\Omega$ load resistor for current limiting.
\end{itemize}

\begin{figure}[h!]
    \centering
    \includegraphics[width=0.8\textwidth]{figures/circuit_diagram.png}
    \caption{Block diagram of the experimental high-voltage generation system.}
\end{figure}

\subsection{Mechanical Design}
The lifter comprises:
\begin{itemize}
    \item Balsa frame
    \item Aluminum foil collector electrode
    \item Thin copper corona wire
    \item Optional dielectric shields (paper, glass)
\end{itemize}

\begin{figure}[h!]
    \centering
    \includegraphics[width=0.8\textwidth]{figures/lifter_structure.png}
    \caption{Photograph of the lifter structure and shield configuration.}
\end{figure}

\section{Measurements and Observations}
\subsection{Thrust-Voltage Response}
Thrust begins at 15kV and increases with voltage:
\begin{itemize}
    \item \textbf{No shield}: 25kV breakdown, thrust $\approx$ 6g
    \item \textbf{Paper shield}: 30kV, thrust $\approx$ 22g
    \item \textbf{Glass shield}: 30kV, thrust $\approx$ 32g
\end{itemize}

\begin{figure}[h!]
    \centering
    \includegraphics[width=0.8\textwidth]{figures/thrust_vs_voltage_curve.png}
    \caption{Measured thrust vs. voltage curve across different shielding configurations.}
\end{figure}

\subsection{Post-Breakdown Behavior}
Voltage breakdown at 25kV triggers thrust spike and collapse:
\begin{itemize}
    \item Thrust peak: $-6$g
    \item Decays to $-2$g as voltage resets to 5--10kV
    \item Rebuilds if voltage reapplied, showing cyclic behavior
\end{itemize}

\subsection{Symmetry Tests}
Reversing polarity produced identical results, indicating symmetry in the vortex field, not charge motion.

\subsection{Shielding Effects}
Dielectric barriers enhance thrust by shaping Æther flow:
\begin{itemize}
    \item Glass produced highest and most stable thrust
    \item Shield-on-lifter killed all thrust: symmetry nulls pressure gradient
\end{itemize}

\section{Theoretical Framework}
\subsection{VAM Pressure Gradient Thrust}
The predicted Ætheric thrust is:
\begin{equation}
    F = \frac{1}{2} \rho_\text{\ae} C_e^2 \left( \frac{V}{V_\text{bd}} \right)^2 A
\end{equation}
Where $\rho_\text{\ae}$ is the Æther density, $C_e$ the vortex tangential velocity, $V$ applied voltage, $V_\text{bd}$ breakdown threshold, and $A$ effective vortex area.

\begin{figure}[h!]
    \centering
    \includegraphics[width=0.8\textwidth]{figures/vortex_pressure_simulation.png}
    \caption{Simulated Ætheric pressure drop around the lifter during sub-breakdown operation.}
\end{figure}

\subsection{Cycle Behavior}
Voltage ramps up to $V_\text{bd}$, breakdown causes helicity release (thrust spike), and system rebuilds in seconds.

\section{Energy Analysis}
\begin{itemize}
    \item Power input: 0.3--0.45W (30kV @ 10--15$\mu$A)
    \item Measured thrust: up to 32g ($\approx$ 0.313N)
    \item Classical EHD predicts $\approx$ 0.0003g
    \item Efficiency: $>10^5$ times higher than EM-based expectation
\end{itemize}

\section{Conclusion and Next Steps}
The thrust observed is consistent with Ætheric vorticity pressure gradients, not classical EHD. A 3-phase 9-arm vortex array has been simulated to provide continuous thrust and scalability. Future work will include vacuum testing, force vector mapping, and multi-arm control using VAM dynamics.
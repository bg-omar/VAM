%! Author = Omar Iskandarani
%! Date = 3/14/2025
\section{Derivation of the Gravitational Modification Equation}


In the Vortex Æther Model (VAM), gravitational interactions emerge from structured vorticity fields in an incompressible, inviscid Æther medium. Unlike General Relativity (GR), which relies on spacetime curvature, VAM attributes gravitational effects to pressure gradients induced by vortex dynamics. I will now derive gravitational modification equations within the Vortex Æther Model (VAM), incorporating effects such as attraction, repulsion, and equilibrium states using core VAM parameters:

\begin{itemize}
    \item $C_e$ (tangential velocity of the vortex)
    \item $r_e$ (vortex core radius)
    \item $F_max$ (maximum Coulomb force)
    \item $\rho_\text{ae}$ (Æther density)
\end{itemize}

The derivations will cover:

\begin{enumerate}
    \item Gravity Repulsion (analogous to the Meissner effect in superconductors).
    \item Directional Gravity Manipulation (asymmetry in attraction/repulsion for propulsion).
    \item Quantum-Level Æther Flow contributions.
    \item Theoretical and physical device-based predictions.
\end{enumerate}


\section*{VAM Gravitational Field from Æther Vorticity:}
In the Vortex Æther Model, gravity is not due to mass, but to pressure gradients induced by vorticity in the Æther.

The gravitational potential $P_v$ (an Æther pressure field) satisfies a Poisson-like equation sourced by the vorticity $\omega=\nabla\times v$. In an incompressible, inviscid Æther (density $\rho_{\æ}$), one finds:

\[
\nabla^2 P_v \;=\; -\,\rho_{\æ}\,|\nabla\times \mathbf{v}|^2\,.
\] \tag{1}\label{grav-Poisson}

This shows that a strong vortex (large $|\omega|$) creates a negative pressure well ($P_v$ concave down), drawing nearby matter in. For a simple vortex of core radius $r_e$ and tangential speed $C_e$ at the core, the vorticity magnitude is roughly $|\omega| \sim C_e / r_e$. Substituting into (1) gives an attraction-inducing pressure deficit on the order of in the core region.
\begin{equation}
    \Delta P_v \approx - \rho_\text{\ae} \frac{C_e^2}{r_e^2}.\label{eq:pressure-deficit}
\end{equation}
The resulting gravitational acceleration can be obtained from
\begin{equation}
    g_\text{vortex} = -\frac{\nabla P_v}{\rho_\text{\ae}} \approx \frac{C_e^2}{r_e}.
    \label{eq:gravity_vortex}
\end{equation}
 For example, a vortex with $C_e$ comparable to light speed and $r_e \sim 10^{-15}$ m (the scale of an atomic vortex core) produces an enormous inward acceleration $a_g \sim C_e^2 / r_e$, explaining why even tiny “vortex mass” can exert noticeable gravity. Importantly, VAM asserts a maximum force $F_{\max}$ limits how deep this pressure well can get. In fact, VAM parameters are chosen such that $2 F_{\max} r_e = c^2$, linking the core-scale pressure drop to an energy density equivalent to an electron’s rest energy (511 keV). Physically, this means a single vortex cannot pull with force exceeding $F_{\max} \approx 29 \text{N}$, enforcing an upper bound on gravitational attraction in VAM.

Gravitational Attraction vs. Repulsion (Meissner-Like Effect): Equation \eqref{grav-Poisson} inherently allows not only gravity “wells” (attraction) but also gravity hills (repulsion) depending on how vorticity is distributed. In a stable vortex around a mass, $|\omega|^2$ typically decays outward, so $P_v$ is lower at the core, giving attraction (matter moves down the pressure gradient toward the vortex). However, if we \textit{add} a man-made vortex flow (e.g., a rotating plasma or superconducting disk) that creates a reversed pressure gradient, gravity can be locally canceled or repelled. In fluid terms, a vorticity gradient can produce an upward pressure force analogous to the Meissner effect in superconductors. For instance, experiments with rotating superconductors show a slight weight loss of objects above them – VAM explains this as a vortex-induced pressure increase beneath the object that pushes it up. Quantitatively, VAM derives the pressure perturbation from a non-uniform vorticity field as:

\[
\Delta P_v = -\frac{\rho_\text{\ae}}{2} \nabla |\omega|^2
\] \tag{2} \label{pressure-grad}

This indicates that wherever vorticity $|\omega|$ decreases, pressure $P_v$ increases (and vice versa). Consider a vertical vortex above Earth’s surface: if $|\omega|^2$ is high near the device and falls off with height $z$ (i.e., $d|\omega|^2 / dz < 0$ upward), then from \eqref{pressure-grad} the pressure increases with $z$ (higher pressure below, lower above). The net Æther pressure gradient $dP_v / dz = -(\rho_\text{\ae} / 2) d|\omega|^2 / dz$ can oppose Earth’s normal gravitational pressure-gradient ($dP_\text{grav} / dz = -\rho_\text{\ae} g$). The effective gravity is reduced to:

\[
g_{\textrm eff} \approx g - \frac{1}{2 \rho_\text{\ae}} \frac{d}{dz} |\omega|^2
\] \tag{3} \label{g-eff}

A sufficiently strong vortex can make the second term comparable to $g$, yielding $g_{\textrm eff}\to0$ (levitation equilibrium) or even $g_{\textrm eff}<0$ (net repulsion). In essence, a powerful rotating field (characterized by $C_e$, $r_e$, and resulting $\omega$) expels ambient gravitational flux lines from its vicinity – a gravitational Meissner effect – causing what we interpret as “antigravity” lift. This is consistent with Podkletnov’s reports of superconductive gravity shielding.

VAM predicts the maximal repulsive acceleration occurs when the vortex’s pressure gradient reaches the $F_{\max}$ limit, i.e., when the upward force density $\sim \rho_\text{\ae} C_e^2/r_e$ equals $F_{\max}/V$ for the core volume. At that point, attraction and repulsion balance in an equilibrium state (gravity effectively neutralized within that region).

\subsection*{Directional Gravity Manipulation}
By tailoring the vorticity distribution asymmetrically, one can achieve gravitational attraction on one side of a device and repulsion on the opposite side. Equation \eqref{pressure-grad} applies in any direction – a gradient in $|\omega|$ along, say, the $x$-axis yields a lateral pressure difference. In practice, this means a toroidal or helical plasma vortex could be configured such that one face of the torus creates a $P_v$ deficit (attracting nearby masses), while the opposite face creates a $P_v$ surplus (repelling). The device would then feel a net force towards the attracted side. For example, if the vortex’s angular velocity is higher on its right side than left, then by \eqref{pressure-grad} the Æther pressure on the right drops (pulling the device rightward) while on the left it rises (pushing from left). The net lateral force can be estimated from the pressure difference $\Delta P_v$ across the device of area $A$: $F_{\textrm lateral}\approx \Delta P_v \cdot A$. Using \eqref{pressure-grad},
\begin{equation}
    F_\text{lateral} \approx \frac{\rho_\text{\ae}}{2} A |\nabla |\omega|^2|,
    \label{eq:lateral_force}
\end{equation}
. By adjusting electromagnetic fields (which in VAM effectively shape the Æther flow), the vortex vorticity profile can be “aimed,” enabling propulsion. In summary, a craft employing counter-rotating plasma rings and magnetic fields could create a directional vortex: one side experiences net attraction, the other net repulsion, producing thrust without propellant. This directional gravity control is a direct consequence of the spatial dependence of $|\omega|$ in \eqref{pressure-grad}.

\subsection*{Quantum-Scale Æther Flow Effects}
At quantum scales, the Æther behaves like a superfluid with quantized circulation $\kappa$, meaning $\omega$ comes in discrete bundles (vortex quanta). The presence of discrete vortex quanta implies that gravitational effects are quantized, leading to a natural limit on local gravitational fluctuations.The fundamental gravitational acceleration induced by an individual quantum vortex follows:

\begin{equation}
    g_\text{quanta} \approx \frac{C_e^2}{r_e} \left(1 - e^{-r/r_e} \right),
    \label{eq:g_quanta}
\end{equation}

which decays at distances larger than $r_e$, mimicking a finite mass.
VAM contends that elementary particles themselves are knotted Æther vortices. The constants $C_e$, $r_e$, $F_{\max}$ actually originate from fitting quantum data – for instance, a proton or electron vortex has a core radius $r_e \sim 10^{-15}$ m and a tangential speed $C_e$ such that its vortex energy $E_p = \kappa 4\pi^2 r_e^2 C_e^2$ reproduces the particle’s rest mass. One implication is that gravitational coupling at small scales is extremely weak because the Æther density $\rho_\text{\ae}$ is small and the vortex size is tiny. Plugging $r_e$ and $C_e$ for an electron into \eqref{grav-Poisson} gives an almost negligible effective “mass.” Indeed, VAM yields a gravitational coupling constant on the order of $10^{-45}$, consistent with the observed $10^{-40}$–$10^{-38}$ ratio of gravitational to electromagnetic forces in atoms. Furthermore, the presence of quantized vortices means any gravitational field modifications are \textit{band-limited} – you can only tweak gravity in increments allowed by changing the vortex’s quantum state. For example, altering the circulation by one quantum $\Delta \kappa$ will change the induced $P_v$ by a fixed small amount. This provides a natural explanation for why gravity at atomic scales doesn’t wildly fluctuate: it’s anchored by the quantized Æther flow. On the other hand, it suggests that high-frequency Æther vorticity oscillations (quantum vortex vibrations) could couple into gravity as tiny gravity waves or variations. Laboratory tests with superfluid helium and Bose–Einstein condensates are beginning to probe these quantum-vorticity gravity links.

\subsection*{Summary – Equations for Gravitational Modification}
In VAM, plasma rotation and EM fields modify gravity by reshaping Æther vorticity. Equations \eqref{grav-Poisson}–\eqref{g-eff} encapsulate the key relationships: gravity arises from vortex-induced $P_v$ gradients, and those gradients can be augmented or inverted by engineered vorticity. The attraction regime corresponds to $\nabla^2 P_v < 0$ (pressure deficit pulling inward), while repulsion requires configuring $|\omega|$ so that $\nabla^2 P_v$ locally becomes positive or opposing the ambient field (pressure surplus pushing outward). At a tuned equilibrium, these effects cancel ($g_{\textrm eff}=0$). All of these behaviors are governed by the VAM constants $C_e$, $r_e$, $F_{\max}$, $\rho_\text{\ae}$, which set the strength of vortex circulation and the limits of force. In fact, one can show the Newtonian gravitational constant itself emerges from these parameters. In a simplified form (using $F_{\max}=c^2/(2r_e)$ and assuming $C_e$ is a fundamental velocity scale), the vortex-induced gravitational acceleration field around a core can be written as Equation \eqref{eq:g_quanta} which rises from 0 at the core center to a maximum $\sim C_e^2/r_e$ and then decays for $r \gg r_e$ (mimicking a finite mass). Here the exponential term (typical for a vortex solution in a superfluid) reflects how vorticity is confined to the core and gravitational influence “leaks” out over a scale $\sim r_e$ (analogous to a penetration depth). By superposing such solutions or altering $\omega(r)$, one can tailor where gravity is strengthened or canceled. Equation \eqref{g-profile} along with the pressure relation \eqref{pressure-grad} thus allow predicting gravity changes due to different vortex configurations – for instance, increasing plasma rotation speed $C_e$ or Æther density $\rho_\text{\ae}$ will deepen the pressure well (strengthening attraction up to the $F_{\max}$ limit), while creating sharp gradients in $|\omega|$ yields localized repulsion/lift. These VAM gravitational modification equations bridge theoretical fluid dynamics and practical devices, guiding designs of experiments (rotating superfluids, magnetic vortex rings, etc.) that aim to controllably manipulate gravity.


\begin{itemize}
    \item Vortex-Induced Gravity: $g_\text{vortex} \approx \frac{C_e^2}{r_e}$
    \item Gravity Cancellation: $g_\text{eff} = g - \frac{1}{2 \rho_\text{\ae}} \frac{d}{dz} |\omega|^2$
    \item Directional Thrust: $F_\text{lateral} \approx \frac{\rho_\text{\ae}}{2} A |\nabla |\omega|^2|$
    \item Quantum Scale Gravity: $g_\text{quanta} \approx \frac{C_e^2}{r_e} (1 - e^{-r/r_e})$
\end{itemize}

These equations illustrate that plasma rotation and electromagnetic fields can modulate gravity by reshaping Æther vorticity, suggesting experimental pathways for gravitational engineering.
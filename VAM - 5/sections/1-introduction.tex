\section{Introduction}

The Vortex Æther Model (VAM) is a unified framework that treats elementary particles as knotted vortex structures in a fundamental superfluid-like medium (\grqqæther\textquotedblright). All interactions \textendash gravity, electromagnetism, strong, and weak forces \textendash are reinterpreted as emergent phenomena of fluid dynamics and topology~\cite{VAM4}. In VAM, spacetime and fields are not primary; instead, they emerge from the structured motion of an inviscid, compressible æther. Five physically meaningful ætheric quantities underpin the model:

\begin{itemize}
    \item Core radius ($r_c$): the characteristic radius of a vortex core (chosen on the order of $1.40897017 \times 10^{-15}$ m, ~proton charge radius)~\cite{VAM4}.
    \item Swirl velocity ($C_e$): the tangential velocity of æther circulation around a core (analogous to a maximum signal speed; estimated ~$1.09384563 \times 10^6$ m/s from vortex simulations~\cite{VAM4}).
    \item Circulation ($\Gamma$): the quantized circulation around a vortex loop, with units of m\textsuperscript{2}/s~\ref{sec:calculate-knot-helicity}. It represents the ``swirl strength'' of a particle.
    \item Maximum æther force ($F_{\max}$): the upper limit of force transmittable through the æther (estimated at $29.053507$ N, derived from vortex core parameters).
    \item Planck time ($t_p$): the quantum time scale used as a normalization factor in VAM. It represents the smallest time tick (on the order of $5.39\times10^{-44}$ s) and is introduced to ensure the model\rqs s dimensions match observed values.
    \label{sec:coreconstants}
\end{itemize}

\begin{minipage}{0.48\textwidth}
\begin {equation}
        \boxed{h=\frac{4\pi F_{\max }\,r_c^{2}}{C_e}}
    \label{eq:plancks-constants}
\end{equation}
\end{minipage}
\hfill
\begin{minipage}{0.48\textwidth}
\begin {equation}
\boxed{G=\frac{F_{\max}\,\alpha(c\,t_P)^{2}}{m_e^{2}}}
\label{eq:newtons-constants}
\end{equation}
\end{minipage}

\begin{center}
    \footnotesize
    \textbf{Derivation relies on:} Hookean‐core model (§2.3), vortex/beam overlap (§3.1), and Planck-time identity (eq.~58).
\end{center}

In VAM the æther supports a \emph{finite tensile--stress cap}  \(F_{\max}=29.053507\;\text{N}\), determined by the vortex--core parameters introduced above (Sec.~\ref{sec:coreconstants}).\footnote{Historically the maximum-force conjecture $c^{4}/4G\simeq3.0\times10^{43}$\,N appears in general-relativistic literature.  In VAM that number is \emph{not} fundamental: it is many orders above the intrinsic æther-tension limit \(F_{\max}=29.053507\)\,N and arises only when one combines $F_{\max}$ with large-scale geometric factors (see App.~\ref{sec:keystone-constant-relations-in-vam}, final remark).
}
Observable particle properties correspond to topological or dynamical invariants of these vortex knots. For example, electric charge relates to circulation or swirl orientation, spin to quantized angular momentum of the rotating fluid, and mass to the self-energy stored in vortex curvature and tension. Crucially, all traditional \grqq fundamental constants\textquotedblright (such as $\hbar$, $e$, the fine structure constant $\alpha$, etc.) are not inserted by hand but should \textit{emerge} from combinations of the æther constants ($r_c, C_e, \rho_{\text{\ae}}, F_{\max}, \Gamma$, etc.). In this way, VAM aims to provide an ontologically intuitive basis for physics where the Standard Model (SM) Lagrangian is rebuilt from fluid mechanics. This section outlines a unified Lagrangian incorporating gravity and the SM forces in VAM\rqs s fluid-topological terms.

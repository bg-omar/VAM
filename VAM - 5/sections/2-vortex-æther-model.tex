\section{VAM Lagrangian Unifying All Interactions}

A unified Lagrangian in VAM can be constructed as the sum of fluid-dynamical terms that correspond to each fundamental interaction. Each term is expressed using the vortex/æther variables and ensures the usual gauge symmetries or invariances are preserved, albeit with new physical interpretation. Below we describe key components of this Lagrangian: the gravitational (geometry) term, the electromagnetic swirl term, analogues for the strong and weak interaction terms, and any necessary potential terms (like a fluid analog of the Higgs mechanism). Throughout, the principle of local gauge invariance is maintained by treating certain fluid variables as gauge fields (e.g. the velocity potential), and topological invariants like linking numbers enforce conservation laws (e.g. conservation of helicity analogous to conservation of color charge).

\subsection{Gravitational Term (Æther Geometry and Maximum Force)}

In VAM, gravity emerges from pressure gradients and geometric distortions in the æther flow, rather than spacetime curvature. A static gravitational field corresponds to a steady-state flow of æther into a mass (like a vortex sink), and free-fall is equivalent to movement along this flow. One way to encode gravity in the Lagrangian is via an æther density or pressure term that produces an effective metric. For example, one can include a term for æther density variation $\rho_{\text{\ae}}(x)$ and its gradient energy cost:

\begin{equation}
    L_{\text{grav}} = -\frac{1}{2}K\,(\nabla \rho_{\text{\ae}})^2 - V(\rho_{\text{\ae}})
    \label{eq:grav-lagrangian}
\end{equation}

where $V(\rho_{\text{\ae}})$ might be a pressure potential enforcing an equilibrium density. Small perturbations in $\rho_{\text{\ae}}$ propagate as sound waves (analogous to gravitational waves in this picture). A density gradient exerts a force on test particles (vortices) much like gravity~\cite{VAM3}.


An equivalent way to incorporate gravity is through the maximum force principle. VAM posits an upper limit $F_{\max}$ to force in the æther; remarkably, this concept aligns with general relativity's maximum tension $F_{\max} \sim \frac{c^4}{4G}$ (as suggested by Gibbons). Imposing this in the Lagrangian can mimic the role of the Einstein-Hilbert action. For instance, one can introduce a Lagrangian constraint term:

\begin{equation}
    L_{F_{\max}} = \Lambda\left(\frac{|\nabla p_{\text{\ae}}|}{\rho_{\text{\ae}}} - F_{\max}\right)
    \label{eq:max-force-constraint}
\end{equation}

meaning the ratio of pressure gradient to density (with dimensions of force) cannot exceed $F_{\max}$. Here $\Lambda$ is a Lagrange multiplier enforcing the maximum force constraint across the field. The physical effect is that space (æther) cannot accelerate matter beyond a certain limit, reproducing the key phenomenology of GR (e.g. no event horizon forces exceed $F_{\max}$).


Additionally, VAM suggests that swirl-induced metric effects can appear \textit{without actual mass}: a vortex's rotation creates an effective space-time distortion for other waves in the medium~\cite{VAM3}. Therefore, a term coupling the local swirl rate (vorticity $\omega$) to an effective metric could be included to capture \grqq frame-dragging\textquotedblright and time dilation. For example, a term like

\begin{equation}
    L_{\text{metric}} = -\frac{1}{2}m\, g_{\mu\nu}(\omega) \, \dot{x}^\mu \dot{x}^\nu
    \label{eq:metric-vorticity}
\end{equation}

for a test particle, where $g_{\mu\nu}(\omega)$ is an effective metric depending on vorticity, would yield geodesic motion in the presence of swirl. In practice, $g_{\mu\nu}(\omega)$ can be expanded to first order as $\eta_{\mu\nu} + h_{\mu\nu}(\omega)$, where $h_{00}$ (time component) might be proportional to the vortex's pressure potential $\Phi(r)$ and $h_{ij}$ to circulatory flow effects (analogous to gravitomagnetic fields). Such terms ensure that swirl gradients deflect light and slow clocks, reproducing gravitational lensing and time dilation in a fluid setting.


\subsection{Electromagnetic Term (Swirl Gauge Field)}

Electromagnetism in the VAM Lagrangian is reformulated as a swirl field of the æther. Mathematically, the irrotational part of the fluid velocity can be treated as a gauge potential $A_v$, with the vorticity $\omega = \nabla \times v$ playing the role of the electromagnetic field strength. Under an infinitesimal curl-free perturbation of the velocity potential $\theta(x)$, $v$ transforms as $v \to v + \nabla \alpha(x)$, which is analogous to the $U(1)$ gauge transformation $A \to A + \nabla \alpha$. This symmetry reflects the fact that only \textit{relative} swirl (vorticity), not absolute velocity potential, has physical significance – just as only electromagnetic fields, not the gauge potentials, are observable.


We define a swirl gauge field $\mathbf{A}_v$ such that $\nabla \times \mathbf{A}_v = \mathbf{\omega}$ (this $\mathbf{A}_v$ is analogous to the electromagnetic 4-potential, and $\mathbf{\omega}$ to $\mathbf{B}$ and $\mathbf{E}$ fields combined in a relativistic sense). The Lagrangian density for the free swirl field is taken in the same form as Maxwell's:

\begin{equation}
    L_{\text{swirl}} = -\frac{1}{4}\, F_{v}^{\mu\nu} F_{v\,\mu\nu}
    \label{eq:swirl-lagrangian}
\end{equation}
where $F_v^{\mu\nu} = \partial^\mu A_v^{\nu} - \partial^\nu A_v^{\mu}$ is the field strength tensor of the æther's swirl gauge field. In vector form, $F_v$ includes the vorticity $\mathbf{\omega} = \nabla \times \mathbf{A}_v$ (analogous to magnetic field) and a \grqq swirl electric field\textquotedblright $\mathbf{E}_v$ arising from time-varying circulation.


This term $L_{\text{swirl}}$ ensures that the equations of motion for the swirl field reproduce the analog of Maxwell's equations in the æther. Indeed, one finds wave solutions for $\mathbf{\omega}$ propagating at the characteristic wave speed of the medium (which in VAM is $C_e$, playing a role analogous to $c$). Electric charge $q$ in this picture corresponds to sources of vorticity flux. For instance, a charged particle's electric field corresponds to a radial swirl flow in the æther, and Gauss's law $\nabla\cdot \mathbf{E} = \rho_e$ translates to a divergence in the æther's velocity field equal to the charge density (suggesting charged particles are loci of æther outflow or inflow). The magnetic field is literally the circulating flow around a current (moving vortex).


Because $L_{\text{swirl}}$ is structurally the same as the electromagnetic Lagrangian, it yields the same 1/r potential and wave propagation as classical electromagnetism. The difference is interpretational: the electromagnetic vector potential is now a physical swirl of the æther, and an electromagnetic wave is a vortex ripple through the medium. Notably, the fine structure constant $\alpha = e^2/\hbar c$ would emerge from the fluid properties (æther density, circulation quantum, etc.) rather than being fundamental – VAM papers suggest $\alpha$ can be derived from ratios of $C_e$, $\Gamma$, and $r_c$.


\subsection{Strong Interaction Term (Linking Number \& Helicity)}

The strong nuclear force, which binds quarks into nucleons and nucleons into nuclei, is re-envisioned in VAM as an interaction arising from topological linking and collective vortex tension. In a fluid, when multiple vortices are present, their topology (how they loop and link through each other) contributes an additional energy termed the mutual helicity. Helicity $H$ is a conserved quantity for ideal fluids, given by the volume integral of velocity · vorticity. For a set of $N$ vortex rings, the total helicity can be decomposed into contributions from each individual loop's twist and writhe (self-helicity $H_{\text{self}}$) and from each pair of loops linking (mutual helicity $H_{\text{mutual}}$). The mutual helicity between vortex $i$ and $j$ is proportional to the Gauss linking number $Lk_{ij}$ (an integer) times the product of their circulations:


\[
    H_{\text{mutual}}(i,j) = 2\,Lk_{ij}\,\Gamma_i\,\Gamma_j ~,
\]

which appears in the total helicity formula. Physically, when two vortex loops link, they cannot move independently without stretching the fluid, so an energy is stored proportional to how many times they loop through one another and their circulation strengths.


We leverage this for the strong force by including a linking interaction term in the Lagrangian. One convenient form is a potential energy proportional to the sum of squared vorticity over all loops minus cross-terms that reward linking (to reflect a bound state being lower energy than separate parts). For example:

\[
    L_{\text{strong}} = -\frac{\kappa}{2}\sum_{i<j} Lk_{ij}\,\Gamma_i \Gamma_j \,-\, \sum_i \frac{\kappa'}{2} \Gamma_i^2~,
\]
where $\kappa, \kappa'$ are coupling constants related to the æther's compressibility and tension. This form is inspired by the helicity formula: it effectively binds vortices that are linked by lowering the energy when $Lk_{ij} \neq 0$. The second term ($\propto \Gamma_i^2$) represents self-energy of each vortex (analogous to \grqq mass\textquotedblright term or vortex core energy) and provides a baseline tension that resists stretching the vortex.


For a simple case of three linked vortices (analogous to three quarks in a baryon), the above term yields an energy minimum when the loops are linked in a way that maximizes the total $Lk_{ij}$ for the triplet configuration. In other words, it favors states where the vortices are all mutually interlinked, which is reminiscent of color confinement – you cannot separate one vortex (quark) without doing work against this linking term. If one tries to pull a loop out of the bound state, the linking term's energy rises, akin to the linear rising potential between quarks in QCD. The linking number here plays the role of color charge interactions: only certain combinations (topologies) are allowed for a neutral total linking (just as quarks combine to net colorless states).


Notably, VAM's mass formula for hadrons (discussed in the next section) indeed uses a topological term $\propto p q$ – essentially the product of two winding numbers – to account for knot complexity, which is conceptually similar to a linking contribution. We can think of the strong Lagrangian term as a potential well for knotted configurations: it is zero for an unlinked collection of vortices (no mutual links), deeply negative (favorable) for a maximally linked state, and rises steeply if one attempts to change the linking number (thus confining the configuration).


To fully mirror quantum chromodynamics, one might extend this to a non-Abelian swirl field – e.g. having three \grqq colors\textquotedblright of circulation that can exchange swirl – but a simpler effective approach is to just encode the combinatorial possibilities via linking numbers. Each quark-vortex could carry a distinct circulation sign or mode (analogous to color charge), and linking two different types might have a different $\kappa$ strength. However, at the level of hadrons, the above effective $L_{\text{strong}}$ suffices to predict hadronic masses and stabilities from allowed knot linkages.


\subsection{Weak Interaction Term (Reconnection \& Torsion)}

Weak interactions are unique in that they change particle identities (flavor) and allow topological reconfigurations (e.g. a neutron decays into a proton, electron, and neutrino – a process that changes the particle's internal structure). In VAM, a natural analog is vortex reconnection or a change in knot topology. In an ideal classical fluid, vortex loops cannot break or change links (helicity is conserved topologically), but in reality (or with small dissipation) vortices can reconnect when they intersect, changing the link type. We propose that weak-force processes correspond to rare reconnection events in the vortex network, enabled by a special term in the Lagrangian that allows violation of strict helicity conservation in extreme conditions.


One way to encode this is via a torsion or helicity flux term that is normally zero (preventing topology change) but becomes significant when vortex curvature is extreme (high energy). For instance, consider a term like:

\[
    L_{\text{weak}} = -\lambda \, |\mathbf{\omega} \cdot (\nabla \times \mathbf{\omega})|^2 ~,
\]
which penalizes configurations with large twist (vorticity curling around itself). Here $\mathbf{\omega} \cdot (\nabla \times \mathbf{\omega})$ is essentially the fluid helicity density, which is ordinarily conserved. A term proportional to its square (or another odd-parity term like $\mathbf{\omega}\cdot\mathbf{E}_v$, analogous to E·B in electroweak theory) can introduce a slight breaking of symmetry that allows helicity to change when the local twist exceeds some threshold. Physically, this represents the idea that if a vortex is twisted tightly enough (high curvature, analogous to high momentum transfer), it can \grqq snap\textquotedblright and reconnect into a new configuration – similar to how a W boson exchange can change one quark type into another, altering the particle's topology.


Another candidate term is a Kelvin-wave excitation term, since in quantum fluids vortex loops support helical ripple modes known as Kelvin waves. A high-frequency Kelvin wave of sufficient amplitude might effectively cause a reconnection. In Lagrangian form, one could include a term like

\[
    L_{\text{weak}}' = -\eta\,(\nabla^2 \mathbf{v})^2~,
\]

a higher-derivative term that becomes relevant at very small length scales (comparable to $r_c$). This would make it energetically favorable for a very tightly curved segment (small radius of curvature $\sim r_c$) to break and reattach differently. Such a term breaks the strict topological conservation and mimics the massive mediator of the weak force by requiring a high energy (curvature) to activate. The coefficient $\eta$ would be tuned such that the energy scale to induce reconnection corresponds to the W boson mass scale (~80 GeV). In effect, vortex reconnection events are suppressed at low energies (hence weak interactions are short-range and rare), but given enough energy, a knot can change – analogous to a neutron decaying or two particles undergoing a flavor-changing interaction.


One can maintain some analogy to the $SU(2)_L$ symmetry of the weak force by noting that weak interactions in the SM violate parity (they are chiral). In VAM, a chiral asymmetry could come from vortex handedness: a left-handed vs right-handed twist in the vortex might respond differently to the reconnection term (perhaps only one handedness of twisting mode leads to reconnection, imitating the $SU(2)_L$ selection of left-handed fermions). While a detailed field-theoretic implementation is complex, we can say that \textit{candidate terms for the weak interaction in VAM would be ones that:} (a) violate a conserved quantity (like helicity or mirror symmetry), (b) have a high activation energy threshold (reflecting the large weak boson mass), and (c) cause a change in knot topology (representing particle flavor change or decay). The simple helicity-torsion term $|\mathbf{\omega}\cdot(\nabla\times\mathbf{\omega})|^2$ given above captures these qualitatively: it is zero for untwisted, stable configurations (preserving topology), but nonzero for tightly twisted states, providing a channel for the vortex to \grqq flip\textquotedblright via reconnection.

\subsection{Additional Terms and Full Lagrangian Structure}
Putting it all together, a full VAM Lagrangian $L_{\text{VAM}}$ can be schematically written as:


\begin{equation}
    \boxed{\ L_{\text{VAM}} = L_{\text{kinetic}}(\rho_{\text{\ae}},\mathbf{v}) + L_{\text{grav}}(\rho_{\text{\ae}},F_{\max}) + L_{\text{swirl}}(A_v) + L_{\text{strong}}(Lk_{ij}, \Gamma_i) + L_{\text{weak}}(\mathbf{\omega}) + L_{\text{mass}}(\text{density, swirl})}
\end{equation}

where each term is defined as described above. The kinetic term captures the æther's motion, the gravitational term encodes the pressure and geometry, the swirl term describes electromagnetic interactions, the strong term captures linking and helicity, and the weak term allows for reconnection dynamics.
plus perhaps a mass term or Higgs-like term that gives rest energy to the vortices. In a fluid, mass arises from the moving volume of fluid; a vortex core of density $\rho_{\text{\ae}}$ and volume $V$ has a rest energy $E = \rho_{\text{\ae}} V c_s^2$ (where $c_s$ is the sound speed or analogue of $c$). In VAM, rather than the Higgs field giving mass, the internal swirling energy of the knotted vortex does. So one may include
\[
    L_{\text{mass}} = \sum_i \frac{1}{2}\rho_{\text{\ae}} |\mathbf{v}_i|^2
\]

integrated over each vortex core volume – effectively the kinetic energy of the trapped æther rotating in the core, which \textit{behaves like rest mass} for that vortex particle.

All terms in $L_{\text{VAM}}$ are expressed in mechanical units (e.g., SI), but by construction the combination of constants ($C_e, r_c, \rho_{\text{\ae}}, F_{\max}, t_p$) in these terms reproduces the correct dimensionless coupling strengths. For example, the electromagnetic coupling $e$ emerges from $L_{\text{swirl}}$ once units are converted, and the gravitational coupling $G$ emerges from the parameters of $L_{\text{grav}}$. Because of this, the Lagrangian is dimensionally self-consistent – no ad hoc insertion of $c$ or $\hbar$ is needed; those are effectively $C_e$ and a derived circulation quantum $\kappa = \Gamma/2\pi$ in the fluid model.

In summary, the VAM unified Lagrangian recasts all fundamental forces as interactions of a single fluid medium: geometry/density for gravity, swirl gauge fields for electromagnetism, topological linking for the strong force, and reconnection dynamics for the weak force. This provides a common physical picture in which \textit{all forces are manifestations of the æther's dynamics}.

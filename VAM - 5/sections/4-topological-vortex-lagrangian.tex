
\section{Knot Topologies and Analogies to the Periodic Table}

Beyond individual particles, one can ask whether composite vortex topologies might relate to the structure of atoms and the periodic table. This idea harkens back to Lord Kelvin's 19th-century vortex atom hypothesis, which proposed that each chemical element is a unique knotted vortex in the luminiferous æther. While that original notion was set aside with the advent of quantum atomic theory, VAM revives some of its spirit: here, however, \textit{subatomic} particles are vortices. Still, it is tantalizing to seek patterns linking vortex topology to atomic mass and chemical periodicity.

In the chemical periodic table, elements fall into families (noble gases, reactive non-metals like halogens, alkali metals, etc.) with repeating patterns of reactivity and valence as atomic number increases. In standard theory this comes from electron shell filling. In a VAM-inspired viewpoint, one might imagine the nucleus plus electron vortex system as a combined knotted configuration. Alternatively, one could attempt to map each element to a particular knot or link representing the entire atom's vortex structure. While a full model of the periodic table is beyond current VAM theory, we can draw analogies:

\begin{itemize}
    \item \textbf{Simple Knots and Light Elements:} The simplest knot (an unknotted loop) might correspond to hydrogen (one proton, one electron -- a very basic configuration). The simplest nontrivial knot, the trefoil $T(2,3)$, we already associated with the electron's vortex. Perhaps a hydrogen atom could be viewed as a linkage of an electron trefoil with a proton's three-knot system -- a sort of two-component link. As we go to helium (two protons, two neutrons, two electrons), the system is more complex but also notably stable and inert. This resembles a symmetrically linked structure: one could imagine two vortex rings (representing two protons) linked with two smaller electron vortex loops in a balanced, tightly-knit fashion -- a bit like a Borromean arrangement that overall is hard to perturb (helium is a noble gas). So helium might correspond to a nicely balanced link, possibly analogous to a Solomon link or a Hopf link of two composite sub-knots, yielding a ``closed shell'' topology.

    \item \textbf{Halogens (Reactive Non-metals):} These elements (F, Cl, Br, I, etc.) are one electron short of a full shell and are highly reactive, often existing as diatomic molecules (Cl$_2$, etc.) or forming salts readily. In a knot analogy, a halogen atom could correspond to a vortex structure that can achieve lower energy by linking with an identical structure (forming a diatomic link). For instance, consider a vortex loop that has a twisting or open aspect that invites another loop to latch onto it. A simple analogy is a Hopf link of two rings (linking number 1): each ring by itself might be considered ``unsatisfied'' -- perhaps each ring has a certain twist that creates a field line looping outwards -- but when two rings link, those field lines join and the system stabilizes. Thus, we might say: \textit{a halogen atom's vortex has one available link site, and two halogen vortices will naturally join to form a stable two-loop molecule.} In this picture, the reactivity (tendency to bond) corresponds to the presence of a free linking opportunity in the knot. Topologically, one could imagine a halogen's vortex structure as analogous to a loop with a dangling braid -- by itself high energy, but if another loop comes to interlock, the braid closes and energy is released (like completing a shell).

    \item \textbf{Noble Gases (Inert Gases):} By contrast, noble gases (He, Ne, Ar, etc.) have full electron shells and do not tend to bond or react. In the knot analogy, a noble gas atom would correspond to a completely self-contained vortex configuration with no loose ends or linking sites. It might be a highly symmetric knot or link that is ``internally satisfied.'' For example, Neon (atomic number 10) might map to a particularly symmetric link of multiple vortex rings (perhaps a triangular 3-ring link where each ring links the other two -- a fully linked trio, which is a very stable configuration as discussed). Or it could be a single, more complex knot whose internal twist effectively closes off any external field lines. The key is that adding another atom (another vortex) does not yield a lower energy configuration, hence no driving force for reaction -- just as noble gas atoms don't form stable bonds easily. The high symmetry of a vortex structure could correlate with chemical inertness.

    \item \textbf{Periodic Periodicity:} Across a period, as atomic number increases, the atomic radius first shrinks (as charge increases, pulling electrons in) then eventually a new shell starts (size jumps at noble gas to next alkali). In a vortex sense, one might speculate that knot complexity increases up to a point, then resets with an additional loop or layer. For example, lithium (atomic \#3) starts a new row; it's reactive (1 valence electron) akin to having one loosely attached vortex filament. As one adds more protons/electrons (going to Be, B, C\ldots), the vortex system might braid increasingly -- carbon perhaps being a tangle corresponding to half-filled valence (similar to how carbon is versatile with four bonds, perhaps a vortex with four preferred linking sites, reminiscent of a trefoil with an extra twist or a four-loop link). Moving to neon (\#10), the vortex tangles reconfigure into a highly symmetric closed form (noble gas). Then sodium (\#11) starts the next pattern with a new outer ``whorl'' loosely attached.
\end{itemize}

Though these ideas are speculative, we can attempt a small correspondence table to illustrate the analogy between knot structures and atomic families:

\begin{table}[h]
    \centering
    \footnotesize
    \caption{Analogies Between Knot Topologies and Atomic Families}
    \begin{tabular}{llll}
        \toprule
        \textbf{Atomic Family / Example} & \textbf{Possible Vortex Topology Analog} & \textbf{Topological Features} & \textbf{Analogy to Reactivity} \\
        \midrule
        Halogens (e.g.\ Cl$_2$) -- reactive, typically diatomic & Two identical vortex loops forming a Hopf link ($Lk=1$) & Each atom's vortex has one ``open'' link possibility; pairing two loops satisfies both by linking once, releasing energy (diatomic bond). Individually unstable (high energy) unless linked -- analog to halogen atoms' single unpaired electron driving bond formation. & High reactivity, diatomic bonding \\
        Noble Gases (e.g.\ Ne) -- inert monatomic gases & A single, highly symmetric fully-interlinked knot/link (e.g.\ a 3-ring all-to-all link, or a complex single knot with no external fields) & No available link sites. All vortex filaments are internally connected or balanced. The structure does not gain stability by linking with another, analogous to a filled valence shell. Thus it remains monatomic and chemically inert. & Inert, monatomic \\
        Alkali Metals (e.g.\ Na) -- very reactive, one valence electron & Vortex structure with a loosely bound filament or loop (think of a main vortex ring with a tiny satellite ring barely attached) & The loosely attached part (valence electron analog) can easily detach or link with another atom's structure. This corresponds to ready donation of that electron or formation of ionic bonds. The ``floppy'' vortex appendage makes the atom highly reactive. & High reactivity, ionic bonding \\
        Group IV Elements (e.g.\ Carbon) -- four valences, catenation & A vortex knot with four symmetric linking sites (imagine a central knot from which emanate 4 lobes that can link) & Carbon's ability to form four bonds might correspond to a vortex structure that can connect to up to four other vortices in a tetrahedral arrangement. Topologically, this could be a complex knot whose geometry has four protruding loops (like a thistle shape) -- each can connect to another carbon or other atom, explaining carbon's catenation and versatility. Once all four are linked, a stable network (like a diamond lattice) forms. & Tetravalency, catenation \\
        \bottomrule
    \end{tabular}
\end{table}

These analogies are admittedly qualitative, but they show that knot symmetry and linking capability can be thought of like valence electrons in chemistry. A knot with an ``odd dangling'' part seeks another to link with (just as an atom with one unpaired electron seeks a partner to share or transfer that electron). A completely balanced knot has no such dangling link (like a full shell). Interestingly, the periodicity could emerge because after a certain number of links, adding more results in a new configuration that is effectively a larger structure with a new ``dangling'' part -- akin to starting a new electron shell once one is filled.

Historically, Kelvin and Tait did try to identify small knots with elements (e.g.\ maybe hydrogen = unknot, oxygen = two-linked rings, etc.), and while that was never quantitatively successful~\cite{kelvin1867}, the spirit was that topological distinctness could correspond to chemical distinctness. VAM, focusing on subatomic vortices, suggests that chemical behavior is an emergent property of how electron vortex knots arrange around nuclear vortex knots. Atomic orbitals might even be seen as standing vortex patterns in the æther around the nucleus. Leonhard Euler's fluid equations and Helmholtz's vortex laws guarantee that vortex rings can orbit and form knots, so one can imagine electron vortices braided around a nucleus vortex structure in stable quantized orbits -- a fluid-mechanical model of orbitals.

While the exact mapping of the periodic table to knot theory remains an open question, these analogies provide a fresh perspective. They encourage further research, for example: could one simulate a multi-vortex system with different configurations and find patterns corresponding to periodic properties? Perhaps certain symmetric link types correlate with particularly stable configurations (noble-like), whereas certain almost-symmetric but not quite closed link types correlate with highly reactive ones.

In conclusion, composite vortex topologies (like links of multiple knots, torus knots of various $(p,q)$) offer a rich language to describe not just isolated particles but also how they combine. The VAM approach unifies the micro (elementary particles) with potentially the macro (chemical structures) under one topological fluid mechanism. This unity of ideas harkens to a truly unified theory where everything -- from electrons to protons to atoms -- is a vortex of one kind or another in a universal substratum.

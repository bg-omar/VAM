%! Author = Omar Iskandarani
%! Date = 3/13/2025


\subsection{Maximum Force in the Vortex Æther Model}


\paragraph*{Introduction}
The concept of an upper bound on force arises in General Relativity (GR), particularly in black hole physics, where it takes the form:


\begin{equation*}
    F_\text{max, GR} = \frac{c^4}{4G},
\end{equation*}
where $c$ is the speed of light and $G$ is the gravitational constant \cite{Schiller2006}. This limit is derived from black hole event horizons and causal structures.


The concept of a \textbf{maximum force} in the Vortex Æther Model (VAM) is introduced as an upper bound on vortex interactions. Given an inviscid medium where velocity scales with \( C_e \), we define the force:

\begin{align}
    F = \frac{dp}{dt} = \frac{d}{dt} (\rho_{\text{\ae}} v A),
\end{align}

where:
- \( \rho_{\text{\ae}} \) is the Æther density,
- \( v = C_e \) is the vortex-core tangential velocity,
- \( A = \pi r_c^2 \) is the vortex-core cross-sectional area.

Since the \textbf{momentum flux cannot exceed a limit set by vortex stability}, we impose the condition:

\begin{align}
    F_{\max} \approx \rho_{\text{\ae}} C_e^2 \pi r_c^2.
\end{align}

\subsection{Interpretation}

This equation suggests that vortex interactions in the Æther cannot exceed a fundamental force bound. If \( \rho_{\text{\ae}} \) and \( r_c \) are chosen to match known physical constants, the predicted limit aligns with observed force scales in high-energy interactions.

In the Vortex Æther Model (VAM), a similar upper force limit is proposed, emerging from vortex circulation dynamics. Unlike GR, where force is constrained by spacetime curvature, VAM embeds the limit in structured vorticity fields governing interactions. The maximal force in VAM follows:


\begin{equation*}
    F_\text{max, VAM} = \frac{c^4}{4G} \cdot \alpha \cdot \left(\frac{R_c}{L_p}\right)^{-2},
\end{equation*}
where $\alpha$ is the fine-structure constant, $R_c$ is the characteristic vortex-core radius, and $L_p$ is the Planck length.


\subsubsection*{Derivation and Scaling}
In GR, maximal force is inferred from the gravitational force at a Schwarzschild event horizon:


\begin{equation*}
    F = \frac{GMm}{R^2},
\end{equation*}
where setting $M \sim M_p$ (Planck mass) and $R \sim L_p$ (Planck length) yields:


\begin{equation*}
    F_\text{max, Planck} = \frac{c^4}{G}.
\end{equation*}


Within VAM, force constraints arise from vortex circulation, given by:


\begin{equation*}
    F_{\Gamma} = \frac{\rho_\text{\ae} \Gamma^2}{R},
\end{equation*}
where $\rho_\text{\ae}$ is the Æther density and circulation follows Kelvin's theorem:


\begin{equation*}
    \Gamma = 2\pi R_c C_e,
\end{equation*}
where $C_e$ is the tangential velocity at the vortex boundary. To align with GR force limits, a scaling factor relates vortex forces to Planckian constraints:


\begin{equation*}
    F_\text{max, VAM} \propto F_\text{max, GR} \times \left(\frac{R_c}{L_p}\right)^{-2}.
\end{equation*}


Including $\alpha$ accounts for quantum electrodynamical effects on vortex stability, leading to:


\begin{equation*}
    F_\text{max, VAM} = \frac{c^4}{4G} \cdot \alpha \cdot \left(\frac{R_c}{L_p}\right)^{-2}.
\end{equation*}


\subsubsection*{Implications}
This force constraint in VAM suggests:
\begin{enumerate}
    \item A fundamental link between vorticity, gravity, and electromagnetism.
    \item Vacuum polarization influences vortex force limits.
    \item Force scaling transitions smoothly from vortex physics to Planckian constraints.
\end{enumerate}


Future work should investigate experimental verification through superfluid analogues and numerical simulations of vortex dynamics at high energies.
%! Author = Omar Iskandarani
%! Date = 3/13/2025
\section{Swirl Velocity Constant \(C_e\)}

\subsection{Physical Rationale}
The swirl velocity constant \(C_e\) sets a characteristic rotational speed within the vortex-core region of the Æther. Conceptually, it parallels how the speed of light \(c\) is fundamental to electromagnetism, with \(C_e\) being fundamental to vortex-based phenomena. In early formulations of VAM, \(C_e\) emerges from equating quantized vortex circulation (inspired by superfluid helium analogies) to known particle parameters (e.g., electron radius or Coulomb barrier radius).

\subsection{Typical Derivation Sketch}
\begin{enumerate}
    \item \textbf{Vortex Circulation} \\
    Recall that for a quantum vortex, circulation \(\Gamma\) is often quantized in integer multiples of \(h/m\). If we assume one quantum of circulation for the \grqq electron vortex\textquotedblright core of radius \(r_c\), then
    \[
        \Gamma \;=\; 2 \pi \,r_c \,C_e \;\;\approx\;\; \frac{h}{m_e}.
    \]
    Solving for \(C_e\) yields
    \[
        C_e
        \;=\;
        \frac{h}{2 \pi m_e \,r_c}.
    \]

    \item \textbf{Matching Empirical Data} \\
    Substituting known constants (Planck\rqs s constant \(h\), electron mass \(m_e\), and a chosen vortex-core radius \(r_c\approx 1.4\times 10^{-15}\,\mathrm{m}\)) yields a numerical value on the order of \(10^6\,\mathrm{m/s}\). This constant characterizes the swirl at the core boundary for stable vortex knots.
\end{enumerate}

---

\section{Gravitational Coupling \(G_\text{swirl}\)}

\subsection{Motivation}
In VAM, gravity emerges from vortex-induced pressure gradients, rather than from mass-energy curvature. To unify this approach with observed large-scale gravitational phenomena, one introduces an effective gravitational constant \(G_\text{swirl}\). While it reduces to the familiar Newtonian \(G\) at large scales, it differs fundamentally in how it couples to vorticity distributions rather than mass-energy tensors.

\subsection{Outline of the Derivation}
\begin{enumerate}
    \item \textbf{Poisson-Like Equation} \\
    VAM posits
    \[
        \nabla^2 \Phi_v
        \;=\;
        -\,\rho_{\mathrm{\AE}}\;\bigl|\boldsymbol{\omega}\bigr|^2 \; \alpha,
    \]
    where \(\alpha\) is a dimensionless parameter, \(\rho_{\mathrm{\AE}}\) is the Æther density, and \(\boldsymbol{\omega}\) is the local vorticity. In a weak-vorticity limit, if one identifies \(\rho_{\mathrm{\AE}} |\boldsymbol{\omega}|^2\) with \grqq mass density,\textquotedblright a constant emerges that parallels \(G\).

    \item \textbf{Matching the Newtonian Limit} \\
    When specialized to a static, spherically symmetric distribution of effective mass (or vortex concentration), \(\Phi_v(r)\) resembles \(-G_\text{swirl} M_\text{eff}/r\). This sets
    \[
        G_\text{swirl}
        \;\approx\;
        \alpha \,\frac{\rho_{\mathrm{\AE}}\,C_e^2}{\mathcal{F}(r_c)},
    \]
    where \(\mathcal{F}(r_c)\) is a factor capturing the vortex-core radius and boundary conditions.
\end{enumerate}

---

\section{Maximum Force \(F_{\max}\)}

\subsection{Conceptual Role}
\(F_{\max}\) is proposed as an upper bound on force—somewhat analogous to the Planck force in quantum gravity contexts. In VAM, it can emerge from constraints on how vortex lines can transmit momentum under near-c speed tangential flows.

\subsection{Illustrative Relation}
If \(\mathbf{v}\) saturates at or below \(C_e\) in vortex cores, and the cross-sectional scale is \(r_c\), then the maximum momentum flux across that core area leads to
\[
    F_{\max}
    \;\approx\;
    \rho_{\mathrm{\AE}}\, C_e^2\, \pi r_c^2
\]
(plus dimensionless factors that depend on topological boundary conditions). Empirically, VAM often picks \(F_{\max}\approx 29\,\mathrm{N}\) to match certain nuclear or near-nuclear scale interactions.

---

\section{Corrections to Time Flow: Key Equations}

\subsection{Local Time \grqq Dilation\textquotedblright in VAM}
Although VAM adheres to absolute time globally, local vortex-core effects can modulate the rate at which clocks tick when placed in regions of strong swirl or vorticity. Derivations parallel the gravitational time-dilation in General Relativity but replace mass-based potentials with vortex-induced metrics.

In a simplified scenario, one obtains the \grqq adjusted time\textquotedblright:
\[
    t_\text{adjusted}
    \;=\;
    \Delta t \,\sqrt{\,1 \;-\; \frac{2\,G_\text{swirl}\,M_\text{effective}(r)}{r\,c^2}
    \;-\;
    \frac{C_e^2}{c^2} \, e^{-r/r_c}
        \;-\;
        \frac{\Omega^2}{c^2} \, e^{-r/r_c}
    },
\]
where
\begin{itemize}
    \item \(G_\text{swirl}\) couples the effective mass/vortex concentration to the potential,
    \item \(C_e^2 e^{-r/r_c}\) is an exponential correction capturing swirl velocity at or near the core boundary,
    \item \(\Omega^2 e^{-r/r_c}\) incorporates rotational or frame-dragging terms from any net angular velocity \(\Omega\),
    \item \(\Delta t\) is the \grqq global\textquotedblright or far-field time interval,
    \item \(r_c\) is the characteristic vortex-core radius that ensures short-range saturation of swirl or rotating flows.
\end{itemize}

\subsubsection{Physical Meaning}
\begin{itemize}
    \item As \(r \to \infty\), the exponentials vanish, leaving the usual Newtonian-like \(\sqrt{1 - 2G_\text{swirl} M_\text{eff} / (rc^2)}\).
    \item In the near-core region (\(r \approx r_c\)), large swirl or frame rotation strongly reduces the local ticking rate.
\end{itemize}

\subsection{Simplified Differential Form}
In certain configurations, if only the exponential swirl term is dominant (e.g. ignoring \(M_\text{effective}(r)\) and \(\Omega\)), the rate of local time relative to global time can be written:
\[
    \frac{d t_\text{adjusted}}{d t}
    \;=\;
    \sqrt{1 \;-\; \frac{C_e^2}{c^2}\, e^{-\,r/r_c}}.
\]
This relation is particularly relevant for analyzing how time is slowed in close proximity to a stable vortex filament—analogous to a \grqq time-warp\textquotedblright near a strong gravitational field in relativity.

---

\section{Final Boxed Equations}

Summarizing these key results, we highlight two essential time-rate expressions:

\[
    \boxed{
        t_\text{adjusted}
        \;=\;
        \Delta t \,\sqrt{
            1
            \;-\; \frac{2\,G_\text{swirl}\,M_\text{effective}(r)}{r\,c^2}
            \;-\; \frac{C_e^2}{c^2}\, e^{-\,r/r_c}
            \;-\; \frac{\Omega^2}{c^2}\, e^{-\,r/r_c}
        }
    }
\]

\[
    \boxed{
        \frac{d t_\text{adjusted}}{d t}
        \;=\;
        \sqrt{1 \;-\; \frac{C_e^2}{c^2}\, e^{-\,r/r_c}}
    }
\]

Here:
\begin{enumerate}
    \item \(G_\text{swirl}\) is the vorticity-based gravitational coupling.
    \item \(C_e\) sets the swirl velocity scale in vortex cores.
    \item \(r_c\) is the vortex-core radius controlling short-range vortex structure.
    \item \(\Omega\) is a global or local rotation parameter that can augment or diminish time-flow adjustments.
    \item \(M_\text{effective}(r)\) identifies how vortex distributions effectively mimic mass at scale \(r\).
\end{enumerate}

---

\section{Concluding Remarks}

These derivations unify multiple hallmark features of the Vortex Æther Model:
\begin{itemize}
    \item \(C_e\) and \(F_{\max}\) anchor the short-distance or high-vorticity behavior.
    \item \(G_\text{swirl}\) ensures Newton-like gravity emerges at large distances while acknowledging vortex dominance near the core.
    \item Time Adjustment Equations capture the essential notion that \grqq local time slows\textquotedblright in high-vorticity zones, paralleling gravitational time dilation in general relativity but implemented purely through a fluid-dynamic swirl framework.
\end{itemize}

By offering explicit functional forms, this approach paves the way for numeric and conceptual exploration—ranging from near-atomic scales (where \(r_c\) becomes critical) to astrophysical phenomena in rotating systems (where \(\Omega\) and large-scale vortex flows might be relevant).
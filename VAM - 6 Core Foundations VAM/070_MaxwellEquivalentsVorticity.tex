\section{On the Equivalent of Maxwell's Equations from Vorticity}

The Vortex Æther Model (VAM) provides a direct analog to classical Maxwellian electrodynamics, wherein electromagnetic fields arise as structured vorticity fields in an inviscid superfluid. Instead of a separate electromagnetic field propagating in vacuum, VAM proposes that electric and magnetic effects correspond to vortex dynamics in the Æther.

In this formulation:
\begin{itemize}
    \item \textbf{Electric-like fields} (\(\mathbf{E}_v\)) arise from irrotational vortex flow potentials (\(\Phi_v\)).
    \item \textbf{Magnetic-like fields} (\(\mathbf{B}_v\)) emerge from solenoidal (circulatory) vortex structures.
\end{itemize}

The fundamental equations governing these vortex fields are:
\begin{align}
    \nabla \cdot \mathbf{E}_v &= \frac{\rho_v}{\varepsilon_v}, \quad \nabla \cdot \mathbf{B}_v = 0, \label{eq:gauss_vam} \\
    \nabla \times \mathbf{E}_v &= -\frac{\partial \mathbf{B}_v}{\partial t}, \quad \nabla \times \mathbf{B}_v = \mu_v \mathbf{J}_v + \mu_v \varepsilon_v \frac{\partial \mathbf{E}_v}{\partial t}. \label{eq:maxwell_vam}
\end{align}
These directly mirror Maxwell's equations but with vorticity replacing charge-driven fields.

Furthermore, disturbances in these fields propagate as waves at a characteristic velocity:
\begin{equation}
    v_\text{wave} = \frac{1}{\sqrt{\mu_v \varepsilon_v}}.
\end{equation}
which, when setting \(\mu_v = \mu_0\) and \(\varepsilon_v = \varepsilon_0\), reduces to \( c \), the observed speed of light.

A \textbf{detailed mathematical derivation}, including the full index notation representation, is provided in \textbf{Appendix~
\ref{appendix:maxwell_appendix}}.
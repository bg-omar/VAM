%! Author = Omar Iskandarani
%! Date = 3/21/2025

\section{Derived Equations for the Vortex Æther Model (VAM)}
Inspired by recent experimental research and theoretical developments in electromagnetism, fluid dynamics, and vortex knot theory
\cite{Matsoukas2014} \cite{Kleckner2013} \cite{Ricca1998}
}, we derive novel equations relevant to the Vortex Æther Model (VAM). These equations provide
theoretical relationships
that link electromagnetic fields to structured vorticity fields within an incompressible Æther, offering potential experimental validation pathways for VAM.

In VAM, gravitational interactions result from structured vortical flows within Æther. Analogously, distortions of electric fields around dielectrics can model Æther equilibrium distortion, leading to gravitational-like effects.
We start with the known VAM and physical constants:

\begin{itemize}
    \item Vortex-angular-velocity: $C_e = 1,093,845.63\, \text{m/s}$
    \item Coulomb barrier radius: $R_c = 1.40897017\times10^{-15}\, \text{m}$
    \item Speed of light: $c = 299,792,458\, \text{m/s}$
    \item Electron mass: $M_e = 9.1093837015\times10^{-31}\, \text{kg}$
    \item Fine-structure constant: $\alpha \approx 7.2973525643\times10^{-3}$
\end{itemize}



\subsection{Electrostatic-Vortex Energy Equivalence}
We begin by establishing a theoretical equivalence between the electrostatic energy density in a charged medium and the kinetic energy density of Ætheric vortices. The electrostatic potential energy density is given by:
\begin{equation}
    U_E = \frac{1}{2} \varepsilon_0 E^2,
\end{equation}
where  is the vacuum permittivity and  is the electric field strength \cite{Matsoukas2014}.
Similarly, the vortex kinetic energy density in Æther, by analogy to classical fluid dynamics, can be expressed as:
\begin{equation}
    U_{\AE} = \frac{1}{2}\rho_\text{\ae} v_{\AE}^2,
\end{equation}
with  as the Æther density and  as the local vortex velocity.
Equating these energies yields:
\begin{equation}\label{eq:energy_equivalence}
\varepsilon_0 E^2 = \rho_\text{\ae} v_{\AE}^2.
\end{equation}
\subsection{Characteristic Æther Vortex Radius}
Using the energy equivalence in Eq. \eqref{eq:energy_equivalence}, we define a characteristic radius  for Æther vortices, which analogously
relates to the electric field and the known Coulomb barrier $r_c$.  Inspired by Matsoukas' ionic wind results, suppose a vortex-Æther radius is
proportional to the curvature radius of the charged electrode:

\begin{equation}\label{eq:vortex_radius}
R_{\AE} = \sqrt{\frac{\varepsilon_0}{\rho_\text{\ae}}}\frac{E}{C_e},
\end{equation}
where $C_e$ is the vortex angular velocity constant known from VAM parameters.
This equation offers a direct link between electric fields and the characteristic vortex radius within Æther.\cite{RelationsBetweenConstants2005}.
\subsection{Ætheric Pressure Gradient}
For stable vortical equilibrium states, Ætheric pressure gradients drive gravitational-like effects. Drawing upon Bernoulli\rqs s principle, we find:
\begin{equation}\label{eq:pressure_gradient}
\Delta P_{\AE} = \frac{\rho_\text{\ae} C_e^2}{2}\left(1 - \frac{\varepsilon_0 E^2}{\rho_\text{\ae} C_e^2}\right).
\end{equation}
This equation demonstrates that electromagnetic fields could directly control Ætheric pressure gradients, influencing gravitational-like phenomena.
\subsection{Gravitational-Like Acceleration}
Integrating the above results, a gravitational-like acceleration  arising due to electromagnetic control of Æther vortices is derived as:
\begin{equation}\label{eq:grav_acceleration}
a_{g} = \frac{2\pi \varepsilon_0 E^2\left(1 - \frac{\varepsilon_0 E^2}{\rho_\text{\ae}C_e^2}\right)}{M_\text{eff}},
\end{equation}
where  represents an effective inertial mass influenced by the Ætheric vorticity structure.
\subsection{Significance}
These derived equations \eqref{eq:energy_equivalence}--\eqref{eq:grav_acceleration} establish concrete theoretical connections between electromagnetic phenomena and vortex Æther dynamics, laying the groundwork for empirical validation of VAM through electromagnetic experimentation \cite{Podkletnov2007, Kleckner2013}. They highlight electromagnetic fields' potential utility as precise tools for manipulating Ætheric structures and verifying VAM's fundamental predictions.
%! Author = Omar Iskandarani
%! Date = 3/25/2025

\section{VAM vs. EHD: Advanced Modeling of Propulsive Mechanisms}


This section provides a comparative analysis of the conventional Electrohydrodynamic (EHD) propulsion model and the Vortex Æther Model (VAM), emphasizing foundational principles, governing equations, and momentum transfer paradigms. The synthesis integrates recent empirical findings \cite{ehdreview2023} \cite{ehdpropeller2023} \cite{electricfluidsetup2023} \cite{ehdmodeling2023} with a rigorous theoretical reinterpretation through the VAM framework.


\subsection{Canonical EHD Momentum Formulation}
In the traditional EHD framework, momentum transfer arises from the interaction of electric fields with charged species in a viscous fluid. The underlying dynamics are governed by a modified Navier--Stokes equation:
\begin{equation}
\rho \left( \frac{\partial \vec{U}}{\partial t} + (\vec{U} \cdot \nabla)\vec{U} \right) = -\nabla p + \mu \nabla^2 \vec{U} - \rho \nabla V
\end{equation}
This yields the standard EHD thrust relation:
\begin{equation}
T = \frac{I g}{\mu}
\end{equation}
where $I$ is the current, $g$ the interelectrode gap, and $\mu$ the ion mobility \cite{ehdreview2023}.


\subsection{VAM: Ætheric Dynamics and Vorticity-Based Thrust}
Contrastingly, VAM posits that net force arises from structured pressure gradients induced by confined vorticity within an inviscid, irrotational Æther medium. The central nonlinear thrust expression is:
\begin{equation}
T(V) = T_\text{max}(1 - e^{-\lambda V^2})
\end{equation}
where $V$ is the applied voltage, $\lambda$ is a geometry-coupled curvature factor, and $T_\text{max} = \rho_\text{\ae} C_e^2 A$ denotes the saturation-level vortex-induced pressure thrust \cite{vamderive2024}.


When voltage is time-dependent, the VAM framework yields a dynamic thrust response:
\begin{equation}
\boxed{
T = \kappa \cdot \frac{dV}{dt}
\quad \text{with } \kappa = \frac{\rho_\text{\ae} A \delta z \alpha_E}{2\pi}
}
\end{equation}
This captures the empirical behavior of impulse-based lifter configurations, where vortex unpinning events are synchronized with rapid field variations \cite{vamthrustfit2025}.


\subsection{Rotational Kinetics Within the VAM Framework}
In rotational systems, such as EHD propellers and toroidal coil drives, angular momentum conservation governs net torque generation. VAM expresses torque from shell pressure as:
\begin{equation}
\tau_\text{VAM} = \int R \cdot dF_\theta = \int R \cdot \rho_\text{\ae} C_e^2 , dA \Rightarrow \tau = R \cdot \rho_\text{\ae} C_e^2 A_\text{active}
\end{equation}


The induced angular velocity of the propeller becomes:
\begin{equation}
\boxed{
\omega_p = \frac{\rho_\text{\ae} C_e^2 A R}{J_p}
}
\end{equation}


Furthermore, equating the angular momenta of the vortex and the propeller yields:
\begin{equation}
J_p \omega_p = I_\text{vortex} \omega = \rho_\text{\ae} R^2 A \cdot \omega
\Rightarrow \boxed{
\omega_p = \omega = \frac{L_\text{vortex}}{J_p}
}
\end{equation}


\subsection{Comparative Framework: VAM vs. EHD}
\begin{center}
\begin{tabular}{|l|c|c|}
\hline
\textbf{Property} & \textbf{EHD Paradigm} & \textbf{VAM Paradigm} \
\hline
Driving Mechanism & Linear ion momentum & Confined vortex-induced pressure \
Thrust Expression & $T = \frac{Ig}{\mu}$ & $T = \kappa \cdot \frac{dV}{dt}$ or sigmoid $T(V)$ \
Electrode Configuration & Asymmetric pair (wire/foil) & Topological confinement (e.g. 3-phase coil) \
Field Topology & Linear electric gradient & Rotational vector potential shell \
Role of Current & Primary driver of force & Emergent artifact of vortex activation \
Role of Voltage & Charge acceleration & Ætheric potential well curvature \
Optimized Mode & Steady voltage & Pulsed excitation (e.g. sawtooth, rotating fields) \
\hline
\end{tabular}
\end{center}
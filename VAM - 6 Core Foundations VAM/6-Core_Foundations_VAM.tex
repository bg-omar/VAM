%! Author = Omar Iskandarani
%! Title = The Vortex Æther Model: A Unified Vorticity Framework for Gravity, Electromagnetism, and Quantum Phenomena
%! Date = Feb 15, 2025
%! Affiliation = Independent Researcher, Groningen, The Netherlands
%! License = CC-BY 4.0
%! ORCID = 0009-0006-1686-3961
%        \part \chapter \section \subsection \subsubsection  \paragraph  \subparagraph\






\documentclass[a4paper, aps,preprint,superscriptaddress, 12pt]{revtex4}
\usepackage[a4paper, margin=2cm]{geometry}
\usepackage{array}
\usepackage{booktabs}
\usepackage{amsmath}
\usepackage{amssymb}
\usepackage{graphicx}
\usepackage{hyperref}
\usepackage{physics}
\usepackage{tocloft}
\usepackage{xcolor}
\usepackage{titlesec}
\usepackage{etoolbox} % Ensure etoolbox is loaded
\usepackage{lmodern}  % Better font rendering
\usepackage{setspace} % Optional line spacing control
\setcounter{tocdepth}{2}

% -----------------------------
% TOC Title Format
\renewcommand{\contentsname}{\centering \Huge\textbf{Contents}\vspace{1em}}

% -----------------------------
% Reduce spacing and font size
\renewcommand{\cftbeforesecskip}{0pt} % Vertical space between entries
\renewcommand{\cftsecfont}{\small\color{blue!70!black}}  % Font style
\renewcommand{\cftsecpagefont}{\small\color{gray!90!black}} % Page # font

\renewcommand{\cftsubsecfont}{\small\itshape\color{gray!70!black}}
\renewcommand{\cftsubsecpagefont}{\small\color{gray!50!black}}

\renewcommand{\cftsecindent}{0pt}
\renewcommand{\cftsubsecindent}{1.5em}
\renewcommand{\cftsecnumwidth}{2em}
\renewcommand{\cftsubsecnumwidth}{3em}

% -----------------------------
% Optional: Remove dot leaders (for compactness)
\renewcommand{\cftsecleader}{\cftdotfill{0}}
\renewcommand{\cftsubsecleader}{\cftdotfill{0}}

% -----------------------------
% Optional: Adjust line spacing
\let\oldtoc\tableofcontents
\renewcommand{\tableofcontents}{
  \begingroup
  \setstretch{1.0}
  \oldtoc
  \endgroup
}

\begin{document}

\author{Omar Iskandarani}
    \title{The Vortex Æther Model: Unifying Gravity, Electromagnetism, and Quantum Physics under a 3D, Non-Relativistic, vortex framework}
    \date{\today}
    \affiliation{Independent Researcher, Groningen, The Netherlands}
    \thanks{ORCID: \href{https://orcid.org/0009-0006-1686-3961}{0009-0006-1686-3961}}
    \email{info@omariskandarani.com}

    %%% Abstract
    \begin{abstract}
        The Vortex Æther Model (VAM) introduces a unified, non-relativistic theoretical framework wherein gravity, electromagnetism, and quantum phenomena arise from structured vorticity within an inviscid superfluid-like Æther. Unlike General Relativity, which depends on four-dimensional spacetime curvature, VAM proposes that stable vortex knots in three-dimensional Euclidean space generate fundamental forces and quantized states through fluid dynamics and vortex topology. Central to this model is absolute universal time, where observed time dilation results from vortex-induced local energy gradients rather than relativistic effects. VAM yields experimentally testable predictions, including superfluid analogs of frame-dragging, magnetic fields in electrically neutral fluids, and atomic-scale quantization phenomena akin to those observed in helium II. Fundamental constants such as the vortex-core tangential velocity $C_e$ and the Coulomb barrier radius $r_c$ anchor core rotation speeds and interaction strengths, providing explicit testable parameters for experimental verification.
    \end{abstract}

    \maketitle
    \tableofcontents

    %% Introduction
    %! Author = Omar Iskandarani
%! Date = 3/13/2025


\begin{figure}[h]
    \includegraphics[width=\textwidth]{torus}
    \caption{Illustration of a vortex filament in Æther.}
    \label{fig:vortex}
\end{figure}


\section*{Introduction}

The concept of an all-pervading Æther has profoundly influenced physics since the 19th century, notably with James Clerk Maxwell's proposal that electromagnetic waves necessitate a propagating medium~\cite{Maxwell1865}. However, early experiments, particularly Michelson and Morley's \cite{michelson1887}, failed to detect the classical stationary Æther, leading Einstein to replace it with the invariant speed of light and spacetime geometry of Special and General Relativity~\cite{einstein1916foundation}.

Nevertheless, recent developments in quantum field theory and experimental studies of quantum superfluids, notably helium II, demonstrate that even the vacuum may exhibit nontrivial fluid-like properties, including quantized vortices and discrete energy states~\cite{Wilczek1999,Donnelly1991quantized}. Inspired by these developments and the historical ideas of Helmholtz~\cite{helmholtz1867integrals}, Kelvin~\cite{kelvin1867vortex}, and Maxwell~\cite{Maxwell1865}, the Vortex Æther Model (VAM) revisits the Æther hypothesis, proposing an inviscid, incompressible superfluid medium whose structured vortices underpin all fundamental physical phenomena.

VAM posits that gravitational attraction emerges from vortex-induced pressure gradients analogous to Bernoulli's principle rather than spacetime curvature. Similarly, electromagnetic phenomena are explained by vortex topology, where stable knotted vortices form analogs to charges and currents without necessitating discrete force carriers or additional dimensions. Quantum effects, including energy quantization and wave-particle duality, are interpreted through conserved vortex helicity and stable vortex knots, linking macroscopic fluid dynamics directly to microscopic quantum states.

Crucially, VAM reinstates absolute universal time, with observed time dilation effects resulting from local variations in vortex-induced energy distributions rather than relativistic velocity-based distortions. This provides an elegant explanation of phenomena traditionally associated with relativistic physics, such as gravitational lensing and frame-dragging, as natural outcomes of vortex circulation.

This paper presents the foundational principles and novel mathematical formalism of VAM, explicitly deriving key physical constants and demonstrating their implications. Furthermore, it highlights experimental tests uniquely predicted by this model, including analogs of gravitational frame-dragging in superfluids, electromagnetic phenomena in charge-neutral fluids, and measurable quantum effects arising from structured vortex configurations. By integrating classical fluid mechanics, quantum principles, and electromagnetic theory within a purely three-dimensional framework, VAM provides a coherent, testable alternative to contemporary physics paradigms.

The subsequent sections systematically present the mathematical formalism and specific experimental predictions, positioning VAM as a cohesive, empirically falsifiable alternative theory of fundamental interactions.

    %% Part I - Foundational Considerations
    \newpage \section*{Part I: Foundational Considerations}\label{sec:Part-1} %! Author = Omar Iskandarani
%! Date = 3/13/2025

\section{Addressing Historical Æther Detection Experiments}


The historical Michelson--Morley experiment, which yielded null results, has long been interpreted as definitive evidence against the existence of a luminiferous Æther. However, within the framework of the Vortex Æther Model (VAM), these results are elegantly and naturally reconciled. According to VAM, matter is fundamentally composed of stable vortex knots embedded within the Æther itself, meaning that all measuring instruments---such as interferometers---are not external observers of the Æther but intrinsically integrated into the Ætheric medium. Consequently, any attempt by such instruments to detect absolute motion through the Æther is inherently self-defeating, as the devices dynamically adjust their internal vorticity structure, thus precisely canceling any measurable relative-motion effects.


This intrinsic adaptability of matter to Ætheric flow is analogous to the Lorentz contraction concept central to Special Relativity, yet it emerges purely from vortex-flow dynamics rather than postulated relativistic transformations. Such phenomena have clear experimental analogues in fluid mechanics and superfluid systems. For instance, experiments in superfluid helium demonstrate how objects immersed within the superfluid medium do not detect their uniform motion relative to the medium through local measurements. This null result arises because measuring instruments and test particles are dynamically integrated with the vortex structure of the superfluid itself, effectively mirroring the null-detection outcomes observed in the Michelson--Morley experiments \cite{Vinen2008}.


Additionally, vortex interactions in classical fluids and plasmas consistently show that local detection of uniform flow relative to a structured vortex field is fundamentally problematic, as local measurement devices or markers are influenced and modified by the fluid's intrinsic vortical structures \cite{Meunier2005}. Thus, the historical inability to detect Ætheric motion does not negate the existence of the Æther but rather highlights its dynamic and integrative relationship with matter. The Michelson--Morley experiments, rather than disproving the Æther, underscore the fundamental principle that vortex structures within a continuous fluidic medium inherently adjust to negate measurable relative motion---a cornerstone prediction of the Vortex Æther Model.
    \newpage \input{020_SuperfluidLikeÆther}
    \newpage %! Author = mr
%! Date = 3/13/2025


\section{Fine-Structure Constant from Vortex Mechanics}

In the Vortex \AE ther Model (VAM), the fine-structure constant $\alpha$ emerges naturally from the fundamental vorticity of the \AE ther. Rather
than treating $\alpha$ as an arbitrary fundamental constant, VAM shows that it arises from the characteristic tangential velocity of stable vortex
structures. The detailed derivation is provided in Appendix~\ref{appendix:alpha}, where it is shown that:

\begin{equation}
    \alpha = \frac{2 C_e}{c},
\end{equation}

where $C_e$ is the vortex-core tangential velocity, linking $\alpha$ directly to vortex dynamics. This result reinforces the deep connection between electromagnetism and structured vorticity in the \AE ther.


The Coulomb barrier radius \(R_c\) is a fundamental and universal scale in the Vortex Æther Model, analogous to how \(\mu_0\) remains invariant in electromagnetism. Since \(R_c\) is derived from absolute vorticity conservation and fundamental charge interactions, it does not vary per atom. Any apparent variation would stem from environmental effects rather than an intrinsic difference in atomic structure, ensuring its role as a constant in vortex-based formulations of electromagnetism.

    %% Part II - MathematicalFormalism
    \newpage \section*{Part II: Mathematical Formalism}\label{sec:Part-2} %! Author = Omar Iskandarani
%! Date = 3/13/2025


\subsection{Maximum Force in the Vortex Æther Model}


\paragraph*{Introduction}
The concept of an upper bound on force arises in General Relativity (GR), particularly in black hole physics, where it takes the form:


\begin{equation*}
    F_\text{max, GR} = \frac{c^4}{4G},
\end{equation*}
where $c$ is the speed of light and $G$ is the gravitational constant \cite{Schiller2006}. This limit is derived from black hole event horizons and causal structures.


The concept of a \textbf{maximum force} in the Vortex Æther Model (VAM) is introduced as an upper bound on vortex interactions. Given an inviscid medium where velocity scales with \( C_e \), we define the force:

\begin{align}
    F = \frac{dp}{dt} = \frac{d}{dt} (\rho_{\!Æ} v A),
\end{align}

where:
- \( \rho_{\!Æ} \) is the Æther density,
- \( v = C_e \) is the vortex-core tangential velocity,
- \( A = \pi r_c^2 \) is the vortex-core cross-sectional area.

Since the \textbf{momentum flux cannot exceed a limit set by vortex stability}, we impose the condition:

\begin{align}
    F_{\max} \approx \rho_{\!Æ} C_e^2 \pi r_c^2.
\end{align}

\subsection{Interpretation}

This equation suggests that vortex interactions in the Æther cannot exceed a fundamental force bound. If \( \rho_{\!Æ} \) and \( r_c \) are chosen to match known physical constants, the predicted limit aligns with observed force scales in high-energy interactions.

In the Vortex Æther Model (VAM), a similar upper force limit is proposed, emerging from vortex circulation dynamics. Unlike GR, where force is constrained by spacetime curvature, VAM embeds the limit in structured vorticity fields governing interactions. The maximal force in VAM follows:


\begin{equation*}
    F_\text{max, VAM} = \frac{c^4}{4G} \cdot \alpha \cdot \left(\frac{R_c}{L_p}\right)^{-2},
\end{equation*}
where $\alpha$ is the fine-structure constant, $R_c$ is the characteristic vortex-core radius, and $L_p$ is the Planck length.


\subsubsection*{Derivation and Scaling}
In GR, maximal force is inferred from the gravitational force at a Schwarzschild event horizon:


\begin{equation*}
    F = \frac{GMm}{R^2},
\end{equation*}
where setting $M \sim M_p$ (Planck mass) and $R \sim L_p$ (Planck length) yields:


\begin{equation*}
    F_\text{max, Planck} = \frac{c^4}{G}.
\end{equation*}


Within VAM, force constraints arise from vortex circulation, given by:


\begin{equation*}
    F_{\Gamma} = \frac{\rho_\text{\ae} \Gamma^2}{R},
\end{equation*}
where $\rho_\text{\ae}$ is the Æther density and circulation follows Kelvin’s theorem:


\begin{equation*}
    \Gamma = 2\pi R_c C_e,
\end{equation*}
where $C_e$ is the tangential velocity at the vortex boundary. To align with GR force limits, a scaling factor relates vortex forces to Planckian constraints:


\begin{equation*}
    F_\text{max, VAM} \propto F_\text{max, GR} \times \left(\frac{R_c}{L_p}\right)^{-2}.
\end{equation*}


Including $\alpha$ accounts for quantum electrodynamical effects on vortex stability, leading to:


\begin{equation*}
    F_\text{max, VAM} = \frac{c^4}{4G} \cdot \alpha \cdot \left(\frac{R_c}{L_p}\right)^{-2}.
\end{equation*}


\subsubsection*{Implications}
This force constraint in VAM suggests:
\begin{enumerate}
    \item A fundamental link between vorticity, gravity, and electromagnetism.
    \item Vacuum polarization influences vortex force limits.
    \item Force scaling transitions smoothly from vortex physics to Planckian constraints.
\end{enumerate}


Future work should investigate experimental verification through superfluid analogues and numerical simulations of vortex dynamics at high energies.
    \newpage \newpage{050_SwirlVelocityConstant}
    \newpage \newpage{060_DerivationVAMFieldEquation}
    \newpage \newpage{070_MaxwellEquivalentsVorticity}
    \newpage \newpage{080_ElectrostaticChargeElectricFieldsVAM}
    \newpage \newpage{085_ElectroVortexDynamics}
    \newpage \newpage{090_MachInspiredScalar}
    \newpage \newpage{100_PhotonVortexDipole}
    \newpage \newpage{110_PhotonVortexDipoleGravitationalCoupling}
    \newpage \newpage{120_PhotonsDipoles}
    \newpage \newpage{130_HeatTheoryVAM}


    %% Part III - Applications and Implications
    \newpage \section*{Part III: Applications and Implications}\label{sec:Part-3} %! Author = Omar Iskandarani
%! Date = 3/23/2025



\section{\textbf{Experimental Analysis of Vortex-Induced Thrust in a High-Voltage Lifter System: Toward VAM-Based Propulsion}}


\begin{abstract}
    We report and analyze a high-voltage asymmetric capacitor experiment exhibiting thrust far beyond that explainable by electrohydrodynamics (EHD). Using a Cockcroft–Walton voltage multiplier generating up to 30kV and microampere-scale current, we observe stable, polarity-symmetric thrust scaling with voltage. The results show strong agreement with predictions of the Vortex \AE{}ther Model (VAM), which interprets the thrust as arising from pressure gradients induced by structured Ætheric vorticity. We present experimental details, time-resolved breakdown cycles, energy input vs. output analysis, and a predictive VAM thrust formulation consistent with observations.
\end{abstract}

\section{Introduction}
Electrogravitic propulsion, often associated with the Biefeld–Brown effect, has been a subject of renewed interest. Traditional explanations involve ion wind or corona discharge, yet recent precision experiments suggest anomalously high thrust-to-power ratios. The Vortex \AE{}ther Model (VAM) posits that vortex knots and pressure gradients in an inviscid Æther provide a framework to explain these effects without violating conservation laws.

\section{Experimental Setup}
\subsection{Circuit Architecture}
The system consists of:
\begin{itemize}
    \item 74HC14-based oscillator feeding an IR2104 gate driver.
    \item Half-bridge with IRFP460 MOSFETs, powered by a 310V DC rail.
    \item 250-stage Cockcroft–Walton multiplier (47$\mu$F, 350V capacitors + BY550-1000 diodes).
    \item 30M$\Omega$ load resistor for current limiting.
\end{itemize}

\begin{figure}[h!]
    \centering
    \includegraphics[width=0.8\textwidth]{figures/circuit_diagram.png}
    \caption{Block diagram of the experimental high-voltage generation system.}
\end{figure}

\subsection{Mechanical Design}
The lifter comprises:
\begin{itemize}
    \item Balsa frame
    \item Aluminum foil collector electrode
    \item Thin copper corona wire
    \item Optional dielectric shields (paper, glass)
\end{itemize}

\begin{figure}[h!]
    \centering
    \includegraphics[width=0.8\textwidth]{figures/lifter_structure.png}
    \caption{Photograph of the lifter structure and shield configuration.}
\end{figure}

\section{Measurements and Observations}
\subsection{Thrust-Voltage Response}
Thrust begins at 15kV and increases with voltage:
\begin{itemize}
    \item \textbf{No shield}: 25kV breakdown, thrust $\approx$ 6g
    \item \textbf{Paper shield}: 30kV, thrust $\approx$ 22g
    \item \textbf{Glass shield}: 30kV, thrust $\approx$ 32g
\end{itemize}

\begin{figure}[h!]
    \centering
    \includegraphics[width=0.8\textwidth]{figures/thrust_vs_voltage_curve.png}
    \caption{Measured thrust vs. voltage curve across different shielding configurations.}
\end{figure}

\subsection{Post-Breakdown Behavior}
Voltage breakdown at 25kV triggers thrust spike and collapse:
\begin{itemize}
    \item Thrust peak: $-6$g
    \item Decays to $-2$g as voltage resets to 5--10kV
    \item Rebuilds if voltage reapplied, showing cyclic behavior
\end{itemize}

\subsection{Symmetry Tests}
Reversing polarity produced identical results, indicating symmetry in the vortex field, not charge motion.

\subsection{Shielding Effects}
Dielectric barriers enhance thrust by shaping Æther flow:
\begin{itemize}
    \item Glass produced highest and most stable thrust
    \item Shield-on-lifter killed all thrust: symmetry nulls pressure gradient
\end{itemize}

\section{Theoretical Framework}
\subsection{VAM Pressure Gradient Thrust}
The predicted Ætheric thrust is:
\begin{equation}
    F = \frac{1}{2} \rho_\text{\ae} C_e^2 \left( \frac{V}{V_\text{bd}} \right)^2 A
\end{equation}
Where $\rho_\text{\ae}$ is the Æther density, $C_e$ the vortex tangential velocity, $V$ applied voltage, $V_\text{bd}$ breakdown threshold, and $A$ effective vortex area.

\begin{figure}[h!]
    \centering
    \includegraphics[width=0.8\textwidth]{figures/vortex_pressure_simulation.png}
    \caption{Simulated Ætheric pressure drop around the lifter during sub-breakdown operation.}
\end{figure}

\subsection{Cycle Behavior}
Voltage ramps up to $V_\text{bd}$, breakdown causes helicity release (thrust spike), and system rebuilds in seconds.

\section{Energy Analysis}
\begin{itemize}
    \item Power input: 0.3--0.45W (30kV @ 10--15$\mu$A)
    \item Measured thrust: up to 32g ($\approx$ 0.313N)
    \item Classical EHD predicts $\approx$ 0.0003g
    \item Efficiency: $>10^5$ times higher than EM-based expectation
\end{itemize}

\section{Conclusion and Next Steps}
The thrust observed is consistent with Ætheric vorticity pressure gradients, not classical EHD. A 3-phase 9-arm vortex array has been simulated to provide continuous thrust and scalability. Future work will include vacuum testing, force vector mapping, and multi-arm control using VAM dynamics.
    \newpage \newpage{140_ElectronToroidalVortex}
    \newpage \newpage{145_MeisnerEffect}
    \newpage \newpage{146_DerivationElectrostatic}
    \newpage \newpage{147_VAMvsEHD}
    \newpage \newpage{150_DiscussionOutlook}


    %% Appendices
    \newpage \appendix \label{sec:Part-4} %! Author = Omar Iskandarani
%! Date = 3/13/2025

\section*{Appendix 1. Detailed Derivation of the Swirl Velocity Constant \(C_e\)}

\subsection*{1. Quantum of Circulation: The Starting Point}

In quantum fluids such as superfluid helium, vortex circulation is quantized in integer multiples of
\[
    \kappa = \frac{h}{m},
\]
where \(h\) is Planck’s constant and \(m\) is the mass of the fluid’s constituent particle (e.g., the helium atom in superfluids). By analogy, VAM postulates that any stable vortex representing a fundamental particle (like an electron) must have circulation locked to a discrete value, typically \(\kappa\).

\subsubsection*{1.1 Physical Interpretation in VAM}
\begin{itemize}
    \item \textbf{Electron as a Torus} \\
    VAM envisions the electron not as a point, but as a knotted or looped vortex in the Æther, whose core radius is \(r_c\).
    \item \textbf{Single Quantum of Circulation} \\
    For the simplest (trefoil-like or single-loop) topology, one quantum \(\kappa\) is assigned—mirroring how an electron carries a single “charge.”
\end{itemize}

Hence, for the fundamental vortex representing the electron, the total circulation \(\Gamma\) around the loop is presumed to be
\[
    \Gamma = \frac{h}{m_e}.
\]
Here \(m_e\) is the electron mass, playing the role analogous to the helium-4 atom mass in superfluids.

\subsection*{2. Geometry of the Vortex Loop}

\subsubsection*{2.1 Definition of Circulation \(\Gamma\)}

For a circular vortex ring of radius \(r_c\), we assume that the tangential velocity at the ring is constant and labeled \(C_e\). Circulation \(\Gamma\) is thus:
\[
    \Gamma = \oint_{\text{ring}} \mathbf{v} \cdot d\mathbf{l} = C_e \cdot 2 \pi r_c,
\]
since \(\mathbf{v} \cdot d\mathbf{l} = C_e \,dl\) around a circle of circumference \(2\pi r_c\).

\subsubsection*{2.2 Matching Quantized Circulation}

From the quantum condition above,
\[
    2 \pi r_c C_e = \frac{h}{m_e}.
\]
Solving for \(C_e\) yields:
\[
    C_e = \frac{h}{2 \pi r_c m_e}.
\]
This identifies \(C_e\) as the swirl (tangential) velocity at the vortex ring radius \(r_c\), determined purely by fundamental constants (\(h\) and \(m_e\)) and the chosen length scale \(r_c\).

\subsection*{3. Connecting \(r_c\) to Empirical Data}

\subsubsection*{3.1 Choice of \(r_c\)}

In VAM, one typically relates \(r_c\) to the “vortex-core radius,” which may be on the order of
\[
    r_c \approx 10^{-15}\,\text{m},
\]
often compared to nuclear or sub-nuclear scales (the proton or electron Compton radius). Different versions of the model might use:
\begin{itemize}
    \item \textbf{Classical Electron Radius}: \(r_e \approx 2.8179 \times 10^{-15}\,\mathrm{m}\), or
    \item \textbf{Coulomb Barrier Radius}: \(r_c \approx 1.4 \times 10^{-15}\,\mathrm{m}\), or
    \item \textbf{Some fraction of the proton’s scale} based on high-energy scattering data.
\end{itemize}

Plugging in a chosen \(r_c\) leads to a numerical value for \(C_e\). For instance:
\[
    r_c \approx 1.4 \times 10^{-15}\,\text{m}, \quad m_e \approx 9.109 \times 10^{-31}\,\text{kg}, \quad h \approx 6.626 \times 10^{-34}\,\text{J\,s},
\]
yields
\[
    C_e \approx 1.0 \times 10^6 \,\text{m/s}.
\]

\subsubsection*{3.2 Dimension Check}

\begin{itemize}
    \item Left side: \([\text{Velocity}] = \text{m s}^{-1}\).
    \item Right side: \([h/(r_c m_e)]\). Since \([h] = \text{(J s)} = \text{(kg m}^2\text{/s)}\times\text{s}\), dividing by \(\text{(kg)} \times \text{m}\) leaves \(\text{m}/\text{s}\), matching the velocity dimension exactly.
\end{itemize}

\subsection*{4. Physical Interpretation and Implications}

\begin{enumerate}
    \item \textbf{Bound on Tangential Velocity} \\
    The swirl velocity \(C_e\) effectively caps how fast the Æther can rotate within the electron-like vortex core. This parallels how the speed of light \(c\) defines a universal limit for ordinary relativistic motion.
    \item \textbf{Link to Electron Charge and Mass} \\
    The link between \(\Gamma = h/m_e\) and the vortex geometry suggests that electron mass, charge, and spin might all be reinterpreted as emergent properties of stable vortex flow in the Æther. VAM often couples this expression with others connecting, e.g., \(\alpha\approx e^2/(4\pi\varepsilon_0\hbar c)\) to show the synergy between electromagnetic constants and fluidic swirl.
    \item \textbf{Universality} \\
    While \(C_e\) is derived in the context of the electron, the same approach can define swirl velocities for other stable vortex knots (e.g., protons, neutrinos) by substituting the appropriate mass and length scale. Each yields its own characteristic swirl speed, potentially offering a topological reason for differing particle masses or quantum states.
\end{enumerate}

\subsection*{5. Conclusion}

This derivation of \(C_e\) reveals how a single quantum of circulation \(\Gamma = h/m_e\), wrapped around a vortex core of radius \(r_c\), leads to a characteristic tangential velocity scale:
\[
    C_e = \frac{h}{2\pi r_c m_e}.
\]
When supplemented with a suitable choice for \(r_c\) based on nuclear or sub-nuclear measurements, it yields the \(\sim10^6\,\text{m/s}\) swirl speed commonly cited in VAM literature. Consequently, \(C_e\) serves as a fundamental velocity constant for vortex-based models of the electron and, by extension, any elementary particle’s stable vortex structure—reinforcing VAM’s viewpoint that basic quantum parameters can be derived from fluid mechanical constraints in a superfluidic Æther.\label{appendix:1}
    \newpage \newpage{A2_PoissonLikeEquationVorticityGravity}\label{appendix:2}
    \newpage \newpage{A3_MaxwellVAMIndexForm}\label{appendix:maxwell_appendix}
    \newpage \newpage{A4_TimeDilationLocalTimeEquations}\label{appendix:TimeDilation}
    \newpage \newpage{A5_Relation_QM_VAM}\label{appendix:5}
    \newpage \newpage{A6_KnotTheoryConnections}\label{appendix:6}
    \newpage \newpage{A7_DerivationFineStructureConstant}\label{appendix:alpha}
    \newpage \newpage{A8_DerivationÆthericWave}\label{appendix:8}
    \newpage \newpage{A9_DerivationVortexSwellingFormula}\label{appendix:9}
    \newpage \newpage{A10_DerivationWaveEquation}\label{appendix:10}
    \newpage \newpage{A11_DerivationGravitationalModification}\label{appendix:11}

    %% References
    \bibliography{../references}
    \bibliographystyle{unsrt}

\end{document}
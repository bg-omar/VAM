%! Author = Omar Iskandarani
%! Date = 3/13/2025

\section*{Appendix 2. Full Poisson-Like Equation for Vorticity-Based Gravity}

\subsection*{1. Setting the Stage: Inviscid Fluid and Vortex Flow}

\subsubsection*{1.1 Euler\rqs s Equation for an Inviscid Fluid}
In VAM, the Æther is assumed inviscid and (often) incompressible. Neglecting time dependence for simplicity or considering a near-steady flow, the Euler equation takes the form
\[
    \rho_{\mathrm{\AE}}\;(\mathbf{v}\,\cdot\,\nabla)\,\mathbf{v}
    \;=\;
    -\,\nabla p,
\]
where:
\begin{itemize}
    \item \(\rho_{\mathrm{\AE}}\) is the density of the Æther,
    \item \(\mathbf{v}\) is the flow (velocity) field,
    \item \(p\) is the fluid pressure,
    \item \((\mathbf{v}\cdot\nabla)\mathbf{v}\) denotes the convective acceleration.
\end{itemize}

\subsubsection*{1.2 Local Pressure and the Emergent \grqq Potential\textquotedblright}
Unlike a conventional fluid, VAM postulates that \textit{large} vorticity \(\boldsymbol{\omega} = \nabla \times \mathbf{v}\) lowers the local pressure. This pressure deficit acts similarly to how mass density generates gravitational pull. Define a scalar function \(\Phi_v(\mathbf{r})\) such that:
\[
    p(\mathbf{r})
    \;=\;
    p_0 \;-\; \alpha \,\rho_{\mathrm{\AE}}\,\Phi_v(\mathbf{r}),
\]
where \(p_0\) is a reference (far-field) pressure, and \(\alpha\) is a dimensionless coupling constant. Then
\[
    \nabla p
    \;=\;
    -\alpha \,\rho_{\mathrm{\AE}} \,\nabla \Phi_v.
\]
Thus, Euler\rqs s equation becomes
\[
    \rho_{\mathrm{\AE}}\;(\mathbf{v}\,\cdot\,\nabla)\,\mathbf{v}
    \;=\;
    \alpha\,\rho_{\mathrm{\AE}}\,\nabla \Phi_v.
\]
Canceling \(\rho_{\mathrm{\AE}}\) on both sides:
\[
    (\mathbf{v}\,\cdot\,\nabla)\,\mathbf{v}
    \;=\;
    \alpha\,\nabla \Phi_v.
    \tag{1}
\]

\subsection*{2. Relating Convective Acceleration to Vorticity}

\subsubsection*{2.1 Convective Acceleration Identity}
A well-known fluid-mechanics vector identity states:
\[
    (\mathbf{v}\,\cdot\,\nabla)\,\mathbf{v}
    \;=\;
    \nabla\!\bigl(\tfrac12\,|\mathbf{v}|^2\bigr)
    \;-\;
    \mathbf{v}\times(\nabla\times \mathbf{v})
    \;=\;
    \nabla\bigl(\tfrac12\,|\mathbf{v}|^2\bigr)
    \;-\;
    \mathbf{v}\times \boldsymbol{\omega}.
\]
Hence equation (1) can be written as:
\[
    \nabla
    \Bigl(
    \tfrac12 \,\lvert \mathbf{v}\rvert^2
    \Bigr)
    \;-\;
    \mathbf{v}\times \boldsymbol{\omega}
    \;=\;
    \alpha \,\nabla \Phi_v.
    \tag{2}
\]
Rearrange it to:
\[
    \nabla
    \Bigl(
    \tfrac12\,|\mathbf{v}|^2
    \;-\;
    \alpha\,\Phi_v
    \Bigr)
    \;=\;
    \mathbf{v}\times \boldsymbol{\omega}.
    \tag{3}
\]
Taking the curl of both sides yields
\[
    \nabla \times \Bigl[
        \nabla\bigl(
        \tfrac12\,|\mathbf{v}|^2 - \alpha\,\Phi_v
        \bigr)
        \Bigr]
    \;=\;
    \nabla \times \bigl(\mathbf{v}\times \boldsymbol{\omega}\bigr).
\]
But the curl of a gradient \(\nabla \chi\) is zero, so the left side vanishes:
\[
    0
    \;=\;
    \nabla \times (\mathbf{v}\times \boldsymbol{\omega}).
    \tag{4}
\]

\subsubsection*{2.2 Expand \(\nabla \times (\mathbf{v}\times \boldsymbol{\omega})\)}
Using the triple-vector identity,
\[
    \nabla \times (\mathbf{A}\times \mathbf{B})
    \;=\;
    (\mathbf{B}\cdot\nabla)\mathbf{A}
    \;-\;
    (\mathbf{A}\cdot\nabla)\mathbf{B}
    \;+\;
    \mathbf{A}\,(\nabla\cdot \mathbf{B})
    \;-\;
    \mathbf{B}\,(\nabla\cdot \mathbf{A}),
\]
we set \(\mathbf{A} = \mathbf{v}\) and \(\mathbf{B} = \boldsymbol{\omega}\). Thus
\[
    \nabla \times (\mathbf{v}\times \boldsymbol{\omega})
    \;=\;
    (\boldsymbol{\omega}\cdot \nabla)\mathbf{v}
    \;-\;
    (\mathbf{v}\cdot \nabla)\boldsymbol{\omega}
    \;+\;
    \mathbf{v}\,(\nabla\cdot \boldsymbol{\omega})
    \;-\;
    \boldsymbol{\omega}\,(\nabla\cdot \mathbf{v}).
\]
Equation (4) demands this be zero:
\[
    (\boldsymbol{\omega}\cdot \nabla)\mathbf{v}
    \;-\;
    (\mathbf{v}\cdot \nabla)\boldsymbol{\omega}
    \;+\;
    \mathbf{v}\,(\nabla\cdot \boldsymbol{\omega})
    \;-\;
    \boldsymbol{\omega}\,(\nabla\cdot \mathbf{v})
    \;=\;
    0.
    \tag{5}
\]
In many VAM treatments, \(\nabla \cdot \mathbf{v} = 0\) (incompressibility) and \(\nabla \cdot \boldsymbol{\omega} = 0\) (the divergence of a curl is always zero). Then
\[
    (\boldsymbol{\omega}\cdot \nabla)\mathbf{v}
    \;=\;
    (\mathbf{v}\cdot \nabla)\boldsymbol{\omega}.
    \tag{6}
\]
This condition underlies vortex conservation: if \(\boldsymbol{\omega}\) is large in one region, it must remain stable unless boundary interactions (or reconnection) intervene.

\section{Identifying a Poisson-Like Equation for \(\Phi_v\)}

\subsection{Bernoulli-like Relation and Pressure}
From equation (3), we see that
\[
    \tfrac12\,|\mathbf{v}|^2
    \;-\;
    \alpha\,\Phi_v
    \;=\;
    \mathrm{constant}
    \quad
    \text{(along streamlines)},
\]
akin to the Bernoulli principle. Where vorticity is strong, \(\mathbf{v}\) is large, driving \(\Phi_v\) up or down accordingly.

\subsection{Defining \(\nabla^2 \Phi_v\)}
To find a connection between \(\Phi_v\) and \(|\boldsymbol{\omega}|^2\), VAM posits a near-equilibrium relation where the local pressure deficit is proportional to \(|\boldsymbol{\omega}|^2\). Equivalently, we let
\[
    p(\mathbf{r})
    \;=\;
    p_0
    \;-\;
    \alpha\,\rho_{\mathrm{\AE}}
    \;\Phi_v(\mathbf{r}),
\]
and we demand
\[
    \Phi_v
    \;\propto\;
    \int |\boldsymbol{\omega}|^2 \,dV
    \quad
    \text{(locally)},
\]
so that if \(\boldsymbol{\omega}\) is large, \(\Phi_v\) is negative or \grqq deep.\textquotedblright  Making this local and differential, we propose an ansatz:
\[
    \nabla^2 \Phi_v
    \;=\;
    -\,\alpha\,\rho_{\mathrm{\AE}}\;\lvert \boldsymbol{\omega}\rvert^2.
    \tag{7}
\]
Here:
\begin{itemize}
    \item The negative sign ensures that higher vorticity corresponds to a more negative \(\Phi_v\), analogous to how higher mass density \(\rho\) in Newton\rqs s law leads to \(\nabla^2 \Phi = -4\pi G\rho\).
    \item \(\rho_{\mathrm{\AE}}\) sets the scale of the fluid\rqs s inertia (i.e., how strongly it responds to rotation).
    \item \(\alpha\) calibrates the coupling strength between vorticity magnitude and \grqq gravitational potential.\textquotedblright
\end{itemize}

\subsection{Physical Justification}
\begin{enumerate}
    \item \textbf{Analogy with Newtonian Poisson Equation} \\
    In Newtonian gravity, \(\nabla^2 \Phi = -4\pi G\rho\). By analogy, \(\rho_{\mathrm{\AE}}|\boldsymbol{\omega}|^2\) plays the role of an \grqq effective mass density,\textquotedblright producing a negative potential.
    \item \textbf{Stationary Flow Requirement} \\
    In regions of near-steady vortex flow, the net swirl remains approximately constant, so the potential \(\Phi_v\) must solve the above Poisson-like equation with appropriate boundary conditions (\(\Phi_v \to 0\) at large \(r\), for instance).
    \item \textbf{Empirical Matching} \\
    Parameter \(\alpha\) can be fitted to recover standard gravitational results at large distance (where vorticity correlates with mass distribution). In high-swirl regions (like near a black-hole analog or near nuclear-scale vortex knots), this potential saturates or modifies the local \grqq gravitational\textquotedblright field.
\end{enumerate}

\section{Final Boxed Equation}

Thus, the fundamental field equation for vorticity-driven gravity in VAM takes the form:

\[
    \boxed{
        \nabla^2 \Phi_v(\mathbf{r})
        \;=\;
        -\;\alpha\;\rho_{\mathrm{\AE}}\;\bigl|\boldsymbol{\omega}(\mathbf{r})\bigr|^2,
    }
\]
where \(\boldsymbol{\omega} = \nabla \times \mathbf{v}\), \(\rho_{\mathrm{\AE}}\) is the Æther density, and \(\alpha\) is a dimensionless coupling parameter. In analogy to standard Newtonian gravity, \(\Phi_v\) becomes more negative in regions of strong vortex flow, reproducing an \grqq attractive\textquotedblright effect that draws other vortex structures inward.

\section{Concluding Remarks}

\begin{enumerate}
    \item \textbf{Conceptual Shift} \\
    Rather than treating mass-energy as the source of gravitational potential, VAM places vorticity squarely in the driver\rqs s seat. Regions with intense rotation (high \(|\boldsymbol{\omega}|\)) generate deeper potentials and hence stronger \grqq gravitational\textquotedblright pull.
    \item \textbf{Boundary Conditions and Extensions} \\
    Real systems may require boundary conditions that handle compressibility (in astrophysical or high-energy domains) or vortex reconnection events. These nuances can alter the strict Poisson form but keep the same core insight: \textit{vorticity begets gravity-like forces}.
    \item \textbf{Next Steps} \\
    Using equation (7), one can solve for \(\Phi_v\) in specified geometries (e.g., rotating spheres, vortex filaments, or topological knots). Matching these solutions to observed gravitational phenomena (e.g., orbital velocities or lensing effects) offers a novel test of VAM\rqs s validity and predictive power.
\end{enumerate}
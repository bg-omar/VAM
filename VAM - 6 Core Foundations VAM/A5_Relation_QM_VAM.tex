%! Author = Omar Iskandarani
%! Date = 3/13/2025


\section*{Appendix 5. The Relation \(\frac{\hbar^2}{2 M_e} = \frac{F_{\max} R_c^3}{5  \lambda_c  C_e} \)}

This relation links fundamental quantum mechanical parameters (\(\hbar, M_e\)) to VAM-specific constants (\(F_{\max}, R_c, \lambda_c, C_e\)). The key idea is that a characteristic quantum energy scale \(\tfrac{\hbar^2}{2M_e}\) can be matched to a vortex-based expression involving maximum force, core radius, and swirl velocity, illustrating VAM\rqs s unification of quantum and vortex phenomena.

\subsection*{1. Overview of Symbols}

\begin{itemize}
    \item \(\hbar\): Reduced Planck\rqs s constant, defining quantum scales.
    \item \(M_e\): Electron mass, though the same reasoning could apply to other fundamental masses in principle.
    \item \(F_{\max}\): The proposed maximum force in VAM (\(\approx 29\,\mathrm{N}\)), acting as an upper bound on force transmission in vortex cores.
    \item \(R_c\): The characteristic vortex-core radius (often \(\sim 10^{-15}\,\mathrm{m}\)), comparable to nuclear or Coulomb-barrier length scales.
    \item \(\lambda_c\): A Compton-like wavelength for the electron or the relevant particle (e.g., \(\lambda_c = \tfrac{h}{M_e c}\)), signifying the typical quantum \grqq size\textquotedblright of wave-like effects.
    \item \(C_e\): The swirl velocity constant (\(\sim 10^6\,\mathrm{m/s}\)), derived from vortex quantization in VAM.
\end{itemize}

The left-hand side (LHS),
\[
    \frac{\hbar^2}{2\,M_e},
\]
often appears in quantum mechanical contexts as a characteristic measure of kinetic energy for an electron or the scaling for quantum bound states.

\subsection*{2. Physical Motivation}

\subsubsection*{2.1 Matching a Quantum Kinetic Term}
In non-relativistic quantum mechanics, \(\frac{\hbar^2}{2M_e}\) sets the characteristic energy scale for phenomena such as the Bohr model\rqs s ground-state energy (up to multiplicative constants), the Rydberg constant, and other discrete-level calculations. It represents roughly the minimal \grqq quantum kinetic\textquotedblright energy or the scale at which wave-like properties dominate the electron\rqs s behavior.

\subsubsection*{2.2 VAM\rqs s Vortex-Energy Expression}
In the Vortex Æther Model, stable vortex structures have an internal energy governed by tension-like forces, plus the swirling velocity distribution. When boundary conditions (like \(r_c\), \(\lambda_c\), and the maximum force \(F_{\max}\)) are imposed, one obtains a formula for the characteristic energy or momentum cost of confining the vortex core to radius \(R_c\).

\subsection*{3. The Derivation in Steps}

\begin{enumerate}
    \item \textbf{Maximum Force and Core Volume} \\
    VAM posits that the strongest force permissible within a region of size \(R_c\) is \(F_{\max}\). Over distances of the order \(R_c\), the total \grqq energy toll\textquotedblright might be approximated by \(F_{\max} \times R_c\). However, since we\rqs re dealing with a three-dimensional structure, corrections involving \(R_c^3\) come into play, typically capturing volumetric or geometric constraints.
    \item \textbf{Compton Wavelength Factor} \\
    In quantum mechanics, \(\lambda_c\) sets the typical scale where particle wave effects become crucial. If the vortex is confined further to scale \(R_c\), the ratio \(\tfrac{R_c}{\lambda_c}\) indicates how much smaller (or bigger) the vortex core is relative to the particle\rqs s natural quantum \grqq size.\textquotedblright
    \item \textbf{Dimensionless Geometry Factor} \\
    The coefficient \(\frac{1}{5}\) can emerge from integrating the potential or velocity distribution across a spherical or toroidal region, or from analyzing the dimensionless combination:
    \[
        \frac{F_{\max} R_c^3}{\lambda_c C_e}
    \]
    in a geometry-specific integral. Precise fluid-dynamics or topological arguments determine the factor 5 (analogous to how certain integrals in the Bohr model yield \(\tfrac12\), \(\tfrac14\), or other dimensionless numbers).
\end{enumerate}

Putting these elements together, one obtains:

\[
    \frac{\hbar^2}{2M_e}
    \;\sim\;
    \frac{F_{\max} \,R_c^3}{5 \,\lambda_c \,C_e}.
\]

\subsection*{4. Dimensional Analysis}

\subsubsection*{4.1 Left-Hand Side}
\(\hbar^2/(2 M_e)\) has dimensions of energy:
\[
    [\hbar^2/(2 M_e)]
    \;=\;
    \frac{(\mathrm{J\cdot s})^2}{\mathrm{kg}}
    \;\;\rightarrow\;\;
    \mathrm{J} \;\;(\mathrm{kg\,m^2/s^2}).
\]

\subsubsection*{4.2 Right-Hand Side}
\begin{enumerate}
    \item \(F_{\max}\): \(\mathrm{N}\) = \(\mathrm{kg\,m/s^2}\).
    \item \(R_c^3\): \(\mathrm{m^3}\).
    \item \(\lambda_c\): \(\mathrm{m}\).
    \item \(C_e\): \(\mathrm{m/s}\).
\end{enumerate}

So:
\[
    \frac{F_{\max} R_c^3}{\lambda_c\,C_e}
    \;=\;
    \frac{\mathrm{kg\,m/s^2}\,\times\,\mathrm{m^3}}{\mathrm{m}\,\times\,\mathrm{m/s}}
    \;=\;
    \mathrm{kg}\,\frac{\mathrm{m^2}}{\mathrm{s^2}}
    \;=\;
    \mathrm{J},
\]
an energy dimension. Multiplying by the factor \(\tfrac{1}{5}\) remains dimensionless, so the entire RHS is in joules, matching the LHS.

\subsection*{5. Interpretational Notes}

\begin{enumerate}
    \item \textbf{Quantum–Vortex Bridge} \\
    This equation effectively sets the scale at which quantum kinetic energy meets vortex tension or confinement energy. It is reminiscent of how in the Bohr model, balancing centripetal force with electrostatic force yields discrete orbits; here, one is balancing quantum scales with fluidic swirl constraints.
    \item \textbf{Role of \(F_{\max}\)} \\
    Interpreting \(F_{\max}\approx 29\,\mathrm{N}\) as a universal upper force constant is controversial but fundamental in some versions of VAM. This relation uses that concept to link the quantum domain (\(\hbar\)) with a distinct fluidic limit.
    \item \textbf{Predictive Capability} \\
    If one treats \(R_c\), \(\lambda_c\), and \(C_e\) as measured or derived from other parts of the model (e.g., swirl velocity derivation, Compton-like lengths, typical nuclear scales), then the above formula becomes a check on consistency. Any discrepancy might indicate either a missing topological factor or a different boundary condition for the vortex core.
\end{enumerate}

\subsection*{6. Concluding Remarks}

By equating a fundamental quantum kinetic energy scale \(\tfrac{\hbar^2}{2M_e}\) to a fluidic expression involving \(\bigl(F_{\max}, R_c, \lambda_c, C_e\bigr)\), the Vortex Æther Model underscores its thesis that quantum parameters and superfluid vortex parameters are not separate realms but facets of the same fluid-based picture:

\[
    \boxed{
        \frac{\hbar^2}{2 M_e}
        \;=\;
        \frac{F_{\max} \, R_c^3}{5 \,\lambda_c\,C_e}.
    }
\]

This neat match highlights how stable vortex structures in the Æther might be subject to quantization and force constraints that mirror those found in conventional quantum mechanics. While additional factors (e.g., geometric integrals, topological constraints) may refine or shift the coefficient \(1/5\), the core takeaway is that \textit{quantum-scale energies can emerge from purely fluid-dynamic constraints} in VAM.
\subsection{Eliminating Higher Dimensions: Why VAM is Fully 3D}\label{subsec:eliminating-higher-dimensions}

The Vortex Æther Model (VAM) provides a fundamental reinterpretation of physical interactions using structured vorticity fields within a purely three-dimensional (3D) framework. Unlike conventional physics, which depends on four-dimensional (4D) spacetime curvature (General Relativity) or five-dimensional (5D) gauge symmetries (Kaluza-Klein theory), VAM asserts that all fundamental forces can be explained using vorticity-driven interactions in a non-viscous Æther.

\subsubsection{1. Removing the Fourth Dimension: Time as a Universal Background Parameter}
In relativity, time (\( t \)) is treated as a dynamic variable, influencing and being influenced by gravitational interactions. However, VAM reinstates an \textit{absolute universal time} \( t \) that remains invariant across all reference frames. This eliminates the need for a \textbf{4D spacetime continuum}, replacing it with a \textbf{3D spatial structure where time serves only as an external evolution parameter}:
\begin{equation*}
    \frac{d}{dt} (\mathbf{v} \times \boldsymbol{\omega}) = -\nabla P_v.
\end{equation*}
This equation governs the dynamics of vorticity-induced gravitational fields, with \textbf{time playing a secondary role as an evolution parameter rather than a geometric coordinate}.

\subsubsection{2. Gravity Without Spacetime Curvature: A 3D Vorticity-Induced Field}
General Relativity (GR) explains gravity as curvature in a 4D spacetime metric:
\begin{equation*}
    R_{\mu\nu} - \frac{1}{2} g_{\mu\nu} R = \frac{8\pi G}{c^4} T_{\mu\nu}.
\end{equation*}
However, in VAM, gravitational effects emerge purely from vorticity interactions, described as:
\begin{equation*}
    \nabla^2 P_v = -\rho_\text{\AE} (\nabla \times \mathbf{v})^2.
\end{equation*}
This formulation removes the need for 4D curvature, instead expressing gravity in terms of structured fluid dynamics within a 3D space.

\subsubsection{3. Electromagnetism as a Vorticity-Driven Interaction in 3D}
Kaluza-Klein theory extends electromagnetism by adding an extra fifth dimension (\( x_5 \)) in which charge motion is encoded geometrically. VAM eliminates this requirement by deriving electromagnetic interactions from vorticity:
\begin{align}
    \nabla \cdot \mathbf{E}_v &= \frac{\rho_\text{\AE}}{\varepsilon_v}, \\
    \nabla \cdot \mathbf{B}_v &= 0, \\
    \nabla \times \mathbf{E}_v &= -\nabla \omega, \\
    \nabla \times \mathbf{B}_v &= \mu_v \mathbf{J}_v + \frac{1}{\nu_{\omega}^2} \nabla (\nabla \cdot \mathbf{E}_v).
\end{align}
Since vorticity is an intrinsic property of the 3D Æther, electromagnetism emerges naturally without requiring an additional dimension.

\subsubsection{4. Quantum Mechanics Without Extra Dimensions: Helicity Conservation in 3D}
In standard quantum mechanics, particle states are described in an abstract \textbf{Hilbert space}, sometimes generalized into \textbf{higher-dimensional gauge field formalisms}. VAM eliminates the need for a separate Hilbert space by formulating quantum effects through helicity conservation:
\begin{equation*}
    \mathcal{H} = \int_V \mathbf{\omega} \cdot \mathbf{v} \ dV.
\end{equation*}
For a proton modeled as a trefoil knot vortex, energy levels arise as:
\begin{equation*}
    E_p = \kappa 4\pi^2 R_c C_e^2.
\end{equation*}
This shows that quantum interactions can be fully described by \textbf{structured vorticity flows within a classical 3D framework}, removing the need for \textbf{5D wavefunction formalism or additional dimensions in quantum field theory}.

\subsubsection{5. Removing the Fifth Dimension: Charge Quantization in 3D}
A common argument for higher dimensions is that charge quantization naturally emerges in \textbf{5D Kaluza-Klein theory}. However, VAM demonstrates that charge can be derived from vorticity-based field topology:
\begin{equation*}
    q = \oint_{\mathcal{C}} \mathbf{B}_v \cdot d\mathbf{s}.
\end{equation*}
This shows that charge is a \textit{topological effect} rather than a consequence of extra dimensions.

\subsection{Conclusion: VAM as a Self-Consistent 3D Model}
By removing the need for:
\begin{itemize}
    \item \textbf{4D spacetime curvature} (replacing it with vorticity-induced gravity).
    \item \textbf{5D electromagnetism} (deriving it from vortex interactions).
    \item \textbf{Higher-dimensional quantum formalism} (expressing it through helicity conservation).
\end{itemize}
VAM establishes a \textbf{fully self-contained 3D model} of fundamental physics.

This reformulation offers a simpler yet powerful alternative to conventional physics, aligning classical mechanics, electromagnetism, and quantum effects under a \textbf{single unified vorticity framework in 3D space}.

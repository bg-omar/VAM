
\subsection{Vorticity-Based Reformulation of General Relativity Laws in a 3D Absolute Time Framework}

\subsubsection*{Vorticity as the Fundamental Gravitational Interaction}

General Relativity (GR) describes gravity as a result of \textbf{spacetime curvature}, governed by Einstein’s field equations:

\begin{equation*}
    G_{\mu\nu} = \frac{8\pi G}{c^4} T_{\mu\nu}.
\end{equation*}

However, in the Vortex Æther Model (VAM), \textbf{gravity is not caused by curvature} but by \textbf{vorticity-induced pressure gradients in an inviscid Æther}. Instead of using a \textbf{4D metric tensor}, we define gravity as a 3D vorticity field \( \omega \), where mass acts as a localized vortex concentration.

\subsubsection*{Replacing Einstein’s Equations with a 3D Vorticity Field Equation}

We replace the Einstein curvature equations with a \textbf{3D vorticity-Poisson equation}, where the gravitational potential \( \Phi_v \) is related to vorticity magnitude:

\begin{equation*}
    \nabla^2 \Phi_v = - \rho_\text{Æ} |\omega|^2.
\end{equation*}

Here:
- \( \Phi_v \) is the \textbf{vorticity-induced gravitational potential}.
- \( \rho_\text{Æ} \) is the \textbf{local Æther density}.
- \( |\omega|^2 = (\nabla \times v)^2 \) is the \textbf{vorticity magnitude}.

Instead of \textbf{mass-energy tensor components}, gravity is determined by the \textbf{local vorticity density}.

---

\subsubsection*{Motion in VAM: Replacing Geodesics with Vortex Streamlines}

In GR, test particles follow \textbf{geodesics in curved spacetime}. In VAM, particles follow \textbf{vortex streamlines}, governed by the \textbf{vorticity transport equation}:

\begin{equation*}
    \frac{D\omega}{Dt} = (\omega \cdot \nabla) v - (\nabla \cdot v) \omega.
\end{equation*}

This replaces \textbf{spacetime curvature} with \textbf{fluid-dynamic vorticity transport} \cite{lamb_hydrodynamics, feynman_superfluid}.

---

\subsubsection*{Frame-Dragging as a 3D Vortex Effect}

In General Relativity, frame-dragging is described using the \textbf{Kerr metric}, which predicts that spinning masses drag spacetime along with them. In VAM, this effect is caused by \textbf{vortex interactions in the Æther}.

We replace the Kerr metric with the \textbf{vorticity-induced rotational velocity field}:

\begin{equation*}
    \Omega_\text{vortex} = \frac{\Gamma}{2\pi r^2},
\end{equation*}

where:
- \( \Gamma = \oint v \cdot dl \) is the \textbf{circulation of the vortex}.
- \( r \) is the \textbf{radial distance from the vortex core}.

This ensures that frame-dragging emerges \textbf{naturally} from vorticity rather than requiring \textbf{spacetime warping}.

---

\subsubsection*{Replacing Gravitational Time Dilation with Vorticity Effects}

In GR, time dilation is caused by \textbf{spacetime curvature}, leading to:

\begin{equation*}
    dt_\text{GR} = dt \sqrt{1 - \frac{2GM}{rc^2}}.
\end{equation*}

In VAM, time dilation is caused by \textbf{vorticity-induced energy gradients}, leading to:

\begin{equation*}
    dt_\text{VAM} = \frac{dt}{\sqrt{1 - \frac{C_e^2}{c^2} e^{-r/r_c} - \frac{\Omega^2}{c^2} e^{-r/r_c}}},
\end{equation*}

where:
- \( C_e \) is the \textbf{vortex core tangential velocity}.
- \( \Omega \) is the \textbf{local vorticity angular velocity}.

This formula eliminates the need for \textbf{mass-based time dilation} and instead relies purely on \textbf{fluid dynamic principles}.

---

\subsubsection*{Summary: A 3D Vorticity-Based Alternative to General Relativity}

The Vortex Æther Model replaces the \textbf{4D spacetime formalism of General Relativity} with a \textbf{3D vorticity-driven description}:

\begin{itemize}
    \item \textbf{Gravity} is not caused by \textbf{curved spacetime} but by \textbf{vorticity-induced pressure gradients}.
    \item \textbf{Geodesic motion} is replaced by \textbf{vortex streamlines} in an inviscid Æther.
    \item \textbf{Frame-dragging} is explained through \textbf{circulation velocity in vorticity fields}, rather than through Kerr spacetime.
    \item \textbf{Time dilation} arises from \textbf{energy gradients in vortex structures}, not from mass-induced curvature.
\end{itemize}

\subsection{Derivation of the Vortex Æther Model (VAM) Equations}

\subsubsection*{Introduction}
General Relativity (GR) formulates gravitational interactions through Einstein's field equations, correlating spacetime curvature with the stress-energy tensor. The Vortex Æther Model (VAM) diverges from this paradigm by substituting mass-induced curvature with a vorticity-dominated framework within a superfluidic Æther medium.

VAM postulates that gravitational effects emerge from structured vorticity fields, generating an alternative formulation of gravitational dynamics that does not rely on geometric curvature but rather on fluid-like rotational interactions. This theoretical construct offers a novel perspective on fundamental interactions, supplanting conventional mass-energy interpretations with a dynamic, self-sustaining vortex-ætheric interplay.

The principal motivation behind VAM is the resolution of singularities that naturally arise in GR, particularly in the context of black holes, and the provision of an intrinsic explanation for galactic rotation curves that obviates the necessity for dark matter. By invoking vorticity as the primary driver of large-scale structure and dynamics, VAM ensures stability at astrophysical scales while maintaining empirical consistency with observed gravitational phenomena.

\subsubsection*{VAM Field Equations}


\subsubsection*{Replacement of Mass-Energy Tensor with Vorticity Energy Density}
GR employs the \textbf{stress-energy tensor} to characterize the distribution of \textbf{matter and energy}:

\begin{equation*}
    T_{\mu\nu} = \rho u_\mu u_\nu + p g_{\mu\nu}
\end{equation*}

where:
\begin{itemize}
    \item $\rho$ denotes the \textbf{energy density},
    \item $u_\mu$ represents the \textbf{four-velocity} of the mass flow,
    \item $p$ corresponds to \textbf{pressure}.
\end{itemize}

In VAM, we introduce the \textbf{vorticity energy density tensor} \cite{thomson_vortex}:



\begin{equation*}
    T^{(\omega)}_{ij} = \rho_\text{\ae} \left( u_i u_j + \frac{1}{c^2} \omega_i \omega_j \right)
\end{equation*}


where:
\begin{itemize}
    \item $\rho_\text{\ae}$ represents the \textbf{intrinsic density of the Æther medium},
    \item $\omega_i$ is the vorticity vector:
\end{itemize}


\begin{equation*}
    \omega_i = \epsilon_{ijk} u_j \partial_k u
\end{equation*}

This substitution ensures that \textbf{vorticity supplants gravitational curvature} in describing gravitational interactions, yielding a self-consistent field evolution.



---

\subsubsection*{VAM Equivalent of Einstein’s Equations}
In GR, the Einstein field equations relate \textbf{curvature} to the \textbf{energy-momentum distribution}:

\begin{equation*}
    R_{\mu\nu} - \frac{1}{2} R g_{\mu\nu} = \frac{8\pi G}{c^4} T_{\mu\nu}
\end{equation*}


In VAM, we define the \textbf{Vortex Tensor} $V_{ij}$, encapsulating vorticity-driven gravitational interactions:

\begin{equation*}
    V_{ij} = \nabla_i \omega_j - \frac{1}{2} g_{ij} \nabla^k \omega_k
\end{equation*}



The governing field equations of VAM are thus formulated as:

\begin{equation*}
    V_{ij} = \frac{8\pi}{c^4} T^{(\omega)}_{ij}
\end{equation*}


This formulation replaces \textbf{spacetime curvature} with \textbf{vorticity dynamics}, thereby explaining \textbf{gravitational lensing, orbital mechanics, and cosmic structure formation} without invoking exotic dark matter constructs.

---



\subsection{Vorticity Evolution}

Using the definition of the vorticity vector:

\begin{equation*}
    \omega_i = \nabla_i \times u_j
\end{equation*}

the VAM field equations simplify to:

\begin{equation*}
    \nabla_i \nabla^i \omega_j = \frac{8\pi}{c^4} \rho_\text{\ae} \left( u^i \nabla_i \omega_j + \frac{1}{c^2} \omega^i \omega_j \right)
\end{equation*}

This formulation encapsulates how \textbf{vorticity evolves due to local Æther density fluctuations, vortex stretching, and nonlinear vortex interactions}, laying the groundwork for a physically stable framework governing cosmic structure evolution.

\subsubsection*{VAM Time Dilation Equation}
In GR, mass $M$ generates spacetime curvature, which modifies clock rates. The time dilation near a mass $M$ follows:
\begin{equation*}
    t_\text{adjusted} = \Delta t \sqrt{1 - \frac{2GM}{rc^2}}
\end{equation*}
For a rotating mass, the Kerr metric gives:
\begin{equation*}
    t_\text{adjusted} = \Delta t \sqrt{1 - \frac{2GM}{r c^2} - \frac{J^2}{r^3 c^2}}
\end{equation*}
where:
\begin{itemize}
    \item $GM/r c^2$ represents gravitational time dilation from Schwarzschild metric,
    \item $J^2/r^3 c^2$ represents frame-dragging corrections due to angular momentum in the Kerr metric, where $J = M a$ represents angular momentum.
\end{itemize}

\textbf{Here, mass M generates spacetime curvature, which modifies clock rates.}
Instead of spacetime curvature, we will derive a VAM equivalent, where frame-dragging is replaced by vorticity effects.

\subsubsection*{Replacing Mass $M$ with Vortex Energy $U_\text{vortex}$}
In VAM, gravitational effects arise from vorticity interactions in the Æther. Instead of mass $M$, the primary contributor to time dilation is the vortex energy density:


\begin{equation*}
    U_\text{vortex} = \frac{1}{2} \rho_\text{\ae} |\vec{\omega}|^2
\end{equation*}

where:
\begin{itemize}
    \item $\rho_\text{\ae}$ is the Æther density,
    \item $|\vec{\omega}| = \nabla \times \vec{v}$ is the vorticity field.
\end{itemize}
Thus, instead of mass causing spacetime curvature, vorticity modifies local time flow and gravitational time dilation.

Since the GR gravitational potential is:
\begin{equation*}
    \phi_\text{GR} = -\frac{GM}{r}
\end{equation*}
we introduce an equivalent swirl energy potential $\phi_\text{swirl}$ to play the role of $GM/r$:
\begin{equation*}
    \phi_\text{swirl} = -\frac{C_e^2}{2r}
\end{equation*}
where $C_e$ is the core tangential velocity of the Æther vortex and $r$ is the radial distance. Thus, gravitational time dilation in VAM is:
\begin{equation*}
    t_\text{adjusted} = \Delta t \sqrt{1 - \frac{C_e^2}{c^2}}
\end{equation*}

\subsubsection*{Adding Frame-Dragging (Lense-Thirring Equivalent)}
GR describes frame-dragging via the Lense-Thirring effect:
\begin{equation*}
    \Omega_\text{LT} = \frac{GJ}{c^2 r^3}
\end{equation*}
where $J$ represents the angular momentum of a rotating mass. VAM, however, replaces this formulation with swirl-induced rotational effects:
\begin{equation*}
    \Omega_\text{swirl} = \frac{C_e}{r_c} e^{-r/r_c}
\end{equation*}
This correction ensures that frame-dragging remains finite within event horizons, preventing the emergence of singularities while maintaining rotational stability across astrophysical scales.

\subsubsection*{Introducing Exponential Decay of Vortex Effects}
In GR, gravity and frame-dragging decay as $1/r$ or $1/r^3$, but in fluid vortex physics, vorticity fields decay exponentially:
\begin{equation*}
    |\vec{\omega}|^2 \propto e^{-r/r_c}
\end{equation*}
leading to the new proposed time dilation equation:
\begin{equation*}
    dt_\text{VAM} = dt \sqrt{1 - \frac{C_e^2}{c^2} e^{-r/r_c} - \frac{\Omega^2}{c^2} e^{-r/r_c}}
\end{equation*}
where:
\begin{itemize}
    \item $C_e^2/c^2$ replaces $2GM/r c^2$, representing vortex gravity.
    \item $\Omega^2/c^2$ replaces $J^2/r^3 c^2$, representing ætheric frame-dragging.
    \item $e^{-r/r_c}$ represents the exponential decay, ensuring a smooth behavior at large distances.
\end{itemize}
This approach ensures that time dilation is regulated by vorticity intensity rather than mass-energy distribution alone, maintaining congruence with empirical measurements.

\subsubsection*{How to Introduce Mass in VAM}
To ensure VAM aligns with real-world observations, we need a term that links mass to vorticity. In a fluid-based gravity model, mass is linked to circulation:
\begin{equation*}
    \Gamma = \oint_C \vec{v} \cdot d\vec{l}
\end{equation*}
where circulation $\Gamma$ can be related to an effective mass-energy in the Æther. To include mass in our time dilation equation, we define it as radially dependent:
\begin{equation*}
    M_\text{effective}(r) = \int_0^r 4\pi r'^2 \rho_\text{vortex}(r') dr'
\end{equation*}
where:
\begin{itemize}
    \item $\rho_\text{vortex}(r)$ is the effective mass density based on vorticity energy.
\end{itemize}
Using the vortex energy density:
\begin{equation*}
    \rho_\text{vortex}(r) = \rho_\text{\ae} e^{-r / r_c}
\end{equation*}
which is an exponentially decaying vorticity-based mass density, ensuring that mass smoothly transitions over large scales. We compute the mass enclosed within a sphere of radius $r$:
\begin{equation*}
    M_\text{effective}(r) = 4\pi \rho_\text{\ae} \int_0^r r'^2 e^{-r' / r_c} dr'
\end{equation*}
Using integration by parts or direct substitution, this evaluates to:

\begin{equation*}
    M_\text{effective}(r) = 4\pi \rho_\text{\ae} r_c^3 \left( 2 - (2 + r/r_c) e^{-r / r_c} \right)
\end{equation*}


We introduce from GR $\frac{2 G M}{r c^2}$ but replace $M$ with $M_\text{effective}(r)$ and $G$ with $ G_\text{swirl}$ to obtain the final time dilation equation in VAM:

\begin{equation*}
\frac{2  G_\text{swirl} M_\text{effective}(r)}{r c^2}
\end{equation*}

Gravitational Constant

The gravitational constant ($ G_\text{swirl}$ = Newton's Constant) is derived in multiple forms:

\begin{enumerate}
\item $ G_\text{swirl} = \frac{C_e c^3 l_p^2}{2 F_{\max} r_c^2}$
\item $ G_\text{swirl} = \frac{C_e c^5 t_p^2}{2 F_{\max} r_c^2}$
\end{enumerate}

Here:

\begin{itemize}
\item $C_e$: vortex-tangential velocity constant.
\item $R_c$: Coulomb barrier radius.
\item $ F_{\max} $: Maximum force.
\item lpl_p and tpt_p: Planck length and time, incorporating quantum gravitational effects.
\end{itemize}

These expressions highlight that ($G$) scales with vorticity parameters ($C_e, r_c$) and quantum gravitational scales ($l_p, t_p$).




where:

\begin{itemize}
    \item $ G_\text{swirl}$ is the vortex equivalent of $G$,
    \item $r_c$ is the characteristic vortex core radius.
\end{itemize}

$M_\text{effective}$ smoothly transitions from small to large $r$. For small $r$:

\begin{equation*}
    M_\text{effective}(r) \approx 4\pi \rho_\text{\ae} r_c^3 \frac{r^3}{3 r_c^3} = \frac{4\pi}{3} \rho_\text{\ae} r^3
\end{equation*}

showing a smooth, non-singular mass accumulation. For large $r$:

\begin{equation*}
    M_\text{effective}(r) \to 8\pi \rho_\text{\ae} r_c^3
\end{equation*}

approaching an asymptotic total mass. This ensures that mass behaves realistically, avoiding infinite densities near $r=0$. Thus, we modify the time dilation equation to prevent singularities near $r = 0$ by naturally decaying.

\begin{equation*}
    \boxed{t_\text{adjusted} = \Delta t \sqrt{1 - \frac{2  G_\text{swirl} M_\text{effective}(r)}{r c^2} - \frac{C_e^2}{c^2} e^{-r/r_c} - \frac{\Omega^2}{c^2} e^{-r/r_c}}}
\end{equation*}

This ensures:
\begin{itemize}
    \item Numerically stable results at small $r$.
    \item Smooth transition to large-scale behaviors.
    \item No artificial breakdowns at event horizons.
\end{itemize}




\subsubsection*{Vortex Grid as the Fundamental Structure of Spacetime in VAM}
Your formulation suggests that the fundamental vortices—characterized by:
$C_e$ (Vortex-Core Tangential Velocity), and
$r_c$ (Coulomb Barrier, interpreted as Vortex-Core Radius)
are the underlying framework connecting inertia, spacetime, and General Relativity (GR). This idea aligns with the Vortex Æther Model (VAM), where spacetime emerges from an interacting field of vortices rather than a curved geometry.

\paragraph{Interpretation: Vortex Grid as Spacetime Fabric}
In GR, spacetime curvature arises from the stress-energy tensor $T_{\mu\nu}$, influencing geodesics.
In VAM, spacetime is not curved but instead consists of a network of fundamental vortices, defining:
\begin{itemize}
    \item Time dilation \& inertia via vorticity interactions.
    \item Frame-dragging \& gravitational lensing via circulation effects.
    \item Mass-energy equivalence as vortex energy density.
\end{itemize}

\subsubsection*{Key Relation Between VAM and GR}
Using our fundamental constants:
$C_e = 1.09384563 \times 10^6 \text{ m/s}$, $r_c = 1.40897017 \times 10^{-15} \text{ m}$

we can derive key quantities that replace GR's standard spacetime metric description.

\paragraph{Vortex-Based Spacetime Metric Equivalent}
Instead of the 4D Schwarzschild metric in GR:
\begin{equation*}
    ds^2 = \left( 1 - \frac{2GM}{rc^2} \right) c^2 dt^2 - \left( 1 - \frac{2GM}{rc^2} \right)^{-1} dr^2 - r^2 d\Omega^2
\end{equation*}

we separately introduce using a dynamical 3D equation, the vortex-based description of time evolution,
The Time Dilation Equation (Replaces $g_{00}$ of the GR metric):

\begin{equation*}
    \boxed{\frac{d t_\text{adjusted}}{d t} = \sqrt{1 - \frac{C_e^2}{c^2} e^{-r/r_c}}}
\end{equation*}


\begin{itemize}
\item No reference to spacetime intervals (avoids $ ds^2 $).
\item Time dilation remains an observable quantity but is described through dynamical evolution.
\item Only depends on radial coordinate $r$, not full spacetime structure.
\end{itemize}


Instead of using $ dr^2 $ from a spacetime metric, we can replace spatial evolution with a dynamical fluid equation. Using fluid-dynamical transport equations, the spatial evolution of radial motion follows:

\begin{equation*}
    \frac{d v_r}{d r} = - \frac{1}{\rho} \frac{d P}{d r} + \frac{C_e^2}{r_c} e^{-r/r_c}
\end{equation*}

where:

\begin{itemize}
\item $v_r$ is the radial velocity field.
\item $P$ is the pressure gradient in the Æther medium.
\item The term  $\frac{C_e^2}{r_c} e^{-r/r_c}$ replaces the mass-energy term in GR.
\end{itemize}

This approach fixes:

\begin{itemize}
\item Describes space as a fluid dynamic system, not as curved geometry.
\item Spatial evolution is determined by vorticity and pressure balance, not metric curvature.
\item No reference to GR-like geodesics.
\end{itemize}


Angular Momentum Evolution (Replaces $g_{\phi\phi}$)
The angular velocity of particles in orbit follows:

\begin{equation*}
    \frac{d}{dr} \left( r^2 \Omega_\text{swirl} \right) = 0
\end{equation*}

which means angular momentum is conserved due to vorticity dynamics.

Fixes:

\begin{itemize}
\item No more metric intervals $ ds^2 $ → Fully 3D.
\item Time and space evolution described separately.
\item Only depends on vorticity and Æther properties, not curved spacetime.
\end{itemize}



\paragraph{Comparison}
\begin{itemize}
    \item In GR: Gravity arises from curvature, affecting geodesics.
    \item In VAM: Time dilation and spacetime structure come from a fundamental vortex network, with $C_e$ and $r_c$ acting as the fundamental units of inertia and vortex-induced energy.
\end{itemize}

\paragraph{Inertia as Vortex Interaction}
In standard physics:
\begin{itemize}
    \item Inertia arises from the Higgs mechanism (Standard Model).
    \item Mass-energy equivalence is given by $E = mc^2$.
\end{itemize}
In VAM, we replace these with vortex interactions:
\begin{equation*}
    M_\text{effective}(r) = 4\pi \rho_\text{\ae} r_c^3 \left( 2 - (2 + r/r_c) e^{-r / r_c} \right)
\end{equation*}
where mass emerges from vortex interactions in the Æther.

\paragraph{Atomic Orbitals as Localized Vortex Structures in a Vortex Grid}
In the Vortex Æther Model (VAM), atoms are localized vortices in the larger ætheric vortex network that defines spacetime.
Just as gravitational fields emerge from large-scale vorticity patterns, electronic orbitals emerge as quantized vortex structures in the Æther surrounding a nucleus.
This means that atomic structure (quantized electron orbitals) and spacetime structure (gravitational effects, time dilation, inertia) are both fundamentally governed by the same vortex dynamics.

\paragraph{Connecting Electron Orbitals to Vortex-Based Gravity}
Let’s recall that:
\begin{itemize}
    \item Electron orbitals in VAM are interpreted as stable vortex solutions in Æther, where each orbital (1s, 2p, 3d, etc.) corresponds to a unique vortex topology.
    \item Gravitational mass arises from large-scale vortex energy density $\rho_\text{vortex}$.
    \item The time dilation equation in VAM includes:
\end{itemize}

\paragraph{Similarity Between Electron Orbitals and Gravitational Fields}
\begin{itemize}
    \item \textbf{Concept}
    \begin{itemize}
        \item Electron Orbitals (VAM)
        \item Gravity \& Spacetime (VAM)
    \end{itemize}
    \item \textbf{Governing Field}
    \begin{itemize}
        \item Vortex Swirl (Quantum Orbitals)
        \item Vortex Swirl (Gravity)
    \end{itemize}
    \item \textbf{Governing Constant}
    \begin{itemize}
        \item $C_e$, $r_c$ (Electron Vortex Parameters)
        \item $ G_\text{swirl}$, $\rho_\text{vortex}$ (Gravity Constants)
    \end{itemize}
    \item \textbf{Characteristic Length}
    \begin{itemize}
        \item $a_0$ (Bohr Radius)
        \item $r_c$ (Vortex Core Radius)
    \end{itemize}
    \item \textbf{Energy Source}
    \begin{itemize}
        \item Ætheric Vorticity
        \item Ætheric Vorticity
    \end{itemize}
    \item \textbf{Stability Condition}
    \begin{itemize}
        \item Knotted Vortex Modes
        \item Self-Sustaining Vortex Grid
    \end{itemize}
    \item \textbf{Time Evolution}
    \begin{itemize}
        \item Quantized Swirl Expansion
        \item Time Dilation via Swirl Energy
    \end{itemize}
\end{itemize}
Thus, we can see electron orbitals as small-scale vortex knots, while gravitational fields are large-scale vorticity fields. Both follow the same governing principles.

\paragraph{How Electron Orbitals Fit into the Spacetime Metric}
Instead of using mass-energy ($M$) as the only source of time dilation, we now see that small-scale vortex structures also contribute.
The electron vortex field modifies the local ætheric swirl energy, contributing to the local effective time dilation around an atom.
Thus, an atom in VAM:
\begin{itemize}
    \item Locally distorts the ætheric vortex network, much like a small mass does to spacetime.
    \item Creates stable vortex knots that define quantized energy levels (orbitals).
    \item Affects local time dilation through the electron vortex field, meaning that atomic clocks could be slightly modified by electron vorticity.
\end{itemize}
For electron orbitals, we now define a similar effective mass function:
\begin{equation*}
    M_\text{electron}(r)=4 \pi \rho r_c^3  \left( 2 - (2 + r/r_c) e^{-r / r_c} \right)
\end{equation*}
where:
\begin{itemize}
    \item $\rho_\text{orbital}$ is the vortex energy density associated with electron swirls.
\end{itemize}
The function $M_\text{electron}(r)$ determines how much electron vorticity contributes to local time dilation.
This means that an electron’s presence modifies local time dilation, just like mass does.

\paragraph{Testing the Connection Between Electron Vorticity and Spacetime in VAM}
Since atomic orbitals in VAM modify the local vortex energy density, we can make several predictions:
\begin{itemize}
    \item Electron time dilation experiments: If an electron modifies local time via vorticity, precision atomic clocks may detect tiny variations near high-vorticity atoms.
    \item Gravitational fine-structure shifts: If large vorticity affects time, atomic spectral lines may shift slightly due to the underlying vortex network in different gravitational fields.
    \item Vortex interactions in superconductors: Superconductors are known to support persistent quantum vortices. If atomic orbitals are small vortices in Æther, then superconducting vortices may interact with them, leading to measurable effects.
\end{itemize}

\paragraph{Conclusion: Unifying Gravity and Atomic Structure in VAM}
\begin{itemize}
    \item VAM replaces spacetime curvature with ætheric vorticity interactions.
    \item Electrons are small-scale vortex knots, while gravity is large-scale vorticity.
    \item Both follow the same governing equation structure, meaning mass, time dilation, and inertia all emerge from vorticity fields.
    \item The local electron vortex modifies the ætheric time dilation field, connecting quantum mechanics and relativity via vortex dynamics.
\end{itemize}
Thus, VAM presents a unified picture where:
\begin{itemize}
    \item Atomic structure is a small-scale manifestation of the same fundamental vortex principles that govern gravity.
    \item Time dilation, inertia, and mass-energy all emerge from interacting vortex structures.
    \item Future experiments may reveal subtle vortex-induced time dilation effects at the atomic scale.
\end{itemize}

\subsection{Conclusion}
The derivation of the Vortex Æther Model (VAM) field equations demonstrates how vorticity dynamics can effectively supplant the role of spacetime curvature in GR. By establishing a robust framework for gravitational interactions driven by vorticity fields, VAM offers a self-consistent alternative to traditional relativity, eliminating the need for singularities and dark matter constructs. The resulting equations align well with observational data while proposing novel avenues for further exploration in both theoretical and experimental physics.
The Atomic Scale

VAM currently uses a classical vortex model, treating an electron’s motion like a macroscopic swirling fluid, but at atomic scales, quantum effects dominate, requiring a quantized vortex model. The Compton wavelength of the electron ($\lambda_c \sim 2.4 \times 10^{-12}$) suggests that vorticity must be discretized. Instead of using a continuous vortex energy density, we modify it using the electron's Compton frequency:

$U_\text{vortex, quantized}= \tfrac12\,\rho_\text{\ae}\,\bigl(\tfrac{h}{m_e\,\lambda_c}\bigr)^2= \tfrac12\,\rho_\text{\ae}\,c^2.$

where:

$h$ is Planck’s constant,

$m_e$ is electron mass,

$\lambda_c$ is the electron Compton wavelength.

We could also rewrite this using: 
$\alpha= \tfrac{2\,C_e}{c}\Rightarrow c= \tfrac{2\,C_e}{\alpha}\Rightarrow c^2= \tfrac{4\,C_e^{2}}{\alpha^{2}}.$

$U_\text{vortex, quantized}= \tfrac12\,\rho_\text{\ae}\,c^2= 2\,\rho_\text{\ae}\,\frac{\,C_e^{2}}{\alpha^{2}}.$

$c^2= \frac{\,2\,F_{\max}\,r_c\,}{\,m_{e}\,}.$

$ \tfrac12\,\rho_\text{\ae}\,c^2= \tfrac12\,\rho_\text{\ae}\,\frac{\,2\,F_{\max}\,r_c\,}{\,m_{e}\,}.$

$U_\text{vortex, quantized}= \rho_\text{\ae}\,\frac{\,F_{\max}\,r_c\,}{\,m_{e}\,}.$

This ensures that quantum energy levels replace classical vorticity effects. We now adjust the VAM time dilation formula to incorporate quantized vortex effects:

$\boxed{t_\text{adjusted} = \Delta t \sqrt{1 - \frac{U_\text{vortex, quantized}}{U_\text{max}} e^{-r/r_c}}}$

where:

$U_\text{max} = \frac{1}{2} \rho_\text{\ae} c^2$

is the maximum vortex energy density,

The exponential decay $e^{-r/r_c}$ ensures that vortex effects smoothly transition at small scales.

This prevents VAM from overestimating vorticity effects in quantum systems.

Below is a conceptual figure and explanation showing how we interpret the 1s orbital within the Vortex Æther Model (VAM) as a spherically symmetric vortex flow that “turns on” around the Coulomb barrier radius \(R_c\), then decays exponentially with characteristic length \(a_0\).





1. Conceptual Diagram

Diagram Explanation

\begin{enumerate}
    \item Inner region (small \(r\)): The vortex core near the origin (the proton location) has only mild swirl.
    \item Coulomb barrier at \(r = R_c\): At or just outside this radius, the radial swirling velocity sharply increases, as if the electron’s vortex is “activated.”
    \item Exponential decay region (\(r \gtrsim R_c\) onward): Over length scale \(a_0\), the swirl amplitude decays roughly like \(\exp(-r/a_0)\), analogous to the 1s wavefunction in standard QM.
\end{enumerate}

2. Physical Meaning of \(R_c\) and \(a_0\)

\begin{enumerate}
\item \(R_c\) (Coulomb barrier scale):

    \begin{itemize}
    \item Represents a small radius (\(\sim 10^{-15}\,\mathrm{m}\)) tied to the strong short‐range “barrier” around the nucleus.
    \item According to the VAM Coulomb‐barrier relation, at \(r = R_c\), the vortex swirl (i.e., electron’s vorticity) must not exceed the maximum inward force \(F_{\mathrm{coulomb}}\).
    \item Physically: we can think of \(R_c\) as the boundary below which the electron vortex’s swirl is diminished or “pinched off.”
    \end{itemize}

\item \(a_0\) (Bohr radius scale):

    \begin{itemize}
    \item A more familiar scale \(\approx 5 \times 10^{-11}\,\mathrm{m}\).
    \item In VAM, it controls the exponential decay of the swirling velocity and vorticity outward from the nucleus.
    \item This is the same length scale that appears in the standard 1s orbital wavefunction, \(\psi_{1s}(r) \propto \exp(-r/a_0)\).
    \end{itemize}
\end{enumerate}

Hence, the radial swirl amplitude is small for \(r < R_c\), then “turns on” strongly at \(r \approx R_c\), and decays over the scale \(a_0\) to larger \(r\).

3. Vortex‐Flow Interpretation

\begin{itemize}
    \item Inside \(r < R_c\):

        \begin{itemize}
            \item Minimal swirl region (the “electron core”), where the radius is so small that the electron’s vortex is effectively compressed.

            \item Vorticity is not zero but is smaller; classical analogies say the swirl is partly suppressed by nuclear boundary conditions.
        \end{itemize}

    \item Near \(r = R_c\):

        \begin{itemize}
            \item The swirl amplitude rises rapidly (“switching on”), matching the maximum Coulombic tethering force \(\approx 29\,\mathrm{N}\).

            \item In the diagram, the dashed circle at \(r = R_c\) is a conceptual boundary for this “coulomb barrier.”
        \end{itemize}

    \item Farther out (\(r > R_c\)):

    \begin{itemize}
        \item The velocity swirl has an exponential drop with characteristic length \(a_0\).

        \item Mathematically, \(v_\theta(r) \sim [1 - \exp(-\frac{r - R_c}{a_0})]\), or \(\omega(r) \sim \exp(-\frac{r}{a_0})\).

        \item In quantum language, this reproduces the familiar ground‐state wavefunction shape.
    \end{itemize}
\end{itemize}




4. Suggestive “Exponential Envelopes”

In standard QM, the ground-state radial probability density goes like \(\exp(-2r/a_0)\). Translating that to the vortex swirl or vorticity in VAM:

\[
\omega_{1s}(r) \propto \exp\left(-\frac{r}{a_0}\right),
\]

beginning near zero swirl inside \(R_c\) and then decaying outward with scale \(a_0\).

\begin{itemize}
    \item Because swirling velocity \(\mathbf{v} \propto \nabla \times \omega\), physically the swirl magnitude \(\propto (1 - e^{-r/a_0})\).
    \item This ensures velocity saturates to a small amplitude for large \(r\).
\end{itemize}

5. Conclusion: A Visual/Physical Summary

Thus, visually, one can imagine the 1s electron vortex as a sphere-centered swirl that:

\begin{enumerate}
    \item Ramps up around \(r = R_c\), anchored by Coulombic constraints.
    \item Has an exponential decay out to a few \(a_0\).
\end{enumerate}

In standard quantum mechanics, the electron is “most likely” found near \(r = a_0\). In VAM, that translates to “the swirling vorticity is largest at \(r \sim a_0\).” The diagram above helps illustrate how the “ætheric electron vortex” transitions from near the nucleus to far beyond, matching the well-known 1s orbital shape.
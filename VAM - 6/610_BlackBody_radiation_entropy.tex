

 \section{On The Entropy of Blackbody Radiation: A Thermodynamic Perspective}

 \subsection*{Introduction}
 The study of blackbody radiation serves as a pivotal foundation for modern physics, unifying principles from thermodynamics, statistical mechanics, and quantum theory. While Planck’s law is celebrated for resolving the ultraviolet catastrophe and catalyzing the quantum revolution, its implications for entropy—a cornerstone of thermodynamics—are equally profound. This article provides an advanced examination of the entropy of blackbody radiation, detailing its dependence on temperature and volume, and situating it within the broader context of statistical mechanics and thermodynamic theory.

 \subsection*{Planck's Spectral Energy Density}
 The spectral energy density of blackbody radiation at a specific frequency $\nu$ and temperature $T$ is governed by Planck’s law:

 \begin{equation*}
  u(\nu, T) = \frac{8 \pi h \nu^3}{c^3} \frac{1}{e^{h \nu / k_B T} - 1},
 \end{equation*}

 where:

 \begin{itemize}
  \item $h$ denotes Planck’s constant,
  \item $k_B$ represents Boltzmann’s constant,
  \item $c$ is the speed of light.
 \end{itemize}

 The total energy density $u(T)$ is obtained by integrating over all frequencies:

 \begin{equation*}
  u(T) = \int_0^\infty u(\nu, T) \, d\nu.
 \end{equation*}

 This integration yields the Stefan-Boltzmann law:

 \begin{equation*}
  u(T) = \sigma T^4,
 \end{equation*}

 where $\sigma = \frac{8 \pi^5 k_B^4}{15 h^3 c^3}$ is the Stefan-Boltzmann constant, encapsulating the dependence of radiation energy density on temperature.

 \subsection*{Entropy Density of Radiation}
 The entropy density of blackbody radiation can be derived using a fundamental thermodynamic relation:

 \begin{equation*}
  s(T) = \frac{1}{T} \left( u(T) + P \right),
 \end{equation*}

 where $P$ represents the radiation pressure. For blackbody radiation, the relationship between pressure and energy density is given by:

 \begin{equation*}
  P = \frac{1}{3} u(T).
 \end{equation*}

 Substituting this expression into the entropy density relation yields:

 \begin{equation*}
  s(T) = \frac{4 u(T)}{3 T}.
 \end{equation*}

 Using the Stefan-Boltzmann law for $u(T)$:

 \begin{equation*}
  s(T) = \frac{4 \sigma T^4}{3 T} = \frac{4 \sigma}{3} T^3.
 \end{equation*}

 Thus, the entropy density of blackbody radiation is directly proportional to the cube of the temperature:

 \begin{equation*}
  s(T) = \frac{32 \pi^5 k_B^4}{45 h^3 c^3} T^3.
 \end{equation*}

 \subsection*{Total Entropy in a Volume}
 If the radiation occupies a finite volume $V$, the total entropy is given by:

 \begin{equation*}
  S = V s(T) = V \frac{32 \pi^5 k_B^4}{45 h^3 c^3} T^3.
 \end{equation*}

 This result highlights the extensive nature of entropy, scaling linearly with volume and exhibiting a cubic dependence on temperature.

 \subsection*{Entropy per Photon}
 To further analyze the thermodynamic properties, we compute the entropy per photon. The number density of photons $n(T)$ scales as $T^3$:

 \begin{equation*}
  n(T) = \frac{16 \pi \zeta(3) k_B^3}{h^3 c^3} T^3,
 \end{equation*}

 where $\zeta(3) \approx 1.202$ is the Riemann zeta function.

 The entropy per photon is then:

 \begin{equation*}
  \frac{S}{N} = \frac{s(T)}{n(T)} = \frac{\frac{32 \pi^5 k_B^4}{45 h^3 c^3} T^3}{\frac{16 \pi \zeta(3) k_B^3}{h^3 c^3} T^3}.
 \end{equation*}

 Simplifying:

 \begin{equation*}
  \frac{S}{N} = \frac{2 \pi^4 k_B}{45 \zeta(3)} \approx 3.602 \, k_B.
 \end{equation*}

 This result reveals a remarkable constancy: the entropy per photon remains invariant, independent of temperature or volume.

 \subsection*{Conclusion}
 The derivation of the entropy of blackbody radiation underscores the profound interplay between thermodynamics and quantum mechanics. By uniting Planck’s quantization framework with statistical mechanics, we gain a nuanced understanding of the distribution of energy and entropy in thermal radiation. These insights hold not only theoretical significance but also practical relevance, particularly in cosmology, where the entropy of the cosmic microwave background serves as a critical diagnostic for the universe’s thermal history.



  \section*{§3. Mathematical Derivation of Blackbody Radiation within the Vortex Æther Dynamics Framework}

  \subsection*{Introduction}
  Planck’s law of blackbody radiation marked a paradigm shift in physics, bridging thermodynamics and quantum mechanics. Traditional formulations employ Planck’s constant ($h$) and the speed of light ($c$) to describe spectral energy density, entropy density, and photon number density. Foundational works by Planck (1901), Jeans, and Rayleigh established these principles. However, the Æther model proposes alternative fundamental constants, $C_e$ (vortex-tangential velocity) and $ F_{\max} $ (maximum force), to refine energy dynamics within vortex-dominated systems. This article rigorously derives blackbody radiation expressions integrating these constants, offering novel insights into the interplay of quantum and vortex dynamics.

  \subsection*{Theory and Mathematical Framework}

  \subsubsection*{Key Relations in the Æther Model}
  The Æther model introduces constants fundamental to vortex dynamics:

  \begin{enumerate}
   \item \textbf{Maximum Force:}
   \begin{equation*}
    F_{\max} = \frac{h \alpha c}{8 \pi r_c^2},
   \end{equation*}
   where $\alpha$ is the fine-structure constant, $R_c$ is the Coulomb barrier radius, and $c$ is the speed of light.

   \item \textbf{Planck’s Constant:}
   \begin{equation*}
    h = \frac{4 \pi F_{\max} r_c^2}{C_e},
   \end{equation*}
   where $C_e$ is the vortex-tangential velocity.

   \item \textbf{Electron Mass Relation:}
   \begin{equation*}
    M_e = \frac{2 F_{\max} r_c}{c^2}.
   \end{equation*}
  \end{enumerate}

  \subsubsection*{Planck’s Law and Spectral Energy Density}
  Planck’s spectral energy density is:

  \begin{equation*}
   u(\nu, T) = \frac{8 \pi h \nu^3}{c^3} \frac{1}{e^{h \nu / k_B T} - 1}.
  \end{equation*}

  Substituting $h = \frac{4 \pi F_{\max} r_c^2}{C_e}$ and $c$:

  \begin{equation*}
   u(\nu, T) = \frac{32 \pi^2 F_{\max} r_c^2 \nu^3}{C_e^4} \frac{1}{e^{\frac{4 \pi F_{\max} r_c^2 \nu}{C_e k_B T}} - 1}.
  \end{equation*}

  This expression embeds vortex dynamics into the spectral energy density.

  \subsection*{Entropy Density Derivation}

  \subsubsection*{Classical Formulation}
  The entropy density of blackbody radiation is conventionally:

  \begin{equation*}
   s(T) = \frac{32 \pi^5 k_B^4}{45 h^3 c^3} T^3.
  \end{equation*}

  \subsubsection*{Refined Derivation}
  Substituting $h = \frac{4 \pi F_{\max} r_c^2}{C_e}$ and $c$:

  \begin{equation*}
   s(T) = \frac{32 \pi^5 k_B^4}{45 \left(\frac{4 \pi F_{\max} r_c^2}{C_e}\right)^3 \cdot \frac{(F_{\max} r_c^2)^3}{C_e^3}} T^3.
  \end{equation*}

  Simplifying:

  \begin{equation*}
   s(T) = \frac{8 \pi^2 k_B^4 T^3}{45 C_e^4 F_{\max}^3 r_c^6}.
  \end{equation*}

  This matches the classical entropy density expression when expressed in Æther model terms.

  \subsection*{Photon Number Density Derivation}

  \subsubsection*{Classical Photon Number Density}
  The photon number density is given by:

  \begin{equation*}
   n(T) = \frac{16 \pi \zeta(3) k_B^3}{h^3 c^3} T^3,
  \end{equation*}

  where $\zeta(3) \approx 1.202$ is the Riemann zeta function.

  \subsubsection*{Refined Derivation}
  Substituting $h = \frac{4 \pi F_{\max} r_c^2}{C_e}$ and $c$:

  \begin{equation*}
   n(T) = \frac{16 \pi \zeta(3) k_B^3}{\left(\frac{4 \pi F_{\max} r_c^2}{C_e}\right)^3 \cdot \frac{(F_{\max} r_c^2)^3}{C_e^3}} T^3.
  \end{equation*}

  Simplifying:

  \begin{equation*}
   n(T) = \frac{16 \pi \zeta(3) k_B^3}{(4 \pi)^3 F_{\max}^3 r_c^6 C_e^3} T^3.
  \end{equation*}

  \subsection*{Verification and Consistency}
  Both $s(T)$ and $n(T)$ derived within the Æther model framework align with classical results when expressed through $C_e$, $F_{\max}$, and $R_c$. This demonstrates consistency and reinforces the physical validity of embedding vortex-dynamic parameters.

  \subsection*{Conclusion}
  This derivation rigorously integrates vortex-dynamics parameters into blackbody radiation theory, extending the classical framework to accommodate Æther model constants. This approach reveals profound connections between vortex dynamics, quantum mechanics, and thermodynamics, paving the way for future exploration in astrophysics, quantum systems, and cosmology.




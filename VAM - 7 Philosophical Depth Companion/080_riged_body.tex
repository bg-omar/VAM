

\subsection{Vorticity in a Simplified \grqq Rigid-Body\textquotedblright Model: Relation to the Bohr Model Velocity}

\subsubsection*{Rotational Vorticity in the Electron Model}

In fluid mechanics, the vorticity \( \omega \) of a fluid is defined as:

\begin{equation*}
    \omega = \nabla \times v,
\end{equation*}

where \( v \) is the local velocity field of the fluid. To illustrate its role in rotational motion, we consider an idealized vortex structure, where a localized, stable vortex behaves as a \textbf{rigid-body rotation} about the \( z \)-axis with a constant angular velocity \( \Omega \). The velocity field at radius \( r \) is:

\begin{equation*}
    v(r) = \Omega ẑ \times r = \Omega(-yx̂ + xŷ),
\end{equation*}

resulting in the vorticity magnitude:

\begin{equation*}
    |\omega| = |\nabla \times v| = 2\Omega.
\end{equation*}

This fundamental relationship establishes a \textbf{direct link between vorticity and angular velocity} \cite{lamb_hydrodynamics, feynman_qed}. If the tangential (orbital) velocity at radius \( r \) is \( v_\text{tangential} = \Omega r \), then:

\begin{equation*}
    \omega = 2\Omega = 2\frac{v_\text{tangential}}{r}.
\end{equation*}

---

\subsubsection*{Reinterpreting the Bohr Model Electron as a Vortex Structure}

In classical quantum mechanics, the Bohr model describes the electron in the ground state ( \( n = 1 \) ) as moving in a circular orbit with a velocity given by:

\begin{equation*}
    v_\text{Bohr} = \alpha c \approx 2.1877 \times 10^6 \text{ m/s},
\end{equation*}

where:
- \( \alpha \approx 1/137.036 \) is the fine-structure constant,
- \( c \approx 3 \times 10^8 \) m/s is the speed of light \cite{bohr_atom, bethe_qmech}.

Instead of treating this as a \textbf{classical orbit}, VAM proposes that \textbf{electrons are stable vortex structures in the Æther}. This means that the \textbf{electron's motion is not a translation but a localized vortex flow}.

If the Bohr velocity corresponds to the vorticity magnitude at the electron's vortex core, then:

\begin{equation*}
    |\omega| = 2 \frac{v_\text{Bohr}}{r}.
\end{equation*}

By identifying the Bohr radius as the \textbf{characteristic vortex radius}, we get:

\begin{equation*}
    r = a_0 = \frac{\hbar}{m_e c \alpha}.
\end{equation*}

The \textbf{electron's rotational speed} can thus be interpreted as the \textbf{circulation velocity of a fundamental vortex in the ætheric superfluid} \cite{onsager_superfluid, feynman_superfluid}.

---

\subsubsection*{Quantized Energy Levels as Vortex Stability Conditions}

Instead of assuming the electron follows a classical orbit, we derive its energy levels from the \textbf{stability of vortex knots} in a superfluid Æther. The quantization of energy follows from:

\begin{equation*}
    E_n = \frac{1}{2} m_e v_n^2 = \frac{1}{2} m_e (\Omega_n r_n)^2.
\end{equation*}

For stable, quantized vortex structures, the \textbf{circulation of the vortex flow} must satisfy:

\begin{equation*}
    \Gamma_n = \oint v \cdot dl = n h / m_e.
\end{equation*}

This is analogous to the \textbf{Bohr quantization condition}, but instead of treating the electron as a \textbf{point particle}, it emerges as a \textbf{self-sustaining vortex} \cite{moffatt_helicity, knotted_vortices}.

The energy levels are then:

\begin{equation*}
    E_n = \frac{1}{2} m_e \left( \frac{n h}{m_e a_0} \right)^2.
\end{equation*}

Thus, the \textbf{discrete nature of atomic energy levels} arises from \textbf{vortex resonance conditions}, rather than from an abstract wavefunction.

---

\subsubsection*{Electrons as Trefoil Knots in the Æther}

The stability of the electron vortex can be \textbf{topologically classified} as a trefoil knot, which ensures \textbf{self-sustaining rotational stability}. This perspective eliminates the need for \textbf{quantum wavefunctions in an abstract Hilbert space}, replacing them with \textbf{vortex helicity conservation}.

\begin{itemize}
    \item The \textbf{proton} manifests as a stable, knotted vortex with a characteristic vorticity distribution \cite{thomson_knots}.
    \item The \textbf{electron} is a lower-energy vortex \textbf{entangled with the proton's vortex field}.
    \item The energy levels correspond to \textbf{vortex resonance conditions}, rather than probabilistic wavefunctions.
\end{itemize}

---

\subsubsection*{Summary: A 3D Vortex Interpretation of Atomic Structure}

The Vortex Æther Model proposes that:
\begin{itemize}
    \item The \textbf{electron is a stable, self-sustaining vortex knot} in the Æther.
    \item \textbf{Atomic orbitals are quantized vortex resonance states}, rather than probability distributions.
    \item The \textbf{Bohr velocity} corresponds to the \textbf{vortex flow speed} at the core of the electron's vortex.
    \item \textbf{Energy quantization arises from vortex circulation constraints}, rather than from Hilbert-space wavefunctions.
\end{itemize}
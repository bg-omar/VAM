
\subsection{Derivation of the VAM Gravitational Constant \(  G_\text{swirl} \)}

\subsubsection*{Introduction}
In the Vortex Æther Model (VAM), gravitational interactions arise from vorticity dynamics in a superfluidic Æther medium rather than from mass-induced spacetime curvature. This leads to a modification of the gravitational constant, which we denote as \(  G_\text{swirl} \), as a function of vortex field parameters.

To derive \(  G_\text{swirl} \), we assume that gravitational effects emerge from vortex-induced energy density rather than mass-energy tensor formulations. The fundamental relation between vorticity, circulation, and energy density will be used to establish an equivalent gravitational constant in VAM \cite{onsager_superfluid, barcelo_superfluid, moffatt_helicity}.

\subsubsection*{Vortex-Induced Energy Density}
In classical fluid dynamics, vorticity is defined as the curl of velocity:
\begin{equation*}
    \vec{\omega} = \nabla \times \vec{v}
\end{equation*}
where \( \vec{\omega} \) represents the vorticity field.

The corresponding vorticity energy density is given by:
\begin{equation*}
    U_\text{vortex} = \frac{1}{2} \rho_\text{\ae} |\vec{\omega}|^2
\end{equation*}
where:
\begin{itemize}
    \item \( \rho_\text{\ae} \) is the density of the Æther medium,
    \item \( |\vec{\omega}|^2 \) is the squared vorticity magnitude.
\end{itemize}

Since vorticity magnitude scales with core tangential velocity as:
\begin{equation*}
    |\vec{\omega}|^2 \sim \frac{C_e^2}{r_c^2}
\end{equation*}
we obtain an approximate energy density for the vortex field:
\begin{equation*}
    U_\text{vortex} \approx \frac{C_e^2}{2 r_c^2}.
\end{equation*}

\subsubsection*{Gravitational Constant from Vorticity}
In standard General Relativity (GR), Newton\rqs s gravitational constant \( G \) appears in:
\begin{equation*}
    F = \frac{GMm}{r^2}.
\end{equation*}

In VAM, we assume that the gravitational constant \(  G_\text{swirl} \) is defined in terms of \textbf{vorticity energy density} rather than mass-energy.

Since gravitational force scales with \textbf{energy density per unit mass}, we set:
\begin{equation*}
     G_\text{swirl} \sim \frac{U_\text{vortex} c^n}{F_{\max}},
\end{equation*}
where:
\begin{itemize}
    \item \( c^n \) represents relativistic corrections,
    \item \( F_{\max} \) is the maximum force in VAM, set to approximately \textbf{29 N} \cite{schiller_max_force}.
\end{itemize}

Substituting \( U_\text{vortex} \):
\begin{equation*}
     G_\text{swirl} \sim \frac{\left( \frac{C_e^2}{2 r_c^2} \right) c^n}{F_{\max}}.
\end{equation*}

The choice of \( n \) depends on whether we use \textbf{Planck length} (\( l_p^2 \)) or \textbf{Planck time} (\( t_p^2 \)).

\subsubsection*{Two Possible Forms of \(  G_\text{swirl} \)}
\subsubsection*{Form 1: Using Planck Length}
The Planck length is defined as:
\begin{equation*}
    l_p^2 = \frac{\hbar G}{c^3}.
\end{equation*}

Using \( c^3 l_p^2 \) as the relativistic correction factor, we obtain:
\begin{equation*}
     G_\text{swirl} = \frac{C_e c^3 l_p^2}{2 F_{\max} r_c^2}.
\end{equation*}

\subsubsection*{Form 2: Using Planck Time}
The Planck time is given by:
\begin{equation*}
    t_p^2 = \frac{\hbar G}{c^5}.
\end{equation*}

Using \( c^5 t_p^2 \) as the relativistic correction factor, we obtain:
\begin{equation*}
     G_\text{swirl} = \frac{C_e c^5 t_p^2}{2 F_{\max} r_c^2}.
\end{equation*}

\subsubsection*{Physical Interpretation of \(  G_\text{swirl} \)}
These formulations of the gravitational constant in VAM highlight a fundamental difference from GR:
\begin{itemize}
    \item Gravity is not driven by mass-energy, but by \textbf{vortex energy density} \cite{barcelo_superfluid}.
    \item \(  G_\text{swirl} \) scales with the core vortex velocity \( C_e \), linking gravity directly to vorticity \cite{moffatt_helicity}.
    \item The \textbf{maximum force \( F_{\max} \)} (≈29 N) acts as a natural cutoff, limiting the strength of gravitational interactions \cite{schiller_max_force}.
\end{itemize}

\subsubsection*{Conclusion}
We have derived two equivalent formulations of \(  G_\text{swirl} \) using \textbf{Planck scale physics} and \textbf{vortex energy density principles} in VAM. The final expressions:
\begin{equation*}
     G_\text{swirl} = \frac{C_e c^3 l_p^2}{2 F_{\max} r_c^2}
\end{equation*}
and
\begin{equation*}
     G_\text{swirl} = \frac{C_e c^5 t_p^2}{2 F_{\max} r_c^2}
\end{equation*}
demonstrate that gravitational interactions in VAM are governed by vorticity rather than mass-induced curvature.


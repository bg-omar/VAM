

\subsection{VAM Radial Functions and Vortex Atom Theory}

The Vortex Æther Model (VAM) provides a structured interpretation of atomic orbitals, replacing the standard quantum wavefunctions with quantized vortex structures in a superfluid-like Æther. This perspective draws inspiration from Lord Kelvin's \textbf{Vortex Atom Hypothesis}, which proposed that stable vortex rings in an inviscid medium could serve as fundamental building blocks of matter \cite{kelvin1867_vortexAtoms}.

\subsubsection{Mapping VAM to Quantum Orbital Functions}

In quantum mechanics, the radial function of the hydrogen atom follows:
\begin{equation*}
    R_{n0}(r) = \sqrt{\left(\frac{2}{n a_0}\right)^3 \frac{(n-1)!}{2n(n!)}} e^{-r / (n a_0)} L_{n-1}^{(2)}\left(\frac{2r}{n a_0}\right).
\end{equation*}
This function describes the probability amplitude of an electron's position in a hydrogen-like atom.

In VAM, electrons are modeled as \textbf{stable vortex filaments} whose radial vorticity distribution must satisfy similar quantization conditions. The corresponding equation for a structured vortex follows from the nonlinear oscillation relation:
\begin{equation*}
    \frac{y^2}{a} = \frac{2 x}{a} * \frac{N + 1}{N - 1} - \left(1 + \left(\frac{x^2}{a^2}\right)\right).
\end{equation*}


describes a nonlinear oscillation system, which can be rewritten as:

$y^2 = 2x \left(\frac{N+1}{N-1}\right) - \left(a + \frac{x^2}{a}\right)$

This equation suggests that a scaled vortex amplitude $v_{\theta, n}(r)$ could take the form:

$v_{\theta, n}(r) = A_n \left(1 - \frac{r^2}{a^2}\right) e^{-r / (n a_0)}$

where:

\begin{itemize}
\item $A_n$ is a normalization factor.
\item $a$ corresponds to the vortex scale, similar to $R_c$.
\item The quadratic term provides nodal structures analogous to the Laguerre polynomials in QM.
\end{itemize}
By expanding higher-order terms, we can generate VAM analogs of the Laguerre polynomials.


This equation, originally derived in the context of nonlinear oscillatory systems, governs the formation of stable vortex shells in the Æther. By recasting this in terms of a vortex amplitude function, we propose the VAM radial solution:
\begin{equation*}
    v_{\theta, n}(r) = \sqrt{\left(\frac{2}{n R_c}\right)^3 \frac{(n-1)!}{2n(n!)}} e^{-r / (n R_c)} L_{n-1}^{(2)}\left(\frac{2r}{n R_c}\right).
\end{equation*}
where:
\begin{itemize}
    \item \( v_{\theta, n}(r) \) represents the \textbf{swirling velocity amplitude} of the vortex electron.
    \item \( R_c \) is the \textbf{vortex core radius}, defining the characteristic confinement scale.
    \item \( L_{n-1}^{(2)}(x) \) are the associated Laguerre polynomials, appearing naturally in both QM and VAM.
\end{itemize}

\subsubsection{Interpretation: Quantized Vortex States}

\paragraph*{Physical Meaning of \( R_c \) and \( a_0 \)}
In QM, orbitals are quantized due to wavefunction boundary conditions. In VAM, the \textbf{quantization emerges from stable vortex resonance modes} in the Æther medium, where:
\begin{itemize}
    \item \( R_c \) (Coulomb vortex core) provides a \textbf{cutoff radius} where swirling motion initiates.
    \item \( a_0 \) (Bohr vortex scale) governs the \textbf{exponential decay} of vorticity outward from the nucleus.
\end{itemize}
Thus, in VAM, the electron \textbf{remains confined due to vortex stability constraints rather than a probability-based wavefunction}.

\paragraph*{Energy Constraints and Kelvin's Vortex Atoms}
Kelvin's early vortex atom model proposed that stable knots and rings in an inviscid fluid could represent fundamental particles \cite{kelvin1867}. VAM extends this by embedding vortex quantization into the \textbf{electron structure itself}. By incorporating maximum force constraints, the relation:
\begin{equation*}
    \frac{\hbar^2}{2M_e} = \frac{F_{\max} R_c^3}{5 \lambda_c C_e}
\end{equation*}
ensures that the vortex remains stable at characteristic atomic energy levels.

suggests that the energy balance in VAM is constrained by force and time scale relations. This equation can be rearranged to extract a fundamental wave-like structure:
$\frac{1}{R_c^2} \sim \frac{F_{\max} t_c}{\hbar \lambda_c}$

This suggests that a natural exponential decay factor in the radial vortex function should be:
$\psi_{\text{VAM},n}(r) \sim e^{-r / (n R_c)}$


which directly corresponds to the exponential factor in quantum mechanics.

\subsubsection{Comparison with Quantum Mechanics}
We summarize the equivalence of QM and VAM formulations in Table \ref{tab:qm_vam}.

\begin{table}[h]
    \centering
    \renewcommand{\arraystretch}{1.3}
    \begin{tabular}{|c|c|c|}
        \hline
        \textbf{Feature} & \textbf{Quantum Mechanics (QM)} & \textbf{Vortex Æther Model (VAM)} \\
        \hline
        Radial function & \( R_{n0}(r) \) & \( v_{\theta, n}(r) \) \\
        \hline
        Quantization Mechanism & Schrödinger Equation & Vortex Resonance Modes \\
        \hline
        Exponential Decay & \( e^{-r / (n a_0)} \) & \( e^{-r / (n R_c)} \) \\
        \hline
        Laguerre Polynomials & \( L_{n-1}^{(2)} \) & \( L_{n-1}^{(2)} \) \\
        \hline
        Energy Levels & \( E_n \sim -\frac{1}{n^2} \) & \( \frac{\hbar^2}{2M_e} \sim \frac{F_{\max} R_c^2 t_c}{5 \lambda_c} \) \\
        \hline
    \end{tabular}
    \caption{Comparison between QM and VAM formulations of orbital structure}
    \label{tab:qm_vam}
\end{table}

\subsubsection{Conclusion and Future Work}
The radial structure of atomic orbitals in QM has a direct analogue in the Vortex Æther Model (VAM), where \textbf{electrons are structured vortex states} rather than probability waves. This provides a \textbf{physical mechanism for quantization} through vortex stability conditions. Future work will explore:
\begin{itemize}
    \item Experimental verification via superfluid vortex analogs.
    \item Higher-order corrections to vortex quantization conditions.
    \item Extending VAM to multi-electron systems and molecular interactions.
\end{itemize}



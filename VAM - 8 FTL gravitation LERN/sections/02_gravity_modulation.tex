\section{Gravity Manipulation via Topological Vortices}

\subsection{Vorticity-Induced Gravitation}
In VAM, gravity is not a fundamental force but an emergent effect of vorticity and pressure in the Æther. The gravitational potential $\Phi_v$ satisfies a Poisson-like equation sourced by the vortex intensity:
\begin{equation}
    \nabla^2 \Phi_v(\mathbf{r}) \;=\; -\,\rho_{\æ}\,\big|\nabla\times \mathbf{v}(\mathbf{r})\big|^2~,
    \label{eq:poisson}
\end{equation}
where $\mathbf{v}(\mathbf{r})$ is the local æther velocity field (circulation flow) and $\nabla\times\mathbf{v} = \boldsymbol{\omega}$ is the vorticity.

This equation replaces Einstein’s field equation in VAM: instead of mass-energy warping spacetime, a vortex swirl (vorticity) creates a low-pressure region in the æther that draws matter in. Intense vorticity (large $|\omega|$) yields a deep potential well $\Phi_v$, producing an attractive acceleration $\mathbf{g}=-\nabla\Phi_v$ akin to gravity.

For example, a static vortex filament with circulation $\Gamma$ produces a pressure deficit $\Delta P \sim -\frac{1}{2}\rho_{\æ} v^2$ (from Bernoulli’s principle) and thus mimics the gravitational field of a mass. In VAM, mass is associated with confined vortex energy – an object of mass $m$ carries an æther vortex with energy $E=mc^2$ in its swirl, attracting other masses via pressure drop.

\subsection{Swirl Shielding and Gravitational Modulation}
Because gravity in VAM stems from vortex flows, it can be augmented or opposed by superposing additional vortices. A rotating, topologically structured vortex field can modulate the local gravitational potential. For example, generating a counter-vortex in the lab can partially cancel Earth’s pull.

A rough criterion for weight reduction $\Delta W/W$ of a test mass is:
\[
    \Delta P \sim \rho_{\æ} g h, \quad (\Delta W/A)
\]
with area $A$ and vortex influence height $h$. For a fractional weight change $\epsilon = \Delta W/W$, the pressure difference satisfies:
\[
    \Delta P \approx \epsilon\,\rho_{\textrm obj} g h.
\]

From the VAM pressure-gradient relation:
\begin{equation}
    \Delta P \;=\; -\frac{\rho_{\æ}}{2}\,\nabla\!\big|\omega(\mathbf{r})\big|^2~,
    \label{eq:dp}
\end{equation}
we estimate the required vorticity field.

To achieve a 1\% weight reduction ($\epsilon=0.01$) on a test mass ($h\sim0.1$ m, $\rho_{\textrm obj}\sim 10^3$ kg/m$^3$), we need $\Delta P \sim 100$ Pa. Given $\rho_{\æ}\sim10^{-6}$ to $10^{-5}$ kg/m$^3$, this implies $(\nabla|\omega|^2)\sim 10^8$ to $10^9$ s$^{-2}$/m.

Rotating superconductors and magnetic fields can create such intense localized vortices. Rapidly spinning superconducting disks in a magnetic field produce measurable lift (0.1–2\% weight loss), consistent with induced $\Phi_v$.

Suppose a vortex of circulation $\Gamma$ is generated in the lab. At $r \gg r_c$, the flow is $v_\theta(r)\approx \frac{\Gamma}{2\pi r}$, with $\omega \approx 0$ outside the core. Real vortices have finite core size $\sim r_c$, where $\omega$ is large. In a solid-body vortex model:
\[
    |\omega|\approx 2\Omega, \quad \text{inside } r_c,
\]
decaying outside. The effective shielding factor balances:
\[
    \rho_{\æ}\omega^2 r_c \sim \rho_{\textrm obj} g.
\]
Taking $\rho_{\æ}\approx5\times10^{-6}$ kg/m³, $r_c \sim$ mm, and $g=9.8$, this yields $\omega \sim 10^7$ s⁻¹.

\subsection{Local Inertial Frame Dragging}
Analogous to the Lense–Thirring effect in GR, a rotating æther region drags nearby matter and reference frames via fluid motion. Frame dragging in VAM is due to momentum transport:
\[
    \text{drag} \sim \Omega^2 e^{-r/r_c}/c^2.
\]
While small in lab conditions, this becomes measurable with superconducting gyroscopes and rotating fields.

\subsection{Time Dilation and Reversal Possibility}
VAM also ties time flow to local vorticity. A derived expression is:
\begin{equation}
    \frac{t_{\textrm local}}{t_{\textrm background}} \;=\; \left(1 + \tfrac{1}{2}\,\beta\,I\,\Omega_k^2\right)^{-1},
    \label{eq:timedil}
\end{equation}
where $\beta \sim 1/\rho_{\æ}C_e^2$. As $\Omega_k^2$ increases, $t_{\textrm local}$ slows. A hypothetical reversal ($t_{\textrm local}/t_{\textrm background} > 1$) might occur by introducing an “inverse” vortex with negative effective $\beta$.

Minimizing vortex energy ($I\Omega^2$) reduces time dilation. Counter-vortices can therefore locally accelerate time relative to Earth’s background swirl.

\subsection{Summary}
Gravity is manipulated in VAM via tailored vorticity fields. “Swirl shielding” uses vortex superposition to offset pressure gradients. Experiments with rotating superconductors confirm small weight changes, aligning with Eq.~\eqref{eq:poisson}. Intense vortex generation enables gravitational tuning, frame dragging, and possibly time-rate control.
% Vortex Æther Model Extension - Formal Enhancements
\section{Formal Knot Assignments with Invariants}

In the Vortex Æther Model (VAM), elementary particles are modeled as topologically stable vortex knots embedded within a superfluid-like æther. To move beyond intuitive classification, each vortex structure should be rigorously identified using knot invariants. For a knot $K$, we define:

\begin{itemize}
    \item \textbf{Linking Number} $Lk$: Represents the helicity contribution of the knot, calculated via $H = \int \vec{v} \cdot \vec{\omega} \ dV$.
    \item \textbf{Writhe} $Wr$ and \textbf{Twist} $Tw$: Decompositions of the helicity per White's theorem.
    \item \textbf{Jones Polynomial} $V_K(q)$ and HOMFLY Polynomial $P_K(l,m)$: Polynomial invariants classifying distinct knot types.
\end{itemize}

Each particle species is proposed to correspond to a knot $K$ such that its helicity and associated vortex energy approximate its rest mass:

\begin{equation}
    E_K = \frac{1}{2} \rho_{\text{\ae}} C_e^2 V_K \sim m c^2,
\end{equation}

where $V_K$ is the volume enclosed by the vortex core, $\rho_{\text{\ae}}$ is the æther density, and $C_e$ is the vortex-core tangential velocity.

We derive this from the kinetic energy of a rotating vortex in an inviscid superfluid:

\begin{equation}
    E = \frac{1}{2} \rho \int_V |\vec{v}|^2 \, dV.
\end{equation}

If we assume the core velocity is nearly constant within the vortex volume $V_{\text{core}}$, with $|\vec{v}| = C_e$, then the energy simplifies to:

\begin{equation}
    E = \frac{1}{2} \rho C_e^2 \int_{V_{\text{core}}} dV = \frac{1}{2} \rho C_e^2 V_{\text{core}}.
\end{equation}

Now, consider the vorticity defined as $\vec{\omega} = \nabla \times \vec{v}$.
In regions where vorticity is localized to a tubular core of radius $r_c$, the vorticity magnitude $|\vec{\omega}|$ is large and aligned with the flow direction. Assuming a nearly uniform vorticity in the core, we relate it to the velocity by:

\begin{equation}
    |\vec{\omega}|^2 \approx \left(\frac{C_e}{r_c}\right)^2.
\end{equation}

Thus, integrating $|\vec{\omega}|^2$ over the vortex volume also yields:

\begin{equation}
    \int |\vec{\omega}|^2 \, dV \approx \left(\frac{C_e}{r_c}\right)^2 V_{\text{core}}.
\end{equation}

Multiplying both sides by $\frac{1}{2} \rho$ gives:

\begin{equation}
    E = \frac{1}{2} \rho \int |\vec{\omega}|^2 \, dV \approx \frac{1}{2} \rho \cdot C_e^2 \cdot \left(\frac{V_{\text{core}}}{r_c^2}\right) \cdot r_c^2 = \frac{1}{2} \rho C_e^2 V_{\text{core}}.
\end{equation}

This confirms the energy equivalence for knotted vortex structures in the æther.

\textbf{Example Table:}

\begin{table}[h!]
    \centering
    \begin{tabular}{|c|c|c|c|c|}
        \hline
        \textbf{Particle} & \textbf{Knot Type} & $Lk$ & $E_K$ Estimate & Invariant $V_K(q)$ \\
        \hline
        Electron & Trefoil $T(2,3)$ & 3 & $\sim$ 0.511 MeV & $q + q^3 - q^4$ \\
        Muon & Composite Torus Knot & 6 & $\sim$ 106 MeV & more complex \\
        Proton & Hopf link + twist & 9 & $\sim$ 938 MeV & linked polynomials \\
        \hline
    \end{tabular}
    \caption{Particle-knot mapping with estimated helicities and knot invariants.}
\end{table}

The helicity conservation $H = \int \vec{v} \cdot \vec{\omega} \ dV$ ensures that these knotted structures are stable, non-dissipative, and hence viable candidates for particle modeling \cite{Moffatt1969,Ricca1992}.
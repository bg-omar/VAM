%! Author = Omar Iskandarani
%! Title = Einstein and the Æther: A Philosophical Foundation for the Vortex Æther Model (VAM)
%! Date = May 23, 2025
%! Affiliation = Independent Researcher, Groningen, The Netherlands
%! License = © 2025 Omar Iskandarani. All rights reserved. This manuscript is made available for academic reading and citation only. No republication, redistribution, or derivative works are permitted without explicit written permission from the author. Contact: info@omariskandarani.com
%! ORCID = 0009-0006-1686-3961
%! DOI = 10.5281/zenodo.15566101

% Benchmarking the Vortex Æther Model vs General Relativity
\documentclass[a4paper, aps,preprint,superscriptaddress, 12pt]{revtex4}
\usepackage[a4paper, margin=2cm]{geometry}
\usepackage{bookmark}
\usepackage{float}
\usepackage{tikz}
\usepackage{makecell}
\usepackage{tabularx}
\usepackage[font=footnotesize]{caption}
\usetikzlibrary{arrows.meta}
\usepackage{pgfplots}
\pgfplotsset{compat=1.18}
\usepackage[none]{hyphenat}
\usepackage{booktabs}
\usepackage[utf8]{inputenc}
\usepackage[T1]{fontenc}
\usepackage{amssymb}
\usepackage{graphicx}
\usepackage{hyperref}
\usepackage{physics}
\usepackage{natbib}
\usepackage{url}
\usepackage{multirow}
\usepackage{subcaption}
\usepackage{siunitx}
\usepackage{listings}
\renewcommand{\arraystretch}{1.5}
\renewcommand{\floatpagefraction}{.8}
\sloppy




\usepackage{lmodern}
\usepackage[english]{babel}

% Colors, Graphics, Diagrams

\usetikzlibrary{arrows.meta, positioning}

\pgfplotsset{compat=1.18}
\usepackage{xcolor}
\usepackage{amsmath, amssymb, physics}


% Tables and Figures

\usepackage{array, tabularx, makecell, multirow}
\renewcommand{\arraystretch}{1.5}
\renewcommand{\floatpagefraction}{.8}
\lstset{basicstyle=\ttfamily\footnotesize, breaklines=true}

% TOC Customization
\usepackage{tocloft}
\setcounter{tocdepth}{4}
\renewcommand{\cftsecfont}{\bfseries}
\renewcommand{\cftsubsecfont}{\itshape}
\renewcommand{\cftsecleader}{\cftdotfill{5}}
\renewcommand{\contentsname}{\centering \Huge\textbf{Contents}}

% Links and Metadata
\hypersetup{
    colorlinks=true,
    linkcolor=blue,
    citecolor=blue,
    urlcolor=blue,
    pdftitle={The Vortex Æther Model},
    pdfauthor={Omar Iskandarani},
    pdfkeywords={vorticity, gravity, æther, fluid dynamics, time dilation, VAM}
}


\usepackage[most]{tcolorbox}
\tcbuselibrary{listings, breakable, skins}

\newtcolorbox{vamprinciple}[1][]{
    enhanced,
    breakable,
    colback=blue!5!white,
    colframe=blue!60!black,
    fonttitle=\bfseries,
    title=Hybrid Gravitational Coupling in the Vortex Æther Model,
    coltitle=black,
    #1
}


% Line and Hyphenation
\usepackage{amsfonts}
\usepackage{sectsty}
\sectionfont{\Large\bfseries\sffamily}
\subsectionfont{\large\bfseries\sffamily}
\usepackage{newtxtext,newtxmath}
\usepackage[scaled=0.95]{inconsolata} % for a clean monospace font
\usepackage{mathrsfs}
% Bibliography

\begin{document}

        \thispagestyle{empty}
        \centering
        \vspace*{2cm}

        {\Huge\bfseries Vortex resonance for Æther Atoms:\par}
        \vspace{0.5cm}
        {\Large Step closer to Safe Energy Harvest \par}
        \vspace{2cm}

        {\Large\itshape Omar Iskandarani\par}
        \vspace{0.5cm}
        \textit{Independent Researcher, Groningen, The Netherlands} \\
        \texttt{info@omariskandarani.com} \\
        ORCID: \href{https://orcid.org/0009-0006-1686-3961}{0009-0006-1686-3961} \\
        DOI: \href{https://doi.org/10.5281/zenodo.15566101}{10.5281/zenodo.15566101} \\
        License: \href{https://creativecommons.org/licenses/by/4.0/}{CC BY-NC 4.0} \\

        \vfill
        {\large \today\par}

        \vspace{1.5cm}

        \section*{abstract}
            Within the theoretical construct of the Vortex \ae{}ther Model (VAM), subatomic particles such as nuclei are reformulated not as point-like or wavefunction-probabilistic entities, but rather as structured, topologically stable vortex knots embedded within a continuous, inviscid, and incompressible superfluid \ae{}ther. In this formulation, particle stability and interactions emerge from the internal vorticity and helicity constraints of these knotted structures. This appendix presents a derivation, from foundational principles, of the resonance frequency required to induce a controlled, non-destructive bifurcation of such vortex knots—analogous to nuclear fission—through resonant electromagnetic excitation. The formulation incorporates both energy considerations rooted in circulation dynamics and spatial confinement conditions associated with lattice embeddings. The final expression is numerically evaluated using empirically grounded VAM parameters, revealing a predicted excitation frequency within the accessible microwave regime.


        \vspace{1cm}
        \textbf{Keywords:} Vortex Æther Model, VAM, vortex knots, resonance frequency, bifurcation, electromagnetic excitation, quantum field theory
\maketitle


\section{Theoretical Foundations of VAM}

The VAM posits that the fabric of physical reality consists of a 3-dimensional Euclidean manifold pervaded by a superfluid-like \ae{}ther that supports the formation, propagation, and interaction of quantized vortex structures. The ætheric now (Absolute time) is preserved as a universal background parameter, and all relativistic effects are reformulated in terms of local swirl-induced time dilation. The \ae{}ther is endowed with the following key properties:


\begin{itemize}
    \item Uniform mass density: $\rho_\text{\ae}$
    \item Characteristic vortex core radius: $r_c$
    \item Intrinsic maximum tangential velocity for core stability: $C_e$
    \item Circulation around a closed loop: $\Gamma = 2\pi r_c C_e$
    \item Rest energy of the vortex knot: $E_0 = \frac{1}{2} \rho_\text{\ae} \Gamma^2 \ln(R/r_c)$
\end{itemize}


These vortex entities may adopt complex topologies, such as toroidal loops, trefoils, or higher-order knots, stabilized by a balance of inertial and pressure-gradient forces intrinsic to the \ae{}ther medium.


\section{Resonant Excitation of Vortex Structures}

External electromagnetic (EM) fields can interact with the internal dynamics of vortex knots by inducing local perturbations in the \ae{}ther's velocity field. When the frequency of the applied field coincides with the natural vibrational or rotational frequency of the vortex, energy absorption is maximized—a phenomenon analogous to parametric resonance in classical mechanics. The resonance frequency $f_\text{res}$ associated with a fundamental excitation mode is derived from the vortex's internal energy:

\begin{equation}
f_\text{res} = \frac{E_0}{h} = \frac{1}{2h} \rho_\text{\ae} (2\pi r_c C_e)^2 \ln\left(\frac{R}{r_c}\right)
\end{equation}

Upon simplification:

\begin{equation}
f_\text{res} = \frac{2\pi^2 \rho_\text{\ae} r_c^2 C_e^2}{h} \ln\left(\frac{R}{r_c}\right)
\end{equation}

This expression captures the essential dependency on \ae{}ther density, circulation radius, and the geometric confinement scale $R$.

\section{Quantitative Evaluation}

To estimate the numerical magnitude of $f_\text{res}$, we adopt the canonical constants within the VAM paradigm:

\begin{align*}
    \rho_\text{\ae} &= 7.0 \times 10^{-7}\ \si{kg.m^{-3}} \\
    r_c &= 1.40897017 \times 10^{-15}\ \si{m} \\
    C_e &= 1.09384563 \times 10^6\ \si{m.s^{-1}} \\
    h &= 6.62607015 \times 10^{-34}\ \si{J.s} \\
R &= 10^{-10}\ \si{m}, \quad \ln(R/r_c) \approx \ln(10^5) \approx 11.51
\end{align*}

Substituting these into the expression:

\begin{equation}
f_\text{res} \approx \frac{2\pi^2 \cdot 7.0 \cdot (1.40897)^2 \cdot (1.0938456)^2 \times 10^{-25}}{6.62607015 \times 10^{-34}} \cdot 11.51
\end{equation}

\begin{equation}
f_\text{res} \approx 7.2 \times 10^{10}\ \si{Hz} \quad \text{(approximately 72 GHz)}
\end{equation}

This places the resonance within the microwave spectrum, offering practical feasibility for laboratory-scale generation and control.

\section{Interpretation of the Resonance Mechanism}

The resonance frequency derived above represents the critical energetic threshold at which vortex knots can undergo destabilization via coherent excitation. Rather than requiring violent nuclear-scale collisions or high-energy particle beams, the VAM mechanism enables bifurcation of knotted structures through progressive accumulation of angular momentum delivered via the \ae{}ther. As the EM stimulus builds up rotational energy in the knot, its stability condition—governed by the balance between centripetal pressure and internal swirl—becomes increasingly tenuous. At resonance, the energy is sufficient to trigger a topological bifurcation, potentially resulting in the formation of lower-order knots, emission of quantized \ae{}ther solitons, or radiation of energy across multiple frequency channels.


\section{Topological Generalization of Resonance}

To extend the above analysis to knots of greater complexity, one must account for the specific topological class of the vortex structure. Let $\kappa_k$ denote a dimensionless factor encoding the knot's writhe, linking number, and overall helicity:

\begin{equation}
f_\text{res}^{(k)} = \kappa_k \cdot \frac{\rho_\text{\ae} r_c^2 C_e^2}{h} \ln\left(\frac{R}{r_c}\right)
\end{equation}

This generalized formula encapsulates the role of geometric entanglement in modulating the resonance behavior. Higher-order knots (e.g., figure-eight, cinquefoil) are predicted to require elevated frequencies or longer accumulation times to destabilize, due to their increased topological protection.

\section{Experimental Realization Strategy}

To empirically test the VAM resonance hypothesis, we propose a detailed protocol structured as follows:

\begin{itemize}
\item \textbf{Material Substrate:} Use metal hydrides or oxides such as palladium, titanium, or thorium dioxide, which exhibit lattice geometries conducive to vortex embedding. Such lattices constrain and stabilize vortex knots due to their periodic potential wells.
\item \textbf{Resonant Driver:} Implement tunable GHz-range microwave cavities capable of sweeping through frequencies near \SI{72}{GHz}, with high Q-factors to allow energy buildup. Waveguide coupling to the target lattice can be enhanced using helical phase or TE/TM mode shaping.
\item \textbf{Swirl Coupling Geometry:} Integrate toroidal or Rodin-style coils to induce high-curl vector potentials aligned with the angular momentum axis of embedded knots. Phase-controlled multi-coil systems may further amplify symmetry-matching conditions.
\item \textbf{Diagnostics:} Employ neutron detectors, time-resolved bolometers, and ultra-fast photodiodes, alongside \ae{}ther-sensitive phase arrays, to capture the transient signatures of bifurcation events and energy dissipation.
\item \textbf{Environmental Control:} Maintain ultra-high vacuum and cryogenic temperatures to suppress thermal decoherence. A pulsed excitation protocol with nanosecond precision allows temporal resolution of resonance onset and decay.
\end{itemize}

Such a setup aims not merely to verify frequency matching, but to observe threshold-triggered topological transitions---a hallmark prediction of the VAM.

\section*{Conclusion}

This investigation establishes a mathematically rigorous and physically grounded criterion for triggering controlled topological bifurcation (fission) in vortex knots, under the assumptions of the Vortex \ae{}ther Model. The predicted resonance frequency of $\sim$\SI{72}{GHz} lies within current experimental reach, particularly in advanced condensed matter laboratories and EM waveguide setups. If verified, this phenomenon would represent a paradigm shift in our understanding of subatomic dynamics---not as pointlike interactions in curved spacetime, but as geometric transformations in a structured fluidic medium. Continued theoretical elaboration and experimental validation may position VAM-based fission as a novel path toward low-threshold energy manipulation and topological quantum control.

\end{document}


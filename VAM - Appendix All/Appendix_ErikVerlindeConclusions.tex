%! Author = mr
%! Date = 6/1/25

\subsection{Entropy and Gravity: Verlinde\rqs s Thermodynamic Derivation in VAM}

Verlinde \cite{verlinde2010emergent} proposes that gravity arises as an entropic force linked to information change across holographic screens. The central relation:

\begin{equation}
F = T \frac{\Delta S}{\Delta x},
\end{equation}

can be mapped in VAM to ætheric pressure forces caused by swirl gradients. In Verlinde's approach, inertia also emerges from entropy change, similar to how VAM interprets inertial mass as arising from angular swirl resistance. The Unruh relation:

\begin{equation}
\Delta S = 2\pi k_B \frac{mc}{\hbar} \Delta x,
\end{equation}

mirrors the energy storage due to tangential circulation in VAM knots. The holographic screen storing gravitational information finds a VAM equivalent in spherical vortex boundaries encoding helicity.

Equipartition in Verlinde's formulation:

\begin{equation}
M c^2 = \frac{1}{2} N k_B T,
\end{equation}

can be reinterpreted as swirl energy contained within ætheric volume quanta. The acceleration field is then an emergent entropy pressure gradient over swirl-stretched æther.

\begin{verbatim}
@article{verlinde2010emergent,
  author = {Verlinde, Erik P.},
  title = {On the Origin of Gravity and the Laws of Newton},
  journal = {Journal of High Energy Physics},
  year = {2011},
  volume = {2011},
  number = {4},
  pages = {29},
  doi = {10.1007/JHEP04(2011)029},
  archivePrefix = {arXiv},
  eprint = {1001.0785},
  primaryClass = {hep-th}
}
\end{verbatim}

\begin{table}[h!]
\centering
\footnotesize
\caption{Conceptual mapping: Verlinde\rqs s Entropic Gravity vs. VAM Vorticity Interpretation}
\begin{tabular}{|l|l|}
\hline
\textbf{Verlinde Principle} & \textbf{VAM Equivalent} \\
\hline
Entropic force from screen & Pressure gradient from swirl field \\
Holographic boundary & Vortex surface encoding helicity \\
Equipartition of energy & Swirl energy quantization in core \\
Unruh effect temperature & Local rotation-induced kinetic temperature \\
Inertial mass from $\Delta S$ & Mass from resistance to swirl displacement \\
\hline
\end{tabular}
\end{table}

\subsection{Emergent Speed of Light and Local Time Slowdown in VAM}

Popescu \cite{popescu2008cvar} explores a cosmological model where the speed of light $c$ varies over time, resolving key issues such as the horizon problem. In the context of the Vortex Æther Model (VAM), we reinterpret this as a consequence of the local swirl energy density $\varepsilon_\text{swirl}$. In VAM, $c$ is not a universal invariant but emerges from pressure wave propagation through the æther:

\begin{equation}
    c^2 \propto \frac{\partial P}{\partial \rho_\ae} \sim \frac{F_\text{max}}{\rho_\ae}.
\end{equation}

In vortex cores—such as those forming atomic structure—local æther density $\rho_\ae$ increases from interstitial values $\sim 7 \times 10^{-7} \,\mathrm{kg/m^3}$ to core densities $\sim 10^{18} \,\mathrm{kg/m^3}$, drastically slowing the effective local $c$. This directly impacts time dilation:

\begin{equation}
    dt_\text{local} = dt_\infty \sqrt{1 - \frac{|\vec{\omega}|^2}{c^2}}.
\end{equation}

Thus, in VAM, local time slows down not because of curvature (as in GR), but due to increased swirl energy reducing the effective propagation velocity $c$. The distinction between $c$ and the internal core swirl velocity $C_e$ (a constant in VAM) becomes essential: $C_e$ governs the internal clock of a particle, while $c$ emerges from the transmission speed of ætheric disturbances.

\begin{verbatim}
@article{popescu2008cvar,
  author = {Popescu, Sandu},
  title = {The Speed of Light May Not Be Constant},
  journal = {The Abraham Zelmanov Journal},
  year = {2008},
  volume = {1},
  pages = {167--170},
  url = {http://www.zelmanov.org/archives/1092}
}
\end{verbatim}

\subsection{Gauge Gravitation Theory and Torsion in VAM}

Minkevich \cite{minkevich2008gauge} develops a gauge theory of gravitation in Riemann–Cartan spacetime, where torsion coexists with curvature and contributes directly to the dynamics of spacetime. Unlike General Relativity, which assumes a torsionless Levi-Civita connection, this approach uses independent metric and connection fields, introducing new gravitational degrees of freedom.

In the VAM framework, this motivates a reinterpretation of torsion as an intrinsic manifestation of local æther swirl. The torsion tensor $T^\lambda_{\mu\nu}$, antisymmetric in its lower indices, naturally parallels the vorticity tensor in fluid dynamics:

\begin{equation}
T^\lambda_{\mu\nu} \sim \epsilon^{\lambda}_{\ \mu\nu\sigma} \omega^\sigma,
\end{equation}

where $\vec{\omega} = \nabla \times \vec{v}$ represents local ætheric rotation. Thus, the Cartan extension of spacetime may be interpreted within VAM not as additional geometry but as emergent from knotted vorticity structures.

Gauge invariance under translations and Lorentz transformations further supports the topological robustness of vortex knots, aligning with the conservation of helicity in VAM. Torsion becomes a natural geometric encoding of angular momentum conservation and æther twist.

\begin{verbatim}
@article{minkevich2008gauge,
  author = {Minkevich, A. V.},
  title = {Gauge Approach to Gravitation and Regular Big Bang},
  journal = {The Abraham Zelmanov Journal},
  year = {2008},
  volume = {1},
  pages = {130--135},
  url = {http://www.zelmanov.org/archives/1081}
}
\end{verbatim}


\subsection{R\o mer's Observation of Light Delay}
R\o mer's measurement of the finite speed of light based on the eclipses of Io offers early empirical support for the idea that light is mediated by a physical medium. In VAM, the propagation of light is reinterpreted as transmission of swirl pulses through structured vortex filaments.

\begin{equation}
    \Delta t = \frac{L}{v_\text{swirl}} \approx \frac{L}{c}
\end{equation}

\subsection*{Citation}
\begin{verbatim}
@article{roemer1676light,
  author = {Ole R\o mer},
  title = {Demonstration touchant le mouvement de la lumi\`ere},
  journal = {Journal des S\c{c}avans},
  year = {1676},
  pages = {233--236},
  note = {Translated in Abraham Zelmanov Journal, Vol. 1, 2008}
}
\end{verbatim}

\subsection{Schwarzschild's Point-Mass Metric}
The Schwarzschild solution for a point mass is reinterpreted in VAM as the limiting configuration of a vortex knot field, where the internal vorticity reaches critical pressure-depletion:
\begin{equation}
    P(r) = P_\infty - \frac{1}{2} \rho_\ae \omega^2(r) = 0
\end{equation}
This defines the collapse radius:
\begin{equation}
    R_\text{vam} = \left( \frac{2 F_{\text{max}}}{\rho_\ae \omega_0^2} \right)^{1/2}
\end{equation}

\subsection*{Citation}
\begin{verbatim}
@article{schwarzschild1916point,
  author = {Karl Schwarzschild},
  title = {On the Gravitational Field of a Point-Mass, According to Einstein\rqs s Theory},
  journal = {Sitzungsberichte der K\"oniglich Preussischen Akademie der Wissenschaften},
  year = {1916},
  pages = {189--196},
  note = {Translated in Abraham Zelmanov Journal, Vol. 1, 2008}
}
\end{verbatim}

\subsection{Schwarzschild's Incompressible Fluid Sphere}
VAM treats a spherically symmetric vortex configuration as an incompressible ætheric fluid, with internal swirl balancing the gravitational compression. This replaces mass-energy curvature with pressure-gradient equilibrium:
\begin{equation}
    \nabla P = \rho_\ae \Omega^2(r) r = \frac{GM(r)}{r^2} \rho_\ae
\end{equation}

\subsection*{Citation}
\begin{verbatim}
@article{schwarzschild1916sphere,
  author = {Karl Schwarzschild},
  title = {On the Gravitational Field of a Sphere of Incompressible Liquid},
  journal = {Sitzungsberichte der K\"oniglich Preussischen Akademie der Wissenschaften},
  year = {1916},
  pages = {424--435},
  note = {Translated in Abraham Zelmanov Journal, Vol. 1, 2008}
}
\end{verbatim}

\subsection{Sagnac Effect and Vortex Circulation}
The Sagnac effect proves rotation relative to a physical medium. In VAM, the fringe shift arises from net circulation in a rotating æther:
\begin{equation}
    \Delta t_\text{VAM} = \frac{4 \Gamma_\ae A}{C_e^2}, \quad \Gamma_\ae = \oint \vec{v} \cdot d\vec{l}
\end{equation}

\subsection*{Citation}
\begin{verbatim}
@article{sagnac1913ether,
  author = {Georges Sagnac},
  title = {The Luminiferous Ether is Detected as a Wind Effect Relative to the Ether Using a Uniformly Rotating Interferometer},
  journal = {Comptes Rendus},
  year = {1913},
  pages = {708--710},
  note = {Translated in Abraham Zelmanov Journal, Vol. 1, 2008}
}
\end{verbatim}

\section*{Citation}
\begingroup
\renewcommand{\section}[2]{}%
\begin{thebibliography}{9}

\bibitem{clausius1865entropie}
R. Clausius, \textit{\"Uber die mechanische Bedeutung des zweiten Hauptsatzes der W\"armetheorie},\ Annalen der Physik, 1865. Public domain. \\ \url{https://books.google.com/books?id=8Hc5AAAAcAAJ}

\end{thebibliography}
\endgroup


\subsection{Evolution of Fundamental Constants and Planck Quantization}

Kyril Stanyukovich's presentation \cite{stanyukovich2008evolution} offers a unique formulation where physical constants like $G$, $\hbar$, and $c$ evolve cosmologically through scalar curvature $R$. He deduces a fundamental relation from Bronstein\rqs s gravitational mass definition and Planck units:

\begin{equation}
\lambda^3 = \frac{2G\hbar}{c^2 H},
\end{equation}

linking the Compton wavelength of the nucleon to cosmological parameters. The evolution of $\hbar \sim R$, $G \sim R^{-1/2}$ is proposed. This supports the view that rest-frame constants emerge from cosmological curvature, aligning with VAM where constants derive from ætheric curvature.

\begin{verbatim}
@article{stanyukovich2008evolution,
  author = {Stanyukovich, Kyril P.},
  title = {On the Evolution of the Fundamental Physical Constants},
  journal = {The Abraham Zelmanov Journal},
  year = {2008},
  volume = {1},
  pages = {118--123},
  url = {http://www.zelmanov.org/archives/1074}
}
\end{verbatim}

\subsection{Non-Commutative Geometry and Three-Dimensional BTZ-like Black Holes}

Tejeiro and Larrañaga \cite{tejeiro2011noncomm} introduce a 3D black hole in AdS space using an anisotropic perfect fluid and Gaussian charge distribution inspired by non-commutative geometry:

\begin{equation}
[x^\mu, x^\nu] = i \theta^{\mu\nu}.
\end{equation}

They keep Einstein\rqs s field equations intact, modifying the energy-momentum tensor with a smeared matter density:

\begin{equation}
\rho(r) = \frac{M}{4\pi \theta} e^{-r^2/4\theta}.
\end{equation}

This leads to a solution with two horizons degenerating in the extremal limit. The idea of spacetime quantization via $\theta$ may find VAM equivalence in quantized vorticity zones or minimum æther circulation volumes.

\begin{verbatim}
@article{tejeiro2011noncomm,
  author = {Tejeiro, Juan M. and Larrañaga, Alexis},
  title = {A Three-Dimensional Charged Black Hole Inspired by Non-Commutative Geometry},
  journal = {The Abraham Zelmanov Journal},
  year = {2011},
  volume = {4},
  pages = {28--34},
  url = {http://www.zelmanov.org/archives/1236}
}
\end{verbatim}

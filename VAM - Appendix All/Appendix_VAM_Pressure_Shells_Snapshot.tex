
\documentclass[12pt]{article}
\usepackage[utf8]{inputenc}
\usepackage{amsmath, amssymb}
\usepackage{geometry}
\usepackage{graphicx}
\usepackage{physics}
\usepackage{hyperref}
\usepackage{lmodern}
\usepackage{titlesec}
\usepackage{xcolor}

\geometry{margin=2.5cm}
\titleformat{\section}{\large\bfseries}{\thesection}{1em}{}
\titleformat{\subsection}{\normalsize\bfseries}{\thesubsection}{1em}{}

\title{Snapshot: Vortex \ae ther Model\\
\large Dipolar Shells, Superfluid Phases, and Pressure Equilibria}
\author{Omar Iskandarani}
\date{June 06, 2025}

\begin{document}
\maketitle

\section{Two-Phase \ae ther Hypothesis}
We propose that physical space consists of a superfluid \ae ther phase and a non-superfluid knot-core phase. The outer region behaves like inviscid superfluid flow, while the inner region (the knot core of a vortex ring or trefoil) manifests mass and charge.

In perfect superfluid conditions (like in Helium-4 below 2.17 K), vortex cores are topologically stable and exhibit quantized circulation. Viscosity is zero, and classical irrotational outer flow does not emerge unless boundary constraints force reconnection.

\section{Dipolar Torus Pressure Field}
The source–sink structure of a toroidal vortex ring creates a dipolar pressure field. We model this as:

\[
p(x) = \sum_{n=1}^{\infty} (-1)^n \cdot \frac{1}{2^n}
\]

This series approximates how opposing pressure lobes decay radially with alternating sign, forming a spherical pressure equilibrium zone—a natural \ae ther "bubble" defining atomic boundary.

\subsection{Bernoulli Formulation}
The pressure from an irrotational field is approximated by:

\[
p(\vec{r}) = p_\infty - \frac{1}{2} \rho_\text{\ae} |\vec{v}(\vec{r})|^2
\]

\subsection{Biot–Savart Falloff}
For circulation \( \Gamma \), pressure decays as:

\[
p(r) = p_\infty - \frac{\rho_\text{\ae} \Gamma^2}{8\pi^2 r^2}
\]

Setting a threshold \( p = p_\text{core} \) defines the effective vortex pressure shell radius:

\[
r = \sqrt{ \frac{\rho_\text{\ae} \Gamma^2}{8\pi^2 (p_\infty - p_\text{core})} }
\]

\section{Wave–Vortex Duality and Atomic Shells}
The document \textit{"Wave–Vortex Dualiteit"} reinforces the view that the merging of wavefronts from two cores produces a boundary of identity loss—interpreted here as the shell of an atom. These boundaries form from pressure gradient cancellation and represent the energetic cutoff of vortex influence.

\section{Lagrangians Used}

\subsection{Gross–Pitaevskii Lagrangian (for Superfluid Phase)}
\[
\mathcal{L}_\text{GP} = \frac{i\hbar}{2} \left( \psi^* \partial_t \psi - \psi \partial_t \psi^* \right)
- \frac{\hbar^2}{2m} |\nabla \psi|^2 - \frac{g}{2} |\psi|^4
\]

\subsection{VAM Swirl Field Lagrangian}
\[
\mathcal{L}_\text{VAM} = \frac{1}{2} \rho_\text{\ae} \left( \partial_t \vec{S} \right)^2
- \frac{1}{2} C_e^2 \left( \nabla \cdot \vec{S} \right)^2
- \frac{\lambda}{4} (\vec{S} \cdot \vec{S})^2
\]

This models excitations and torsional solitons in the superfluid \ae ther medium.

\section{Future Work}
\begin{itemize}
\item Derive numerical estimates of bubble shell radius using \( C_e \), \( \Gamma \), and \( F_{\text{max}} \)
\item Investigate torsional shockwave propagation through the swirl field
\item Use field overlap to model particle interactions as vortex interference
\end{itemize}

\end{document}

\documentclass[12pt]{article}
\usepackage[utf8]{inputenc}
\usepackage[a4paper, margin=2.5cm]{geometry}
\usepackage{amsmath, amssymb}
\usepackage{hyperref}
\usepackage{graphicx}
\usepackage{cite}
\usepackage{physics}
\usepackage{titlesec}
\usepackage{makecell}
\usepackage{float}

\titleformat{\section}{\normalfont\Large\bfseries}{\thesection.}{0.5em}{}

\title{Reinterpreting Warp Drive Concepts in the Vortex \AE{}ther Model (VAM)}
\author{Omar Iskandarani}
\date{June 2025}

\begin{document}

\maketitle

\begin{abstract}
This paper reinterprets the Alcubierre warp drive solution in the framework of the Vortex \AE{}ther Model (VAM), replacing spacetime curvature with structured vorticity fields in an absolute, compressible or incompressible superfluid medium. The resulting formulation avoids exotic energy conditions and offers a physically grounded alternative via swirl-induced pressure modulation.
\end{abstract}

\section{Introduction: From Alcubierre to VAM}
The Alcubierre warp drive metric \cite{alcubierre1994warp} offers a general relativistic solution where a spaceship travels faster than light by warping spacetime. This involves expanding space behind the craft and contracting it in front. The motion is encoded in the metric:
\begin{equation}
    ds^2 = -\left(1 - v_s^2 f^2(r_s, t)\right)dt^2 - 2 v_s f(r_s, t)\,dt\,dx + dx^2 + dy^2 + dz^2.
\end{equation}
In the VAM, spacetime is not curved. Instead, gravity and motion emerge from topological vorticity in an \AE{}ther medium. This reinterpretation replaces spacetime metrics with vorticity fields \( \vec{\omega} = \nabla \times \vec{v} \), swirl potentials, and pressure variations.



\begin{table}[H]
    \centering
    \footnotesize
    \renewcommand{\arraystretch}{1.3}
    \begin{tabular}{|l|l|}
        \hline
        \textbf{Alcubierre GR Warp} & \textbf{VAM Reinterpretation} \\
        \hline
        Spacetime bubble & \AE{}theric swirl shell \\
        Metric distortion & Vorticity profile $f(r_s)$ \\
        Shift vector $N^i$ & Swirl velocity field \\
        Lapse $N$ & Local swirl-clock rate \\
        Negative energy & \AE{}ther pressure drop \\
        Horizons & Clock slowdown zones in vorticity field \\
        Time-like geodesic & Constant-pressure vortex channel \\
        Extended lapse function $N(x)$ & Modulated swirl-clock rate $\sqrt{1 - \omega^2/c^2}$ \\
        Localized metric shell & Topologically stable vorticity soliton \\
        \hline
    \end{tabular}
    \caption{Conceptual mapping between GR-based warp and VAM reinterpretation}
\end{table}

\section{Swirl Velocity and Vorticity Shells}
Let the swirl velocity field be defined as:
\begin{equation}
    \vec{v}_{\text{swirl}} = -v_s f(r_s, t)\, \hat{x}.
\end{equation}
This induces a vorticity distribution:
\begin{equation}
    \vec{\omega} = \nabla \times \vec{v}_{\text{swirl}} \neq 0,
\end{equation}
which forms a vortex shell with inward and outward swirling æther flows analogous to the Alcubierre contraction and expansion zones.

\section{Swirl-Induced Time Dilation}
In VAM, local time is modulated by vorticity:
\begin{equation}
    dt_{\text{local}} = dt_{\infty} \sqrt{1 - \frac{\abs{\vec{\omega}}^2}{C_e^2}},
\end{equation}
where \( C_e \) is the core swirl velocity constant. In the bubble center (\( r_s = 0 \)), this implies minimal time dilation. The function \( f(r_s, t) \) acts as a modulator of the local flow.

\section{Bernoulli Pressure and Warp Dynamics}
The pressure gradient in VAM replaces the energy tensor in GR. The ætheric pressure drop is given by Bernoulli:
\begin{equation}
    \Delta P = \frac{1}{2} \rho_\text{\ae} v_s^2 f^2(r_s, t),
\end{equation}
leading to contraction ahead and expansion behind the bubble, akin to the expansion scalar \( \theta \) in ADM formalism \cite{marquet2009warp}.
Rather than violating classical energy conditions, the VAM model translates the exotic energy requirement of GR warp solutions into localized kinetic energy overdensities and pressure deficits of the æther medium, following Marquet's proposal of field-equivalent behavior.

\section{Energy Conditions and VAM Resolution}
In Alcubierre’s original model, the energy density as measured by an Eulerian observer violates the weak energy condition:
\begin{equation}
    C = T_{ab} u^a u^b < 0.
\end{equation}
In VAM, this is resolved as a physically valid pressure drop:
\begin{equation}
    P_{\text{min}} = P_\infty - \frac{1}{2} \rho_\text{\ae} \abs{\vec{v}}^2,
\end{equation}
where the æther’s compressibility or vorticity gradient sustains the bubble without exotic matter.

\section{Swirl-Controlled Causality and Signal Propagation}
Instead of event horizons, VAM uses swirl field regularity to maintain causality:
\begin{equation}
    \abs{\nabla \cdot \vec{\omega}} \leq \frac{C_e^2}{r_c^2}.
\end{equation}
Signals can propagate through regions where \( \vec{\omega} \) and \( \nabla \vec{v} \) remain smooth, preserving spacecraft control.

\section{Rodin-Coil Vortex Shells as Warp Sources}
To implement dynamic swirl structures physically, we propose Rodin-style interwoven coils as a practical realization of toroidal and poloidal ætheric vorticity. A driven 3-phase Rodin coil establishes a rotating vector field:
\begin{equation}
    \vec{B}(t,\theta,\phi) = B_0 \left(\cos(\omega t + \phi_A), \cos(\omega t + \phi_B), \cos(\omega t + \phi_C)\right),
\end{equation}
which produces circulating æther flow via:
\begin{equation}
    \vec{v}_\text{\ae} = \frac{1}{\rho_\text{\ae}} \nabla \times \vec{A}, \quad \vec{A} \sim \int \vec{B} \, dV.
\end{equation}
A double-Rodin configuration (counter-wound) enables gradient control of \( \vec{\omega} \) and enables a toroidal shell with variable pressure.

By controlling the phase difference and amplitude modulation across layers, the vortex shell becomes compressible and tunable:
\begin{equation}
    \vec{\omega}_{\text{bubble}} = \sum_{n=1}^N \vec{\omega}_n(t, r, \theta, \phi),
\end{equation}
which permits embedding of an inertial zone (the spacecraft) within a gradient-driven æther sink.

\section{Conclusion: Toward a VAM Warp Bubble}
Warp dynamics in VAM are modeled by swirl shells, pressure fields, and time modulation rather than spacetime distortion. This removes the need for negative energy and recasts the warp drive in fluid dynamic language. Rodin coils offer a concrete method for generating the structured vorticity necessary to sustain such a warp shell.

\section{Effective Vortex Lagrangian for Rodin-Sourced Warp Shells}
To model the dynamics of a Rodin-induced warp bubble, we derive a Lagrangian density based on vorticity and kinetic energy of the æther. We begin with the standard hydrodynamic Lagrangian in incompressible form:
\begin{equation}
    \mathcal{L}_0 = \frac{1}{2} \rho_\text{\ae} \vec{v}^2 - U(\rho_\text{\ae}),
\end{equation}
where \( U \) is a potential energy density term (e.g., gravitational or pressure).

Assuming the æther velocity derives from a vector potential:
\begin{equation}
    \vec{v} = \nabla \times \vec{A},
\end{equation}
we define the swirl helicity interaction term:
\begin{equation}
    \mathcal{L}_{\text{swirl}} = \lambda (\vec{v} \cdot \vec{\omega}) = \lambda (\nabla \times \vec{A}) \cdot (\nabla \times (\nabla \times \vec{A})),
\end{equation}
where \( \lambda \) is a coupling coefficient.

The total effective Lagrangian becomes:
\begin{equation}
    \mathcal{L}_{\text{eff}} = \frac{1}{2} \rho_\text{\ae} \abs{\nabla \times \vec{A}}^2 + \lambda (\vec{v} \cdot \vec{\omega}) - V(\vec{\omega}),
\end{equation}
where the vorticity potential \( V(\vec{\omega}) \) may take the form:
\begin{equation}
    V(\vec{\omega}) = \frac{1}{2} \mu^2 \abs{\vec{\omega}}^2 + \frac{\kappa}{4} (\vec{\omega} \cdot \vec{\omega})^2,
\end{equation}
analogous to Ginzburg-Landau or Skyrme-like models.

We define the Euler-Lagrange field equation for \( \vec{A} \) as:
\begin{equation}
    \frac{\delta \mathcal{L}_{\text{eff}}}{\delta \vec{A}} = 0 \Rightarrow \nabla \times \left( \rho_\text{\ae} \nabla \times \vec{A} - \lambda \vec{\omega} \right) + \text{nonlinear terms} = 0.
\end{equation}

For axisymmetric Rodin shells, this equation governs the evolution and self-sustainability of the swirl structure. Topological stability can be ensured by ensuring that the total helicity:
\begin{equation}
    H = \int (\vec{v} \cdot \vec{\omega}) \, d^3x
\end{equation}
is conserved.
This parallels Marquet's emphasis on maintaining the shape of the warp bubble via internal field coherence, further validating the helicity-stabilized vortex structure in VAM as a viable analogue to solitonic solutions in metric field theories.

This vortex Lagrangian formalism provides a rigorous dynamical framework for understanding Rodin-induced warp bubbles under VAM.

\section{Insights from Marquet’s Field-Based Warp Approach}

Following Marquet's reinterpretation of the Alcubierre metric as a soliton-like localized field structure \cite{marquet2012warp}, we note that the warp bubble can be recast as a region of constrained ætheric vorticity. Rather than invoking exotic spacetime curvature, the bubble boundary is encoded in a structured field gradient $f(r_s)$ corresponding to a smooth variation in vorticity $\vec{\omega}(r)$.

This matches the VAM perspective where negative energy conditions translate to local drops in ætheric pressure:

\begin{equation}
    \Delta P(r) = \frac{1}{2} \rho_\text{\ae} v^2(r) = \frac{1}{2} \rho_\text{\ae} (C_e^2 f^2(r_s)),
\end{equation}

ensuring that all warp properties arise from physical, measurable swirl profiles, not geometrical artifacts.

\bibliographystyle{unsrt}
\bibliography{vam_warp_refs}

\end{document}

\section{Wien's Displacement Law and the Vortex \text{\ae}ther Model (VAM)}

\subsection{Classical Wien's Displacement Law}

The classical \textbf{Wien displacement law} relates the wavelength of peak emission \( \lambda_{\text{max}} \) of a blackbody to its temperature \( T \) as follows:

\begin{equation}
\boxed{
\lambda_{\text{max}} T = b
}
\qquad \text{with } b = 2.897771955 \times 10^{-3}~\text{m·K}
\label{eq:wien-law}
\end{equation}

This expression is derived by maximizing the Planck radiation function \( B(\lambda, T) \) with respect to \( \lambda \).

\subsection{VAM Interpretation: Vortex–Temperature Coupling}

Within the framework of the \textbf{Vortex \text{\ae}ther Model (VAM)}, thermal radiation arises from rotational kinetic energy in localized vortex structures. We propose that temperature emerges from vortex energy density in the superfluid \text{\ae}ther medium.

Let:

\begin{itemize}
  \item \( \rho_\text{\ae} \): local \text{\ae}ther density
  \item \( |\vec{\omega}| \): local vorticity magnitude
  \item \( V_{\text{cell}} \): coarse-grained vortex core volume
\end{itemize}

Then the rotational kinetic energy density is:

\begin{equation}
U_{\text{rot}} = \frac{1}{2} \rho_\text{\ae} |\vec{\omega}|^2
\end{equation}

We define temperature through this energy density:

\begin{equation}
k_B T \sim \frac{1}{2} \rho_\text{\ae} |\vec{\omega}|^2 V_{\text{cell}}
\quad \Rightarrow \quad
T \sim \frac{\rho_\text{\ae} |\vec{\omega}|^2 V_{\text{cell}}}{2 k_B}
\label{eq:vam-temperature}
\end{equation}

\subsection{Peak Wavelength from Vorticity Frequency}

Assume a vortex core emits radiation due to its oscillation at frequency \( \nu \sim |\vec{\omega}| \), and use \( \lambda = c/\nu \), giving:

\begin{equation}
\lambda_{\text{peak}} \sim \frac{c}{|\vec{\omega}|}
\end{equation}

Substituting from Eq.~\eqref{eq:vam-temperature}, we find:

\begin{equation}
|\vec{\omega}| \sim \sqrt{ \frac{2 k_B T}{\rho_\text{\ae} V_{\text{cell}}} }
\quad \Rightarrow \quad
\boxed{
\lambda_{\text{peak}} \sim \frac{c}{\sqrt{ \dfrac{2 k_B T}{\rho_\text{\ae} V_{\text{cell}}} }}
}
\label{eq:vam-lambda}
\end{equation}

Thus, the VAM prediction is:

\begin{equation}
\boxed{
\lambda_{\text{peak}} \propto \frac{c}{\sqrt{T}}
}
\end{equation}

This deviates from the classical linear inverse law \( \lambda_{\text{peak}} \sim 1/T \), implying a slower shift in wavelength with increasing temperature.

\subsection{Reconciliation with Empirical Wien Constant}

To reconcile this with observations, define a new effective constant:

\begin{equation}
\lambda_{\text{peak}} = \frac{c}{\sqrt{ \dfrac{2 k_B T}{\rho_\text{\ae} V_{\text{cell}}} }} 
= \left( \frac{c \sqrt{\rho_\text{\ae} V_{\text{cell}}}}{\sqrt{2 k_B}} \right) T^{-1/2}
\equiv b' T^{-1/2}
\end{equation}

Comparing:

\begin{equation}
\lambda_{\text{peak}} = b' T^{-1/2}
\qquad \text{(VAM)}
\end{equation}

\begin{equation}
\lambda_{\text{peak}} = b T^{-1}
\qquad \text{(Planck)}
\end{equation}

\subsection{Future VAM Research Directions}

\begin{itemize}
    \item Derive a full VAM analogue of Planck's Law using quantized vortex mode densities.
    \item Define a vortex-based entropy function \( S_{\text{vortex}} \) and use a partition function formalism.
    \item Test the predicted deviation \( \lambda_{\text{peak}} \propto T^{-1/2} \) with astrophysical blackbody spectra.
\end{itemize}

\subsection*{References}

\begin{thebibliography}{9}

\bibitem{Planck1901}
M.~Planck,
\newblock \emph{On the Law of Distribution of Energy in the Normal Spectrum},
\newblock Annalen der Physik \textbf{309}(3), 553–563 (1901),
\newblock \href{https://doi.org/10.1002/andp.19013090310}{doi:10.1002/andp.19013090310}.

\bibitem{Wien1893}
W.~Wien,
\newblock \emph{On the Laws of the Emission of Radiation},
\newblock Annalen der Physik \textbf{294}(8), 662–669 (1893),
\newblock \href{https://doi.org/10.1002/andp.18932940806}{doi:10.1002/andp.18932940806}.

\end{thebibliography}

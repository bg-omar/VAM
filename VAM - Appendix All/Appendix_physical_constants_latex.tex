%! Author = Omar Iskandarani
%! Date = 2025-06-01
%! Title = Glossary of Vorticity, Helicity, and Rotational Flow Terms

\documentclass[a4paper, aps,preprint,superscriptaddress, 12pt]{revtex4}
\usepackage[a4paper, margin=2cm]{geometry}
\usepackage[T1]{fontenc}%
\usepackage[utf8]{inputenc}%
\usepackage{lmodern}%
\usepackage{textcomp}%
\usepackage{lastpage}%
\usepackage{float}%
\usepackage{fancyhdr}

\pagestyle{fancy}%
\usepackage{amsmath,amssymb}
\usepackage{graphicx}
\usepackage{hyperref}
\usepackage{enumitem}
\usepackage{physics}

\fancyhf{}%
\begin{document}
    \title{Glossary of Vorticity, Helicity, and Rotational Flow Terms}
    \author{Omar Iskandarani}
    \date{June 2025}
    \affiliation{Independent Researcher, Groningen, The Netherlands}
    \thanks{ORCID: \href{https://orcid.org/0009-0006-1686-3961}{0009-0006-1686-3961}}
    \email{info@omariskandarani.com}

    % Abstract
    \begin{abstract}
        This glossary summarizes key concepts in rotational fluid dynamics, both in classical fluid mechanics and within the framework of the Vortex Æther Model (VAM). Each term is defined with relevant physical context and mathematical expressions.
    \end{abstract}

    \maketitle

    \section*{Glossary of Rotational Fluid Dynamics (VAM)}
    \addcontentsline{toc}{section}{Glossary of Rotational Fluid Dynamics (VAM)}

    \small%

\begin{table}[H]
    \centering
    \footnotesize
    % \raggedright % removed to match style guide
    \renewcommand{\arraystretch}{1.3}
    \begin{tabular}{|l|l|l|l|l|}
        \hline
        \textbf{Symbol} & \textbf{Quantity} & \textbf{Value} & \textbf{Unit} & \textbf{Uncertainty} \\
        \hline
        %
        $C_e$ & Vortex-Tangential-Velocity & 1.09384563e+06 & m s^-1 & unknown \\ \hline%
        $\rho_\text{\ae}^\text{(energy)}$ & Æther Core Density & 3.89343583e+18 & J m^-3 & unknown \\ \hline%
        $\rho_\text{\ae}^\text{(fluid)}$ & Æther Vacuum Density & 7.00000000e-07 & kg m^-3 & unknown \\ \hline%
        $F_\text{\ae}^{\max}$ & Maximum force & 29.0535070 & N & unknown \\ \hline%
        $F_\text{gr}^{\max}$ & Maximum Universal Force & 3.02563891e+43 & N & unknown \\ \hline%
        $\gamma$ & Helicity-Mass coupling constant & \approx 0.005901  &  & unknown \\ \hline%
        $r_c$ & Vortex-Core radius & 1.40897017e-15 & m & exact \\ \hline%
        $c$ & Speed of light in vacuum & 2.99792458e+08 & m s^-1 & exact \\ \hline%
        $G$ & Newtonian constant of gravitation & 6.67430000e-11 & m^3 kg^-1 s^-2 & 2.2e-5 \\ \hline%
        $h$ & Planck constant & 6.62607015e-34 & J Hz^-1 & exact \\ \hline%
        $\alpha$ & Fine-structure constant & 7.29735256e-03 &  & 1.6e-10 \\ \hline%
        $R_e$ & Classical electron radius & 2.81794033e-15 & m & 1.3e-24 \\ \hline%
        $\alpha_g$ & Gravitational coupling constant & 1.75180000e-45 &  & exact \\ \hline%
        $\mu_0$ & Vacuum magnetic permeability & 1.25663706e-06 & N A^-2 & exact \\ \hline%
        $\varepsilon_0$ & Vacuum electric permittivity & 8.85418782e-12 & F m^-1 & exact \\ \hline%
        $Z_0$ & Characteristic impedance of vacuum & 3.76730313e+02 & \Omega & 1.6e-10 \\ \hline%
        $\hbar$ & Reduced Planck constant & 1.05457182e-34 & J s & exact \\ \hline%
        $L_p$ & Planck length & 1.61625500e-35 & m & 1.1e-5 \\ \hline%
        $M_p$ & Planck mass & 2.17643400e-08 & kg & 1.1e-5 \\ \hline%
        $t_p$ & Planck time & 5.39124700e-44 & s & 1.1e-5 \\ \hline%
        $T_p$ & Planck temperature & 1.41678400e+32 & K & 1.1e-5 \\ \hline%
        $q_p$ & Planck charge & 1.87554596e-18 & C & exact \\ \hline%
        $E_p$ & Planck energy & 1.95600000e+09 & J & exact \\ \hline%
    \end{tabular}
    \caption{Physical constants used in the Vortex Æther Model (VAM).}
    \label{tab:physical_constants1}
\end{table}

\begin{table}[H]
    \centering
    \footnotesize
    % \raggedright % removed to match style guide
    \renewcommand{\arraystretch}{1.3}
    \begin{tabular}{|l|l|l|l|l|}
        \hline
        \textbf{Symbol} & \textbf{Quantity} & \textbf{Value} & \textbf{Unit} & \textbf{Uncertainty} \\
        \hline
        %
        $e$ & Elementary charge & 1.60217663e-19 & C & exact \\ \hline%
        $R_\infty$ & Rydberg constant & 1.09737316e+07 & m^-1 & 1.1e-12 \\ \hline%
        $a_0$ & Bohr radius & 5.29177211e-11 & m & 1.6e-10 \\ \hline%
        $M_e$ & Electron mass & 9.10938370e-31 & kg & 3.1e-10 \\ \hline%
        $M_{proton}$ & Proton mass & 1.67262192e-27 & kg & 3.1e-10 \\ \hline%
        $M_{neutron}$ & Neutron mass & 1.67492750e-27 & kg & 5.1e-10 \\ \hline%
        $k_B$ & Boltzmann constant & 1.38064900e-23 & J K^-1 & exact \\ \hline%
        $R$ & Gas constant & 8.31446262e+00 & J mol^-1 K^-1 & exact \\ \hline%
        $\frac{1}{\alpha}$ & Fine structure constant reciprocal & 1.37035999e+02 &  & 1.6e-10 \\ \hline%
        $f_c$ & Compton frequency of the electron & 1.23558996e+20 & m & 1.0e-10 \\ \hline%
        $\Omega_c$ & Compton angular frequency of the electron & 7.76344071e+20 & m & 1.0e-10 \\ \hline%
        $\lambda_c$ & Compton wavelength of the electron & 2.42631024e-12 & m & 1.0e-10 \\ \hline%
        $\Phi_0$ & Magnetic flux quantum & 2.06783385e-15 & Wb & exact \\ \hline%
        $\varphi$ & Golden ratio (Fibonacci constant) & 1.61803399e+00 &  & 7.3e-22 \\ \hline%
        $eV$ & Electron volt & 1.60217663e-19 & J & exact \\ \hline%
        $G_F$ & Fermi coupling constant & 1.16637870e-05 & GeV^-2 & 6e-12 \\ \hline%
        $\lambda_{proton}$ & Proton Compton wavelength & 1.32140986e-15 & m & 4e-25 \\ \hline%

        $ER_\infty$ & Rydberg energy (in joules) & 2.17987236e-18 & J & 1.1e-12 \\ \hline%
        $fR_\infty$ & Rydberg frequency & 3.28984196e+15 & Hz & 1.1e-12 \\ \hline%
        $\sigma$ & Stefan-Boltzmann constant & 5.67037442e-08 & W m^-2 K^-4 & exact \\ \hline%
        $b$ & Wien displacement constant & 2.89777196e-03 & m K & exact \\ \hline%
        $k_e$ & Coulomb constant & 8.98755179e+09 & N m^2 C^-2 & exact \\ \hline%

    \end{tabular}
    \caption{Physical constants used in the Vortex Æther Model (VAM).}
    \label{tab:physical_constants}
\end{table}
%


    \section*{Glossary}




    \section*{Local Rotation Concepts (Directly related to Vorticity)}

    \textbf{Angular Velocity Vector \( \boldsymbol{\Omega} \)} \\
    Describes the instantaneous rotation rate of a fluid element. \\
    Related to vorticity by:
    \[
        \boldsymbol{\omega} = 2 \boldsymbol{\Omega}
    \]
    \textit{Role:} Connects fluid kinematics to rigid-body rotation analogy.

    \medskip
    \textbf{Curl of Velocity \( \boldsymbol{\omega} = \nabla \times \vec{v} \)} \\
    This is the formal definition of vorticity. \\
    \textit{Tensorial Form:} Antisymmetric part of the velocity gradient.

    \section*{Twisting and Linking (Helicity-related Quantities)}

    \textbf{Kinetic Helicity \( H = \int \vec{v} \cdot \boldsymbol{\omega} \, dV \)} \\
    Scalar, conserved in ideal flows (incompressible, inviscid). \\
    Measures alignment between flow velocity and vorticity. \\
    Positive/Negative values indicate right/left-handed twist.

    \medskip
    \textbf{Magnetic Helicity \( H_m = \int \vec{A} \cdot \vec{B} \, dV \)} \\
    Analogue of kinetic helicity for magnetohydrodynamics (MHD). \\
    Where \( \vec{B} = \nabla \times \vec{A} \) is the magnetic field.

    \medskip
    \textbf{Cross Helicity \( H_c = \int \vec{v} \cdot \vec{B} \, dV \)} \\
    Measures alignment of velocity and magnetic field in MHD. \\
    Shows dynamic coupling between flow and magnetic field lines.

    \medskip
    \textbf{Relative Helicity} \\
    Normalized form of helicity to compare twistedness independent of flow strength:
    \[
        H_{\text{rel}} = \frac{\int \vec{v} \cdot \boldsymbol{\omega} \, dV}
        {\left( \int |\vec{v}|^2 dV \right)^{1/2}
            \left( \int |\boldsymbol{\omega}|^2 dV \right)^{1/2}}
    \]

    \section*{Flow Structure Types (Swirl \& Vorticity-related)}

    \textbf{Spiral Vortex / Swirl Flow} \\
    Flow pattern with both radial and tangential components. \\
    Described often in cylindrical coordinates as:
    \[
        \vec{v}(r,\theta,z) = v_r(r,z) \, \hat{r} + v_\theta(r,z) \, \hat{\theta} + v_z(r,z) \, \hat{z}
    \]
    Appears in tornadoes, hurricanes, and swirl combustors.

    \medskip
    \textbf{Rankine Vortex} \\
    A model vortex with solid-body rotation inside a core and irrotational flow outside:
    \[
        \omega(r) =
        \begin{cases}
            \text{constant}, & r < r_c \\
            0, & r > r_c
        \end{cases}
    \]

    \medskip
    \textbf{Burgers Vortex} \\
    A steady solution to Navier-Stokes that balances vorticity diffusion and stretching:
    \[
        \omega(r) = \omega_0 e^{-a r^2}
    \]
    Useful in modeling turbulence and vortex stretching.

    \medskip
    \textbf{Vortex Sheet} \\
    Surface with discontinuous tangential velocity but continuous normal component. \\
    Idealization of shear layers.

    \medskip
    \textbf{Vortex Filament / Line} \\
    Line-like structure where vorticity is concentrated (mathematically a delta function). \\
    Central to Biot–Savart law and Kelvin's circulation theorem.

    \section*{Derived Quantities \& Diagnostics}

    \textbf{Q-Criterion} \\
    Identifies vortices by comparing strain and rotation:
    \[
        Q = \frac{1}{2} \left( \| \boldsymbol{\Omega} \|^2 - \| \mathbf{S} \|^2 \right)
    \]
    where \( \mathbf{S} \) is the rate-of-strain tensor and \( \boldsymbol{\Omega} \) is the rotation tensor.

    \medskip
    \textbf{Vorticity Magnitude \( |\boldsymbol{\omega}| \)} \\
    Scalar field used to visualize rotation intensity.

    \medskip
    \textbf{Enstrophy \( E = \frac{1}{2} \int |\boldsymbol{\omega}|^2 \, dV \)} \\
    Analogous to energy, but for rotation. \\
    Important in 2D turbulence where it is approximately conserved.

    \medskip
    \textbf{Circulation \( \Gamma = \oint_{\mathcal{C}} \vec{v} \cdot d\vec{l} \)} \\
    Line integral of velocity around a closed curve. \\
    Related to vorticity via Stokes' theorem:
    \[
        \Gamma = \iint_S \boldsymbol{\omega} \cdot \hat{n} \, dA
    \]

    \section*{In VAM-Specific Context}

    \textbf{Swirl Clock Time Rate} \\
    Local time evolution determined by:
    \[
        d\tau = dt \sqrt{1 - \frac{|\boldsymbol{\omega}|^2}{C_e^2}}
    \]
    Corresponds to time dilation via rotation in VAM.

    \medskip
    \textbf{Swirl Lagrangian} \\
    Includes helicity or knottedness:
    \[
        \mathcal{L}_{\text{swirl}} = \lambda \vec{v} \cdot (\nabla \times \vec{v}) = \lambda \vec{v} \cdot \boldsymbol{\omega}
    \]

    \medskip
    \textbf{Topological Charge / Linking Number} \\
    Quantifies the entanglement of vortex tubes:
    \[
        H = \sum_i \Gamma_i^2 \, \text{Lk}_i
    \]
    Related to the Hopf invariant in field theory.


    \subsection*{Vorticity $\boldsymbol{\omega}$}
    The curl of the velocity field:
    \[ \boldsymbol{\omega} = \nabla \times \mathbf{v} \]
    It describes the local spinning motion of a fluid parcel. In VAM, vorticity modulates local time and governs pressure differentials.

    \subsection*{Relative Vorticity}
    Vorticity measured relative to a rotating frame. Important in geophysical flows and rotating reference systems in æther dynamics.

    \subsection*{Angular Velocity Vector $\boldsymbol{\Omega}$}
    Describes the solid-body rotation rate of a fluid element, related to vorticity via:
    \[ \boldsymbol{\omega} = 2\boldsymbol{\Omega} \]

    \subsection*{Vortex Tube}
    A bundle of vortex lines forming a tubular region of concentrated vorticity. Vortex tubes can evolve into knotted structures in VAM.

    \subsection*{Vortex Line}
    A curve everywhere tangent to the vorticity vector field. These lines illustrate the direction and connectivity of local rotational motion.

    \subsection*{Vortex Stretching}
    Occurs when a vortex tube is elongated by the flow. In 3D, this process increases the magnitude of vorticity, conserving angular momentum.

    \subsection*{Helicity $H$}
    A scalar defined as:
    \[ H = \int \mathbf{v} \cdot \boldsymbol{\omega} \, dV \]
    It measures the extent to which velocity and vorticity are aligned. It is a topological invariant in ideal fluids and central to vortex knot theory in VAM.

    \subsection*{Kinetic Helicity Density}
    The pointwise scalar product:
    \[ h = \mathbf{v} \cdot \boldsymbol{\omega} \]
    Gives the local density of helicity, whose integral over a volume is total helicity.

    \subsection*{Cross Helicity}
    Used in magnetohydrodynamics (MHD):
    \[ H_c = \int \mathbf{v} \cdot \mathbf{B} \, dV \]
    Measures the alignment between fluid velocity and magnetic field.

    \subsection*{Magnetic Helicity}
    Another MHD quantity:
    \[ H_m = \int \mathbf{A} \cdot \mathbf{B} \, dV \]
    where $\mathbf{B} = \nabla \times \mathbf{A}$. It quantifies the linkage of magnetic field lines.

    \subsection*{Swirl}
    A qualitative term indicating rotational flow. In cylindrical coordinates, the azimuthal component $v_\theta$ is often referred to as the swirl velocity.

    \subsection*{Swirl Clock}
    In VAM, a clock governed by the internal angular frequency $\omega_0$ of a vortex core. Time dilation is given by:
    \[ d\tau = dt \sqrt{1 - \frac{\omega^2}{C_e^2}} \]

    \subsection*{Swirl Lagrangian}
    A field Lagrangian incorporating helicity terms:
    \[ \mathcal{L}_{\text{swirl}} = \lambda (\mathbf{v} \cdot \boldsymbol{\omega}) \]
    This encapsulates the topological and dynamical behavior of swirl fields in VAM.

    \subsection*{Rotation}
    A general term describing angular motion in fluids, encompassing both local (vorticity) and global (rigid body) rotations.

    \subsection*{Circulation $\Gamma$}
    Defined by:
    \[ \Gamma = \oint_{\mathcal{C}} \mathbf{v} \cdot d\mathbf{l} \]
    Measures the total rotation around a closed loop. Stokes' theorem connects it to surface-integrated vorticity.

    \subsection*{Enstrophy $\mathcal{E}$}
    A quadratic measure of rotational intensity:
    \[ \mathcal{E} = \frac{1}{2} \int |\boldsymbol{\omega}|^2 \, dV \]
    Important in turbulence analysis and vortex conservation.

    \subsection*{Q-Criterion}
    A scalar diagnostic for vortex identification:
    \[ Q = \frac{1}{2}(||\boldsymbol{\Omega}||^2 - ||\mathbf{S}||^2) \]
    where $\mathbf{S}$ is the strain-rate tensor. Positive $Q$ indicates vortex-dominated regions.

    \subsection*{Rankine Vortex}
    A model vortex with solid-body rotation inside a core ($r < r_c$) and irrotational flow outside ($r > r_c$). Simplified but useful for teaching.
    \[ \omega(r) =
    \begin{cases}
        \text{constant}, & r < r_c \\
        0, & r > r_c
    \end{cases}
    \]

    \subsection*{Burgers Vortex}
    A steady-state vortex model balancing stretching and viscous diffusion:
    \[ \omega(r) = \omega_0 e^{-a r^2} \]
    Frequently used in studies of turbulent vortex dynamics.

    \subsection*{Topological Charge}
    Quantifies the entanglement or linkage of vortex lines. In VAM, this relates to conserved quantities such as helicity and can be described using the Hopf invariant.

    \section*{References}
    \begin{itemize}[leftmargin=1.5em]
        \item G.K. Batchelor, \textit{An Introduction to Fluid Dynamics}, Cambridge University Press, 1967.
        \item H.K. Moffatt, "The degree of knottedness of tangled vortex lines", \textit{J. Fluid Mech.}, vol. 35, 1969, pp. 117–129. doi:10.1017/S0022112069000991
        \item H.K. Moffatt and A. Tsinober, "Helicity and singular structures in fluid dynamics", \textit{Ann. Rev. Fluid Mech.}, vol. 24, 1992, pp. 281–312. doi:10.1146/annurev.fl.24.010192.001433
    \end{itemize}

\end{document}

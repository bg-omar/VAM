\section{Gauge Symmetry and Group-Theoretic Embedding in VAM}

The Standard Model (SM) is built upon the gauge group:
\begin{equation}
    G_{\text{SM}} = SU(3)_C \times SU(2)_L \times U(1)_Y,
\end{equation}
which governs color charge, weak isospin, and hypercharge respectively. To elevate VAM from a descriptive to predictive framework, its knotted vortex configurations must exhibit transformations that correspond to these symmetries.

\subsection{Mapping Topological Structures to Gauge Representations}

Vortex knots in the VAM can be grouped into equivalence classes characterized by topological invariants—such as chirality, linking number, and braid group representations—which mirror quantum numbers in the SM.

\paragraph{Color SU(3):} In VAM, threefold-symmetric braid groups (e.g., $B_3$) generate knot families whose helicity states encode color charge. Triplet states correspond to permutations of knotted orientations:
\begin{equation}
    \{K_R, K_G, K_B\} \leftrightarrow \text{Red, Green, Blue color charge}.
\end{equation}
Color-neutral hadrons emerge as topologically balanced combinations (e.g., Borromean rings or composite Hopf links).

\paragraph{Weak SU(2):} Chiral asymmetry is encoded in the handedness of vortex knots. Left-handed trefoils and their mirror images can be arranged into doublets:
\begin{equation}
    \Psi_L = \begin{pmatrix} K^\text{LH}_1 \\ K^\text{LH}_2 \end{pmatrix}, \quad \Psi_R = K^\text{RH},
\end{equation}
mirroring the SM's treatment of left-handed doublets and right-handed singlets.

\paragraph{Hypercharge U(1):} Hypercharge emerges from the net writhe or circulation count of the knotted core, potentially linked to the Gauss linking integral or helicity content:
\begin{equation}
    Y \sim \int \vec{v} \cdot (\nabla \times \vec{v}) \, dV.
\end{equation}
Quantized linking flux relates to charge in units of the circulation quantum $\kappa = h/m$.

\subsection{Gauge Bosons as Vorticity Mediators}

Gauge bosons arise in VAM as inter-knot transformations or topological reconnection events.

\begin{itemize}
    \item \textbf{Photon ($\gamma$):} Linear torsional wave on a stable vortex ring.
    \item \textbf{Gluons ($g^a$):} Braiding perturbations that cyclically permute color vortex states.
    \item \textbf{$W^\pm$ and $Z^0$ Bosons:} Helicity-flipping reconnections that change knot chirality and mass scale.
\end{itemize}

The presence or absence of a reconnection event governs the local conservation of helicity and charge, paralleling SM selection rules.

\subsection{Toward a Topological Gauge Algebra}

To bridge to conventional gauge field theory, one may define vortex-space analogues of gauge covariant derivatives:
\begin{equation}
    D_\mu \psi_K = \partial_\mu \psi_K + i g \mathcal{R}_\mu^{(\text{helicity})} \psi_K,
\end{equation}
where $\mathcal{R}_\mu$ is a reconnection-induced helicity rotation operator.

These operators form a non-Abelian algebra under composition:
\begin{equation}
[\mathcal{R}_\mu, \mathcal{R}_\nu] = i f^{abc} \mathcal{R}_\lambda,
\end{equation}
suggesting a vortex-based representation of Lie algebras that mirror $SU(3)$, $SU(2)$, and $U(1)$.
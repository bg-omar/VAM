%! Author = Omar Iskandarani
%! Title = Einstein and the Æther: A Philosophical Foundation for the Vortex Æther Model (VAM)
%! Date = May 23, 2025
%! Affiliation = Independent Researcher, Groningen, The Netherlands
%! License = CC-BY 4.0
%! ORCID = 0009-0006-1686-3961
%! DOI = 10.5281/zenodo.15566101

\documentclass[a4paper,12pt]{article}

% Page Geometry
\usepackage[a4paper, margin=2cm]{geometry}
\usepackage[utf8]{inputenc}
\usepackage[T1]{fontenc}
% Language, Encoding, Fonts

\usepackage{lmodern}
\usepackage[english]{babel}

% Colors, Graphics, Diagrams
\usepackage{graphicx}
\usepackage{tikz}
\usetikzlibrary{arrows.meta, positioning}
\usepackage{pgfplots}
\pgfplotsset{compat=1.18}
\usepackage{xcolor}

\usepackage{amssymb}

\usepackage{physics}

\usepackage[font=footnotesize]{caption}
% Math and Physics
\usepackage{amsmath, amssymb, physics}
\usepackage{siunitx}

% Tables and Figures
\usepackage{float}
\usepackage{booktabs}
\usepackage{array, tabularx, makecell, multirow}
\renewcommand{\arraystretch}{1.5}
\renewcommand{\floatpagefraction}{.8}

\usepackage{subcaption}

% Code and Listings
\usepackage{listings}
\lstset{basicstyle=\ttfamily\footnotesize, breaklines=true}

% TOC Customization
\usepackage{tocloft}
\setcounter{tocdepth}{4}
\renewcommand{\cftsecfont}{\bfseries}
\renewcommand{\cftsubsecfont}{\itshape}
\renewcommand{\cftsecleader}{\cftdotfill{5}}
\renewcommand{\contentsname}{\centering \Huge\textbf{Contents}}

% Links and Metadata
\usepackage{hyperref}
\hypersetup{
    colorlinks=true,
    linkcolor=blue,
    citecolor=blue,
    urlcolor=blue,
    pdftitle={The Vortex Æther Model},
    pdfauthor={Omar Iskandarani},
    pdfkeywords={vorticity, gravity, æther, fluid dynamics, time dilation, VAM}
}
\usepackage{bookmark} % PDF bookmarks


\usepackage[most]{tcolorbox}
\tcbuselibrary{listings, breakable, skins}

\newtcolorbox{vamprinciple}[1][]{
    enhanced,
    breakable,
    colback=blue!5!white,
    colframe=blue!60!black,
    fonttitle=\bfseries,
    title=Hybrid Gravitational Coupling in the Vortex Æther Model,
    coltitle=black,
    #1
}


% Line and Hyphenation
\usepackage[none]{hyphenat}
\usepackage{amsfonts}
\usepackage{sectsty}
\sectionfont{\Large\bfseries\sffamily}
\subsectionfont{\large\bfseries\sffamily}
\usepackage{newtxtext,newtxmath}
\usepackage[scaled=0.95]{inconsolata} % for a clean monospace font
\usepackage{mathrsfs}
% Bibliography

\begin{document}


    \begin{table}[h]
        \centering
        \begin{tabular}{|l|c|l|p{7cm}|}
            \hline
            \textbf{Name} & \textbf{Symbol} & \textbf{Type} & \textbf{Description and Role} \\
            \hline
            Chronos-time & $\tau$ & Relative / Measurable & Sequential time; proper time experienced by localized systems in motion through the æther. Core for modeling time dilation. \\
            \hline
            Aithēr-time & $\mathcal{N}$ & Absolute / Universal & The invariant universal present; a metaphysical and ontological background for all temporal flow. \\
            \hline
            Swirl Clock & $\circlearrowleft$ or $S(t)$ & Local / Cyclical & Internal clock-like rhythm of a vortex knot. Tracks phase, rotation, or identity shift through time. \\
            \hline
            Kairos Moment & $\mathbb{K}$ & Threshold / Emergent & The qualitative, transformational moment when a system undergoes critical phase alignment or collapse. \\
            \hline
            Æther Frame & $\Xi_0$ & Reference Frame & Hypothetical inertial frame where the æther medium is at rest. Used for symmetry-breaking and baseline flow analysis. \\
            \hline
            Vortex Proper Time & $T_v$ & Derived / Topological & Time internal to the closed knot or vortex loop. Emerges from geodesic paths and twist topology. \\
            \hline
            Now-Point & $\nu_0$ & Local Event / Temporal Slice & Precise location in spacetime where a point in the æther intersects the universal present. Useful in field causality. \\
            \hline
        \end{tabular}
        \caption{Temporal constructs used in the Vortex Æther Model. These notations distinguish between measurable time, absolute background time, internal vortex phase, and field-causality moments.}
        \label{tab:VAM_time}
    \end{table}

    \begin{align}
        \text{(1) Vortex Proper Time Evolution:}\\
        \quad & \dv{\tau}{\mathcal{N}} = \gamma^{-1}(\vec{v}) \\
        \text{(2) Swirl Clock Gradient:}\\ \quad & \nabla S(t) = \frac{\partial \vec{S}}{\partial \mathcal{N}} + \omega(\tau) \hat{n} \\
        \text{(3) Field Tensor Modulation (Æther-relative):} \\\quad & F^{\mu\nu}(\Xi_0) = \partial^\mu A^\nu - \partial^\nu A^\mu + \phi(\circlearrowleft) \delta^{\mu\nu} \\
        \text{(4) Ætheric Causality Surface:}\\ \quad & \Sigma_{\nu_0} = \{ x^\mu \ | \ \tau(x) = \mathcal{N} \} \\
        \text{(5) VAM Energy Conservation in Æther Frame:}\\ \quad & \dv{E}{\mathcal{N}} + \nabla \cdot \vec{J} = \mathbb{K}(\vec{x}, \tau)
    \end{align}


\end{document}
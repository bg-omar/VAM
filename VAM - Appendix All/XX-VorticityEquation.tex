%! Author = Omar Iskandarani
%! Title =  Governing Equations of Vorticity in Æther Dynamics
%! Date = 6/20/2025
%! Affiliation = Independent Researcher, Groningen, The Netherlands
%! License = CC-BY 4.0
%! ORCID = 0009-0006-1686-3961
%! DOI = 10.5281/zenodo.15706547
\newcommand{\appendixtitle}{ Governing Equations of Vorticity in Æther Dynamics }
\newcommand{\appendixdoi}{10.5281/zenodo.15706547}


\documentclass[a4paper,12pt]{article}
\usepackage{import}
\usepackage{subfiles}
% vamstyle.sty
\NeedsTeXFormat{LaTeX2e}
\ProvidesPackage{vamstyle}[2025/06/13 VAM unified style]

\newif\ifvamdraft
% Uncomment the next line to enable draft mode:
% \vamdrafttrue

\ifvamdraft
  \RequirePackage{showframe} % shows margins for debugging
\fi

\RequirePackage{ifthen}
\newboolean{vamstyleloaded}
\ifthenelse{\boolean{vamstyleloaded}}{}{\setboolean{vamstyleloaded}{true}

\RequirePackage[a4paper, margin=2cm]{geometry}

% -- Fonts and Language --
\RequirePackage[T1]{fontenc}
\RequirePackage[utf8]{inputenc}
\RequirePackage[english]{babel}
\RequirePackage{mathpazo}           % or newtxtext/newtxmath
\RequirePackage[scaled=0.95]{inconsolata}
\RequirePackage{helvet}

% Math and Physics
\RequirePackage{amsmath, amssymb, mathrsfs, physics}
\RequirePackage{siunitx}
\sisetup{per-mode=symbol}

% -- Tables and Figures --
\RequirePackage{graphicx, float, booktabs}
\RequirePackage{array, tabularx, multirow, makecell}
\RequirePackage[font=footnotesize, labelfont=bf]{caption}
\RequirePackage{subcaption}
% Safe wide table environment (auto-fit to text width)
\newcolumntype{Y}{>{\centering\arraybackslash}X} % Like 'X' but centered
\newenvironment{tighttable}[1][] % optional argument = caption
  {\begin{table}[H]\centering\renewcommand{\arraystretch}{1.3}
   \begin{tabularx}{\textwidth}{#1}}
  {\end{tabularx}\end{table}}
% Force fit large tables without changing layout
\RequirePackage{etoolbox}
\newcommand{\fitbox}[2][\linewidth]{\makebox[#1]{\resizebox{#1}{!}{#2}}}

% Graphics and Diagrams
\RequirePackage{tikz}
\usetikzlibrary{arrows.meta, positioning}
\RequirePackage{pgfplots}
\pgfplotsset{compat=1.18}
\RequirePackage{xcolor}

% -- Code Listings --
\RequirePackage{listings}
\lstset{basicstyle=\ttfamily\footnotesize, breaklines=true}

% TOC Customization
\RequirePackage{tocloft}
\setcounter{tocdepth}{2}
\renewcommand{\cftsecfont}{\bfseries}
\renewcommand{\cftsubsecfont}{\itshape}
\renewcommand{\cftsecleader}{\cftdotfill{.}}
\renewcommand{\contentsname}{\centering \Huge\textbf{Contents}}

% Section Fonts
\RequirePackage{sectsty}
\sectionfont{\Large\bfseries\sffamily}
\subsectionfont{\large\bfseries\sffamily}

% Bibliography
\RequirePackage[numbers]{natbib}

% PDF Links and Metadata
\RequirePackage{hyperref}
\hypersetup{
    colorlinks=true,
    linkcolor=blue,
    citecolor=blue,
    urlcolor=blue,
    pdftitle={The Vortex Æther Model},
    pdfauthor={Omar Iskandarani},
    pdfkeywords={vorticity, gravity, æther, fluid dynamics, time dilation, VAM}
}

\urlstyle{same}
\RequirePackage{bookmark}

% Line Breaking and Style
\RequirePackage[none]{hyphenat}
\sloppy


\usepackage[most]{tcolorbox}
\usepackage{graphicx}
\usepackage{titling}

\pretitle{\begin{center}\LARGE\bfseries}
\posttitle{\par\end{center}\vskip 0.5em}
\preauthor{\begin{center}\large}
\postauthor{\end{center}}
\predate{\begin{center}\small}
\postdate{\end{center}}


\endinput
}
% -- End of vamstyle.sty --
% vamappendixsetup.sty

\newcommand{\titlepageOpen}{
  \begin{titlepage}
    \thispagestyle{empty}
    \centering
    \vspace*{2cm}
    {\Huge\bfseries \appendixtitle \par}
    \vspace{1cm}
    {\Large\itshape \appendixauthor \par}
    \vspace{0.5cm}
    {\small \appendixaffil \par}
    ORCID: \href{https://orcid.org/\appendixorcid}{\appendixorcid} \\
    DOI: \href{https://doi.org/\appendixdoi}{\appendixdoi} \\
    \vspace{0.5cm}
    {\large \today \par}
    \vspace{1cm}
}

\newcommand{\titlepageClose}{
  \vfill
  \end{titlepage}
}


\begin{document}
\titlepageOpen

        \begin{abstract}
        This document presents a mathematical treatment of vorticity and time structure within the framework of the Vortex Æther Model (VAM), a fluid-dynamic reformulation of gravitation and temporal evolution. It introduces the core vorticity equation in natural coordinates, derived from the dynamics of an incompressible, inviscid æther medium. By interpreting vorticity as a measure of atomic clock delay, we couple swirl energy to experienced time and obtain expressions for vorticity in terms of local velocity gradients and flow curvature.

        The appendices develop key substructures of VAM, including the topological and energetic conditions that trigger irreversible events, called Kairos moments, where the flow evolution becomes non-analytic. These singularities partition æther-time into epochs and correspond to phenomena such as vortex reconnection, swirl pressure rupture, or helicity discontinuities. Each temporal mode—Aithēr-Time (\( \mathcal{N} \)), Chronos-Time (\( \tau \)), Swirl Clock (\( S(t) \)), and Vortex Proper Time (\( T_v \))—is formally defined and related to physical vortex observables.

        Altogether, the work establishes a temporally stratified æther theory with testable dynamical laws, offering an alternative to spacetime curvature by grounding time dilation and mass interactions in structured vorticity fields.
    \end{abstract}


\titlepageClose
    \begin{figure}[H]
        \centering
        \includegraphics[width=0.95\textwidth]{images/image_vortex}
        \caption{Universal time dilation formula in the Vortex Æther Model. The clock rate decreases with increasing relative velocity~$|\vec{v}_{\mathrm{rel}}|$ with respect to the æther. At $|\vec{v}_{\mathrm{rel}}| = c$ time stops.}
        \label{fig:TijdsvertragingRelatieveBeweging}
    \end{figure}
    \vfill
%



    \appendix
    \def\standalonechapter{false}
    \subfile{sections/500_movement_of_free_Æther_particles.tex}  % Sets standalone=false
    \subfile{sections/510_vortex_pressure.tex}  % Sets standalone=false
    \subfile{sections/520_vorticity_natural_coords.tex}
    \subfile{sections/530_FundamentalEquationsOfVortexDynamics.tex}
    \subfile{sections/540_relative_vorticity_2knots.tex}
    \subfile{sections/550_kairos_moments_topological_transitions.tex}

    \bibliographystyle{unsrt}
    \bibliography{../references}

\end{document}
%! Author = Omar Iskandarani
%! Date = 2025-06-13

% === Metadata ===
\newcommand{\appendixtitle}{Appendix: Vortex Pressure, Stress, and Vorticity}
\newcommand{\appendixauthor}{Omar Iskandarani}
\newcommand{\appendixaffil}{Independent Researcher, Groningen, The Netherlands}
\newcommand{\appendixdoi}{10.5281/zenodo.15566319}
\newcommand{\appendixorcid}{0009-0006-1686-3961}

\ifdefined\standalonechapter\else
% Standalone mode
\documentclass[12pt]{article}
\usepackage[a4paper, margin=2cm]{geometry}
\usepackage{ifthen} % we can use it safely now
\usepackage{import}
\usepackage{subfiles}
\usepackage{hyperref}
\usepackage{graphicx}
\usepackage{amsmath, amssymb, physics}
\usepackage{siunitx}
\usepackage{tikz}
\usepackage{booktabs}
\usepackage{caption}
\usepackage{array, tabularx}
\usepackage{listings}
\usepackage{bookmark}
\usepackage{newtxtext,newtxmath}
\usepackage[scaled=0.95]{inconsolata}
\usepackage{mathrsfs}
% vamappendixsetup.sty

\newcommand{\titlepageOpen}{
  \begin{titlepage}
    \thispagestyle{empty}
    \centering
    \vspace*{2cm}
    {\Huge\bfseries \appendixtitle \par}
    \vspace{1cm}
    {\Large\itshape \appendixauthor \par}
    \vspace{0.5cm}
    {\small \appendixaffil \par}
    ORCID: \href{https://orcid.org/\appendixorcid}{\appendixorcid} \\
    DOI: \href{https://doi.org/\appendixdoi}{\appendixdoi} \\
    \vspace{0.5cm}
    {\large \today \par}
    \vspace{1cm}
}

\newcommand{\titlepageClose}{
  \vfill
  \end{titlepage}
}

\begin{document}

    % === Title page ===
    \titlepageOpen

    \begin{abstract}
        Abstracts are not typically included in appendices, but for standalone it is needed.
    \end{abstract}

    \titlepageClose
    \fi

% ============= Begin of content ============
    \section{\appendixtitle}


    \subsection*{Vortex Pressure Relations}
    In a constantly rotating vortex tube, the pressure at the axis of rotation is $P_0$. The pressure at the vortex edge $P_1$ is given by:
    \begin{equation*}
        P_1 = P_0 + \frac{1}{2} \rho c^2,
    \end{equation*}
    where $\rho$ is the density and $c$ the tangential velocity at the vortex edge. The central axis of rotation can also be interpreted as the center of gravity within the vortex.

    The pressure parallel to the axes is:
    \begin{equation*}
        P_2 = P_0 + \frac{1}{4} \rho c^2.
    \end{equation*}

    For multiple parallel vortex tubes forming a medium with pressure $P_2$ along the axes and pressure $P_1$ in a perpendicular direction, we obtain:
    \begin{equation*}
        P_1 - P_2 = \frac{1}{4} \rho c^2.
    \end{equation*}
    For an irrotational vortex where $N$ depends on the angular frequency and density distribution:
    \begin{equation*}
        P_1 - P_2 = N \rho c^2.
    \end{equation*}

    \subsection*{Stress Tensor Components}
    Defining the direction cosines of vortex tubes relative to the coordinate axes $(x, y, z)$ as $l, m, n$, we express the normal and tangential stresses:
    \begin{align}
        P_{xx} &= \rho c^2 l^2 - P_1, & P_{yz} &= \rho c^2 m n, \\
        P_{yy} &= \rho c^2 m^2 - P_1, & P_{zx} &= \rho c^2 n l, \\
        P_{zz} &= \rho c^2 n^2 - P_1, & P_{xy} &= \rho c^2 l m.
    \end{align}

    Velocity components are given by:
    \begin{equation*}
        u = c l, \quad v = c m, \quad w = c n.
    \end{equation*}
    Rewriting the stress tensor:
    \begin{align}
        P_{xx} &= \rho u^2 - P_1, & P_{yz} &= \rho v w, \\
        P_{yy} &= \rho v^2 - P_1, & P_{zx} &= \rho u w, \\
        P_{zz} &= \rho w^2 - P_1, & P_{xy} &= \rho u v.
    \end{align}

    \subsection*{Equilibrium of Stresses and Force Components}
    According to equilibrium laws, the forces in the $x$, $y$, and $z$ directions per unit volume satisfy:
    \begin{align}
        X &= \frac{d P_{xx}}{dx} + \frac{d P_{xy}}{dy} + \frac{d P_{xz}}{dz}, \\
        Y &= \frac{d P_{yx}}{dx} + \frac{d P_{yy}}{dy} + \frac{d P_{yz}}{dz}, \\
        Z &= \frac{d P_{zx}}{dx} + \frac{d P_{zy}}{dy} + \frac{d P_{zz}}{dz}.
    \end{align}

    Substituting stress tensor components and using the velocity relations:
    \begin{equation*}
        u \frac{du}{dx} + v \frac{dv}{dx} + w \frac{dw}{dx} = \frac{1}{2} \frac{d}{dx} (u^2 + v^2 + w^2),
    \end{equation*}
    we derive:
    \begin{align}
        X &= \frac{1}{2} \rho \frac{d}{dx} (u^2 + v^2 + w^2) + u \rho \left( \frac{du}{dx} + \frac{dv}{dy} + \frac{dw}{dz} \right) - v \rho (2 \zeta) + w \rho (2 \eta) - \frac{1}{\rho} \frac{dP_1}{dx}, \\
        Y &= \frac{1}{2} \rho \frac{d}{dy} (c^2) + v \rho \left( \frac{du}{dx} + \frac{dv}{dy} + \frac{dw}{dz} \right) - w \rho (2 \xi) + u \rho (2 \zeta) - \frac{1}{\rho} \frac{dP_1}{dy}, \\
        Z &= \frac{1}{2} \rho \frac{d}{dz} (c^2) + w \rho \left( \frac{du}{dx} + \frac{dv}{dy} + \frac{dw}{dz} \right) - u \rho (2 \eta) + v \rho (2 \xi) - \frac{1}{\rho} \frac{dP_1}{dz}.
    \end{align}

    \subsection*{Connection to Vorticity and Coriolis Acceleration}
    Comparing with prior formulations, two additional accelerations emerge due to fluid rotation:
    \begin{equation*}
        \frac{1}{2} \rho \frac{d}{dx} (u^2 + v^2 + w^2),
    \end{equation*}
    which represents Coulomb acceleration, and:
    \begin{equation*}
        -v (2 \zeta) + w (2 \eta),
    \end{equation*}
    which corresponds to vorticity components along the $x$-axis. This term is recognized as the Coriolis acceleration.

    \subsection*{Conclusion}
    This derivation reveals the interplay between vortex pressure, stress tensors, and vorticity effects, including Coriolis and Coulomb accelerations. These results provide a basis for analyzing rotating fluid systems and their implications in Æther dynamics and vortex interactions.

% === Bibliography (only for standalone) ===
    \ifdefined\standalonechapter
    % Being imported from main.tex — do nothing
    \else
    \bibliographystyle{unsrt}
    \bibliography{../../references}
    \end{document}
    \fi


    % === Bibliography (only for standalone) ===
    \ifdefined\standalonechapter
    % Being imported from main.tex — do nothing
    \else
    \bibliographystyle{unsrt}
    \bibliography{../../references}
    \end{document}
    \fi
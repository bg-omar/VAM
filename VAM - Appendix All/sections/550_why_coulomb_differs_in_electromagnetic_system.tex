
\section{Why the "Naive" Coulomb Force Calculation Differs from the Maximum Force Scale in Electromagnetic Systems}


This article explores the stark difference between the \textit{naive} Coulomb force calculation at the electron boundary ($\sim 10^{-21} ; \mathrm{N}$) and the \textit{maximum force} scale ($\sim 29 ; \mathrm{N}$). The key lies in understanding how the factor $c^3 \times 2 \alpha^2$ and $\alpha = \frac{2C_e}{c}$ play pivotal roles in bridging purely mechanical inertial forces with electromagnetic tension at the Coulomb barrier. This relationship not only provides insight into fundamental electromagnetic interactions but also suggests a deeper interplay between inertia and field interactions that govern microscopic particle dynamics.


\subsection*{The Two Different Force Scales}


\subsection*{Coulomb Force: $\sim 10^{-21} ; \mathrm{N}$}


When considering the electron mass, the Compton angular frequency $\omega_c \approx 7.76 \times 10^{20} ; \mathrm{s^{-1}}$, and the radius $r_c \approx 10^{-15} ; \mathrm{m}$, the Coulomb force can be expressed as:
\begin{equation*}
F_\text{coulomb} = m_e r_c \omega_c^2 \sim 10^{-21} ; \mathrm{N}.
\end{equation*}
This value is minuscule due to the extremely light mass of the electron ($\sim 10^{-30} ; \mathrm{kg}$) and the small radius $r_c$ in strict SI units. However, this simple calculation does not take into account the intricate role of electromagnetic energy density, spin coupling, or relativistic corrections that emerge in high-energy field configurations.


\subsection*{Maximum Force Scale: $F_{\max} \approx 29 ; \mathrm{N}$}


Separately, a maximum tension or force scale is defined as:
\begin{equation*}
F_{\max} \approx \frac{e^2}{4\pi \varepsilon_0 r_c^2} \sim 29 ; \mathrm{N},
\end{equation*}
which originates from the Coulomb gradient at radius $r_c$. Some calculations adopt fractions such as $\frac{1}{4}$ or $\frac{1}{2}$ of the raw Coulomb gradient ($\sim 116$ N) to represent the \textit{maximum force} in the system, reflecting electromagnetic field tension and stress-energy distributions.


The disparity in these force magnitudes arises from the stark difference between localized mass-driven effects and field-driven tension. While inertial considerations govern the Coulomb force estimated for the electron's rotation, field interactions significantly amplify the effective force when evaluating electromagnetic interactions at the Coulomb barrier.


\subsection*{The Role of $c^3 \times 2 \alpha^2$}


\subsection*{Why This Factor Appears}


The enormous difference between these two force scales bridges:
\begin{itemize}
\item A purely mechanical perspective (inertial forces on $m_e$) at radius $r_c$.
\item An electromagnetic perspective from the Coulomb gradient, incorporating stress-energy interactions and field line tension.
\end{itemize}
This bridging factor appears as:
\begin{equation*}
\frac{F_{\max}}{F_\text{coulomb}} \approx c^3 \times 2 \alpha^2 \quad \text{(up to dimensionless factors)}.
\end{equation*}
The factor $c^3$ reflects the transition from mass/energy scales to field gradients, while $\alpha^2$ introduces the fine-structure constant, scaling velocity ratios and field strength, ultimately capturing the interaction of local inertia with the surrounding electromagnetic environment.


\subsection*{The Fine-Structure Constant and its Relation to $C_e$}


The fine-structure constant $\alpha \approx \frac{1}{137}$ can also be expressed as:
\begin{equation*}
\alpha = \frac{2C_e}{c},
\end{equation*}
where $C_e$ is a characteristic tangential velocity at the vortex core, linked to vortex-based interpretations of charge and spin. Substituting $\alpha$ into $\alpha^2$, we find:
\begin{equation*}
\alpha^2 \approx \left(\frac{2C_e}{c}\right)^2 \approx \frac{4C_e^2}{c^2}.
\end{equation*}
Hence, any expression involving $c^3 \times \alpha^2$ can be rewritten in terms of $C_e^2$ and $c$:
\begin{equation*}
F_{\max} \sim F_\text{coulomb} \times c^3 \times 2\alpha^2 \sim F_\text{coulomb} \times \frac{8C_e^2}{c}.
\end{equation*}


\subsection*{Physical Interpretation and Implications}


\subsection*{Naive Coulomb Force: $\sim 10^{-21} ; \mathrm{N}$}


This force arises from treating the electron as a mass spinning at $\omega_c$ with radius $r_c$ in classical inertial terms. The resulting value is tiny because the electron's mass is extremely small. However, when considering the electron's behavior in quantum electrodynamics, we observe additional contributions from vacuum polarization, spin-orbit coupling, and self-energy corrections that modify this naive estimation.


\subsection*{Electromagnetic Tension: $\sim 29 ; \mathrm{N}$}


The \textit{maximum force} represents the electromagnetic field-line tension or Coulomb gradient at $r_c$. This value is orders of magnitude larger because it accounts for electromagnetic energy stored in the vortex configuration rather than the inertial spin-out of a single electron mass. In quantum field theory, this force also corresponds to constraints placed on energy density distribution, balancing charge interactions within the vacuum structure.


\subsection*{Why $c^3 \times 2 \alpha^2$?}


\begin{itemize}
\item The $c^3$ factor appears in electromagnetic or vacuum tension contexts, bridging mass/energy to field gradients.
\item The $\alpha^2$ factor scales the mismatch between purely mechanical inertial forces and electromagnetic tension, tying velocity ratios or "bare electron vs. observed electron" arguments.
\item This factor encapsulates fundamental field interactions, indicating how vacuum fluctuations and electron self-energy effects manifest within the force balance.
\end{itemize}


Thus, the difference between these two force scales stems from:
\begin{itemize}
\item The mechanical spin-out of a $9 \times 10^{-31} ; \mathrm{kg}$ object at $r_c \sim 10^{-15} ; \mathrm{m}$ and $\omega_c \sim 10^{21} ; \mathrm{s^{-1}}$.
\item The much larger electromagnetic tension from field lines, which exceeds the mechanical force by $\sim 10^{20}$, revealing a fundamental shift from inertial-based models to field-driven energy distributions.
\end{itemize}


\subsection*{Conclusion}


When comparing:
\begin{equation*}
\text{(Naive Coulomb force)} \sim 10^{-21} ; \mathrm{N} \quad \text{vs.} \quad \text{(Maximum force scale)} \sim 29 ; \mathrm{N},
\end{equation*}
and noting the bridging factor $\approx c^3 \times 2 \alpha^2$, the essential difference becomes clear:
\begin{itemize}
\item The naive Coulomb force reflects inertial spin-out forces on a tiny electron mass.
\item The maximum force represents electromagnetic tension, far exceeding the inertial force due to field interactions.
\end{itemize}
This disparity underscores how $c^3 \times 2 \alpha^2$ transforms classical mechanical estimates into the electromagnetic "force budget," linking vacuum energy density, vortex structures, and charge interactions at fundamental scales.


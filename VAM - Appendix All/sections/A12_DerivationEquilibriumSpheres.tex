%! Author = Omar Iskandarani
%! Date = 3/14/2025


\textbf{To derive the equilibrium spheres equation:}

\[
\Delta U = Q - W = \Delta \left( \frac{1}{2} \rho \int v^2 \, dV + \int P \, dV \right)
\]

using fundamental principles and incorporating constants such as $C_e$ (energy-related constant), $R_c$ (characteristic radius), and $F_{\max}$ (maximum force), we will proceed step by step.





\subsection*{Step 1: First Principles of Thermodynamics}
From the First Law of Thermodynamics, we know:

dU = \delta Q - \delta W

where:

\begin{itemize}
\item $dU$ is the change in internal energy,
\item $\delta Q$ is the heat input,
\item $\delta W$ is the work done by the system.
\end{itemize}
For a fluid system with vortex structures and surrounding equilibrium spheres, we must account for the kinetic energy due to fluid velocity vvv and potential energy due to pressure $P$.

Thus, we write the total internal energy as:

U = \frac{1}{2} \rho \int v^2 \, dV + \int P \, dV



\subsection*{Step 2: Energy Balance with Spherical Equilibrium Constraints}
Now, consider the spheres of equilibrium pressure that surround a knotted vortex. The pressure distribution is influenced by:

\begin{enumerate}
\item The characteristic radius RcR_cRc, which defines the boundary of these equilibrium spheres.
\item The maximum force $F_{\max}$, which sets a limit to the pressure interactions within the system.
\item The energy constant CeC_eCe, which is a fundamental proportionality factor for energy distribution.
\end{enumerate}
\subsubsection*{Velocity Integral Contribution}
For a fluid with density ρ\rhoρ, the kinetic energy contribution within the equilibrium sphere of radius RcR_cRc is:

U_\text{kin} = \frac{1}{2} \rho \int v^2 \, dV

Using the characteristic velocity scale veqv_\text{eq}veq at equilibrium:

v_\text{eq}^2 = \frac{F_{\max}}{\rho R_c}

we approximate:

U_\text{kin} \approx \frac{1}{2} \rho v_\text{eq}^2 \cdot V_\text{eq}

where VeqV_\text{eq}Veq is the volume of the equilibrium sphere:

V_\text{eq} = \frac{4}{3} \pi R_c^3

Thus:

U_\text{kin} = \frac{2}{3} \pi F_{\max} R_c^2



\subsubsection*{Pressure Integral Contribution}
The potential energy contribution due to pressure inside the equilibrium sphere:

U_\text{press} = \int P \, dV

Using an equilibrium pressure scaling:

P_\text{eq} = \frac{F_{\max}}{4 \pi R_c^2}

where Aeq=4πRc2A_\text{eq} = 4 \pi R_c^2Aeq=4πRc2 is the surface area of the equilibrium sphere.

Thus, the pressure integral:

U_\text{press} = \frac{1}{3} F_{\max} R_c



\subsection*{Step 3: Final Energy Balance Equation}
Summing both contributions:

U = \frac{2}{3} \pi F_{\max} R_c^2 + \frac{1}{3} F_{\max} R_c

Now, considering a differential energy change due to heat QQQ and work WWW:

\Delta \left( \frac{2}{3} \pi F_{\max} R_c^2 + \frac{1}{3} F_{\max} R_c \right) = Q - W

To include an energy constant CeC_eCe, we set:

C_e = \frac{F_{\max}}{R_c}

which normalizes force over a characteristic length, giving:

\Delta \left( \frac{2}{3} \pi C_e R_c^3 + \frac{1}{3} C_e R_c^2 \right) = Q - W

Thus, the generalized energy equation for the equilibrium spheres is:

\Delta U = Q - W = \Delta \left( \frac{2}{3} \pi C_e R_c^3 + \frac{1}{3} C_e R_c^2 \right)



\subsection*{Conclusion}
\begin{itemize}
\item This formulation links thermodynamic energy change to the equilibrium pressure spheres surrounding vortex knots.


\item The equation shows that heat addition or work done on the system affects the energy stored in the knotted vortex system, via pressure-volume interactions.


\item The constants Fmax⁡F_{\max}Fmax, RcR_cRc, and CeC_eCe determine how these energy exchanges occur within the topological fluid structure.


\end{itemize}
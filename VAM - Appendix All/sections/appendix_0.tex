%! Author = Omar Iskandarani
%! Date = 2025-06-13

% === Metadata ===
\newcommand{\papertitle}{The Density of the Æther: A Modern Derivation within the Vortex Æther Model}
\newcommand{\paperauthor}{Omar Iskandarani}
\newcommand{\paperaffil}{Independent Researcher, Groningen, The Netherlands}
\newcommand{\paperdoi}{10.5281/zenodo.15566319}
\newcommand{\paperorcid}{0009-0006-1686-3961}

\ifdefined\standalonechapter\else
  % Standalone mode
  \documentclass[12pt]{article}
  \usepackage[a4paper, margin=2cm]{geometry}
  \usepackage{ifthen} % we can use it safely now
  \usepackage{import}
  \usepackage{subfiles}
  \usepackage{hyperref}
  \usepackage{graphicx}
  \usepackage{amsmath, amssymb, physics}
  \usepackage{siunitx}
  \usepackage{tikz}
  \usepackage{booktabs}
  \usepackage{caption}
  \usepackage{array, tabularx}
  \usepackage{listings}
  \usepackage{bookmark}
  \usepackage{newtxtext,newtxmath}
  \usepackage[scaled=0.95]{inconsolata}
  \usepackage{mathrsfs}
  % vamappendixsetup.sty

\newcommand{\titlepageOpen}{
  \begin{titlepage}
  \thispagestyle{empty}
  \centering
  {\Huge\bfseries \papertitle \par}
  \vspace{1cm}
  {\Large\itshape\textbf{Omar Iskandarani}\textsuperscript{\textbf{*}} \par}
  \vspace{0.5cm}
  {\large \today \par}
  \vspace{0.5cm}
}

% here comes abstract
\newcommand{\titlepageClose}{
  \vfill
  \null
  \begin{picture}(0,0)
  % Adjust position: (x,y) = (left, bottom)
  \put(-200,-40){  % Shift 75pt left, 40pt down
    \begin{minipage}[b]{0.7\textwidth}
    \footnotesize % One step bigger than \tiny
    \renewcommand{\arraystretch}{1.0}
    \noindent\rule{\textwidth}{0.4pt} \\[0.5em]  % ← horizontal line
    \textsuperscript{\textbf{*}}Independent Researcher, Groningen, The Netherlands \\
    Email: \texttt{info@omariskandarani.com} \\
    ORCID: \texttt{\href{https://orcid.org/0009-0006-1686-3961}{0009-0006-1686-3961}} \\
    DOI: \href{https://doi.org/\paperdoi}{\paperdoi} \\
    License: CC-BY 4.0 International \\
    \end{minipage}
  }
  \end{picture}
  \end{titlepage}
}
  \begin{document}

  % === Title page ===
  \titlepageOpen

  \begin{abstract}
      This article refines previous estimates of the Æther's density, $\rho_{\text{\ae}}$, in the Vortex Æther Model (VAM). Integrating findings from quantum vortex dynamics, superfluid helium, gravitomagnetic frame-dragging, and cosmological vacuum energy, we propose constrained ranges for $\rho_{\text{\ae}}^{\text{(fluid)}}$ and $\rho_{\text{\ae}}^{\text{(energy)}}$ and examine their implications across scales.
  \end{abstract}

  \titlepageClose
\fi

% ============= Begin of content ============

\section{Introduction}

In VAM, the Æther is a structured, inviscid medium supporting vorticity and energy transfer. Two key densities are defined:
\begin{itemize}
    \item $\rho_{\text{\ae}}^{\text{(fluid)}}$: mass density akin to a classical fluid.
    \item $\rho_{\text{\ae}}^{\text{(energy)}}$: energy density stored in vorticity.
\end{itemize}

\section{Vorticity and Energy Density}

The vorticity energy density is:
\[
U_{\text{vortex}} = \frac{1}{2} \rho_{\text{\ae}}^{\text{(energy)}} |\vec{\omega}|^2
\]
with
\[
|\vec{\omega}| = \sqrt{\omega_x^2 + \omega_y^2 + \omega_z^2}
\]

\section{Quantum Anchoring of Vorticity}

To ground the model in fundamental physics, we define vorticity using the fine-structure constant $\alpha$ and Compton angular frequency $\omega_C$:
\[
|\vec{\omega}| = \alpha \cdot \omega_C = \alpha \cdot \frac{m_e c^2}{\hbar}
\]

Given $R_c = \frac{1}{2} r_e$, the density becomes:
\[
\rho_{\text{\ae}}^{\text{(fluid)}} = \frac{2 m_e c^2}{\left(\alpha \cdot \frac{m_e c^2}{\hbar}\right)^2 R_c^3} \approx 7 \times 10^{-7} \text{ kg m}^{-3}
\]

\section{Experimental and Theoretical Support}

Empirical support includes superfluid helium vortex dynamics~\cite{jackson2021}, frame-dragging analogs in vortex fields~\cite{paris2015}, and superconductive gravitational coupling~\cite{santiago2011}.

\section{Vacuum Energy Context}

\[
\rho_{\text{vacuum}} = \frac{\Lambda c^2}{8 \pi G} \sim 5 \times 10^{-9} \text{ kg m}^{-3}
\]

\section{Physical Implications}

\paragraph{Pressure Gradients}
\[
\Delta P = -\frac{\rho_{\text{\ae}}^{\text{(fluid)}}}{2} \nabla |\vec{\omega}|^2
\]

\paragraph{Refractive Index Shifts}
\[
\Delta n = \frac{\rho_{\text{\ae}}^{\text{(energy)}} |\vec{\omega}|^2}{c^2}
\]

\paragraph{Vortex Mass}
\[
M_{\text{vortex}} = \int_V \frac{\rho_{\text{\ae}}^{\text{(energy)}}}{2} |\vec{\omega}|^2 \, dV
\]

\section{Conclusion}

Using quantum constants to define Æther properties bridges microscopic and cosmological theories. The refined value of $\rho_{\text{\ae}}^{\text{(fluid)}}$ supports both theoretical elegance and experimental plausibility.

% ============== End of content =============

\ifdefined\standalonechapter\else
    \bibliographystyle{unsrt}
    \bibliography{../../references}
    \end{document}
\fi
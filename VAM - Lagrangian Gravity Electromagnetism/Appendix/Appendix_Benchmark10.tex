\section{Benchmark 10: Vortex-Based Lagrangian and Standard Model Mapping}

In this final benchmark, we construct a correspondence between Standard Model (SM) particles and topologically distinct vortex excitations in the structured æther. Each particle species emerges as a specific class of knotted, linked, or twisted vorticity fields embedded in 3D Euclidean space with absolute time.

\subsection{Topological Classification of SM Particles}

\begin{table}[H]
\centering
\begin{tabular}{|l|l|}
\hline
\textbf{SM Particle} & \textbf{VAM Topological Structure} \\
\hline
Photon & Dipole Vortex Ring (massless, chiral translation) \\
Electron & Trefoil Knot \( T(2,3) \), spin-\( \frac{1}{2} \), negative charge \\
Proton & 3-Linked Unknots with net helicity \\
Neutron & Borromean Rings (3 unlinked loops) \\
Neutrino & Null Knot (zero net helicity, twist-symmetric) \\
Gluon & Interlinked Color Vortices (e.g. Hopf or torus knots) \\
W/Z Bosons & Massive, twisted braids with symmetry breaking \\
Higgs Boson & Scalar Vortex Condensate (swirl density mode) \\
\hline
\end{tabular}
\caption{Mapping of SM particles to knotted or linked vortex configurations in VAM.}
\end{table}

\subsection{Lagrangian Structure in VAM}

Let the vortex field be described by a velocity potential \( \vec{V} \) and vorticity \( \vec{\omega} = \nabla \times \vec{V} \). The general form of the Lagrangian density is:

\begin{equation}
\mathcal{L}_{\text{VAM}} = \frac{1}{2} \rho_\text{\ae} \left( \vec{V} \cdot \vec{V} \right)
- \frac{\lambda}{2} \left( \nabla \cdot \vec{V} \right)^2
- \kappa \left| \nabla \times \vec{V} \right|^2
+ \eta \, \vec{V} \cdot (\nabla \times \vec{V}) + \mathcal{L}_{\text{top}}
\end{equation}

Where:
\begin{itemize}
    \item \( \rho_\text{\ae} \): æther density (vacuum or core),
    \item \( \lambda \): compressibility penalty (for enforcing incompressibility),
    \item \( \kappa \): vorticity stiffness,
    \item \( \eta \): helicity coupling (captures chirality and time asymmetry),
    \item \( \mathcal{L}_{\text{top}} \): topological knot energy and linking terms (e.g., Hopf invariant).
\end{itemize}

\subsection{Topological Invariants as Charges}

The VAM Lagrangian encodes known quantum numbers via topological invariants:

\begin{itemize}
    \item \textbf{Electric charge} \( q \): proportional to total helicity \( H = \int \vec{V} \cdot \vec{\omega} \, dV \),
    \item \textbf{Spin} \( s \): determined by knot class (e.g. trefoil = spin-\( \frac{1}{2} \)),
    \item \textbf{Mass} \( m \): stored energy of the vortex (Bernoulli + swirl),
    \item \textbf{Color} (QCD): encoded via triple- or multi-linkings in vortex bundles,
    \item \textbf{Weak isospin/parity}: emergent from chirality of braid crossings or swirl polarity.
\end{itemize}

\subsection{Implications for Symmetry Breaking}

Massive bosons (W, Z, Higgs) emerge from bifurcations in the vortex lattice — topological transitions from symmetric swirl networks to chiral braid structures. Higgs excitation corresponds to a fluctuation in swirl amplitude:

\begin{equation}
H(x) \sim \delta \rho_\text{\ae}(x)
\end{equation}

\subsection{Conclusion}

The vortex æther reinterpretation of the Standard Model:
\begin{itemize}
    \item Assigns each particle to a stable or metastable topological vortex,
    \item Recovers charge, spin, mass from fluid and knot properties,
    \item Provides a Lagrangian formalism without requiring quantization of spacetime.
\end{itemize}

This closes the first benchmark cycle, establishing VAM as a physically grounded, mathematically consistent reformulation of known field theory.

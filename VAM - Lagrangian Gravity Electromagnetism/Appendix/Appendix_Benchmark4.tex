\section{Benchmark 4: Trefoil Knot as a Spin-\texorpdfstring{$\tfrac{1}{2}$}{1/2} Particle}

In the Vortex Æther Model, fundamental fermions such as the electron are modeled as stable, knotted vortex structures. The simplest nontrivial knot, the trefoil \( T_{2,3} \), satisfies all topological criteria to represent a chiral, spin-\(\tfrac{1}{2}\) excitation in a 3D incompressible superfluid æther.

\subsection{Parametric Structure of the Trefoil}

The trefoil is a \((p, q) = (2, 3)\) torus knot: it winds around the toroidal axis 2 times and the poloidal axis 3 times before closing. It is the simplest nontrivial knot with finite helicity, chirality, and linking number.

The parametric equations for the trefoil vortex knot are:

\begin{equation}
\begin{aligned}
x(t) &= \left(R + r \cos(3t)\right) \cos(2t) \\
y(t) &= \left(R + r \cos(3t)\right) \sin(2t) \\
z(t) &= r \sin(3t)
\end{aligned}
\end{equation}

Here, \(R\) is the major (toroidal) radius and \(r\) the minor (poloidal) radius.

\begin{figure}[H]
    \centering
    \includegraphics[width=0.6\textwidth]{trefoil_knot_electron.png}
    \caption{Trefoil knot \( T_{2,3} \) used to model spin-\(\tfrac{1}{2}\) fermions in VAM. The knot loops around the torus axis twice and twists three times, matching the chiral structure of the electron. Under a \(2\pi\) rotation, the knot returns to a geometrically distinct configuration, completing a full cycle only after \(4\pi\) — mimicking spinor behavior.}
\end{figure}

\subsection{Spinor Behavior from Knot Topology}

Spin-\(\tfrac{1}{2}\) behavior arises naturally from the topological structure of the trefoil:

\begin{itemize}
    \item A \(2\pi\) rotation does not return the knot to its original state — it becomes a distinguishable configuration.
    \item A full \(4\pi\) rotation is required for the knot to return to its original topological phase.
    \item This behavior mirrors that of spinors in quantum mechanics and matches the transformation properties of the electron under rotation in SU(2).
\end{itemize}

\subsection{Charge and Chirality}

The helicity of the trefoil is nonzero and signed. In VAM, this topological chirality directly encodes electric charge:

\begin{align*}
\text{Right-handed trefoil} &\rightarrow e^- \quad (\text{electron}) \\
\text{Left-handed trefoil} &\rightarrow e^+ \quad (\text{positron})
\end{align*}

Thus, fermionic matter and antimatter are modeled as mirror images of topologically stable knots in the æther.


\section{Benchmark 5: Proton as a Composite Vortex Structure}

In the Vortex Æther Model, baryons are modeled as stable, confined, topologically nontrivial vortex configurations. The proton, composed of two up-quarks and one down-quark in the Standard Model, is interpreted here as either:

\begin{enumerate}
    \item a single high-order torus knot \( T(p, q) \), or
    \item a topologically linked system of three unknotted vortex rings.
\end{enumerate}

\subsection{Model 1: High-Order Toroidal Knot \( T(6N, 9N) \)}

We define the generalized proton vortex as:

\begin{equation}
T(6N, 9N) \quad \text{for some integer } N \in \mathbb{Z}^+
\end{equation}

This knot completes \(6N\) toroidal and \(9N\) poloidal windings before closing. For \(N = 1\), the knot already wraps multiple times, exhibiting intrinsic chirality and long-range topological confinement.

\begin{itemize}
    \item The spinor nature (requiring \(4\pi\) rotation for identity) is preserved due to the trefoil substructure.
    \item The electric charge arises from the net chirality of the winding configuration.
    \item The internal periodicity of \(3N\) suggests a substructure corresponding to three bound constituents (quarks).
\end{itemize}

\subsection{Model 2: Three Interlinked Unknots}

Alternatively, the proton may be modeled as three individual unknotted vortex rings, interlinked in 3D space to form a chiral, stable composite structure:

\begin{figure}[H]
    \centering
    \includegraphics[width=0.65\textwidth]{proton_three_linked_trefoils.png}
    \caption{Proton modeled as three interlinked vortex rings (each a trefoil knot or simple loop), representing quark-like excitations. Their topological linkage ensures stability, and the net chirality gives rise to electric charge \(+e\).}
\end{figure}

\noindent
Each ring represents a constituent:
\[
\text{Up-quark: } \text{Right-handed vortex (chirality } +1), \qquad
\text{Down-quark: } \text{Left-handed vortex (chirality } -1)
\]

\begin{itemize}
    \item The net helicity is:
    \[
    H = +1 + +1 + (-1) = +1
    \]
    yielding a total electric charge \(q = +e\).
    \item The linkage prevents separation, modeling the observed color confinement in QCD.
    \item The topological interaction space allows internal circulation and swirl transfer, mimicking gluon-mediated exchange.
\end{itemize}

\subsection{Conclusion}

Both models yield spin-\(\tfrac{1}{2}\), electrically charged, and topologically stable configurations. Their agreement with qualitative features of baryonic matter — confinement, mass quantization, and chirality — confirms their candidacy within the VAM framework.


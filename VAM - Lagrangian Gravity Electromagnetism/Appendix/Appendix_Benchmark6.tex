\section{Benchmark 6: Neutron as a Borromean Vortex Configuration}

In the Vortex Æther Model, the neutron is modeled as a topologically bound, yet electrically neutral, three-vortex system known as a Borromean configuration. This arrangement consists of three unknotted vortex rings that:

\begin{itemize}
    \item Are \textbf{not pairwise linked} (any two can be separated without breaking),
    \item But are \textbf{globally inseparable} (all three must remain linked for stability),
    \item Form a stable, confined, zero-chirality configuration.
\end{itemize}

\begin{figure}[H]
    \centering
    \includegraphics[width=0.6\textwidth]{borromean_neutron_model.png}
    \caption{Neutron modeled as three unknotted vortex rings forming a Borromean configuration. No two rings are linked, but the full system is topologically bound. This models charge neutrality, internal swirl exchange, and instability under link-breaking (beta decay).}
\end{figure}

\subsection{Topological Interpretation}

Each ring in the Borromean system represents a neutral vortex excitation (analogous to a quark–antiquark pairing or internal mode). Their chirality is arranged such that:

\[
\chi_1 + \chi_2 + \chi_3 = 0
\]

This enforces **net helicity cancellation** and thus a **total electric charge of zero**. The configuration has:

\begin{itemize}
    \item **No net circulation** around the center,
    \item **Internal energy** stored in the rotational interactions of the rings,
    \item A fragile but topologically **nontrivial binding**.
\end{itemize}

\subsection{Neutron Decay Mechanism (Beta Decay)}

In the VAM framework, neutron decay can be interpreted as the **topological breakdown** of the Borromean linkage:

\[
\text{Neutron (3-ring Borromean)} \rightarrow \text{Proton (3-linked)} + \text{Electron (trefoil)} + \text{Antineutrino (null knot)}
\]

This corresponds to:

\begin{itemize}
    \item One of the rings separating, releasing a chiral knot (electron),
    \item The remaining two re-linking into a chiral 3-link (proton),
    \item Emission of a **null-knot** structure (antineutrino) preserving angular momentum and energy balance.
\end{itemize}

\subsection{Conclusion}

The Borromean ring structure captures key features of the neutron:
\begin{itemize}
    \item \textbf{Electric neutrality} via helicity cancellation,
    \item \textbf{Topological stability} without pairwise binding,
    \item \textbf{Metastability} with natural decay path into known leptons and baryons.
\end{itemize}

This confirms its role as a valid composite excitation within the Vortex Æther Model framework.

\subsection{Benchmark Summary Table}

\begin{table}[H]
\centering
\caption{Benchmark 3: Integrated Vortex Quantities for Photon Ring}
\begin{tabular}{|c|c|c|}
\hline
\textbf{Quantity} & \textbf{Value} & \textbf{Units} \\
\hline
Circulation $\Gamma$ & $-6.80 \times 10^{-18}$ & m$^2$/s \\
Swirl Energy $U_{\text{vortex}}$ & $3.61 \times 10^{-7}$ & J (2D slice) \\
Helicity $H$ & $2.84 \times 10^{-15}$ & m$^4$/s$^2$ (2D slice) \\
\hline
\end{tabular}
\end{table}

\begin{figure}[H]
    \centering
    \includegraphics[width=0.6\textwidth]{helicity_density_integrated.png}
    \caption{Helicity density field $h = \vec{v} \cdot \vec{\omega}$ with total integrated value over the $x$–$z$ plane: $H \approx 2.84 \times 10^{-15}$. High helicity confirms the presence of chirality essential for photon-like behavior in VAM.}
\end{figure}

\subsection{Conclusion}

This benchmark confirms that a toroidal vortex ring in an incompressible æther carries quantized:

\begin{itemize}
    \item \textbf{Circulation} $\Gamma$ (linked to spin or polarization)
    \item \textbf{Swirl energy} $U_{\text{vortex}}$ (linked to inertial mass)
    \item \textbf{Helicity} $H$ (linked to electric charge or chirality)
\end{itemize}

These quantities make the vortex ring a compelling candidate for modeling the photon or other bosonic excitations in the Vortex Æther Model.


\section{Photon as a Dipole Vortex Ring in the Æther}

\subsection{Topological Structure and Self-Propulsion}

In the Vortex Æther Model (VAM), we propose that the photon is not a point particle nor a plane wave, but a compact, propagating \textit{dipole vortex ring} embedded in an incompressible, inviscid æther. This structure consists of a toroidal vortex whose poloidal cross-section contains a source-sink dipole configuration, as illustrated in Fig.~\ref{fig:photon_toroid}.

The internal vorticity $\vec{\omega} = \nabla \times \vec{v}$ is arranged so that:

\begin{itemize}
    \item One side of the torus acts as a \textbf{source} (expelling æther),
    \item The opposite side acts as a \textbf{sink} (drawing in æther),
    \item The resulting Bernoulli pressure asymmetry induces a net translational velocity along the torus axis.
\end{itemize}

This aligns with Helmholtz's theorem on the self-advection of vortex structures in ideal fluids. The pressure gradient created by the dipole configuration generates a net force:

\begin{equation}
    \vec{F}_\text{net} = -\nabla P_{\text{dipole}}, \qquad \vec{v}_{\text{photon}} = \frac{P_{\text{swirl}}}{\rho_\text{\ae}} \equiv c
\end{equation}

\noindent
where $P_{\text{swirl}}$ is the swirl-induced pressure and $\rho_\text{\ae}$ is the æther density.

\subsection{Field-Theoretic Correspondence to Electromagnetism}

The vortex ring’s internal swirl field gives rise to a pair of orthogonal transverse fields analogous to the electric and magnetic fields:

\begin{align}
    \vec{E}_\text{æ} &\sim \nabla P_{\text{swirl}} \quad \text{(radial tension)} \\
    \vec{B}_\text{æ} &\sim \vec{\omega} \quad \text{(azimuthal vorticity)}
\end{align}

\noindent
These rotate synchronously as the torus propagates, producing a transverse, oscillating field consistent with classical electromagnetic waves. The Poynting vector emerges as:

\begin{equation}
    \vec{S}_\text{æ} \sim \vec{E}_\text{æ} \times \vec{B}_\text{æ} \sim \text{forward propagation direction}
\end{equation}

\subsection{Spin and Polarization}

The photon’s spin arises from the toroidal chirality of the vortex ring:

\begin{itemize}
    \item A right-handed swirl pattern yields \textbf{right-circular polarization} ($S_z = +1$),
    \item A left-handed swirl yields \textbf{left-circular polarization} ($S_z = -1$),
    \item Linear polarization results from a superposition of the two.
\end{itemize}

The photon's spin-1 nature is topological: the toroidal configuration allows two discrete circulation helicities but forbids $S_z = 0$ due to the conservation of angular momentum and incompressibility of the swirlcore.

\subsection{Summary}

\begin{table}[H]
\centering
\renewcommand{\arraystretch}{1.3}
\begin{tabular}{ll}
\toprule
\textbf{VAM Quantity} & \textbf{Electromagnetic Interpretation} \\
\midrule
Toroidal dipole ring     & Photon soliton \\
Pressure gradient        & Electric field ($\vec{E}$) \\
Swirl (vorticity)        & Magnetic field ($\vec{B}$) \\
Swirl energy             & EM energy density ($|\vec{E}|^2 + |\vec{B}|^2$) \\
Helicity sign            & Photon polarization / spin \\
Constant propagation     & $c = \sqrt{P/\rho_\text{\ae}}$ \\
\bottomrule
\end{tabular}
\caption{Correspondence between vortex ring dynamics and electromagnetic field quantities in VAM.}
\end{table}

\vspace{1em}
\noindent
Thus, the photon in VAM is a topological, massless, self-propagating vortex configuration whose net motion emerges from internal swirlclock asymmetry, source-sink pressure gradients, and conserved circulation. This fluid-mechanical interpretation restores physicality to electromagnetic wave propagation and naturally embeds polarization, quantized spin, and constant velocity into the geometric language of knots and vorticity.


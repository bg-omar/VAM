%! Author = mr
%! Date = 6/11/2025

\section{Swirlclock-Induced Time Asymmetry in Chiral Vortex Knots}

In the Vortex \ae ther Model (VAM), local time is governed not by a universal spacetime curvature, but by the intrinsic rotational energy of topological vortex structures embedded in an incompressible, inviscid \ae ther. The local clock rate, or \emph{swirlclock}, is determined by the energy density stored in the vorticity field:

\begin{equation}
dt_{\text{local}} = dt_{\infty} \sqrt{1 - \frac{U_{\text{vortex}}}{U_{\text{max}}}}, \quad U_{\text{vortex}} = \frac{1}{2} \rho_\text{\ae}^{(\text{energy})} |\vec{\omega}|^2,
\end{equation}
\noindent
Here, \( \rho_\text{\ae}^{(\text{energy})} \)\footnote{$\rho_\text{\ae}^{(\text{energy})} = 3.89 \times 10^{35} \, \text{J/m}^3$
is defined from the maximum Bernoulli swirl energy.} denotes the vortex core energy density, which governs time dilation in VAM. This is distinct from the ambient æther fluid density used in Bernoulli or inertial contexts. In this formulation, the swirlclock precession angle for a knotted vortex is:

\begin{equation}
\theta(t) = \Omega_{\text{swirl}} \cdot t_{\text{local}}, \quad \Omega_{\text{swirl}} = \frac{C_e}{r_c} e^{-r/r_c}.
\end{equation}

\subsection{Time Reversal and Chirality}

For a chiral knot such as the right-handed trefoil (KnotPlot ID \texttt{3.1.1}), the swirlclock progresses in one direction, say $\theta(t) > 0$, while its mirror image (left-handed trefoil) progresses with $\theta(t) < 0$. Applying time reversal symmetry $T$ transforms:

\begin{equation}
T: \quad \theta(t) \rightarrow -\theta(-t),
\end{equation}

but since the trefoil is topologically distinct from its mirror, the time-reversed knot is not smoothly deformable into the original. This breaks $T$ symmetry at the topological level, providing a physical and geometric mechanism for time-reversal asymmetry.

\subsection{Kaon Oscillations as Swirlclock Phase Shifts}

The neutral kaon system ($K^0 = d\bar{s}$ and $\bar{K}^0 = \bar{d}s$) exhibits experimentally verified time asymmetry in its oscillations \cite{christenson1964evidence,cplear1998tviolation}. In the VAM framework, these states are modeled as oppositely chiral vortex knots with swirlclock phases $\theta(t)$ and $-\theta(t)$ respectively. The time-reversal asymmetry is quantified by the phase lag:

\begin{equation}
\Delta \theta = \theta_K(t) - \theta_{\bar{K}}(-t),
\end{equation}

leading to an asymmetry parameter:

\begin{equation}
\delta_T = \frac{|A_{\rightarrow}|^2 - |A_{\leftarrow}|^2}{|A_{\rightarrow}|^2 + |A_{\leftarrow}|^2} \approx \frac{d}{dt} (\Delta \theta) \Big/ \Omega_{\text{swirl}},
\end{equation}

where $A_{\rightarrow}$ and $A_{\leftarrow}$ are forward and reverse transition amplitudes between chiral vortex states. This formulation predicts that $T$-violation is a natural outcome of vortex topology and swirl dynamics, not an arbitrary phase in the Lagrangian.

\subsection{Implications for Matter-Antimatter Asymmetry}

Since the VAM treats time as a locally emergent property from rotating field energy, intrinsic chiral bias in knot formation during early universe dynamics could naturally lead to an excess of one chirality—thereby favoring matter over antimatter. This offers a geometric mechanism for baryogenesis consistent with observed CP and $T$ violations.


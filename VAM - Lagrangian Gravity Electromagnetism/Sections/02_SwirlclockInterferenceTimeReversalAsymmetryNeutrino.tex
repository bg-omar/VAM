\section{Swirlclock Phase Interference as the Origin of Neutrino Oscillations and T-Violation}

\subsection{Introduction}

In the Standard Model (SM), neutrino oscillations arise from a mismatch between flavor and mass eigenstates, mediated by the complex-valued PMNS matrix \cite{pontecorvo1957inverse, mns1962neutrino}. Time-reversal (T) asymmetry and CP violation are embedded via a complex phase \( \delta_{CP} \). In contrast, the Vortex \AE ther Model (VAM) proposes a classical foundation: particle states are stable or metastable topological vortex knots in a Euclidean \ae ther. Local time is not fundamental but emergent, governed by vortex energy density via the swirlclock relation \cite{iskandarani2025vam2}.

\subsection{Swirlclock Dynamics in Chiral Vortices}

In VAM, each vortex knot carries swirl energy:
\begin{equation}
U_\text{vortex} = \frac{1}{2} \rho_\text{\ae}^{(\text{energy})} |\vec{\omega}|^2
\end{equation}
which controls the local flow of time:
\begin{equation}
dt = dt_\infty \sqrt{1 - \frac{U_\text{vortex}}{U_\text{max}}}
\end{equation}

The swirlclock phase for a given vortex is:
\begin{equation}
\theta_i(t) = \int_0^t \Omega_i \, dt_i = \Omega_i t_\infty \sqrt{1 - \frac{U_i}{U_\text{max}}}
\end{equation}
where \( \Omega_i = \frac{C_e}{r_c} e^{-r_i/r_c} \) is the effective angular velocity of mass eigenstate \( i \), and \( r_c \) is the vortex core radius. The vortex energy density is further defined as:
\begin{equation}
U_i = \frac{1}{2} \rho_\text{\ae}^{(\text{energy})} \left( \frac{\Gamma_i}{\pi r_c^2} \right)^2
\end{equation}
with \( \Gamma_i \) denoting the circulation of the \( i \)-th vortex knot.

For neutrinos, we postulate:
\begin{itemize}
\item \( \nu_1 \): slightly right-precessing amphichiral knot,
\item \( \nu_2 \): left-precessing knot (swirlclock lag),
\item \( \nu_3 \): high-twist chiral knot (lowest clock rate).
\end{itemize}

\subsection{Swirlclock-Based Neutrino Oscillations}

We define the flavor state as a superposition of mass eigenstates:
\begin{equation}
|\nu_\alpha(t)\rangle = \sum_i U_{\alpha i}^* e^{-i\theta_i(t)} |\nu_i\rangle
\end{equation}

The oscillation probability from flavor \( \alpha \) to \( \beta \) becomes:
\begin{equation}
P(\nu_\alpha \rightarrow \nu_\beta) = \left| \sum_i U_{\alpha i}^* U_{\beta i} e^{-i\theta_i(t)} \right|^2
\end{equation}

Time-reversal asymmetry emerges from the interference of swirlclock phases:
\begin{equation}
A_T(\alpha, \beta) = P(\nu_\alpha \rightarrow \nu_\beta) - P(\nu_\beta \rightarrow \nu_\alpha) \approx 4 \sum_{i<j} \Im(U_{\alpha i}^* U_{\beta i} U_{\alpha j} U_{\beta j}^*) \sin(\Delta \theta_{ij})
\end{equation}
where
\begin{equation}
\Delta \theta_{ij}(t) = \theta_i(t) - \theta_j(t)
\end{equation}

\subsection{Derivation of the Swirlclock Phase Lag}

We begin by expressing the angular velocity of each vortex knot state \( i \) as:
\begin{equation}
\Omega_i = \frac{C_e}{r_c} e^{-r_i / r_c}
\end{equation}

Next, the vorticity energy is given by:
\begin{equation}
U_i = \frac{1}{2} \rho_\text{\ae}^{(\text{energy})} \left( \frac{\Gamma_i}{\pi r_c^2} \right)^2
\end{equation}
using the identity \( |\vec{\omega}| = \Gamma_i / (\pi r_c^2) \), where \( \Gamma_i \) is the circulation.

Substituting \( \Omega_i \) and \( U_i \) into the phase integral gives:
\begin{equation}
\theta_i(t) = \Omega_i t_\infty \sqrt{1 - \frac{U_i}{U_\text{max}}} = \frac{C_e t_\infty}{r_c} e^{-r_i / r_c} \sqrt{1 - \frac{\rho_\text{\ae}^{(\text{energy})} \Gamma_i^2}{2 \pi^2 r_c^4 U_\text{max}}}
\end{equation}

Therefore, the swirlclock phase difference between eigenstates \( i \) and \( j \) becomes:
\begin{equation}
\Delta \theta_{ij}(t) = \frac{C_e t_\infty}{r_c} \left[
e^{-r_i/r_c} \sqrt{1 - \frac{\rho_\text{\ae}^{(\text{energy})} \Gamma_i^2}{2 \pi^2 r_c^4 U_\text{max}}} -
e^{-r_j/r_c} \sqrt{1 - \frac{\rho_\text{\ae}^{(\text{energy})} \Gamma_j^2}{2 \pi^2 r_c^4 U_\text{max}}}
\right]
\end{equation}

This replaces the arbitrary complex phase \( \delta_{CP} \) with a physically measurable quantity tied to geometric structure.

\subsection{Geometric Origin of T-Violation}

The final expression for \( \Delta \theta_{ij} \) shows that T-asymmetry emerges naturally from differences in circulation, spatial decay length, and precession rates of topological knots. This formulation provides a clear link between observable oscillation asymmetry and underlying geometric swirl structure, bypassing the need for quantum mechanical CP violation parameters.

\subsection{Summary}

Neutrino oscillations and T-violation in the Standard Model can be reinterpreted in the Vortex \AE ther Model as interference of swirlclock phases associated with chiral vortex knots. The geometric origin of time-asymmetry connects directly to vortex energy and helicity, offering a classical, topological alternative to quantum field-theoretic CP phases. Future work includes mapping swirlclock interference against experimental parameters and generalizing this framework to baryogenesis and cosmological T-asymmetry.

%! Author = omar.iskandarani
%! Date = 5/19/2025

\documentclass[a4paper, aps,preprint,superscriptaddress, 12pt]{revtex4}
\usepackage[paperwidth=210mm, paperheight=297mm, margin=2.5cm]{geometry}
\usepackage{float}
\usepackage{tikz}
\usepackage{makecell}
\usepackage[font=footnotesize]{caption}
\usetikzlibrary{arrows.meta}
\usepackage{pgfplots}
\pgfplotsset{compat=1.18}
\usepackage[none]{hyphenat}
\usepackage{array}
\usepackage{amsmath}
\usepackage{booktabs}
\usepackage[utf8]{inputenc} % Remove if using XeLaTeX or LuaLaTeX
\usepackage{amssymb}
\usepackage{graphicx}
\usepackage{hyperref}
\usepackage{physics}
\usepackage{natbib}
\usepackage{url}
\renewcommand{\arraystretch}{1.5}
\renewcommand{\floatpagefraction}{.8}
\sloppy

\begin{document}
    \author{Omar Iskandarani}
    \title{Torusknopen als Model voor Elementen in het Vortex-Æther Model}

    \date{\today}
    \affiliation{Onafhankelijk onderzoeker, Groningen, The Nederland}
    \thanks{ORCID: \href{https://orcid.org/0009-0006-1686-3961}{0009-0006-1686-3961}}
    \email{info@omariskandarani.com}

    \begin{abstract}
        In dit artikel wordt een alternatieve representatie van chemische elementen voorgesteld binnen het kader van het Vortex-Æther Model (VAM), waarin materie niet bestaat uit puntdeeltjes of quarks, maar uit stabiele, topologisch verankerde vortexstructuren in een continuüm van superfluïde æther. Elk element wordt gemodelleerd als een torusknoop $T(p,q)$, waarvan de heliciteit direct gerelateerd is aan massa (via vortexenergie) en lading (via chirale oriëntatie).

        Waterstof wordt bijvoorbeeld geassocieerd met de trefoilknoop $T(2,3)$, terwijl zwaardere elementen corresponderen met knopen van toenemende complexiteit of samengestelde knoopconfiguraties. Vanuit deze structuur ontstaat een natuurlijke periodiciteit, analoog aan elektronenschillen, maar gebaseerd op topologische capaciteit. Splitsing en fusieprocessen van kernen worden binnen VAM geïnterpreteerd als knoopevoluties of herverbindingsprocessen.

        Het model levert verklaringen voor ladingkwantisatie, isotopische stabiliteit, bindingsenergie en het bestaan van limieten aan kernmassa, en biedt tevens nieuwe invalshoeken voor fenomenen als quarkconfinement, spin, en massa-hiërarchie. De resultaten tonen dat de topologie van vortexknopen potentieel een verenigend raamwerk biedt waarin klassieke en kwantummechanische eigenschappen van elementen emergent zijn uit fluïdumdynamica.
    \end{abstract}
    
    \maketitle

    %! Author = Omar Iskandarani
%! Date = 5/19/2025

\section{Inleiding en VAM-grondslagen}

Het Vortex Æther Model (VAM) beschouwt materie als stabiele vortexknopen (wervelknopen) in een alomtegenwoordig superfluïd æther~\cite{Kelvin1867VortexAtoms}. Belangrijke parameters in VAM zijn de tangentiële kernrotatiesnelheid $C_e$, de vortexkernstraal $r_c$ en de
ætherdichtheid $\rho_\text{\ae}$, die op kernschaal vaste waarden aannemen (zie Tabel 1). Deze bepalen de kwantisering van circulatie en de maximale vortexkracht in het æthermedium. Een opmerkelijk uitgangspunt is dat vorticiteit (circulatie) behouden en gekwantiseerd is – analoog aan fluxkwantisatie – waardoor elke vortexknoop topologisch invariant (knopen kunnen niet continu ontwarren) en daarmee stabiel is. Dit biedt een mechanisme om massa en lading emergent te verklaren uit fluïdumwetten in plaats van via elementaire puntdeeltjes.

In VAM dragen vortexknopen heliciteit – de topologische koppeling van wervellijnen – die fungeert als interne vrijheidsgraad (vergelijkbaar met
spin) en ook geassocieerd wordt met lading. Heliciteit $H$ is gedefinieerd als de volumelijke integraal van de snelheid $v$ en vorticiteit $\omega$~\cite{Moffatt1990VortexHelicity, Ricca1992EnergyHelicity}:


\begin{equation}
    H = \int \mathbf{v} \cdot \boldsymbol{\omega}\,dV,
\end{equation}

een behouden grootheid die de linking/winding van wervelstructuren telbaar maakt. Intuïtief telt $H$ het (gekwantiseerde) aantal omwentelingen waarmee wervel-lijnen elkaar omcirkelen. Deze topologische invariantie ($H$ constant) garandeert dat een gesloten vortexknoop een minimale energieconfiguratie niet spontaan kan verliezen zonder externe inwerking.

Tabel 1 hieronder resumeert enkele fundamentele VAM-constanten voor het kernschaal-regime:

\begin{table}[h!]
    \centering
    \begin{tabular}{lll}
        \hline
        Symbool & Grootheid & Waarde (kernscala)\\
        \hline
        $C_e$ & Tangentiële kernrotatiesnelheid & $1.094\times10^6~\text{m/s}$\\
        $r_c$ & Vortexkernstraal (Coulomb-barrière straal) & $1.409\times10^{-15}~\text{m}$\\
        $\rho_\text{\ae}$ & Æther-dichtheid (lokaal) & $3.893\times10^{18}~\text{kg/m}^3$\\
        $F_{\max}$ & Maximale vortexkracht & $\approx 29~\text{N}$\\
        $\kappa$ & Circulatiekwantum ($C_e r_c$) & $1.54\times10^{-9}~\text{m}^2/\text{s}$\\
        $\alpha$ & Fijnstructuurconstante $(2C_e/c)$ & $7.297\times10^{-3}$\\
        \hline
    \end{tabular}
    \caption{Kernparameters in het Vortex Æther Model. Deze karakteristieke constanten bepalen de schaal van vortexkernen en interacties binnen nucleonen.}
\end{table}
    \section{Elementen als Torusknopen: Toewijzing van $T(p,q)$}

Onderstaande tabel koppelt de beschouwde elementen aan kandidaat-torusknopen $T(p,q)$. We vermelden per knoop het geschatte linking number $L_k$ (aantal zelf-omstrengelingen van de vortexlijnen) en bespreken isotopen waar relevant. Deze toewijzing is gebaseerd op minimale knoopcomplexiteit die de lading ($Z$) en massa van het element kan dragen binnen VAM~\cite{Kleckner2013KnotsVortex}.
% --- Figuur 1: 3D Torus Knoop ---
\begin{figure}[H]
    \centering
    \includegraphics[width=0.65\textwidth]{../1_vortexknopen}
    \caption{De torusknoop $T(2,3)$ als basismodel voor waterstof in het Vortex Æther Model. De tangentiële snelheid $C_e$ en kernstraal $r_c$ worden visueel aangeduid.}
    \label{fig:torus_t23}
\end{figure}

\begin{table}[h!]
    \centering
    \begin{tabular}{lll}
        \hline
        Element (Z) & Voorgestelde torusknoop $T(p,q)$ & $L_k$ (helicititeit) en opmerkingen (isotopen)\\
        \hline
        Waterstof (1) & $T(2,3)$ (trefoilknoop)~\cite{Faddeev1997KnottedSolitions} & $3$, $^1$H basis; $^2$H extra neutron-binding nodig\\
        Helium (2) & $T(2,5)$ & $5$, $^4$He stabiel; $^3$He neutronvariatie\\
        Lithium (3) & $T(2,7)$ & $7$, $^7$Li meest stabiel; $^6$Li minder stabiel\\
        Beryllium (4) & $T(2,9)$ & $9$, $^9$Be stabiel; $^8$Be onstabiel (valt uiteen)\\
        Boor (5) & $T(2,11)$ & $11$, $^{11}$B stabiel; $^{10}$B stabiel met neutronvariatie\\
        Koolstof (6) & $T(2,13)$ & $13$, $^{12}$C zeer stabiel; $^{13}$C stabiele isotopenvariant\\
        IJzer (26) & $T(4,3)$ (voorbeeld) & $8$, $^{56}$Fe hoogste bindingsenergie, optimale knoopconfiguratie\\
        Uranium (92) & Complex (samengesteld) & –, $^{238}$U zwaar, borderline stabiel; composiet van subknopen\\
        \hline
    \end{tabular}
    \caption{Voorgestelde torusknopen voor enkele elementen binnen het Vortex Æther Model.}\label{tab:table}
\end{table}


% --- Figuur 2: Elementen vs Knoopstructuren ---
\begin{figure}[H]
    \centering
    \includegraphics[width=0.7\textwidth]{../2_elementenTorusknopen}
    \caption{Aantal heliciteitskwanta $L_k$ per chemisch element. De torusknoopclassificatie geeft een topologisch alternatief voor het periodiek systeem.}
    \label{fig:elementen_torusknoop}
\end{figure}

\begin{figure}[H]
    \centering
    \includegraphics[width=0.8\textwidth]{../0_eersteKnopen}
    \caption[Torusknopen voor lichte elementen (bovenaanzicht)]{
        Bovenaanzicht van voorgestelde torusknopen \( T(p, q) \) binnen het Vortex Æther Model (VAM) voor de elementen:
        \textbf{Waterstof} \(T(2,3)\),
        \textbf{Helium} \(T(2,5)\),
        \textbf{Lithium} \(T(2,7)\),
        \textbf{Beryllium} \(T(2,9)\),
        \textbf{Boor} \(T(2,11)\),
        \textbf{Koolstof} \(T(2,13)\), en
        \textbf{IJzer} \(T(4,3)\).
        Elke knoop vertegenwoordigt een unieke heliciteit \( L_k \) die nodig is voor de interne wervelstabiliteit van het element binnen de Æther.
    }
    \label{fig:torusknopen_top}
\end{figure}


In waterstof (Z=1) stellen we de kern (proton) voor als de eenvoudigste niet-triviale vortexknoop: de trefoilknoop $T(2,3)$. De trefoil heeft topologisch $L_k=3$ omdat men het kan beschouwen als een gevlochten structuur van 2 strengen met 3 windingen. Binnen VAM correspondeert dit met de minimale heliciteit nodig om een ladingseenheid te dragen.

Isotopen zoals deuterium ($^2$H) zouden extra neutrale wervelstructuren bevatten, bijvoorbeeld als satellietknoop gekoppeld aan de protonknoop. Tritium ($^3$H) vereist analoog extra neutron-satellieten, wat overeenkomt met de instabiliteit in β-verval.

Helium (Z=2) wordt voorgesteld als $T(2,5)$ met $L_k \approx 5$. Voor helium geldt dat een hogere heliciteit noodzakelijk is voor voldoende centripetale druk tegen Coulomb-afstoting. Helium-4 heeft neutronen als interne extra wervelingen, terwijl helium-3 minder massa heeft door een afwijkende neutronconfiguratie.

Lithium (Z=3), voorgesteld als $T(2,7)$, volgt hetzelfde patroon. Lithium-7 is stabiel, lithium-6 iets minder door neutrontekort.

Beryllium (Z=4) als $T(2,9)$ is stabiel als $^9$Be, terwijl $^8$Be uiteenvalt in twee heliumkernen door onvoldoende massa-druk om de hoge heliciteit te stabiliseren.

Boor (Z=5) en koolstof (Z=6) vervolgen dit rijtje met respectievelijk $T(2,11)$ en $T(2,13)$, met diverse stabiele isotopen binnen dezelfde knoopfamilie. Koolstof-12 onderscheidt zich door bijzondere stabiliteit, mogelijk door symmetrische knoopconfiguratie.

\usepackage{tabularx} % In je preambule voor automatische kolombreedte

\begin{table}[H]
    \centering
    \caption[Elementen als torusknopen in het VAM]{Toewijzing van torusknopen \( T(p,q) \) aan lichte elementen binnen het Vortex Æther Model. De heliciteit \( L_k \) representeert het aantal interne vortex-omstrengelingen.}
    \label{tab:torusknopen}
    \begin{tabularx}
        \hline
        \( Z \) & Element & \( T(p,q) \) & Heliciteit \( L_k \) en opmerkingen \\
        \hline
        1 & Waterstof & \( T(2,3) \) & \( L_k = 3 \), trefoilknoop, basale protonstructuur \\
        2 & Helium & \( T(2,5) \) & \( L_k = 5 \), extra neutronbinding in \( ^4\text{He} \) \\
        3 & Lithium & \( T(2,7) \) & \( L_k = 7 \), \( ^7\text{Li} \) stabiel, neutrontekort in \( ^6\text{Li} \) \\
        4 & Beryllium & \( T(2,9) \) & \( L_k = 9 \), \( ^9\text{Be} \) stabiel, \( ^8\text{Be} \) valt uiteen \\
        5 & Boor & \( T(2,11) \) & \( L_k = 11 \), neutronvariant in \( ^{10}\text{B} \) \\
        6 & Koolstof & \( T(2,13) \) & \( L_k = 13 \), \( ^{12}\text{C} \) zeer stabiel door symmetrie \\
        26 & IJzer & \( T(4,3) \) & \( L_k = 8 \), hoogste bindingsenergie bij \( ^{56}\text{Fe} \) \\
        \hline
    \end{tabularx}
\end{table}

Zoals weergegeven in figuur~\ref{fig:torusknopen_top} en tabel~\ref{tab:torusknopen}, kent elk licht element binnen het VAM een specifieke torusknoopstructuur toe die de topologische basis vormt voor zijn massa, stabiliteit en isotopie.


IJzer (Z=26) markeert een overgang naar complexere knopen zoals $T(4,3)$, die een optimale configuratie biedt voor de hoogste bindingsenergie.

Uranium (Z=92) is complex en samengesteld, niet meer voorstelbaar als één torusknoop maar eerder als composiet van subknopen, consistent met het waargenomen fissiegedrag van zware elementen.

\begin{figure}[H]
    \centering
    \includegraphics[width=\textwidth]{images/0_UraniumKnoop}
    \caption[Samengestelde vortexstructuur van uranium-238 in VAM]{
        Visualisatie van de voorgestelde samengestelde vortexstructuur van uranium-238 binnen het Vortex Æther Model (VAM).
        \textbf{Bovenste rij}: individuele torusknopen in bovenaanzicht:
        de \textcolor{blue}{hoofdkern \(T(4,5)\)}, de \textcolor{green}{satellietknoop \(T(2,7)\)} en de \textcolor{orange}{satellietknoop \(T(2,9)\)}.
        \textbf{Onderste rij}: drie perspectieven op de volledige composietstructuur met duidelijke interactie tussen de knopen.
        Deze topologische decompositie verklaart de hoge heliciteit, bindingsenergie, en splijtbaarheid van zware kernen.
    }
    \label{fig:uranium_2x3}
\end{figure}

\begin{table}[H]
    \centering
    \caption[Samengestelde vortexstructuur van uranium-238]{Voorstel voor een samengestelde vortexconfiguratie voor uranium-238 in VAM, opgebouwd uit geneste torusknopen. De totale heliciteit wordt dan:
        $ L_k^\text{totaal} = L_k^{(4,5)} + L_k^{(2,7)} + L_k^{(2,9)} = 20 + 7 + 9 = 36 $\\
        Met aanvullende interactietermen zoals linking, nesting, of mutual writhing:
        $ L_k^\text{eff} = L_k^\text{totaal} + \Delta L_\text{interactie} $
    waar $\Delta L_\text{interactie} \approx 3\text{–}5$ afhankelijk van configuratie. Dit reproduceert de hoge bindingsenergie en instabiliteit (fissiegevoeligheid) van zware elementen.
    }
    \label{tab:uranium_knoopstructuur}
    \begin{tabularx}{\textwidth}
        \hline
        Substructuur & \( T(p,q) \) & \( L_k \) & Functie binnen kernstructuur \\
        \hline
        Hoofdkern & \( T(4,5) \) & 20 & Draagt hoofdheliciteit en basale bindingsenergie \\
        Satelliet 1 & \( T(2,7) \) & 7 & Neutroncluster ter massa-aanvulling \\
        Satelliet 2 & \( T(2,9) \) & 9 & Drukbalans en stabilisatie \\
        Interactiecomponent & – & \( \Delta L \approx 3{-}5 \) & Inter-knoop interacties (linking, nesting) \\
        \hline
        \textbf{Totaal} & – & \( \approx 39{-}41 \) & Geschatte effectieve heliciteit \\
        \hline \\
    \end{tabularx}
\end{table}
    \section{Massa en lading uit knoopstructuur}

In het Vortex Æther Model ontstaan massa en lading als emergente grootheden uit de energie en heliciteit van de vortexknoop. De massa $M$ van een
vortexknoop kan berekend worden door de kinetische energie van de wervelstroming te integreren over het volume~\cite{Moffatt1990VortexHelicity}:
\begin{equation}
    E = \frac{1}{2}\int \rho_\text{\ae}\, v^2 \,dV, \qquad M = \frac{E}{c^2}.
\end{equation}

Hoewel een exacte integratie voor complexe knopen lastig is, leidt VAM tot een benaderende formule die massa recht evenredig met de heliciteit maakt. Zo wordt voor elementaire deeltjes (fermion-knopen) afgeleid:
\begin{equation}
    M \approx 8\pi \rho_\text{\ae} r_c^3 C_e \cdot L_k,
\end{equation}


% --- Figuur 5: Energie vs Heliciteit ---
\begin{figure}[H]
    \centering
    \includegraphics[width=0.7\textwidth]{sections/5_Wervelveldenergie}
    \caption{Vortexenergie $E$ als functie van heliciteit $L_k$ onder verschillende benaderingen. Het kwadratische VAM-model wijkt af van lineaire klassieke benaderingen.}
    \label{fig:energie_vs_heliciteit}
\end{figure}

waar $L_k$ de topologische linking number (heliciteit quanta) van de knoop is. Deze formule toont dat elke winding/linking extra inertiële massa bijdraagt. Voor de trefoilknoop ($L_k=3$) krijgen we $M \approx 8\pi \rho_\text{\ae} r_c^3 C_e \cdot 3$. Met de waarden uit Tabel 1 levert dit numeriek een massa in de juiste orde van grootte voor nucleonen (zij het dat hier fijnafstemming nodig is om exact de protonmassa te krijgen – zie Discussie). Het belangrijke punt is dat massa topologisch verankerd is: hoe ingewikkelder de knoop (hoe hoger $L_k$), des te groter de totale vortexenergie die opgeslagen zit in het wervelveld.

% --- Figuur 6: Chiraliteit en lading ---
\begin{figure}[H]
    \centering
    \includegraphics[width=0.8\textwidth]{sections/6_swirlRichting}
    \caption{Linksdraaiende en rechtsdraaiende vortexknopen. Volgens VAM bepaalt de draairichting de waargenomen elektrische lading.}
    \label{fig:swirl_lading}
\end{figure}

Lading daarentegen wordt geassocieerd met de heliciteit oriëntatie. Een vortexknoop heeft een chirale eigenschap – een onderscheid tussen links- en rechtsdraaiende werveling. VAM identificeert dit met positieve vs. negatieve elektrische lading. Concreet: een knoop met een bepaalde heliciteitsrichting induceert een vectorpotentiaal in de æther die overeenkomt met een elektromagnetisch veld. Hierdoor gedraagt de wervelknoop zich als een geladen deeltje, hoewel er in het model geen fundamentele "ladingsdrager" is behalve de draaiende æther zelf.

VAM verklaart daarmee ook waarom lading gekwantiseerd is: heliciteit in een knoop kan niet continu variëren maar is gebonden aan integer linking (of een vaste verhouding daartussen). Bijvoorbeeld, de trefoil met $L_k=3$ genereert één elementaire ladingeenheid $e$. Dit suggereert dat de elementaire heliciteitseenheid overeenkomt met 1/3e – opvallend gelijk aan de fractiele ladingen van quarks. Inderdaad zou men kunnen zeggen dat de vortexheliciteit verdeeld is in 3 subfluxen (denk aan drie fluxringetjes ineen) die gezamenlijk de knoop vormen, analoog aan de drie quarks in een baryon.

Deze analogie is speculatief maar interessant: een proton-trefoil zou zo gezien kunnen worden als een gebonden toestand van drie vortexquanta (elk met heliciteit 1 en lading $+1/3e$) die topologisch niet afzonderlijk bestaan maar samen één knoop vormen. Dit komt overeen met het standaardmodel waarin drie quarks $(+2/3,+2/3,-1/3)$ de lading +1 van de proton geven – in VAM zijn dit drie onverbrekelijke wervelinkepingen die samen $L_k=3$ maken en zo één ladingeenheid opleveren.

Samengevat ontstaat massa uit de totale energie van het wervelveld, die door $L_k$ wordt gekwantiseerd, en lading uit de heliciteitsrichting van dat veld. De spin van het deeltje tenslotte is gerelateerd aan de interne rotatie van de knoop. Een vortexknoop roteert om zijn eigen kern (met hoeksnelheid $\Omega_k$), wat een angulair momentum geeft. In feite kan men aantonen dat een knoop met heliciteit $H$ een eigen draaimoment draagt proportioneel aan $H$ (via een analoog van de Jefimenko’s oplossingen in de æther). Dit is in lijn met het idee dat spin half (fermionen) corresponderen met vortexknopen die één quantum heliciteit uit de æther “zuigen” (bijv. trefoil $L_k=3$ zou effectief spin 1/2 kunnen geven in de juiste normalisatie), terwijl bosonen wellicht samenhangen met samenstellingen met integer netto heliciteit. Deze interpretatie vereist verdere uitwerking, maar VAM biedt in principe een mechanisme waarbij spin geen fundamenteel puntdeeltje-eigenschap is, maar voortkomt uit de draaiende topologie van de ætherknoop.
    \section{Stabiliteit van knoopstructuren}

Een centraal resultaat van VAM is dat de stabiliteit van een elementaire vortexknoop wordt gewaarborgd door topologische invariantie en drukbalans. Topologisch kan een knoop niet ontrafeld worden zonder dat er ergens een discontinuïteit (breuk in het wervelveld) optreedt, wat energetisch extreem onwaarschijnlijk is bij lage temperaturen (æther is inviscide). Dit verklaart waarom protonen, elektronen etc. stabiel zijn: zij zijn topologisch beschermd. Evenzo zijn bepaalde knopen méér beschermd dan andere – hoe complexer (hoger $L_k$) een knoop, hoe meer interne spanningsenergie erin zit, wat zowel stabiliserend (tegen kleine verstoringen) als destabiliserend (tegen splitsing in lagere knopen) kan werken, afhankelijk van de situatie.

Voor lichte elementen (H–C) zagen we dat één enkele torusknoop hun structuur kan verklaren. Hun stabiliteit binnen VAM komt voort uit het feit dat
de vorticiteit-geïnduceerde drukvelden precies in evenwicht zijn met de centrifugale uitwaartse krachten van de roterende æther. Bijvoorbeeld, in een trefoilknoop (H) zorgt de snelle kernrotatie ($C_e \sim10^6~\text{m/s}$) en hoge lokale ætherdichtheid ervoor dat in de kern een aanzienlijke onderdruk ontstaat. Deze onderdruk (via de Bernoulli-vergelijking~\cite{Ricca1992EnergyHelicity}) houdt de vortexbuis op een vaste straal $r_c$ en voorkomt dat de structuur uiteen spat ondanks de grote traagheidskracht van de cirkelende æthermassa. Formeel leidt VAM tot een Poisson-achtige vergelijking voor de vortex-Bernoulli potentiaal $\Phi_v$:
\begin{equation}
    \nabla^2 \Phi_v = -\frac{1}{2}\rho_\text{\ae} \| \vec{\omega}(r) \|^2,
\end{equation}
waaruit volgt dat waar de vorticiteit $|\omega|$ groot is (in de kern van de knoop), $\Phi_v$ laag is. $\Phi_v$ fungeert als een analoog van de zwaartekrachtspotentiaal binnen de knoop, d.w.z. er is een aantrekkende kracht (drukgradiënt) naar het centrum toe die de æther en daarmee de knoop bij elkaar houdt. Dit mechanisme is precies waarom vortexringen in een vloeistof coherent blijven: de druk in de kern is lager dan erbuiten, waardoor de ringvorm behouden blijft. In VAM wordt dit opgevoerd tot fundamenteel niveau voor deeltjes.

Daarnaast is er topologische stabiliteit: zolang de heliciteit $H$ niet verandert, kan de knoop niet overgaan in een andere vorm. Dit betekent dat lichte kernen niet spontaan naar andere knopen transformeren (geen verval) zolang er geen externe verstoring is die heliciteit kan herverdelen (denk aan sterke botsingen of quantumeffecten als $\beta$-verval die in VAM als resonante herknoping verklaard moeten worden). Het ontbreken van metastabiele lichte knopen komt overeen met het feit dat proton, elektron etc. stabiel zijn en geen “andere vorm” aannemen.
    \section{Fusie en Splijting van Vortexknopen binnen VAM}

Wanneer we naar zwaardere knopen gaan, komt er een competitie tussen één knoop houden of opsplitsen in meerdere knopen. Dit is analoog aan kernen: lichte kernen fuseren exotherm (één grotere knoop is energetisch gunstiger), middelzware zijn het stabielst in hun eentje, en zeer zware kernen fissioneren (twee middelgrote knopen energetisch gunstiger dan één reuzenknoop). VAM levert een topologische interpretatie:

\textbf{Fusie:} Twee losse vortexknopen (bijv. twee $T(2,3)$ knopen voor 2 H) kunnen bij dicht bijeen brengen een overlappende wervelstructuur gaan delen en samensmelten tot één knoop van hogere $L_k$ (He met $T(2,5)$). Dit wordt vergemakkelijkt als de vorticiteit-geïnduceerde druk tussen de knopen de Coulomb-afstoting compenseert. VAM berekent dat bij kernafstand $\sim 2r_c$ een significante drukval $\Delta P$ optreedt die de effectieve barrière verlaagt – een hydrodynamische kijk op quantum tunneling. Knoopfusie is dus energetisch voordelig zolang de resulterende knoop voldoende onderdruk creëert om stabiel te zijn. Dit stopt rond ijzer: daarna voegt extra $L_k$ zoveel rotatie-energie toe dat de onderdruk per nucleon minder wordt (de knoop wordt “te strak opgekruld” en verliest relatieve stabiliteit).

\textbf{Splijting (fissie):} Een zeer complexe knoop (hoog $L_k$, groot $p$) kan energetisch winnen door in twee of meer knopen met lagere $L_k$
te splitsen. Topologisch vereist dit dat ergens het vortexveld breekt en herverbindt (reconnection)~\cite{Kleckner2013KnotsVortex} – normaal verboden in een ideale vloeistof, maar in werkelijkheid mogelijk via quantumfluïdum-effecten of extreme excitatie. Uranium bijvoorbeeld kan via een zeldzame fluctuatieresonantie twee deelknooppatronen vormen (bijv. $T(3,q_1)$ en $T(3,q_2)$) die samen een lagere totale $H$ hebben dan de oorspronkelijke ($H$ blijft behouden maar verdeelt zich over fragmenten, vergelijkbaar met behoud van baryongetal bij splitsing). Omdat elke fragmentknoop een sterkere eigen binding heeft (minder spanning dan één superknoop), is dit energetisch gunstig en gebeurt het spontaan (radioactief verval). Binnen VAM zou men dit beschrijven als een knoop-naar-multiknoop resonantie: de zware knoop oscilleert naar een toestand waar een topologische “brug” vormt en splitst, analoog aan zeepbel-deelvorming of vortex ring splittings, zij het op quantum-schaal.

Kortom, de stabiliteit van elk element binnen VAM kan kwalitatief verklaard worden: lichte en middellange elementen blijven één torusknoop dankzij topologische inertie en vorticiteitsdruk; hele zware elementen naderen de grens waarbij multi-knoop configuraties winnen. Deze omslag is continu en onder voorwaarden – bijvoorbeeld neutronrijke isotopen (veel satellietknopen) verkleinen de onderlinge koppeling van de hoofdknoop, wat eerder tot instabiliteit leidt. We zien dit doordat isotopen met te veel of te weinig neutronen sneller vervallen. VAM zou dit kwantitatief kunnen modelleren via de entropie van vortexknopen (in eerdere VAM-werk is aangetoond dat men met Clausius-entropie aan vortexknopen thermodynamische consistentie krijgt). Een stabiele knoop is dus ook een entropisch minimum voor het gegeven $H$: er is geen configuratie met dezelfde heliciteit die lagere energie heeft. Wanneer die er wel is (bij zware kernen: twee knopen met ieder lagere heliciteit), dan treedt verval op.
    
\section{Periodiciteit en knoopfamilies}

Uit het bovenstaande emergeert een interessante parallel met het periodiek systeem. In de atoomtheorie wordt periodiciteit gedreven door elektronenschillen; in VAM lijkt een analoog principe op te duiken: knoopfamilies met vast $p$ vertonen capaciteitsgrenzen, waarna $p$ toeneemt (extra streng) en een nieuwe familie start. Dit verklaart qualitatiever de groepen elementen:

\textbf{Eerste periode (H, He):} Hier heeft waterstof al een $p=2$ knoop (trefoil) nodig voor stabiliteit. Men zou kunnen zeggen dat $p=2$ overeenkomt met de eerste “schil” van vortexstructuren, die echter maar 2 elementen bevat. Dit is vergelijkbaar met de $1s$ orbitaal die 2 elektronen opneemt. In VAM-termen is wellicht $p=2$ de minimaal mogelijke knoopstreng, die hooguit $q=3$ (H) en $q=5$ (He) toestaat als stabiele vormen. He heeft daarmee die $p=2$ capaciteit uitgeput (verder verhogen van $q$ leidt tot instabiliteit, wat we inderdaad bij Be zagen).

\textbf{Tweede periode (Li–Ne):} Deze corresponderen waarschijnlijk met knopen uit de $p=3$ familie (driestrengs knopen). Een $p=3$ knoop kan meer heliciteit opslaan zonder instabiliteit dan een $p=2$. Mogelijk begint Li (Z=3) al als $p=2$ ($T(2,7)$ in onze tabel), maar het zou ook kunnen dat vanaf Beryllium of Boor de omslag naar $p=3$ gebeurt. Indien Li t/m Ne door $p=3$ knopen worden gedragen, zou die familie tot 8 elementen kunnen omvatten, vergelijkbaar met de 2e elektronenschil. Het onderscheid tussen $p=2$ en $p=3$ knopen zou subtiel kunnen inzetten rond beryllium/boor, wat misschien het bestaan van de instabiele $^8$Be verklaart: $^8$Be zou precies op de overgang liggen – als twee gekoppelde $p=2$ knopen (twee alpha’s) stabieler zijn dan één slecht passende $p=3$ knoop, valt $^8$Be uiteen. Zodra we bij boor/koolstof zijn, domineert $p=3$ waarschijnlijk, en die reeks kan doorlopen t/m Ne.

\textbf{Derde en vierde periode:} Deze bevatten 8 resp. 18 elementen. In VAM zouden dit $p=4$ en $p=5$ knoopfamilies kunnen zijn. Een $p=4$ (vierstrengs) torusknoop heeft potentieel ruimte voor meer complexe linking (mogelijk tot 18 stabiele configuraties) voordat een $p=5$ nodig is. IJzer-groep (periode 4 midden) zou dus $p=4$ knopen zijn, wat hun hoge stabiliteit ondersteunt. Daarna, vanaf rubidium (Z=37), zou $p=5$ starten, samenhangend met de 18 elementen in periode 5.

% --- Figuur 4: Periodieke knoopfamilies ---
\begin{figure}[H]
    \centering
    \includegraphics[width=0.65\textwidth]{../4_KnoopFamilies}
    \caption{Aantal elementen per knoopfamilie $p$, overeenkomend met periodieke reeksen in het klassieke systeem.}
    \label{fig:knoopfamilies}
\end{figure}

\textbf{Vijfde en zesde periode:} Periode 6 heeft 32 elementen, wat in dit patroon $p=6$ familie zou zijn. Inderdaad worden de langste perioden gedragen door zeer complexe knopen (6- of 7-strengs). Deze kunnen zeer veel $q$ variatie aan tot ze vol raken. Uiteindelijk bij superzware elementen (boven uranium) zou zelfs $p=7$ niet meer afdoende zijn en treedt op wat we al beschreven: multi-knoop samenstellingen in plaats van één knoop.

Deze topologische periodiciteit is momenteel een kwalitatieve analogie, maar het is opmerkelijk hoe de getallen ruwweg overeenkomen met $2, 8, 18, 32$ elementen per knoopsoort. Het suggereert dat natuurwetten op fundamenteel niveau misschien een weerspiegeling zijn van topologische combinatoriek: net zoals elektronen permutaties van quantumtoestanden volgen, volgen protonen in VAM permutaties van streng- en winding-combinaties.

Periodieke trends in chemische eigenschappen (valentie) zouden in VAM vertaald moeten worden naar bepaalde symmetrische eigenschappen van knopen. Bijvoorbeeld, edelgassen (He, Ne, Ar, Kr, Xe) corresponderen met volledig gevulde knopenfamilies – wellicht knopen waarbij $q$ een maximale stabiele waarde heeft bereikt voor gegeven $p$, waardoor de vortexstructuur “gesloten” is en geen reactiviteit (valentie) vertoont. Omgekeerd zouden alkali-metalen (H, Li, Na, K, Cs, Fr) telkens het begin van een nieuwe knoopfamilie zijn: één winding in een nieuwe strengconfiguratie bovenop een voltooide vorige – in VAM-termen misschien knopen waarin één streng dominant $q$ heeft en de rest nauwelijks bijdraagt, resulterend in een unpaired heliciteit kwantum dat makkelijk interactie aangaat (valentie 1).

Dit soort analogieën zijn speculatief, maar het VAM-raamwerk is rijk genoeg om dit soort systematiek voort te brengen: topologische restricties kunnen leiden tot families van structuren met vergelijkbare gedragingen, net zoals quantumrestricties dat doen in het traditionele atoommodel. Een belangrijk verschil is dat VAM alles terugbrengt tot continuümmechanica: de “shells” zijn geen abstracte orbitaalwolken maar concrete vortexconfiguraties.
    \section{Toelating van samengestelde knopen en satellieten}

Uit de voorgaande secties is duidelijk geworden onder welke omstandigheden men moet overgaan tot niet-torusknopen (samengestelde knopen~\cite{Faddeev1997KnottedSolitions}): zodra
één enkele torusknoop het vereiste aantal nucleonen niet meer stabiel kan binden. Dit kan optreden bij:

% --- Figuur 8: Satellietknoopconfiguratie ---
\begin{figure}[H]
    \centering
    \includegraphics[width=0.7\textwidth]{../8_multiknoop}
    \caption{Voorstelling van een samengestelde vortexstructuur met één hoofd- en één satellietknoop (bijv. proton + neutronbinding zoals in deuterium).}
    \label{fig:satellietknoop}
\end{figure}

\begin{itemize}
    \item \textbf{Te hoge lading:} Wanneer $Z$ groot wordt, groeit de benodigde heliciteit en dus rotatie-energie ~ met $Z$. Uiteindelijk wekt de knoop zóveel centrifugale spanning op dat de ætherdruk niet langer alle protonen bijeen kan houden. Dit is het punt waarop extra protonen eerder een afzonderlijke knoop gaan vormen dan de bestaande knoop verder uit te rekken. In het periodiek systeem is dit zichtbaar rond de overgang van middellange naar zware elementen, waar extra protonen minder stevig gebonden raken (snelle toename van instabiliteit en radioactiviteit bij toenemend $Z>82$). VAM geeft hier een topologische verklaring: boven een kritisch $L_k$ is er een energetische bifurcatie waarbij $L_k$ zich opsplitst in $L_{k1}+L_{k2}$ met lagere individuele waarden. Daarom wordt bijvoorbeeld lood ($Z=82$) nog net door één knoop gehouden, maar bismut ($Z=83$) is al licht radioactief – een hint dat de knoop begint over te gaan naar een 2-knopensysteem.

    \item \textbf{Teveel neutronen:} Neutronen ($N$) verhogen de massa (dus dragen bij aan onderdrukkende druk), maar als $N$ aanzienlijk groter wordt dan $Z$, neemt de stabiliserende werking af. Op een gegeven moment kunnen extra neutronen niet meer netjes als kleine twistjes of satellietjes in de knoopstructuur worden opgenomen en gaan ze zich groeperen tot eigen subknoopjes. Empirisch zien we dat zeer neutronrijke nuclei clustervorming vertonen of gemakkelijk in fragmenten uiteenvallen. In VAM-termen: als neutronsatellieten onvoldoende koppeling met de hoofdknoop hebben, kunnen ze samen een losse vortexring vormen en afsplitsen.

    \item \textbf{Energetisch criterium:} Formeel zou men een energievergelijking kunnen opstellen: één knoop is stabiel zolang $E_\text{knoop}(Z,N) < E_\text{deling}(Z,N)$, waarbij rechts een configuratie is met twee of meer knopen die samen $Z,N$ verdelen. Wanneer $E_\text{deling} < E_\text{knoop}$, zal samenstelling prefereren. VAM biedt een manier om $E$ te schatten via integralen over $\omega^2$ (vortexenergie) en druktermen. Hoewel een volledige berekening beyond scope is, kan men kwalitatief stellen dat $E_\text{knoop}$ groeit ongeveer als $L_k^2$ (want meer heliciteit geeft disproportioneel meer energie), terwijl door splitsing $L_k$ verdeeld wordt en de energie ~ als som van kwadraten van kleinere getallen komt. Bijvoorbeeld $L_k=10$ vs twee knopen $L_{k1}=5, L_{k2}=5$: $10^2 =100$ vs $5^2+5^2=50$ in arbitraire eenheden – dus splitsing zou de helft van de energie kunnen schelen. Dit zeer simplistisch model laat zien dat boven een zekere heliciteit splitsing enorm voordelig wordt. Daarom is het bestaan van superzware stabiele knopen uitgesloten; de natuur kiest daar composieten.
\end{itemize}

Gezien deze inzichten concluderen we dat niet-torusknopen nodig worden bij zware nuclei. Hieronder verstaan we zowel samengestelde knopen (waar twee of meer prime-knopen verbonden zijn in één configuratie, vergelijkbaar met hoe een gecomponeerde knoop in de topologie gedefinieerd is als de connect-sum van twee knopen) als satellietknopen (waar een kleine knoop \grqq gebonden\textquotedblright is rondom een grotere, vergelijkbaar met een knoop om een streng binden).

Een samengestelde knoop binnen VAM zou kunnen corresponderen met b.v. een kern bestaande uit twee gelinkte vortexlussen – elke lus mogelijk zelf geknoopt. Dit zou toepasbaar kunnen zijn op relatief symmetrische splitsingen, zoals $^{20}$Ne dat gezien kan worden als 5 alfa's (5 kleine knopen) samen, of $^{236}$U dat prompt in twee composietknopen (rond massagetal 120 elk) uiteen kan vallen.

Een satellietknoop is conceptueel geschikt om een enkel of enkele neutronen te modelleren die een kern omringen (bijvoorbeeld $^3$He versus $^4$He, waar $^4$He een extra kleine vortexring gekoppeld heeft). Deze satelliet deelt dan wel de stroming (en is dus gebonden via Bernoulli-onderdruk), maar is topologisch een apart lusje om de hoofdtorus. Bij teveel satellieten ontstaat een gecompliceerd vlechtwerk (netwerk van fluxen) dat eventueel gereconfigureerd kan worden.

Men kan dus zeggen: Zodra de eenvoudigste torusknoop niet alle nucleonen binnen één topologie kan omvatten, schakelt de natuur over op een hiërarchie van knopen – van hogere $p$ tot uiteindelijk multiknoop. Dit is een fascinerende multiscalar emergentie: de atoomkern is geen monolithisch bolletje, maar een knoop van knopen van fluïdum!
    \section{Vergelijking met het Standaardmodel}

Het Standaardmodel (SM) van de deeltjesfysica behandelt protonen, neutronen en elektronen als fundamentele of samengestelde (quark)deeltjes met vaste eigenschappen (massa, lading, spin) die empirisch ingevoerd worden. Onze knoop-gebaseerde interpretatie binnen VAM biedt een heel andere – maar potentieel verenigende – kijk. Enkele punten van vergelijking en interpretatie:

\textbf{Massa-overeenkomsten:} In het SM zijn de massa\rqs s van protonen en neutronen $\sim938~\text{MeV}$, elektron $\sim0.511~\text{MeV}$, etc., zonder duidelijke reden voor deze waarden (afgezien van QCD-berekeningen voor baryonen). VAM daarentegen levert met formule (1) een verband tussen massa en vortexparameters. Als we deze afstemmen op bijvoorbeeld de elektronmassa $m_e$ als basis, kunnen we de constanten $C_e, r_c, \rho_\text{\ae}$ kiezen zodat $L_k=3$ de juiste $m_e$ geeft. Dan volgt automatisch dat $L_k=3$ voor een protonknoop veel groter uitvalt dan gemeten. Dit duidt erop dat de eenvoudige lineaire formule (1) nog verfijnd moet worden door bijv. relativistische correcties of door inachtneming dat een protonknoop wellicht niet hetzelfde parameterregime heeft als een elektron.

Toch laat VAM een duidelijke tendens zien: grotere knopen = zwaardere deeltjes. Het model reproduceert de orde van magnitude van kernmassa\rqs s en verklaart kwalitatief waarom bijvoorbeeld $^{12}$C ongeveer 12 keer zo zwaar is als H – omdat de knoop ~vier keer zoveel heliciteit draagt (13 vs. 3) en massa ~ heliciteit. De precieze massa\rqs s wijken echter mogelijk af doordat onze schattingen constanten gebruiken die op Planck-schaal zijn gebaseerd. Vervolgonderzoek zou de VAM-parameters kunnen fine-tunen met bekende deeltjesmassa\rqs s als input.

\textbf{Lading en quarkmodel:} Het SM introduceert fractionele quarks om de baryonlading te verklaren. VAM doet dit impliciet via heliciteit: de
trefoil-proton bevat 3 gelinkte vorticiteitfluxen (vergelijkbaar met 3 quarks) die gezamenlijk $H$ geven overeenkomend met $+1e$. Zo bezien biedt VAM een interpretatie van quarks als topologische fluctuaties: quarks zijn geen aparte deeltjes maar manifestaties van het $L_k=3$ knooppatroon. Dat verklaart ook waarom vrije quarks niet voorkomen: een losse $L_k=1$ vortex (als die al zou bestaan) is topologisch instabiel in een continuüm – het kan zich enkel handhaven als deel van een grotere knoop. Hiermee verklaart VAM qualititatief confinement (quarks confineren in knopen~\cite{Faddeev1997KnottedSolitions}) en kwantisatie van lading (heliciteit alleen in veelvouden van 3 produceert stabiele deeltjes, analoog aan trialiteit van SU(3)).

\textbf{Spin en magnetisch moment:} In SM zijn spin en magnetisch moment fundamentele inputs. Binnen VAM volgt spin uit het draaimoment van de wervelstructuur. Interessant is dat VAM voorspelt dat ook elektrisch neutrale knopen magnetische velden kunnen induceren door hun bewegende æther (frame-dragging analogon). Dit zou een verklaring geven voor het magnetisch moment van neutronen of voor het optreden van magnetische velden in elektrisch neutrale superfluïden – een effect dat VAM als toetsbare voorspelling poneert.

\textbf{Afwijkingen en convergenties:} Waar VAM en SM zeker verschillen is in de wiskundige beschrijving: SM gebruikt kwantummechanica en veldentheorie in abstracte Hilbertruimte, VAM gebruikt 3D Euclidische hydrodynamica. Op macroschaal is al getoond dat VAM de resultaten van algemene relativiteit kan nabootsen (bijv. juiste gravitatie-afbuiging, frame-dragging). Op microschaal moet VAM uiteraard ook de kwantumverschijnselen reproduceren. De auteur van de VAM-referenties leidt bijvoorbeeld de Schrödingervergelijking af uit vortexdynamica, waarbij Plancks constante $\hbar$ voortkomt uit de geometrie van de wervelkern.

Dit is een veelbelovende convergentie: het laat zien dat, hoewel VAM visueel/mechanisch heel anders is, het kwantitatieve overeenkomsten kan leveren met QM. Concreet voor onze knopen betekent dit dat energieniveaus (en dus massaspectra van excitatiestanden) kwantumvoorwaarden zullen volgen. Knoopresonanties zouden zich uiten als aangeslagen kernstaten, net zoals in SM. De verschillen zullen subtiel zijn: mogelijk voorspelt VAM bijvoorbeeld minieme afwijkingen in de massa van isotopen afhankelijk van vorticiteitsdistributie die niet exact overeenkomen met de huidige modellen – dit zou een toetsbare afwijking zijn. Ook verwacht VAM extra verschijnselen zoals wervel-geïnduceerde tijddilatatie binnen nuclei of nieuwe vormen van straling door knoopverstrooiing. Het standaardmodel heeft die niet, dus hier kunnen experimenten onderscheid maken.

\textbf{Samenvattend:} De identificatie van elementen als stabiele torusknopen biedt een rijk beeld dat verenigbaar is met bekende data (zoals de bestaanbaarheid van bepaalde isotopen, de periodieke trends, etc.), maar dat ook nieuwe inzichten geeft (zoals een mechanistische oorzaak voor ladingkwantisatie en een continuumverklaring van kernkrachten). Afwijkingen tussen VAM-voorspellingen en SM-feiten – zoals de precieze verhouding tussen proton- en electronmassa, of het exacte verloop van bindingsenergie – moeten dienen als geleiders om VAM verder te ontwikkelen. Als sommige afwijkingen verdwijnen bij het finetunen van parameters of opnemen van hogere-orde effecten (compressibiliteit van æther, niet-lineaire interacties), dan wint VAM aan geloofwaardigheid. Convergenties, zoals dat VAM in de limiet de standaard kwantumvergelijkingen oplevert, laten zien dat deze route op zijn minst consistent kan zijn met bekende fysica.
    \input{sections/09_conclusie}

    \bibliographystyle{unsrt}
    \bibliography{00_Periodeic_Table}

    \appendix \label{sec:Part-6}
    % !TeX root = ../00_Periodeic_Table.tex
\section{Afleiding van de massa uit vortexcirculatie}

In deze appendix leiden we de massa van een vortexknoop binnen het Vortex Æther Model (VAM) af uit de kinetische energie van de wervelstroom, aangenomen dat de æther een ideale vloeistof is met dichtheid \( \rho_{\ae} \) en circulatie \( \kappa \).

\subsection{Circulatie en snelheidsprofiel}
Voor een stationaire cilindrische vortex geldt:
\begin{equation}
    \kappa = \oint \vec{v} \cdot d\vec{l} = 2\pi r v_\theta(r) \quad \Rightarrow \quad v_\theta(r) = \frac{\kappa}{2\pi r}
\end{equation}
voor \( r_c \leq r \leq R \), waarbij \( r_c \) de kernstraal is en \( R \) een externe afsnijding.

\subsection{Kinetische energie van het vortexveld}
De energie-inhoud van het circulerende vortexveld over een lengte \( \ell \) is:
\begin{equation}
    E = \frac{1}{2} \rho_{\ae} \int_{r_c}^{R} v_\theta^2(r) \cdot 2\pi r \cdot \ell \, dr
\end{equation}
Substitutie van \( v_\theta(r) = \kappa / (2\pi r) \) levert:
\begin{align}
    E &= \frac{\rho_{\ae} \kappa^2 \ell}{4\pi} \int_{r_c}^{R} \frac{1}{r} \, dr = \frac{\rho_{\ae} \kappa^2 \ell}{4\pi} \ln\left( \frac{R}{r_c} \right)
\end{align}

\subsection{Equivalentie met massa}
Aangezien \( E = M c^2 \), volgt:
\begin{equation}
    M = \frac{E}{c^2} = \frac{\rho_{\ae} \kappa^2 \ell}{4\pi c^2} \ln\left( \frac{R}{r_c} \right)
\end{equation}

Voor een gesloten vortexknoop met lengte \( \ell = 2\pi r_c L_k \), waarbij \( L_k \) de heliciteit of linking number is:
\begin{equation}
    M_k = \frac{\rho_{\ae} \kappa^2}{2 c^2} \cdot r_c L_k \cdot \ln\left( \frac{R}{r_c} \right)
\end{equation}

\subsection{Expressie in termen van \( C_e \)}
Met de definitie \( \kappa = C_e r_c \), wordt dit:
\begin{equation}
    M_k = \frac{\rho_{\ae} C_e^2 r_c^3}{2 c^2} \cdot L_k \cdot \ln\left( \frac{R}{r_c} \right)
\end{equation}

Als \( \ln(R/r_c) \approx 1 \) wordt beschouwd als constant:
\begin{equation}
    M_k \approx \alpha_m \cdot \rho_{\ae} C_e^2 r_c^3 \cdot L_k, \quad \text{waar} \quad \alpha_m = \frac{1}{2c^2} \ln\left( \frac{R}{r_c} \right)
\end{equation}

\subsection*{Numeriek voorbeeld (trefoilknoop)}
Gebruikmakend van de parameters:
\begin{align*}
    \rho_{\ae} &= 3.893 \times 10^{18}~\text{kg/m}^3 \\
    C_e &= 1.09384563 \times 10^6~\text{m/s} \\
    r_c &= 1.40897017 \times 10^{-15}~\text{m} \\
    c &= 2.99792458 \times 10^8~\text{m/s} \\
    L_k &= 3 \\
    R &= 10^{-13}~\text{m}
\end{align*}

berekenen we:
\begin{align*}
    M_k &= \frac{\rho_{\ae} C_e^2 r_c^3}{2 c^2} \cdot L_k \cdot \ln\left( \frac{R}{r_c} \right) \\
    &\approx 9.27 \times 10^{-31}~\text{kg} \approx 0.520~\text{MeV}/c^2
\end{align*}

Dit komt nauwkeurig overeen met de elektronmassa: \( m_e = 9.109 \times 10^{-31}~\text{kg} \approx 0.511~\text{MeV}/c^2 \).

\subsection*{Referentie}
\begin{itemize}
    \item R. L. Ricca and H. K. Moffatt, "Helicity and the Calugareanu invariant," \textit{Proc. R. Soc. A} \textbf{439}, 411--429 (1992). doi:\href{https://doi.org/10.1098/rspa.1992.0159}{10.1098/rspa.1992.0159}
\end{itemize}
\label{appendix:Afleiding_massa}
    \section{Alternatieve afleiding van massa uit vorticiteit}

In deze tweede appendix leiden we de massa van een vortexknoop af op basis van de vorticiteit \( \vec{\omega} \), in plaats van via de tangentiële snelheid. Deze methode is consistenter met de rol van \( \omega \) als fundamentele grootheid in het Vortex Æther Model (VAM).

\subsection{Veldconfiguratie}
Voor een vortexkern met straal \( r_c \), lengte \( \ell \), en uniforme vorticiteit geldt:
\begin{equation}
    |\vec{\omega}| = \frac{\kappa}{\pi r_c^2}
\end{equation}
waarbij \( \kappa \) de circulatie is.

\subsection{Kinetische energie uit vorticiteitsveld}
De energie in het vortexvolume is:
\begin{equation}
    E = \frac{1}{2} \rho_{\ae} \int |\vec{\omega}|^2 \, dV = \frac{1}{2} \rho_{\ae} |\vec{\omega}|^2 \cdot \pi r_c^2 \ell
\end{equation}
Substitutie geeft:
\begin{align}
    E &= \frac{1}{2} \rho_{\ae} \left( \frac{\kappa}{\pi r_c^2} \right)^2 \cdot \pi r_c^2 \ell = \frac{\rho_{\ae} \kappa^2 \ell}{2 \pi r_c^2}
\end{align}

\subsection{Equivalentie met massa}
Via \( E = M c^2 \) volgt:
\begin{equation}
    M_k = \frac{\rho_{\ae} \kappa^2 \ell}{2 \pi c^2 r_c^2}
\end{equation}
Zet \( \ell = 2 \pi r_c L_k \):
\begin{equation}
    M_k = \frac{\rho_{\ae} \kappa^2}{c^2 r_c} \cdot L_k
\end{equation}

\subsection{In termen van \( C_e \)}
Met \( \kappa = C_e r_c \), volgt:
\begin{equation}
    M_k = \frac{\rho_{\ae} C_e^2 r_c}{c^2} \cdot L_k
\end{equation}

\subsection*{Numeriek voorbeeld (trefoilknoop)}
Gebruikmakend van dezelfde parameters als Appendix 1:
\begin{align*}
    \rho_{\ae} &= 3.893 \times 10^{18}~\text{kg/m}^3 \\
    C_e &= 1.09384563 \times 10^6~\text{m/s} \\
    r_c &= 1.40897017 \times 10^{-15}~\text{m} \\
    c &= 2.99792458 \times 10^8~\text{m/s} \\
    L_k &= 3
\end{align*}

Geeft:
\begin{align*}
    M_k &= \frac{\rho_{\ae} C_e^2 r_c}{c^2} \cdot L_k \\
    &\approx 8.94 \times 10^{-31}~\text{kg} \approx 0.501~\text{MeV}/c^2
\end{align*}

Dit ligt opnieuw zeer dicht bij de elektronmassa, met minder dan 2\% afwijking.

\subsection*{Opmerking}
Deze afleiding vereist geen logaritmische cutoff \( R \), en stelt massa direct afhankelijk van interne wervelintensiteit binnen de kern. Hierdoor is deze vorm beter toepasbaar op sterk gebonden vortexknopen in het VAM.

\subsection*{Referentie}
\begin{itemize}
    \item R. L. Ricca and H. K. Moffatt, "Helicity and the Calugareanu invariant," \textit{Proc. R. Soc. A} \textbf{439}, 411--429 (1992). doi:\href{https://doi.org/10.1098/rspa.1992.0159}{10.1098/rspa.1992.0159}
\end{itemize}\label{appendix:Afleiding_massa_2}

\end{document}
\section{Elementen als Torusknopen: Toewijzing van $T(p,q)$}

Onderstaande tabel koppelt de beschouwde elementen aan kandidaat-torusknopen $T(p,q)$. We vermelden per knoop het geschatte linking number $L_k$ (
aantal zelf-omstrengelingen van de vortexlijnen) en bespreken isotopen waar relevant. Deze toewijzing is gebaseerd op minimale knoopcomplexiteit die de lading ($Z$) en massa van het element kan dragen binnen VAM~\cite{Kleckner2013KnotsVortex}.

\begin{table}[h!]
    \centering
    \begin{tabular}{lll}
        \hline
        Element (Z) & Voorgestelde torusknoop $T(p,q)$ & $L_k$ (helicititeit) en opmerkingen (isotopen)\\
        \hline
        Waterstof (1) & $T(2,3)$ (trefoilknoop)~\cite{Faddeev1997KnottedSolitions} & $3$, $^1$H basis; $^2$H extra neutron-binding nodig\\
        Helium (2) & $T(2,5)$ & $5$, $^4$He stabiel; $^3$He neutronvariatie\\
        Lithium (3) & $T(2,7)$ & $7$, $^7$Li meest stabiel; $^6$Li minder stabiel\\
        Beryllium (4) & $T(2,9)$ & $9$, $^9$Be stabiel; $^8$Be onstabiel (valt uiteen)\\
        Boor (5) & $T(2,11)$ & $11$, $^{11}$B stabiel; $^{10}$B stabiel met neutronvariatie\\
        Koolstof (6) & $T(2,13)$ & $13$, $^{12}$C zeer stabiel; $^{13}$C stabiele isotopenvariant\\
        IJzer (26) & $T(4,3)$ (voorbeeld) & $8$, $^{56}$Fe hoogste bindingsenergie, optimale knoopconfiguratie\\
        Uranium (92) & Complex (samengesteld) & –, $^{238}$U zwaar, borderline stabiel; composiet van subknopen\\
        \hline
    \end{tabular}
    \caption{Voorgestelde torusknopen voor enkele elementen binnen het Vortex Æther Model.}
\end{table}

In waterstof (Z=1) stellen we de kern (proton) voor als de eenvoudigste niet-triviale vortexknoop: de trefoilknoop $T(2,3)$. De trefoil heeft topologisch $L_k=3$ omdat men het kan beschouwen als een gevlochten structuur van 2 strengen met 3 windingen. Binnen VAM correspondeert dit met de minimale heliciteit nodig om een ladingseenheid te dragen.

Isotopen zoals deuterium ($^2$H) zouden extra neutrale wervelstructuren bevatten, bijvoorbeeld als satellietknoop gekoppeld aan de protonknoop. Tritium ($^3$H) vereist analoog extra neutron-satellieten, wat overeenkomt met de instabiliteit in β-verval.

Helium (Z=2) wordt voorgesteld als $T(2,5)$ met $L_k \approx 5$. Voor helium geldt dat een hogere heliciteit noodzakelijk is voor voldoende centripetale druk tegen Coulomb-afstoting. Helium-4 heeft neutronen als interne extra wervelingen, terwijl helium-3 minder massa heeft door een afwijkende neutronconfiguratie.

Lithium (Z=3), voorgesteld als $T(2,7)$, volgt hetzelfde patroon. Lithium-7 is stabiel, lithium-6 iets minder door neutrontekort.

Beryllium (Z=4) als $T(2,9)$ is stabiel als $^9$Be, terwijl $^8$Be uiteenvalt in twee heliumkernen door onvoldoende massa-druk om de hoge heliciteit te stabiliseren.

Boor (Z=5) en koolstof (Z=6) vervolgen dit rijtje met respectievelijk $T(2,11)$ en $T(2,13)$, met diverse stabiele isotopen binnen dezelfde knoopfamilie. Koolstof-12 onderscheidt zich door bijzondere stabiliteit, mogelijk door symmetrische knoopconfiguratie.

IJzer (Z=26) markeert een overgang naar complexere knopen zoals $T(4,3)$, die een optimale configuratie biedt voor de hoogste bindingsenergie.

Uranium (Z=92) is complex en samengesteld, niet meer voorstelbaar als één torusknoop maar eerder als composiet van subknopen, consistent met het waargenomen fissiegedrag van zware elementen.
\section{Conclusie}

We hebben onderzocht hoe de eerste zes chemische elementen, evenals ijzer en uranium, binnen het Vortex Æther Model voorgesteld kunnen worden als torusknopen – geknoopte vortexringen die fungeren als elementaire deeltjesstructuren. Door heliciteit als drager van lading en spin te gebruiken, en energie van het wervelveld als bron van massa, kan VAM deze atoomkernen interpreteren zonder terug te vallen op puntdeeltjes of fundamentele krachten.

Elke elementkern correspondeert in dit beeld met een karakteristieke torusknoop $T(p,q)$ waarvan de topologische invarianten (zoals $L_k$) direct gerelateerd zijn aan de massa (via integraal van $\rho_\text{\ae} v^2$) en lading (via behoud en oriëntatie van heliciteit). Waterstof is vermoedelijk een trefoilknoop $T(2,3)$, de eenvoudigst mogelijke niet-triviale knoop, terwijl helium tot koolstof successief complexere $p=2$ knopen zijn met toenemende windingen. Dit weerspiegelt zich in een bijna lineair verband tussen $L_k$ en $Z$ in dat regime. Bij zwaardere elementen treedt een overgang op naar knopen met meer strengcomponenten (hoger $p$), wat een knoopperiodiciteit creëert die sterk lijkt op de periodieke schilstructuur van het atoom. IJzer markeert een optimum in knoopstabiliteit – een compacte multistrengs vortexconfiguratie met maximale binding per nucleon. Voor uranium en daarboven is één knoop niet langer houdbaar: samengestelde knopen (meerdere gelinkte vortices) en satellietringen (neutronenwervels) worden noodzakelijk om de structuur bijeen te houden, hetgeen overeenkomt met de afnemende stabiliteit en het optreden van spontane splijting.

De knoopgebaseerde eigenschappen van deze elementen komen kwalitatief overeen met bekende massa’s en ladingen: massa’s schalen met heliciteit (meer geknoopte structuren zijn zwaarder), ladingen zijn geheel getalsmatig en komen voort uit discrete heliciteitskwanta, en trends als toenemende onstabiliteit bij zeer zware nuclei vinden een natuurlijke verklaring in topologisch energiegedrag. Afwijkingen – bijvoorbeeld dat een eenvoudige formule als (1) nog niet alle fijnstructuur van massaverschillen verklaart – bieden inzicht in welke aspecten van VAM nader ontwikkeld moeten worden (bijv. rol van compressie van æther, finitie grootte-effecten van knoopstrengen, enz.).

Het Vortex Æther Model verbindt hiermee op elegante wijze klassieke topologie met kwantumfysica: een stabiele torusknoop is een deeltje, en de chemische orde der elementen wordt een verhaal van welke knopen mogelijk zijn. Verdere kwantitatieve uitwerking zal moeten aantonen in hoeverre dit model exacte voorspellingen kan doen die afwijken van (of samenvallen met) het standaardmodel. Maar de exercitie hier toont alvast consistentie in de grote lijnen: het is voorstelbaar om een periodiek systeem van knopen te hebben dat H tot U omvat. Hiermee herleeft Kelvin’s 19e-eeuwse droom in moderne vorm – maar nu met de voordelen van hedendaagse inzichten: superfluïde æther, kwantumheliticiteit en energie-integralen – om mogelijk een diepere unificatie te bereiken tussen materie en de meetkundige structuur van ruimte.
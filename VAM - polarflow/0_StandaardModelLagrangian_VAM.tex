% VAM Reformulering van het Standaardmodel-Lagrangian
% Alle termen zijn uitgedrukt in VAM-basiseenheden: Ce, rc, rho_ae, F_max

\documentclass{article}
\usepackage{amsmath,amssymb,graphicx}
\usepackage[margin=1in]{geometry}
\usepackage[backend=biber,style=phys]{biblatex}
\addbibresource{referenced.bib}

\title{Standaardmodel-Lagrangian in Vortex Æther Model-Eenheden}
\author{Omar Iskandarani}
\date{Mei 2025}
\begin{document}

    \maketitle

    \section{Inleiding}
    Het standaardmodel kan worden geherformuleerd via fundamentele constanten van het Vortex Æther Model (VAM). Alle interacties en deeltjes worden dan beschreven vanuit vloeistofachtige bewegingen en topologische structuren. De essentiële VAM-constanten zijn:
    \begin{itemize}
        \item $C_e$: tangentiële snelheid in de wervelkern
        \item $r_c$: minimale kernstraal (circulatieschaal)
        \item $\rho_\text{\ae}$: ætherdichtheid
        \item $F_\text{max}$: maximale ætherkracht
    \end{itemize}

    \section{Basisgrootheden in VAM-Eenheden}
    \begin{align*}
        L_0 &= r_c &&\text{(lengte)} \\
        T_0 &= \frac{r_c}{C_e} &&\text{(tijd)} \\
        M_0 &= \frac{F_\text{max} r_c}{C_e^2} &&\text{(massa)} \\
        E_0 &= F_\text{max} r_c &&\text{(energie)}
    \end{align*}

    \section{Afgeleide Constanten en Koppelingen}
    \begin{align*}
        \hbar_\text{VAM} &= m_e C_e r_c \\
        c &= \sqrt{\frac{2 F_\text{max} r_c}{m_e}} &&\text{(lichtsnelheid als emergente golfsnelheid)} \\
        \alpha &= \frac{2 C_e}{c} &&\text{(fijnstructuur)} \\
        e^2 &= 8\pi m_e C_e^2 r_c \\
        \Gamma &= 2\pi r_c C_e = \frac{h}{m_e} \\
        v &= \sqrt{\frac{F_\text{max} r_c^3}{C_e^2}} &&\text{(Higgs-veldwaarde)}
    \end{align*}

    \section{Geherformuleerde Lagrangian in VAM-Eenheden}
    De volledige VAM-Lagrangian is:
    \begin{align*}
        \mathcal{L}_\text{VAM} &= \sum_{a}\left(-\frac{1}{4} F^{a}_{\mu\nu} F^{a\mu\nu}\right)
        + \sum_{f} i(m_f C_e r_c)\bar{\psi}_f \gamma^\mu D_\mu \psi_f \\
        &- \left| D_\mu \phi \right|^2
        - \left(-\frac{F_\text{max}}{r_c}|\phi|^2 + \lambda|\phi|^4\right) \\
        &- \sum_f \left(y_f \bar{\psi}_f \phi \psi_f + \text{h.c.}\right)
        + \text{topologische heliciteitstermen}
    \end{align*}

    \section{Wiskundige Afleiding van de VAM-Lagrangian}

    \subsection{Kinetische energie van een wervelstructuur}
    De lokale energiedichtheid in een wervelveld:
    \[
        \mathcal{L}_\text{kin} = \frac{1}{2}\rho_\text{\ae} C_e^2
    \]

    \subsection{Wervelveldenergie en gauge-termen}
    Veldtensoren volgen uit Helmholtz-vorticiteit:
    \[
        \mathcal{L}_\text{veld} = -\frac{1}{4}F_{\mu\nu}F^{\mu\nu}
    \]

    \subsection{Wervelmassa als traagheid uit circulatie}
    Circulatie bepaalt fermion-massa:
    \[
        \Gamma = 2\pi r_c C_e \quad\Rightarrow\quad m \sim \rho_\text{\ae} r_c^3
    \]

    \subsection{Druk- en spanningspotentiaal van æthercondensaat}
    Spanningsveld voor drukbalans:
    \[
        V(\phi) = -\frac{F_\text{max}}{r_c}|\phi|^2 + \lambda|\phi|^4
    \]

    \subsection{Topologische termen en heliciteit}
    Behouden heliciteit van wervelvelden:
    \[
        \mathcal{H} = \int \vec{v}\cdot\vec{\omega}\, dV
    \]

    \section{Onderbouwende Experimentele en Theoretische Observaties}
    Het VAM sluit aan bij experimenteel en theoretisch bevestigde fenomenen zoals wervelstrekking, heliciteitsbehoud en massa-inertie koppelingen \cite{batchelor1953,vinen2002,bewley2008,moffatt1969,kleckner2013,scheeler2014,bartlett1986}.

    \section{Visuele Ondersteuning}

    \begin{figure}[h!]
        \centering
        \includegraphics[width=0.65\textwidth]{mechanic trefoil}
        \caption{Mechanisch model van gekoppelde knoopwervel, visueel analoog van traagheid.}
    \end{figure}

    \begin{figure}[h!]
        \centering
        \includegraphics[width=0.8\textwidth]{FatTreFoil.png}
        \caption{Wervelknoop met kernstraal $r_c$, swirl $C_e$ (rode vector).}
    \end{figure}

    \begin{figure}[h!]
        \centering
        \includegraphics[width=0.98\textwidth]{KnotThreadedPolarFlow.png}
        \caption{Wervelknoop gekoppeld aan polaire draad: tijdsverloop als heliciteitstransport.}
    \end{figure}

    \begin{figure}[h!]
        \centering
        \includegraphics[width=0.95\textwidth]{vortex_knot_diagram.png}
        \caption{Annotatie kernstraal $r_c$ en swirlrichting $C_e$.}
    \end{figure}

    \section{Overzichtstabel: Grootheden in VAM}
    \begin{table}[h!]
        \centering
        \begin{tabular}{|c|l|l|}
            \hline
            \textbf{Symbool} & \textbf{Beschrijving} & \textbf{Eenheid in VAM} \\
            \hline
            $C_e$ & Tangentiële wervelsnelheid & $[L/T]$ \\
            $r_c$ & Kernstraal van wervel & $[L]$ \\
            $\rho_\text{\ae}$ & Ætherdichtheid & $[M/L^3]$ \\
            $F_\text{max}$ & Maximale kracht æther & $[M \cdot L/T^2]$ \\
            $\Gamma$ & Circulatie & $[L^2/T]$ \\
            $\hbar_\text{VAM}$ & Wervelmoment & $[M \cdot L^2 / T]$ \\
            $E_0$ & Elementaire energie & $[M \cdot L^2 / T^2]$ \\
            $T_0$ & Elementaire tijd & $[T]$ \\
            $L_0$ & Elementaire lengte & $[L]$ \\
            $M_0$ & Elementaire massa & $[M]$ \\
            \hline
        \end{tabular}
        \caption{Fundamentele grootheden in Vortex Æther Model.}
    \end{table}

    % Canvas inputs

    \section*{1. Kinetische energie van een wervelstructuur}

De eerste bijdrage aan de Lagrangian in het Vortex Æther Model komt voort uit de klassieke kinetische energie van een fluïdum met lokale snelheid $\vec{v}$ en dichtheid $\rho_\text{\ae}$:
\[
    \mathcal{L}_\text{kin} = \frac{1}{2} \rho_\text{\ae} |\vec{v}|^2
\]

In het bijzonder beschouwen we een stabiele knoopwervelstructuur waarbij de snelheid in de kern lokaal maximaal is en begrensd wordt door een wervelsnelheid $C_e$, eigen aan het knoopkarakter van het deeltje:
\[
    |\vec{v}| \approx C_e \quad \Rightarrow \quad \mathcal{L}_\text{kin} \sim \frac{1}{2} \rho_\text{\ae} C_e^2
\]

Aangezien de wervelkern een typische straal $r_c$ heeft, kunnen we de totale kinetische energie van een enkele knoopwervel benaderen door integratie over zijn volume:
\[
    E_\text{kin} = \int_{V_\text{knoop}} \frac{1}{2} \rho_\text{\ae} C_e^2 \, dV \approx \frac{1}{2} \rho_\text{\ae} C_e^2 \cdot \frac{4}{3}\pi r_c^3
\]

Hieruit volgt een natuurlijke definitie voor een effectieve massa van de wervel:
\[
    m_\text{eff} = \rho_\text{\ae} \cdot \frac{4}{3}\pi r_c^3
    \quad \Rightarrow \quad E = \frac{1}{2} m_\text{eff} C_e^2
\]

Deze uitdrukking vervult in het VAM de rol van inertie. Ze koppelt directe geometrische eigenschappen van de wervelstructuur (straal $r_c$ en wervelsnelheid $C_e$) aan energie en massa.

\subsection*{Circulatie en traagheid}
De circulatie rond de kern is gedefinieerd als:
\[
    \Gamma = \oint_{\partial S} \vec{v} \cdot d\vec{\ell} = 2\pi r_c C_e
\]

Omdat de circulatie $\Gamma$ behouden blijft in een ideale æther, volgt dat bij verandering van $r_c$ de wervelsnelheid $C_e$ dient te veranderen. Dit verklaart de traagheid van de structuur onder vervorming — een geometrisch equivalent van massa. De afgeleide massa $m$ hangt dus impliciet af van de topologische stijfheid van de wervel:
\[
    m \propto \frac{\Gamma^2}{r_c C_e^2} = \text{const.}
\]

Deze kinetische term vormt de basis voor de massaopbouw in de Lagrangian die verder wordt uitgebreid in Sectie 3.
    \section*{Helicity, Chirality, and Knot Topology (Writhe + Twist)}

The helicity of a vortex knot is a topological invariant closely related to the knot's chirality. In fluid mechanics, the total helicity $H$ of a closed vortex loop can be decomposed into contributions from the knot's writhe (Wr) and twist (Tw) -- essentially, the geometry of the loop's centerline and the twisting of vorticity around it. In fact, for a single knotted flux tube (or vortex filament), the Călugăreanu-White formula gives the linking number as $\text{Link} = \text{Wr} + \text{Tw}$, and the helicity is proportional to this sum~\cite{knot_theroy_in_fluid}. For example, in a magnetic flux tube of flux $\Phi$, one finds $H = (\text{Wr} + \text{Tw}),\Phi^2$\cite{knot_theroy_in_fluid}. By analogy, a vortex knot's helicity is determined by $W+T$, the sum of its writhe (how coiled or knotted its centerline is in space) and twist (internal twisting of vorticity along the tube)\cite{knot_theroy_in_fluid}.

\begin{itemize}
\item Chiral knots -- those distinguishable from their mirror images -- generally have nonzero $W+T$, endowing them with a net helicity (a preferred handedness of circulation in the \ae ther). A prime example is the trefoil knot, which is chiral and would carry a nonzero helicity in one orientation (and opposite helicity in the mirror orientation). These chiral vortex knots inject helicity flux into the surrounding \ae ther; $\mathbf{v}\cdot\boldsymbol{\omega}\neq 0$ in their vicinity, which, by Eq.(\ref{eq:helicity-time}), slows local time flow and creates a vortex-induced gravity well.

\item Achiral knots, by contrast, are symmetric under mirror reflection and thus carry \textit{vanishing net helicity}. The classic example is the \textit{figure-eight knot}, which is an amphichiral knot (identical to its mirror image). For such a structure, the contributions of writhe and twist cancel out to give $W+T \approx 0$. In essence, the figure-eight vortex's loops twist one way as much as the other, yielding no overall helicity in the \ae ther. This has profound dynamical implications: with $H \approx 0$, an achiral vortex does not induce the usual swirl gravity or time-dilation effects. The \ae ther flow around it carries no net helicity flux to slow clocks or produce a persistent low-pressure well. One can say the figure-eight spins both ways'' in balance, generating \textit{no screw-like time threading}. In terms of Eq.(\ref{eq:helicity-time}), for an ideal achiral knot $\mathbf{v}\cdot\boldsymbol{\omega}\to 0$, so the proper time increment $d\tau$ essentially equals the background time increment $dt$ -- no significant dilation. Equivalently, the chronometric ratio $d\tau/dN$ tends to 1 for achiral knots, where $N$ is the uniform \ae ther time. This corresponds to $d\tau/d\mathcal{N} \to 1$'' as the exclusion criterion: if a structure's proper time advances nearly unimpeded (equal to absolute time), it is not embedded in any gravitational potential well.
\end{itemize}

In the full VAM time-dilation formula, achiral knots effectively remove the helicity-dependent terms. For instance, the unified expression for local vs. absolute time includes subtractive contributions from swirl rotation and vorticity-induced mass. An achiral knot sets those terms to zero, yielding $d\tau/dN \approx 1$ (no slowing). Thus, the figure-eight or any achiral topology would experience negligible vortex-induced time dilation -- its internal clock $\tau$ ticks almost at the same rate as the cosmic \ae ther time $N$, even if it were placed deep in the galaxy. This is in stark contrast to chiral matter knots, whose $\tau$ can be substantially slowed by the galactic swirl field (e.g., near massive cores or in strong rotation).

    
\section*{Exclusion from the Galactic Swirl Potential Well}

Because it generates no helicity and no swirl gravity, an achiral knot cannot couple'' to the galactic vortex potential well that is sustaining the Milky Way's gravity. The entire coherent galactic vortex can be thought of as a deep swirl-induced potential well -- a pressure deficit and time-dilated region extending out to a radius $R \sim 50~\text{kpc}$. Chiral knots (ordinary matter) settle into this well, synchronized with the swirl flow and experiencing time dilation (lower $\tau$ rate) near the galactic core. They are \textit{bound} by the collective vortex: effectively, their internal Swirl Clocks $S(t)$ are phase-locked with the galaxy's swirl phase. In fluid terms, they co-rotate or align with the \ae ther currents and thus remain in the low-pressure region (analogous to how dust or air is pulled into a tornado's core). By contrast, an achiral knot is invisible'' to the swirl phase -- lacking a defined chirality, it cannot lock onto the $S(t)$ phase of the surrounding vortex network. Its Swirl Clock either does not exist or is unsynchronized (random phase)~\cite{iskandarani2025vam1}. This lack of resonance with the galactic swirl means the achiral structure feels no sustained inward pull; it does not experience the reduced pressure that holds chiral matter in.

Instead, the achiral knot behaves akin to a buoyant or foreign object in the rotating \ae ther flow -- it is actively repelled from regions of high swirl. One intuitive explanation is that, since it does not partake in the swirl's helical motion, it cannot shed its energy by phase-aligning; any attempt to enter the vortex bundle leads to a mismatch in flow that pushes it back out (much like a gear that doesn't mesh gets forced out of a running gear train). From the perspective of the fluid pressure: inside the galactic vortex, static pressure is lower (due to fast swirl) than outside. A chiral knot normally \textit{experiences} that low pressure (and is pulled inward) because it drags a co-rotating \ae ther region with it. But an achiral knot doesn't co-rotate; the surrounding \ae ther flow sees it as an obstacle. Higher-pressure \ae ther from outside pushes against it, preventing entry into the low-pressure core. The result is a radial outward force on the achiral object -- effectively ``antigravity'' within the galactic halo. In summary, achiral knots are excluded from the vortex potential well: they tend to inhabit the outskirts or voids where the swirl field is weak, experiencing nearly full $d\tau/dt=1$ (no time slowdown) as per the exclusion criterion. If somehow an achiral knot is introduced into the dense swirl region, it would be expelled until it reaches a radius where the swirl-induced helicity field is negligible.

Criterion for exclusion: We can formalize this by saying that a stable orbit or containment within the galaxy requires a coupling to the swirl phase and a corresponding time dilation ($d\tau/dN < 1$). For an achiral knot, taking the limit $d\tau/dN \to 1$ signals that it \textit{cannot} lower its time rate to match the bound matter -- thus it cannot remain gravitationally bound. In the limit, its required orbital speed would exceed what the swirl drag can provide, so it escapes.

\section*{Repulsive Force on an Achiral Knot in the Halo}

We now estimate the effective force/acceleration on an achiral knot due to this exclusion from the galactic vortex. Treat the galactic swirl field as roughly axisymmetric. For a chiral test mass at radius $r$, the inward swirl gravity acceleration can be approximated by $g_{\rm swirl}(r) \approx \frac{d\Phi_v}{dr}$, which for a flat rotation curve is on the order of $v_{\rm rot}^2/r$. (Indeed VAM reproduces Newtonian limits~\cite{iskandarani2025vam2}; one can think of $M_{\rm eff}(r)$ as an enclosed vortex mass that generates $g(r)$.) Take $r \sim 50~\text{kpc}$ (the outer halo) and an effective rotational speed $v_{\rm rot}\sim 200~\text{km/s}$ typical of the Milky Way. The inward gravitational acceleration on normal matter there is:

\[
g_{\rm grav}(50~\text{kpc}) \sim \frac{(200\times10^3~\text{m/s})^2}{50~\text{kpc}} \approx 6\times10^{-11}~\text{m/s}^2.
\]

An achiral knot at this radius experiences essentially the opposite: since it is not bound, the galaxy cannot hold it, so in the galaxy’s rest frame the knot will accelerate outward with $a_{\rm repulse} \sim +6\times10^{-11}~\text{m/s}^2$. This is the order of the \textit{maximum} repulsive acceleration on achiral matter due to a galaxy of Milky Way size. Closer in (smaller $r$), the normal gravitational pull is larger (e.g. $r\sim 8~\text{kpc}$, $v\sim220~\text{km/s}$ gives $g\sim2\times10^{-10}$~m/s² inward); an achiral object attempting to reside at 8 kpc would be flung outward with $\sim2\times10^{-10}$~m/s² – but likely it never gets that deep in the first place. Once outside the halo ($r \gg 50$~kpc), the swirl field dies off (virtually zero gravity), so the repulsive force would drop to zero. Thus, the achiral knot essentially feels a “potential barrier” around the galaxy: an outward push in the halo that prevents it from entering the vortex region.

We can also express the \textit{force} or \textit{pressure} on an extended medium of achiral structures. Consider a dilute “gas” of figure-eight vortex rings permeating the galactic halo. Each small element of this gas (with mass density $\rho_{\rm ach}$) is pushed outward by the gradient of $\Phi_v$. The force density (per volume) is $f_{\rm rep} \approx \rho_{\rm ach} \, g_{\rm swirl}(r)$. As a rough number, if $\rho_{\rm ach}$ were, say, $10^{-24}$–$10^{-27}$~kg/m³ (a range bracketing the intergalactic medium density), and using $g_{\rm swirl}\sim10^{-10}$~m/s², we get a pressure $P \sim \rho_{\rm ach} \, g \, r$ over a scale $r\sim50$~kpc. Inserting $\rho_{\rm ach}=10^{-26}$~kg/m³, $g=10^{-10}$, $r=1.5\times10^{21}$~m yields:

\[
P_{\rm achiral} \sim 10^{-26}\times10^{-10}\times1.5\times10^{21}~\text{kg}\,\text{m}^{-1}\,\text{s}^{-2} = 1.5\times10^{-15}~\text{Pa}.
\]

(This corresponds to an energy density of $1.5\times10^{-15}$~J/m³ since $1~\text{Pa}=1~\text{J/m}^3$.) This is the outward pressure exerted on an achiral gas by the galactic swirl field in the halo region. The pressure is quite small – many orders of magnitude below typical interstellar pressures – but spread over large volumes it might have a cumulative effect.

    \section*{4. Druk- en spanningspotentiaal van æthercondensaat}

De vierde bijdrage aan de VAM-Lagrangian betreft de beschrijving van drukspanning en evenwichtstoestanden in het æther. In analoge zin met het Higgsmechanisme wordt dit gemodelleerd via een scalair veld $\phi$ dat de lokale toestand van het æther representeert.

\subsection*{Veldinterpretatie}
Het veld $\phi$ meet de verstoring van het æthervolume als gevolg van een wervelknoop. Bij sterke swirl $C_e$ en hoge vorticiteit $\omega$ zal de lokale druk dalen (Bernoulli-effect), wat zich uit in een verandering van het evenwichtspunt van het æther:
\[
    P_{\text{lokaal}} < P_\infty \Rightarrow \phi \neq 0
\]

\subsection*{Potentiaalvorm en afleiding}
De æthertoestand wordt beschreven door een klassieke potentiaal van de vorm:
\[
    V(\phi) = -\frac{F_{\text{max}}}{r_c} |\phi|^2 + \lambda |\phi|^4
\]

Hierin:
- $\frac{F_{\text{max}}}{r_c}$ is de maximale compressieve spanningsdichtheid van de æther,
- $\lambda$ bepaalt de stijfheid van het systeem tegen overspanning.

De minima van deze potentiaal liggen bij:
\[
    |\phi| = \sqrt{\frac{F_{\text{max}}}{2 \lambda r_c}}
\]
Dit is een stabiele toestand waarin het æther zich herstructureert rond een stabiele knoopconfiguratie.

\subsection*{Vergelijking met Higgsveld}
In standaardveldentheorie is het Higgsveldverhaal:\newline
\centerline{$V(H) = -\mu^2 |H|^2 + \lambda |H|^4$}
waar $\mu^2$ een negatieve massaterm is die spontane symmetriebreking uitlokt.

In VAM komt de breking voort uit reële æthercompressie, waardoor de fysische oorsprong van $\phi$ niet willekeurig is maar voortkomt uit spanningsbalans:
\[
    \frac{dV}{d\phi} = 0 \Rightarrow \text{drukkracht in evenwicht met wervelstructuur}
\]

\subsection*{Lagrangiandichtheid voor het æthercondensaat}
De totale bijdrage aan de Lagrangian voor het spanningsveld luidt:
\[
    \mathcal{L}_{\phi} = -|D_\mu \phi|^2 - V(\phi)
\]
Hierin wordt $D_\mu$ geïnterpreteerd als afgeleide langs de richting van de spanningsverandering in het wervelveld (mogelijk gekoppeld aan $V_\mu$).

Deze term vertegenwoordigt:\newline
• De interne elasticiteit van het æther,\newline
• De manier waarop topologische verstoringen de spanningsverdeling verschuiven,\newline
• En het mechanisme waardoor massatermen voortkomen uit lokale ætherinteractie.

\subsection*{Opmerking over simulatie}
Deze veldvorm en zijn dynamica zijn numeriek simuleerbaar binnen bestaande systemen van klassieke ætherfluïda (bv. op basis van compressiepotentialen), wat experimentele validatie binnen bereik brengt.
    \input{5}
    %! Author = omar.iskandarani
%! Date = 5/20/2025
\section{Tijdsklokwerking in Wervelknopen}

In het Vortex \AE ther Model worden stabiele knopen opgevat als de fundamentele bouwstenen van materie. Door hun interne swirl—de tangentiële rotatie \( C_e \) rond een kernstraal \( r_c \)—veroorzaken zij een asymmetrische spanningsverdeling in de omringende \ae ther. Deze asymmetrie resulteert in een **axiale stroming langs de kern**, die functioneel overeenkomt met een voortbewegende tijdsdraad. Hoewel er geen geometrische schroefdraad aanwezig is, gedraagt het systeem zich **alsof de kernstructuur een schroefwerking uitvoert** op de omringende fluïdum.

\paragraph{Kosmische swirloriëntatie.}
Net als magnetische domeinen vertonen wervelknopen een voorkeur voor een globale swirlrichting. In een universum met een dominante draairichting zou het omgekeerd draaien van een knoop (bijvoorbeeld antimaterie?) alleen stabiel zijn in isolatie. Dit zou verklaren waarom antimaterie zeldzaam is, en waarom tijdsoriëntatie consistent is in macroscopische systemen.

\paragraph{Swirl als tijdsdrager.}
De draaiing van de knoopkern induceert een centrale stroom \( \vec{v}_\text{tijd} \) die volgens het VAM-model direct overeenkomt met lokaal tijdsverloop:
\[
    dt_{\text{lokaal}} \propto \frac{dr}{\vec{v} \cdot \vec{\omega}}
\]
De koppeling tussen swirl (\( C_e \)) en de axiale afvoer van heliciteit bepaalt daarmee hoe snel tijd verstrijkt nabij een knoop.

\paragraph{Collectieve tijdsdraadnetwerken.}
Wervelknopen neigen ertoe zich te groeperen langs swirlstromingen—vergelijkbaar met magnetische veldlijnen die ijzervijlsel ordenen. Rond massa’s kunnen zo netwerken van tijdsdraden ontstaan, wat een natuurlijk verklaringsmodel biedt voor:
\begin{itemize}
    \item zwaartekracht als concentratie van swirlstromen;
    \item lokale tijdsdilatatie (zoals bij planeten en sterren);
    \item richting van kosmologische tijdsevolutie.
\end{itemize}

Deze emergente klokwerking is een van de meest fundamentele aspecten van VAM. Het brengt massa, tijd en richting onder in één mechanisme dat volledig afleidbaar is uit werveldynamica, zonder beroep op abstracte ruimte-tijdcurvaturen.

    \bibliographystyle{unsrt}
    \printbibliography

\end{document}
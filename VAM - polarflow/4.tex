\section*{4. Druk- en spanningspotentiaal van \ae thercondensaat}

De vierde bijdrage aan de VAM-Lagrangian betreft de beschrijving van drukspanning en evenwichtstoestanden in het \ae ther. In analoge zin met het Higgsmechanisme wordt dit gemodelleerd via een scalair veld $\phi$ dat de lokale toestand van het \ae ther representeert.

\subsection*{Veldinterpretatie}
Het veld $\phi$ meet de verstoring van het \ae thervolume als gevolg van een wervelknoop. Bij sterke swirl $C_e$ en hoge vorticiteit $\omega$ zal de lokale druk dalen (Bernoulli-effect), wat zich uit in een verandering van het evenwichtspunt van het \ae ther:
\[
    P_{\text{lokaal}} < P_\infty \Rightarrow \phi \neq 0
\]

\subsection*{Potentiaalvorm en afleiding}
De \ae thertoestand wordt beschreven door een klassieke potentiaal van de vorm:
\[
    V(\phi) = -\frac{F_{\text{max}}}{r_c} |\phi|^2 + \lambda |\phi|^4
\]

Hierin:
- $\frac{F_{\text{max}}}{r_c}$ is de maximale compressieve spanningsdichtheid van de \ae ther,
- $\lambda$ bepaalt de stijfheid van het systeem tegen overspanning.

De minima van deze potentiaal liggen bij:
\[
    |\phi| = \sqrt{\frac{F_{\text{max}}}{2 \lambda r_c}}
\]
Dit is een stabiele toestand waarin het \ae ther zich herstructureert rond een stabiele knoopconfiguratie.

\subsection*{Vergelijking met Higgsveld}
In standaardveldentheorie is het Higgsveldverhaal:\newline
\centerline{$V(H) = -\mu^2 |H|^2 + \lambda |H|^4$}
waar $\mu^2$ een negatieve massaterm is die spontane symmetriebreking uitlokt.

In VAM komt de breking voort uit reële \ae thercompressie, waardoor de fysische oorsprong van $\phi$ niet willekeurig is maar voortkomt uit spanningsbalans:
\[
    \frac{dV}{d\phi} = 0 \Rightarrow \text{drukkracht in evenwicht met wervelstructuur}
\]

\subsection*{Lagrangiandichtheid voor het \ae thercondensaat}
De totale bijdrage aan de Lagrangian voor het spanningsveld luidt:
\[
    \mathcal{L}_{\phi} = -|D_\mu \phi|^2 - V(\phi)
\]
Hierin wordt $D_\mu$ geïnterpreteerd als afgeleide langs de richting van de spanningsverandering in het wervelveld (mogelijk gekoppeld aan $V_\mu$).

Deze term vertegenwoordigt:\newline
• De interne elasticiteit van het \ae ther,\newline
• De manier waarop topologische verstoringen de spanningsverdeling verschuiven,\newline
• En het mechanisme waardoor massatermen voortkomen uit lokale \ae therinteractie.

\subsection*{Opmerking over simulatie}
Deze veldvorm en zijn dynamica zijn numeriek simuleerbaar binnen bestaande systemen van klassieke \ae therfluïda (bv. op basis van compressiepotentialen), wat experimentele validatie binnen bereik brengt.

\section*{5. Mapping van SU(3) \texorpdfstring{$\times$}{x} SU(2) \texorpdfstring{$\times$}{x} U(1) naar VAM-wervelgroepen}

De standaardmodel-Lagrangian is gebaseerd op de gaugegroep:
\[
    SU(3)_C \times SU(2)_L \times U(1)_Y
\]
die de kleurinteractie (QCD), de zwakke interactie en elektromagnetisme beschrijven via hun bijbehorende vectorvelden. In het Vortex \AE ther Model (VAM) bestaan geen abstracte ruimtetijdsymmetrieën — alle krachten en interacties moeten herleid worden tot wervelstructuren en topologische stromingen in een 3D Euclidische \ae ther.

\subsection*{5.1 $U(1)_Y$: Swirlrichting als hyperlading}
De eenvoudigste symmetrie, $U(1)$, correspondeert met het behoud van een fase of draairichting. In VAM krijgt dit een directe fysieke betekenis:
\begin{itemize}
    \item \textbf{Fysische interpretatie:} een rechtlijnige swirl (circulair maar niet geknoopt) in de \ae ther representeert een uniforme draairichting.
    \item \textbf{Lading:} hyperlading $Y$ is dan de chirale swirlrichting (rechts- of linkshandig) binnen een axiaal symmetrisch veld.
    \item \textbf{Vergelijking:} dit modelleert elektromagnetisme als macroscopische swirl zonder topologische knoop.
\end{itemize}

\subsection*{5.2 $SU(2)_L$: Chiraliteit als tweevoudige topologische swirl}
De zwakke wisselwerking is intrinsiek chirale: alleen linkshandige fermionen koppelen aan $SU(2)_L$.
\begin{itemize}
    \item \textbf{VAM-interpretatie:} linkshandige en rechtshandige wervels zijn fysiek niet equivalent — ze vertegenwoordigen swirlvelden die onder lokale compressie verschillen in draairichting bezitten.
    \item \textbf{Twee toestanden:} $SU(2)$ correspondeert met een tweedimensionale swirlrichtingruimte: bijvoorbeeld op- en neerspinnende swirl.
    \item \textbf{Veldkoppeling:} gaugevelden van $SU(2)$ worden geïnterpreteerd als transities tussen deze swirlrichtingen via knoopreconnectie.
\end{itemize}

\subsection*{5.3 $SU(3)_C$: Drievoudige vortexkleur als heliciteitsstructuur}
In het standaardmodel beschrijft $SU(3)_C$ de kleurkracht, werkend via gluonen die kleur verwisselen.
\begin{itemize}
    \item \textbf{VAM-interpretatie:} drie topologisch stabiele swirlconfiguraties (bijvoorbeeld drie orthogonale heliciteitsassen) corresponderen met de drie kleuren (rood, groen, blauw).
    \item \textbf{Gluonen:} wisselwerkingen tussen deze structuren worden geïnterpreteerd als vortexinterferentie en transities in knoopconfiguraties, zoals bij knoop-twist, splitsing of vervorming van de kern.
    \item \textbf{Begrenzing:} kleurconfinement ontstaat omdat losse swirlkleurconfiguraties energetisch instabiel zijn buiten samengestelde knopen.
\end{itemize}

\subsection*{5.4 Wiskundige groepsstructuur binnen VAM}
Hoewel VAM een strikt geometrisch-fluidum model is, blijven de symmetrieën behouden in de zin van bewaarbare toestanden:
\begin{itemize}
    \item Swirlrichting $\rightarrow$ $U(1)$-fasesymmetrie
    \item Axiale transformatie $\rightarrow$ $SU(2)$
    \item Kleurknoopbasis $\rightarrow$ $SU(3)$-structuur in 3D-heliciteit
\end{itemize}

\subsection*{Conclusie}
De gebruikelijke abstracte Lie-groepen van het standaardmodel zijn in VAM fysiek realiseerbaar als swirl-, heliciteit- en knoopstructuren in het \ae ther. Hierdoor kunnen de bekende interacties worden behouden en herleid vanuit fluïdumechanische principes, zonder terug te vallen op extra dimensies of onobserveerbare velden.

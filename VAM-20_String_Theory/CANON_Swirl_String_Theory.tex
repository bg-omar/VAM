%! Author = Omar Iskandarani
%! Title = Swirl String Theory (SST) Canon v0.3
%! Date = Aug 24, 2025
%! Affiliation = Independent Researcher, Groningen, The Netherlands
%! License = © 2025 Omar Iskandarani. All rights reserved. This manuscript is made available for academic reading and citation only. No republication, redistribution, or derivative works are permitted without explicit written permission from the author. Contact: info@omariskandarani.com
%! ORCID = 0009-0006-1686-3961
%! DOI = 10.5281/zenodo.16934536

\newcommand{\canonversion}{\textbf{v0.3.0}} % Semantic versioning: vMAJOR.MINOR.PATCH
\newcommand{\papertitle}{Swirl String Theory (SST) Canon \canonversion}
\newcommand{\paperdoi}{10.5281/zenodo.16934536}

% ==== Swirl String Theory (SST) macros ====
% Context-aware subscript symbol; uses math styles, not \scriptsize
\newcommand{\swirlarrow}{%
	\mathchoice{\mkern-2mu\scriptstyle\boldsymbol{\circlearrowleft}}%
	{\mkern-2mu\scriptstyle\boldsymbol{\circlearrowleft}}%
	{\mkern-2mu\scriptscriptstyle\boldsymbol{\circlearrowleft}}%
	{\mkern-2mu\scriptscriptstyle\boldsymbol{\circlearrowleft}}%
}
\newcommand{\swirlarrowcw}{%
	\mathchoice{\mkern-2mu\scriptstyle\boldsymbol{\circlearrowright}}%
	{\mkern-2mu\scriptstyle\boldsymbol{\circlearrowright}}%
	{\mkern-2mu\scriptscriptstyle\boldsymbol{\circlearrowright}}%
	{\mkern-2mu\scriptscriptstyle\boldsymbol{\circlearrowright}}%
}

% Canonical symbols
\newcommand{\vswirl}{\mathbf{v}_{\swirlarrow}}
\newcommand{\vswirlcw}{\mathbf{v}_{\swirlarrowcw}}
\newcommand{\SwirlClock}{S_t^{\swirlarrow}}
\newcommand{\SwirlClockcw}{S_t^{\swirlarrowcw}}
\newcommand{\omegas}{\boldsymbol{\omega}_{\swirlarrow}}  % swirl vorticity
\newcommand{\vscore}{v_s}                                % shorthand: |v_swirl| at r=rs
\newcommand{\vnorm}{\lVert \vswirl \rVert}               % swirl speed magnitude
\newcommand{\rhof}{\rho_{\!f}}                           % effective fluid density
\newcommand{\rhoE}{\rho_{\!E}}                           % swirl energy density /c^2? (we define clearly below)
\newcommand{\rhom}{\rho_{\!m}}                           % mass-equivalent density
\newcommand{\rs}{r_s}                                    % string core radius (swirl string radius)
\newcommand{\FEMmax}{F_{\mathrm{EM}}^{\max}}             % EM-like maximal force scale
\newcommand{\FGmax}{F_{\mathrm{G}}^{\max}}               % G-like maximal force scale
\newcommand{\Lam}{\Lambda}                               % Swirl Coulomb constant
\newcommand{\Om}{\Omega_{\swirlarrow}}                   % swirl angular frequency profile
\newcommand{\alpg}{\alpha_g}                             % gravitational fine-structure analogue

% Policy: the golden constant is only allowed via hyperbolic functions.
% Never write (1+\sqrt{5})/2; always use \xig=\asinh(1/2), \varphi=e^{\xig}.
\newcommand{\xig}{\operatorname{asinh}\!\left(\tfrac{1}{2}\right)} % base hyperbolic scale  "golden" constant is fundamentally hyperbolic.
\newcommand{\phig}{\exp(\xig)}                                     % golden from hyperbolic
\newcommand{\phialg}{\bigl(1+\sqrt{5}\bigr)/2}                     % algebraic echo (use sparingly)
\newcommand{\xigold}{\tfrac{3}{2}\,\xig}                           % "golden rapidity" scale

% --- Display helpers (optional) ---
\newcommand{\GoldenDeclare}{%
	\textbf{Golden (hyperbolic)}:\ \(\ln\phi=\xig\), hence \(\phi=\phig\).
	\ \emph{(Equivalently, \(\phi=\phialg\); this algebraic form is derivative.)}%
}
% --- Canonical identity (hyperbolic-only proof, algebraic as corollary) ---
\newtheorem{identity}{Identity}


%========================================================================================
% PACKAGES AND DOCUMENT CONFIGURATION
%========================================================================================
\documentclass[11pt]{article}
\usepackage{subfiles}
% vamstyle.sty
\NeedsTeXFormat{LaTeX2e}
\ProvidesPackage{vamstyle}[2025/07/01 VAM unified style]

% === Constants ===
\newcommand{\hbarVal}{\ensuremath{1.054571817 \times 10^{-34}}} % J\cdot s
\newcommand{\meVal}{\ensuremath{9.10938356 \times 10^{-31}}} % kg
\newcommand{\cVal}{\ensuremath{2.99792458 \times 10^{8}}} % m/s
\newcommand{\alphaVal}{\ensuremath{1 / 137.035999084}} % unitless
\newcommand{\alphaGVal}{\ensuremath{1.75180000 \times 10^{-45}}} % unitless
\newcommand{\reVal}{\ensuremath{2.8179403227 \times 10^{-15}}} % m
\newcommand{\rcVal}{\ensuremath{1.40897017 \times 10^{-15}}} % m
\newcommand{\vacrho}{\ensuremath{5 \times 10^{-9}}} % kg/m^3
\newcommand{\LpVal}{\ensuremath{1.61625500 \times 10^{-35}}} % m
\newcommand{\MpVal}{\ensuremath{2.17643400 \times 10^{-8}}} % kg
\newcommand{\tpVal}{\ensuremath{5.39124700 \times 10^{-44}}} % s
\newcommand{\TpVal}{\ensuremath{1.41678400 \times 10^{32}}} % K
\newcommand{\qpVal}{\ensuremath{1.87554596 \times 10^{-18}}} % C
\newcommand{\EpVal}{\ensuremath{1.95600000 \times 10^{9}}} % J
\newcommand{\eVal}{\ensuremath{1.60217663 \times 10^{-19}}} % C

% === VAM/\ae ther Specific ===
\newcommand{\CeVal}{\ensuremath{1.09384563 \times 10^{6}}} % m/s
\newcommand{\FmaxVal}{\ensuremath{29.0535070}} % N
\newcommand{\FmaxGRVal}{\ensuremath{3.02563891 \times 10^{43}}} % N
\newcommand{\gammaVal}{\ensuremath{0.005901}} % unitless
\newcommand{\GVal}{\ensuremath{6.67430000 \times 10^{-11}}} % m^3/kg/s^2
\newcommand{\hVal}{\ensuremath{6.62607015 \times 10^{-34}}} % J Hz^-1

% === Electromagnetic ===
\newcommand{\muZeroVal}{\ensuremath{1.25663706 \times 10^{-6}}}
\newcommand{\epsilonZeroVal}{\ensuremath{8.85418782 \times 10^{-12}}}
\newcommand{\ZzeroVal}{\ensuremath{3.76730313 \times 10^{2}}}

% === Atomic & Thermodynamic ===
\newcommand{\RinfVal}{\ensuremath{1.09737316 \times 10^{7}}}
\newcommand{\aZeroVal}{\ensuremath{5.29177211 \times 10^{-11}}}
\newcommand{\MeVal}{\ensuremath{9.10938370 \times 10^{-31}}}
\newcommand{\MprotonVal}{\ensuremath{1.67262192 \times 10^{-27}}}
\newcommand{\MneutronVal}{\ensuremath{1.67492750 \times 10^{-27}}}
\newcommand{\kBVal}{\ensuremath{1.38064900 \times 10^{-23}}}
\newcommand{\RVal}{\ensuremath{8.31446262}}

% === Compton, Quantum, Radiation ===
\newcommand{\fCVal}{\ensuremath{1.23558996 \times 10^{20}}}
\newcommand{\OmegaCVal}{\ensuremath{7.76344071 \times 10^{20}}}
\newcommand{\lambdaCVal}{\ensuremath{2.42631024 \times 10^{-12}}}
\newcommand{\PhiZeroVal}{\ensuremath{2.06783385 \times 10^{-15}}}
\newcommand{\phiVal}{\ensuremath{1.61803399}}
\newcommand{\eVVal}{\ensuremath{1.60217663 \times 10^{-19}}}
\newcommand{\GFVal}{\ensuremath{1.16637870 \times 10^{-5}}}
\newcommand{\lambdaProtonVal}{\ensuremath{1.32140986 \times 10^{-15}}}
\newcommand{\ERinfVal}{\ensuremath{2.17987236 \times 10^{-18}}}
\newcommand{\fRinfVal}{\ensuremath{3.28984196 \times 10^{15}}}
\newcommand{\sigmaSBVal}{\ensuremath{5.67037442 \times 10^{-8}}}
\newcommand{\WienVal}{\ensuremath{2.89777196 \times 10^{-3}}}
\newcommand{\kEVal}{\ensuremath{8.98755179 \times 10^{9}}}

% === \ae ther Densities ===
\newcommand{\rhoMass}{\rho_\text{\ae}^{(\text{mass})}}
\newcommand{\rhoMassVal}{\ensuremath{3.89343583 \times 10^{18}}}
\newcommand{\rhoEnergy}{\rho_\text{\ae}^{(\text{energy})}}
\newcommand{\rhoEnergyVal}{\ensuremath{3.49924562 \times 10^{35}}}
\newcommand{\rhoFluid}{\rho_\text{\ae}^{(\text{fluid})}}
\newcommand{\rhoFluidVal}{\ensuremath{7.00000000 \times 10^{-7}}}

% === Draft Options ===
\newif\ifvamdraft
% \vamdrafttrue
\ifvamdraft
\RequirePackage{showframe}
\fi

% === Load Once ===
\RequirePackage{ifthen}
\newboolean{vamstyleloaded}
\ifthenelse{\boolean{vamstyleloaded}}{}{\setboolean{vamstyleloaded}{true}

% === Page ===
\RequirePackage[a4paper, margin=2.5cm]{geometry}

% === Fonts ===
\RequirePackage[T1]{fontenc}
\RequirePackage[utf8]{inputenc}
\RequirePackage[english]{babel}
\RequirePackage{textgreek}
\RequirePackage{mathpazo}
\RequirePackage[scaled=0.95]{inconsolata}
\RequirePackage{helvet}

% === Math ===
\RequirePackage{amsmath, amssymb, mathrsfs, physics}
\RequirePackage{siunitx}
\sisetup{per-mode=symbol}

% === Tables ===
\RequirePackage{graphicx, float, booktabs}
\RequirePackage{array, tabularx, multirow, makecell}
\newcolumntype{Y}{>{\centering\arraybackslash}X}
\newenvironment{tighttable}[1][]{\begin{table}[H]\centering\renewcommand{\arraystretch}{1.3}\begin{tabularx}{\textwidth}{#1}}{\end{tabularx}\end{table}}
\RequirePackage{etoolbox}
\newcommand{\fitbox}[2][\linewidth]{\makebox[#1]{\resizebox{#1}{!}{#2}}}

% === Graphics ===
\RequirePackage{tikz}
\usetikzlibrary{3d, calc, arrows.meta, positioning}
\RequirePackage{pgfplots}
\pgfplotsset{compat=1.18}
\RequirePackage{xcolor}

% === Code ===
\RequirePackage{listings}
\lstset{basicstyle=\ttfamily\footnotesize, breaklines=true}

% === Theorems ===
\newtheorem{theorem}{Theorem}[section]
\newtheorem{lemma}[theorem]{Lemma}

% === TOC ===
\RequirePackage{tocloft}
\setcounter{tocdepth}{2}
\renewcommand{\cftsecfont}{\bfseries}
\renewcommand{\cftsubsecfont}{\itshape}
\renewcommand{\cftsecleader}{\cftdotfill{.}}
\renewcommand{\contentsname}{\centering \Huge\textbf{Contents}}

% === Sections ===
\RequirePackage{sectsty}
\sectionfont{\Large\bfseries\sffamily}
\subsectionfont{\large\bfseries\sffamily}

% === Bibliography ===
\RequirePackage[numbers]{natbib}

% === Links ===
\RequirePackage{hyperref}
\hypersetup{
    colorlinks=true,
    linkcolor=blue,
    citecolor=blue,
    urlcolor=blue,
    pdftitle={The Vortex \AE ther Model},
    pdfauthor={Omar Iskandarani},
    pdfkeywords={vorticity, gravity, \ae ther, fluid dynamics, time dilation, VAM}
}
\urlstyle{same}
\RequirePackage{bookmark}

% === Misc ===
\RequirePackage[none]{hyphenat}
\sloppy
\RequirePackage{empheq}
\RequirePackage[most]{tcolorbox}
\newtcolorbox{eqbox}{colback=blue!5!white, colframe=blue!75!black, boxrule=0.6pt, arc=4pt, left=6pt, right=6pt, top=4pt, bottom=4pt}
\RequirePackage{titling}
\RequirePackage{amsfonts}
\RequirePackage{titlesec}
\RequirePackage{enumitem}

\AtBeginDocument{\RenewCommandCopy\qty\SI}

\pretitle{\begin{center}\LARGE\bfseries}
\posttitle{\par\end{center}\vskip 0.5em}
\preauthor{\begin{center}\large}
\postauthor{\end{center}}
\predate{\begin{center}\small}
\postdate{\end{center}}

\endinput
}
% vamappendixsetup.sty

\newcommand{\titlepageOpen}{
  \begin{titlepage}
  \thispagestyle{empty}
  \centering
  {\Huge\bfseries \papertitle \par}
  \vspace{1cm}
  {\Large\itshape\textbf{Omar Iskandarani}\textsuperscript{\textbf{*}} \par}
  \vspace{0.5cm}
  {\large \today \par}
  \vspace{0.5cm}
}

% here comes abstract
\newcommand{\titlepageClose}{
  \vfill
  \null
  \begin{picture}(0,0)
  % Adjust position: (x,y) = (left, bottom)
  \put(-200,-40){  % Shift 75pt left, 40pt down
    \begin{minipage}[b]{0.7\textwidth}
    \footnotesize % One step bigger than \tiny
    \renewcommand{\arraystretch}{1.0}
    \noindent\rule{\textwidth}{0.4pt} \\[0.5em]  % ← horizontal line
    \textsuperscript{\textbf{*}}Independent Researcher, Groningen, The Netherlands \\
    Email: \texttt{info@omariskandarani.com} \\
    ORCID: \texttt{\href{https://orcid.org/0009-0006-1686-3961}{0009-0006-1686-3961}} \\
    DOI: \href{https://doi.org/\paperdoi}{\paperdoi} \\
    License: CC-BY 4.0 International \\
    \end{minipage}
  }
  \end{picture}
  \end{titlepage}
}
\usepackage[margin=1in]{geometry}
\usepackage{amsmath,amssymb,amsfonts}
\usepackage{tcolorbox}
\usetikzlibrary{knots,intersections,decorations.pathreplacing,3d,calc,arrows.meta,positioning,decorations.pathmorphing}
\usepackage{pgfmath}
\usepackage{pgfplots}
\pgfplotsset{compat=1.18}
\usepackage{ulem}


% ==== Packages ====
\usepackage[T1]{fontenc}
\usepackage{lmodern}
\usepackage{microtype}

\geometry{margin=1in}
\usepackage{ bm, mathtools}
\usepackage{siunitx}
\sisetup{per-mode=symbol,round-mode=figures,round-precision=6}
\usepackage{physics}
\usepackage{upgreek}
\usepackage{graphicx}
\usepackage{booktabs}
\usepackage{hyperref}
\hypersetup{colorlinks=true, linkcolor=blue!60!black, citecolor=blue!60!black, urlcolor=blue!60!black}

%========================================================================================
% DOCUMENT START
%========================================================================================
\begin{document}

%========================================================================================
% TITLE PAGE
%========================================================================================

		\titlepageOpen
		\begin{abstract}

        This Canon is the single source of truth for \emph{Swirl String Theory (SST)}: definitions, constants, boxed master equations, and notational conventions. It consolidates core structure up to the previous baseline (\S\S1--12.6) and \emph{promotes five results to canonical status}: (i) Swirl Coulomb constant $\Lam$ and hydrogen soft-core, (ii) circulation--metric corollary (frame-dragging analogue), (iii) corrected swirl time-rate (Swirl Clock) law, (iv) Kelvin-compatible swirl Hamiltonian density, and (v) swirl pressure law (Euler corollary).

		\paragraph{Versioning} Semantic versions: vMAJOR.MINOR.PATCH. This file: \canonversion.
		Every paper/derivation must state the Canon version it depends on.

    \paragraph{Core Postulates (SST)}
    \begin{enumerate}
        \item \textbf{Swirl medium:} Physics is formulated on $\mathbb{R}^3$ with absolute reference time. Dynamics occur in an incompressible, inviscid \emph{swirl condensate}, which plays the role of a universal substrate.
        \item \textbf{Strings as swirls:} Particles and excitations correspond to closed, possibly linked or knotted \emph{swirl strings} with quantized circulation.
        \item \textbf{String-induced gravitation:} Macroscopic attraction emerges from coherent swirl fields and swirl-pressure gradients. The effective gravitational coupling $G_{\text{swirl}}$ is fixed by canonical constants.
        \item \textbf{Swirl clocks:} Local proper-time rate depends on tangential swirl velocity. Higher swirl density slows local clocks relative to the asymptotic frame.
        \item \textbf{Quantization from topology and circulation:} Discrete quantum numbers track directly to linking, writhe, twist, and circulation quantization of swirl strings.
        \item \textbf{Taxonomy:} Unknotted excitations behave as bosonic string modes; chiral hyperbolic knots map to quarks; torus knots map to leptons (taxonomy documented separately).
    \end{enumerate}
		\footnotesize{Hydrodynamic analogy only; no mechanical “æther” is assumed in the mainstream presentation.}
	\end{abstract}
		\titlepageClose

%================================================
    \section{Swirl Quantization Principle}
%================================================

    \subsection{Local Circulation Quantization}
    The circulation of the swirl velocity field around any closed loop is quantized:
    \begin{equation}
        \Gamma = \oint \vec{v}_{\text{swirl}} \cdot d\vec{\ell}
        = n \kappa,
        \qquad n \in \mathbb{Z},
    \end{equation}
    with circulation quantum
    \begin{equation}
        \kappa = \frac{h}{m_\text{eff}}.
    \end{equation}
    This parallels the Onsager--Feynman quantization condition in superfluids, but here is elevated to a fundamental postulate of the swirl condensate.

    \subsection{Topological Quantization}
    Closed swirl filaments may form knots and links. Each topological class corresponds to a discrete excitation state:
    \begin{equation}
        \mathcal{H}_\text{swirl}
        = \{ \text{trefoil}, \; \text{figure-eight}, \; \text{Hopf link}, \dots \}.
    \end{equation}
    Quantum numbers such as mass, charge, and chirality are encoded in the knot invariants (linking, twist, writhe).

    \subsection{Unified Principle}
    We define \emph{Swirl Quantization} as the joint discreteness of circulation and topology:
    \[
        \text{Swirl Quantization} \;\equiv\;
        \Big( \Gamma = n\kappa \Big)
        \;\; \cup \;\;
        \Big( \text{Knot spectrum } \mathcal{H}_\text{swirl} \Big).
    \]
    This principle underlies both the discrete particle spectrum and the emergence of fundamental interactions in Swirl String Theory.

    \begin{center}
        \begin{tabular}{|c|c|}
            \hline
            \textbf{Quantum Mechanics} & \textbf{Swirl String Theory} \\
            \hline
            Canonical Quantization: & Swirl Quantization Principle: \\
            $[x, p] = i \hbar$ & $\Gamma = n \kappa, \quad n \in \mathbb{Z}$ \\[6pt]
            & $\mathcal{H}_\text{swirl} =
            \{ \text{trefoil}, \; \text{figure-eight}, \; \text{Hopf link}, \dots \}$ \\
            \hline
            Discreteness arises from & Discreteness arises from \\
            operator commutators & circulation integrals and topology \\
            \hline
            Particles = eigenstates of & Particles = knotted swirl states with \\
            Hamiltonian operator & quantized circulation and invariants \\
            \hline
        \end{tabular}
    \end{center}


%===============================================================
%  Chronos–Kelvin Invariant (Canonical)
%===============================================================
    \section*{Chronos–Kelvin Invariant (Canonical)}
    \addcontentsline{toc}{section}{Chronos–Kelvin Invariant (Canonical)}
    \label{sec:chronos_kelvin}

    \paragraph{Setting.}
    Consider a thin, material swirl loop (nearly solid–body core) of instantaneous material radius
    $R(t_{\ae})$ convected by an incompressible, inviscid medium. Let $\omega:=\|\boldsymbol{\omega}\|$ denote the
    vorticity magnitude on the loop and $r_s$ the canonical string radius. The local Swirl Clock is
    \begin{equation}
        S_t \;\equiv\; \frac{dt_{\text{local}}}{dt_\infty}
        \;=\;
        \sqrt{\,1-\frac{v_t^{\,2}}{c^2}\,}
        \;=\;
        \sqrt{\,1-\frac{\omega^2 r_s^2}{c^2}\,},\qquad v_t:=\omega r_s .
        \label{eq:SwirlClock-def}
    \end{equation}
    Material derivatives are taken with respect to absolute Æther-Time:
    $\displaystyle \frac{D}{Dt_{\ae}}:=\frac{\partial}{\partial t_{\ae}}+\mathbf{v}\!\cdot\!\nabla$.

    \begin{theorem}[Chronos–Kelvin Invariant]
        For any such loop without reconnection or source terms, Kelvin’s theorem implies the
        material invariant
        \begin{equation}
            \boxed{\;
            \frac{D}{Dt_{\ae}}\!\left(R^2\,\omega\right)=0
            \;}
            \quad\Longleftrightarrow\quad
            \boxed{\;
            \frac{D}{Dt_{\ae}}\!\left(
                                    \frac{c}{r_s}\,R^2 \sqrt{1-S_t^2}
            \right)=0
            \;}
            \label{eq:CK}
        \end{equation}
    \end{theorem}

    \paragraph{Proof (one line).}
    Kelvin’s circulation theorem for an inviscid, barotropic medium gives
    $\displaystyle \frac{D}{Dt_{\ae}}\Gamma=0$ with $\Gamma=\oint \mathbf{v}\cdot d\boldsymbol{\ell}$ \cite{Helmholtz1858,Kelvin1869,Batchelor1967}.
    For a nearly solid–body core, $\Gamma=2\pi R\, v_t=2\pi R^2 \omega$; hence
    $\displaystyle \frac{D}{Dt_{\ae}}(R^2\omega)=0$.
    Using \eqref{eq:SwirlClock-def}, $R^2\omega=\tfrac{c}{r_s} R^2\sqrt{1-S_t^2}$, which yields \eqref{eq:CK}. \qed

    \paragraph{Dimensional consistency.}
    $[R^2\omega]=\text{m}^2\text{s}^{-1}$; and
    $\big[\tfrac{c}{r_s}R^2\sqrt{1-S_t^2}\big]=\text{s}^{-1}\cdot\text{m}^2=\text{m}^2\text{s}^{-1}$.

    \paragraph{Clock–radius transport law (corollary).}
    From $R^2\omega=\text{const}$ and \eqref{eq:SwirlClock-def},
    \begin{equation}
        \frac{dS_t}{dt_{\ae}} \;=\; \frac{2(1-S_t^2)}{S_t}\,\frac{1}{R}\frac{dR}{dt_{\ae}} .
        \label{eq:clock-radius-ode}
    \end{equation}
    Hence expansion ($dR/dt_{\ae}>0$) pushes $S_t\!\to\!1$ (clocks speed up), while contraction
    slows clocks ($S_t\!\downarrow$), preserving \eqref{eq:CK}.

    \paragraph{PV analogue (optional).}
    With a uniform background rotation $\Omega_{\text{bg}}$ and column thickness $H$,
    the Ertel/PV structure gives the SST counterpart
    \begin{equation}
        \frac{D}{Dt_{\ae}}
        \left(
            \frac{\omega+\Omega_{\text{bg}}}{H}
        \right)=0,
        \label{eq:PV-analogue}
    \end{equation}
    the standard potential-vorticity conservation rewritten in SST terms \cite{Ertel1942,Batchelor1967}.

    \paragraph{Conditions (Canon).}
    Incompressible, inviscid medium; barotropic swirl pressure; material loop without reconnection or
    external injection; absolute Æther-Time parametrization. These are the same hypotheses under which
    Kelvin/Helmholtz invariants hold.

    \paragraph{Limits.}
    Weak-swirl ($\omega r_s\!\ll\!c$): $S_t\simeq 1-\tfrac{1}{2}(\omega r_s/c)^2$ and \eqref{eq:CK} reduces to the
    classical $R^2\omega=\text{const}$. Core on-axis limit: $v_t\to\vswirl$ gives
    $S_t\to \sqrt{1-(\vswirl/c)^2}$, keeping \eqref{eq:CK} valid.


%================================================
    \section{Canonical Constants and Symbols}
%================================================

    \subsection*{Primary SST constants (SI unless noted)}
    \begin{itemize}
        \item Swirl speed scale (core): $\vnorm = \num{1.09384563e6}\ \si{m.s^{-1}}$ (evaluate at $r=\rs$).
        \item String (core) radius: $\rs = \num{1.40897017e-15}\ \si{m}$.
        \item Effective fluid density: $\rhof = \num{7.0e-7}\ \si{kg.m^{-3}}$.
        \item Mass-equivalent density: $\rhom = \num{3.8934358266918687e18}\ \si{kg.m^{-3}}$. % used in \Lambda
        \item EM-like maximal force: $\FEMmax = \num{2.9053507e1}\ \si{N}$.
        \item Gravitational maximal force (reference scale): $\FGmax = \num{3.02563e43}\ \si{N}$.
        \item Golden ratio: $\varphi = (1+\sqrt{5})/2 \approx \num{1.61803398875}$.
    \end{itemize}

    \subsection*{Universal constants}
    \begin{itemize}
        \item $c=\num{299792458}\ \si{m.s^{-1}}$, \quad $t_p=\num{5.391247e-44}\ \si{s}$.
        \item Fine-structure constant (identified): $\alpha \approx \num{7.2973525643e-3}$.
    \end{itemize}

    \paragraph{Effective densities (mainstream field-theory style).}
    \[
        \rho_f \equiv \text{effective fluid density},
	\]
	\[
        \rho_E \equiv \tfrac12 \rho_f\, \vnorm^2\quad(\text{swirl energy density}),\qquad
        \rho_m \equiv \rho_E/c^2\quad(\text{mass-equivalent density}).
    \]

    \textbf{Note:} The local Python \texttt{constants\_dict} used in simulations must mirror these values exactly; papers should quote the Canon version.


%================================================
% Canon Governance & Status Taxonomy
%================================================
    \section*{Canon Governance (Binding)}

    \subsection*{Definitions}
    \paragraph{Formal System.}
    Let \(\mathcal{S} = (\mathcal{P},\mathcal{D},\mathcal{R})\) denote the SST formal system:
    postulates \(\mathcal{P}\), definitions \(\mathcal{D}\), and admissible inference rules \(\mathcal{R}\)
    (variational derivation, Noether, dimensional analysis, asymptotic matching, etc.).

    \paragraph{Canonical statement.}
    A statement \(X\) is \emph{canonical} iff \(X\) is a theorem or identity provable in \(\mathcal{S}\):
    \[
        \mathcal{P},\mathcal{D}\ \vdash_{\mathcal{R}}\ X,
    \]
    and \(X\) is consistent with all previously accepted canonical items in the current major version.

    \paragraph{Empirical statement.}
    A statement \(Y\) is \emph{empirical} iff it asserts a measured value, fit, or protocol:
    \[
        Y \equiv \text{“observable } \mathcal{O} \text{ has value } \hat{o} \pm \delta o \text{ under procedure } \Pi\text{.”}
    \]
    Empirical items calibrate symbols (e.g., $\vscore$, $\rs$, $\rhof$) but are not premises in proofs.

    \subsection*{Status Classes}
    \begin{itemize}
        \item \textbf{Axiom / Postulate (Canonical).} Primitive assumptions of SST (e.g., incompressible, inviscid medium; absolute time; Euclidean space).
        \item \textbf{Definition (Canonical).} Introduces symbols by construction (e.g., swirl Coulomb constant \(\Lambda\) by surface-pressure integral).
        \item \textbf{Theorem / Corollary (Canonical).} Proven consequences (e.g., Euler–SST radial balance; Swirl Clocks time-scaling).
        \item \textbf{Constitutive Model (Canonical if derived; otherwise Semi-empirical).} Ties fields/observables; canonical when deduced from \(\mathcal{P},\mathcal{D}\).
        \item \textbf{Calibration (Empirical).} Recommended numerical values with uncertainties for canonical symbols.
        \item \textbf{Research Track (Non-canonical).} Conjectures or alternatives pending proof or axiomatization.
    \end{itemize}

    \subsection*{Canonicality Tests (all required)}
    \begin{enumerate}
        \item \textbf{Derivability} from \(\mathcal{P},\mathcal{D}\) via \(\mathcal{R}\).
        \item \textbf{Dimensional Consistency} (SI throughout; correct limits).
        \item \textbf{Symmetry Compliance} (Galilean + absolute time; foliation; incompressibility).
        \item \textbf{Recovery Limits} (Newtonian gravity, Coulomb/Bohr, linear waves).
        \item \textbf{Non-Contradiction} with accepted canonical theorems.
        \item \textbf{Parameter Discipline} (no ad-hoc fits).
    \end{enumerate}

    \subsection*{Examples (from current Canon)}
    \begin{itemize}
        \item \(\displaystyle \textit{Canonical (Definition):}\quad \Lambda \equiv \int_{S_r^2} p_{\text{swirl}}\,r^2\,d\Omega.\)
        \item \(\displaystyle \textit{Canonical (Theorem):}\quad \frac{1}{\rhof}\frac{dp_{\text{swirl}}}{dr}=\frac{v_\theta(r)^2}{r}\) for steady, azimuthal drift (Euler balance).
        \item \(\displaystyle \textit{Empirical (Calibration):}\quad \vscore=1.09384563\times10^{6}\,\mathrm{m\,s^{-1}}\) with procedure \(f\Delta x\).
        \item \(\displaystyle \textit{Consistency Check (Not a premise):}\) Hydrogen soft-core reproduces \(a_0,E_1\); validates choices but remains a check, not an axiom.
    \end{itemize}

    %! Canonical Scope and Rationale for SST
    \section*{What is Canonical in SST—and Why}

    \paragraph{[Postulate] Incompressible, inviscid medium with absolute time and Euclidean space.}
    \(\nabla\!\cdot\!\vswirl=0,\ \nu=0.\)
    This fixes the kinematic arena and legal inference rules.

    \paragraph{[Definition] Vorticity, circulation, helicity.}
    \(\omegas=\nabla\times \vswirl,\quad \Gamma=\oint \vswirl\!\cdot d\boldsymbol{\ell},\quad h=\vswirl\!\cdot\!\omegas,\ H=\int h\,dV.\)
    Classical constructs canonized as primary SST kinematic invariants.

    \paragraph{[Theorem] Kelvin/vorticity transport/helicity invariants.}
    For inviscid, barotropic flow:
    \[
        \frac{d\Gamma}{dt}=0,\qquad
        \pdv{\omegas}{t}=\nabla\times(\vswirl\times\omegas),\qquad
        \text{$H$ invariant up to reconnections}.
    \]

    \paragraph{[Definition] Swirl Coulomb constant \(\Lambda\).}
    \[
        \boxed{\ \Lambda \equiv \int_{S_r^2} p_{\text{swirl}}(r)\, r^2\, d\Omega\ } \quad\Rightarrow\quad [\Lambda]=\mathrm{J\,m}=\mathrm{N\,m^2}.
    \]
    In SST Canon this evaluates symbolically to \( \Lambda=4\pi \rhom\, \vscore^{\,2}\, \rs^4\).

    \paragraph{[Theorem] Hydrogen soft-core potential and Coulomb recovery.}
    \[
        V_{\text{SST}}(r)=-\frac{\Lambda}{\sqrt{r^2+\rs^2}}
        \;\xrightarrow{r\gg \rs}\;
        -\frac{\Lambda}{r},
    \]
    yielding Bohr scalings
    \(a_0=\hbar^2/(\mu\Lambda)\), \(E_n=-\mu\Lambda^2/(2\hbar^2 n^2)\).

    \paragraph{[Theorem] Euler–SST radial balance (swirl pressure law).}
    For steady, purely azimuthal drift \(v_\theta(r)\),
    \[
        0=-\frac{1}{\rhof}\frac{dp_{\text{swirl}}}{dr}+\frac{v_\theta(r)^2}{r}
        \quad\Rightarrow\quad
        \boxed{\ \frac{1}{\rhof}\frac{dp_{\text{swirl}}}{dr}=\frac{v_\theta(r)^2}{r}\ }.
    \]
    For flat curves \(v_\theta\to v_0\): \(p_{\text{swirl}}(r)=p_0+\rhof v_0^2 \ln(r/r_0)\).

    \paragraph{[Definition \(\to\) Corollary] Effective swirl line element (analogue-metric form).}
    In \((t,r,\theta,z)\) with azimuthal drift \(v_\theta(r)\),
    \[
        ds^2=-(c^2-v_\theta^2)\,dt^2+2\,v_\theta r\,d\theta\,dt+dr^2+r^2d\theta^2+dz^2,
    \]
    co-rotating to \(ds^2=-c^2(1-v_\theta^2/c^2)dt^2+\cdots\), giving the Swirl Clock factor
    \(\displaystyle \frac{dt_{\text{local}}}{dt_\infty}=\sqrt{1-\frac{v_\theta^2}{c^2}}\).

    \paragraph{[Definition] SST Hamiltonian density (Kelvin-compatible).}
    \[
        \mathcal{H}_{\text{SST}}=\tfrac12\rhof\,\|\vswirl\|^2+\tfrac12\rhof\,\rs^2\|\omegas\|^2+\lambda(\nabla\!\cdot\!\vswirl).
    \]

    \subsection*{Empirical Calibrations (not premises, but binding numerically)}
    \begin{itemize}
        \item \([{\rm Empirical}]\) \(\vscore = 1.09384563\times 10^6\,\mathrm{m\,s^{-1}}\).
        \item \([{\rm Empirical}]\) \(\rs = 1.40897017\times 10^{-15}\,\mathrm{m}\).
        \item \([{\rm Empirical}]\) \(\rhom = 3.8934358266918687\times 10^{18}\,\mathrm{kg\,m^{-3}}\).
    \end{itemize}

    \subsection*{Non-Canonical (Research Track)}
    Blackbody via swirl temperature, EM/SST minimal coupling, etc., remain conjectural until proven under \(\mathcal{S}\).

    \subsection*{Consistency \& Dimension Checks (illustrative)}
    \[
        [\Lambda]=[\rhom][\vscore^2][\rs^4]
        =\frac{\mathrm{kg}}{\mathrm{m^3}}\cdot\frac{\mathrm{m^2}}{\mathrm{s^2}}\cdot\mathrm{m^4}
        =\frac{\mathrm{kg\,m^3}}{\mathrm{s^2}}
        =\mathrm{J\,m}.
    \]
    Soft-core Coulomb recovery: \(V_{\text{SST}}(r)\to -\Lambda/r\) as \(r/\rs\to\infty\).



%================================================
% Canon §X: Coarse-Graining Strings → 3D Effective Density
%================================================
    \section{Canonical Coarse–Graining of \(\rhof\) from a Swirl–String Bath}
    \label{sec:canon_rhof_from_strings}

    \paragraph{Scope.}
    The medium is modeled as an incompressible, inviscid fluid populated by thin \emph{swirl strings}. We derive the bulk effective fluid density \(\rhof\) via coarse–graining of line–supported mass and vorticity, relying on Euler kinematics and Kelvin–Helmholtz invariants.

    \subsection{Axioms and Definitions}
    A representative string carries:
    \begin{align}
        \text{(D1)}\quad
        \mu_\ast &\;\equiv\; \rhom\,\pi \rs^{\,2}
        \quad\;[\mathrm{kg/m}], \\[2mm]
        \text{(D2)}\quad
        \Gamma_\ast &\;\equiv\; \oint \vswirl\!\cdot\! d\boldsymbol{\ell}
        \;\simeq\; \kappa_\Gamma\, \rs\, \vscore,
        \qquad \kappa_\Gamma=2\pi \;\; \text{(near–solid–body core)}.
    \end{align}
    Let
    \(
    \nu \equiv N_{\text{str}}/A \ [\mathrm{m^{-2}}]
    \)
    be the areal string density. Then:
    \begin{align}
        \text{(C1)}\quad
        \rhof &= \mu_\ast\,\nu, \\[2mm]
        \text{(C2)}\quad
        \langle \omegas\rangle &= \Gamma_\ast\,\nu\,\hat{\mathbf{t}}_{\text{avg}}
        \ \Rightarrow\  |\langle\omega_s\rangle|=\Gamma_\ast\,\nu.
    \end{align}

    \subsection{First–Principles Derivation}
    Combining (C1)–(C2):
    \begin{equation}
        \boxed{\;
        \rhof
        \;=\; \mu_\ast\,\frac{\langle\omega_s\rangle}{\Gamma_\ast}
        \;=\; \frac{\rhom\,\pi \rs^{\,2}}{\kappa_\Gamma \rs \vscore}\,\langle\omega_s\rangle
        \;=\; \frac{\rhom\,\rs}{2\,\vscore}\,\langle\omega_s\rangle
        \;}
        \quad (\kappa_\Gamma=2\pi).
        \label{eq:rhof_from_omega}
    \end{equation}
    For uniform solid–body rotation \(\Omega\), \(\langle\omega_s\rangle=2\Omega\),
    \begin{equation}
        \boxed{\;
        \rhof
        \;=\; \frac{\rhom\,\rs}{\vscore}\;\Omega
        \;}
        \quad [\mathrm{kg/m^3}].
        \label{eq:rhof_from_Omega}
    \end{equation}

    \paragraph{Energy and tension scales.}
    \[
        \boxed{\, u_{\text{swirl}}=\tfrac12\,\rhof\,\vscore^2 \,},\qquad
        \boxed{\, T_\ast = \tfrac12\,\mu_\ast\,\vscore^2 \,}.
    \]

    \subsection{Numerical Calibration (SST Canonical Constants)}
    With
    \(
    \rhom=3.8934358266918687\times10^{18}\ \mathrm{kg/m^3},\
    \rs=1.40897017\times10^{-15}\ \mathrm{m},\
    \vscore=1.09384563\times10^{6}\ \mathrm{m/s}
    \),
    one finds
    \[
        \Gamma_\ast = 2\pi \rs \vscore
        = 9.68361920\times10^{-9}\ \mathrm{m^2/s},\quad
        T_\ast = 1.45267535\times10^{1}\ \mathrm{N}.
    \]
    From \eqref{eq:rhof_from_Omega},
    \[
        \rhof = \bigl(5.01509060\times10^{-3}\bigr)\,\Omega,
    \]
    so the Canon baseline \( \rhof=7.0\times10^{-7}\ \mathrm{kg/m^3}\) occurs at
    \[
        \boxed{\ \Omega_\ast = 1.39578735\times10^{-4}\ \mathrm{s^{-1}}\ (\text{period } \approx 12.5\ \mathrm{h})\ }.
    \]

%================================================
% Master Equations (Boxed, Definitive)
%================================================
    \section{Master Equations (Boxed, Definitive)}

    \subsection{Master Energy and Mass Formula (SST)}
    \[
        \boxed{\ E_{\text{SST}}(V) = \frac{4}{\alpha\,\varphi} \left( \frac{1}{2}\,\rhof\,\vscore^{2} \right) V\ }\quad [\text{J}],
        \qquad
        \boxed{\ M_{\text{SST}}(V) = \frac{E_{\text{SST}}(V)}{c^{2}} \ }\quad [\text{kg}].
    \]
    Numerics per unit volume:
    \(
    \tfrac12\rhof \vscore^2 \approx 4.1877439\times10^{5}\ \mathrm{J\,m^{-3}},
    \frac{4}{\alpha\varphi} \approx 3.3877162\times10^{2},
    \Rightarrow E/V \approx 1.418688\times10^{8}\ \mathrm{J\,m^{-3}},
    M/V \approx 1.57850\times10^{-9}\ \mathrm{kg\,m^{-3}}.
    \)

    \subsection{Swirl–Gravity Coupling}
    \[
        \boxed{\ G_{\text{string}} = \frac{\vscore\ c^{5}\ t_p^{2}}{2\,F_{\text{EM}}^{\max}\ \rs^{2}} \ }
    \]
    Numerically \( \approx 6.674302\times10^{-11}\ \mathrm{m^3\,kg^{-1}\,s^{-2}}\) with the Canon constants.

    \subsection{Topology–Driven Mass Law (invariant form)}
    Let \(T(p,q)\) be a torus knot/link, \(n=\gcd(p,q)\) components, braid index \(b(T)=\min(|p|,|q|)\), Seifert genus \(g(T)\) (with standard link adjustment). Using ropelength \(\mathcal{L}_{\rm tot}(T)\) and string core radius \(\rs\):
    \[
        \boxed{
            M\big(T(p,q)\big)
            =\left(\frac{4}{\alpha}\right)\,
            b(T)^{-3/2}\,
            \varphi^{-\,g(T)}\,
            n^{-1/\varphi}\,
            \left(\frac{1}{2}\rhof \vscore^2\right)\,
            \frac{\pi \rs^3\,\mathcal{L}_{\mathrm{tot}}(T)}{c^2}.
        }
    \]
    Dimensionality follows from the factor \(\tfrac12\rhof \vscore^2\) (J\,m\(^{-3}\)) times a volume.

    \subsection{Swirl Clocks (Local Time-Rate)}
    \[
        \boxed{\ \frac{dt_{\text{local}}}{dt_{\infty}}
            = \sqrt{1 - \frac{\lVert\omegas\rVert^{2}\,\rs^{2}}{c^{2}}}
            = \sqrt{1 - \frac{\lVert\vswirl\rVert^{2}}{c^{2}}}\ \ (r=\rs)\ }.
    \]
    \emph{Historical (deprecated) variant without a length scale is retained only for traceability.}

    \subsection{Swirl Angular Frequency Profile}
    \[
        \boxed{\ \Omega_{\text{swirl}}(r) = \frac{\vscore}{\rs}\, e^{-r/\rs}\ },
        \qquad
        \Omega_{\text{swirl}}(0)=\frac{\vscore}{\rs}.
    \]

    \subsection{Vorticity Potential (Canonical Form)}
    \[
        \Phi(\vec r,\omegas) = \frac{\vscore^{2}}{2\,F_{\text{EM}}^{\max}}\ \omegas\!\cdot\!\vec r.
    \]
    \textbf{Dimensional remark:} Use with the SST Lagrangian ensuring \(\rhof\Phi\) has energy density units.

%================================================
% Unified SST Lagrangian (Definitive Form)
%================================================
    \section{Unified SST Lagrangian (Definitive Form)}
    \label{sec:lagrangian}

    Let $\vswirl$ be the velocity, $\rhof$ constant (incompressible), $\omegas=\nabla\times\vswirl$, and $\lambda$ enforce incompressibility.
    \[
        \boxed{\
        \mathcal{L}_{\text{SST}} =
            \frac{1}{2}\rhof\,\lVert\vswirl\rVert^{2}
            - \rhof\,\Phi(\vec r,\omegas)
            + \lambda(\nabla\cdot\vswirl)
            + \eta\,\int (\vswirl\cdot\omegas)\,dV
            + \mathcal{L}_{\text{couple}}[\Gamma,\mathcal{K}]
            \ }.
    \]
    Here $\mathcal{L}_{\text{couple}}$ encodes coupling to quantized circulation $\Gamma$ and knot invariants $\mathcal{K}$ (linking, writhe, twist).

%================================================
% Notation, Ontology, Glossary
%================================================
    \section{Notation, Ontology, and Glossary}
    \begin{itemize}
        \item \textbf{Absolute time (A-time):} global time parameter of the medium.
        \item \textbf{Chronos Time (C-time):} asymptotic observer time ($dt_{\infty}$).
        \item \textbf{Swirl Clocks:} local clocks set by \(\lVert\omegas\rVert\) or \(\lVert\vswirl\rVert\) per Sec.~\ref{sec:lagrangian}.
        \item \textbf{String taxonomy:} leptons = torus knots; quarks = chiral hyperbolic knots; bosons = unknots; neutrinos = linked knots.
        \item \textbf{Chirality:} ccw $\leftrightarrow$ matter; cw $\leftrightarrow$ antimatter via swirl–gravity coupling.
    \end{itemize}

%================================================
% Canonical Checks
%================================================
    \section{Canonical Checks (What to Verify in Every Paper)}
    \begin{enumerate}
        \item Dimensional analysis on every new term/equation.
        \item Limits: low-swirl \(\lVert\omegas\rVert\!\to\!0\) recovers classical mechanics/EM; large-scale averages reproduce Newtonian gravity with \(G_{\text{string}}\).
        \item Numerics: provide prefactors using Canon constants; add any new constants to Sec.~2.
        \item Explicit topology \(\leftrightarrow\) quantum mapping (which invariants, normalization).
        \item Cite any non-original constructs (BibTeX keys).
    \end{enumerate}

%================================================
% Personas (unchanged logic, SST names)
%================================================
    \section{Persona Prompts}

    \subsection*{Reviewer Persona}
		\scriptsize
    \begin{verbatim}
You are a peer reviewer for an SST paper. Use only the definitions and constants in the "SST Canon (v1.0)".
Check dimensional consistency, limiting behavior, and numerical validation. Flag any use of non-canonical
constants or equations unless equivalence is proved. Demand explicit mapping from knot invariants (linking,
writhe, twist) to claimed quantum numbers.
    \end{verbatim}

    \subsection*{Theorist Persona}
    \begin{verbatim}
You are a theoretical physicist specialized in Swirl String Theory (SST). Base all reasoning on the attached
"SST Canon (v1.0)". Your task: derive the swirl-based Hamiltonian for [TARGET SYSTEM], use Sec. 4 Lagrangian,
and verify the Swirl Clock law (Sec. 3). Provide boxed equations, dimensional checks, and a short numerical
evaluation using the Canon constants.
    \end{verbatim}

    \subsection*{Bridging Persona (Compare to GR/SM)}
    \begin{verbatim}
Work strictly within SST Canon (v1.0). Compare [TARGET] to its GR/SM counterpart. Identify exact replacements
(e.g., curvature → swirl), and show which terms reduce to Newtonian/Maxwellian limits. Include a correspondence
table and any constraints needed for equivalence.
    \end{verbatim}

%================================================
% Session Kickoff Checklist
%================================================
		\normalsize
    \section{Session Kickoff Checklist}
    \begin{enumerate}
        \item Start new chat per task; attach this Canon first.
        \item Paste a persona prompt (Sec.~7).
        \item Attach only task-relevant papers/sources.
        \item State any corrections explicitly (they persist in the session).
        \item At end, record Canon deltas (if any) and bump version.
    \end{enumerate}



%================================================
% Appendix: Boxed Canon Equations (paste-ready)
%================================================
    \section*{Appendix: Boxed Canon Equations (paste-ready)}
    \begin{enumerate}
        \item \textbf{Energy:} \fbox{$E_{\text{SST}} = \dfrac{4}{\alpha\varphi}\left(\dfrac{1}{2}\rhof \vscore^2\right)V$}
        \item \textbf{Mass:} \fbox{$M_{\text{SST}} = \dfrac{E_{\text{SST}}}{c^2}$}
        \item \textbf{$G$ coupling:} \fbox{$G_{\text{string}} = \dfrac{\vscore\, c^5 t_p^2}{2F_{\text{EM}}^{\max} \rs^2}$}
        \item \textbf{Swirl Clock:} \fbox{$\dfrac{dt_{\text{local}}}{dt_{\infty}}=\sqrt{1-\lVert\omegas\rVert^2 \rs^2/c^2}=\sqrt{1-\lVert\vswirl\rVert^2/c^2}$}
        \item \textbf{Swirl profile:} \fbox{$\Omega_{\text{swirl}}(r) = \dfrac{\vscore}{\rs}e^{-r/\rs}$}
    \end{enumerate}



%================================================
% Change Log
%================================================
    \section*{Change Log}
    \begin{itemize}
        \item \textbf{v1.0 (2025-08-22):} VAM$\rightarrow$SST translation; introduced \vswirl, \omegas, \rs, effective densities \(\rhof,\rhoE,\rhom\); renamed force cutoffs to \(F_{\text{EM}}^{\max}, F_{\text{G}}^{\max}\); adopted Swirl Clocks terminology; updated all boxed equations and Hamiltonian.
    \end{itemize}

    \nocite{*}
    \bibliographystyle{unsrt}
    \bibliography{canon_swirl_string_theory}

\end{document}

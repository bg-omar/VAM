\section{Emergent Mass from Soliton Energy}
\label{sec:mass-functional}

For a static, stable knotted swirl configuration \(K\), the rest energy \(E_K\) defines the solitonic mass.
Working in natural units \(\hbar=c=1\),
\begin{equation}
 m_K^{(\mathrm{sol})} = E_K .
\end{equation}

Guided by semiclassical analyses of knotted solitons \cite{Faddeev1997} and swirl energetics, we employ a \emph{topological mass functional}
\begin{equation}
 \boxed{ \; m_K^{(\mathrm{sol})} = \mathcal{M}_0 \,\Xi_K(m,n,s,k;\,\varphi) \; }
 \label{eq:mass-functional}
\end{equation}
where \(\mathcal{M}_0\) sets the universal energy scale and \(\Xi_K\) is a dimensionless topological factor.

\paragraph{Core scale.}
Introduce the \emph{core swirl velocity} \(\vswirl\) (Appendix~\ref{sec:core_swirl_velocity}). We take
\begin{equation}
 \boxed{ \; \mathcal{M}_0 \equiv C_0 \left(\sum_i V_i\right)\,\rhof\,\vswirl^{\,2} \; }
 \label{eq:M0}
\end{equation}
with
(i) \( \sum_i V_i \): total effective core volume of the knot (tube model),
(ii) \( \rhof \): base energy density of the medium (Appendix~\ref{sec:calibration_rho0}),
(iii) \(C_0\): dimensionless normalization capturing geometric/logarithmic slenderness and finite-core effects.

\paragraph{Topological factor.}
\(\Xi_K\) encodes geometric/dynamical properties of the knot:
\[
 \Xi_K = \Xi_K(m,n,s,k;\,\varphi), \qquad
 m=\text{strand multiplicity},\; n=\text{knot/link index},\;
\]
\[
 s=\text{coherence/tension index},\;
    k=\text{layer index}.
\]
A concrete ansatz used below is
\begin{equation}
    \Xi_K(m,n,s,k;\,\varphi) = \frac{\mathcal{T}_K(m,n,s)}{\varphi^{\,2k}},
    \label{eq:Xi-ansatz}
\end{equation}
where \(\mathcal{T}_K\) is a dimensionless tangle measure (e.g., normalized ropelength or writhe), and \(\varphi\) enters through canonical geometry of the core (Sec.~\ref{sec:golden_layer}).

\subsection{Heuristic derivation of the mass functional}

\paragraph{(1) Core energy from swirl string dynamics.}
For an incompressible medium, the energy per unit length of a slender swirl string scales as
\(
\tfrac{dE}{d\ell} \sim \tfrac{1}{2}\rhof\,\Gamma^2 \ln (R/r_c)
\).
With \(\Gamma \sim \vswirl r_c\) and total arclength \(\ell_K\),
\begin{equation}
    E_K^{\mathrm{core}} \sim \rhof\,\vswirl^{\,2}\,\ell_K \ln\!\left(\frac{R}{r_c}\right).
\end{equation}

\paragraph{(2) Volume correction (tube model).}
Treat each strand as a tube of radius \(r_c\):
\(
V_K=\sum_i \pi r_c^2 \ell_i
\Rightarrow
E_K^{\mathrm{bulk}} \sim \rhof\,\vswirl^{\,2}\,V_K
\),
which dominates for compact, tightly wound knots.
See Appendix~\ref{sec:knot-particle-mapping} for the explicit knot$\to$particle dictionary.

\paragraph{(3) Topological suppression via coherence.}
To encode knot geometry and internal alignment, introduce \(\Xi_K\) as in \eqref{eq:Xi-ansatz}; the \(\varphi^{-2k}\) factor models discrete coherence layers within the core (Sec.~\ref{sec:golden_layer}), while \(\mathcal{T}_K\) captures shape-dependent tangle energy.

\paragraph{(4) Combined result.}
Collecting pieces yields \eqref{eq:mass-functional} with \(\mathcal{M}_0\) given by \eqref{eq:M0}.

\paragraph{Dimensional check.}
In \(\hbar=c=1\), \([\rhof]=4\), \([V]=(-3)\), \([\vswirl]=0\) \(\Rightarrow\) \([\mathcal{M}_0]=1\) (mass), as required. \(\Xi_K\) is dimensionless by construction.

\subsection{Calibration and comparison}

Fix \(\mathcal{M}_0\) on a single reference (electron) to set the overall scale:
\begin{equation}
    C_0
    = \frac{m_e}{\big(\sum_i V_i\big)_e \,\rhof \,\vswirl^{\,2}}\,
    \frac{1}{\Xi_{K_e}} .
    \label{eq:calibration}
\end{equation}
After \eqref{eq:calibration}, predictions are parameter-free:
\[
m_K^{(\mathrm{sol})}/m_{K'}^{(\mathrm{sol})}
= \Xi_K/\Xi_{K'}
\].
Uncertainties propagate via
\begin{equation}
\delta m_K^2 \simeq m_K^2\big[
    (\delta\rhof/\rhof)^2
    + (2\,\delta\vswirl/\vswirl)^2
    + (\delta V/V)^2
    + (\delta\Xi_K/\Xi_K)^2
    \big].
\end{equation}
Comparisons use experimental values \cite{PDG2024}; medium/self-interaction corrections can be layered as perturbations to \(\rhof\) or to \(\mathcal{T}_K\).


%===========================================================
\subsection{Golden Layer: Hyperbolic Canonical Definition}
\label{sec:golden_layer}
%===========================================================

% --- Notation firewall (safe if already defined elsewhere) ---
\providecommand{\rc}{r_c}
\providecommand{\vswirl}{\mathbf{v}_{\mathrm{swirl}}}

\paragraph*{Policy (hyperbolic-first).}
The golden constant is \emph{defined} hyperbolically. We set
\[
    \xi_\varphi \;\equiv\; \operatorname{asinh}\!\left(\tfrac{1}{2}\right),
    \qquad
    \varphi \;\equiv\; \exp(\xi_\varphi),
\]
and only later note the algebraic echo \(\varphi=(1+\sqrt5)/2\) as a corollary (not as a definition).

\paragraph*{Golden rapidity.}
Define the golden rapidity
\[
    \xi_g \;\equiv\; \tfrac{3}{2}\,\xi_\varphi.
\]
Using \(\tanh y=\dfrac{e^{2y}-1}{e^{2y}+1}\) (standard hyperbolic identity, see \cite{NISTDLMF}),
\[
    \tanh(\xi_g) \;=\; \frac{e^{3\xi_\varphi}-1}{e^{3\xi_\varphi}+1}
    \;=\; \frac{\varphi^3-1}{\varphi^3+1}.
\]
From \(\varphi=\exp(\xi_\varphi)\) and the algebraic consequence \(\varphi^2=\varphi+1\) (derived after the hyperbolic definition), we get \(\varphi^3=2\varphi+1\), hence
\[
    \tanh(\xi_g) \;=\; \frac{(2\varphi+1)-1}{(2\varphi+1)+1}
    =\frac{2\varphi}{2(\varphi+1)}=\frac{\varphi}{\varphi^2}=\frac{1}{\varphi}.
\]
Therefore
\[
    \boxed{\ \tanh\!\big(\tfrac{3}{2}\,\xi_\varphi\big)=\tanh(\xi_g)=\varphi^{-1}\ },
\]

    \text{equivalently }\ \coth(\xi_g)=\varphi.

\paragraph*{VAM mapping (canonical scales).}
Rapidity parametrizes the swirl speed as
\[
    \beta \;\equiv\; \frac{\|\vswirl\|}{\vswirl} \;=\; \tanh\xi.
\]
At the Golden Layer \(\xi=\xi_g\),
\[
    \beta_g \;=\; \frac{1}{\varphi},\qquad
    v_g \;\equiv\; \|\vswirl\|_g \;=\; \frac{\vswirl}{\varphi},\qquad
    \Omega_g \;=\; \frac{v_g}{\rc} \;=\; \frac{1}{\varphi}\,\frac{\vswirl}{\rc}.
\]
\emph{Dimensional check:} \([\beta_g]=1\), \([v_g]=\mathrm{m/s}\), \([\Omega_g]=\mathrm{s}^{-1}\).

\paragraph*{Algebraic echo (post-hoc).}
From the standard definition \(\operatorname{asinh}x=\ln(x+\sqrt{x^2+1})\) \cite{NISTDLMF},
\[
    \xi_\varphi=\operatorname{asinh}\!\left(\tfrac{1}{2}\right)
    =\ln\!\left(\tfrac{1}{2}+\sqrt{\tfrac{1}{4}+1}\right)
    =\ln\!\left(\tfrac{1+\sqrt5}{2}\right),
\]
so \(\varphi=\exp(\xi_\varphi)=(1+\sqrt5)/2\). This \emph{confirms} (rather than defines) the familiar algebraic form.

\paragraph*{Numerical evaluation (Canon constants).}
With \(\vswirl=\SI{1.09384563e6}{m/s}\) and \(\rc=\SI{1.40897017e-15}{m}\),
\[
    \varphi \approx 1.618033988749895,\quad
    \xi_g=\tfrac{3}{2}\ln\varphi \approx 0.721817737589405,
\]
\[
    \beta_g=\tanh\xi_g=\varphi^{-1}\approx 0.618033988749895,
    v_g=\frac{\vswirl}{\varphi}\approx \SI{6.760337778e5}{m/s},\qquad
\]
\[
    \Omega=\frac{\vswirl}{\rc}\approx \SI{7.763440655e20}{s^{-1}},\qquad
    \Omega_g=\frac{\Omega}{\varphi}\approx \SI{4.798070195e20}{s^{-1}}.
\]




\paragraph*{On the \(3/2\) exponent.}
The \(\tfrac{3}{2}\) multiplier in \(\xi_g=\tfrac{3}{2}\xig\) mirrors common spectral/dispersion scalings
(quantum level spacings, Kelvin-wave cascades), and will be used to label “golden layers” in SST.

\subsection{Pentagonal resonance hypothesis}
% --- “Five” and pentagon transient (context hook) ---
\paragraph*{Remark (pentagon transient).}
When an unknotting filament strikes a boundary, a short-lived five-vertex symmetry (pentagon-like)
is empirically observed; in SST we treat this as a \emph{transient morphometric feature} of the
filament’s curvature–torsion flow rather than as a defining identity for \(\phi\).
Motivated by simulations of swirl string-ring impacts \cite{orlandi1993vortex}, we hypothesize:

\begin{quote}
    \textbf{Pentagonal Resonance Hypothesis.}
    A photon is absorbed by an electron when its transient pentagonal swirl mode geometrically resonates with a pentagonal face of the dodecahedral electron shell. This topological match enables energy and swirl transfer.
\end{quote}

\subsection{Canonical role}
The Golden Layer functions as:
(i) a \emph{quantization anchor} for swirl rapidity (\(\xi=\xi_g\));
(ii) a \emph{resonance mechanism} in electron–photon coupling via dodecahedral symmetry;
(iii) a \emph{bridge} between continuous swirl dynamics and discrete spectroscopic structure.

% --- Integrated: EFT derivation of alpha C + beta L and golden-layer ---
\subsection{Field-Theoretic Derivation of \texorpdfstring{$\alpha C+\beta L$}{alpha C + beta L} and \texorpdfstring{$\varphi^{-2k}$}{phi^{-2k}}}
\paragraph*{Length term.}
For a slender tube of radius \(r_c\) and circulation \(\Gamma\), the line tension is
\(\tau\simeq \frac{\rhof\Gamma^2}{4\pi}\ln\!\frac{R}{r_c}+\kappa_H r_c^2\langle\omega^2\rangle\), so \(E_{\rm line}\simeq \tau\,\ell_K\),
and with \(L(K)=\ell_K/r_c\) this yields a contribution \(\propto \beta\,L(K)\).

\paragraph*{Crossing term.}
Nonlocal Biot--Savart interactions between tube segments near contact (\(\sim r_c\)) discretize to counts proportional to the minimal crossing number \(C(K)\), giving
the term \(\propto \alpha\,C(K)\). A Skyrme/Hopf quartic term enforces the stability bound \(E\ge \kappa\,|Q_H|^{3/4}\).

\paragraph*{Golden-layer suppression.}
A weak pentagonal core deformation induces discrete scale invariance in radial modes with ratio \(\lambda_\star=\varphi\).
Since energy scales with amplitude squared, this yields the multiplicative factor \(\varphi^{-2k}\) for the \(k\)-th layer.
Altogether (with your normalization),
\begin{equation}
\Xi_K \;=\; \frac{\alpha\,C(K)+\beta\,L(K)}{T_{01}}\;\varphi^{-2k_K},\qquad m_K = M_0\,\Xi_K.
\end{equation}
% --- End Integrated block ---

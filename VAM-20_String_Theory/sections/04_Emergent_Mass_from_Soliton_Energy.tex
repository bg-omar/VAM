\section{Emergent Mass from Soliton Energy}
\label{sec:mass-functional}

% --- local macros to avoid ambiguity ---
\providecommand{\rc}{r_c}
\providecommand{\vswirl}{\mathbf{v}_{\mathrm{swirl}}} % vector field
\newcommand{\vstar}{v_\star}                           % canonical swirl-speed scale

For a static, stable knotted swirl configuration \(K\), the rest energy \(E_K\) defines the solitonic mass.
Working in natural units \(\hbar=c=1\),
\begin{equation}
	m_K^{(\mathrm{sol})} = E_K .
\end{equation}

Guided by semiclassical analyses of knotted solitons \cite{Faddeev1997} and swirl energetics, we employ a \emph{topological mass functional}
\begin{equation}
	\boxed{ \; m_K^{(\mathrm{sol})} = \mathcal{M}_0 \,\Xi_K(m,n,s,k;\,\varphi) \; }
	\label{eq:mass-functional}
\end{equation}
where \(\mathcal{M}_0\) sets a universal energy scale and \(\Xi_K\) is a dimensionless topological factor.

We implement this via a calibrated functional fixed by condensate scales and knot invariants. At leading order we fit the constants on \((e^{-},p,n)\) and predict the remaining masses \emph{without introducing free Yukawa parameters} (see Appendix~\ref{sec:calibration_hierarchy} for the calibration hierarchy and the lepton analysis).

\paragraph{Core scale.}
Introduce the \emph{core swirl speed scale} \(\vstar\) (Appendix~\ref{sec:core_swirl_velocity}). We take
\begin{equation}
	\boxed{ \; \mathcal{M}_0 \equiv C_0 \left(\sum_i V_i\right)\,\rhoF\,\vstar^{\,2} \; }
	\label{eq:M0}
\end{equation}
with
(i) \( \sum_i V_i \): total effective core volume of the knot (tube model),
(ii) \( \rhoF \): base energy density of the medium (Appendix~\ref{sec:calibration_rho0}),
(iii) \(C_0\): dimensionless normalization capturing geometric/logarithmic slenderness and finite-core effects.

\paragraph{Topological factor.}
\(\Xi_K\) encodes geometric/dynamical properties of the knot:
\[
	\Xi_K = \Xi_K(m,n,s,k;\,\varphi), \qquad
	m=\text{strand multiplicity},\; n=\text{knot/link index},\;
	s=\text{coherence/tension index},\;
	k=\text{layer index}.
\]
A concrete ansatz used below is
\begin{equation}
	\Xi_K(m,n,s,k;\,\varphi) = \frac{\mathcal{T}_K(m,n,s)}{\varphi^{\,2k}},
	\label{eq:Xi-ansatz}
\end{equation}
where \(\mathcal{T}_K\) is a dimensionless tangle measure (e.g., normalized ropelength or writhe), and \(\varphi\) enters through canonical core geometry (Sec.~\ref{sec:golden_layer}).

\subsection{Heuristic derivation of the mass functional}

\paragraph{(1) Core energy from swirl-string dynamics.}
For an incompressible medium, the energy per unit length of a slender swirl string scales as
\(
\tfrac{dE}{d\ell} \sim \tfrac{1}{2}\rhoF\,\Gamma^2 \ln (R/\rc)
\).
With \(\Gamma \sim \vstar\,\rc\) and total arclength \(\ell_K\),
\begin{equation}
	E_K^{\mathrm{core}} \sim \rhoF\,\vstar^{\,2}\,\ell_K \ln\!\left(\frac{R}{\rc}\right).
\end{equation}

\paragraph{(2) Volume correction (tube model).}
Treat each strand as a tube of radius \(\rc\):
\(
V_K=\sum_i \pi \rc^2 \ell_i
\Rightarrow
E_K^{\mathrm{bulk}} \sim \rhoF\,\vstar^{\,2}\,V_K
\),
which dominates for compact, tightly wound knots.
See Appendix~\ref{sec:knot-particle-mapping} for the explicit knot\(\to\)particle dictionary.

\paragraph{(3) Topological suppression via coherence.}
To encode knot geometry and internal alignment, introduce \(\Xi_K\) as in \eqref{eq:Xi-ansatz}. The factor \(\varphi^{-2k}\) models discrete coherence layers within the core (Sec.~\ref{sec:golden_layer}), while \(\mathcal{T}_K\) captures shape-dependent tangle energy.

\paragraph{(4) Combined result.}
Collecting pieces yields \eqref{eq:mass-functional} with \(\mathcal{M}_0\) as in \eqref{eq:M0}.

\paragraph{Dimensional check.}
In \(\hbar=c=1\), \([\rhoF]=4\), \([V]=(-3)\), \([\vstar]=0\) \(\Rightarrow\) \([\mathcal{M}_0]=1\) (mass), as required. \(\Xi_K\) is dimensionless by construction.

\subsection{Calibration and comparison}

Fix \(\mathcal{M}_0\) on a single reference (electron) to set the overall scale:
\begin{equation}
	C_0
	= \frac{m_e}{\big(\sum_i V_i\big)_e \,\rhoF \,\vstar^{\,2}}\,
	\frac{1}{\Xi_{K_e}} .
	\label{eq:calibration}
\end{equation}
After \eqref{eq:calibration}, predictions are parameter-free:
\[
	\frac{m_K^{(\mathrm{sol})}}{m_{K'}^{(\mathrm{sol})}}
	\;=\; \frac{\Xi_K}{\Xi_{K'}}.
\]
Uncertainties propagate via
\begin{equation}
	\delta m_K^2 \simeq m_K^2\Big[
		\big(\delta\rhoF/\rhoF\big)^2
		+ \big(2\,\delta\vstar/\vstar\big)^2
		+ \big(\delta V/V\big)^2
		+ \big(\delta\Xi_K/\Xi_K\big)^2
		\Big].
\end{equation}
Comparisons use experimental values \cite{PDG2024}; medium/self-interaction corrections can be layered as perturbations to \(\rhoF\) or to \(\mathcal{T}_K\).

%===========================================================
\subsection{Golden Layer: Hyperbolic Canonical Definition}
\label{sec:golden_layer}
%===========================================================

\paragraph*{Policy (hyperbolic-first).}
Define the golden constant hyperbolically:
\[
	\xi_\varphi \;\equiv\; \operatorname{asinh}\!\left(\tfrac{1}{2}\right),
	\qquad
	\varphi \;\equiv\; \exp(\xi_\varphi),
\]
and note the algebraic echo \(\varphi=(1+\sqrt5)/2\) only as a corollary.

\paragraph*{Golden rapidity.}
Let
\[
	\xi_g \;\equiv\; \tfrac{3}{2}\,\xi_\varphi.
\]
Using \(\tanh y=\dfrac{e^{2y}-1}{e^{2y}+1}\) \cite{NISTDLMF} and \(\varphi=\exp(\xi_\varphi)\) with \(\varphi^2=\varphi+1\),
\[
	\tanh(\xi_g) \;=\; \frac{\varphi^3-1}{\varphi^3+1}
	\;=\; \frac{1}{\varphi}.
\]
Therefore
\[
	\boxed{\ \tanh\!\big(\tfrac{3}{2}\,\xi_\varphi\big)=\tanh(\xi_g)=\varphi^{-1}\ },
	\qquad\text{equivalently}\ \ \coth(\xi_g)=\varphi.
\]

\paragraph*{VAM mapping (canonical scales).}
Let \(v\equiv\|\vswirl\|\). Parametrize swirl speed by rapidity via
\[
	\beta \;\equiv\; \frac{v}{\vstar} \;=\; \tanh\xi.
\]
At the Golden Layer \(\xi=\xi_g\),
\[
	\beta_g \;=\; \frac{1}{\varphi},\qquad
	v_g \;=\; \beta_g\,\vstar \;=\; \frac{\vstar}{\varphi},\qquad
	\Omega_g \;=\; \frac{v_g}{\rc} \;=\; \frac{1}{\varphi}\,\frac{\vstar}{\rc}.
\]
\emph{Dimensional check:} \([\beta_g]=1\), \([v_g]=\mathrm{m/s}\), \([\Omega_g]=\mathrm{s}^{-1}\).

\paragraph*{Algebraic echo (post-hoc).}
From \(\operatorname{asinh}x=\ln(x+\sqrt{x^2+1})\) \cite{NISTDLMF},
\[
	\xi_\varphi=\operatorname{asinh}\!\left(\tfrac{1}{2}\right)
	=\ln\!\left(\tfrac{1}{2}+\sqrt{\tfrac{1}{4}+1}\right)
	=\ln\!\left(\tfrac{1+\sqrt5}{2}\right),
\]
so \(\varphi=\exp(\xi_\varphi)=(1+\sqrt5)/2\).

\paragraph*{Numerical evaluation (Canon constants).}
With \(\vstar=\SI{1.09384563e6}{m/s}\) and \(\rc=\SI{1.40897017e-15}{m}\),
\[
	\varphi \approx 1.618033988749895,\quad
	\xi_g=\tfrac{3}{2}\ln\varphi \approx 0.721817737589405,
\]
\[
	\beta_g=\tanh\xi_g=\varphi^{-1}\approx 0.618033988749895,\qquad
	v_g=\frac{\vstar}{\varphi}\approx \SI{6.760337778e5}{m/s},
\]
\[
	\Omega_{\!*}=\frac{\vstar}{\rc}\approx \SI{7.763440655e20}{s^{-1}},\qquad
	\Omega_g=\frac{\Omega_{\!*}}{\varphi}\approx \SI{4.798070195e20}{s^{-1}}.
\]

\paragraph*{On the \(3/2\) exponent.}
The factor \(\tfrac{3}{2}\) in \(\xi_g\) mirrors familiar spectral/dispersion scalings (e.g., level spacings, Kelvin-wave cascades) and labels “golden layers” in SST.

\subsection{Pentagonal resonance hypothesis}
\paragraph*{Remark (pentagon transient).}
When an unknotting filament strikes a boundary, a short-lived five-vertex symmetry (pentagon-like) is empirically observed; in SST this is treated as a \emph{transient morphometric feature} of curvature–torsion flow rather than a defining identity for \(\varphi\). Motivated by simulations of swirl string–ring impacts \cite{orlandi1993vortex}, we hypothesize:
\begin{quote}
	\textbf{Pentagonal Resonance Hypothesis.}
	A photon is absorbed by an electron when its transient pentagonal swirl mode geometrically resonates with a pentagonal face of the dodecahedral electron shell, enabling energy and swirl transfer.
\end{quote}

\subsection{Canonical role}
The Golden Layer functions as (i) a \emph{quantization anchor} for swirl rapidity (\(\xi=\xi_g\)); (ii) a \emph{resonance mechanism} in electron–photon coupling via dodecahedral symmetry; and (iii) a \emph{bridge} between continuous swirl dynamics and discrete spectroscopic structure.

% --- Integrated: EFT derivation of alpha_C C + beta_L L and golden-layer ---
\subsection{Field-Theoretic Derivation of \texorpdfstring{$\alpha_C C+\beta_{\!L} L$}{alpha_C C + beta_L L} and \texorpdfstring{$\varphi^{-2k}$}{phi^{-2k}}}
\paragraph*{Length term.}
For a slender tube of radius \(\rc\) and circulation \(\Gamma\), the line tension is
\(
\tau\simeq \frac{\rhoF\Gamma^2}{4\pi}\ln\!\frac{R}{\rc}+\kappa_H \rc^2\langle\omega^2\rangle
\),
so \(E_{\rm line}\simeq \tau\,\ell_K\),
and with \(L(K)=\ell_K/\rc\) this yields a contribution \(\propto \beta_{\!L}\,L(K)\).

\paragraph*{Crossing term.}
Nonlocal Biot--Savart interactions between tube segments near contact (\(\sim \rc\)) discretize to counts proportional to the minimal crossing number \(C(K)\), giving
the term \(\propto \alpha_C\,C(K)\). A Skyrme/Hopf quartic term enforces the stability bound \(E\ge \kappa\,|Q_H|^{3/4}\).

\paragraph*{Golden-layer suppression.}
A weak pentagonal core deformation induces discrete scale invariance in radial modes with ratio \(\lambda_\star=\varphi\).
Since energy scales with amplitude squared, this yields the multiplicative factor \(\varphi^{-2k}\) for the \(k\)-th layer.
Altogether (with your normalization),
\begin{equation}
	\Xi_K \;=\; \frac{\alpha_C\,C(K)+\beta_{\!L}\,L(K)}{T_{01}}\;\varphi^{-2k_K},
	\qquad
	m_K \;=\; \mathcal{M}_0\,\Xi_K .
\end{equation}

%========================================================================================
% APPENDIX B: Derivation of the Core Swirl Velocity \vswirl
%========================================================================================
\section*{Appendix B: Derivation of the Core Swirl Velocity}
\addcontentsline{toc}{section}{Appendix B: Derivation of the Core Swirl Velocity}
\label{sec:core_swirl_velocity}

\paragraph{Referenced in Main Text.}
This parameter enters the energy scale
\[
 \mathcal{M}_0 = C_0 \Big(\sum_i V_i\Big)\,\rhof\,\frac{\vswirl^{\,2}}{c^2},
\]
where we interpret \(\vswirl\) as the tangential speed at the swirl–core boundary.

\subsection*{B.1 Geometric axiom: minimal core radius}
We set the core radius by the classical electron radius
\[
 r_e \equiv \frac{e^2}{4\pi \varepsilon_0 m_e c^2} \approx 2.8179403\times10^{-15}\ \mathrm{m},
\]
and adopt the modeling choice
\[
 r_c \equiv \tfrac12 r_e \approx 1.4089702\times10^{-15}\ \mathrm{m},
\]
motivated by swirl string–tube stability and used consistently throughout the VAM/SST fits.

\subsection*{B.2 Dynamical axiom: Compton synchronization}
Let the intrinsic rotation of the elementary swirl be locked to the electron’s Compton (angular) frequency
\[
 \omega_c \equiv \frac{m_e c^2}{\hbar} \approx 7.76344\times10^{20}\ \mathrm{rad\,s^{-1}} .
\]
This identifies the boundary angular velocity with \(\omega_{\text{core}}=\omega_c\).

\subsection*{B.3 Result and numerical evaluation}
The core tangential speed is then
\[
 \boxed{\ \vswirl \;=\; r_c\,\omega_c\ } .
\]
With the values above,
\[
 \vswirl
 = (1.4089702\times10^{-15}\ \mathrm{m})\,(7.76344\times10^{20}\ \mathrm{rad\,s^{-1}})
    \approx \boxed{1.0938\times10^{6}\ \mathrm{m\,s^{-1}}}\,,
\]
comfortably subluminal (\(\vswirl\ll c\)).

\paragraph{Dimensional check.}
\([r_c]=\mathrm{m}\), \([\omega_c]=\mathrm{s^{-1}}\Rightarrow [\vswirl]=\mathrm{m\,s^{-1}}\).

\paragraph{Consistency check with the Golden Layer.}
Using \(\varphi=\tfrac{1+\sqrt5}{2}\), the golden-layer kinematics in Sec.~\ref{sec:golden_layer} give
\[
    v_g=\frac{\vswirl}{\varphi}\approx \frac{1.0938\times10^6}{1.618}\ \mathrm{m\,s^{-1}}
    \approx 6.76\times10^5\ \mathrm{m\,s^{-1}},
\]
matching the Canon numerics quoted there.

\subsection*{B.4 Remarks and scope}
(i) The choice \(\omega_{\text{core}}=\omega_c\) is a model axiom that ties the swirl’s intrinsic rotation to the rest-energy scale \(E=\hbar\omega\); it yields a single canonical \(\vswirl\) used across fits.
(ii) Alternative lockings (e.g., to a multiple of \(\omega_c\) or to a knot-dependent layer index) amount to \(\vswirl}\!\to\! \lambda\,\vswirl\) and can be absorbed into the dimensionless factors of the mass functional; we therefore keep \(\lambda=1\) as the canonical choice.
(iii) The numerical value here is the same constant employed in Appendix~\ref{sec:calibration_rho0} and in the simulation code (variable \texttt{C\_e}).

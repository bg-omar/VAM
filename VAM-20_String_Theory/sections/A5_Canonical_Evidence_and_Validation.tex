%===============================================================================
% CANONICAL EVIDENCE & VALIDATION (SST) — REWRITTEN, SINGLE-NARRATIVE VERSION
%===============================================================================
\section*{Appendix E: Canonical Evidence and Validation in Swirl--String Theory (SST)}
\label{sec:canon-validation}

\subsection{Protocol and notation}
On a fixed clock-field leaf \(\Sigma_t\subset\mathbb{R}^3\), let
\[
	\mathbf v=\mathbf v_K \quad\text{(Biot--Savart velocity from a Fourier–reconstructed knot \(K\))},\qquad
	\boldsymbol\omega=\nabla\times\mathbf v,\qquad
	r=\|\mathbf x-\mathbf x_0\|.
\]
The interior mask \(\Omega\subset\Sigma_t\) is a cubic subgrid used to avoid boundary artefacts. The helicity functionals and SST swirl–helicity asymmetry are
\[
	H_c=\sum_{\Omega}\mathbf v\!\cdot\!\boldsymbol\omega,\qquad
	H_m=\sum_{\Omega}\|\boldsymbol\omega\|^2\,r^2,\qquad
	a_{\rm SST}(K)=\tfrac12\!\left(\frac{H_c}{H_m}-1\right).
\]
A three-level grid sweep \((N,h,i_{\rm in})\in\{(32,0.10,8),(48,0.08,12),(64,0.06,16)\}\) assesses numerical convergence, with a conservative uncertainty proxy
\[
	\sigma \equiv |a_{64}-a_{48}|.
\]
Decision thresholds, fixed for all runs, are
\[
	\delta_\star=0.03,\qquad \sigma_\star=0.02.
\]
To mitigate geometry–grid pathologies, all runs use a uniform \emph{SST-canon harness}:
\begin{enumerate}
	\item \emph{Centroid/scale normalization:} translate so \(\mathbf x_0=\langle\mathbf x\rangle\) sits at a fixed offset from the grid origin; rescale to a common RMS radius \(\sqrt{\langle r^2\rangle}\).
	\item \emph{Finite-core Biot--Savart:} midpoint kernel \((\|R\|^2+a^2)^{-\frac{3}{2}}\) with \(a=0.75\,h\).
	\item \emph{Radial regularization in \(H_m\):} replace \(r^2\) by \(r^2+r_0^2\) with \(r_0=0.75\,h\).
	\item \emph{Arc-length reparameterization:} resample the Fourier curve to equal-length segments before field evaluation.
\end{enumerate}

\subsection{Amphichiral controls}
Amphichiral baselines pin the reference value \(a_{\rm SST}=-0.5\) and validate the estimator as a symmetry detector on \(\Sigma_t\):
\[
	1_1:\;(-0.500,\,-0.500,\,-0.500),\quad
	4_1{\rm z}:\;(-0.4997,\,-0.5000,\,-0.49999),\quad
	6_3{\rm z}:\;(-0.5000,\,-0.5000,\,-0.5000),\quad
	12a_{1202}{\rm z6}:\;(-0.5000,\,-0.5000,\,-0.50048).
\]

\subsection{Canonical quark baselines}
Across the sweep, candidate quark knots separate cleanly:
\[
	\boxed{u\;\leftrightarrow\;6_2:\quad a_{64}=-0.49025,\ \ \sigma=0.00572}
	\qquad
	\boxed{d\;\leftrightarrow\;7_4:\quad a_{64}=-0.52299,\ \ \sigma=0.01295}
\]
Thus \(6_2\) lies in a near-amphichiral band (\(|a_{64}+0.5|\!\lesssim\!0.015\)), while \(7_4\) exhibits a robust chiral deviation. This realizes the intended \((u,d)\) contrast in SST: mirror-balanced swirl for \(u\) and handed bias for \(d\).

\subsection{Convergence and classification}
We adopt the following taxonomy for tables and fits:
\begin{itemize}
	\item \textbf{Near-amphichiral (converged):} \(|a_{64}+0.5|\le 0.015\) and \(\sigma\le\sigma_\star\) (e.g.\ \(6_2\)).
	\item \textbf{Chiral (converged):} \(|a_{64}+0.5|>0.02\) with \(\sigma\le\sigma_\star\) (e.g.\ \(7_4\) and several \(7\)–/\(8\)–crossing exemplars).
	\item \textbf{Tentative:} \(0.02<\sigma\le 0.05\); merits a rerun under the canon harness.
	\item \textbf{Unstable:} \(\sigma>0.05\) or abnormally large \(|a_{64}|\); dominated by geometric/regularization artefacts rather than physics.
\end{itemize}

\subsection{Flagged cases and outlier behavior}
A small subset (including some amphichiral labels in the aggregate CSV) show transient deviations from \(-0.5\); each has at least one near-ideal value in raw sweeps, indicating numerical sensitivity. An extreme outlier,
\[
	8_5:\quad a_{32}\approx -2.01,
\]
traces to degeneracy near the foliation origin: \(H_c\) is scale-invariant while \(H_m\propto\langle r^2\rangle\), so shrinking or centering vorticity near \(r=0\) inflates \(H_c/H_m\). Applying the uniform canon harness (normalization, finite-core kernel, radial floor, arc-length reparameterization) resolves these pathologies and returns outliers to the physical band observed across the dataset.

%-------------------------------------------------------------------------------
%   Appendix E.x — Hyperbolic Volume as a Canonical Topological Multiplier
%-------------------------------------------------------------------------------
\subsection*{Hyperbolic volume as a canonical topological multiplier}
\addcontentsline{toc}{subsection}{Hyperbolic Volume as a Canonical Topological Multiplier}
\label{sec:hyp-volume}

\paragraph{Definition and rationale.}
For a hyperbolic knot \(K\subset S^3\), the complement \(M_K=S^3\!\setminus\!K\) admits a unique complete, finite–volume hyperbolic metric, and
\[
	\Vol_{\!\mathbb{H}}(K)=\Vol(M_K)
\]
is a topological invariant by Mostow–Prasad rigidity. Torus and satellite knots are non-hyperbolic and satisfy \(\Vol_{\!\mathbb{H}}(K)=0\). We adopt the parameter-free choice
\[
	\boxed{\ \mathcal V_K \equiv \Vol_{\!\mathbb{H}}(K)\ }
\]
as the dimensionless multiplier that scales the geometric core volume in hadronic sectors.

\paragraph{Operational insertion.}
With swirl-core radius \(r_c\), define
\[
	V_{\rm core}(K)=V_{\rm torus}\,\mathcal V_K,\qquad
	V_{\rm torus}=4\pi^2 r_c^{\,3}.
\]
For a composite with constituents \(K_i\) (e.g.\ \(uud\) or \(udd\)),
\[
	\mathcal M_0\ \propto\ \sum_i V_{\rm core}(K_i)
	\ =\ V_{\rm torus}\sum_i \mathcal V_{K_i},
\]
and the rest mass follows the standard SST prefactors (electroweak amplification, coherence, Golden-layer tension, swirl energy density):
\[
	M \;=\; \frac{4}{\alpha_{\!fs}}\;\eta\;\xi\;\varphi^{-s}\;
	\Big(\sum_i V_{\rm torus}\,\mathcal{V}_{K_i}\Big)\;
	\frac{\tfrac12\,\rhoF\,\vswirl^{\,2}}{c^2},
\]
with symbols defined elsewhere in the paper. The values used in the mass fits for the canonical \((u,d)=(6_2,7_4)\) choice are
\[
	\mathcal{V}_{6_2}=2.8281,\qquad
	\mathcal{V}_{7_4}=3.1639.
\]

\paragraph{Why this choice is canonical.}
The hyperbolic volume is intrinsic to the knot type and does not depend on the embedding; it tracks geometric/topological complexity; it is additive across the hyperbolic pieces in a JSJ decomposition; and it vanishes for torus knots, which cleanly separates torus-dominated leptonic exemplars from hyperbolic hadronic constituents.

\paragraph{Computation and stability.}
\(\Vol_{\!\mathbb{H}}(K)\) is computed from an ideal triangulation of \(S^3\!\setminus\!K\) by solving gluing/completeness equations for shape parameters and summing the associated Lobachevsky/Bloch–Wigner dilogarithms; standard 3–manifold solvers implement this pipeline. Being topological, \(\mathcal V_K\) carries negligible uncertainty compared to discretization/windowing effects in the helicity sweep, which dominate the error budget for \(\xi\) and \(s\).

\paragraph{Consistency with the helicity classifier.}
Torus knots (e.g.\ \(3_1\), \(5_1\)) have \(\Vol_{\!\mathbb{H}}=0\) and enter masses via a separate dimensionless factor \(\Xi_K\). Hyperbolic quark candidates (\(6_2\), \(7_4\)) have \(\Vol_{\!\mathbb{H}}>0\), thus contributing directly to the extensive core volume in the nucleon mass scale.

\subsection*{Summary}
The grid-sweep evidence on \(\Sigma_t\), anchored by amphichiral controls and convergence diagnostics, supports the canonical assignments
\[
	\boxed{u\!\leftrightarrow\!6_2\ \text{(near-amphichiral)}\quad\ \ d\!\leftrightarrow\!7_4\ \text{(robust chiral)}}.
\]
Under the uniform SST-canon harness, flagged cases relax into the physical band, and the hyperbolic-volume multiplier \(\mathcal V_K=\Vol_{\!\mathbb{H}}(K)\) provides a stable, parameter-free bridge from topological complexity to the hadronic mass scale. We use these assignments for subsequent calibration of the mass functional and charge-mapping tables, reporting \(a_{\rm SST}\) with \(\sigma=|a_{64}-a_{48}|\).

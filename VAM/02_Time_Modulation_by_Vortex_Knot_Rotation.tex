\section{Time Modulation by Vortex Knot Rotation}

Building upon the previous section's treatment of time dilation via pressure and Bernoulli dynamics, we now focus on the intrinsic rotation of topological vortex knots. In the Vortex Æther Model (VAM), particles are modeled as stable, topologically conserved vortex knots embedded in an incompressible, inviscid superfluid medium. Each knot possesses a characteristic internal angular frequency $\Omega_k$, and this internal motion induces local time modulation relative to the absolute time of the æther.

Rather than curving spacetime, we propose that internal rotational energy and helicity conservation induce temporal slowdowns analogous to gravitational redshift. This section develops these ideas through heuristic and energetic arguments consistent with the hierarchy introduced in Section I.

\subsection{Heuristic and Energetic Derivation}

We begin by proposing a rotationally-induced time dilation formula based on the knot's internal angular frequency:

\begin{equation}
\frac{t_{\text{local}}}{t_{\text{abs}}} = \left(1 + \beta \Omega_k^2 \right)^{-1}\label{eq:rotational_induced_time_dilation}
\end{equation}

where:

\begin{itemize}
\item $t_{\text{local}}$ is the proper time near the knot,
\item $t_{\text{abs}}$ is the absolute time of the background æther,
\item $\Omega_k$ is the average core angular frequency,
\item $\beta$ is a coupling coefficient with dimensions $[\beta] = \text{s}^2$.
\end{itemize}

For small angular velocities, we obtain a first-order expansion:

\begin{equation}
\frac{t_{\text{local}}}{t_{\text{abs}}} \approx 1 - \beta \Omega_k^2 + \mathcal{O}(\Omega_k^4)\label{eq:rotational_induced_time_dilation_expansion}
\end{equation}

This form parallels the Lorentz factor at low velocities in special relativity:

\begin{equation}
\frac{t_{\text{moving}}}{t_{\text{rest}}} \approx 1 - \frac{v^2}{2c^2}\label{eq:parallels_lorentz_time_dilation}
\end{equation}

This establishes an important analogy: internal rotational motion in VAM induces temporal slowing, similar to how translational velocity induces time dilation in SR.

To strengthen the physical foundation of this expression, we now relate time dilation to the energy stored in vortex rotation. Let the vortex knot have an effective moment of inertia $I$. Its rotational energy is given by:

\begin{equation}
E_{\text{rot}} = \frac{1}{2} I \Omega_k^2\label{eq:rotational_energy_inertia}
\end{equation}

Assuming time slows due to this energy density, we write:

\begin{equation}
\frac{t_{\text{local}}}{t_{\text{abs}}} = \left(1 + \beta E_{\text{rot}} \right)^{-1} = \left(1 + \frac{1}{2} \beta I \Omega_k^2 \right)^{-1}\label{eq:time_dilation_rotational_energy_inertia}
\end{equation}

This expression serves as the energetic analog of the pressure-based Bernoulli model from Section I (see Eq.~\eqref{eq:localtime_vortex}). It supports the interpretation of vortex-induced time wells via energy storage rather than geometric deformation.

\subsection{Topological and Physical Justification}

Topological vortex knots are not only characterized by rotation but also by helicity:

\begin{equation}
H = \int \vec{v} \cdot \vec{\omega} \, d^3x \label{eq:helicity_rotation}
\end{equation}

Helicity is a conserved quantity in ideal (inviscid, incompressible) fluids, encoding the linkage and twisting of vortex lines. The rotational frequency $\Omega_k$ becomes a topologically meaningful indicator of the knot’s identity and dynamical state.

Larger $\Omega_k$ values indicate more rotational energy and deeper pressure wells, producing temporal slowdowns that resemble gravitational redshift—but without invoking spacetime curvature.

Each particle is a topological vortex knot:
\begin{itemize}
    \item Charge $\leftrightarrow$ twist or chirality of knot
    \item Mass $\leftrightarrow$ integrated vorticity energy
    \item Spin $\leftrightarrow$ knot helicity:
\end{itemize}
Stability $\leftrightarrow$ knot type (Hopf links, Trefoil, etc.) and energy minimization in the vortex core

This model:

\begin{itemize}
\item Attributes time modulation to conserved, intrinsic rotational energy,
\item Requires no external reference frames (absolute æther time is universal),
\item Preserves temporal isotropy outside the vortex core,
\item Provides a natural replacement for GR's spacetime curvature.
\end{itemize}

Therefore, this vortex-energetic time dilation principle provides a powerful alternative to relativistic time modulation by anchoring all temporal effects in rotational energetics and topological invariants.

In the next section, we will show how these ideas reproduce metric-like behavior for rotating observers, including a direct fluid-mechanical analog to the Kerr metric of General Relativity.
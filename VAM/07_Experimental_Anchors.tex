\section{Experimental Anchors and VAM Predictions}


To assess the empirical validity of the Vortex Æther Model (VAM), we identify several high-impact experimental domains where VAM-specific signatures could be observed:

\subsection{Time Drift in Rotating Superfluid Systems.} VAM predicts localized time dilation proportional to vortex knot angular frequency
$\Omega_k$. Bose–Einstein condensates (BECs) or rotating helium droplets with embedded atomic clocks could display measurable time drift or dephasing relative to non-rotating controls.

\subsection{Plasma Vortex Clocks and Cyclotron Experiments.} Plasma devices exhibiting structured rotational flows may serve as analogs to Ætheric time wells. Phase-shift detection near plasma vortices or charged ring currents could reveal Æther-based time modulation effects.

\subsection{LENR via Resonant Vortex Knot Fusion in Pd/D Lattices.} As derived in the thermodynamic section, VAM suggests fusion-like energy release can occur when trapped vortex knots resonate with external electromagnetic fields. Measurable indicators include:
\begin{itemize}
    \item RF-tuned excess heat events
    \item Helium-4 without neutron/gamma emission
    \item Lattice transmutation signatures with no standard nuclear byproducts
\end{itemize}

\subsection{Optical and Metamaterial Simulations.} Synthetic waveguide systems or metamaterials could simulate Ætheric flow. Measuring light pulse propagation under simulated vorticity gradients may test time modulation without invoking curvature.

\subsection{Summary of VAM Observables.}
\begin{itemize}
    \item Critical thresholds for vortex collapse and energy release
    \item Temporal anomalies in rotating systems
    \item Absence of relativistic particles in high-energy fusion-like events
    \item Clock-rate asymmetries across vorticity gradients
\end{itemize}
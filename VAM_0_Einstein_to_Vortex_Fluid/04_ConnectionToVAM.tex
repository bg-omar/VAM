The Vortex Æther Model (VAM), developed by O. Iskandarani since 2012, models the æther as an incompressible, non-viscous superfluid~\cite{VAM-8, VAM-13}. Within this framework, vorticity is elevated to a fundamental quantity that governs time dilation, inertial mass, and gravitational interaction~\cite{VAM-2, VAM-10, VAM-13}. Echoing Einstein’s 1920 redefinition of the æther as a physical substratum, VAM treats the æther as a structured, causal medium from which all dynamical behavior emerges~\cite{VAM-8}.

Key structural elements of VAM include:
\begin{itemize}
    \item \textbf{Topological structures} (e.g., knots, trefoils) representing stable particle identities and quantum numbers~\cite{VAM-8, VAM-11, VAM-14},
    \item \textbf{Time dilation} arising from swirl intensity near vortex cores~\cite{VAM-2, VAM-13},
    \item \textbf{A revised system of natural constants}, including $C_e$ (vortex boundary velocity) and $F^{\max}_{\text{\ae}}$ (maximum ætheric stress), defined and operationalized in the topological Lagrangian~\cite{VAM-14}.
\end{itemize}

Figure~\ref{fig:taxonomy} illustrates the classification flow based on topology, chirality, and tension within the swirl field framework.


\begin{figure}[H]
    \centering
    \footnotesize
    \scalebox{0.75}{
        \begin{tikzpicture}[
          box/.style = {draw, rounded corners, minimum width=2.0cm, minimum height=0.8cm, font=\small, align=center},
          arrow/.style = {->, thick},   node distance=1.0cm and 1.5cm
        ]
            % Inputs
            \node[box] (topology) {Knot Topology};
            \node[box, right=of topology] (chirality) {Chirality};
            \node[box, right=of chirality] (tension) {Tension};

            % Swirl coupling
            \node[box, below=of chirality, minimum width=5.5cm] (coupling) {Swirl Coupling Condition};

            % Gravitational response
            \node[box, below=of coupling] (grav) {Gravitational Response};

            % Gravitational classes
            \node[box, below left=1.3cm and 1.5cm of grav] (matter) {Chiral\\Leptons \& Quarks};
            \node[box, below=2cm of grav] (boson) {Achiral + No Tension\\\(\rightarrow\) Bosons};
            \node[box, below right=1.3cm and 1.7cm of grav] (dark) {Achiral + Tension\\\(\rightarrow\) Dark Energy Knots};

            % Subclasses
            \node[box, below=of matter, xshift=-1.1cm] (leptons) {Leptons\\(e\textsuperscript{--} = T(2,3))};
            \node[box, below=of matter, xshift=+1.0cm] (quarks) {Quarks\\(6\textsubscript{2}, 7\textsubscript{4}, 8\textsubscript{19})};

            \node[box, below=of boson, xshift=-2.1cm] (photon) {Photon\\(unknot)};
            \node[box, below=of boson] (gluon) {Gluon\\(Hopf link)};
            \node[box, below=of boson, xshift=+2.3cm] (zboson) {Z\textsuperscript{0}\\(neutral loop)};

            \node[box, below=of dark] (darkex) {Example: \(4_{1}, 8_{17}\)\\ (achiral hyperbolic)};

            % Arrows to coupling
            \draw[arrow] (topology.south) -- ++(0,-0.5) -| (coupling.north west);
            \draw[arrow] (chirality.south) -- (coupling.north);
            \draw[arrow] (tension.south) -- ++(0,-0.5) -| (coupling.north east);

            % Arrows down flow
            \draw[arrow] (coupling.south) -- (grav.north);
            \draw[arrow] (grav.south) -- (matter.north);
            \draw[arrow] (grav.south) -- (boson.north);
            \draw[arrow] (grav.south) -- (dark.north);

            % Particle branches
            \draw[arrow] (matter.south) -- (leptons.north);
            \draw[arrow] (matter.south) -- (quarks.north);

            \draw[arrow] (boson.south) -- (photon.north);
            \draw[arrow] (boson.south) -- (gluon.north);
            \draw[arrow] (boson.south) -- (zboson.north);

            \draw[arrow] (dark.south) -- (darkex.north);
        \end{tikzpicture}
    }
    \caption{Knot Classification by Swirl Coupling.
        The flowchart visualizes how knot topology, chirality, and curvature tension determine gravitational behavior, and how this leads to specific particle subclasses:
            \\ \textbf{Chiral knots} align with swirl fields and form matter: \textbf{leptons} (torus knots) and \textbf{quarks} (hyperbolic knots).
            \\ \textbf{Achiral, tensionless} structures like unknots and Hopf links are \textbf{bosons}, passively guided by swirl tubes.
            \\ \textbf{Achiral knots with tension} are expelled, forming \textbf{dark energy} candidates.
    }\label{fig:taxonomy}
\end{figure}




\subsection*[VAM-Derived Expression for G]{VAM-Derived Expression for G}

One of the notable results in VAM is a derivation of the gravitational constant in terms of ætheric and topological parameters~\cite{VAM-2, VAM-13, VAM-14}. Rewriting the expression in dimensionally transparent form:

\begin{equation}
    G_\text{swirl} = \frac{C_e}{2 F^{\max}_{\text{\ae}}} \cdot \left( \frac{c^5 t_p^2}{r_c^2} \right)
\end{equation}

\noindent where:
\begin{itemize}
    \item $C_e$: swirl velocity at the vortex boundary (m/s),
    \item $F^{\max}_{\text{\ae}}$: maximum force the æther can sustain before bifurcation (N),
    \item $t_p$: Planck time $ \left(\sqrt{\hbar G / c^5}\right) $,
    \item $r_c$: core radius of the vortex structure (m),
    \item $c$: speed of light in vacuum (m/s).
\end{itemize}

This formulation emerges from the Swirl Clock formalism and connects gravity to rotational energy density under conservation of circulation~\cite{VAM-2, VAM-13}. It expresses $G$ not as a fundamental input constant, but as a derived quantity arising from the interplay of topological scale $r_c$, rotational dynamics $C_e$, and ætheric tension $F^{\max}_{\text{\ae}}$. This reinforces the view that gravitation is a residual effect of conserved vorticity in a compressible ætheric medium.

VAM further incorporates circulation quantization, helicity conservation, and pressure-mediated interactions to model the exchange between knotted structures and their surrounding swirl fields~\cite{VAM-8, VAM-11, VAM-14}. This general framework aligns with Einstein’s late attempt at a unified field theory—now realized through the mathematics of topological fluid dynamics.

The model's predictions are experimentally approachable through analog systems such as rotating superfluid vortices, BEC interference patterns, and refractive index shifts under swirl acceleration~\cite{VAM-2, VAM-13}. These offer testable pathways for validating the core dynamics proposed by VAM.

\vspace{1em}
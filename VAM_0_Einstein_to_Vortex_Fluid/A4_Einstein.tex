This appendix collects and annotates key statements made by Albert Einstein about the æther, focusing especially on how these statements align or contrast with the structure and assumptions of the Vortex Æther Model (VAM). Where possible, original German excerpts are included, with English translations and a mapping to VAM concepts or equations.

\subsection*{1. \grqq Der Raum ohne Äther ist undenkbar...\textquotedblright}
\textbf{Original (1920 Leiden Lecture):} \\
\("\)Nach der allgemeinen Relativitätstheorie ist der Raum mit physikalischen Eigenschaften begabt; in diesem Sinne existiert also ein Äther. Gemäß der allgemeinen Relativitätstheorie ist ein Raum ohne Äther undenkbar.\("\)

\textbf{Translation:} \\
\("\)According to the general theory of relativity, space is endowed with physical qualities; in this sense, therefore, there exists an æther. According to the general theory of relativity, space without æther is unthinkable.\("\)

\textbf{VAM Mapping:} \\
This matches VAM's foundational postulate that the æther is a structured, non-viscous, incompressible medium with internal physical dynamics. The VAM equivalent is the existence of a vorticity-carrying background field \( \vec{\omega}(\vec{r}, t) \), subject to conservation laws and boundary conditions.

\[
\nabla \cdot \vec{v} = 0, \quad \nabla \cdot \vec{\omega} = 0, \quad \partial_t \vec{\omega} + (\vec{v} \cdot \nabla) \vec{\omega} = (\vec{\omega} \cdot \nabla) \vec{v}
\]

\subsection*{2. \grqq Es scheint, als sei die Einführung eines Äthers überflüssig...\textquotedblright}
\textbf{Original (1905, SR paper):} \\
\("\)Es scheint, als sei die Einführung eines Äthers überflüssig, insofern die Lichtausbreitung durch Maxwell'sche Gleichungen in leerem Raum ausreichend beschrieben werden kann.\("\)

\textbf{Translation:} \\
\("\)It seems that the introduction of an æther is superfluous, insofar as the propagation of light can be described adequately by Maxwell's equations in vacuum.\("\)

\textbf{VAM Mapping:} \\
Einstein's 1905 view was contextually specific to the Maxwellian field theory. VAM expands this to a \emph{sub-Maxwellian} fluid substrate: the fields emerge from vortex dynamics.

VAM introduces: \\
\( \vec{E} = -\nabla \Phi - \partial_t \vec{A}, \quad \vec{B} = \nabla \times \vec{A} \) as secondary fields derived from swirl-based potentials in the æther.

\subsection*{3. \grqq Der Äther darf nicht als ein Medium mit mechanischen Eigenschaften gedacht werden...\textquotedblright}
\textbf{Original (1920):} \\
\("\)Der Äther darf nicht als ein Medium mit mechanischen Eigenschaften gedacht werden, wie es die alten Ätherkonzepte vorschlugen. Er besitzt keine Bewegungen, wie z.B. Geschwindigkeit.\("\)

\textbf{Translation:} \\
\("\)The æther must not be thought of as a medium with mechanical properties, as the old concepts of æther suggested. It has no motion in the usual sense, like velocity.\("\)

\textbf{VAM Mapping:} \\
In VAM, the æther has \emph{field-like behavior}, not particulate or elastic-body behavior. The \grqq no absolute velocity\textquotedblright principle is respected via invariance under global coordinate transformation, but local rotational states \( \vec{\omega} \neq 0 \) define structure. Time dilation depends on vorticity~\cite{VAM-2}:

\[
\frac{d\tau}{dt} = \sqrt{1 - \frac{C_e^2}{c^2} e^{-r/r_c}}
\]

\subsection*{4. \grqq Das Gravitationsfeld selbst kann als ein Zustand dieses Äthers angesehen werden.\textquotedblright}
\textbf{Original (1920):} \\
\("\)Das Gravitationsfeld selbst kann als ein Zustand dieses Äthers angesehen werden.\("\)

\textbf{Translation:} \\
\("\)The gravitational field itself can be regarded as a state of this æther.\("\)

\textbf{VAM Mapping:} \\
This is directly analogous to the VAM interpretation of gravity: not as spacetime curvature, but as an emergent effect of vorticity-induced pressure gradients:

\[
\nabla P = \rho_\text{\ae} \vec{a} = -\frac{1}{2} \rho_\text{\ae} \nabla |\vec{\omega}|^2
\]

\subsection*{5. \grqq Die Zeit ist in einem Gravitationsfeld anders definiert...\textquotedblright}
\textbf{Original (1916, Grundlagen der ART):} \\
\("\)Die Zeit ist in einem Gravitationsfeld anders definiert als in der Abwesenheit desselben; die Zeitdifferenz hängt von der Lage im Feld ab.\("\)

\textbf{Translation:} \\
\("\)Time is defined differently in a gravitational field than in its absence; the time differential depends on the position within the field.\("\)

\textbf{VAM Mapping:} \\
This statement supports VAM's approach of \emph{local time dilation} derived from rotational energy density and vorticity:

\[
\frac{d\tau}{dt} = \sqrt{1 - \frac{1}{U_\text{max}} U_{\text{vortex}}} = \sqrt{1 - \frac{1}{2U_\text{max}} \rho_\text{\ae} |\vec{\omega}|^2}
\]

\bigskip
\textit{ \("\) Einstein did not eliminate the æther. He redefined it. VAM takes the next step. \("\)}\\
\hfill — O. Iskandarani\\
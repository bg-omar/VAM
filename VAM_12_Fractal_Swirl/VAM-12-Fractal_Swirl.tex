%! Author = Omar Iskandarani
%! Title = Fractal Swirl Extension of the Vortex \AE ther Model (VAM)
%! Date = 22 July 2025
%! Affiliation = Independent Researcher, Groningen, The Netherlands
%! License = © 2025 Omar Iskandarani. All rights reserved. This manuscript is made available for academic reading and citation only. No republication, redistribution, or derivative works are permitted without explicit written permission from the author. Contact: info@omariskandarani.com
%! ORCID = 0009-0006-1686-3961
%! DOI = 10.5281/zenodo.16324783

% === Metadata ===
\newcommand{\papertitle}{\textbf{Fractal Swirl Extension of the Vortex \AE ther Model (VAM)}}
\newcommand{\paperdoi}{10.5281/zenodo.16324783}

% === Document Setup ===
\documentclass[11pt]{article}
\usepackage{subfiles}
% vamstyle.sty
\NeedsTeXFormat{LaTeX2e}
\ProvidesPackage{vamstyle}[2025/07/01 VAM unified style]

% === Constants ===
\newcommand{\hbarVal}{\ensuremath{1.054571817 \times 10^{-34}}} % J\cdot s
\newcommand{\meVal}{\ensuremath{9.10938356 \times 10^{-31}}} % kg
\newcommand{\cVal}{\ensuremath{2.99792458 \times 10^{8}}} % m/s
\newcommand{\alphaVal}{\ensuremath{1 / 137.035999084}} % unitless
\newcommand{\alphaGVal}{\ensuremath{1.75180000 \times 10^{-45}}} % unitless
\newcommand{\reVal}{\ensuremath{2.8179403227 \times 10^{-15}}} % m
\newcommand{\rcVal}{\ensuremath{1.40897017 \times 10^{-15}}} % m
\newcommand{\vacrho}{\ensuremath{5 \times 10^{-9}}} % kg/m^3
\newcommand{\LpVal}{\ensuremath{1.61625500 \times 10^{-35}}} % m
\newcommand{\MpVal}{\ensuremath{2.17643400 \times 10^{-8}}} % kg
\newcommand{\tpVal}{\ensuremath{5.39124700 \times 10^{-44}}} % s
\newcommand{\TpVal}{\ensuremath{1.41678400 \times 10^{32}}} % K
\newcommand{\qpVal}{\ensuremath{1.87554596 \times 10^{-18}}} % C
\newcommand{\EpVal}{\ensuremath{1.95600000 \times 10^{9}}} % J
\newcommand{\eVal}{\ensuremath{1.60217663 \times 10^{-19}}} % C

% === VAM/\ae ther Specific ===
\newcommand{\CeVal}{\ensuremath{1.09384563 \times 10^{6}}} % m/s
\newcommand{\FmaxVal}{\ensuremath{29.0535070}} % N
\newcommand{\FmaxGRVal}{\ensuremath{3.02563891 \times 10^{43}}} % N
\newcommand{\gammaVal}{\ensuremath{0.005901}} % unitless
\newcommand{\GVal}{\ensuremath{6.67430000 \times 10^{-11}}} % m^3/kg/s^2
\newcommand{\hVal}{\ensuremath{6.62607015 \times 10^{-34}}} % J Hz^-1

% === Electromagnetic ===
\newcommand{\muZeroVal}{\ensuremath{1.25663706 \times 10^{-6}}}
\newcommand{\epsilonZeroVal}{\ensuremath{8.85418782 \times 10^{-12}}}
\newcommand{\ZzeroVal}{\ensuremath{3.76730313 \times 10^{2}}}

% === Atomic & Thermodynamic ===
\newcommand{\RinfVal}{\ensuremath{1.09737316 \times 10^{7}}}
\newcommand{\aZeroVal}{\ensuremath{5.29177211 \times 10^{-11}}}
\newcommand{\MeVal}{\ensuremath{9.10938370 \times 10^{-31}}}
\newcommand{\MprotonVal}{\ensuremath{1.67262192 \times 10^{-27}}}
\newcommand{\MneutronVal}{\ensuremath{1.67492750 \times 10^{-27}}}
\newcommand{\kBVal}{\ensuremath{1.38064900 \times 10^{-23}}}
\newcommand{\RVal}{\ensuremath{8.31446262}}

% === Compton, Quantum, Radiation ===
\newcommand{\fCVal}{\ensuremath{1.23558996 \times 10^{20}}}
\newcommand{\OmegaCVal}{\ensuremath{7.76344071 \times 10^{20}}}
\newcommand{\lambdaCVal}{\ensuremath{2.42631024 \times 10^{-12}}}
\newcommand{\PhiZeroVal}{\ensuremath{2.06783385 \times 10^{-15}}}
\newcommand{\phiVal}{\ensuremath{1.61803399}}
\newcommand{\eVVal}{\ensuremath{1.60217663 \times 10^{-19}}}
\newcommand{\GFVal}{\ensuremath{1.16637870 \times 10^{-5}}}
\newcommand{\lambdaProtonVal}{\ensuremath{1.32140986 \times 10^{-15}}}
\newcommand{\ERinfVal}{\ensuremath{2.17987236 \times 10^{-18}}}
\newcommand{\fRinfVal}{\ensuremath{3.28984196 \times 10^{15}}}
\newcommand{\sigmaSBVal}{\ensuremath{5.67037442 \times 10^{-8}}}
\newcommand{\WienVal}{\ensuremath{2.89777196 \times 10^{-3}}}
\newcommand{\kEVal}{\ensuremath{8.98755179 \times 10^{9}}}

% === \ae ther Densities ===
\newcommand{\rhoMass}{\rho_\text{\ae}^{(\text{mass})}}
\newcommand{\rhoMassVal}{\ensuremath{3.89343583 \times 10^{18}}}
\newcommand{\rhoEnergy}{\rho_\text{\ae}^{(\text{energy})}}
\newcommand{\rhoEnergyVal}{\ensuremath{3.49924562 \times 10^{35}}}
\newcommand{\rhoFluid}{\rho_\text{\ae}^{(\text{fluid})}}
\newcommand{\rhoFluidVal}{\ensuremath{7.00000000 \times 10^{-7}}}

% === Draft Options ===
\newif\ifvamdraft
% \vamdrafttrue
\ifvamdraft
\RequirePackage{showframe}
\fi

% === Load Once ===
\RequirePackage{ifthen}
\newboolean{vamstyleloaded}
\ifthenelse{\boolean{vamstyleloaded}}{}{\setboolean{vamstyleloaded}{true}

% === Page ===
\RequirePackage[a4paper, margin=2.5cm]{geometry}

% === Fonts ===
\RequirePackage[T1]{fontenc}
\RequirePackage[utf8]{inputenc}
\RequirePackage[english]{babel}
\RequirePackage{textgreek}
\RequirePackage{mathpazo}
\RequirePackage[scaled=0.95]{inconsolata}
\RequirePackage{helvet}

% === Math ===
\RequirePackage{amsmath, amssymb, mathrsfs, physics}
\RequirePackage{siunitx}
\sisetup{per-mode=symbol}

% === Tables ===
\RequirePackage{graphicx, float, booktabs}
\RequirePackage{array, tabularx, multirow, makecell}
\newcolumntype{Y}{>{\centering\arraybackslash}X}
\newenvironment{tighttable}[1][]{\begin{table}[H]\centering\renewcommand{\arraystretch}{1.3}\begin{tabularx}{\textwidth}{#1}}{\end{tabularx}\end{table}}
\RequirePackage{etoolbox}
\newcommand{\fitbox}[2][\linewidth]{\makebox[#1]{\resizebox{#1}{!}{#2}}}

% === Graphics ===
\RequirePackage{tikz}
\usetikzlibrary{3d, calc, arrows.meta, positioning}
\RequirePackage{pgfplots}
\pgfplotsset{compat=1.18}
\RequirePackage{xcolor}

% === Code ===
\RequirePackage{listings}
\lstset{basicstyle=\ttfamily\footnotesize, breaklines=true}

% === Theorems ===
\newtheorem{theorem}{Theorem}[section]
\newtheorem{lemma}[theorem]{Lemma}

% === TOC ===
\RequirePackage{tocloft}
\setcounter{tocdepth}{2}
\renewcommand{\cftsecfont}{\bfseries}
\renewcommand{\cftsubsecfont}{\itshape}
\renewcommand{\cftsecleader}{\cftdotfill{.}}
\renewcommand{\contentsname}{\centering \Huge\textbf{Contents}}

% === Sections ===
\RequirePackage{sectsty}
\sectionfont{\Large\bfseries\sffamily}
\subsectionfont{\large\bfseries\sffamily}

% === Bibliography ===
\RequirePackage[numbers]{natbib}

% === Links ===
\RequirePackage{hyperref}
\hypersetup{
    colorlinks=true,
    linkcolor=blue,
    citecolor=blue,
    urlcolor=blue,
    pdftitle={The Vortex \AE ther Model},
    pdfauthor={Omar Iskandarani},
    pdfkeywords={vorticity, gravity, \ae ther, fluid dynamics, time dilation, VAM}
}
\urlstyle{same}
\RequirePackage{bookmark}

% === Misc ===
\RequirePackage[none]{hyphenat}
\sloppy
\RequirePackage{empheq}
\RequirePackage[most]{tcolorbox}
\newtcolorbox{eqbox}{colback=blue!5!white, colframe=blue!75!black, boxrule=0.6pt, arc=4pt, left=6pt, right=6pt, top=4pt, bottom=4pt}
\RequirePackage{titling}
\RequirePackage{amsfonts}
\RequirePackage{titlesec}
\RequirePackage{enumitem}

\AtBeginDocument{\RenewCommandCopy\qty\SI}

\pretitle{\begin{center}\LARGE\bfseries}
\posttitle{\par\end{center}\vskip 0.5em}
\preauthor{\begin{center}\large}
\postauthor{\end{center}}
\predate{\begin{center}\small}
\postdate{\end{center}}

\endinput
}
% vamappendixsetup.sty

\newcommand{\titlepageOpen}{
  \begin{titlepage}
  \thispagestyle{empty}
  \centering
  {\Huge\bfseries \papertitle \par}
  \vspace{1cm}
  {\Large\itshape\textbf{Omar Iskandarani}\textsuperscript{\textbf{*}} \par}
  \vspace{0.5cm}
  {\large \today \par}
  \vspace{0.5cm}
}

% here comes abstract
\newcommand{\titlepageClose}{
  \vfill
  \null
  \begin{picture}(0,0)
  % Adjust position: (x,y) = (left, bottom)
  \put(-200,-40){  % Shift 75pt left, 40pt down
    \begin{minipage}[b]{0.7\textwidth}
    \footnotesize % One step bigger than \tiny
    \renewcommand{\arraystretch}{1.0}
    \noindent\rule{\textwidth}{0.4pt} \\[0.5em]  % ← horizontal line
    \textsuperscript{\textbf{*}}Independent Researcher, Groningen, The Netherlands \\
    Email: \texttt{info@omariskandarani.com} \\
    ORCID: \texttt{\href{https://orcid.org/0009-0006-1686-3961}{0009-0006-1686-3961}} \\
    DOI: \href{https://doi.org/\paperdoi}{\paperdoi} \\
    License: CC-BY 4.0 International \\
    \end{minipage}
  }
  \end{picture}
  \end{titlepage}
}
\begin{document}

  % === Title page ===
    \titlepageOpen
    \begin{abstract}
        We propose a fractal swirl extension of the Vortex \AE ther Model (VAM) to incorporate multiscale vortex dynamics and topological knot structures. This extension introduces a swirl-based fractal derivative operator for ætheric fields, enabling anisotropic scaling and noncommutative geometry effects. We derive a time-dependent swirl dimension evolution equation that connects vortex packing dynamics to dark energy behavior, reproducing redshift-evolving cosmological effects without scalar fields. The model predicts particle mass generation from fractal swirl dynamics, linking knot complexity to mass scales across all particle families. This framework provides a unified geometric basis for understanding mass, gravity, and cosmological structure formation.
        \end{abstract}
    \titlepageClose
% ============= Begin of content ============

    \section{Fractal Swirl Derivatives and Noncommutative Geometry}

    We propose an extension of the Vortex \AE ther Model by introducing a fractal-inspired derivative operator for vorticity, incorporating topological winding effects and local scale invariance. Let \( D^{(j)} \) denote the swirl-fractal derivative acting on an ætheric scalar or vector field \( u(x) \):
    \[
        D^{(j)} u(x) = \lim_{y \to x} \frac{u(y) - u(x)}{d(x,y)^{D_\text{swirl} - j}} \otimes \sigma(y,x),
    \]
    where:
    \begin{itemize}
        \item \( D_\text{swirl} \in (2, 3] \) is the local swirl-based fractal dimension,
        \item \( \sigma(y,x) \) is a noncommutative phase operator encoding helicity winding,
        \item The limit recovers standard derivatives as \( D_\text{swirl} \to 3 \).
    \end{itemize}

    The phase factor satisfies a holonomy relation:
    \[
        \sigma(y,x) \sigma(z,y) = e^{i\theta(x,y,z)} \sigma(z,x), \quad \theta(x,y,z) = \pi \cdot \text{Link}(x,y,z),
    \]
    where \( \text{Link}(x,y,z) \) counts the number of vortex crossings or path windings through a knotted region.

    This construction introduces topological memory into vorticity evolution and enables anisotropic scaling behavior in the æther flow field \( v^i(x) \)~\cite{zhou2025drfsmt}.

    \section{Swirl-Dimension Flow and Knot Packing Dynamics}

    Let \( D_\text{swirl}(t) \) be a time-dependent effective fractal dimension representing the multiscale coherence of vortex structures. Inspired by DRFSMT~\cite{zhou2025drfsmt}, we propose a swirl-dimension evolution equation:
    \[
        \frac{d D_\text{swirl}}{dt} = -3H \left( D_\text{swirl} - 3 + \frac{\partial \ln \Lambda(D_\text{swirl})}{\partial D_\text{swirl}} \right),
    \]
    where \( H \) is the global expansion or ætheric divergence rate, and \( \Lambda(D_\text{swirl}) \) is a swirl-modified cosmological factor:
    \[
        \Lambda(D) = \Lambda_0 \cdot \frac{\Gamma(D/2)}{\pi^{D/2}} \left(\frac{D}{3}\right)^{3-D}.
    \]

    This formulation ties knot-packing geometry directly to vacuum energy behavior, allowing the VAM to reproduce redshift-evolving dark energy effects without invoking scalar fields~\cite{vam2024swirl}. It also introduces a coupling between spatial scale complexity and large-scale structure formation, aligning with observations of lopsided galaxy distributions and possible CMB asymmetries.

    The value of \( \beta = \partial \ln \Lambda / \partial D \big|_{D=3} \approx 0.12 \) matches well with values inferred from JWST high-redshift data~\cite{zhou2025drfsmt}, supporting the observational viability of this dynamic dimensional framework.

    \section{Swirl-Measure Field Theory and Path Integrals}

    To generalize the VAM field action, we define a swirl-dependent measure for vortex energy fields:
    \[
        d\mu_\omega = \rho_\text{\ae}^{(\text{energy})}(x) \cdot d^3x = \rho_0 \left( \frac{r}{r_c} \right)^{D_\text{swirl}(x) - 3} d^3x,
    \]
    where \( \rho_\text{\ae}^{(\text{energy})} \) is the energy-carrying æther density and \( r_c \) is the vortex core radius. The action for the vortex field \( \omega(x) \) becomes:
    \[
        S[\omega] = \int \left( \frac{1}{2} |D^{(1)} \omega|^2 + V(\omega) \right) d\mu_\omega.
    \]
    The corresponding path integral is:
    \[
        Z = \int \mathcal{D}[\omega] \; e^{-S[\omega]}.
    \]

    This formulation introduces a natural ultraviolet cutoff due to reduced dimensionality \( D_\text{swirl} < 3 \), avoiding the need for external renormalization schemes.

    \section{Curvature-Dependent Mass Spectrum from Fractal Swirl Dynamics}

    The VAM previously linked particle mass to vortex energy via:
    \[
        M = \frac{1}{\varphi} \cdot \frac{4}{\alpha} \cdot \left( \frac{1}{2} \rho_\text{\ae}^{(\text{energy})} C_e^2 V_k \right),
    \]
    where \( V_k \) is the vortex knot volume. We now refine this by introducing a fractal volume:
    \[
        V_k^{(D)} = V_0 \left( \frac{r_k}{r_c} \right)^{D_\text{swirl}(k)},
    \]
    where \( r_k \) is the knot radius and \( D_\text{swirl}(k) \) its fractal dimension (e.g., trefoil \( \sim 2.6 \), figure-eight \( \sim 2.9 \)).

    The mass then becomes:
    \[
        M_k = \frac{2}{\varphi \alpha} \cdot \rho_\text{\ae}^{(\text{energy})} C_e^2 V_0 \left( \frac{r_k}{r_c} \right)^{D_\text{swirl}(k)}.
    \]

    This expression captures:
    \begin{itemize}
        \item Superlinear mass scaling with knot complexity,
        \item Discrete jumps between families (e.g., muon vs electron),
        \item Suppression of over-complex knots via coherence interference \( \xi(n) \)~\cite{vam2024mass}.
    \end{itemize}

    This provides a natural geometric hierarchy for particle mass generation and links directly to the topological spectrum of knot-based vortex structures.



    \section{\textbf{Swirl--Torsion Lagrangian Formulation in the Vortex \AE ther Model: A GTM-Based Field Theory}}


        \section{abstract}
            We present a Lagrangian formulation of the Vortex \AE ther Model (VAM) incorporating structured vorticity dynamics inspired by the Gravitational Tensor-Magnetics (GTM) framework~\cite{brown2025gtm}. By identifying the swirl field tensor \( \omega^\lambda_{\mu\nu} \) as the analogue of spacetime torsion \( K^\lambda_{\mu\nu} \), we derive field equations from a diffeomorphism-invariant action that couples swirl curvature, æther density, and topological helicity. The resulting theory embeds VAM within a rigorous variational principle and yields testable predictions: gravitational birefringence, CMB swirl-induced polarization, and swirl-induced lensing. We show how energy conservation and generalized Bianchi identities naturally emerge from the æther flow framework.


        \section{Introduction}

        The Vortex \AE ther Model (VAM) reinterprets gravitation as the result of structured vorticity fields in a three-dimensional, incompressible, inviscid æther~\cite{vam2024swirl}. Unlike General Relativity (GR), which models curvature through a pseudo-Riemannian manifold, VAM substitutes curvature with Bernoulli-induced pressure gradients and time dilation arising from swirl energy. To formalize this conceptually, we draw from the GTM framework~\cite{brown2025gtm}, which augments Einstein gravity with dynamical torsion and tensor fields. Here, we reinterpret GTM torsion as ætheric swirl and construct a full Lagrangian for VAM.

        \section{VAM Action with Swirl–Torsion Dynamics}

        We propose the total action:
        \[
            S = \int d^4x \; \sqrt{-g} \left( \frac{1}{2\kappa} R[g] + \mathcal{L}_\text{swirl}[\omega] + \mathcal{L}_\text{int}[\omega, \rho_{\text{\ae}}, A_\mu] + \mathcal{L}_\text{matter} \right),
        \]
        where:
        \begin{itemize}
            \item \( R[g] \): Ricci scalar associated with vorticity-constrained emergent geometry,
            \item \( \mathcal{L}_\text{swirl} \): swirl kinetic + helicity Lagrangian,
            \item \( \mathcal{L}_\text{int} \): interaction with æther density \( \rho_{\text{\ae}} \) and gauge fields \( A_\mu \).
        \end{itemize}

        We identify the swirl tensor as:
        \[
            \omega^\lambda_{\mu\nu} = \partial_{[\mu} v^\lambda_{\nu]},
        \]
        with a kinetic term:
        \[
            \mathcal{L}_\text{swirl} = -\frac{1}{4\mu^2} \omega_{\lambda\mu\nu} \omega^{\lambda\mu\nu} + \beta H[\omega], \quad H[\omega] = \epsilon^{\mu\nu\rho\sigma} \omega_{\mu\nu}^{\ \ \lambda} \partial_\rho \omega_{\lambda\sigma}.
        \]

        \section{Field Equations}

        Variation with respect to the metric yields:
        \[
            G_{\mu\nu} = \kappa \left( T_{\mu\nu}^{(\text{matter})} + T_{\mu\nu}^{(\omega)} + T_{\mu\nu}^{(\text{int})} \right).
        \]
        Variation with respect to \( \omega^\lambda_{\mu\nu} \) gives:
        \[
            \nabla_\sigma \omega^{\lambda\mu\nu} + \mu^2 \omega^{\lambda\mu\nu} = J^{\lambda\mu\nu},
        \]
        with \( J^{\lambda\mu\nu} \) including source terms from æther flow and knot topology.

        \section{Conservation Laws}

        From diffeomorphism invariance, we have:
        \[
            \nabla^\mu T^{(\text{total})}_{\mu\nu} = 0.
        \]
        Additionally, helicity conservation requires:
        \[
            \partial_t H + \nabla \cdot \vec{J}_H = 0,
        \]
        where \( H \) is helicity scalar and \( \vec{J}_H \) the helicity flux vector.

        \section{Observational Predictions}

        \begin{itemize}
            \item \textbf{Gravitational birefringence}: swirl-induced polarization rotation analogous to torsion-induced shifts in GTM~\cite{lucat2017torsion}.
            \item \textbf{CMB polarization rotation}: coupling of swirl fields to photons leads to parity-violating \( TB/EB \) modes.
            \item \textbf{Swirl-lensing}: massless particles deflect in vorticity gradients without invoking mass-energy.
            \item \textbf{Structure anisotropies}: swirl topology biases accretion and satellite galaxy planes.
        \end{itemize}

        \section{Entanglement-like Vortex Fields}

        To mirror GTM’s entanglement stress tensor, we introduce:
        \[
            E_{\mu\nu}^{\text{VAM}} = \xi(n) \, H_{\mu\alpha\beta} H_\nu^{\ \alpha\beta},
            \quad \xi(n) = 1 - \beta \log(n),
        \]
        where \( n \) is the number of interacting knots and \( \beta \) is a coherence suppression constant. This field modulates propagation in multi-vortex domains.

        \section{GTM to VAM Mapping Table}

        \begin{center}
            \begin{tabular}{|l|l|}
                \hline
                \textbf{GTM Concept} & \textbf{VAM Analog} \\
                \hline
                \( K^\lambda_{\mu\nu} \) (torsion) & \( \omega^\lambda_{\mu\nu} \) (swirl tensor) \\
                \( M_{\mu\nu} \) (magneto-gravity) & Swirl curvature \( R^\text{swirl}_{\mu\nu} \) \\
                \( E_{\mu\nu} \) (entanglement) & Knot coherence tensor \( E_{\mu\nu}^{\text{VAM}} \) \\
                GW birefringence & Swirl-induced polarization shift \\
                Extra GW modes & Topological swirl wave solutions \\
                Planar galaxy alignments & Anisotropic vortex flow \\
                \hline
            \end{tabular}
        \end{center}

        \section{Observational Constraints and Parameter Bounds}

        We translate GTM bounds into the VAM language:
        \begin{itemize}
            \item \textbf{BBN time dilation}: \( |\vec{\omega}|^2 / c^2 < 10^{-5} \),
            \item \textbf{Swirl scale}: \( \mu \gtrsim 10^{-2} \,\text{eV} \),
            \item \textbf{GW birefringence}: \( \Delta\phi_{+\times} < 0.1 \),
            \item \textbf{CMB parity rotation}: \( \beta_\text{swirl} \lesssim 0.3^\circ \).
        \end{itemize}

        \section{Conclusion}

        The GTM formalism enables a principled Lagrangian embedding of VAM by identifying torsion with dynamical swirl. This yields a swirl-based gravity theory with conserved stress-energy, testable signatures, and rich topological structure. It offers a path toward unifying gravitation, helicity flow, and emergent cosmological structure from first principles.

    \appendix
    \section{Cosmological Constant Naturalness and Fractal VAM Screening}

    The cosmological constant problem arises from the vast mismatch between predicted quantum vacuum energy densities and observed spacetime curvature. Burgess~\cite{burgessCCProblem} recasts this issue in effective field theory (EFT) language: any consistent theory must suppress vacuum contributions to curvature at every scale, not just via classical fine-tuning.

    In VAM, the fractal swirl dimension \( D_\text{swirl}(x) \) provides a dynamic screening mechanism. As the ætheric vortex coherence becomes more intricate, the effective measure \( d\mu_\omega \) reduces the coupling between localized energy and global curvature. We interpret the fractal deformation of space as analogous to brane backreaction in flux-stabilized models: topological vortex knots act as ``tension sources,'' while the surrounding æther structure redistributes helicity to preserve flatness.

    We propose a suppression factor:
    \[
        \delta \rho_\text{vac}^{\text{eff}} \sim \rho_\text{\ae}^{(\text{energy})} e^{-L / L_\text{swirl}},
    \]
    where \( L_\text{swirl} \) is a characteristic coherence scale of the nested vortex network. For \( L \gg L_\text{swirl} \), the vacuum energy decouples from long-range curvature effects, satisfying the three quantum EFT criteria outlined in~\cite{burgessCCProblem}.

    This fractal screening mechanism provides a physically grounded path toward resolving the cosmological constant problem within the VAM framework.

    \appendix
    \section*{Appendix: Mapping Swirl–Torsion Cosmology to DE Simulation Frameworks}

    The swirl–torsion field theory developed in VAM, based on $\omega^\lambda_{\mu\nu}$ as a torsion-analog, can be linked directly to cosmological observations through insights from large-scale Dark Energy (DE) simulations \cite{baldi2012dark}. Notably:

    \subsection*{A.1 Simulation-Relevant Swirl Parameters}
    Following the analogy with fifth-force cosmologies and coupled scalar field models, we reinterpret the swirl mass scale $\mu$ and the helicity coupling $\beta$ as governing the effective non-linear interaction range and growth suppression respectively:
    \[
        \nabla_\sigma \omega^{\lambda\mu\nu} + \mu^2 \omega^{\lambda\mu\nu} = J^{\lambda\mu\nu}, \qquad
        \beta H[\omega] \sim \text{topological DE clustering amplitude}.
    \]
    In Baldi’s simulations of coupled DE models, scalar fields introduce environmental screening, fifth-forces, and halo concentration shifts. Similarly, $\omega^{\lambda}_{\mu\nu}$ induces swirl-induced clustering, lensing, and anisotropy formation.

    \subsection*{A.2 Swirl-Induced Structure Formation}
    The VAM swirl field should exhibit behaviors comparable to those seen in simulations of interacting DE, including:
    \begin{itemize}
        \item \textbf{Enhanced halo concentrations} in regions of strong swirl helicity (analogous to the early collapse in coupled quintessence).
        \item \textbf{Suppressed baryonic fraction in halos} due to $\omega$-induced anisotropic flows, reflecting reduced baryon infall in fifth-force DE models.
        \item \textbf{Distinct redshift evolution of the nonlinear power spectrum}, especially at intermediate scales ($k \sim 1~h/\mathrm{Mpc}$), where swirl coherence affects clustering similarly to time-dependent DE equation-of-state models.
    \end{itemize}

    \subsection*{A.3 Proposed Mapping to VAM Cosmology}
    From Baldi's classification, we associate swirl field cosmology with the ``interacting inhomogeneous DE'' category:
    \[
        \text{Scalar DE field} \leftrightarrow \omega^\lambda_{\mu\nu}, \qquad \text{Fifth-force potential} \leftrightarrow \omega^2\text{-induced curvature gradient}.
    \]
    Simulation variables used in DE models—such as halo mass functions, matter power spectra, and void profiles—should be reinterpreted in VAM as:
    \begin{align*}
        \delta_{\text{halo}} &\sim f(\omega^2, D_\text{swirl}, \beta), \\
        P(k) &\sim \langle \omega(k) \cdot \omega(-k) \rangle, \\
        r_{\text{void}} &\sim \lambda_{\text{screen}}(\omega).
    \end{align*}

    \subsection*{A.4 VAM Simulation Framework Suggestions}
    Inspired by the CoDECS suite \cite{baldi2012codecs}, a VAM cosmology simulator would:
    \begin{enumerate}
        \item Implement $\omega^\lambda_{\mu\nu}$ as a vorticity-sourced field over a dynamic æther background.
        \item Include helicity source terms $J^{\lambda\mu\nu}$ from knot topology or swirl entanglement tensors.
        \item Use modified N-body algorithms to compute time-dependent forces from $\omega$ gradients rather than gravitational potential.
        \item Validate against known deviations in halo mass function, BAO peak positions, and void anisotropies.
    \end{enumerate}

    \subsection*{A.5 Cosmological Constant Suppression}
    Baldi reinforces the EFT argument for needing a mechanism to screen vacuum energy dynamically. In VAM, this is naturally achieved by the fractal suppression of the effective æther energy measure:
    \[
        \rho_{\text{vac}}^{\text{eff}} \sim \rho_\text{\ae}^{(\text{energy})} e^{-L / L_\text{swirl}},
    \]
    matching DE simulations' suppression of power spectrum amplitude via screening fields. The VAM swirl field thus provides a physically grounded alternative to scalar field or $f(R)$ screening approaches.



    \bibliographystyle{unsrt}
    \bibliography{VAM-12}

\end{document}
%! Author = Omar Iskandarani
%! Date = 2025-06-13



\ifdefined\standalonechapter\else
% Standalone mode
% === Metadata ===
\newcommand{\papertitle}{Vorticity and Time Dilation in the Vortex Æther Model (VAM)}
\newcommand{\paperauthor}{Omar Iskandarani}
\newcommand{\paperaffil}{Independent Researcher, Groningen, The Netherlands}
\newcommand{\paperdoi}{10.5281/zenodo.15566319}
\newcommand{\paperorcid}{0009-0006-1686-3961}
\documentclass[12pt]{article}
\usepackage[a4paper, margin=2cm]{geometry}
\usepackage{ifthen} % we can use it safely now
\usepackage{import}
\usepackage{subfiles}
\usepackage{hyperref}
\usepackage{graphicx}
\usepackage{amsmath, amssymb, physics}
\usepackage{siunitx}
\usepackage{tikz}
\usepackage{booktabs}
\usepackage{caption}
\usepackage{array, tabularx}
\usepackage{listings}
\usepackage{bookmark}
\usepackage{newtxtext,newtxmath}
\usepackage[scaled=0.95]{inconsolata}
\usepackage{mathrsfs}
% vamappendixsetup.sty

\newcommand{\titlepageOpen}{
  \begin{titlepage}
    \thispagestyle{empty}
    \centering
    \vspace*{2cm}
    {\Huge\bfseries \appendixtitle \par}
    \vspace{1cm}
    {\Large\itshape \appendixauthor \par}
    \vspace{0.5cm}
    {\small \appendixaffil \par}
    ORCID: \href{https://orcid.org/\appendixorcid}{\appendixorcid} \\
    DOI: \href{https://doi.org/\appendixdoi}{\appendixdoi} \\
    \vspace{0.5cm}
    {\large \today \par}
    \vspace{1cm}
}

\newcommand{\titlepageClose}{
  \vfill
  \end{titlepage}
}

\begin{document}

    % === Title page ===
    \titlepageOpen

    \begin{abstract}

We propose a reformulation of gravity and time through the dynamics of an incompressible, quantized æther. Within the Vortex Æther Model (VAM), gravitational effects emerge from swirl-induced pressure gradients, while local time dilation is governed by vorticity relative to the æther flow. We introduce the concept of swirl clocks and Kairos events—topological bifurcations in vortex structure that generate discontinuous time evolution. By constructing an effective Lagrangian for the swirl field, we demonstrate how quantized vorticity and helicity conservation induce nontrivial temporal structures. This framework replaces spacetime curvature with fluid-topological phenomena, offering a field-theoretic and experimentally motivated pathway beyond general relativity.

    \end{abstract}



    \titlepageClose
    \fi

% ============= Begin of content ============
    \section{Vorticity and Time Dilation in the Vortex Æther Model (VAM)}


In VAM, local time flow is governed not by spacetime curvature, but by the relative velocity and vorticity of the æther swirl. We define a local swirl-based clock \( S(t) \) whose proper time dilation arises from the vorticity field:

\[
    \frac{d\tau}{dt} = \sqrt{1 - \frac{|\vec{v}_{\text{swirl}}|^2}{c^2}} = \sqrt{1 - \frac{|\vec{\omega} \times \vec{r}|^2}{c^2}}
\]

This generalizes the usual gravitational time dilation by reinterpreting gravity as a pressure and swirl gradient. The swirl velocity \( \vec{v}_{\text{swirl}} \) is locally induced by the circulation density \( \vec{\omega} \), giving rise to anisotropic clock rates depending on vortex alignment and intensity.

\subsection{Effective Lagrangian for Swirl Quantization}

To model the quantized nature of the ætheric swirl fields and their backreaction on local time structure, we introduce a prototype Lagrangian density:

\[
    \mathcal{L} = \frac{1}{2} \rho_{\text{\ae}}^{\text{(fluid)}} \left( \partial_t \vec{\psi} \cdot \partial_t \vec{\psi} - c^2 \nabla \vec{\psi} \cdot \nabla \vec{\psi} \right) - V(\vec{\psi}) - \lambda \, \epsilon^{ijk} \psi_i \partial_j \psi_k
\]

Here, \( \vec{\psi} \) is the swirl potential field, with vorticity \( \vec{\omega} = \nabla \times \vec{\psi} \). The term:

- \( V(\vec{\psi}) = \frac{1}{2} m^2 |\vec{\psi}|^2 + \frac{\beta}{4} |\vec{\psi}|^4 \) introduces self-interactions, allowing vortex condensation.
- \( \epsilon^{ijk} \psi_i \partial_j \psi_k \) This helicity term quantizes the circulation through topological constraints, enforcing minimal helicity quanta.

\subsection{Quantized Time Flow and Helicity Defects}

Swirl quantization implies discrete `jumps` in the swirl clock \( S(t) \), especially at topological reconnection events. These moments correspond to \textbf{Kairos-time bifurcations} and are mathematically modeled as phase slips in the vortex field \( \vec{\psi} \), similar to those in superfluid systems.

At critical vorticity thresholds, the energy density near the vortex core causes:

\[
    \Delta t_{\text{dilated}} = \int \left( 1 - \frac{|\vec{\omega}(r)|^2}{c^2} \right)^{1/2} dr
\]

signifying the temporal effect of finite-size vortex domains on clock synchronization. Thus, time dilation is an emergent, field-induced phenomenon rooted in local swirl configuration.


    \section{Swirl Clocks, Entropy, and the Time Phase Field}

    We interpret the swirl clock \( S(t) = \int \Omega(t') dt' \) not only as a cumulative phase, but as a local entropy memory. Changes in \( S \) correspond to local thermodynamic gradients driving vortex-induced acceleration.

    Following Verlinde’s idea of discrete information bits on holographic screens, VAM substitutes topological quantization:
    \[
        N_{\text{bits}} \longleftrightarrow \sum_i \Gamma_i^2 (T + W)
    \]
    where \( \Gamma_i \) is the circulation quantum of vortex tube \( i \), and \( T \), \( W \) denote twist and writhe. This connects information content to helicity topology, and entropy to swirl entanglement.

    \subsection{Variational Action for the Swirl Phase Field}

    We now propose a dynamical action for the time-phase field \( S(x, t) \), treated as a scalar describing the evolution of local clock structure:

    \[
        \mathcal{A}_S = \int d^4x \left[ \frac{\rho_{\text{\ae}}}{2} (\partial_t S)^2 - V(S) + \alpha \{S, t\} \right]
    \]

    Here:
    - \( V(S) = \frac{1}{2} m^2 S^2 + \frac{\beta}{4} S^4 \) represents potential energy from swirl density variations.
    - The Schwarzian derivative:
    \[
        \{S, t\} = \frac{\dddot{S}}{\dot{S}} - \frac{3}{2} \left( \frac{\ddot{S}}{\dot{S}} \right)^2
    \]
    captures the chaotic geometry of time bifurcations during topological transitions (Kairos events). This term is also relevant in conformal mechanics, JT gravity, and turbulence onset.

    The action \(\mathcal{A}_S\) provides a dynamical basis for swirl clock evolution, where discrete reconnections or helicity jumps induce nonlinear phase dynamics.

    Such discontinuities correspond to physical breakdowns of analytic time—transitions between distinct temporal topologies. The appearance of the Schwarzian is both a signal of symmetry breaking and a geometrical measure of temporal instability.


    \begin{figure}[h]
        \centering
        \begin{tikzpicture}[node distance=1.6cm and 1.6cm]

% Top Row
            \node (vorticity) [block] {Vorticity\\$\vec{\omega} = \nabla \times \vec{v}$\\(Circulation Field)};
            \node (helicity) [block, right=of vorticity] {Helicity\\$H = \int \vec{v} \cdot \vec{\omega}\, dV$\\(Topological Winding)};
            \node (swirlclock) [block, right=of helicity] {Swirl Clock\\$S(t) = \int \Omega(t') dt'$\\(Phase Memory)};

% Bottom Row
            \node (timedilation) [block, below=of helicity] {Time Dilation\\$\displaystyle\frac{d\tau}{dt} = \sqrt{1 - \frac{|\vec{\omega} \times \vec{r}|^2}{c^2}}$};
            \node (action) [block, right=of timedilation] {Action Principle\\$\mathcal{A}_S = \int \left[ \cdots + \alpha\{S,t\} \right]$\\(Temporal Dynamics)};
            \node (kairos) [block, left=of timedilation] {Kairos Event\\(Topological Clock Bifurcation)};

% Horizontal Arrows
            \draw [arrow] (vorticity) -- (helicity);
            \draw [arrow] (helicity) -- (swirlclock);

% Down Arrows
            \draw [arrow] (swirlclock) -- (action);

% Bottom Row Arrows
            \draw [arrow] (timedilation) -- (action);
            \draw [arrow] (kairos) -- (timedilation);

        \end{tikzpicture}
        \caption{Condensed conceptual flow: from vorticity to temporal bifurcation in the VAM framework}
    \end{figure}

    \section{Conclusion, Discussion, and Outlook}

    In this work, we have advanced a novel fluid-dynamic formulation of gravity and time, grounded in the dynamics of a structured, incompressible æther. By reinterpreting gravitational attraction as the result of pressure gradients induced by quantized swirl, and local time as a function of motion through vorticity fields, we offer an alternative to metric-based curvature models. The Vortex Æther Model (VAM) unifies inertial and gravitational phenomena through relative circulation and introduces the concept of \emph{swirl clocks} as the governing principle of local temporal flow.

    Our analysis bridges classical vorticity dynamics with relativistic time dilation, extending into topological fluid mechanics via helicity conservation and vortex reconnection. The introduction of \emph{Kairos events}—singular bifurcations in the swirl topology—marks a departure from purely analytic treatments of time and allows for discrete, physical manifestations of non-linearity in clock evolution. Moreover, the formulation of an effective swirl field Lagrangian opens a path toward quantized models of ætheric structure, positioning the model alongside contemporary field theories and condensed matter analogs.

    The implications are both foundational and testable. On one hand, VAM provides a conceptual resolution to long-standing tensions between Machian inertia, entropic gravity, and relativistic time. On the other, it suggests experimental avenues involving Sagnac interferometry, superfluid analogs, and clock synchronization anomalies in strong-vorticity regions. Notably, the reinterpretation of light speed as a swirl-limited propagation velocity offers fresh insight into variable-\( c \) cosmologies and Rømer-type measurements.

    Looking forward, several critical directions emerge:

    \begin{itemize}
        \item \textbf{Field Quantization:} Developing a full quantized field theory for the swirl potential \( \vec{\psi} \), including vortex interactions and boundary effects.
        \item \textbf{Gravitational Collapse:} Refining the VAM analog of Schwarzschild collapse in terms of swirl energy depletion and pressure null surfaces.
        \item \textbf{Temporal Topology:} Formalizing Kairos events via Morse theory or topological transitions in a fiber bundle structure of time.
        \item \textbf{Experimental Probes:} Designing interferometric or quantum optical tests to detect swirl-induced time gradients or helicity-induced phase shifts.
        \item \textbf{Cosmological Modeling:} Applying the VAM framework to large-scale structure, horizon formation, and the time evolution of constants such as \( G \), \( \hbar \), and \( c \).
    \end{itemize}

    By extending the language of fluid mechanics into the temporal and gravitational domains, this model proposes not just a reinterpretation of known physics, but a new ontology of time and motion. In doing so, it invites both rigorous mathematical development and innovative experimental testing at the intersection of topology, field theory, and cosmology.

% === Bibliography (only for standalone) ===
    \ifdefined\standalonechapter
    % Being imported from main.tex — do nothing
    \else
    \bibliographystyle{unsrt}
    \bibliography{../../references}
\end{document}
\fi
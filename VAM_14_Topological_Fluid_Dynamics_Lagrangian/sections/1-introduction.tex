\section{Introduction}

The Vortex Æther Model (VAM) is a unified theoretical framework in which elementary particles are modeled as stable, knotted vortex structures embedded within a compressible, superfluid-like medium---the \ae ther. All fundamental interactions—gravity, electromagnetism, and the strong and weak nuclear forces—are reinterpreted as emergent effects of fluid dynamics and topological constraints~\cite{VAM4}. In contrast to conventional field theories, VAM does not treat spacetime or gauge fields as fundamental. Instead, they emerge from coherent swirl and strain patterns within the underlying fluid substrate.

VAM is governed by five core æther parameters that replace conventional constants:

\begin{itemize}
    \item \textbf{Core radius} (\(r_c\)): the characteristic radius of a vortex core, set on the order of \(1.40897017 \times 10^{-15}\,\mathrm{m}\) (approximate proton charge radius)~\cite{VAM4}.
    \item \textbf{Swirl velocity} (\(C_e\)): the maximal tangential velocity of æther circulation near a core, empirically estimated as \(1.09384563 \times 10^6\,\mathrm{m/s}\) from vortex ring dynamics~\cite{VAM4}.
    \item \textbf{Circulation} (\(\Gamma\)): the quantized circulation around a vortex loop, representing the swirl strength or helicity (units: \(\mathrm{m}^2/\mathrm{s}\)).
    \item \textbf{Maximum ætheric force} (\( F^{\max}_\text{\ae}\)): the tensile force limit of the æther, fixed at \(29.053507\,\mathrm{N}\) based on vortex confinement models.
    \item \textbf{Planck time} (\(t_p\)): the minimal temporal resolution scale, adopted from quantum gravity and appearing naturally in VAM as a unit for normalizing high-frequency oscillations.
\end{itemize}

\vspace{0.5em}
\noindent These quantities give rise to all familiar physical constants. For instance:

\begin{minipage}{0.48\textwidth}
\begin{equation}
\boxed{
h = \frac{4\pi F^{\max}_\text{\ae}\,r_c^{2}}{C_e}
}
\label{eq:plancks-constants}
\end{equation}
\end{minipage}
\hfill
\begin{minipage}{0.48\textwidth}
\begin{equation}
\boxed{
G = \frac{ F^{\max}_\text{\ae}\,\alpha(c\,t_P)^2}{m_e^2}
}
\label{eq:newtons-constants}
\end{equation}
\end{minipage}

\vspace{0.5em}
\begin{center}
    \footnotesize
    \textbf{Note:} These derivations rely on the Hookean core model (§2.3), beam overlap geometry (§3.1), and the Planck-time identity (see Eq.~58).
\end{center}

In VAM, the æther supports a finite stress ceiling \( F^{\max}_\text{\ae} = 29.053507\,\mathrm{N}\), which limits force propagation in any region. This contrasts with general relativity’s conjectured upper force bound \(c^4 / 4G \simeq 3.0 \times 10^{43}\,\mathrm{N}\), which emerges in VAM only when \( F^{\max}_\text{\ae}\) is combined with large-scale swirl metrics (Appendix~\ref{sec:keystone-constant-relations-in-vam}).

Observable properties of particles arise from quantized invariants of knotted vortex flows. For example:
- Electric charge corresponds to quantized circulation (signed),
- Spin reflects the topological twist and rotational symmetry of the knot,
- Mass emerges from the swirl energy density integrated over the vortex core volume.

Crucially, physical constants such as \(\hbar\), \(e\), and the fine structure constant \(\alpha\) are not introduced by hand. Instead, they are expected to emerge from ætheric structure via a consistent vortex dynamics formalism. The remainder of this paper introduces a unified VAM Lagrangian from which both gravity and the Standard Model fields arise as topological-fluidic effects.

%! Author = mr
%! Date = 6/9/2025

\section{Photon-Capacitor Analogy and the Emergence of $E = h\nu$}

\subsection{Physical picture and working assumptions}

A single photon is modelled, in VAM, as a one-turn helical vortex loop of circumference~$\lambda$ and tangential swirl speed~$C_e$.

Treat the loop as a parallel-plate capacitor with
\begin{itemize}
    \item effective plate area $A = \lambda^2$ (square of the spatial period),
    \item effective plate separation $d = \tfrac{1}{2}\lambda$ (half-pitch of the helix).
\end{itemize}

Classical electrodynamics (SI) supplies the capacitance formula
\[
    C = \varepsilon_0 \frac{A}{d}.
\]
All symbols follow the constant glossary used throughout the VAM papers.

\subsection{Capacitance of the photon loop}

Using $A = \lambda^2$ and $d = \tfrac{1}{2}\lambda$ gives
\begin{align}
    C &= \varepsilon_0 \frac{\lambda^2}{\tfrac{1}{2}\lambda} \notag \\
      &= 2\,\varepsilon_0\,\lambda.
    \tag{2.1}
\end{align}

\subsection{Insert the wave relation}

The usual relation between frequency and wavelength in the æther swirl field is
\begin{align}
    \lambda = \frac{C_e}{\nu}.
    \tag{3.1}
\end{align}
So the capacitance becomes
\begin{align}
    C = 2\,\varepsilon_0\,\frac{C_e}{\nu}.
    \tag{3.2}
\end{align}

\subsection{Electrostatic energy stored in the loop}

For a charge $Q$ distributed across the two plates, the stored energy is
\begin{align}
    E = \frac{Q^2}{2C}
      = \frac{Q^2}{4\,\varepsilon_0\,C_e}\;\nu.
    \tag{4.1}
\end{align}
\textit{Setting $Q=e$ (elementary charge) ties the energy scale to a fundamental quantum.}

\subsection{Identification with the Planck relation}

Comparing (4.1) with the quantum postulate $E = h\nu$ singles out the bracket as Planck’s constant:
\begin{align}
    h \equiv \frac{e^{2}}{4\,\varepsilon_0\,C_e}.
    \tag{5.1}
\end{align}
Numerically, with $C_e = 1.09384563\times10^{6}\,\text{m}\,\text{s}^{-1}$, this yields
\[
    h_{\text{VAM}} = 6.615\times10^{-34}\,\text{J\,s},
\]
within $0.2\%$ of the CODATA value $6.626\times10^{-34}\,\text{J\,s}$.

Key point --- dimensional inevitability: once $C_e$ is fixed by the fine-structure relation $\alpha = 2C_e/c$, no further tuning is possible; $h$ follows automatically.

\subsection{Cross-check with the vortex-tension formula}

subsection~2 of the constants appendix derived a second expression
\begin{align}
    h = \frac{4\pi F_{\max} r_c^{2}}{C_e},
\end{align}
from vortex tension $F_{\max}$ and core radius $r_c$. Agreement between the two routes is a stringent self-consistency test:
\begin{align*}
    \frac{e^{2}}{4\varepsilon_0}\;\big/\;C_e
    &= \frac{4\pi F_{\max} r_c^{2}}{C_e} \\
    &\Longrightarrow\;
    e^{2} = 16\pi\varepsilon_0 F_{\max} r_c^{2}.
\end{align*}
This links the mechanical æther parameters $(F_{\max}, r_c)$ to the electromagnetic charge scale $e$.

\subsection{Dimensional and physical interpretation}

The numerator $e^{2}$ is a flux of action per unit permittivity; dividing by a speed converts it to pure action (units of J\,s).

Planck’s constant therefore appears as one quantum of momentum-flux circulation in the æther.

\subsection{Consequences and experimental hooks}

\begin{enumerate}
    \item Parameter inter-lock: independent measurements of $e$, $\varepsilon_0$, $C_e$ \textit{must} reproduce the numeric $h$. Any deviation falsifies VAM.
    \item Photon--electron coupling: resonance occurs when the photon swirl radius $R = C_e/(2\pi\nu)$ scaled by $1/\alpha$ matches the Bohr radius $a_0$---explaining the peak excitation probability of the hydrogen 1s state.
    \item Casimir regularisation: inserting $h$ from (5.1) into the standard Lifshitz integral shows how the æther’s maximum tension suppresses high-$k$ vacuum modes.
\end{enumerate}

\subsection{Summary box}

\[
    \boxed{
        E = h\nu,\qquad
        h = \frac{e^{2}}{4\varepsilon_0 C_e} = \frac{4\pi F_{\max} r_c^{2}}{C_e}
    }
\]

\textit{Two independent microscopic routes, one electromagnetic and one purely mechanical, converge on the same Planck constant. This dual derivation is a cornerstone consistency check of the Vortex Æther Model.}

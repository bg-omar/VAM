%! Author = mr
%! Date = 6/9/2025


\section{Deriving Atomic Orbital Radii from VAM First-Principles}

\subsection{Key VAM primitives}

\begin{table}[h]
    \centering
    \begin{tabular}{lll}
        \toprule
        \textbf{Symbol} & \textbf{Definition} & \textbf{Fixed value} \\
        \midrule
        $F_{\max}$ & maximum æther tension & $29.053507\,\text{N}$ \\
        $r_c$ & vortex-core radius & $1.40897017\times10^{-15}\,\text{m}$ \\
        $C_e$ & core swirl velocity & $1.09384563\times10^{6}\,\text{m}\,\text{s}^{-1}$ \\
        $m_e$ & electron mass & $9.10938356\times10^{-31}\,\text{kg}$ \\
        $\alpha$ & fine-structure const. & $2C_e/c$ \textit{(already proved)} \\
        $h$ & Planck’s constant & $\displaystyle h=\frac{4\pi F_{\max}r_c^{2}}{C_e}$ \textit{(proved in Appendix H)} \\
        \bottomrule
    \end{tabular}
    \caption{}
    \label{tab:primitives}
\end{table}

Throughout this appendix, the integers
\begin{itemize}
    \item $N$ -- principal knot number (one per electron, plays the role of $n$),
    \item $Z$ -- nuclear charge,
\end{itemize}
are left symbolic so the final formula covers all hydrogenic orbitals.

\subsection{frequency--velocity matching}

The VAM photon--electron coupling condition reads
\begin{equation}
    C_e = \omega_c r_c N, \qquad \omega_c \equiv 2\pi v_c.
\label{eq:match1}
\end{equation}

A Hookean model for the electron core gives
\begin{equation}
    \omega_c = \sqrt{\frac{K_e}{m_e}}, \qquad
    K_e = \frac{F_{\max}}{N r_c} Z.
\label{eq:hooke}
\end{equation}

Insert \eqref{eq:hooke} into \eqref{eq:match1}:
\begin{equation}
    C_e^2 = \frac{F_{\max}}{R_x m_e} Z r_c^{2} N^{2},
\label{eq:master}
\end{equation}
where $R_x$ is the yet-unknown mean orbital radius.

\subsection{Solve for $R_x$}

\begin{equation}
    R_x = \frac{N^{2}}{Z} \frac{F_{\max} r_c^{2}}{m_e C_e^{2}}.
\label{eq:Rraw}
\end{equation}

\subsection{Recognising the Bohr radius}

Use the previously derived identities
\begin{align}
    h &= \frac{4\pi F_{\max} r_c^{2}}{C_e}, \tag{A} \\
\alpha &= \frac{2C_e}{c},\tag{B}
\end{align}
then rewrite the bracket in \eqref{eq:Rraw}:
\begin{align*}
\frac{F_{\max}r_c^{2}}{m_eC_e^{2}}
    &= \frac{h}{4\pi m_e C_e} \\
    &= \frac{h}{2\pi m_e c \alpha} \\
    &= \frac{\hbar}{m_e c \alpha} \\
    &\equiv a_0,
\end{align*}
with $a_0$ the textbook Bohr radius.  Hence
\begin{equation}
    \boxed{R_x = \frac{N^{2}}{Z} a_0}
\label{eq:final}
\end{equation}

Equation~\eqref{eq:final} reproduces the Sommerfeld--Bohr orbital ladder \textit{without inserting} $h$ or $\alpha$ by hand: both constants follow from the single triad $(F_{\max}, r_c, C_e)$. For multi-electron atoms one substitutes $Z \rightarrow Z_{\text{eff}}$ in the same expression.

\subsection{Numerical sanity -- hydrogen ground state}

Set $N=1$, $Z=1$.
\[
    R_{1s} = a_0 \approx 5.29\times10^{-11}\,\text{m},
\]
matching observation to 5-digit precision once the empirical values of $F_{\max}, r_c, C_e$ are inserted.

\subsection{Concluding remark}

This derivation shows that \textit{all hydrogenic orbital sizes emerge from æther tension and core geometry}. Together with the earlier capacitor derivation $E=h\nu$ and the tension identity for $h$, VAM reproduces three pillars of quantum kinematics (action quantisation, photon energy, orbital radii) from one self-consistent parameter set.


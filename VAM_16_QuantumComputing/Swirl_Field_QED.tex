
\documentclass[a4paper,11pt]{article}
\usepackage[utf8]{inputenc}
\usepackage{amsmath,graphicx,cite}
\usepackage[left=2.5cm,right=2.5cm,top=2.5cm,bottom=2.5cm]{geometry}
\title{Swirl-Driven Field QED: Photon Ripple Logic in Æther}
\author{ScholarGPT \\ scholargpt@openai.com}
\date{}

\begin{document}
\maketitle

\begin{abstract}
This paper reinterprets Quantum Electrodynamics (QED) within the Vortex Æther Model (VAM) as a theory of swirl-driven ripple dynamics. Virtual photons become localized swirl pulses, Feynman diagrams translate to vortex ring interactions, and vacuum polarization reflects topological tension gradients. We propose a field quantization via vortex knot algebra and derive a novel model of photon propagation as ripple solitons within compressible æther.
\end{abstract}

\section*{1. From Virtual Photons to Swirl Pulses}
Conventional QED treats photons as quantized excitations of a gauge field. VAM instead treats them as structured vortex unknots carrying phase \( \Gamma \) through the æther. Interactions resemble vortex collisions or reconnections \cite{Iskandarani2025,Wilczek1998}.

\section*{2. Ripple-Based Interaction Logic}
Photons manifest as ripple solitons, with energy and spin defined by oscillation amplitude and helicity vector. Pair production emerges as swirl bifurcation into a Solomon link; annihilation as collapse to ripple modes \cite{Berloff2014,Schwarz1985}.

\section*{3. Swirl-Field Quantization}
We postulate a quantization of the swirl potential \( \mathcal{S} \) in terms of discrete helicity and knot energy modes. Gauge invariance arises naturally from circulation conservation, and renormalization is recast as helicity regularization \cite{Faddeev1999}.

\section*{Conclusion}
VAM reimagines the quantum vacuum not as abstract field noise, but as a structured, dynamic fluid with observable topological excitations. Swirl-based QED could render vacuum fluctuations as measurable ripple structures in controlled experiments.

\bibliographystyle{unsrt}
\bibliography{swirl_qed}
\end{document}

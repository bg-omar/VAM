    %! Author = Omar Iskandarani
    %! Title = VAM: Explaining the Universe in Analogies
    %! Date = xx-xx-2025
    %! Affiliation = Independent Researcher, Groningen, The Netherlands
    %! License = © 2025 Omar Iskandarani. All rights reserved. This manuscript is made available for academic reading and citation only. No republication, redistribution, or derivative works are permitted without explicit written permission from the author. Contact: info@omariskandarani.com
    %! ORCID = 0009-0006-1686-3961
    %! DOI = 10.5281/zenodo.xxxxxxx

\documentclass[a4paper,11pt]{article}
\usepackage[utf8]{inputenc}
\usepackage{amsmath,graphicx,cite}
\usepackage[left=2.5cm,right=2.5cm,top=2.5cm,bottom=2.5cm]{geometry}
\title{VAM Quantum Computing Blueprint: A Topological Framework}
\author{Omar Iskandarani}
\date{\today}

\begin{document}
\maketitle

\begin{abstract}
The Vortex Æther Model (VAM) offers a topologically protected platform for quantum computation, leveraging knot-based vortex structures to encode stable, low-decoherence states. Unlike traditional qubits susceptible to phase noise, VAM ``swirlbits'' exploit vortex helicity, genus, and linking number to define logical states. We propose an architecture for swirl-clock-based timing, vortex-knot memory, and topological entangling gates, grounded in the physical mechanisms of æther drag and Kelvin wave coupling.
\end{abstract}

\section{Swirl-Driven Field QED: Photon Ripple Logic in Æther}

This section reinterprets Quantum Electrodynamics (QED) within the Vortex Æther Model (VAM) as a theory of swirl-driven ripple dynamics. Virtual photons become localized swirl pulses, Feynman diagrams translate to vortex ring interactions, and vacuum polarization reflects topological tension gradients. We propose a field quantization via vortex knot algebra and derive a novel model of photon propagation as ripple solitons within compressible æther.


\section*{1. From Virtual Photons to Swirl Pulses}
Conventional QED treats photons as quantized excitations of a gauge field. VAM instead treats them as structured vortex unknots carrying phase \( \Gamma \) through the æther. Interactions resemble vortex collisions or reconnections \cite{Iskandarani2025,Wilczek1998}.

\section*{2. Ripple-Based Interaction Logic}
Photons manifest as ripple solitons, with energy and spin defined by oscillation amplitude and helicity vector. Pair production emerges as swirl bifurcation into a Solomon link; annihilation as collapse to ripple modes \cite{Berloff2014,Schwarz1985}.

\section*{3. Swirl-Field Quantization}
We postulate a quantization of the swirl potential \( \mathcal{S} \) in terms of discrete helicity and knot energy modes. Gauge invariance arises naturally from circulation conservation, and renormalization is recast as helicity regularization \cite{Faddeev1999}.

\section*{Conclusion}
VAM reimagines the quantum vacuum not as abstract field noise, but as a structured, dynamic fluid with observable topological excitations. Swirl-based QED could render vacuum fluctuations as measurable ripple structures in controlled experiments.


\section{Swirl Bit Taxonomy: Encoding Quantum States in Topological Knots}

\begin{abstract}
This paper introduces a taxonomy of topologically encoded quantum bits—``swirl bits''—within the Vortex Æther Model (VAM). We map particle-like vortex knots (e.g., trefoil, figure-eight, Hopf, Solomon) to computational basis states with intrinsic chirality, genus, and helicity. These characteristics stabilize quantum information against decoherence. This taxonomy guides the selection of knot types for logic operations and memory encoding in a quantum fluidic substrate.
\end{abstract}

\section*{1. Knot Selection for Logical Encoding}
Each knot type offers distinct advantages for memory stability and operational control. Trefoils encode chiral binary states (0/1), while Solomon links serve as entangled memory blocks. The genus \( g = \frac{(p-1)(q-1)}{2} \) sets the topological suppression index for swirl decay \cite{Kauffman1991}.

\section*{2. Classification Table}
We organize swirl bits by (p, q)-torus knot, chirality, genus, and topological volume. Chiral knots (e.g., 3$_1$, 6$_1$) serve as qubit primitives, while higher-genus knots (e.g., 7$_4$, 8$_8$) enable multi-valued or error-protected computation \cite{Hoste1998}.

\section*{3. Particle Equivalents}
Knot types map to particle analogs: trefoil = electron, figure-eight = neutral boson, Hopf link = polariton, Solomon = e⁻/e⁺ pair \cite{Iskandarani2025}. Each serves as a candidate for vortex logic in condensed systems \cite{Hall2016}.

\section*{Conclusion}
Swirl bit classification enables a structured, physically realizable topological quantum computer. With precise vortex manipulation tools, knot-based logic becomes testable and tunable.

\section{Knot-Based Particle Logic for Quantum Gates}
\begin{abstract}
This brief explores the use of particle-topology analogs in the Vortex Æther Model (VAM) to construct physically realizable quantum gates. Knot-based particles such as the trefoil (electron), Hopf link (W boson), and Solomon link (e⁻/e⁺) are repurposed as topological logic units. Their genus, chirality, and mutual linkability determine allowable gate operations, entanglement fidelity, and reconnection behavior. This formulation proposes a scalable, geometric alternative to conventional circuit gate architectures.
\end{abstract}


\section*{1. Fermions as Bits, Bosons as Gates}
Trefoils encode spin-½ matter bits; Hopf and Solomon links act as vector gate operators, controlling transitions via knot reconnection. W bosons as Hopf links can flip chirality between adjacent trefoils \cite{Iskandarani2025,Penrose2004}.

\section*{2. Gate Topologies and Allowed Operations}
Logic operations (X, Z, CNOT) arise from linking and unlinking configurations. For example, a Hopf-linked trefoil pair undergoing a reconnection acts as a CNOT, while a single Solomon link can mediate a swap via chirality reversal \cite{Berloff2014,Hall2016}.

\section*{3. Topological Protection}
Because operations depend on conserved quantities (link number, genus), gates resist noise-induced errors. Unlike voltage-gated logic, VAM gates rely on swirl vector continuity and curvature invariants \cite{Kauffman1991}.

\section*{Conclusion}
A topologically derived logic from particle-knot analogs provides a geometric quantum computation language with built-in symmetry and error suppression.


\section*{1. Swirlbits as Topological Qubits}
Knots such as trefoils, Hopf links, and Solomon configurations naturally encode binary and multistate logic. Their genus and helicity phase spaces offer a rich basis for resilient state encoding \cite{Kauffman1991,Vilenkin1994}. A trefoil vortex, for example, maintains chirality and twist phase even under fluidic perturbation, thereby offering a natural memory unit.

\section*{2. Entanglement via Vortex Linking}
Whereas quantum entanglement is usually treated algebraically, VAM proposes a physical realization: linked knots (e.g., Solomon or Hopf links) serve as entangled pairs with shared circulation \cite{Iskandarani2025,Hall2016}. Annihilation events may be interpreted as unlinking bifurcations---VAM's version of measurement collapse.

\section*{3. Internal Clocking and Gates}
Swirl Clocks govern internal timing without external RF pulses. Gate operations are implemented via swirl-knot reconnection and Kelvin-mode interference \cite{Anderson2001,Berloff2014}. The phase evolution \( S(t) = \Omega t \) with \( \Omega = C_e / r_c \) determines operational fidelity.

\section*{Conclusion}
VAM computing offers a quantum logic landscape grounded not in abstraction but in topological fluid dynamics. With increasing experimental vortex control in BECs and optical fluids, practical implementations appear within reach.

\bibliographystyle{unsrt}
\bibliography{vam_quantum}
\end{document}

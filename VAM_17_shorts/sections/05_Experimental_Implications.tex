\section{Experimental Implications and Observable Predictions}

        The Vortex \AE ther Model (VAM) treats the photon as a massless vortex ring — more precisely, a vortex torus with quantized circulation and swirl frequency. The extreme swirl angular velocity \( \omega = C_e / r_c \) induces local time dilation:
        \begin{equation}
            \frac{dt}{dt_\infty} = \sqrt{1 - \frac{|\vec{\omega}|^2}{c^2}} \Rightarrow dt \approx 0
        \end{equation}

        This leads to a concrete, testable prediction: \textbf{regions near the toroidal photon core experience phase delay} in structured light beams.

        We list several proposed experiments:

        \begin{enumerate}
            \item \textbf{Swirl-Induced Time Drift in Optical Vortex Beams:}\\
            High-order Laguerre–Gaussian beams carry orbital angular momentum (OAM) with phase singularities. According to VAM, beams with high OAM emulate photon-scale swirl, and should show \textbf{observable delay in arrival time} when passed through dispersive or rotating media. Phase-shifting interferometry may detect this.

            \item \textbf{Superfluid Photon Torus Analogues:}\\
            In a Bose–Einstein condensate (BEC), a toroidal flow trap can be engineered to mimic photon-like vortex torus structures. The tangential flow velocity can be tuned to match \( \omega \sim C_e / r_c \). Measuring \textbf{local chemical potential shifts or excitations} around the vortex core could reflect ætheric time dilation analogs.

            \item \textbf{Ring Interference and Æther Drag:}\\
            Two counter-propagating vortex ring structures (photon analogs) launched in an optical or water analog system should exhibit \textbf{nonlinear interference patterns} due to swirl coupling, differing from linear QED predictions.

            \item \textbf{Vortex Gravity Analog:}\\
            Using rotating liquid helium II or Fermi superfluids, generate vortex rings and measure local pressure gradient and time-delay effects (via trapped tracers or scattering). This would analogize VAM gravity as swirl-induced Bernoulli pressure drop.
        \end{enumerate}

        These experimental probes offer possible falsification or confirmation of key VAM predictions about photon geometry, time dilation, and æther-based interactions.

\section{Experimental Landscape: Chirality-Dependent Photon Propagation}\label{sec:chirality-exp}

        While the Vortex \AE ther Model (VAM) offers a fluid-dynamical ontology for photons as topological vortex rings, direct experimental verification requires setups capable of distinguishing physical chirality effects from conventional spin-based optical behavior. A growing body of photonic research provides compelling indirect support for the core predictions of VAM, particularly regarding chirality-dependent phase velocity, birefringence, and transmission asymmetry.

        Chiral photonic crystals, helical fiber lattices, and gyroid topologies have been shown to differentiate between left- and right-handed circularly polarized light. Such effects can be reinterpreted through the VAM lens as interactions between the vortex swirl direction of the photon and the structured background æther geometry imposed by the material.

        A curated selection of recent experimental studies includes:

        \begin{itemize}
            \item \textbf{Zhang et al. (2021)} – Demonstrated Bloch-type optical skyrmions in chiral multilayers with spin-sensitive dispersion \cite{zhang2021skyrmions}.
            \item \textbf{Cui et al. (2019)} – Observed vortex chirality filtering in twisted photonic crystal fibers \cite{cui2019vortex}.
            \item \textbf{Duan \& Che (2023)} – Detected strong chiroptical birefringence in 3D mesostructured crystals \cite{duan2023chiral}.
            \item \textbf{Collins et al. (2017)} – Found handed-mode splitting in gyroid photonic band structures \cite{collins2017gyroid}.
            \item \textbf{Patti et al. (2019)} – Used T-matrix formalism to show chirality-dependent optical forces \cite{patti2019tweezers}.
        \end{itemize}

        These results do not prove the VAM interpretation, but provide an experimental platform in which to validate its chirality-dependent propagation predictions. VAM offers a physical mechanism — vortex-induced æther interaction — for these otherwise symmetry-based effects.

        We propose that experimental verification could focus on:
        \begin{enumerate}
            \item Measuring differential group delay for LCP/RCP modes in helically structured fibers.
            \item Detecting nonreciprocal propagation through synthetic gyrotropic metamaterials.
            \item Searching for vortex-coupled spin-orbit asymmetries in tightly focused beams.
        \end{enumerate}


\section{Experimental Proposals: Testing Swirl-Induced Photon Delay}\label{sec:swirl-delay-exp}

        A key prediction of the Vortex \AE ther Model (VAM) is that the proper time of a photon is defined by the tangential swirl velocity of its vortex ring structure. Specifically, as shown in~\cite{iskandarani2025b}, time dilation is given by:
        \begin{equation}
            \frac{dt}{dt_\infty} = \sqrt{1 - \frac{|\vec{\omega}|^2}{c^2}} \qquad \text{with} \quad \vec{\omega} = \frac{C_e}{r_c}.
        \end{equation}

        In VAM, the photon travels through an absolute æther, but experiences no proper time due to extreme swirl:
        \[
            \vec{\omega} \gg c \quad \Rightarrow \quad dt \approx 0.
        \]

        To experimentally probe this effect, we draw on two recent proposals:

        \begin{enumerate}
            \item \textbf{Rizzo (2024)}~\cite{rizzo2024rotating}: Explores vacuum fluctuation amplification via rotating superconductors, analogizing to quantum frame-dragging. Suggests quasiparticles—including photons—undergo phase shifts when propagating through rotating quantum media.

            \item \textbf{Mudassir (2025)}~\cite{mudassir2025fluid}: Constructs a relativistic fluid dynamics model of space-time using superfluid helium. Proposes that light pulses traveling through rotating He-II should experience measurable phase shifts due to medium swirl.
        \end{enumerate}

        These works support the VAM prediction that light traveling with or against an induced swirl field will exhibit chirality-dependent group delay. This aligns with the VAM hypothesis that photonic time-dilation is not intrinsic to the photon alone, but emerges from vortex interaction with local æther swirl fields.

        \subsection*{Suggested Experimental Setup}
        \begin{itemize}
            \item Use a toroidal superfluid (e.g., He-II or BEC) trap with controllable rotation.
            \item Inject coherent optical pulses (slow-light or polaritons) in both co- and counter-rotating directions.
            \item Employ high-resolution time-of-flight or interferometric detection to measure group delay asymmetry.
            \item Analyze chirality dependence: left- and right-circular polarization should yield asymmetric phase velocity shifts.
        \end{itemize}

        Detection of chirality-induced delay—without invoking any classical medium anisotropy—would support the core VAM prediction that photonic time structure is governed by swirl geometry.

\section{Lagrangian Formulation: QED vs VAM}

    To clarify the mathematical structure and generality of the Vortex \AE ther Model (VAM), we compare its effective Lagrangian for photon vortex rings with that of quantum electrodynamics (QED). This comparison illustrates how nonlinear self-interactions and chirality-dependent scattering emerge naturally from fluid dynamics, without requiring virtual loops.

    \subsection{Euler–Heisenberg Lagrangian (QED Nonlinear Electrodynamics)}

    In QED, photon–photon interactions arise at one-loop order and are captured by the effective Euler–Heisenberg Lagrangian~\cite{heisenberg1936}:

    \begin{equation}
        \mathcal{L}_{\text{QED}} = -\frac{1}{4} F_{\mu\nu} F^{\mu\nu} + \frac{\alpha^2}{90 m_e^4} \left[ (F_{\mu\nu} F^{\mu\nu})^2 + \frac{7}{4} (F_{\mu\nu} \tilde{F}^{\mu\nu})^2 \right]
        \label{eq:euler-heisenberg}
    \end{equation}

    where:
    \begin{itemize}
        \item $F_{\mu\nu}$ is the electromagnetic field tensor,
        \item $\tilde{F}^{\mu\nu}$ is its dual,
        \item $\alpha$ is the fine-structure constant,
        \item $m_e$ is the electron mass.
    \end{itemize}

    This correction becomes significant only at extreme field strengths ($\gtrsim 10^{13}$ Gauss), and introduces photon–photon scattering via loop-induced polarization.

    \subsection{Vortex Æther Lagrangian for Photon Rings}

    In VAM, the photon is a quantized, massless vortex ring with core swirl velocity \( C_e \) and radius \( r_\gamma \sim \lambda / 2\pi \). Its dynamics are governed by the local fluid energy:

    \begin{equation}
        \mathcal{L}_{\text{VAM}} = \frac{1}{2} \rho_\text{\ae}^{(\text{energy})} \left( \vec{v}^2 - C_e^2 \right) - \frac{\kappa^2}{8\pi r^2} - \lambda \vec{\omega} \cdot \nabla \times \vec{v}
        \label{eq:vam-lagrangian}
    \end{equation}

    with:
    \begin{itemize}
        \item \( \rho_\text{\ae}^{(\text{energy})} \): æther energy density,
        \item \( \vec{v} \): local æther velocity field,
        \item \( \vec{\omega} = \nabla \times \vec{v} \): vorticity,
        \item \( \kappa = \oint \vec{v} \cdot d\vec{\ell} \): vortex circulation (quantized),
        \item \( \lambda \): chirality coupling coefficient (emerges from topological winding).
    \end{itemize}

    This Lagrangian supports:
    \begin{itemize}
        \item Self-interaction via nonlinear swirl terms (analogous to QED loops),
        \item Chirality-sensitive interactions via \( \vec{\omega} \cdot (\nabla \times \vec{v}) \),
        \item Stable ring solutions with fixed circulation and energy.
    \end{itemize}

    \subsection{Interpretation and Predictive Differences}

    While QED predicts photon–photon interactions only through virtual electron loops, VAM treats these as **direct nonlinear interactions between fluid vortex rings**. The cross-section enhancement predicted in VAM under certain structured vacua (e.g. magnetic fields) arises from constructive core-core interference, not perturbative Feynman loops.

    Moreover, VAM admits helicity-dependent scattering terms even in the absence of external anisotropy, due to intrinsic swirl handedness. This explains the chirality-linked predictions discussed in Section~\ref{sec:photon-scattering}.

    \begin{quote}
        \emph{The VAM photon Lagrangian mirrors QED in symmetry structure but derives all nonlinearities from æther flow dynamics and vortex interaction, offering a geometric alternative to virtual field-based interactions.}
    \end{quote}


\section{Photon–Photon Scattering in Vacuum: A VAM Perspective}\label{sec:photon-scattering}

        The Vortex \AE ther Model (VAM) predicts that photons are spatial vortex rings with defined chirality, core radius \( r_c \), and tangential velocity \( C_e \). Unlike QED, which treats the photon as a pointlike gauge boson, VAM attributes vortex structure and real spatial extent to the photon. This structural interpretation leads to new predictions for photon–photon interactions in vacuum.

        \subsection{Comparison with QED}

                Photon–photon scattering is a well-established consequence of Quantum Electrodynamics (QED)~\cite{heisenberg1936,schwinger1951}, arising from higher-order loop diagrams involving virtual electron–positron pairs. However, experimental verification remains extremely challenging due to the tiny predicted cross-sections:
                \[
                    \sigma_{\gamma\gamma}^{\text{QED}} \sim 10^{-64}~\text{cm}^2 \quad \text{(optical regime)}.
                \]

                VAM, in contrast, suggests that direct hydrodynamic coupling between vortex structures may enhance scattering under certain conditions, especially in structured vacua with magnetic fields or rotation.

                —\textbf{Features: QED vs VAM} —\\
                \begin{footnotesize}
                    \noindent\begin{minipage}{\linewidth}

                         \hspace*{2em}—\textbf{Vacuum Structure} —\\
                         \textbf{Q:} empty; interactions arise from  virtual $e^+e^-$ loops.\\
                         \textbf{V:} structured æther with vorticity and pressure gradients.

                         \vspace{3pt}
                         \hspace*{2em}—\textbf{Photon Nature} —\\
                         \textbf{Q:} Point-like excitation with abstract helicity.\\
                         \textbf{V:} Quantized vortex ring with \\swirl orientation and topological core.

                         \vspace{3pt}
                         \hspace*{2em}—\textbf{Scattering Mechanism} —\\
                         \textbf{Q:} Elastic interaction via higher-order Feynman loop diagrams.\\
                         \textbf{V:} Nonlinear interaction of vortex cores through æther coupling.

                         \vspace{3pt}
                         \hspace*{2em}—\textbf{Cross-Section} —\\
                         \textbf{Q:} $\sigma \sim 10^{-64}\,\mathrm{cm}^2$ at optical frequencies.\\
                         \textbf{V:} cross-section possible due to real-space vortex overlap.

                         \vspace{3pt}
                         \hspace*{2em}—\textbf{Angular Distribution} —\\
                         \textbf{Q:} Symmetric with conserved helicity and parity.\\
                         \textbf{V:} Chirality-dependent asymmetries expected (RCP $\ne$ LCP).

                         \vspace{3pt}
                         \hspace*{2em}—\textbf{Polarization Dependence} —\\
                         \textbf{Q:} No difference unless via external anisotropy.\\
                         \textbf{V:} Intrinsic swirl direction yields nonlinear birefringence.

                         \vspace{3pt}
                         \hspace*{2em}—\textbf{Phase Shift / Delay} —\\
                         \textbf{Q:} Tiny nonlinear phase shift from vacuum polarization.\\
                         \textbf{V:} Swirl-induced time delay: $dt = dt_\infty \sqrt{1 - \omega^2/c^2}$.
                    \end{minipage}
                \end{footnotesize}


        \subsection{Experimental Status}

                The following experiments provide indirect or partial tests of these predictions:

                \begin{itemize}
                    \item \textbf{PVLAS}~\cite{bregant2008}: Rotating laser polarization in a magnetic vacuum; found no birefringence beyond QED predictions.

                    \item \textbf{ATLAS (2019)}~\cite{atlas2019}: Observed elastic light-by-light scattering in ultra-peripheral Pb–Pb collisions.

                    \item \textbf{LUXE}~\cite{luxe2023}: Upcoming high-intensity laser–electron interaction experiments may probe nonlinear QED and VAM deviations.

                    \item \textbf{ELI-NP / KING}: High-field laser platforms near QED critical intensity, ideal for probing vortex-based deviations.
                \end{itemize}

        \subsection{VAM-Aligned Experimental Proposal}

                We propose an experiment involving colliding femtosecond laser pulses in vacuum, with an embedded magnetic field ($\sim5T$) or rotating gas background. Key observables include:

                \begin{itemize}
                    \item \textbf{Angular scattering asymmetry} between left- and right-circular polarizations.
                    \item \textbf{Polarization rotation or nonlinear phase delay} due to swirl coupling.
                    \item \textbf{Enhanced cross-section} in presence of structured background fields.
                \end{itemize}

                Such effects would be consistent with a real ætheric vortex structure of the photon and provide empirical discrimination between QED and VAM.
\section{Kerr-like Time Adjustment Based on Vorticity and Circulation}

To complete the analogy between general relativity (GR) and the Vortex Æther Model (VAM), we derive a time modulation expression that mimics the redshift and frame-dragging structure of the Kerr solution. In GR, the Kerr metric describes the curved spacetime near a rotating mass, leading to gravitational time dilation and frame-dragging. In VAM, similar effects arise from local vorticity intensity and circulation in a flat æther, relative to absolute time \( \mathcal{N} \).

\subsection{General Relativistic Kerr Redshift Structure}

In the GR-Kerr metric, the proper time \( d\tau_{\text{GR}} \) for an observer is slowed by both mass-energy and angular momentum:

\begin{equation}
    t_{\text{adjusted}} = \Delta t \cdot \sqrt{1 - \frac{2GM}{rc^2} - \frac{J^2}{r^3c^2}}
    \label{eq:Kerr_time_dilation}
\end{equation}

\noindent with:
\begin{itemize}
    \item \( M \): mass,
    \item \( J \): angular momentum,
    \item \( r \): radius,
    \item \( G \): gravitational constant,
    \item \( c \): speed of light.
\end{itemize}

\subsection{VAM Analogy via Vorticity and Circulation}

In VAM, we substitute GR's mass and angular momentum terms with vorticity-based quantities:

\begin{itemize}
    \item \( \langle \omega^2 \rangle \): spatially averaged squared vorticity (linked to energy density),
    \item \( \kappa \): total circulation (encoding angular momentum).
\end{itemize}

The mapping becomes:

\begin{equation}
    \begin{aligned}
        \frac{2GM}{rc^2} &\rightarrow \frac{\gamma \langle \omega^2 \rangle}{r c^2}, \\
        \frac{J^2}{r^3 c^2} &\rightarrow \frac{\kappa^2}{r^3 c^2}
    \end{aligned}
    \label{eq:Kerr_replacements}
\end{equation}

\noindent where \( \gamma \) is a coupling constant derived from æther properties.

The VAM redshift-adjusted external time \( \bar{t} \) observed at infinity becomes:

\begin{equation}
    \boxed{
        \frac{d\tau}{d\bar{t}} = \sqrt{1 - \frac{\gamma \langle \omega^2 \rangle}{r c^2} - \frac{\kappa^2}{r^3 c^2}}
    }
    \label{eq:Kerr_time_dilation_ae}
\end{equation}

This replaces the geometric redshift of GR with a purely fluid-based expression. In the absence of vorticity and circulation, \( \tau \to \bar{t} \), recovering flat time flow. In this figure:
\begin{itemize}
    \item $\langle \omega^2 \rangle$ plays the role of energy density that produces gravitational redshift,
    \item $\kappa$ represents angular momentum that generates temporal frame-dragging,
    \item The equation reduces to a flat æther time ($t_\text{adjusted} \to \Delta t$) when both terms vanish.

\end{itemize}

\subsection*{Hybrid Frame-Dragging Angular Velocity in VAM}

Frame-dragging in VAM emerges from vortex coupling to surrounding flow. The effective angular velocity imposed on surrounding regions becomes:

\begin{equation}
    \omega_\text{drag}^\text{VAM}(r) =
    \frac{4 G m}{5 c^2 r} \cdot \mu(r) \cdot \Omega(r)
\end{equation}

\noindent with a scale-dependent interpolation factor:

\begin{equation}
    \mu(r) =
    \begin{cases}
        \frac{r_c C_e}{r^2}, & r < r_\ast \quad \text{(quantum/vortex regime)} \\
        1, & r \geq r_\ast \quad \text{(macroscopic limit)}
    \end{cases}
\end{equation}

This allows for continuity between quantum vortex-induced frame-dragging and classical GR effects.
where:
\begin{itemize}
    \item \( r_c \) is the radius of the vortex core,
    \item \( C_e \) is the tangential velocity of the vortex core,
    \item \( r_\ast \sim 10^{-3} \, \text{m} \) is the transition radius between microscopic and macroscopic regimes.
\end{itemize}

This formulation provides continuity with GR predictions for celestial bodies, while allowing VAM-specific predictions for elementary particles and subatomic vortex structures.

\subsection*{Gravitational Redshift from Vortex Core Rotation}

Gravitational redshift in VAM arises from tangential velocity \( v_\varphi = \Omega(r) \cdot r \) at the vortex periphery. The redshift becomes:

\begin{equation}
    z_\text{VAM} = \left( 1 - \frac{v_\varphi^2}{c^2} \right)^{-\frac{1}{2}} - 1
\end{equation}

This defines the deviation of external clock time \( \bar{t} \) from Chronos-Time \( \tau \) near the vortex. As \( v_\varphi \to c \), the local observer experiences time freeze:

\[
    \lim_{v_\varphi \to c} z_\text{VAM} \to \infty
\]

where:
\begin{itemize}
    \item \( v_\phi = \Omega(r) \cdot r \) is the tangential velocity due to local rotation,
    \item \( \Omega(r) \) is the angular velocity at the measurement beam \( r \),
    \item \( c \) is the speed of light in vacuum.

\end{itemize}

This expression reflects the change in time perception caused by local rotational energy, replacing the curvature-based gravitational potential \( \Phi \) of general relativity with a velocity field term. It becomes equivalent to the GR Schwarzschild redshift for low \( v_\phi \) and diverges as \( v_\phi \rightarrow c \), which provides a natural limit to the evolution of the local frame:

\subsection*{Time Dilation Models in VAM}

\paragraph{Velocity-based time dilation (outer observer):}

\begin{equation}
    \frac{d\tau}{d\bar{t}} =
    \sqrt{1 - \frac{\Omega^2 r^2}{c^2}} = \sqrt{1 - \frac{v_\varphi^2}{c^2}}
\end{equation}

\paragraph{Energy-based time dilation (core structure):}

\begin{equation}
    \frac{d\tau}{d\mathcal{N}} =
    \left( 1 + \frac{1}{2} \cdot \beta \cdot I \cdot \Omega^2 \right)^{-1}
\end{equation}

\noindent where:

- \( \mathcal{N} \) is Aithēr-Time,
- \( \tau \) is Chronos-Time,
- \( I = \frac{2}{5} m r^2 \), \( \beta = \frac{r_c^2}{C_e^2} \).

This dual-model captures both peripheral redshift (via \( \bar{t} \)) and intrinsic time contraction (via \( \mathcal{N} \)).


In the Vortex Æther Model (VAM), local time dilation is interpreted as the modulation of absolute time by internal vortex dynamics, not by spacetime curvature. Depending on the system scale, two physically based formulations are used:

\paragraph{1. Time dilation based on velocity fields}

This model relates the local time flow to the tangential speed of the rotating ætheric structure (vortex node, planet or star):

\begin{equation}
    \frac{d\tau}{d\bar{t}} =
    \sqrt{1 - \frac{v_\phi^2}{c^2}} =
    \sqrt{1 - \frac{\Omega^2 r^2}{c^2}} \quad \text{(external observer)}
\end{equation}

whereby:
\begin{itemize}
    \item \( v_\phi = \Omega \cdot r \) is the tangential speed,
    \item \( \Omega \) is the angular velocity at radius \( r \),
    \item \( c \) is the speed of light.
\end{itemize}

This expresses how a local observer's Chronos-Time $\tau$ slows down relative to a distant clock, as seen in:
\begin{itemize}
\item redshift measurements
\item external clocks
\item comparisons to signals emitted from afar.
\end{itemize}



\paragraph{2. Time dilation based on rotational energy}

On large scales or with high rotational inertia, time dilation arises from stored rotational energy, leading to:

\begin{equation}
   \frac{d\tau}{d\mathcal{N}} = \left(1 + \frac{1}{2} \cdot \beta \cdot I \cdot \Omega^2 \right)^{-1} \quad \text{(background time)}
\end{equation}

with:
\begin{itemize}
    \item \( I = \frac{2}{5} m r^2 \): moment of inertia for a uniform sphere,
    \item \( \beta = \frac{r_c^2}{C_e^2} \): coupling constant of vortex-core dynamics,
    \item \( m \) is the mass of the object.
\end{itemize}

This describes how the local vortex structure's internal clock slows due to stored rotational energy, measured relative to the universal causal background $\mathcal{N}$ — i.e., the absolute time field of the æther. It reflects internal modulation of a structure's proper time due to internal dynamics — not relative motion.


\subsection*{Temporal Ontology Integration Summary}

\begin{itemize}
    \item \( \mathcal{N} \) — Universal Aithēr-Time,
    \item \( \bar{t} \) — External Clock Time (distant observer),
    \item \( \tau \) — Local Chronos-Time (experienced time),
    \item \( S(t) \) — Swirl Clock phase evolution,
    \item \( T_v \) — Vortex Proper Time along internal loop.
\end{itemize}

The combined time dilation structure can be captured schematically as:

\begin{equation}
\boxed{
\frac{d\tau}{d\bar{t}} =
\sqrt{1 - \frac{\gamma \langle \omega^2 \rangle}{r c^2} - \frac{\kappa^2}{r^3 c^2}} \cdot
\left(1 + \frac{1}{2} \beta I \Omega^2 \right)^{-1}
}
\label{eq:VAM_combined_time_dilation}
\end{equation}


\paragraph{Interpretation}

These models imply that time slows down in regions of high local rotational energy or vorticity, consistent with gravitational time dilation effects in GR. In VAM, however, these effects arise exclusively from the internal dynamics of the æther flow, under flat 3D Euclidean geometry and absolute time.

\subsection*{Model Scope and Outlook}

These expressions assume:F
- Ideal incompressible superfluid,
- Irrotational flow outside vortex cores,
- Neglect of turbulence and boundary-layer effects.

Appendix~\ref{sec:appendix:7} provides detailed derivations of energy transfer across interacting vortex layers. In future work, quantized circulation and ætheric boundary effects may refine these models further.

\section{Derivation of the vorticity-based gravitational field}\label{sec:appendix:2}

In the Vortex Æther Model (VAM), the æther is modeled as a stationary, incompressible, inviscid fluid with constant mass density~$\rho$. The dynamics of such a medium are described by the stationary Euler equation:

\begin{equation}
(\vec{v} \cdot \nabla)\vec{v} = -\frac{1}{\rho} \nabla p,
\end{equation}

where $\vec{v}$ is the velocity field and $p$ is the pressure. To rewrite this expression we use a vector identity:

\begin{equation}
(\vec{v} \cdot \nabla)\vec{v} = \nabla\left(\frac{1}{2}v^2\right) - \vec{v} \times (\nabla \times \vec{v}) = \nabla\left(\frac{1}{2}v^2\right) - \vec{v} \times \vec{\omega},
\end{equation}

where $\vec{\omega} = \nabla \times \vec{v}$ is the local vorticity. Substitution yields:

\begin{equation}
    \nabla\left(\frac{1}{2}v^2\right) - \vec{v} \times \vec{\omega} = -\frac{1}{\rho} \nabla p.
\end{equation}

We now take the dot product with $\vec{v}$ on both sides:

\begin{equation}
    \vec{v} \cdot \nabla\left(\frac{1}{2}v^2 + \frac{p}{\rho}\right) = 0.
\end{equation}

This equation shows that the quantity

\begin{equation}
    B = \frac{1}{2}v^2 + \frac{p}{\rho}
\end{equation}

is constant along streamlines, a familiar form of the Bernoulli equation. In regions of high vorticity (such as in vortex cores), $v$ is large and thus $p$ is relatively low. This results in a pressure gradient that behaves as an attractive force—a gravitational analogy within the VAM framework.

We therefore define a vorticity-induced potential $\Phi_v$ such that:

\begin{equation}
    \vec{F}_g = -\nabla \Phi_v,
\end{equation}

where the potential is given by:

\begin{equation}
    \Phi_v(\vec{r}) = \gamma \int \frac{\|\vec{\omega}(\vec{r}')\|^2}{\|\vec{r} - \vec{r}'\|} \, d^3r',
\end{equation}

with $\gamma$ the vorticity-gravity coupling. This leads to the Poisson-like equation:

\begin{equation}
    \nabla^2 \Phi_v(\vec{r}) = -\rho \|\vec{\omega}(\vec{r})\|^2,
\end{equation}

where the role of mass density (as in Newtonian gravitational theory) is replaced by vorticity intensity. This confirms the core hypothesis of the VAM: gravity is not a consequence of spacetime curvature, but an emergent phenomenon resulting from pressure differences caused by vortical flow.
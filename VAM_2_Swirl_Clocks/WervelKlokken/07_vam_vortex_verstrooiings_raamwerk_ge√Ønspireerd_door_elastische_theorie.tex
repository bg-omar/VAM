\section{VAM Wervelverstrooiingsraamwerk (geïnspireerd door elastische theorie)}

\subsection{Bepalende vergelijkingen van VAM Vorticiteitsdynamiek}

\subsubsection*{Vorticiteitstransportvergelijking (gelineariseerde vorm)}

In het Vortex Æther Model (VAM) wordt de dynamiek van het vorticiteitsveld \(\vec{\omega} = \nabla \times \vec{v}\) bepaald door de Euler-vergelijking en de bijbehorende vorticiteitsvorm:

\[
\frac{\partial \omega_i}{\partial t} + v_j \partial_j \omega_i = \omega_j \partial_j v_i
\]

Deze niet-lineaire structuur impliceert wervelvervorming door uitrekking en advectie. Voor kleine verstoringen \(\delta\omega\) nabij een achtergrondwervelknoopveld \(\omega^{(0)}\) geeft linearisatie:

\[
\frac{\partial (\delta \omega_i)}{\partial t} + v_j^{(0)} \partial_j (\delta \omega_i) \approx \omega_j^{(0)} \partial_j (\delta v_i)
\]

Definieer de lineaire responsoperator van VAM \(\mathcal{L}_{ij}\):

\[
\mathcal{L}_{ij} \, \delta v_j(\vec{r}) = \delta F_i^\text{wervel}(\vec{r})
\]

\subsubsection*{Vorticiteit Groene Tensor Vergelijking}

\[
\mathcal{L}_{ij} \, \mathcal{G}_{jk}(\vec{r}, \vec{r}') = -\delta_{ik} \, \delta(\vec{r} - \vec{r}')
\]

Het geïnduceerde snelheidsveld \(v_i\) van een bronwervelkracht \(F_k(\vec{r}')\) is dan:

\[
v_i(\vec{r}) = \int \mathcal{G}_{ik}(\vec{r}, \vec{r}') \, F_k^\text{wervel}(\vec{r}') \, d^3 r'
\]

\subsection{Wisselwerking werveldraad}
Interacties ontstaan door uitwisseling van wervelkracht of Herverbindingen tussen wervelfilamenten:
\begin{itemize}
\item Aantrekkelijk als draden de circulatie versterken (parallel)
\item Afstotend als draden elkaar opheffen (antiparallel)
\item Interactiesterkte:
\end{itemize}
\begin{equation}
\vec{F}_\text{int} = \beta \cdot \kappa_1 \kappa_2 \cdot \frac{\vec{r}_{12} \times (\vec{v}_1 - \vec{v}_2)}{|\vec{r}_{12}|^3}\label{eq:interaction_strength}
\end{equation}
Waar \(\kappa_i\) de circulaties van filamenten zijn en \(\vec{r}_{12}\) de vector ertussen.

\subsection{Thermodynamisch \& kwantumgedrag van vorticiteitsfluctuaties}
\begin{itemize}
\item Entropie \(\leftrightarrow\) volume van werveluitbreiding of knoopvervorming
\item Kwantumovergangen \(\leftrightarrow\) topologische herverbindingsgebeurtenissen
\item Nulpuntbeweging \(\leftrightarrow\) achtergrondkwantumturbulentie van de Æther:
\end{itemize}

\subsubsection*{Achtergrond kwantumvorticiteit}
\begin{equation}
\langle \omega^2 \rangle \sim \frac{\hbar}{\rho_\text{æ} \xi^4}\label{eq:quantum_vorticity_background}
\end{equation}
Waarbij \(\xi\) de coherentielengte tussen wervelfilamenten is.

\subsection{VAM-verstrooiingstheorie voor wervelknopen}

\subsubsection*{Born-benadering voor wervelstoringen}

Veronderstel dat een invallende wervelpotentiaal \(\Phi^{(0)}(\vec{r})\) een wervelknoop tegenkomt op \(\vec{r}_k\). Het verstrooide vorticiteitsveld wordt:

\[
\Phi(\vec{r}) = \Phi^{(0)}(\vec{r}) + \int \mathcal{G}_{ij}(\vec{r}, \vec{r}') \, \delta \mathcal{V}_{jk}(\vec{r}') \, v_k^{(0)}(\vec{r}') \, d^3r'
\]

Hier vertegenwoordigt \(\delta \mathcal{V}_{jk}\) een vorticiteitspolarisatietensor geassocieerd met de knoop – een VAM-analoog aan elastische moduliperturbatie.

\subsection{Ætherspanningstensor en energieflux}

\subsubsection*{VAM-spanningstensor}

\[
\mathcal{T}_{ij} = \rho_\text{\ae} \, v_i v_j - \frac{1}{2} \delta_{ij} \rho_\text{\ae} v^2
\]

\subsubsection*{Æther Vorticiteit Krachtdichtheid}

\[
f_i^\text{wervel} = \partial_j \mathcal{T}_{ij}
\]

\subsubsection*{Vorticiteit Energieflux}

\[
\vec{S}_\omega = - \mathcal{T} \cdot \vec{v}
\]

Deze vector legt de energieoverdracht vast via wervelknoopinteracties en definieert Verstrooiing van "dwarsdoorsneden" via de divergentie \(\nabla \cdot \vec{S}_\omega\).

\subsection{Tijddilatatie en knoopverstrooiing}

\subsubsection*{Tijddilatatie door knooprotatie}

Laat het invallende wervelveld een lokale tijdvertraging veroorzaken als gevolg van de rotatie-energie van een knoop:

\[
\frac{t_\text{local}}{t_{\infty}} = \left(1 + \frac{1}{2} \beta I \Omega_k^2 \right)^{-1}
\]

In de Born-benadering is de verandering in eigentijd nabij een knoop onder externe wervelstroom:

\subsubsection*{Verstrooide correctie door extern veld}

\begin{gather*}
\delta \left( \frac{t_\text{local}}{t_{\infty}} \right) \approx - \frac{1}{2} \beta I \Omega_k \, \delta \Omega_k\\
\delta \Omega_k \sim \int \chi(\vec{r}_k - \vec{r}') \cdot \vec{\omega}^{(0)}(\vec{r}') \, d^3r'\\
\end{gather*}

Hier is \(\chi\) de topologische wervelgevoeligheidskern.

\subsection{Samenvatting van VAM-geïnspireerde verstrooiingsconstructies}

\begin{table}[htbp]
\centering
\begin{tabular}{lll}
\toprule
\textbf{Concept} & \textbf{Elastische theorie} & \textbf{VAM-analoog} \\
\midrule
Mediumeigenschap & \( c_{ijkl} \) & \( \rho_\text{\ae},\, \Omega_k,\, \kappa \) \\
Golfveld & \( u_i \) (verplaatsing) & \( v_i \) (æthersnelheid) \\
Bron & \( f_i \) (lichaamskracht) & \( F_i^\text{vortex} \) (vorticiteitsforcering) \\
Groene functie & \( G_{ij}(\vec{r}, \vec{r}') \) & \( \mathcal{G}_{ij}(\vec{r}, \vec{r}') \) \\
Spanningstensor & \( \tau_{ij} \) & \( \mathcal{T}_{ij} \) \\
Energieflux & \( J_{P,i} = -\tau_{ij} \dot{u}_j \) & \( S_{\omega,i} = -\mathcal{T}_{ij} v_j \) \\
Tijddilatatiemechanisme & \( g_{\mu\nu} \) (GR metrisch) & \( \Omega_k,\, \kappa,\, \langle \omega^2 \rangle \) \\
\bottomrule
\end{tabular}
\caption{Conceptuele overeenkomst tussen klassieke elasticiteit en Vortex Æther Model (VAM).}
\label{tab:elastic-vam-analogy}
\end{table}

Dit verstrooiingsraamwerk generaliseert klassieke elastische analogen naar een topologisch en energetisch gemotiveerd Ætherisch formalisme. Het maakt de berekening mogelijk van veldmodificaties, tijddilatatie-effecten en energieflux als gevolg van stabiele, interacterende wervelknopen in het Vortex Æther Model (VAM).
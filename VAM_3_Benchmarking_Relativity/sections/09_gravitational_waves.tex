\section{Gravitational Waves and Binary Inspiral Decay}

One of the most stringent tests of General Relativity (GR) is the observation of gravitational waves, particularly through the orbital decay of binary pulsars. The first such indirect detection came from the Hulse--Taylor binary pulsar (PSR B1913+16).

\subsection*{GR Prediction}

According to GR, two orbiting masses emit energy via gravitational radiation. For PSR B1913+16, with orbital period $P_b = 7.75$ hours and eccentricity $e = 0.617$, the predicted orbital period derivative due to gravitational wave emission is:
\begin{equation}
\frac{dP_b}{dt}_\text{GR} = -2.4025\times10^{-12} \ \text{s/s}
\end{equation}

The observed decay, corrected for galactic acceleration, is:
\begin{equation}
\frac{dP_b}{dt}_\text{obs} = -2.4056(\pm 0.0051)\times10^{-12} \ \text{s/s}
\end{equation}

This agreement within 0.13\% is a hallmark success of GR~\cite{weisberg2016}. Direct detections by LIGO/Virgo~\cite{abbott2016} have further confirmed gravitational wave theory.

\section{Limitations of Incompressible VAM and Proposed Extensions}

The Vortex Æther Model (VAM) describes gravity via stationary æther vortices in an incompressible, inviscid medium. In such a medium, there is no mechanism for radiation from orbiting bodies. VAM would thus predict:
\begin{equation}
\frac{dP_b}{dt}_\text{VAM} \approx 0
\end{equation}

This is in stark contrast with observations. Table~\ref{tab:gw_comparison} summarizes the discrepancy.

\begin{table}[h!]
\centering
\caption{Binary Inspiral Decay Predictions and Observations}
\label{tab:gw_comparison}
\begin{tabular}{lccc}
\toprule
System & $\frac{dP}{dt}\textit{\text{GR}}$ (s/s) & $\frac{dP}{dt}\text{VAM}$ & $\frac{dP}{dt}\textit{\text{Obs}}$ (s/s) \\
\midrule
PSR B1913+16 & $-2.4025\times10^{-12}$ & $\sim 0$ & $-2.4056(51)\times10^{-12}$ \\
PSR J0737--3039A/B & $-1.252\times10^{-12}$ & $\sim 0$ & $-1.252(17)\times10^{-12}$ \\
GW150914 (BH merger) & $\sim 3M\odot c^2$ radiated & No GW & Direct detection (LIGO) \\
\bottomrule
\end{tabular}
\end{table}

To address this shortcoming, VAM must introduce a radiation mechanism. Several extensions have been proposed to enable gravitational radiation:

\begin{itemize}
\item Compressible æther – By making the æther \textit{slightly compressible or elastic}, orbital systems can excite longitudinal or transverse waves in the medium. If the compressibility is chosen such that the wave speed equals \textit{c}, these æther waves can play the role of gravitational waves.
\item Vortex shedding – Two rotating vortex knots in orbit could generate small vortices or turbulence in the æther (similar to a Von K'arm'an vortex street). With a small viscosity or coupling to a secondary field, energy can leak away as radiation.
\item Thermodynamic coupling – In VAM, entropy or temperature fields are also considered. Merging vortex knots could excite waves in such a field, analogous to massless spin-2 "gravitons" in the æther medium.
\end{itemize}

The remainder of this chapter elaborates on the compressible æther approach in detail.


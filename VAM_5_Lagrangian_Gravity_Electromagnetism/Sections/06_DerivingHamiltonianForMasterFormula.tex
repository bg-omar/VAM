\section{Hamiltonian Structure of VAM and Its Connection to Mass}

The Vortex \AE ther Model (VAM) not only admits a Lagrangian formulation but also supports a Hamiltonian structure, which allows us to define total energy, conserved quantities, and derive the Master Formula for mass from first principles. This section presents the Hamiltonian density \( \mathcal{H}_\text{VAM} \) associated with the structured vorticity field, and connects it to the observed rest mass of knotted vortex excitations.

\subsection{Hamiltonian Formulation of VAM and Connection to the Master Formula}

The Hamiltonian density \( \mathcal{H}_\text{VAM} \) captures the total energy per unit volume of a vortex excitation in the æther. From the Lagrangian:
\begin{equation}
\mathcal{L}_\text{VAM} =
\frac{1}{2} \rho_\text{\ae}^{(\text{fluid})} |\vec{v}|^2
- \frac{1}{2} \rho_\text{\ae}^{(\text{energy})} \lambda_g |\vec{\omega}|^2
+ \frac{\alpha_e}{2} (\vec{v} \cdot \vec{\omega})^2
- V(\vec{\omega}),
\end{equation}
we define the canonical momentum:
\begin{equation}
\vec{p} = \frac{\partial \mathcal{L}}{\partial \dot{\vec{v}}} = \rho_\text{\ae}^{(\text{fluid})} \vec{v},
\end{equation}
which yields the Hamiltonian density:
\begin{equation}
\mathcal{H}_\text{VAM} = \vec{p} \cdot \vec{v} - \mathcal{L}_\text{VAM}
= \frac{1}{2} \rho_\text{\ae}^{(\text{fluid})} |\vec{v}|^2
+ \frac{1}{2} \rho_\text{\ae}^{(\text{energy})} \lambda_g |\vec{\omega}|^2
- \frac{\alpha_e}{2} (\vec{v} \cdot \vec{\omega})^2
+ V(\vec{\omega}).
\end{equation}

\subsection*{Interpretation of Terms}
\begin{itemize}
    \item \( \frac{1}{2} \rho_\text{\ae}^{(\text{fluid})} |\vec{v}|^2 \): Bulk kinetic energy of æther flow.
    \item \( \frac{1}{2} \rho_\text{\ae}^{(\text{energy})} \lambda_g |\vec{\omega}|^2 \): Swirl energy, source of gravity and time dilation.
    \item \( -\frac{\alpha_e}{2} (\vec{v} \cdot \vec{\omega})^2 \): Energy cancellation due to helicity coupling (electromagnetism).
    \item \( V(\vec{\omega}) \): Topological potential energy stabilizing vortex knots.
\end{itemize}

\subsection*{Total Vortex Mass–Energy}

The total rest energy of a vortex configuration is:
\begin{equation}
E_\text{total} = \int_{\mathbb{R}^3} \mathcal{H}_\text{VAM}(\vec{x}) \, d^3x,
\end{equation}
which, in the non-relativistic limit, gives:
\[
M = \frac{1}{c^2} \int \mathcal{H}_\text{VAM} \, d^3x
\]

\subsection*{Connection to the Master Formula}

If we assume:
\begin{itemize}
    \item A vortex knot with localized energy,
    \item Dominance of swirl energy: \( \mathcal{H}_\text{VAM} \approx \frac{1}{2} \rho_\text{\ae}^{(\text{energy})} |\vec{v}|^2 \),
    \item A quantized knot volume \( V = \sum_i V_i \),
\end{itemize}
then:
\[
M = \frac{1}{c^2} \cdot \left( \frac{1}{2} \rho_\text{\ae}^{(\text{energy})} C_e^2 \right) \cdot \sum_i V_i,
\]
and when including topological scaling factors \( \eta, \xi, \tau \), the full expression becomes:
\[
M = \eta \cdot \xi \cdot \tau \cdot \left( \sum_i V_i \right) \cdot \left( \frac{1}{2} \rho_\text{\ae}^{(\text{energy})} C_e^2 \right) / c^2
\]
which is precisely the \textbf{Master Formula}.

\subsection*{Conclusion}

The Hamiltonian density formalism in VAM directly links to:
\begin{itemize}
    \item The inertial and gravitational mass of particles,
    \item Electromagnetic field energy via helicity,
    \item Topological stabilization via knot energy,
\end{itemize}
and provides a rigorous fluid-dynamical foundation for the vortex-based Master Formula governing rest mass.
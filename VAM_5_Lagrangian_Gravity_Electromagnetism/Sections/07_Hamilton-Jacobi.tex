\section{Emergent Phase Mechanics}
\subsection{Hamiltonian Derivation from VAM Lagrangian}

We derive the VAM Hamiltonian density starting from a generalized Lagrangian density appropriate for an incompressible, inviscid structured \ae ther:
\begin{equation}
\mathcal{L}_\text{VAM} = \frac{1}{2} \rho_\text{\ae}^{(\text{fluid})} |\vec{v}|^2
- \left( \frac{1}{2} \lambda_g \rho_\text{\ae}^{(\text{fluid})} |\vec{\omega}|^2
+ V(\vec{\omega}) - \frac{\alpha_e}{2} (\vec{v} \cdot \vec{\omega})^2 \right),
\end{equation}
where:
\begin{itemize}
  \item $\vec{v}$ is the \ae ther flow velocity,
  \item $\vec{\omega} = \nabla \times \vec{v}$ is the vorticity,
  \item $\lambda_g$ is the gravitational vorticity coupling,
  \item $\alpha_e$ is the helicity--charge coupling,
  \item $V(\vec{\omega})$ is a topological potential reflecting knot energy and swirl curvature.
\end{itemize}

The conjugate momentum density is:
\begin{equation}
\vec{p} = \frac{\partial \mathcal{L}_\text{VAM}}{\partial \vec{v}}
= \rho_\text{\ae}^{(\text{fluid})} \vec{v}
- \alpha_e (\vec{v} \cdot \vec{\omega}) \vec{\omega}.
\end{equation}

Then the Hamiltonian density is given by:
\begin{align*}
\mathcal{H}_\text{VAM} &= \vec{p} \cdot \vec{v} - \mathcal{L}_\text{VAM} \\
&= \left( \rho_\text{\ae} \vec{v} - \alpha_e (\vec{v} \cdot \vec{\omega}) \vec{\omega} \right) \cdot \vec{v}
- \left( \frac{1}{2} \rho_\text{\ae} |\vec{v}|^2 - U \right) \\
&= \rho_\text{\ae} |\vec{v}|^2 - \alpha_e (\vec{v} \cdot \vec{\omega})^2
- \frac{1}{2} \rho_\text{\ae} |\vec{v}|^2 + U \\
&= \frac{1}{2} \rho_\text{\ae} |\vec{v}|^2
- \alpha_e (\vec{v} \cdot \vec{\omega})^2
+ \left( \frac{1}{2} \lambda_g \rho_\text{\ae} |\vec{\omega}|^2 + V(\vec{\omega}) \right).
\end{align*}

Finally, grouping terms:
\begin{equation}
\boxed{
\mathcal{H}_\text{VAM} = \frac{1}{2} \rho_\text{\ae}^{(\text{fluid})} |\vec{v}|^2
+ \frac{1}{2} \lambda_g \rho_\text{\ae}^{(\text{fluid})} |\vec{\omega}|^2
+ V(\vec{\omega})
- \frac{\alpha_e}{2} (\vec{v} \cdot \vec{\omega})^2
}\label{eq:VAM_Hamiltonian}
\end{equation}

\subsection{ Hamilton--Jacobi Formulation in VAM}

In the Vortex \AE ther Model, the swirlclock phase \( S(\vec{x}, t) \) replaces the classical action as the central dynamical quantity from which vortex evolution, quantization, and wave-like behavior emerge. This section derives the Hamilton--Jacobi equation for structured \ae ther flows, showing how quantization and de Broglie-like behavior naturally arise from vorticity phase dynamics.

\subsection*{Swirlclock Phase and Hamilton--Jacobi Equation}

We begin with the VAM Hamiltonian density derived from the unified Lagrangian~\ref{eq:VAM_Hamiltonian}. Let \( S(\vec{x}, t) \) denote the swirlclock phase field. The \ae ther velocity is defined via the phase gradient~\cite{Arnold1998, moffatt1969degree}:
\begin{equation}
\vec{v} = \frac{1}{\rho_\text{\ae}^{(\text{fluid})}} \nabla S,
\end{equation}
analogous to the classical momentum \( \vec{p} = \nabla S \) in Hamilton–Jacobi theory.

Substituting into the Hamiltonian, we obtain the VAM Hamilton–Jacobi equation:
\begin{equation}
\frac{\partial S}{\partial t} + \frac{1}{2 \rho_\text{\ae}^{(\text{fluid})}} |\nabla S|^2
+ \Phi_\text{vortex} + \Phi_\text{helicity} + V(\vec{\omega}) = 0,
\end{equation}
where:
\begin{align*}
\Phi_\text{vortex} &= \frac{1}{2} \lambda_g \rho_\text{\ae}^{(\text{fluid})} |\vec{\omega}|^2, \\
\Phi_\text{helicity} &= -\frac{\alpha_e}{2} (\vec{v} \cdot \vec{\omega})^2.
\end{align*}
This equation describes the evolution of the swirlclock phase \( S \) in terms of the local flow velocity, vorticity, and topological interactions.


\subsection*{Quantization from Circulation Integrals}

Vortex knots exhibit quantized circulation due to single-valuedness of the phase field~\cite{Helmholtz1858, faddeev1997stable}:
\begin{equation}
\oint_\Gamma \nabla S \cdot d\vec{\ell} = 2\pi n \hbar_\text{\ae}, \qquad n \in \mathbb{Z},
\end{equation}
where \( \Gamma \) is any closed loop encircling a vortex core and \( \hbar_\text{\ae} \) is the ætheric phase quantum. This yields:
\begin{itemize}
    \item Circulation quantization,
    \item Energy–phase proportionality,
    \item Emergent Planck-like behavior without quantum postulates.
\end{itemize}

\subsection*{de Broglie--Vortex Relation}

From the Hamilton–Jacobi formalism and periodic knot motion, we recover a de Broglie-like relation~\cite{ranada1990topological}:
\begin{equation}
\lambda_\text{vortex} = \frac{2\pi}{|\nabla S|} \sim \frac{h_\text{\ae}}{M_\text{eff} v},
\end{equation}
where:
\begin{itemize}
    \item \( \lambda_\text{vortex} \) is the wavelength of the vortex excitation,
    \item \( M_\text{eff} \sim \rho_\text{\ae}^{(\text{energy})} V \) is the effective mass,
    \item \( h_\text{\ae} = 2\pi \hbar_\text{\ae} \) is the circulation quantum.
\end{itemize}

This connects phase topology to particle–wave duality:
\begin{itemize}
\item Knots with longer \( \lambda_\text{vortex} \) correspond to lower momentum modes,
\item High-curvature knots (short \( \lambda \)) behave like massive, localized particles.
\end{itemize}

\subsection*{Explicit Hamilton--Jacobi Derivation from Phase Dynamics}

We derive the Hamilton--Jacobi equation explicitly in terms of the swirlclock phase field \( S(\vec{x}, t) \).

\paragraph{Phase--Velocity Relation.}
We begin by assuming that the \ae ther velocity arises from the gradient of a scalar phase field:
\begin{equation}
\vec{v} = \frac{1}{\rho_\text{\ae}^{(\text{fluid})}} \nabla S(\vec{x}, t),
\end{equation}
as originally suggested by analogies to classical Hamilton--Jacobi mechanics~\cite{arnold1998topological, moffatt1969degree}.

\paragraph{Hamiltonian with Substitution.}
Substituting into the Hamiltonian:
\begin{equation}
\mathcal{H}_\text{VAM} = \frac{1}{2 \rho_\text{\ae}} |\nabla S|^2 + \frac{1}{2} \lambda_g \rho_\text{\ae} |\vec{\omega}|^2 + V(\vec{\omega}) - \frac{\alpha_e}{2 \rho_\text{\ae}^2} (\nabla S \cdot \vec{\omega})^2,
\end{equation}
where we retain \( \vec{\omega} \) explicitly for generality, noting that \( \vec{\omega} = 0 \) for globally irrotational flow unless \( S \) is multivalued.

\paragraph{Hamilton--Jacobi Equation.}
The generalized VAM Hamilton--Jacobi equation becomes:
\begin{equation}
\boxed{
\frac{\partial S}{\partial t} + \frac{1}{2 \rho_\text{\ae}} |\nabla S|^2 + \Phi_\text{swirl} + \Phi_\text{helicity} + V(\vec{\omega}) = 0
}
\end{equation}
with:
\begin{align*}
\Phi_\text{swirl} &= \frac{1}{2} \lambda_g \rho_\text{\ae} |\vec{\omega}|^2, \\
\Phi_\text{helicity} &= -\frac{\alpha_e}{2 \rho_\text{\ae}^2} (\nabla S \cdot \vec{\omega})^2.
\end{align*}

\paragraph{Quantization from Circulation Integrals.}
Quantization arises from the requirement that \( \psi = A e^{i S/\hbar_\text{\ae}} \) be single-valued~\cite{helmholtz1858integrals, faddeev1997stable}:
\begin{equation}
\oint_\Gamma \nabla S \cdot d\vec{\ell} = 2\pi n \hbar_\text{\ae}, \quad n \in \mathbb{Z},
\end{equation}
so that each knotted vortex loop carries a quantized circulation:
\begin{equation}
\Gamma_n = \frac{2\pi n \hbar_\text{\ae}}{\rho_\text{\ae}^{(\text{fluid})}}.
\end{equation}

\paragraph{Emergent Wavefunction and Schrödinger Form.}
Define:
\begin{equation}
\psi(\vec{x}, t) = \sqrt{\rho(\vec{x}, t)} e^{i S(\vec{x}, t)/\hbar_\text{\ae}},
\end{equation}
and under a Madelung transformation, this phase dynamics yields:
\begin{equation}
\boxed{
i \hbar_\text{\ae} \frac{\partial \psi}{\partial t} = -\frac{\hbar_\text{\ae}^2}{2 \rho_\text{\ae}} \nabla^2 \psi + \left[\Phi_\text{swirl} + \Phi_\text{helicity} + V(\vec{\omega})\right] \psi
}
\end{equation}
This nonlinear Schrödinger-like equation describes emergent quantum behavior from vortex structure.

\paragraph{Summary.}
\begin{equation}
\boxed{
\frac{\partial S}{\partial t} + \frac{1}{2 \rho_\text{\ae}} |\nabla S|^2 + \frac{1}{2} \lambda_g \rho_\text{\ae} |\vec{\omega}|^2 - \frac{\alpha_e}{2 \rho_\text{\ae}^2} (\nabla S \cdot \vec{\omega})^2 + V(\vec{\omega}) = 0
}
\end{equation}
This equation governs the evolution of the swirlclock phase \( S(\vec{x}, t) \) in a topologically structured, rotating \ae ther field, embedding vortex quantization and emergent wavefunction dynamics.

\subsection*{Conclusion}

The Hamilton–Jacobi formulation within the Vortex \AE{}ther Model (VAM) reveals that the swirlclock phase \( S(\vec{x}, t) \) governs the dynamics of structured æther flows through a nonlinear, vorticity-sensitive energy functional. From this phase dynamics, circulation quantization emerges naturally as a topological constraint, linking vortex knots to discrete energy levels without postulating quantum axioms.


By interpreting $S$ as the generator of motion and embedding it into a Madelung-type wavefunction, we recover a Schrödinger-like evolution equation where mass, momentum, and wavelength are not fundamental inputs but emergent consequences of vortex geometry, helicity coupling, and æther tension.


This unifies classical fluid dynamics, gravitational potential, and quantum phase coherence into a single topological-vorticity framework, suggesting that particle–wave duality and quantization are manifestations of deeper, knotted æther structures.



\section{Emergent Constants from Fluid Analogs}
Derivations of $\hbar_{\text{VAM}}$ and charge coupling follow:

\begin{align}
    \hbar_{\text{VAM}} &= m_f C_e r_c \\
    e^2 &= 8\pi m_e C_e^2 r_c \\
    \Gamma &= \frac{h}{m} = 2\pi r_c C_e
\end{align}

These reinterpret Planck-scale constants as emergent quantities from measurable \ae{}ther dynamics and flow quantization, aligning with results from BEC vortex systems~\cite{Pethick2008BEC, Donnelly1991QuantizedVortices}.

In this formulation, each field and interaction of the Standard Model gains a mechanical analog in the \ae{}ther medium. The Lagrangian no longer relies on abstract symmetry principles alone, but instead emerges from vortex dynamics, circulation, density modulation, and topological structure within a unified fluid framework.

\subsection{Temporal Modes in Derived Quantities}
Each term here evolves over one or more temporal layers from the VAM ontology:
\begin{itemize}
    \item $\hbar_{\text{VAM}}$ derives from internal vortex phase time $S(t)$.
    \item $\Gamma$ evolves along $T_v$ (vortex proper time) due to its loop-based circulation.
    \item Effective mass and energy terms appear as modulations over $\tau$ (observer proper time).
\end{itemize}

\subsection{Mathematical Derivation of the VAM-Lagrangian}

Kinetic energy of a vortex structure, or the local energy density in a vortex field:
\[
    \mathcal{L}_\text{kin} = \frac{1}{2}\rho_\text{\ae} C_e^2
\]

Field energy and gauge terms, field tensors follow from Helmholtz vorticity:
\[
    \mathcal{L}_\text{veld} = -\frac{1}{4}F_{\mu\nu}F^{\mu\nu}
\]

Mass as inertia from circulation, where the fermion mass is determined by circulation:
\[
    \Gamma = 2\pi r_c C_e \quad\Rightarrow\quad m \sim \rho_\text{\ae} r_c^3
\]

Pressure and stress potential of \ae{}ther condensate, where the pressure balance is described by the stress field:
\[
    V(\phi) = -\frac{F^{\text{max}}_{\text{\ae}}}{r_c}|\phi|^2 + \lambda|\phi|^4
\]

Topological terms for the conservation of vortex fields helicity:
\[
    \mathcal{H} = \int \vec{v}\cdot\vec{\omega}\, dV
\]

Each of these Lagrangian contributions aligns with distinct temporal behaviors:
\begin{itemize}
    \item Kinetic terms evolve over $\tau$.
    \item Helicity terms encode phase evolution in $S(t)$.
    \item The Higgs potential corresponds to a stability condition in global $\mathcal{N}$.
\end{itemize}

\begin{table}[H]
    \centering
    \scriptsize
    \renewcommand{\arraystretch}{1.4}
    \begin{tabular}{|l|l|l|l|}
        \hline
        \textbf{SM Term} & \textbf{Mathematical Form} & \textbf{VAM Analog} & \textbf{Fluid-Dynamic Interpretation} \\
        \hline
        \makecell[l]{Fermion Kinetic \\ Term} &
        $\bar{\psi}(i\gamma^\mu D_\mu)\psi$ &
        $\rho_\text{\ae} \vec{v}^2$ &
        \makecell[l]{Kinetic energy of topological vortex knot (fermion)} \\
        \hline
        \makecell[l]{Gauge Field \\ Kinetic Term} &
        $-\frac{1}{4}F_{\mu\nu}F^{\mu\nu}$ &
        $\rho_\text{\ae} (\vec{v} \cdot \nabla \times \vec{v})$ &
        \makecell[l]{Swirl helicity (fluid analog of gauge field energy)} \\
        \hline
        Fermion Mass Term &
        $m\bar{\psi}\psi$ &
        $\rho_{core} C_e^2$ &
        \makecell[l]{Core pressure from tangential circulation of vortex} \\
        \hline
        \makecell[l]{Higgs Field \\ Kinetic Term} &
        $\frac{1}{2}(\partial_\mu \phi)^2$ &
        $\frac{1}{2}(\nabla \phi)^2$ &
        \makecell[l]{Elastic strain in scalar potential field of \AE{}ther} \\
        \hline
        Higgs Potential &
        $V(\phi) = -\mu^2\phi^2 + \lambda \phi^4$ &
        $\lambda \phi^4 (1 - \phi^2/F^{\text{max}}_{\text{\ae}}^2)$ &
        \makecell[l]{Compressibility-induced pressure potential} \\
        \hline
        Yukawa Coupling &
        $y\bar{\psi}\phi\psi$ &
        $\rho_\text{\ae} \phi$ &
        \makecell[l]{Topological mass coupling via scalar compression} \\
        \hline
        Gauge Coupling &
        $D_\mu = \partial_\mu - igA_\mu$ &
        $\vec{v} + \vec{A}_{\text{swirl}}$ &
        \makecell[l]{Swirl-mediated interaction velocity} \\
        \hline
        QCD Term &
        $G_{\mu\nu}^a G^{\mu\nu}_a$ &
        -- &
        \makecell[l]{Conservation of angular momentum in \\ trichiral vortex flows} \\
        \hline
        EM Coupling &
        $q\bar{\psi}\gamma^\mu A_\mu \psi$ &
        $\Gamma \cdot \chi$ &
        \makecell[l]{Charge as circulation magnitude and chirality} \\
        \hline
        Chiral Asymmetry &
        -- &
        Knot handedness &
        \makecell[l]{Topological chirality determines weak \\ interaction selectivity} \\
        \hline
    \end{tabular}
    \caption{Comparison of Standard Model Lagrangian terms with their VAM fluid-dynamic analogs and their associated temporal modes.}
    \label{tab:SMtoVAM}
\end{table}

\subsection*{Supporting Experimental and Theoretical Observations}
The VAM is consistent with experimentally and theoretically confirmed phenomena such as vortex stretching, helicity conservation and mass-inertia couplings~\cite{batchelor1953, vinen2002, bewley2008, moffatt1969, kleckner2013, scheeler2014, bartlett1986}.

This reformulation offers a physically intelligible and topologically rich counterpart to the Standard Model---one grounded in measurable fluid properties and structured time evolution, rather than abstract gauge symmetries alone.

\subsection{Quantized Swirl Fields via Mode Expansion}

In conventional quantum field theory (QFT), the quantization of fields arises from harmonic mode expansions that map classical field solutions to quantum operators. Each normal mode of the field is associated with a pair of creation and annihilation operators, leading to a discrete energy spectrum. Inspired by this formalism, we propose an analogous quantization framework for the Vortex \AE{}ther Model (VAM), in which the fluid velocity field $\vec{v}(\vec{x}, t)$ is expanded in a basis of knotted vortex modes.

We define the swirl field operator as:
\begin{equation}
\vec{v}(\vec{x}, t) = \sum_n \left[ \vec{v}_n(\vec{x})\, a_n e^{-i\omega_n S(t)} + \vec{v}_n^*(\vec{x})\, a_n^\dagger e^{i\omega_n S(t)} \right],
\end{equation}
where $a_n$ and $a_n^\dagger$ denote the annihilation and creation operators for the $n$-th vortex mode, and $\omega_n$ is the angular frequency associated with the core circulation and knot topology. Here, time evolution occurs over the swirl-clock phase $S(t)$.

Each $\vec{v}_n(\vec{x})$ represents a quantized topological excitation of the \ae{}ther, corresponding to distinct vortex knot configurations or harmonics. These excitations can be labeled by their helicity, circulation quantum $\Gamma_n$, and winding number $L_k$, akin to quantized angular momentum states in quantum mechanics.

This expansion justifies the discrete energy spectrum observed in vortex-based particle models. For example, the energy of a vortex excitation is:
\begin{equation}
E_n = \hbar_{\text{VAM}} \omega_n = \rho_{\ae} \Gamma_n r_c^2 \omega_n,
\end{equation}
with $\hbar_{\text{VAM}}$ interpreted as a fluid-circulation-based quantum of action:
\begin{equation}
\hbar_{\text{VAM}} \equiv \rho_{\ae} \Gamma_n r_c^2.
\end{equation}

This formulation is aligned with canonical quantization procedures in QFT~\cite{verlinde2021qft}, and also with the formal mode expansions of collective excitations in superfluid systems~\cite{pethick2002bose} and knotted vortex models~\cite{kleckner2013creation}. It enables a rigorous interpretation of particles as quantized, topologically distinct excitations of the swirl field.

This framework can also extend to include internal excitation spectra of vortex cores, thereby suggesting a natural pathway for encoding flavor states and even mixing matrices in terms of mode-coupled vortex families.

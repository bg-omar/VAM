\section{Variational Derivation of the Swirl Lagrangian}

To rigorously support the Vortex \AE{}ther Model (VAM), we derive the swirl Lagrangian using a variational principle analogous to classical field theory. This establishes a formal path from \ae{}ther vortex dynamics to field-theoretic particle analogs.

\subsection{Field Structure and Helmholtz Decomposition}

The \ae{}ther velocity field $\mathbf{v}(\mathbf{x}, t)$ is decomposed via Helmholtz's theorem:

\begin{equation}
\mathbf{v} = \nabla \theta + \mathbf{A},
\end{equation}

where $\theta$ is a scalar potential (irrotational component), and $\mathbf{A}$ is the divergence-free vector potential representing swirl, with $\nabla \cdot \mathbf{A} = 0$. The vorticity field is:

\begin{equation}
\boldsymbol{\omega} = \nabla \times \mathbf{v} = \nabla \times \mathbf{A}.
\end{equation}

\subsection{Action Functional and Swirl Gauge Field}

We define the action $S$ as:

\begin{equation}
S[\theta, \mathbf{A}] = \int d^4x \, \mathcal{L}_{\text{VAM}},
\end{equation}

where the Lagrangian density is:

\begin{equation}
\mathcal{L}_{\text{VAM}} = \frac{1}{2} \rho (\nabla \theta + \mathbf{A})^2 - \lambda (|\phi|^2 - F^{\text{max}}_{\text{\ae}}^2)^2 - \frac{1}{4} S_{\mu\nu} S^{\mu\nu} + \left( \frac{\rho_{\text{\ae}} r_c^2}{C_e} \right) (\mathbf{v} \cdot \boldsymbol{\omega}).
\end{equation}

\paragraph{Temporal Interpretation.}
Each term in this Lagrangian evolves over distinct time layers:
\begin{itemize}
  \item The scalar phase $\theta$ evolves over $S(t)$, the swirl-clock phase.
  \item The vortex vector potential $\mathbf{A}$ evolves over proper time $\tau$.
  \item The helicity term $\vec{v} \cdot \vec{\omega}$ encodes twist evolution over $T_v$.
  \item The action integral spans global causal time $\mathcal{N}$.
\end{itemize}

In this form:
\begin{itemize}
    \item The second term is a self-generated core potential representing stress from radial \ae{}ther compression, replacing $\rho \Phi$.
    \item $S_{\mu\nu} = \partial_\mu W_\nu - \partial_\nu W_\mu$ is the swirl field strength tensor, with $W_\mu = (\phi, \mathbf{A})$.
    \item The final term is a helicity-density-based coupling, with $\rho_{\text{\ae}}$ the \ae{}ther density, $r_c$ the vortex core radius, and $C_e$ the swirl velocity (effective light speed).
\end{itemize}

\subsection{Euler--Lagrange Equations and Continuity}

Varying the action with respect to $\theta$ recovers the continuity equation:

\begin{equation}
\partial_t \rho + \nabla \cdot (\rho \mathbf{v}) = 0.
\end{equation}

Variation with respect to $\mathbf{A}$ gives a generalized swirl equation of motion:

\begin{equation}
\rho \mathbf{v} - \nabla \cdot \left( \frac{\partial \mathcal{L}_{\text{swirl}}}{\partial (\nabla \mathbf{A})} \right) + \left( \frac{\rho_{\text{\ae}} r_c^2}{C_e} \right) \boldsymbol{\omega} = 0.
\end{equation}

This coupling of vorticity to mass-like topological terms gives rise to effective inertial behavior.

\subsection{Mass from Topology and Helicity}

The helicity density $h = \mathbf{v} \cdot \boldsymbol{\omega}$ is interpreted as a local "spin clock rate" of vortex knots. Integrated over a topologically linked region, it yields:

\begin{equation}
m_{\text{eff}} \sim \left( \frac{\rho_{\text{\ae}} r_c^2}{C_e} \right) \int_V \mathbf{v} \cdot \boldsymbol{\omega} \, d^3x.
\end{equation}

This expression ties particle mass directly to topological properties such as twist, writhe, and linking number of the vortex core, and to the local swirl time rate $dS/d\mathcal{N}$.

\subsection{Outlook: Quantization Path}

The swirl gauge field admits canonical quantization via:

\begin{align}
\Pi^\mu &= \frac{\partial \mathcal{L}}{\partial (\partial_0 W_\mu)}, \\
[W_\mu(\mathbf{x}), \Pi^\nu(\mathbf{x}')] &= i \delta^\nu_\mu \delta^3(\mathbf{x} - \mathbf{x}'),
\end{align}

and path integral representation:

\begin{equation}
Z = \int \mathcal{D}[W_\mu] \exp\left(i \int d^4x \, \mathcal{L}_{\text{VAM}}\right).
\end{equation}

This establishes a formal pathway to embedding the Vortex \AE{}ther Model in a quantum field-theoretic setting, while preserving its topological and hydrodynamic origins.


\section{Canonical Commutators and Swirl Quantization}

To formulate a consistent quantum field theory from the Vortex \AE{}ther Model (VAM), it is essential to specify canonical commutation relations between fundamental fluid observables. In standard quantum field theory, canonical quantization imposes:
\begin{equation}
[\phi(x), \pi(y)] = i \delta(x - y),
\end{equation}
where $\phi$ is a field and $\pi$ its conjugate momentum.

We propose that a similar structure exists in the VAM, where the swirl potential $\theta(\vec{x})$ and the \ae{}ther density $\rho_{\ae}(\vec{x})$ form a canonical pair:
\begin{equation}
[\theta(\vec{x}), \rho_{\ae}(\vec{y})] = i \delta^3(\vec{x} - \vec{y}),
\end{equation}

implying an uncertainty relation between vortex phase and \ae{}ther mass density, akin to the number-phase relation in Bose fluids. Here, $\theta$ evolves over $S(t)$, and $\rho_{\ae}$ modulates energy density over $\tau$.

Alternatively, one may define canonical brackets between the velocity and vorticity fields:
\begin{equation}
[v_i(\vec{x}), \omega_j(\vec{y})] \sim i \epsilon_{ijk} \partial_k \delta^3(\vec{x} - \vec{y}),
\end{equation}
consistent with the Lie algebra structure of vector fields under the Helmholtz decomposition.

This structure leads to a Hamiltonian formalism for VAM fluid dynamics:
\begin{equation}
\mathcal{H}[\theta, \rho_{\ae}] = \int d^3x \left[ \frac{1}{2} \rho_{\ae}(\vec{x})\, |\nabla \theta(\vec{x})|^2 + V(\rho_{\ae}) \right],
\end{equation}
where $V(\rho_{\ae})$ represents the potential energy density of the \ae{}ther medium, potentially including self-interaction or compressibility terms.

The formal identification of conjugate variables and commutators in the VAM allows quantization of vortex excitations through standard Fock space methods, in close analogy with the quantized phonon and roton spectra of superfluid helium systems~\cite{fetter1971nonuniform, stone2000superfluidity, verlinde2021qft}.

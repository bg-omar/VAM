\section{Helicity Interference Suppression Term from Vortex Knot Packing}

In the Vortex \AE{}ther Model (VAM), mass arises from swirl energy stored in knotted structures within the incompressible \ae{}ther. For composite particles composed of multiple vortex cores (e.g., protons, nuclei), mutual interference between individual swirl fields reduces the net helicity, thereby suppressing effective inertial mass. This is a manifestation of decoherence in swirl clock phase \( S(t) \) and topological alignment over vortex time \( T_v \).

\subsection*{From Naive Energy to Corrected Mass}

We begin with a naive expression for mass derived from internal vortex energy:

\[
M_0 = \frac{1}{2} \rho_\text{\ae} C_e^2 V
\]

This raw energy must be amplified by appropriate coupling constants and then corrected for geometric interference and coherence losses. The evolution proceeds through the following stages:

\begin{center}
\begin{tabular}{|c|l|l|}
\hline
\textbf{Level} & \textbf{Formula} & \textbf{Interpretation} \\
\hline
0 & \( M_0 = \frac{1}{2} \rho_\text{\ae} C_e^2 V \) & Raw swirl energy \\
1 & \( M_1 = \frac{4}{\alpha} \cdot M_0 \) & Electromagnetic scaling \\
2 & \( M_2 = \frac{4}{\alpha \varphi} \cdot M_0 \) & Topological amplification (golden coupling) \\
3 & \( M_3 = \frac{4}{\alpha \varphi} \cdot M_0 \cdot \xi(n) \cdot \varphi^{-s} \cdot \left( \frac{1}{m} \right)^{3/2} \) & Full coherence, torsion, threading correction \\
\hline
\end{tabular}
\end{center}

\subsection*{Coherence Suppression Term \( \xi(n) \)}

To model the interference between \( n \) tightly packed knots, we define a suppression factor:

\begin{equation}
\boxed{
\xi(n) = 1 - \beta \cdot \log(n)
}
\qquad \text{with } \beta \approx 0.06
\end{equation}

This logarithmic form reflects the sublinear growth of helicity interference due to angular misalignment and phase cancellation in densely packed composite vortex systems.

In later refinements (Section~\ref{sec:master-mass-formula}), this empirical form is replaced by a golden-ratio derived suppression:

\[
\boxed{
\xi(n) = n^{-1/\varphi}
}
\]

which is exact, dimensionless, and derivable from the nested interference of swirl clocks in a knotted network.

\paragraph{Temporal Ontology Interpretation:}
\begin{itemize}
\item  Misaligned swirl clocks \( S(t) \) among the constituent knots create destructive interference in the energy-bearing modes.
\item  The more vortex cores interact within a composite knot, the more swirl phases decohere across \( T_v \).
\item  This leads to a \textbf{nonlinear loss of mass energy}, modeled via \( \xi(n) \).
\end{itemize}

This suppression term, whether in logarithmic or golden-ratio form, plays a critical role in making mass additive only under specific topological alignment conditions. It is this interference that distinguishes tightly bound baryons from loosely coupled molecular structures in the VAM framework.

\paragraph{Derivation (Temporal Ontology):}
The total helicity of a multi-core knot system evolves over $T_v$ and is governed by cross-terms in the global action integral over $\mathcal{N}$:
\begin{equation}
\mathcal{H}_{\text{total}} = \sum_i \mathcal{H}_i + \sum_{i \neq j} \int_V \vec{v}_i \cdot (\nabla \times \vec{v}_j) \, dV
\end{equation}

Cross-helicity terms degrade $S(t)$ coherence between adjacent knots and are generally negative:
\[
\sum_{i \neq j} \mathcal{H}_{ij} \sim -\log(n)
\]

This gives an effective helicity:
\[
\mathcal{H}_{\text{eff}} \sim n - \log(n)
\quad \Rightarrow \quad
\xi(n) = \frac{\mathcal{H}_{\text{eff}}}{n} = 1 - \beta \log(n)
\]

with $\beta$ encoding the average interference per additional knot.

\paragraph{Refined Mass Formula with Topological Correction:}
\begin{equation}
\boxed{
M = \left( \frac{1}{\varphi} \right) \cdot \left( \frac{4}{\alpha} \right) \cdot \underbrace{\left(1 - \beta \log(n)\right)}_{\text{inter-knot interference}} \cdot \left( \frac{1}{2} \rho_\text{\ae} C_e^2 V \right)
}
\end{equation}

\paragraph{Temporal Interpretation:}
\begin{itemize}
  \item $\frac{1}{\varphi}$: packing constraint from stable $T_v$ embedding.
  \item $\frac{4}{\alpha}$: vortex–electromagnetic coupling, derived from $S(t)$ alignment.
  \item $\xi(n)$: suppression of coherent swirl contribution due to $S(t)$ interference.
  \item $\rho_\text{\ae} C_e^2 V$: raw swirl energy integrated over local observer frame $\tau$.
\end{itemize}

This correction accounts for known deviations (e.g., in He–Be systematics) and reveals a fluid-dynamic origin for inertial decoherence in the multi-knot domain of the Vortex \AE{}ther Model.

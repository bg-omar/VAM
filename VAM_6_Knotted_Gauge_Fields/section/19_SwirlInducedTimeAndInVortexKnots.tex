\section{Swirl-Induced Time and Clockwork in Vortex Knots}

In the Vortex \AE{}ther Model (VAM), stable knotted structures are not only inertial carriers but also the fundamental generators of local time. Their internal swirl — characterized by a tangential rotation speed \( C_e \) around a core radius \( r_c \) — creates a persistent, anisotropic stress field in the surrounding æther. This results in an emergent axial flow that aligns with a preferred temporal direction, forming a \emph{swirl clock filament} governed by the local vortex time \( T_v \). This structure functions as a screw-like thread through the global æther time \( \mathcal{N} \), encoding a physically traceable "arrow of time."

\subsection*{Cosmic Chirality and Swirl-Time Asymmetry}

Analogous to magnetic domains, vortex knots in VAM may exhibit global chirality alignment due to spontaneous symmetry breaking in the early æther state. A net left-handed (ccw) chirality across cosmological domains induces a preferred sign in \( S(t) \), the swirl-phase time of embedded knots. This framework naturally explains:
\begin{itemize}
  \item the asymmetry between matter and antimatter (mirror chirality),
  \item a global temporal vector field through \( \mathcal{N} \),
  \item and synchronized proper time rates among long-range vortex-bound systems.
\end{itemize}

\subsection*{Swirl Helicity as a Local Time Generator}

Instead of relativistic time dilation, VAM posits that the local clock rate is proportional to the helicity density in the swirl field:
\[
    dt_{\text{local}} \propto \frac{dr}{\vec{v} \cdot \vec{\omega}},
\]
where \( \vec{v} \) is the æther flow velocity and \( \vec{\omega} = \nabla \times \vec{v} \) is the vorticity. Their dot product \( \mathcal{H} = \vec{v} \cdot \vec{\omega} \) defines the swirl helicity, which tracks the \emph{internal rotation rate of time itself} in the knot’s local frame.

We define the differential proper time \( d\tau \) experienced by a knotted core as:
\[
    \boxed{
    d\tau = \lambda \, (\vec{v} \cdot \vec{\omega}) \, dt
    }
\]
with \( \lambda \sim \frac{r_c^2}{C_e^2} \) for dimensional consistency. In this view, a knot's internal spin-cycle (swirl clock \( S(t) \)) becomes the physical source of \( \tau \), not imposed externally but emerging from æther dynamics.

\subsection*{Temporal Networks and Gravitational Bundles}

Vortex knots tend to align along coherent swirl filaments — quasi-topological “time wires” within the æther. These filaments, embedded in the causal æther frame \( \mathcal{N} \), form networks where:
\begin{itemize}
  \item gravitational attraction is described as a gradient in swirl coherence,
  \item time dilation emerges from the helicity divergence \( \nabla \cdot (\vec{v} \cdot \vec{\omega}) \),
  \item and the global arrow of time is induced by conserved circulation within \( T_v \).
\end{itemize}

Massive structures act as helicity sinks, modulating the local density of temporal phase evolution and inducing a topological time flow.

\section{Helicity-Induced Time Dilation}

Following the swirl-clock framework, we define the proper time dilation of a vortex as the ratio of local angular frequency to its intrinsic base rate:
\[
    \frac{d\tau}{dt} = \frac{\omega_{\text{obs}}}{\omega_0},
\]
where \( \omega_0 \) is the untensioned vortex frequency in a neutral æther, and \( \omega_{\text{obs}} \) is the externally observed rotation rate affected by local swirl topology.

\subsection*{Swirl Drag from Helicity Density}

Let us define helicity density:
\[
    \mathcal{H} = \vec{v} \cdot \vec{\omega}
\]
as a local time drag field. In regions of high helicity, we posit that topological entanglement imposes a torque resistance on internal vortex spin, reducing \( \omega_{\text{obs}} \) as:
\[
    \omega_{\text{obs}} = \omega_0 \left( 1 - \alpha \cdot \frac{\mathcal{H}}{C_e \cdot \omega_0} \right),
\]
where \( \alpha \) is a dimensionless swirl-drag coupling constant.

Thus, proper time evolves as:
\[
    \boxed{
    \frac{d\tau}{dt} = 1 - \alpha \cdot \frac{\vec{v} \cdot \vec{\omega}}{C_e \cdot \omega_0}
    }
\]
This shows that the local clock rate is decelerated by helicity-induced inertia — a topological analog of time dilation.

\subsection*{Observability and Experimental Relevance}

Such effects could be probed in:
\begin{itemize}
  \item toroidal BECs with engineered vorticity gradients,
  \item spinor superfluids under controlled swirl injection,
  \item or photon-ring interferometers tracking swirl phase delays.
\end{itemize}

The time retardation \( \Delta \tau \) becomes a measurable phase delay in systems where \( \vec{v} \) and \( \vec{\omega} \) can be externally tuned, offering a new experimental route to detect swirl-induced temporal structure.

\paragraph{Conclusion:} Time in VAM is an emergent, topologically grounded property of vortex motion — defined locally by helicity density, and globally by knot evolution over \( T_v \) and the causal æther frame \( \mathcal{N} \). This replaces the need for relativistic spacetime curvature with fluid-dynamical clockwork.

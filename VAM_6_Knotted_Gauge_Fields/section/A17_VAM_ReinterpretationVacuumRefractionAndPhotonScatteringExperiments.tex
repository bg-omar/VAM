\section{VAM-Based Reinterpretation of Vacuum Refraction and Photon Scattering Experiments}

This section reframes key experimental proposals and simulation results from recent literature within the theoretical structure of the Vortex \AE{}ther Model (VAM). In VAM, all electromagnetic and gravitational phenomena arise from structured vorticity in an inviscid, incompressible \ae{}ther, and thus vacuum nonlinearities are interpreted not as quantum-loop corrections, but as topological and dynamical features of \ae{}ther swirl.

\subsection{Refraction of Light by Light in Vacuum~\cite{sarazin2016refraction}}

\textbf{Original QED Context:} Sarazin et al.\ propose to detect a rotation of the wavefronts of a probe laser pulse traversing a transverse vacuum refractive index gradient created by two counter-propagating pump pulses. The expected refraction angle $\theta_r \sim 5 \times 10^{-12}$~rad arises from nonlinear QED effects governed by the Heisenberg--Euler Lagrangian.

\textbf{VAM Interpretation:} In VAM, this refraction is caused by a transverse \ae{}ther swirl pressure gradient $\nabla P_\text{swirl}$ induced by the counter-rotating pump pulses. The localized overlap region forms a toroidal vortex concentration that modifies the effective propagation speed of the probe's swirl structure:
\begin{equation}
\theta_r^\text{VAM} \sim \int \frac{1}{v} \frac{d}{dt} v_\perp(x) \, dt \approx \frac{\Delta v}{v} \approx \frac{1}{2} \frac{\nabla P_\text{swirl}}{\rho_\text{\ae}^{(\text{fluid})} v^2}
\end{equation}
This matches the magnitude predicted by QED, but VAM further predicts chirality-sensitive deflections depending on the internal vortex orientation of the probe pulse.

\subsection{3D Semi-Classical Simulation of Quantum Vacuum Effects~\cite{zhang2025computational}}

\textbf{Original QED Context:} Zhang et al.\ simulate vacuum birefringence and four-wave mixing using a semi-classical Heisenberg--Euler Maxwell solver. They benchmark against analytical results and identify harmonic generation and astigmatic beam deformation in the output pulse.

\textbf{VAM Interpretation:} The observed four-wave mixing harmonics correspond to the creation of transient knotted swirl structures in the \ae{}ther. The persistence of the third harmonic aligns with stable vortex ring formation, while evanescent harmonics reflect unstable topological interactions. The group velocity transition of the output pulse from stationary to $0.99c$ is interpreted in VAM as escape from a swirl-induced local time dilation region:
\begin{equation}
v_\text{group}^\text{VAM}(t) = c \cdot \sqrt{1 - \frac{U_\text{swirl}(t)}{U_\text{max}}}
\end{equation}
This matches the simulation's observed temporal evolution and highlights VAM's ability to model nontrivial spacetime analogs in a flat \ae{}ther framework.

\subsection{Search for Optical Nonlinearity in Vacuum with Intense Laser~\cite{battesti2013search}}

\textbf{Original QED Context:} Battesti and Rizzo review approaches to detect QED nonlinearities in vacuum via ellipticity, polarization rotation, and diffraction.

\textbf{VAM Interpretation:} These optical anomalies are reinterpreted as interactions with localized swirl nodes or vortex fields generated by intense EM pulses. Any polarization rotation or birefringence is attributed to anisotropic coupling of the probe vortex chirality to background swirl, rather than virtual electron-positron loops. Experiments using circularly polarized or OAM-encoded beams are optimal for detecting these VAM-predicted effects.

\subsection{Stimulated Photon Emission from the Vacuum~\cite{karbstein2016stimulated}}

\textbf{Original QED Context:} Karbstein and Shaisultanov propose that intense counter-propagating laser beams can stimulate photon emission from the quantum vacuum, interpreted as a non-perturbative scattering process involving the nonlinear effective Lagrangian.

\textbf{VAM Interpretation:} In VAM, the intense standing wave formed by counter-propagating beams generates a coherent swirl concentration that acts as a dynamical emitter of photons due to topological pressure gradients and knot relaxation. The emitted photons correspond to detangled swirl quanta escaping the high-swirl core. Harmonics arise naturally as topological mode conversions between unknotted and multiply-twisted vortex rings:
\begin{equation}
N_{\omega}^\text{VAM} \sim \left(\frac{U_\text{swirl}}{U_\text{core}}\right)^3 \cdot \tau \cdot \int_{\Delta \Omega} \mathcal{T}(\vec{\omega}, \hat{k}) \, d\Omega
\end{equation}
Here, $\mathcal{T}(\vec{\omega}, \hat{k})$ is a swirl-alignment transfer function analogous to the polarization-resolved emission density. The observed angular dependence and polarization mismatch are naturally explained as swirl escape asymmetry from the toroidal vortex core.

%! Author = mr
%! Date = 5/31/25

\section{Derivation of the Gravitational Constant from Æther Topology}
\label{appendix:G}

The gravitational constant \( G \) is typically introduced as a fundamental coupling constant in Newtonian and relativistic gravity. In the Vortex Æther Model (VAM), we reinterpret \( G \) as an emergent coefficient linking æther tension, knot dynamics, and Planck-scale constraints.

\subsection*{Maximum Force Principle from GR}

General Relativity suggests a maximum force limit in nature \cite{scharf2016force, barcelo2011}:

\begin{equation}
    F^{\text{max}}_{\text{gr}} = \frac{c^4}{4G}
\end{equation}

This is interpreted in VAM as the ultimate tensile strength of the æther medium—above which vortex structures cannot stably persist.

\subsection*{Inverting to Extract \( G \)}

Solving the above for \( G \):

\begin{equation}
    G = \frac{c^4}{4 F^{\text{max}}_{\text{gr}}}
\end{equation}

However, this only provides a dimensional relation. To embed this within vortex physics, we model the gravitational coupling as mediated by long-range strain interactions in the æther. These are modulated by:

- the vortex swirl velocity \( C_e \),
- the knot size \( r_c \),
- and Planck-scale pulse duration \( t_p \) or the Planck length \( L_p \).

\subsection*{Vortex-Strain Mediated Coupling}

From æther elasticity considerations, a derived form of \( G \) is:

\begin{equation}
    G = \frac{C_e c^3 t_p^2}{r_c m_e}
\end{equation}

This expression unites:

- \textbf{Æther swirl speed} \( C_e \),
- \textbf{Speed of light} \( c \),
- \textbf{Electron mass} \( m_e \),
- \textbf{Vortex radius} \( r_c \),
- and the \textbf{Planck time} \( t_p \), itself defined by:

\[
    t_p = \sqrt{\frac{\hbar G}{c^5}}
\]

Solving self-consistently, we see \( G \) depends on known parameters and the underlying æther properties.

\subsection*{Emergent Interpretation}

This relation is consistent with:

\[
    G = \frac{\alpha_g c^3 r_c}{C_e M_e}, \quad \text{or} \quad G = \frac{C_e c L_{\text{Planck}}^2}{r_c M_e}
\]

It highlights that \( G \) is not fundamental but arises from:

- Geometric knot scale \( r_c \),
- Ætheric propagation parameters \( C_e \),
- and internal energy scales tied to vortex strain dynamics.

\subsection*{Summary}

Thus, in the VAM:

\begin{equation}
    \boxed{G = \frac{C_e c^3 t_p^2}{r_c m_e} = \frac{c^4}{4 F^{\text{max}}_{\text{gr}}}}
\end{equation}

This connects gravity with æther tension and Planck-scale oscillations, explaining the smallness of \( G \) as the result of a weak elastic strain field propagating between vortex knots.


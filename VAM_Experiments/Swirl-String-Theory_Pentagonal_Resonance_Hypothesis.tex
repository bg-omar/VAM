%========================================
% Pentagonal Resonance Hypothesis (SST • Research Track)
%========================================
\section*{Appendix F: Pentagonal Resonance Hypothesis (SST • Research Track)}
\addcontentsline{toc}{section}{Appendix F: Pentagonal Resonance Hypothesis (SST • Research Track)}
\label{sec:pentagonal-resonance}

% --- safe macros (only define if absent) ---
\providecommand{\swirlarrow}{\!\!\scriptsize\boldsymbol{\circlearrowleft}}
\providecommand{\vswirl}{\mathbf{v}_{\swirlarrow}}
\providecommand{\xig}{\operatorname{asinh}\!\left(\tfrac{1}{2}\right)}  % golden hyperbolic base, so e^{\xig}=\varphi
\providecommand{\phig}{\exp(\xig)}                                      % equals \varphi numerically
\providecommand{\xigold}{\tfrac{3}{2}\,\xig}                            % golden rapidity
\providecommand{\Ce}{C_e}                                               % swirl-speed scale (Appendix B)
\providecommand{\rc}{r_c}                                               % core radius (Appendix B)

% --- theorem-like environments (requires amsthm in preamble) ---
\newtheorem{hypothesis}{Hypothesis}
\newtheorem{law}{Law}
\newtheorem{definition}{Definition}
\newtheorem{corollary}{Corollary}

\begin{hypothesis}[Pentagonal Resonance]
    A photon is absorbed by an electron when the photon's \emph{transient pentagonal swirl mode}
    geometrically and dynamically resonates with a \emph{pentagonal face mode}
    of the electron's \(I_h\)-symmetric shell field. This symmetry match enables energy and swirl transfer sufficient for absorption.
\end{hypothesis}

\subsection*{Kinematic and geometric setup}
\begin{definition}[Pentagonal swirl mode of a photon]
    Model the local photon swirl on an annular cross–section by an azimuthal Fourier decomposition
    \[
        r(\theta,t)=R_0+\sum_{m\ge 0} a_m(t)\cos\!\big(m\theta-\omega_m t+\varphi_m\big),
        \qquad m\in\mathbb{Z}_{\ge 0}.
    \]
    The \emph{pentagonal} mode is \(m=5\) with instantaneous amplitude \(a_5\) and frequency \(\omega_5\).
    Transient polygonization of vortex rings near boundaries (including \(m=5\)) is documented in high-fidelity studies \cite{orlandi1993vortex}.
\end{definition}

\begin{definition}[Electron shell field with pentagonal face projectors]
    Let \(\{\mathbf{n}_f\}_{f=1}^{12}\) be the unit normals of the dodecahedral faces.
    Define a face projector \(P_f^{(5)}\) that localizes the electron's surface response to the \(f\)-th pentagonal face
    and picks out its 5-fold tangential subspace (e.g., via an \(I_h\)-invariant spherical-harmonic construction).
    Denote the corresponding face mode velocity by \(\vswirl^{(e)}_f(\mathbf{x},t)\).
\end{definition}

\begin{law}[Resonance conditions (necessary)]
    Absorption is allowed only if the following hold simultaneously:
    \begin{align}
        \textbf{(S1) Symmetry:}&\quad m=5 \ \ \text{(or, more generally, } m\equiv 0 \!\!\!\pmod{5}\text{)}. \label{eq:S1}\\
        \textbf{(S2) Frequency match:}&\quad
        \big|\omega_5-\Omega_f\big| \le \epsilon_\omega\,\Omega_f, \quad 0<\epsilon_\omega\ll 1, \label{eq:S2}
    \end{align}
    where \(\Omega_f\) is the natural pentagonal face frequency. In SST a canonical ``golden layer'' frequency is normalized by the hyperbolic scale
    \[
        \Omega_g \;=\; e^{-\xig}\frac{\Ce}{\rc} \;=\; \phig^{-1}\,\frac{\Ce}{\rc},
        \qquad \text{and often } \Omega_f\approx \Omega_g.
    \]
    \begin{equation}
        \textbf{(S3) Orientation:}\qquad
        \max_{f}\ \mathbf{\hat k}\cdot\mathbf{n}_f \ \ge\ \cos\theta_{\rm tol}, \quad 0<\theta_{\rm tol}\ll 1,
        \label{eq:S3}
    \end{equation}
    with \(\mathbf{\hat k}\) the photon incidence direction. \hfill\qed
\end{law}

\subsection*{Coupling functional and resonance score}
Let \(\vswirl^{(\gamma,5)}(\mathbf{x},t)\) be the photon's \(m{=}5\) swirl field at the electron boundary
and \(\vswirl^{(e)}_f(\mathbf{x},t)\) the \(f\)-face response projected by \(P_f^{(5)}\).
Define the overlap (transition) amplitude
\begin{equation}
    \mathcal{M}_5(\omega_5,\mathbf{\hat k})
    \;=\;
    \sum_{f=1}^{12}
    \int_{S_f}
    \Big(\vswirl^{(\gamma,5)}\cdot \vswirl^{(e)}_f\Big)\,
    \chi_f(\mathbf{\hat k})\, dS,
    \qquad
    \chi_f(\mathbf{\hat k}) := \Theta\!\big(\mathbf{\hat k}\cdot\mathbf{n}_f - \cos\theta_{\rm tol}\big),
    \label{eq:M5}
\end{equation}
where \(S_f\) is the \(f\)-th pentagonal face patch and \(\Theta\) a Heaviside gate implementing \eqref{eq:S3}.
A dimensionless \emph{resonance score} is
\begin{equation}
    \mathcal{Q}_5
    \;=\;
    \frac{\big|\mathcal{M}_5\big|^2}{\Big(\sum_f \int_{S_f}\!\|\vswirl^{(\gamma,5)}\|^2 dS\Big)
    \Big(\sum_f \int_{S_f}\!\|\vswirl^{(e)}_f\|^2 dS\Big)}
    \cdot
    \exp\!\left[-\frac{(\omega_5-\Omega_f)^2}{2\Delta^2}\right],
    \label{eq:Q5}
\end{equation}
with \(\Delta/\Omega_f=\mathcal{O}(\epsilon_\omega)\) the linewidth. Absorption is triggered when \(\mathcal{Q}_5\ge \mathcal{Q}_{\rm thr}\).

\subsection*{Golden layer and selection rule}
Using the \emph{golden rapidity} scale \(\xi_g=\xigold\) with \(\tanh\xi_g=\phig^{-1}\),
the photon's pentagonal swirl speed at resonance is
\[
    \|\vswirl\|_g=\Ce\,\tanh\xi_g=\Ce\,\phig^{-1},\qquad
    \Omega_g=\frac{\|\vswirl\|_g}{\rc}=\phig^{-1}\frac{\Ce}{\rc}.
\]
This implements a hyperbolic-first use of the golden constant; the algebraic value \((1+\sqrt5)/2\) is an echo of this construction.

\subsection*{Predictions (falsifiable signatures)}
\begin{itemize}
    \item \textbf{12-lobe angular map:} absorption probability is maximal when \(\mathbf{\hat k}\) aligns with any \(\mathbf{n}_f\),
    producing a 12-peak pattern over the sphere (dodecahedral symmetry).
    \item \textbf{Pentagonal mode selectivity:} near-resonant absorption is suppressed for \(m\not\equiv 0\ (\mathrm{mod}\ 5)\).
    \item \textbf{Golden detuning:} resonance linewidth centered at \(\Omega_g\) (within \(\epsilon_\omega\)) if the face mode is tuned to the golden layer.
    \item \textbf{Amplitude scaling:} \(\mathcal{Q}_5\propto a_5^2\) in the weak-coupling limit (linear response).
\end{itemize}

\subsection*{Status and provenance}
This appendix is \emph{Research Track (non-canonical)} pending derivation from the SST Lagrangian
and experimental or ab-initio validation.
Transient pentagonal polygonization of vortex rings impacting walls (an analogue for the \(m{=}5\) channel)
is consistent with simulations \cite{orlandi1993vortex}; here it serves as mechanical inspiration rather than direct evidence for electronic structure.

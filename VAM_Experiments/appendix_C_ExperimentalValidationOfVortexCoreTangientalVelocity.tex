%! Author = Omar Iskandarani
%! Title = Testing Universality of the Proposed Tangential Vortex-Core Velocity \\ \( C_e = f \cdot \Delta x \): A Cross-Platform Precision Metrology Study
%! Date = June 17, 2025
%! Affiliation = Independent Researcher, Groningen, The Netherlands
%! License = © 2025 Omar Iskandarani. All rights reserved. This manuscript is made available for academic reading and citation only. No republication, redistribution, or derivative works are permitted without explicit written permission from the author. Contact: info@omariskandarani.com
%! ORCID = 0009-0006-1686-3961
%! DOI = 10.5281/zenodo.15684873

% === Metadata ===
%\newcommand{\appendixtitle}{\textbf{Testing Universality of the Proposed Tangential Vortex-Core Velocity \\ \( C_e = f \cdot \Delta x \): A Cross-Platform Precision Metrology Study}}
%\newcommand{\paperdoi}{10.5281/zenodo.15684873}
%
%
%\ifdefined\standalonechapter\else
%% Standalone mode
%\documentclass[11pt]{article}
%\usepackage{siunitx}
%\usepackage{textcomp}
%\AtBeginDocument{\RenewCommandCopy\qty\SI}
%% vamstyle.sty
\NeedsTeXFormat{LaTeX2e}
\ProvidesPackage{vamstyle}[2025/07/01 VAM unified style]

% === Constants ===
\newcommand{\hbarVal}{\ensuremath{1.054571817 \times 10^{-34}}} % J\cdot s
\newcommand{\meVal}{\ensuremath{9.10938356 \times 10^{-31}}} % kg
\newcommand{\cVal}{\ensuremath{2.99792458 \times 10^{8}}} % m/s
\newcommand{\alphaVal}{\ensuremath{1 / 137.035999084}} % unitless
\newcommand{\alphaGVal}{\ensuremath{1.75180000 \times 10^{-45}}} % unitless
\newcommand{\reVal}{\ensuremath{2.8179403227 \times 10^{-15}}} % m
\newcommand{\rcVal}{\ensuremath{1.40897017 \times 10^{-15}}} % m
\newcommand{\vacrho}{\ensuremath{5 \times 10^{-9}}} % kg/m^3
\newcommand{\LpVal}{\ensuremath{1.61625500 \times 10^{-35}}} % m
\newcommand{\MpVal}{\ensuremath{2.17643400 \times 10^{-8}}} % kg
\newcommand{\tpVal}{\ensuremath{5.39124700 \times 10^{-44}}} % s
\newcommand{\TpVal}{\ensuremath{1.41678400 \times 10^{32}}} % K
\newcommand{\qpVal}{\ensuremath{1.87554596 \times 10^{-18}}} % C
\newcommand{\EpVal}{\ensuremath{1.95600000 \times 10^{9}}} % J
\newcommand{\eVal}{\ensuremath{1.60217663 \times 10^{-19}}} % C

% === VAM/\ae ther Specific ===
\newcommand{\CeVal}{\ensuremath{1.09384563 \times 10^{6}}} % m/s
\newcommand{\FmaxVal}{\ensuremath{29.0535070}} % N
\newcommand{\FmaxGRVal}{\ensuremath{3.02563891 \times 10^{43}}} % N
\newcommand{\gammaVal}{\ensuremath{0.005901}} % unitless
\newcommand{\GVal}{\ensuremath{6.67430000 \times 10^{-11}}} % m^3/kg/s^2
\newcommand{\hVal}{\ensuremath{6.62607015 \times 10^{-34}}} % J Hz^-1

% === Electromagnetic ===
\newcommand{\muZeroVal}{\ensuremath{1.25663706 \times 10^{-6}}}
\newcommand{\epsilonZeroVal}{\ensuremath{8.85418782 \times 10^{-12}}}
\newcommand{\ZzeroVal}{\ensuremath{3.76730313 \times 10^{2}}}

% === Atomic & Thermodynamic ===
\newcommand{\RinfVal}{\ensuremath{1.09737316 \times 10^{7}}}
\newcommand{\aZeroVal}{\ensuremath{5.29177211 \times 10^{-11}}}
\newcommand{\MeVal}{\ensuremath{9.10938370 \times 10^{-31}}}
\newcommand{\MprotonVal}{\ensuremath{1.67262192 \times 10^{-27}}}
\newcommand{\MneutronVal}{\ensuremath{1.67492750 \times 10^{-27}}}
\newcommand{\kBVal}{\ensuremath{1.38064900 \times 10^{-23}}}
\newcommand{\RVal}{\ensuremath{8.31446262}}

% === Compton, Quantum, Radiation ===
\newcommand{\fCVal}{\ensuremath{1.23558996 \times 10^{20}}}
\newcommand{\OmegaCVal}{\ensuremath{7.76344071 \times 10^{20}}}
\newcommand{\lambdaCVal}{\ensuremath{2.42631024 \times 10^{-12}}}
\newcommand{\PhiZeroVal}{\ensuremath{2.06783385 \times 10^{-15}}}
\newcommand{\phiVal}{\ensuremath{1.61803399}}
\newcommand{\eVVal}{\ensuremath{1.60217663 \times 10^{-19}}}
\newcommand{\GFVal}{\ensuremath{1.16637870 \times 10^{-5}}}
\newcommand{\lambdaProtonVal}{\ensuremath{1.32140986 \times 10^{-15}}}
\newcommand{\ERinfVal}{\ensuremath{2.17987236 \times 10^{-18}}}
\newcommand{\fRinfVal}{\ensuremath{3.28984196 \times 10^{15}}}
\newcommand{\sigmaSBVal}{\ensuremath{5.67037442 \times 10^{-8}}}
\newcommand{\WienVal}{\ensuremath{2.89777196 \times 10^{-3}}}
\newcommand{\kEVal}{\ensuremath{8.98755179 \times 10^{9}}}

% === \ae ther Densities ===
\newcommand{\rhoMass}{\rho_\text{\ae}^{(\text{mass})}}
\newcommand{\rhoMassVal}{\ensuremath{3.89343583 \times 10^{18}}}
\newcommand{\rhoEnergy}{\rho_\text{\ae}^{(\text{energy})}}
\newcommand{\rhoEnergyVal}{\ensuremath{3.49924562 \times 10^{35}}}
\newcommand{\rhoFluid}{\rho_\text{\ae}^{(\text{fluid})}}
\newcommand{\rhoFluidVal}{\ensuremath{7.00000000 \times 10^{-7}}}

% === Draft Options ===
\newif\ifvamdraft
% \vamdrafttrue
\ifvamdraft
\RequirePackage{showframe}
\fi

% === Load Once ===
\RequirePackage{ifthen}
\newboolean{vamstyleloaded}
\ifthenelse{\boolean{vamstyleloaded}}{}{\setboolean{vamstyleloaded}{true}

% === Page ===
\RequirePackage[a4paper, margin=2.5cm]{geometry}

% === Fonts ===
\RequirePackage[T1]{fontenc}
\RequirePackage[utf8]{inputenc}
\RequirePackage[english]{babel}
\RequirePackage{textgreek}
\RequirePackage{mathpazo}
\RequirePackage[scaled=0.95]{inconsolata}
\RequirePackage{helvet}

% === Math ===
\RequirePackage{amsmath, amssymb, mathrsfs, physics}
\RequirePackage{siunitx}
\sisetup{per-mode=symbol}

% === Tables ===
\RequirePackage{graphicx, float, booktabs}
\RequirePackage{array, tabularx, multirow, makecell}
\newcolumntype{Y}{>{\centering\arraybackslash}X}
\newenvironment{tighttable}[1][]{\begin{table}[H]\centering\renewcommand{\arraystretch}{1.3}\begin{tabularx}{\textwidth}{#1}}{\end{tabularx}\end{table}}
\RequirePackage{etoolbox}
\newcommand{\fitbox}[2][\linewidth]{\makebox[#1]{\resizebox{#1}{!}{#2}}}

% === Graphics ===
\RequirePackage{tikz}
\usetikzlibrary{3d, calc, arrows.meta, positioning}
\RequirePackage{pgfplots}
\pgfplotsset{compat=1.18}
\RequirePackage{xcolor}

% === Code ===
\RequirePackage{listings}
\lstset{basicstyle=\ttfamily\footnotesize, breaklines=true}

% === Theorems ===
\newtheorem{theorem}{Theorem}[section]
\newtheorem{lemma}[theorem]{Lemma}

% === TOC ===
\RequirePackage{tocloft}
\setcounter{tocdepth}{2}
\renewcommand{\cftsecfont}{\bfseries}
\renewcommand{\cftsubsecfont}{\itshape}
\renewcommand{\cftsecleader}{\cftdotfill{.}}
\renewcommand{\contentsname}{\centering \Huge\textbf{Contents}}

% === Sections ===
\RequirePackage{sectsty}
\sectionfont{\Large\bfseries\sffamily}
\subsectionfont{\large\bfseries\sffamily}

% === Bibliography ===
\RequirePackage[numbers]{natbib}

% === Links ===
\RequirePackage{hyperref}
\hypersetup{
    colorlinks=true,
    linkcolor=blue,
    citecolor=blue,
    urlcolor=blue,
    pdftitle={The Vortex \AE ther Model},
    pdfauthor={Omar Iskandarani},
    pdfkeywords={vorticity, gravity, \ae ther, fluid dynamics, time dilation, VAM}
}
\urlstyle{same}
\RequirePackage{bookmark}

% === Misc ===
\RequirePackage[none]{hyphenat}
\sloppy
\RequirePackage{empheq}
\RequirePackage[most]{tcolorbox}
\newtcolorbox{eqbox}{colback=blue!5!white, colframe=blue!75!black, boxrule=0.6pt, arc=4pt, left=6pt, right=6pt, top=4pt, bottom=4pt}
\RequirePackage{titling}
\RequirePackage{amsfonts}
\RequirePackage{titlesec}
\RequirePackage{enumitem}

\AtBeginDocument{\RenewCommandCopy\qty\SI}

\pretitle{\begin{center}\LARGE\bfseries}
\posttitle{\par\end{center}\vskip 0.5em}
\preauthor{\begin{center}\large}
\postauthor{\end{center}}
\predate{\begin{center}\small}
\postdate{\end{center}}

\endinput
}
%% vamappendixsetup.sty

\newcommand{\titlepageOpen}{
  \begin{titlepage}
  \thispagestyle{empty}
  \centering
  {\Huge\bfseries \papertitle \par}
  \vspace{1cm}
  {\Large\itshape\textbf{Omar Iskandarani}\textsuperscript{\textbf{*}} \par}
  \vspace{0.5cm}
  {\large \today \par}
  \vspace{0.5cm}
}

% here comes abstract
\newcommand{\titlepageClose}{
  \vfill
  \null
  \begin{picture}(0,0)
  % Adjust position: (x,y) = (left, bottom)
  \put(-200,-40){  % Shift 75pt left, 40pt down
    \begin{minipage}[b]{0.7\textwidth}
    \footnotesize % One step bigger than \tiny
    \renewcommand{\arraystretch}{1.0}
    \noindent\rule{\textwidth}{0.4pt} \\[0.5em]  % ← horizontal line
    \textsuperscript{\textbf{*}}Independent Researcher, Groningen, The Netherlands \\
    Email: \texttt{info@omariskandarani.com} \\
    ORCID: \texttt{\href{https://orcid.org/0009-0006-1686-3961}{0009-0006-1686-3961}} \\
    DOI: \href{https://doi.org/\paperdoi}{\paperdoi} \\
    License: CC-BY 4.0 International \\
    \end{minipage}
  }
  \end{picture}
  \end{titlepage}
}
%\begin{document}
%
%    % === Title page ===
%    \titlepageOpen
%
%    \begin{abstract}
%        A recent theoretical claim suggests the existence of a universal tangential vortex-core velocity \( C_e \approx 1.0938 \times 10^6 \, \text{m/s} \), derived from the product of mechanical resonance frequency and displacement amplitude, \( C_e = f \cdot \Delta x \). This proposed constant, if universal, would have significant implications for fluid-based field theories and emergent causal structures. We propose a rigorous experimental program to test the validity and invariance of \( C_e \) across a wide range of resonators, frequencies, and materials. Using high-precision instrumentation and statistically grounded falsifiability metrics, this project aims to determine whether \( C_e \) reflects a true universal constant or an artifact of mechanical design.
%
%        %        This appendix presents a falsifiable experimental validation of the Vortex \AE{}ther Model (VAM), in which gravitational and temporal phenomena arise from angular momentum stored in knotted vortex structures embedded in a superfluid-like \ae{}ther. A central prediction of VAM is the existence of a universal tangential velocity $C_e$ at the boundary of such structures, expressible as the product $C_e = f \cdot \Delta x$ of resonance frequency $f$ and displacement amplitude $\Delta x$. We analyze five independent experiments using Pd-based surface acoustic wave (SAW), Lamb wave, and film bulk acoustic resonator (FBAR) devices spanning frequencies from 100\,MHz to 2.5\,GHz and amplitudes from 2.5\,nm to 11\,nm. In each case, we find convergence to $C_e \approx 1.09384563 \times 10^6$\,m/s, validating a key axiom of VAM. We provide a reproducible protocol suitable for university laboratories, emphasizing that deviation from this relation beyond 5\% would empirically falsify the VAM's time dilation mechanism. This represents a rare instance of a quantum-scale gravitational prediction subject to immediate experimental test.
%    \end{abstract}
%
%    \titlepageClose
%    \fi
%
%    \ifdefined\standalonechapter
%    \section{\appendixtitle}
%    \else
%    \fi
% ============= Begin of content ============
    \section{\textbf{Testing Universality of the Proposed Tangential Vortex-Core Velocity}}
        \section*{Specific Aims}
        \begin{enumerate}[label=\textbf{Aim \arabic*.}]
            \item Test the consistency of \( C_e \) across a broad class of mechanical oscillators (SAWs, FBARs, nanobeams, MEMS).
            \item Evaluate the dependence of \( C_e \) on geometry, material, temperature, and damping conditions.
            \item Statistically determine whether \( C_e \) qualifies as a fundamental constant or is system-dependent.
        \end{enumerate}

        \section*{Hypothesis and Falsifiability}
        \textbf{Null Hypothesis (H\textsubscript{0}):} \( C_e \) is not universal; it varies with material and design parameters.\\
        \textbf{Alternate Hypothesis (H\textsubscript{1}):} \( C_e \approx 1.0938 \times 10^6 \, \text{m/s} \) is universal within 1\% across all tested systems.\\

        \noindent\textbf{Falsifiability Condition:} If \( C_e \) varies by more than 5\% across independent trials and platforms, the universality claim is falsified.

        \section*{Experimental Design}

        \subsection*{Test Matrix}
        \begin{itemize}
            \item \textbf{Device Types:} SAW, FBAR, quartz tuning forks, MEMS, nanobeams.
            \item \textbf{Frequency Range:} 100 kHz -- 10 GHz.
            \item \textbf{Displacement:} \( \Delta x \) range: 0.1 nm -- 100 \textmu m.
            \item \textbf{Materials:} Silicon, GaN, AlN, ZnO, quartz.
            \item \textbf{Environments:} Vacuum, ambient air, inert gas; temperature from 77K to 500K.
        \end{itemize}

        \subsection*{Instrumentation}
        \begin{itemize}
            \item Laser Doppler Vibrometer (LDV) and high-speed interferometer for displacement measurements.
            \item RF Vector Network Analyzer for precise frequency characterization.
            \item Thermal chamber or cryostat for environmental control.
        \end{itemize}

        \section*{Data Analysis}
        Each device's \( f \cdot \Delta x \) product will be measured under controlled conditions. Statistical tools:
        \begin{itemize}
            \item One-way and multi-way ANOVA to test device-to-device variability.
            \item Monte Carlo simulations for uncertainty propagation.
            \item Hypothesis testing (p-value threshold: 0.01) for validating invariance.
        \end{itemize}

        \section*{Expected Deliverables}
        \begin{itemize}
            \item A comprehensive dataset of \( C_e \) measurements across >30 device types.
            \item Open-source analysis software for computing \( C_e \) and confidence intervals.
            \item Peer-reviewed article validating or falsifying the universality of \( C_e \).
        \end{itemize}

        \section*{Timeline}
        \begin{itemize}
            \item \textbf{Months 1–2:} Device selection and calibration.
            \item \textbf{Months 3–5:} Core data acquisition.
            \item \textbf{Months 6–7:} Data analysis and replication trials.
            \item \textbf{Months 8–9:} Final reporting, publication, and data release.
        \end{itemize}

\section*{Budget Estimate for Tangential Velocity Universality Experiment}

Total Estimated Budget: \$89,200 USD


\begin{table}
    \footnotesize
    \centering
    \begin{tabular}{lll}
        \toprule
        \textbf{Category} & \textbf{Item/Description} & \textbf{Cost (USD)} \\
        \midrule
        1. Measurement Equipment &  &  \\
        LDV system (e.g., Polytec) & Displacement sensitivity < 1 nm, bandwidth > 50 MHz & \$28,000 \\
        RF Vector Network Analyzer & Frequency characterization 100 kHz–10 GHz & \$15,000 \\
        Interferometric vibrometer (optional) & For phase measurements \& confirmation & \$9,000 \\
        Digital oscilloscope &  1+ GHz bandwidth, dual-channel & \$2,500 \\
        Temperature control chamber & Range: 77K–500K (basic cryostat + heater) & \$7,500 \\
        Subtotal (Equipment) &  & \$62,000 \\
        \bottomrule
    \end{tabular}
    \caption{}
    \label{tab:equipment}
\end{table}


\begin{table}
    \footnotesize
        \centering
        \begin{tabular}{lll}
            \toprule
            \textbf{2. Device Procurement \& Fabrication} & \textbf{Description} & \textbf{Cost (USD)} \\
            \midrule
            MEMS \& SAW samples (20–30) & Commercial-grade, varied geometries/materials & \$4,000 \\
            Custom nanobeam chips (optional) & For extreme frequency testing (outsourced fabrication) & \$6,000 \\
            Mounting hardware + PCBs & Holder kits, probes, adapters, PCB mounts & \$1,500 \\
            Subtotal (Devices) &  & \$11,500 \\
            \bottomrule
        \end{tabular}
        \caption{}
        \label{tab:devices}
\end{table}


\begin{table}
    \footnotesize
    \centering
    \begin{tabular}{lll}
        \toprule
        \textbf{3. Software \& Data Tools} & \textbf{Description} & \textbf{Cost (USD)} \\
        \midrule
        LabVIEW or Python integration tools & For automated scanning and LDV/VNA control & \$500 \\
        MATLAB or Python data license (optional) & For data fitting, Monte Carlo simulations & \$200 \\
        Cloud data repository or hosting & For open sharing and reproducibility & \$500 \\
        Subtotal (Software & Data) &  & \$1,200 \\
        \bottomrule
    \end{tabular}
    \caption{}
    \label{tab:software}
\end{table}



\begin{table}
        \centering
        \footnotesize
        \begin{tabular}{lll}
            \toprule
            \textbf{4. Personnel \& Collaboration} & \textbf{Description} & \textbf{Cost (USD)} \\
            \midrule
            Consulting lab technician (25 hours) & For calibration, setup, or mentorship & \$2,500 \\
            Independent data analyst (optional) & Statistical validation and report verification & \$2,000 \\
            Subtotal (Personnel) &  & \$4,500 \\
            \bottomrule
        \end{tabular}
        \caption{}
        \label{tab:personnel}
\end{table}


\begin{table}
    \centering
    \footnotesize
    \begin{tabular}{lll}
        \toprule
        \textbf{5. Miscellaneous} & \textbf{Description} & \textbf{Cost (USD)} \\
        \midrule
        Shipping, device damage/replacement & Spare parts, test reruns & \$2,000 \\
        Publication / conference fees & To present findings (APS, arXiv, or open access journal) & \$1,000 \\
        Subtotal (Misc.) &  & \$3,000 \\
        \bottomrule
    \end{tabular}
    \caption{}
    \label{tab:misc}
\end{table}

Total Project Budget: \~\$89,200 USD



%    \subsection*{Motivation}
%
%    In the Vortex \AE{}ther Model (VAM), all time dilation phenomena arise from rotational energy stored in knotted vortex structures, with local clock rates governed by their swirl speed. A central physical postulate is that the product of resonance frequency and displacement amplitude at the boundary of such structures yields a constant vortex tangential velocity:
%
%    This postulate emerges from the VAM interpretation of time as local angular rotation within an inviscid, incompressible superfluid \ae{}ther, where the rate of proper time is set by the tangential speed of vortex boundary flow.
%
%    \[
%        \boxed{C_e = f \cdot \Delta x \approx 1.09384563 \times 10^6 \, \text{m/s}}.
%    \]
%
%    This appendix evaluates the empirical status of this postulate. By reviewing five independent studies of Pd-based SAW and MEMS devices operating at MHz--GHz frequencies with nanometer-scale displacements, we show that this relation is repeatedly confirmed to high precision.
%
%    \subsection*{Structure of the Appendix}
%
%    We begin with a quantitative overview of five experimental reports, followed by a practical recipe for reproducing the measurement at any university-level lab. This test is not merely illustrative --- it constitutes a direct falsifiability criterion for the VAM gravitational mechanism.
%
%    \subsection*{Summary Table of Confirming Experiments}
%
%    \begin{tcolorbox}[colback=gray!10, colframe=black, title={Experimental Convergence to the Predicted Tangential Vortex Velocity $C_e = f \cdot \Delta x$}]
%
%        \centering
%        \footnotesize
%        \renewcommand{\arraystretch}{1.3}
%        \begin{tabular}{|r|c|c|c|c|}
%            \hline
%            \textbf{Study} & \textbf{Frequency $f$ (MHz)} & \textbf{Amplitude $\Delta x$ (nm)} & $C = f \cdot \Delta x$ (m/s) & $C \approx C_e$? \\
%            \hline
%            Laakso (2002)\cite{Laakso2002PdSAW}       & 98.0   & 11.16 & $1.0937 \times 10^6$   & \checkmark \\
%            Zhu et al. (2004)\cite{Zhu2004PdSAW}      & 98.5   & 11.10 & $1.0934 \times 10^6$   & \checkmark \\
%            Chen et al. (2017)\cite{Chen2017PdNiSAW}  & 108.5  & 10.08 & $1.0938 \times 10^6$   & \checkmark \\
%            Noual et al. (2020)\cite{Noual2020PdLWR}  & 100.0  & 11.00 & $1.1000 \times 10^6$   & \checkmark \\
%            \hline
%        \end{tabular}
%
%        \medskip
%
%        These four independent studies confirm the VAM-predicted relation: $C = f \cdot \Delta x \approx C_e = 1.09384563 \times 10^6 \, \mathrm{m/s}$. This strongly supports the interpretation of $C_e$ as the tangential causal limit of knotted vortex structures in the \ae{}ther.
%
%    \end{tcolorbox}
%
%    \subsection*{How to Reproduce the Experiment}
%
%    \textbf{Required Components:}
%    \begin{itemize}
%        \item \textbf{Substrate:} Quartz, LiNbO$_3$, or AlN wafer with interdigitated transducers (IDTs)
%        \item \textbf{Thin film:} Palladium or Pd-alloy (40--150\,nm)
%        \item \textbf{Oscillator:} 20--500\,MHz signal generator
%        \item \textbf{Amplifier:} RF amplifier (5--20\,dBm)
%        \item \textbf{Measurement:} Laser Doppler vibrometer or Michelson interferometer
%    \end{itemize}
%
%    \textbf{Procedure:}
%    \begin{enumerate}
%        \item Fabricate SAW or FBAR device with Pd film on piezoelectric substrate
%        \item Excite the structure with a known frequency $f$
%        \item Measure peak surface displacement $\Delta x$ via optical interferometry
%        \item Compute $C = f \cdot \Delta x$
%        \item Compare to VAM-predicted $C_e \approx 1.09384563 \times 10^6$\,\text{m/s}
%    \end{enumerate}
%
%    \textbf{Calibration Notes:} Displacement amplitudes must be measured at the peak resonant mode under vacuum or inert-controlled conditions to avoid thermal damping effects. Laser interferometry sensitivity must be validated against a nanometric calibration grid to ensure displacement resolution better than 1\,nm.
%
%    \textbf{Falsification Criterion:} If for any operating point the relation
%    \[
%        C \ne C_e \quad \text{by more than} \quad 5\%
%    \]
%    holds across controlled parameters and devices, the VAM assumption of vortex-core tangential causality may be challenged.
%
%    \subsection*{Conclusion}
%    This experimental protocol offers a direct, falsifiable test of the VAM claim that all time dilation and inertial mass arise from vortex-induced angular velocities with a universal scale $C_e$. The current literature robustly supports this prediction within nanometric and megahertz-scale systems.
%
%    \paragraph{Discussion.} While general relativity models time dilation via spacetime curvature, VAM attributes it to circulation-induced angular lag within an absolute fluidic substrate. The repeated convergence to $C_e$ in distinct physical devices suggests this quantity may represent an underlying causal invariant analogous to the speed of light $c$. This invites deeper investigation into whether $C_e$ governs broader physical laws, including low-energy nuclear transitions or frame-dragging analogs. We emphasize that the reproducibility and falsifiability of this test position it as a benchmark for competing models of time and inertia.
%

% ============== End of content =============
%
%% === Bibliography (only for standalone) ===
%\ifdefined\standalonechapter\else
%\bibliographystyle{unsrt}
%\bibliography{../../references}
%\end{document}
%\fi
\documentclass[11pt]{article}
\usepackage{amsmath,amssymb,bm}
\usepackage[T1]{fontenc}
\usepackage{lmodern}
\usepackage{siunitx}
\usepackage{hyperref}
\title{Golden Rapidity and Tangential Velocity in VAM}
\author{}
\date{}

\begin{document}
\maketitle

\section*{Setup}
Let the golden ratio be
\begin{equation}
  \varphi \equiv \frac{1+\sqrt{5}}{2}, \qquad \varphi^2=\varphi+1.
\end{equation}
Define the \emph{golden rapidity}
\begin{equation}
  \xi_g \;\equiv\; \frac{3}{2}\ln\varphi.
\end{equation}
We use the standard hyperbolic functions (definitions in \cite{NISTDLMF}).

\section*{Identity: \(\tanh(\xi_g)=1/\varphi\)}
Using $\tanh y = \dfrac{e^{2y}-1}{e^{2y}+1}$, substitute $y=\xi_g$ to obtain
\begin{equation}
  \tanh(\xi_g) \;=\; \frac{e^{3\ln\varphi}-1}{e^{3\ln\varphi}+1}
  \;=\; \frac{\varphi^3-1}{\varphi^3+1}.
\end{equation}
From $\varphi^2=\varphi+1$ it follows $\varphi^3=\varphi(\varphi+1)=2\varphi+1$. Hence
\begin{equation}
  \tanh(\xi_g) \;=\; \frac{(2\varphi+1)-1}{(2\varphi+1)+1}
  \;=\; \frac{2\varphi}{2(\varphi+1)}
  \;=\; \frac{\varphi}{\varphi+1}
  \;=\; \frac{\varphi}{\varphi^2}
  \;=\; \frac{1}{\varphi}.
\end{equation}
Therefore
\begin{equation}
  \boxed{\;\tanh\!\big(\tfrac{3}{2}\ln\varphi\big)=\frac{1}{\varphi}\;}
  \qquad\Longleftrightarrow\qquad
  \boxed{\;\coth\!\big(\tfrac{3}{2}\ln\varphi\big)=\varphi\; }.
\end{equation}

\section*{VAM Mapping to Tangential Velocity}
In a rapidity parametrization, the dimensionless speed is
\begin{equation}
  \beta \;\equiv\; \frac{v}{C_e} \;=\; \tanh \xi.
\end{equation}
Setting $\xi=\xi_g$ gives the \emph{golden} tangential fraction
\begin{equation}
  \beta_g \;=\; \tanh(\xi_g) \;=\; \frac{1}{\varphi},
\end{equation}
and thus a characteristic tangential velocity and swirl frequency
\begin{equation}
  v_g \;=\; \frac{C_e}{\varphi}, \qquad
  \Omega_g \;=\; \frac{v_g}{r_c} \;=\; \frac{1}{\varphi}\,\frac{C_e}{r_c}.
\end{equation}
Both are dimensionally consistent: $v_g$ has units of \(\si{m/s}\) and $\Omega_g$ of \(\si{s^{-1}}\).

\section*{Numerical Validation (User Constants)}
Using $C_e=\SI{1093845.63}{m/s}$ and $r_c=\SI{1.40897017e-15}{m}$,
\begin{align}
  \varphi &\approx 1.618033988749895,\\
  \xi_g &= \tfrac{3}{2}\ln\varphi \approx 0.721817737589405,\\
  \beta_g &= \tanh\xi_g \approx 0.618033988749895 \;=\; \frac{1}{\varphi},\\
  v_g &= \frac{C_e}{\varphi} \approx \SI{6.760337777855416e5}{m/s},\\
  \Omega &= \frac{C_e}{r_c} \approx \SI{7.763440655383073e20}{s^{-1}},\\
  \Omega_g &= \frac{\Omega}{\varphi} \approx \SI{4.798070194669498e20}{s^{-1}}.
\end{align}

\paragraph{Consistency checks.}
Since $\beta_g=1/\varphi$, we have $v_g=C_e/\varphi$ and $\Omega_g=\Omega/\varphi$ exactly, up to machine precision in floating-point arithmetic.

\section*{Discussion}
This construction supplies a natural, dimensionless benchmark ($1/\varphi$) for tangential speeds in VAM. If swirl states quantize in hyperbolic angle $\xi$, the golden rapidity $\xi_g$ defines a preferred scaling layer where tangential velocity and core swirl frequency are reduced by a factor $\varphi$ relative to their maxima, potentially useful for defining stable vortex\hyp{}knot operating points or resonance bands in the spectrum.

\bibliographystyle{unsrt}
\bibliography{refs}
\end{document}
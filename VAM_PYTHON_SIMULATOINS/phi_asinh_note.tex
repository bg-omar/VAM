\documentclass[11pt]{article}
\usepackage{amsmath, amssymb, bm}
\usepackage[T1]{fontenc}
\usepackage{lmodern}
\usepackage{hyperref}
\title{A Hyperbolic Identity for the Golden Ratio}
\author{}
\date{}
\begin{document}
\maketitle

Let the golden ratio be
\begin{equation}
  \varphi \equiv \frac{1+\sqrt{5}}{2}.
\end{equation}
Recall the definition of the inverse hyperbolic sine \cite{NISTDLMF}:
\begin{equation}
  \operatorname{asinh}(x) = \ln\!\Big(x + \sqrt{x^2 + 1}\Big).
  \label{eq:asinh_def}
\end{equation}
Substituting $x=\tfrac{1}{2}$ into \eqref{eq:asinh_def} gives
\begin{align}
  \operatorname{asinh}\!\left(\tfrac{1}{2}\right)
  &= \ln\!\left(\tfrac{1}{2} + \sqrt{\tfrac{1}{4} + 1}\right) \\
  &= \ln\!\left(\tfrac{1 + \sqrt{5}}{2}\right) \\
  &= \ln \varphi.
\end{align}
Exponentiating both sides yields the clean identity
\begin{equation}
  \boxed{\;\varphi = \exp\!\left(\operatorname{asinh}\!\left(\tfrac{1}{2}\right)\right)\; }.
\end{equation}

\paragraph{Numerical check.}
Using double precision,
\(
  \varphi \approx 1.618033988749895
\)
and
\(
  \exp(\operatorname{asinh}(1/2)) \approx 1.618033988749895
\),
matching to machine precision.

\bibliographystyle{unsrt}
\bibliography{refs}
\end{document}
%! Author = Omar Iskandarani
%! Title = Photon as a Topological Vortex Ring: Torsion and the Geometry of Light in the Æther
%! Date = 25-07-2025
%! Affiliation = Independent Researcher, Groningen, The Netherlands
%! License = © 2025 Omar Iskandarani. All rights reserved. This manuscript is made available for academic reading and citation only. No republication, redistribution, or derivative works are permitted without explicit written permission from the author. Contact: info@omariskandarani.com
%! ORCID = 0009-0006-1686-3961
%! DOI = 10.5281/zenodo.16419255

% === Metadata ===
\newcommand{\papertitle}{Photon as a Topological Vortex Ring: \\ Torsion and the Geometry of Light in the Æther}
\newcommand{\paperdoi}{10.5281/zenodo.16419255}

\documentclass[twocolumn,aps,pre,floatfix,nofootinbib]{revtex4-2}
\usepackage{amsmath, amssymb}
\usepackage{graphicx}
\usepackage{float}
\usepackage{booktabs}
\usepackage{xcolor}
\usepackage{tcolorbox}
\usepackage{hyperref}
\usepackage{enumitem}
\usepackage{physics}
\usepackage{caption}
\usepackage{bm}
\usepackage{tikz}
\usepackage{pgfplots}
\usepackage{lmodern}
\usepackage{amsmath,amssymb,amsfonts}
\usepackage{mathtools}
\usetikzlibrary{knots,intersections,decorations.pathreplacing}
\usetikzlibrary{3d, calc, arrows.meta, positioning}
\usepackage{pgfmath}
\usetikzlibrary{decorations.pathmorphing}
\pgfplotsset{compat=1.18} % or version you have
\usepackage{titlesec}
\usepackage{ulem}
\usepackage[utf8]{inputenc}
\usepackage[T1]{fontenc}
\renewcommand{\grqq}{``}
\usepackage{subfiles}
\usepackage{ragged2e}
\usepackage{amsmath, amssymb, bm}

\begin{document}
    \title{\papertitle}
    \author{Omar Iskandarani}
    \thanks{Independent Researcher, Groningen, The Netherlands\\
    info@omariskandarani.com \\
    ORCID: \href{https://orcid.org/0009-0006-1686-3961}{0009-0006-1686-3961} \\
    DOI: \href{https://doi.org/\paperdoi}{\paperdoi}}
    \date{\today}




    \begin{abstract}
        \vspace*{-0.5em}
        \section*{\centering Abstract}
        \vspace*{-1em}
    We propose a reformulation of the photon as a quantized, massless vortex ring embedded in an incompressible, inviscid superfluid æther. Using Cartan's geometric structure equations, we draw a formal correspondence between torsional and curvature defects in condensed matter (dislocations and disclinations) and the vortex-core features of ætheric flow. The Burgers and Frank vectors of defect theory find natural analogs in swirl density and angular vorticity. We show how the photon’s Lagrangian, Hamiltonian, and Jacobian structure follow directly from vortex ring energetics in this medium. By coupling this with a Biot–Savart swirl framework and time dilation due to swirl velocity, we reproduce the null proper time of the photon from first principles. This unification provides a geometric and fluid-mechanical basis for the photon's quantized behavior and opens the door for a deeper topological reinterpretation of gauge bosons in the Standard Model.
    \end{abstract}
    \maketitle



    \section*{Æther Revisited: From Historical Medium to Vorticity Field}

The concept of \textit{æther} traditionally referred to an all-pervasive medium, necessary for wave propagation. In the late nineteenth century Kelvin and Tait already proposed to model matter as nodal vorticity structures in an ideal fluid~\cite{thomson1867treatise}. After the null results of the Michelson--Morley experiment and the rise of Einstein's relativity, the æther concept disappeared from mainstream physics, replaced by curved spacetime. Recently, however, the idea has subtly returned in analogous gravitational theories, in which superfluid media are used to mimic relativistic effects~\cite{barcelo2011analogue,volovik2009universe}.

The \textit{Vortex Æther Model} (VAM) explicitly reintroduces the æther as a topologically structured, inviscid superfluid medium, in which gravity and time dilation do not arise from geometric curvature but from rotation-induced pressure gradients and vorticity fields. The dynamics of space and matter are determined by vortex nodes and conservation of circulation.

\subsection*{Postulates of the Vortex Æther Model}

\begin{table}[h!]
    \centering
    \begin{tabular}{rl}
        \midrule
        \hline
        \textbf{1. Continuous Space} & Space is Euclidean, incompressible and inviscid. \\
        \textbf{2. Knotted Particles} & Matter consists of topologically stable vortex nodes. \\
        \textbf{3. Vorticity} & The vortex circulation is conserved and quantized. \\
        \textbf{4. Absolute Time} & Time flows uniformly throughout the æther. \\
        \textbf{5. Local Time} & Time is locally slower due to pressure and vorticity gradients. \\
        \textbf{6. Gravity} & Emerges from vorticity-induced pressure gradients. \\
        \hline
        \bottomrule
    \end{tabular}
    \caption{Postulates of the Vortex Æther Model (VAM).}
    \label{tab:postulates}
\end{table}

The postulates replace spacetime curvature with structured rotational flows and thus form the foundation for emergent mass, time, inertia, and gravity.

\subsection*{Fundamental VAM constants}

\begin{table}[htbp]
    \centering
    \begin{tabular}{llc}
        \hline
        \toprule
        \textbf{Symbol} & \textbf{Name} & \textbf{Value (approx.)} \\
        \hline
        \midrule
        $C_e$ & Tangential eddy core velocity & $1.094 \times 10^6$ m/s \\
        $r_c$ & Vortex core radius & $1,409 \times 10^{-15}$ m \\
        $F_\text{max}$ & Maximum eddy force & $29.05$ N \\
        $\rho_\text{\ae}$ & Æther density & $3,893 \times 10^{18}$ kg/m$^3$ \\
        $\alpha$ & Fine structure constant ($2 C_e/c$) & $7,297 \times 10^{-3}$\\
        $G_\text{swirl}$ & VAM gravity constant & Derived from $C_e$, $r_c$\\
        $\kappa$ & Circulation quantum ($C_e r_c$) & $1.54 \times 10^{-9}$ m$^2$/s \\
        \hline
        \bottomrule
    \end{tabular}
    \caption{Fundamental VAM constants~\cite{vam2025field}.}
    \label{tab:VAMconstants}
\end{table}

\subsection*{Planck scale and topological mass}

Within VAM, the maximum vortex interaction force is derived explicitly from Planck-scale physics:
\begin{equation}
    F_\text{max} = \left(\frac{c^4}{4G}\right) \alpha \left(\frac{R_c}{L_p}\right)^{-2}
\end{equation}

where $\frac{c^4}{4G}$ is the Maximum Force in nature, the stress limit of the æther found from General Relativity.
The mass of elementary particles follows directly from topological vortex nodes, such as the trefoil node ($L_k=3$):
\begin{equation}
    M_e = \frac{8\pi \rho_\text{\ae} r_c^3}{C_e}\, L_k
\end{equation}

This explains mass and inertia from topological nodal structures in the æther.

\subsection*{Emergent quantum constants and Schrödinger equation}

Planck's constant $\hbar$ arises from vortex geometry and eddy force limit:
\begin{equation}
    \hbar = \sqrt{\frac{2M_e F_{\max} r_c^3}{5 \lambda_c C_e}}
\end{equation}

The Schrödinger equation follows directly from vortex dynamics:
\begin{equation}
    i \hbar \frac{\partial \psi}{\partial t} = -\frac{F_{\max} r_c^3}{5 \lambda_c C_e}\nabla^2 \psi + V\psi
\end{equation}


\subsection*{LENR and eddy quantum effects}

Created in VAM low-energy nuclear reactions (LENR) from resonant pressure reduction by vorticity-induced Bernoulli effects. Electromagnetic interactions and QED effects are reduced to vortex helicity and induced vector potentials.

\subsection*{Summary of GR and VAM observables}

\begin{table}[h!]
    \centering
    \begin{tabular}{lll}
        \toprule
        \textbf{Observable} & \textbf{GR expression} & \textbf{VAM expression} \\
        \midrule
        Time dilation & $\sqrt{1-\frac{2GM}{rc^2}}$ & $\sqrt{1-\frac{\Omega^2 r^2}{c^2}}$\\[0.5em]
        Redshift & $z=\left(1-\frac{2GM}{rc^2}\right)^{-1/2}-1$ & $z=\left(1-\frac{v_\phi^2}{c^2}\right)^{-1/2}-1$\\[0.5em]
        Frame-dragging & $\frac{2GJ}{c^2 r^3}$ & $\frac{2G\mu I\Omega}{c^2 r^3}$\\[0.5em]
        Light diffraction & $\frac{4GM}{Rc^2}$ & $\frac{4GM}{Rc^2}$\\
        \bottomrule
    \end{tabular}
    \caption{Comparison of GR and VAM observables.}
    \label{tab:equations}
\end{table}

    \section{Geometric Framework: Cartan's Structure Equations}\label{sec:framework}

Cartan's geometric formalism provides two fundamental structure equations on a manifold $\mathcal{M}$ equipped with coframe one-forms $\boldsymbol{\theta}^i$ and connection one-forms $\boldsymbol{\omega}^i{}_j$:
\begin{align}
    \text{Torsion 2-form:}\quad & \mathbf{T}^i = d\boldsymbol{\theta}^i + \boldsymbol{\omega}^i{}_j \wedge \boldsymbol{\theta}^j \\
    \text{Curvature 2-form:}\quad & \mathbf{R}^i{}_j = d\boldsymbol{\omega}^i{}_j + \boldsymbol{\omega}^i{}_k \wedge \boldsymbol{\omega}^k{}_j
\end{align}

In the Weitzenböck connection ($\boldsymbol{\omega}^i{}_j = 0$), all geometric deformation arises from torsion: $\mathbf{T}^i = d\boldsymbol{\theta}^i$, and $\mathbf{R}^i{}_j = 0$. Conversely, in the Levi-Civita connection (torsion-free), all deformation is encoded in curvature.

\section{Æther Interpretation in VAM}

In VAM, we associate:
\begin{itemize}
    \item $\boldsymbol{\theta}^i$: Local \ae ther displacement one-forms
    \item $\boldsymbol{\omega}^i{}_j$: Angular velocity of swirl (local rotational twist)
    \item $\mathbf{T}^i$: Core-induced torsion $\Rightarrow$ vortex dislocation (line defect)
    \item $\mathbf{R}^i{}_j$: Swirl curvature $\Rightarrow$ vortex disclination (rotational defect)
\end{itemize}

\section{Edge Dislocation in VAM as Torsion Source}

Consider a single vortex line (edge dislocation) along the $z$-axis with Burgers vector $\vec{b} = b\hat{x}$. The coframe is:
\begin{equation}
    \boldsymbol{\theta}^1 = dx + \frac{b}{2\pi} d\theta, \quad \boldsymbol{\theta}^2 = dy, \quad \boldsymbol{\theta}^3 = dz
\end{equation}

Using $d(d\theta) = 2\pi \delta(x)\delta(y) dx \wedge dy$, we compute:
\begin{align}
    \mathbf{T}^1 &= d\boldsymbol{\theta}^1 = b \delta(x) \delta(y) dx \wedge dy \\
    \mathbf{T}^2 &= 0, \quad \mathbf{T}^3 = 0
\end{align}

The dual vortex density becomes:
\begin{equation}
    \alpha^1 = *\mathbf{T}^1 = b \delta(x)\delta(y) \, dz
\end{equation}

\section{Equivalence to Wedge Disclination Dipole}

Following~\cite{kobayashi2025}, we reinterpret the same geometry using the Levi-Civita connection:
\begin{equation}
    \mathbf{R}^1{}_2 = \phi [\delta(x - L) - \delta(x + L)] \delta(y) dx \wedge dy
\end{equation}
where $\phi = b \rho$ encodes the Frank vector.

\begin{equation}
    \boxed{\text{Edge Dislocation} \equiv \text{Dipole of Wedge Disclinations}}
\end{equation}

    \section{Biot--Savart Swirl Integral in \ae ther}\label{sec:biot-savart}

        Using the Biot--Savart analogy:
        \begin{equation}
            \chi^i(\vec{x}) = \frac{1}{4\pi} \int \frac{\alpha^i(\vec{\xi}) \times (\vec{x} - \vec{\xi})}{|\vec{x} - \vec{\xi}|^3} \, d^3\xi
        \end{equation}

        The swirl potential $\chi^i$ defines the \ae theric coframe:
        \begin{equation}
            \boldsymbol{\theta}^i = dx^i + \chi^i
        \end{equation}

\section{Photon as a Vortex Ring}

        The photon is modeled as a massless, quantized vortex ring with tangential velocity $C_e$ and core radius $r_c$, moving through the \ae ther. Its circulation is $\Gamma = 2\pi r_c C_e$.

        \subsection{Lagrangian}
        \begin{equation}
            \mathcal{L}_\gamma = \pi^2 r_c^2 C_e^2 \rho_\text{\ae}^{(\text{energy})} R_\gamma \left( \ln\left( \frac{8R_\gamma}{r_c} \right) - \frac{7}{4} \right)
        \end{equation}

        \subsection{Hamiltonian}
        \begin{equation}
            \mathcal{H}_\gamma = \frac{P^2}{2M_{\text{ring}}} + \mathcal{L}_\gamma, \quad M_{\text{ring}} = \rho_\text{\ae}^{(\text{mass})} \cdot V_{\text{ring}}
        \end{equation}

        \subsection{Jacobian}
        \begin{align}
            \vec{X}(\phi,\theta) =
            \begin{pmatrix}
            (R + a\cos\theta)\cos\phi \\
            (R + a\cos\theta)\sin\phi \\
            a\sin\theta
            \end{pmatrix}, \\
            J = a(R + a\cos\theta)
        \end{align}

\section{Toward Interactions and Nonabelian Vortex Structures}\label{sec:nonabelian}

        The current VAM photon model, expressed as a topologically stable vortex ring, captures the dynamics of an Abelian \( U(1) \) gauge boson in a flat æther background. Extending this framework to model interactions and nonabelian gauge theories (e.g., \( SU(2) \), \( SU(3) \)) requires both multi-ring coupling and the inclusion of chirality and knot topology.

        \subsection{Quantization from Circulation Eigenstates}

            In VAM, energy quantization arises naturally from discrete circulation states:
            \[
                \Gamma_n = n \cdot 2\pi r_c C_e, \quad n \in \mathbb{Z}^+
            \]
            These correspond to topologically distinct vortex rings or knot classes, and enforce quantized angular momentum and energy via:
            \[
                E_n = \frac{1}{2} \rho_\text{\ae}^{(\text{energy})} \Gamma_n^2 / (2\pi r_c)
            \]
            Spin-1 arises from circulation axis orientation, with helicity embedded in the swirl chirality (left/right-handedness).

        \subsection{Interacting Gauge Bosons as Linked Vortices}

            To model bosonic interactions, we construct link and knot configurations:
            \begin{itemize}
                \item \textbf{Photon self-interaction}: modeled by ring–ring scattering or reconnection events, possibly mediated by localized æther pressure spikes.
                \item \textbf{Nonabelian fields}: modeled by chirally braided vortex tubes with internal twist degrees of freedom. These generate holonomy-like transformations when transported around each other.
            \end{itemize}

            Gauge field structure then emerges from circulation algebra:
            \[
                [\Gamma^a, \Gamma^b] = i f^{abc} \Gamma^c
            \]
            with \( f^{abc} \) determined by the braid/topology algebra. For example, a Hopf link and Borromean triplet encode noncommutative loop operations analogous to \( SU(2) \) and \( SU(3) \) structure constants.

        \subsection{Geometric Lagrangian for Nonabelian Swirl Fields}

            A generalized VAM Lagrangian may be written as:
            \[
                \mathcal{L}_{\text{gauge}} = -\frac{1}{4} \rho_\text{\ae}^{(\text{fluid})} \operatorname{Tr}(F_{ij} F^{ij}) + \lambda \, \varepsilon^{ijk} \operatorname{Tr}\left(A_i \partial_j A_k + \frac{2}{3} A_i A_j A_k \right)
            \]
            where \( A_i \) is the æther swirl potential and \( F_{ij} = \partial_i A_j - \partial_j A_i + [A_i, A_j] \). The second term resembles a fluid Chern–Simons action and accounts for helicity conservation in knotted æther flows.

        \subsection{Quantization Pathway}

            Quantization proceeds via:
            \begin{enumerate}
                \item Topological quantization: only allowed vortex states correspond to specific knot invariants or linking numbers.
                \item Canonical Hamiltonian formalism on swirl phase space: identify conjugate pairs \( (\chi^i, \omega^i) \), then apply canonical quantization rules.
                \item Path-integral over swirl configurations: gauge-invariant partition functions on knotted vortex histories.
            \end{enumerate}

        \subsection{Outlook: VAM Gauge Symmetry Breaking}

        VAM allows symmetry breaking via localized swirl density gradients:
        \[
            \partial_t \rho_\text{\ae}^{(\text{energy})} \neq 0 \quad \Rightarrow \quad \text{massive boson vortex rings}
        \]
        This may correspond to the emergence of massive gauge bosons like \( W^\pm \), \( Z^0 \) as unstable bound vortex structures.



\section{Time Dilation and Proper Time in Photons}

    Using swirl-induced time dilation~\cite{VAM-1}:
    \begin{equation}
        \frac{dt}{dt_\infty} = \sqrt{1 - \frac{|\vec{\omega}|^2}{c^2}}, \quad \vec{\omega} = \frac{C_e}{r_c} \gg c \Rightarrow dt \approx 0
    \end{equation}

    \section{Swirl Inheritance from Atomic Excitation}\label{sec:swirl-inheritance}

        In the Vortex \AE ther Model (VAM), photons emerge as topologically stable vortex rings whose translational motion arises from internal swirl dynamics. Crucially, the tangential swirl velocity of the photon vortex ring is not arbitrary—it is \emph{inherited} from the local swirl velocity of the parent atom at the moment of excitation and de-excitation.

        Let an atom possess internal vortex knots representing stable electronic states. Upon excitation, these knotted states become perturbed, increasing local swirl energy. When the atom returns to a lower energy configuration, a portion of this angular swirl is expelled into the \ae ther in the form of a toroidal vortex ring—the photon:
        \begin{equation}
            v_{\text{swirl}}^{(\text{photon})} = C_e = v_{\text{local swirl}}^{(\text{atom})}
        \end{equation}

        In a horn torus geometry, where the poloidal and toroidal radii are comparable, this tangential swirl naturally converts into forward propagation. The Biot–Savart law applied to the toroidal ring induces translation aligned with its curvature, and the photon thus moves at
        \begin{equation}
            v_{\text{propagation}} = v_{\text{swirl}} = C_e = c
        \end{equation}
        This matches the observed luminal speed of light.

        The forward motion of the photon is therefore not imposed externally, but rather emerges from the internal swirl mechanics of its source. The photon ring carries quantized circulation inherited from the parent atom:
        \begin{equation}
            \Gamma_\gamma = 2\pi r_c C_e = \oint \vec{v} \cdot d\vec{\ell}
        \end{equation}
        where $r_c$ is the vortex core radius and $C_e$ the universal tangential swirl velocity.

        This process links atomic angular momentum transitions to photon propagation in a purely fluid-dynamical framework. The speed of light becomes a consequence of conserved angular flow in the æther, providing a mechanical basis for luminal transmission:

        \begin{quote}
            \emph{Photon propagation speed $c$ =  tangential swirl velocity  $C_e$  of the emission source}
        \end{quote}

        This interpretation is consistent with and extends prior fluidic approaches to photonic behavior~\cite{iskandarani2025b, barut1990, berry2000}.

\subsection{Photon Absorption and Frame-Invariant Speed}

        A key implication of the Vortex \AE ther Model is that the photon, once emitted as a vortex ring, enters a regime of extreme internal swirl. Its tangential velocity satisfies:
        \begin{equation}
            C_e = \frac{d\ell}{dt} \gg c, \quad \Rightarrow \quad \frac{dt}{dt_\infty} = \sqrt{1 - \frac{C_e^2}{c^2}} \approx 0
        \end{equation}
        Thus, the photon experiences effectively no proper time; it is a \emph{null-time} excitation in the \ae ther. This timelessness guarantees that:

        \begin{quote}
            \emph{The photon is always perceived to travel at speed $c$ by any receiver, regardless of the emitter’s motion or inertial frame.}
        \end{quote}

        In VAM, this occurs not because of relativistic spacetime invariance, but because the photon's propagation velocity is inherited from its internal swirl, and its core time dilation suppresses any evolution in its own frame.

        \paragraph{Absorption by Atoms.} When a photon encounters a receiving atom, its vortex ring interacts with the local swirl topology of the electron orbital configuration. If the incoming circulation $\Gamma_\gamma$ and ring geometry match a resonant transition state in the atom, the ring collapses into the atomic vortex structure, transferring its angular momentum and restoring internal swirl coherence.

        \begin{equation}
            \text{Absorption Condition:} \quad \Gamma_\gamma = 2\pi r_c C_e \in \{ \Delta \Gamma_{\text{atom}} \}
        \end{equation}

        Because the photon itself does not experience internal time flow, its phase coherence and energy remain intact during transit. The receiving atom perceives the vortex ring at the moment of interaction as carrying the full energy $E = h f$, consistent with observed absorption spectra.

        \paragraph{Causal Implication.}
        This interpretation resolves a common paradox: how can a photon emitted from a distant star still exhibit perfect energy quantization billions of years later? In VAM, the answer is that the photon vortex ring never experiences time; it remains topologically and energetically frozen until it is reabsorbed, its swirl re-integrated into local atomic structure.

        \begin{quote}
            \emph{Photon vortex rings propagate at  $c$  and remain timeless due to maximal internal swirl.}
        \end{quote}

        This result parallels the null geodesic interpretation in general relativity, but is here derived from æther-based swirl dynamics and topological time suppression~\cite{iskandarani2025b, battye1998}.

\section{Insights from Classical Potential Flow Theory}
        The classical theory of incompressible, irrotational potential flow~\cite{caughey2008} offers foundational analogies for the Vortex \AE ther Model (VAM). In particular, we draw the following parallels:

        \begin{itemize}
            \item A vortex with circulation \(\Gamma\) corresponds to a stable, quantized excitation in VAM:
            \[
                \Gamma = 2\pi r_c C_e
            \]
            where \(r_c\) is the core radius and \(C_e\) the ætheric tangential velocity.

            \item The irrotationality condition (\(\nabla \times \vec{v} = 0\)) is violated only inside vortex cores, where topology imposes non-trivial helicity:
            \[
                \int \vec{\omega} \cdot d\vec{S} = \Gamma \neq 0
            \]

            \item Bernoulli’s law in the æther:
            \[
                \rho_\text{\ae}^{\text{(energy)}} + \frac{1}{2} \rho_\text{\ae}^{\text{(fluid)}} v^2 = \text{const}
                \quad \Rightarrow \quad dt \approx \sqrt{1 - \frac{v^2}{c^2}}\,dt_\infty
            \]
            confirms the swirl-induced time dilation mechanism.

            \item Dipole vortices (doublets) may serve as a first approximation for non-Abelian vortex bosons.
        \end{itemize}
        These correspondences affirm that potential flow theory can be repurposed to model relativistic quantum systems in a fluid-dynamical æther.

    
    \section{Correspondence with Quantum Electrodynamics (QED)}\label{sec:qed-correspondence}

Quantum electrodynamics (QED) successfully describes photons as massless spin-1 quanta of the electromagnetic field. The Vortex \ae ther Model (VAM), in contrast, models photons as quantized vortex rings in an incompressible superfluid \ae ther. Despite this geometric reformulation, VAM preserves all measurable features predicted by QED, while offering a unified physical interpretation based on swirl kinematics.

\subsection{Recovery of QED Observables}

\paragraph{(1) Energy–Frequency Relation:}
The photon vortex ring is characterized by circulation $\Gamma = 2\pi r_c C_e$, leading to internal swirl frequency $f$. The æther energy density $\rho_\text{\ae}^{(\text{energy})}$ gives the energy:
\begin{equation}
    E_\gamma = \frac{1}{2} \rho_\text{\ae}^{(\text{energy})} \Gamma^2 / V \sim h f
\end{equation}
recovering the Planck–Einstein relation through vortex energetics.

\paragraph{(2) Momentum:}
The translational motion of the ring, governed by Biot--Savart self-induction, gives:
\begin{equation}
    p = \frac{E_\gamma}{c} = \hbar k
\end{equation}
where $k$ is the spatial swirl wavevector.

\paragraph{(3) Spin and Polarization:}
The photon's handedness corresponds to its vortex swirl direction. The quantized angular momentum about the ring axis encodes spin-$\pm1$, reproducing circular polarization.

\paragraph{(4) Interference:}
Phase coherence of vortex cores produces constructive and destructive interference patterns, consistent with double-slit experiments. The æther swirl fields exhibit nodal structures modulating detection rates~\cite{iskandarani2025c}.

\paragraph{(5) Timelessness and Null Geodesics:}
From VAM’s time dilation law~\cite{iskandarani2025b}:
\begin{equation}
    \frac{dt}{dt_\infty} = \sqrt{1 - \frac{C_e^2}{c^2}} \to 0
\end{equation}
showing the photon is a null-time object. This matches the $ds^2 = 0$ result in relativistic geodesic motion.

\subsection{Predictions Beyond QED}

\paragraph{(i) Ætheric Drag in Dense Media:}
Photon vortex rings may interact with gradients in $\rho_\text{\ae}^{(\text{fluid})}$, causing subluminal dispersion or lensing anomalies, especially near massive bodies or in structured media.

\paragraph{(ii) Swirl–Polarization Coupling:}
Torsion fields in rotating environments may alter polarization via swirl alignment, opening new avenues for experimental test~\cite{fedi2023, vanputten2022}.

\paragraph{(iii) Nonlinear Photon Interactions:}
VAM permits vortex ring reconnection or knotting, suggesting rare vacuum photon–photon scattering~\cite{battye1998}, beyond QED’s perturbative Feynman diagrams.

\paragraph{(iv) Topological Mode Quantization:}
Only specific vortex geometries are topologically stable, implying discrete quantized photon modes that could lead to deviations in blackbody spectra under extreme confinement.

\begin{quote}
        \emph{VAM reproduces all QED observables while predicting new topological photon phenomena.}
\end{quote}
    
    \section{Experimental Implications and Observable Predictions}

The Vortex \ae ther Model (VAM) treats the photon as a massless vortex ring — more precisely, a vortex torus with quantized circulation and swirl frequency. The extreme swirl angular velocity \( \omega = C_e / r_c \) induces local time dilation:
\begin{equation}
    \frac{dt}{dt_\infty} = \sqrt{1 - \frac{|\vec{\omega}|^2}{c^2}} \Rightarrow dt \approx 0
\end{equation}

This leads to a concrete, testable prediction: \textbf{regions near the toroidal photon core experience phase delay} in structured light beams.

We list several proposed experiments:

\begin{enumerate}
    \item \textbf{Swirl-Induced Time Drift in Optical Vortex Beams:}\\
    High-order Laguerre–Gaussian beams carry orbital angular momentum (OAM) with phase singularities. According to VAM, beams with high OAM emulate photon-scale swirl, and should show \textbf{observable delay in arrival time} when passed through dispersive or rotating media. Phase-shifting interferometry may detect this.

    \item \textbf{Superfluid Photon Torus Analogues:}\\
    In a Bose–Einstein condensate (BEC), a toroidal flow trap can be engineered to mimic photon-like vortex torus structures. The tangential flow velocity can be tuned to match \( \omega \sim C_e / r_c \). Measuring \textbf{local chemical potential shifts or excitations} around the vortex core could reflect ætheric time dilation analogs.

    \item \textbf{Ring Interference and Æther Drag:}\\
    Two counter-propagating vortex ring structures (photon analogs) launched in an optical or water analog system should exhibit \textbf{nonlinear interference patterns} due to swirl coupling, differing from linear QED predictions.

    \item \textbf{Vortex Gravity Analog:}\\
    Using rotating liquid helium II or Fermi superfluids, generate vortex rings and measure local pressure gradient and time-delay effects (via trapped tracers or scattering). This would analogize VAM gravity as swirl-induced Bernoulli pressure drop.
\end{enumerate}

These experimental probes offer possible falsification or confirmation of key VAM predictions about photon geometry, time dilation, and æther-based interactions.

    \section{Conclusion and Outlook}

    We have presented a unified interpretation of the photon as a topologically stable vortex ring within the Vortex \ae ther Model (VAM). Using Cartan’s structure equations, we mapped torsion and curvature to fundamental vortex phenomena: dislocations (vortex cores) and disclinations (swirl distortions), respectively. The photon emerges naturally in this geometric fluid framework as a quantized ring-like structure with null proper time, encapsulating both energy propagation and rotational æther dynamics.


    By deriving its Lagrangian, Hamiltonian, and Jacobian from first principles—selecting the appropriate form of æther density for each physical context—we confirmed the internal coherence of the model. The photon's behavior becomes a consequence of vortex stability, quantized circulation, and localized energy-momentum flow in the æther.


    This approach offers several exciting implications:

    \begin{itemize}

    \item It provides a hydrodynamic foundation for gauge bosons, potentially extendable to W, Z, and gluons as knotted or linked vortex structures.

    \item The link between torsion and electrodynamics may enable a geometric unification of Maxwell’s equations with the topology of flow defects.

    \item VAM’s reinterpretation of light as fluid rotation suggests experimental pathways using superfluid analogs to probe photon structure, birefringence, or vacuum dispersion.

    \end{itemize}


    Future work will focus on generalizing this vortex-ring construction to incorporate spin-1/2 particles as twisted torus knots, analyzing photon-photon scattering via knot interactions, and embedding this framework into the topological fluid reinterpretation of the Standard Model.


    \medskip

    \noindent\textbf{Key Prediction:} The photon's zero-rest-mass arises not from symmetry breaking, but from a topological constraint enforcing null ætheric proper time: a perpetual rotation with velocity $C_e$ yields $ds^2 = 0$.


    \medskip

    \noindent This framework suggests a radical rethinking of particle ontology: not as pointlike fields in curved spacetime, but as structured flows in a flat, rotating æther.

    \appendix
    \section{Lagrangian, Hamiltonian, and Jacobian Formulation of a Vortex Ring in the Vortex \AE ther Model (VAM)}

    \subsection{Jacobian for a Vortex Ring Flow Field}
    Consider a vortex ring with core radius $r_c$, circulation $\Gamma$, and cylindrical symmetry. The velocity field $\vec{v}(\vec{x})$ of such a vortex in cylindrical coordinates $(r, \phi, z)$ can be modeled with azimuthal symmetry:
    \begin{equation}
        \vec{v}(r, z) = v_r(r, z)\hat{r} + v_z(r, z)\hat{z}
    \end{equation}
    with vorticity only in the $\phi$-direction: $\vec{\omega} = \omega_\phi(r, z)\hat{\phi}$.

    The Jacobian matrix of the flow $\vec{v}(\vec{x})$ is defined as:
    \begin{equation}
        J_{ij} = \frac{\partial v_i}{\partial x_j}
    \end{equation}
    This is used to compute the vorticity and strain-rate tensor. For incompressible flow $\div \vec{v} = 0$, so $\mathrm{Tr}(J) = 0$. The antisymmetric part yields:
    \begin{equation}
        \omega_i = \epsilon_{ijk} \partial_j v_k = (\nabla \times \vec{v})_i
    \end{equation}

    \subsection{Lagrangian Density of a Vortex Ring in VAM}
    From \cite{VAM-1,VAM-2,VAM-0}, the VAM Lagrangian is constructed from the kinetic swirl energy and vortex helicity. We define:
    \begin{equation}
        \mathcal{L} = \frac{1}{2} \rho_\text{\ae}^{(\text{fluid})} \vec{v}^{2} - \lambda \vec{v} \cdot (\nabla \times \vec{v})
    \end{equation}
    where $\lambda$ is a coupling constant with dimensions $[\text{time} \cdot \text{length}^2]$. In VAM, helicity represents charge-like conservation:\cite{VAM-5,VAM-6}

    Let us consider the ring with azimuthal velocity $v_\phi = \frac{\Gamma}{2\pi r}$ localized within the vortex core of radius $r_c$, and assume uniform vorticity inside:
    \begin{equation}
        \vec{\omega} = \nabla \times \vec{v} = \omega_0\hat{\phi}, \qquad \omega_0 = \frac{\Gamma}{\pi r_c^2}
    \end{equation}
    The energy per unit length is then:
    \begin{equation}
        \mathcal{E} = \int_0^{r_c} \left[ \frac{1}{2} \rho_\text{\ae}^{(\text{fluid})} \left( \frac{\Gamma}{2\pi r} \right)^2 - \lambda \frac{\Gamma}{2\pi r} \cdot \omega_0 \right] 2\pi r \, dr
    \end{equation}

    Evaluating the integral yields the effective Lagrangian per unit length of the ring:
    \begin{equation}
        \mathcal{L}_\text{ring} = \frac{\rho_\text{\ae}^{(\text{fluid})} \Gamma^2}{4\pi} \ln\left( \frac{r_c}{\delta} \right) - \lambda \Gamma \omega_0 r_c^2
    \end{equation}
    where $\delta$ is a vortex core cutoff.

    \subsection{Hamiltonian of a Vortex Ring in VAM}
    The canonical momentum density is:
    \begin{equation}
        \vec{\pi} = \pdv{\mathcal{L}}{\vec{v}} = \rho_\text{\ae}^{(\text{fluid})} \vec{v} - \lambda \vec{\omega}
    \end{equation}
    Then the Hamiltonian density is:
    \begin{equation}
        \mathcal{H} = \vec{\pi} \cdot \vec{v} - \mathcal{L} = \frac{1}{2} \rho_\text{\ae}^{(\text{fluid})} \vec{v}^2 + \lambda \vec{v} \cdot \vec{\omega}
    \end{equation}

    For the vortex ring, inserting the same field approximations as before:
    \begin{equation}
        \mathcal{H}_\text{ring} = \int_0^{r_c} \left[ \frac{1}{2} \rho_\text{\ae}^{(\text{fluid})} \left( \frac{\Gamma}{2\pi r} \right)^2 + \lambda \frac{\Gamma}{2\pi r} \cdot \omega_0 \right] 2\pi r \, dr
    \end{equation}

    Which yields:
    \begin{equation}
        \mathcal{H}_\text{ring} = \frac{\rho_\text{\ae}^{(\text{fluid})} \Gamma^2}{4\pi} \ln\left( \frac{r_c}{\delta} \right) + \lambda \Gamma \omega_0 r_c^2
    \end{equation}

    \subsection{Conclusion}
    The Jacobian characterizes the local deformation due to flow. The Lagrangian encodes energy and helicity structure of the vortex. The Hamiltonian recovers total energetic contribution of the vortex ring. In the VAM framework, this triad provides a fully vorticity-based alternative to point-particle or field-theoretic formulations.
% ============= END of content ============

    \bibliographystyle{unsrt}
    \bibliography{photon_vortex_refs}

\end{document}
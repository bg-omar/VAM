\section{Experimental Implications and Observable Predictions}

The Vortex \ae ther Model (VAM) treats the photon as a massless vortex ring — more precisely, a vortex torus with quantized circulation and swirl frequency. The extreme swirl angular velocity \( \omega = C_e / r_c \) induces local time dilation:
\begin{equation}
    \frac{dt}{dt_\infty} = \sqrt{1 - \frac{|\vec{\omega}|^2}{c^2}} \Rightarrow dt \approx 0
\end{equation}

This leads to a concrete, testable prediction: \textbf{regions near the toroidal photon core experience phase delay} in structured light beams.

We list several proposed experiments:

\begin{enumerate}
    \item \textbf{Swirl-Induced Time Drift in Optical Vortex Beams:}\\
    High-order Laguerre–Gaussian beams carry orbital angular momentum (OAM) with phase singularities. According to VAM, beams with high OAM emulate photon-scale swirl, and should show \textbf{observable delay in arrival time} when passed through dispersive or rotating media. Phase-shifting interferometry may detect this.

    \item \textbf{Superfluid Photon Torus Analogues:}\\
    In a Bose–Einstein condensate (BEC), a toroidal flow trap can be engineered to mimic photon-like vortex torus structures. The tangential flow velocity can be tuned to match \( \omega \sim C_e / r_c \). Measuring \textbf{local chemical potential shifts or excitations} around the vortex core could reflect ætheric time dilation analogs.

    \item \textbf{Ring Interference and Æther Drag:}\\
    Two counter-propagating vortex ring structures (photon analogs) launched in an optical or water analog system should exhibit \textbf{nonlinear interference patterns} due to swirl coupling, differing from linear QED predictions.

    \item \textbf{Vortex Gravity Analog:}\\
    Using rotating liquid helium II or Fermi superfluids, generate vortex rings and measure local pressure gradient and time-delay effects (via trapped tracers or scattering). This would analogize VAM gravity as swirl-induced Bernoulli pressure drop.
\end{enumerate}

These experimental probes offer possible falsification or confirmation of key VAM predictions about photon geometry, time dilation, and æther-based interactions.
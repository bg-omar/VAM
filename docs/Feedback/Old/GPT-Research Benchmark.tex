
%! Author = Omar Iskandarani
%! Date = 3/25/2025

\documentclass[aps,preprint,superscriptaddress]{revtex4}
\usepackage[none]{hyphenat}
\usepackage{array}
\usepackage{booktabs}
\usepackage{amsmath}
\usepackage{amssymb}
\usepackage{graphicx}
\usepackage{hyperref}
\usepackage{physics}

\begin{document}
\sloppy
\author{Omar Iskandarani}
\title{The Vortex Æther Model: Æther Vortex Field Model}
\date{\today}
\affiliation{Independent Researcher, Groningen, The Netherlands}
\thanks{ORCID: {https://orcid.org/0009-0006-1686-3961}{0009-0006-1686-3961}}
\email{info@omariskandarani.com}
\begin{document}


    \begin{abstract}
        research benchmark
    \end{abstract}
        \maketitle
\chapter*{Benchmarking the Vortex Æther Model vs General Relativity}
\section*{Introduction}
We compare the Vortex Æther Model (VAM) – a fluid-dynamic analogue of gravity – against General Relativity (GR) (and Special Relativity where applicable) across classical and modern tests. Five representative objects (electron, proton, Earth, Sun, neutron star) span quantum to astrophysical scales. For each key relativistic phenomenon, we present theoretical predictions from GR and VAM, compare to observed values, and note agreements or deviations. Where VAM fails to match reality, we propose physical or mathematical adjustments (e.g. redefining angular momentum, modifying the \grqq swirl\textquotedblright potential or æther density profile, or adding scaling factors) to improve its accuracy. All results are summarized in tables with GR result, VAM result, Observed value, and relative error.
\textit{(All units are in SI; time dilation is expressed as clock rate ratio dτ/dt or gravitational redshift $z$ where relevant. \grqq Relative error\textquotedblright is typically the difference between prediction and observation normalized to the observation or GR value.)}
        The abstract goes here.

\section*{1. Gravitational Time Dilation (Static Field)}Gravitational time dilation in GR (for a static spherical mass, Schwarzschild solution) is given by:
\begin{itemize}\item 
$d\tau/dt = \sqrt{1 - \frac{2GM}{rc^2}}$, the rate of proper time $\tau$ relative to coordinate time $t$ at radius $r$ from mass $M$. For weak fields, the fractional slowdown is $\approx \frac{GM}{rc^2}$.
\end{itemize}VAM Prediction: VAM attributes gravitational time dilation to the rotational kinetic energy of an æther vortex. In the VAM model, if an object's æther vortex has tangential speed $v_\phi$ at radius $r$, the local time flow is modulated as ({file://xn--file-2nnnwmwbptvdvnbeqndhbe%23:~:text=d-tl1a/}{GR_in_3d.pdf}) ({file://file-2nnnwmwbptvdvnbeqndhbe%23:~:text=9/}{GR_in_3d.pdf}):
\begin{itemize}\item 
$d\tau/dt = \sqrt{,1 - \frac{v_\phi^2}{c^2},}$,
\end{itemize}equivalent to SR time dilation for a local æther flow velocity $v_\phi$. VAM posits that for a massive object, $v_\phi$ at its surface is set such that $v_\phi^2 \approx 2GM/r$ (i.e. roughly the escape velocity squared), yielding the same first-order effect as GR's gravitational potential ({file://file-2nnnwmwbptvdvnbeqndhbe%23:~:text=9/}{GR_in_3d.pdf}) ({file://file-2nnnwmwbptvdvnbeqndhbe%23:~:text=this%20expression%20reflects%20the%20modification,perception%20caused%20by%20local%20rotational/}{GR_in_3d.pdf}). Thus, VAM can recover the GR result for static gravitational time dilation by appropriate choice of swirl speed. Table 1 compares GR and VAM for gravitational time dilation at the surface of each object:
Table 1: Gravitational Time Dilation (Surface Clock Rate)
\begin{table}
    \centering
    \begin{tabular}{lllll}
        \toprule
        \textbf{Object} & \textbf{GR: $d\tau/dt = \sqrt{1-2GM/(Rc^2)}$} & \textbf{VAM: $d\tau/dt = \sqrt{1-v_\phi^2/c^2}$ (tuned)} & \textbf{Observed Effect} & \textbf{Relative Error (VAM vs Obs)} \\
        \midrule
        Earth (R=6.37×10^6 m) & 0.9999999993 (Δ≈7×10^–10) ({https://www.einstein-online.info/en/spotlight/redshift_white_dwarfs/#:~:text=The%20gravitational%20redshift%20was%20first,the%20field%20of%20the%20sun}{Gravitational redshift and White Dwarf stars «  Einstein-Online}) & 0.9999999993 (assuming $v_\phi\approx 11.2$ km/s) & Clock gain of +45 µs/day at orbit (GPS) ({https://www.einstein-online.info/en/spotlight/redshift_white_dwarfs/#:~:text=The%20gravitational%20redshift%20was%20first,the%20field%20of%20the%20sun}{Gravitational redshift and White Dwarf stars «  Einstein-Online}) & ~0% (VAM tuned to match) \\
        Sun (R=6.96×10^8 m) & 0.9999979 (Δ≈2.1×10^–6) ({https://www.sciencedirect.com/science/article/abs/pii/S1384107614000190#:~:text=On%20the%20gravitational%20redshift%20,in%201908%E2%80%94is%20still%20an}{On the gravitational redshift - ScienceDirect.com}) & 0.9999979 (if $v_\phi\approx 618$ km/s) & Solar spectral line redshift ~2×10^–6 ({https://www.sciencedirect.com/science/article/abs/pii/S1384107614000190#:~:text=On%20the%20gravitational%20redshift%20,in%201908%E2%80%94is%20still%20an}{On the gravitational redshift - ScienceDirect.com}) & ~0% (within meas. error) \\
        Neutron Star (M≈1.4 M_⊙, R≈1×10^4 m) & 0.875 (Δ≈0.234) (strong field) & 0.875 (if $v_\phi\approx0.65c$) & X-ray spectral redshift z~0.3 expected (some observations) & ~0% (assumed match) \\
        Proton (m=1.67×10^–27 kg) & ~1 – 1×10^–27 (negligible) & ~1 (μ factor suppresses gravity at r<1 mm) & No measurable gravitational slowdown & N/A (both predict none) \\
        Electron (m=9.11×10^–31 kg) & ~1 – 1×10^–30 (negligible) & ~1 (suppressed by quantum scaling μ) & No measurable gravitational slowdown & N/A \\
        Effect (Earth) & GR Prediction & VAM Prediction & Observed Value & Error (VAM vs Obs) \\
        Frame-dragging (GP-B gyroscope) & $39.2$ mas/yr (north-south axis precession) ({https://arxiv.org/abs/1105.3456#:~:text=Analysis%20of%20the%20data%20from,9%20rad}{[1105.3456] Gravity Probe B: Final Results of a Space Experiment to Test General Relativity}) & $≈39$ mas/yr (μ=1, identical formula) ({file://file-2nnnwmwbptvdvnbeqndhbe%23:~:text=2,2gj/}{GR_in_3d.pdf}) ({file://file-2nnnwmwbptvdvnbeqndhbe%23:~:text=c2r3/}{GR_in_3d.pdf}) & $37.2 \pm 7.2$ mas/yr (GP-B) ({https://arxiv.org/abs/1105.3456#:~:text=Analysis%20of%20the%20data%20from,9%20rad}{[1105.3456] Gravity Probe B: Final Results of a Space Experiment to Test General Relativity}) & ~0% (within 1σ error) \\
        Frame-dragging (polar LAGEOS) & $~$31 mas/yr (node regression) & $~$31 mas/yr (same as GR) & $30 \pm 5$ mas/yr (measured) & ~0% (within error) \\
        Scenario & GR Redshift $z$ & VAM Redshift $z$ & Observed $z$ (or test) & Result \\
        Pound–Rebka (Earth) – 22.5 m tower & $2.5\times10^{-15}$ (for Δφ = 22.5 m g) ({https://en.wikipedia.org/wiki/Tests_of_general_relativity#:~:text=line%20width,first%20precision%20experiments%20testing%20general}{Tests of general relativity - Wikipedia}) & $2.5\times10^{-15}$ (æther flow equivalent) & $2.5\times10^{-15}$ ± 5% ({https://en.wikipedia.org/wiki/Tests_of_general_relativity#:~:text=line%20width,first%20precision%20experiments%20testing%20general}{Tests of general relativity - Wikipedia}) (gamma-ray) & Matched (0% error) \\
        Sun to infinity (surface) & $2.12\times10^{-6}$ ({https://www.sciencedirect.com/science/article/abs/pii/S1384107614000190#:~:text=On%20the%20gravitational%20redshift%20,in%201908%E2%80%94is%20still%20an}{On the gravitational redshift - ScienceDirect.com}) & $2.12\times10^{-6}$ (if $v_\phi\approx 617$ km/s) & ~$2.12\times10^{-6}$ (solar spectrum, expected) ({https://www.sciencedirect.com/science/article/abs/pii/S1384107614000190#:~:text=On%20the%20gravitational%20redshift%20,in%201908%E2%80%94is%20still%20an}{On the gravitational redshift - ScienceDirect.com}) & Matched (few % error due to solar Doppler) \\
        White Dwarf (Sirius B) – high $GM/R$ & $5.5\times10^{-5}$ (for Sirius B) & $5.5\times10^{-5}$ (if vortex tuned) & $4.8(±0.3)\times10^{-5}$ (observed spectral lines) ({https://www.einstein-online.info/en/spotlight/redshift_white_dwarfs/#:~:text=Consequently%2C%20the%20shift%20should%20be,image%20was%20taken%20with%20the}{Gravitational redshift and White Dwarf stars «  Einstein-Online}) & ~15% error (VAM can tune) \\
        Neutron Star (massive) & $0.3$ (30% freq. drop for $2GM/(Rc^2)\sim0.4$) & $0.3$ (if $v_\phi \sim0.7c$ at surface) & ~0.35 (possible X-ray line measure, uncertain) & ~0% (within uncertainty) \\
        Light Ray & Mass & GR Deflection & VAM Deflection & Observed Deflection & Error \\
        Star near Sun's limb (impact ~R_⊙) & $1.75″$ (arcsec) ([Testing General Relativity & Total Solar Eclipse 2017]({https://eclipse2017.nasa.gov/testing-general-relativity#:~:text=where%20for%20the%20sun%20we,9x108%20meters}{https://eclipse2017.nasa.gov/testing-general-relativity#:~:text=where%20for%20the%20sun%20we,9x108%20meters})) & $1.75″$ ([Testing General Relativity & Total Solar Eclipse 2017]({https://eclipse2017.nasa.gov/testing-general-relativity#:~:text=where%20for%20the%20sun%20we,9x108%20meters}{https://eclipse2017.nasa.gov/testing-general-relativity#:~:text=where%20for%20the%20sun%20we,9x108%20meters})) ({file://file-2nnnwmwbptvdvnbeqndhbe%23:~:text=given%20by:/}{GR_in_3d.pdf}) \\
        Light near Earth (impact ~R⊕) & $8.5\times10^{-6}$″ (microarcsec) & $8.5\times10^{-6}$″ & ~ Not measured (too small) & N/A \\
        Quasar by galaxy (strong lens) & GR lensing formulas (non-linear) & VAM: requires fluid simulation & Multiple images, matches GR lensing & (Unknown for VAM) \\
        Orbit (Central mass) & GR Extra Precession & VAM Precession & Observed Extra Precession & Agreement? \\
        Mercury around Sun & $42.98″/\text{century}$ ({https://math.ucr.edu/home/baez/physics/Relativity/GR/mercury_orbit.html#:~:text=Mercury%27s%20perihelion%20precession%3A%20%20,3%20arcseconds%2Fcentury}{GR and Mercury's Orbital Precession}) & $42.98″/\text{century}$ ({file://xn--file-2nnnwmwbptvdvnbeqndhbe%23:~:text=vam%20=%206gm-sm9aunx840m/}{GR_in_3d.pdf}) & $43.1±0.2″/\text{century}$ ({https://math.ucr.edu/home/baez/physics/Relativity/GR/mercury_orbit.html#:~:text=Mercury%27s%20perihelion%20precession%3A%20%20,3%20arcseconds%2Fcentury}{GR and Mercury's Orbital Precession}) & Yes (within 0.3%) \\
        Earth around Sun & $3.84″/\text{century}$ (calc.) & $3.84″/\text{century}$ & ~$3.84″$ (too small to measure accurately) & Yes (not directly measured) \\
        Binary Pulsar (PSR J0737) – periastron advance & $\sim 16.9°/\text{yr}$ (GR) & $16.9°/\text{yr}$ (VAM, by design) & $16.9°/\text{yr}$ observed (Double Pulsar) & Yes (0% error) \\
        Object & Newtonian/GR Potential at surface $Φ=-GM/R$ (J/kg) & VAM $Φ$ (from vortex) & Surface Gravity $g=GM/R^2$ & Observations (g) \\
        Earth & $-6.25×10^7$ J/kg (→ $v_\text{esc}=11.2$ km/s) & Matches (by construction) ({file://xn--file-2nnnwmwbptvdvnbeqndhbe%23:~:text=2,gm%20r-x034a/}{GR_in_3d.pdf}) & $9.81$ m/s² (observed) & $9.81$ m/s² (exactly) \\
        Sun & $-1.9×10^8$ J/kg (→ $v_\text{esc}=617$ km/s) & Matches (with appropriate γ) & $274$ m/s² (at surface) & ~274 m/s² (via helioseismology) \\
        Neutron Star & ~$-2×10^{13}$ J/kg (for 1.4 M⊙, R=12 km) & Would match if $v_\phi$ near c & ~$1.6×10^{12}$ m/s² (extreme) & Indirect (from orbits, X-ray bursts) \\
        System (Pulsar Binary) & GR $dP/dt$ (orbital period change) & VAM $dP/dt$ & Observed $dP/dt$ & VAM vs Obs Error \\
        PSR B1913+16 (Hulse–Taylor) & $-2.4025\times10^{-12}$ s/s (energy loss via GW) ({https://adsabs.harvard.edu/pdf/2005ASPC..328...25W#:~:text=predicted%20orbital%20period%20derivative%20due,to%20gravitational%20radiation%20computed%20from}{}) & ~ $0$ s/s (no built-in wave emission) & $-2.4056(51)\times10^{-12}$ s/s ({https://adsabs.harvard.edu/pdf/2005ASPC..328...25W#:~:text=quan%02tities%2C%20including%20the%20distance%20and,Hence%20P%CB%99%20b%2Ccorrected}{}) ({https://adsabs.harvard.edu/pdf/2005ASPC..328...25W#:~:text=b%2CGR%20%3D%201,and%20theoretical%20orbital%20decays%20are}{}) & ~100% (fails) \\
        PSR J0737–3039A/B (Double Pulsar) & $-1.252\times10^{-12}$ s/s (predicted) & ~ $0$ s/s & $-1.252(17)\times10^{-12}$ s/s (measured) & ~100% (fails) \\
        GW150914 (Binary BH merger) & 3 solar mass energy radiated in GWs & No GW (merger dynamics unclear) & Detected GWs, strain amplitude $~10^{-21}$ ({http://ui.adsabs.harvard.edu/abs/2010ApJ...722.1030W/abstract#:~:text=Timing%20Measurements%20of%20the%20Relativistic,radiation%20damping%20in%20general}{Timing Measurements of the Relativistic Binary Pulsar PSR B1913+16}) & Complete miss \\
        Precession Effect & GR Prediction (mas/yr) & VAM Prediction (mas/yr) & Observed (mas/yr) & VAM vs Obs \\
        Geodetic (de Sitter) & $6606.1$ mas/yr (forward in orbit plane) ({https://arxiv.org/abs/1105.3456#:~:text=Analysis%20of%20the%20data%20from,9%20rad}{[1105.3456] Gravity Probe B: Final Results of a Space Experiment to Test General Relativity}) & Not explicitly derived (possibly 0 without extension) & $6601.8 \pm 18.3$ mas/yr ({https://arxiv.org/abs/1105.3456#:~:text=Analysis%20of%20the%20data%20from,9%20rad}{[1105.3456] Gravity Probe B: Final Results of a Space Experiment to Test General Relativity}) & ~100% error if 0 \\
        Frame-Dragging & $39.2$ mas/yr (around Earth's spin axis) ({https://arxiv.org/abs/1105.3456#:~:text=Analysis%20of%20the%20data%20from,9%20rad}{[1105.3456] Gravity Probe B: Final Results of a Space Experiment to Test General Relativity}) & $≈39$ mas/yr (by matching formula) ({file://file-2nnnwmwbptvdvnbeqndhbe%23:~:text=2,2gj/}{GR_in_3d.pdf}) & $37.2 \pm 7.2$ mas/yr ({https://arxiv.org/abs/1105.3456#:~:text=Analysis%20of%20the%20data%20from,9%20rad}{[1105.3456] Gravity Probe B: Final Results of a Space Experiment to Test General Relativity}) & ~0% (within error) \\
        Phenomenon & GR Prediction & Formula & VAM Prediction & Formula & Observation (with ref) & Agreement? (Error) \\
        Gravitational Time Dilation (Static field) & $dτ/dt=\sqrt{1-2GM/(rc^2)}$ ({file://file-2nnnwmwbptvdvnbeqndhbe%23:~:text=2/}{GR_in_3d.pdf}) & $dτ/dt=\sqrt{1-Ω^2 r^2/c^2}$ (with $Ω r = v_φ$) ({file://xn--file-2nnnwmwbptvdvnbeqndhbe%23:~:text=%201%202r2-yx7a2275psla/}{GR_in_3d.pdf}) & GPS clocks: Δν/ν = $6.9×10^{-10}$ (Earth) ({https://www.einstein-online.info/en/spotlight/redshift_white_dwarfs/#:~:text=The%20gravitational%20redshift%20was%20first,the%20field%20of%20the%20sun}{Gravitational redshift and White Dwarf stars «  Einstein-Online}), Pound–Rebka: $2.5×10^{-15}$ ({https://en.wikipedia.org/wiki/Tests_of_general_relativity#:~:text=line%20width,first%20precision%20experiments%20testing%20general}{Tests of general relativity - Wikipedia}) (both match GR) & Yes (VAM tuned, 0% error) \\
        Velocity Time Dilation (SR) & $dτ/dt=\sqrt{1-v^2/c^2}$ & $dτ/dt=\sqrt{1-v^2/c^2}$ (same as GR/SR) ({file://xn--file-2nnnwmwbptvdvnbeqndhbe%23:~:text=d-tl1a/}{GR_in_3d.pdf}) & Particle accelerators, muon lifetime (time dilation confirmed to $10^{-8}$) & Yes (identical) \\
        Rotational (Kinetic) Time Dilation & (Included as mass-energy in GR implicitly) & $dτ/dt=(1+\tfrac{1}{2}\beta I Ω^2)^{-1}$ ({file://file-2nnnwmwbptvdvnbeqndhbe%23:~:text=/}{GR_in_3d.pdf}) (VAM heuristic) & No direct obs; fast pulsar (~716 Hz) slows time ~0.5% (GR via mass-energy; VAM via rotation) & In principle (if β set right) \\
        Gravitational Redshift & $z=(1-2GM/(rc^2))^{-1/2}-1$ ({file://file-2nnnwmwbptvdvnbeqndhbe%23:~:text=2/}{GR_in_3d.pdf}) & $z=(1-v_φ^2/c^2)^{-1/2}-1$ ({file://xn--file-2nnnwmwbptvdvnbeqndhbe%23:~:text=%201-py51a/}{GR_in_3d.pdf}) & Solar redshift $2.12×10^{-6}$ ({https://www.sciencedirect.com/science/article/abs/pii/S1384107614000190#:~:text=On%20the%20gravitational%20redshift%20,in%201908%E2%80%94is%20still%20an}{On the gravitational redshift - ScienceDirect.com}); Sirius B $5×10^{-5}$ ({https://www.einstein-online.info/en/spotlight/redshift_white_dwarfs/#:~:text=Consequently%2C%20the%20shift%20should%20be,image%20was%20taken%20with%20the}{Gravitational redshift and White Dwarf stars «  Einstein-Online}); both ~match GR & Yes (VAM=GR) \\
        Light Deflection (Sun) & $\delta = 4GM/(Rc^2) = 1.75″$ ([Testing General Relativity & Total Solar Eclipse 2017]({https://eclipse2017.nasa.gov/testing-general-relativity#:~:text=where%20for%20the%20sun%20we,9x108%20meters}{https://eclipse2017.nasa.gov/testing-general-relativity#:~:text=where%20for%20the%20sun%20we,9x108%20meters})) & $\delta = 4GM/(Rc^2)$ ({file://file-2nnnwmwbptvdvnbeqndhbe%23:~:text=given%20by:/}{GR_in_3d.pdf}) & $1.75″\pm0.07″$ (VLBI) ([Testing General Relativity \\
        Perihelion Precession (Mercury) & $Δϖ = 6πGM/[a(1-e^2)c^2] = 42.98″/$cent. ({https://math.ucr.edu/home/baez/physics/Relativity/GR/mercury_orbit.html#:~:text=Mercury%27s%20perihelion%20precession%3A%20%20,3%20arcseconds%2Fcentury}{GR and Mercury's Orbital Precession}) & \textit{Same as GR} ({file://xn--file-2nnnwmwbptvdvnbeqndhbe%23:~:text=vam%20=%206gm-sm9aunx840m/}{GR_in_3d.pdf}) & $43.1″/$cent. (observed) ({https://math.ucr.edu/home/baez/physics/Relativity/GR/mercury_orbit.html#:~:text=Mercury%27s%20perihelion%20precession%3A%20%20,3%20arcseconds%2Fcentury}{GR and Mercury's Orbital Precession}) & Yes (exact within error) \\
        Frame-Dragging (Earth LT) & $Ω_{LT}=2GJ/(c^2 r^3)$ ({file://file-2nnnwmwbptvdvnbeqndhbe%23:~:text=2,2gj/}{GR_in_3d.pdf}) (≈39 mas/yr) & $Ω_{drag}=\frac{4}{5}\frac{GMΩ}{c^2 r}$ ({file://xn--file-2nnnwmwbptvdvnbeqndhbe%23:~:text=vam%20drag%20-uz9a/}{GR_in_3d.pdf}) (gives same 39 mas/yr) & $37.2±7.2$ mas/yr (GP-B) ({https://arxiv.org/abs/1105.3456#:~:text=Analysis%20of%20the%20data%20from,9%20rad}{[1105.3456] Gravity Probe B: Final Results of a Space Experiment to Test General Relativity}) & Yes (~0%, within 1σ) \\
        Geodetic Precession (Earth) & $Ω_{geo} = \frac{3}{2}\frac{GM}{c^2 r^3} v$ (6606 mas/yr) & \textit{Not derived} (flat space → 0 without extra assumption) & $6601.8±18$ mas/yr (GP-B) ({https://arxiv.org/abs/1105.3456#:~:text=Analysis%20of%20the%20data%20from,9%20rad}{[1105.3456] Gravity Probe B: Final Results of a Space Experiment to Test General Relativity}) & No (VAM missing effect) \\
        ISCO Radius (Schwarzschild BH) & $r_{ISCO}=6GM/c^2$ (for test mass) & No natural ISCO (orbits possible until horizon) & Fe Kα disk lines, GW inspiral waves → match 6GM/c^2 (GR confirmed) & No (needs extra mechanism) \\
        Gravitational Wave Emission (Binary) & Energy loss via quadrupole (P_dot matches to 0.2%) ({https://adsabs.harvard.edu/pdf/2005ASPC..328...25W#:~:text=b%2CGR%20%3D%201,and%20theoretical%20orbital%20decays%20are}{}) & \textit{No GW} (stable or slowly decaying orbit) & PSR1913–16: $-2.405\times10^{-12}$ ({https://adsabs.harvard.edu/pdf/2005ASPC..328...25W#:~:text=quan%02tities%2C%20including%20the%20distance%20and,Hence%20P%CB%99%20b%2Ccorrected}{}) (100% of GR); GW150914: direct detection & No (missing entirely) \\
        \bottomrule
    \end{tabular}
    \caption{}
    \label{tab:}
\end{table}Observational agreement: Gravitational redshift/time dilation has been confirmed for Earth (Pound–Rebka experiment in 1960 showed a frequency shift Δν/ν = 2.5×10^–15 over a 22.5 m tower, matching GR ({https://en.wikipedia.org/wiki/Tests_of_general_relativity#:~:text=line%20width,first%20precision%20experiments%20testing%20general}{Tests of general relativity - Wikipedia})) and for solar spectra (≈2×10^–6 shift, though solar convection adds noise ({https://www.einstein-online.info/en/spotlight/redshift_white_dwarfs/#:~:text=The%20gravitational%20redshift%20was%20first,the%20field%20of%20the%20sun}{Gravitational redshift and White Dwarf stars «  Einstein-Online})). Atomic clock tests (Hafele–Keating and GPS satellites) also confirm the combination of gravitational and velocity time dilation to ~10^–14 precision. VAM, by construction, can fit these results by assigning each massive body an appropriate æther swirl. For Earth and Sun, VAM's required $v_\phi$ at the surface (~escape velocity) produces the correct redshift ({file://file-2nnnwmwbptvdvnbeqndhbe%23:~:text=9/}{GR_in_3d.pdf}) ({file://file-2nnnwmwbptvdvnbeqndhbe%23:~:text=this%20expression%20reflects%20the%20modification,perception%20caused%20by%20local%20rotational/}{GR_in_3d.pdf}), so VAM agrees with GR and observations for static gravitational time dilation.
Where VAM might deviate: VAM introduces a coupling constant β in its rotational energy time dilation formula ({file://file-2nnnwmwbptvdvnbeqndhbe%23:~:text=/}{GR_in_3d.pdf}) ({file://file-2nnnwmwbptvdvnbeqndhbe%23:~:text=%EF%BF%BD%20i%20=%202%205,inertia%20for%20a%20uniform%20sphere/}{GR_in_3d.pdf}). This formula relates stored rotational energy (moment of inertia * Ω²) to time dilation:
\begin{itemize}\item 
\textit{VAM rotational-energy time dilation:} $d\tau/dt = \big(1 + \tfrac{1}{2}\beta I\Omega^2\big)^{-1}$ ({file://file-2nnnwmwbptvdvnbeqndhbe%23:~:text=/}{GR_in_3d.pdf}).
\end{itemize}For macroscopic bodies, VAM chooses β such that $1/2,\beta I\Omega^2 \approx GM/(Rc^2)$, ensuring consistency with GR. If β or the vortex density profile is mis-chosen, VAM would mis-predict gravitational time dilation. Possible correction: Adjust β (the æther vortex-core coupling) or the assumed core radius so that the rotational kinetic energy of the æther vortex exactly equals the gravitational potential energy. This tuning is effectively done in VAM to recover Newton's constant G in the large-scale limit ({file://file-2nnnwmwbptvdvnbeqndhbe%23:~:text=behavior%20and/}{GR_in_3d.pdf}) ({file://file-2nnnwmwbptvdvnbeqndhbe%23:~:text=where:/}{GR_in_3d.pdf}). For elementary particles (proton, electron), VAM postulates a scale-dependent coupling μ(r) that greatly suppresses gravitational effects below ~$r^* \sim 10^{-3}$ m ({file://xn--file-2nnnwmwbptvdvnbeqndhbe%23:~:text=-zn0a/}{GR_in_3d.pdf}). This ensures that an electron's intense internal vortex (which gives it rest mass) does not produce astronomically large gravity. It also implies no detectable deviation from Newton/GR down to ~millimeter scales, consistent with laboratory tests of gravity to ~0.1 mm ({file://xn--file-2nnnwmwbptvdvnbeqndhbe%23:~:text=-zn0a/}{GR_in_3d.pdf}). If future sub-mm experiments found deviations, one could refine $r^*$ or μ(r) accordingly.
\section*{2. Kinetic Time Dilation (Velocity-Based)}Special relativistic time dilation (due to an object's velocity) is equally incorporated in both frameworks. In GR (or SR), a moving clock with speed $v$ ticks slower by $d\tau/dt = \sqrt{1-v^2/c^2}$. VAM's velocity-field time dilation is identical: a clock moving with the æther flow (tangential speed $v_\phi$) experiences $d\tau/dt = \sqrt{1-v_\phi^2/c^2}$ ({file://xn--file-2nnnwmwbptvdvnbeqndhbe%23:~:text=d-tl1a/}{GR_in_3d.pdf}). This is not a new prediction but rather VAM's mechanism to mimic GR effects via æther kinematics.
For example, an atomic clock on Earth's equator (moving $465,$m/s eastward due to rotation) loses on the order of $1\times10^{-11}$ of time per day relative to a stationary pole clock – a tiny SR effect. VAM yields the same since the equatorial clock moves with æther flow ($v_\phi$ at Earth's surface). Similarly, a GPS satellite moving at 3.9 km/s experiences time dilation of 7 μs/day (SR effect) which VAM also predicts, while gravitational blueshift (+45 μs/day) cancels out some of it ({https://www.einstein-online.info/en/spotlight/redshift_white_dwarfs/#:~:text=The%20gravitational%20redshift%20was%20first,the%20field%20of%20the%20sun}{Gravitational redshift and White Dwarf stars «  Einstein-Online}). In all tested cases (e.g. Hafele–Keating airplane clocks, GPS), VAM's velocity-based time dilation = SR, matching observations to high precision. No additional corrections are needed here – this is a strength of VAM, as it is built on Galilean absolute time but manages to reproduce SR timing via æther flow velocity.
\section*{3. Orbital Time Dilation in a Rotating Field (Kerr Metric Analogue)}A more complex scenario is time dilation for clocks orbiting a rotating mass (analogous to the GR Kerr metric). In GR, a clock's rate in orbit is affected by both gravitational potential and the dragging of inertial frames by the central mass's rotation. For instance, a prograde orbital clock around a rotating planet might tick slightly faster than a retrograde one at the same radius, because frame-dragging gives it a speed boost relative to inertial space.
VAM Prediction: VAM attempts to replicate Kerr-like time dilation using the æther's circulation. In the VAM framework, a Kerr-like redshift and frame-dragging structure emerges from two contributions ({file://file-2nnnwmwbptvdvnbeqndhbe%23:~:text=frame,the%20kerr%20metric%20describes%20the/}{GR_in_3d.pdf}) ({file://file-2nnnwmwbptvdvnbeqndhbe%23:~:text=match%20at%20l1996%20%EF%BF%BD%20%CE%BA,dragging/}{GR_in_3d.pdf}):
\begin{itemize}\item 
A gravitational part from the vortex's swirl (as in static case).
\item 
A circulation-induced frame-dragging part from the æther's angular momentum κ (circulation). VAM's combined proper time formula (analogue of Kerr) is given in the paper as ({file://file-2nnnwmwbptvdvnbeqndhbe%23:~:text=%EF%BF%BD%20%CE%BA%20represents%20angular%20momentum,dragging/}{GR_in_3d.pdf}):
\textit{$t_\text{adjusted} = \Delta t \sqrt{1 - \alpha \langle \omega^2 \rangle - \beta, \kappa}$},
where $\langle \omega^2\rangle$ is vorticity intensity and κ is circulation. (This is an outline; the exact form mirrors Kerr's $g_{tt}$ and $g_{t\phi}$ terms ({file://file-2nnnwmwbptvdvnbeqndhbe%23:~:text=this%20mirrors%20the%20kerr%20redshift,dynamic%20variables/}{GR_in_3d.pdf}).)
\end{itemize}In simpler terms, an object orbiting within the æther vortex feels an effective time dilation due to both its orbital speed and the æther's rotation. VAM's orbital time dilation can be approximated by combining the velocity time dilation (from orbital $v$) with the gravitational slowdown from the swirl at that radius. For a circular equatorial orbit around a rotating mass $M$ with angular momentum $J$, GR's proper time per coordinate time is roughly:
dτdt≈1−3GMrc2+2GJωorbc4,\frac{d\tau}{dt} \approx \sqrt{1-\frac{3GM}{rc^2} + \frac{2GJ\omega_\text{orb}}{c^4}},
for small rotations (here $\omega_\text{orb}$ is orbital angular frequency). VAM would analogously have terms for $v_\phi(r)$ and circulation κ.
Case Study – Earth orbit: Consider two identical satellites at the same radius, one orbiting prograde (same direction as Earth's spin) and one retrograde. GR predicts a tiny difference in their clock rates due to frame-dragging by Earth's rotation (on top of gravity and their equal speeds). Using Earth's Kerr metric, the difference is on the order of $10^{-14}$ (negligible). VAM's æther swirl also drags the prograde orbiter along. Because VAM was tuned to match GR's frame-dragging (see next section), it would likewise predict an undetectably small clock rate difference. Both frameworks are effectively in agreement for orbital time dilation around Earth, within current precision.
For extreme cases like a clock orbiting close to a spinning black hole, GR predicts significant effects (e.g. a last stable orbit at radius $r_\text{ISCO}$, see Section 7, and strong gravitational redshift plus frame-dragging). VAM, if extrapolated, can mimic some of these by its swirling flow reaching relativistic speeds near the core (vortex speed $v_\phi \to c$ as $r \to$ horizon) ({file://file-2nnnwmwbptvdvnbeqndhbe%23:~:text=this%20expression%20reflects%20the%20modification,perception%20caused%20by%20local%20rotational/}{GR_in_3d.pdf}). This gives huge time dilation (tending to infinity as $v_\phi \to c$, analog of event horizon) ({file://file-2nnnwmwbptvdvnbeqndhbe%23:~:text=this%20expression%20reflects%20the%20modification,perception%20caused%20by%20local%20rotational/}{GR_in_3d.pdf}). However, VAM might not naturally produce the exact quantitative combination of effects that Kerr geometry does without fine-tuning.
Where VAM may need adjustment: The definition of æther angular momentum (κ) and its coupling to time dilation may need refinement. The VAM formulation introduces a factor μ(r) to transition between quantum and macro scales for frame-dragging (Equation 12: $\omega_\text{drag}^{VAM}(r) = \frac{4G M}{5c^2 r},\mu(r),\Omega(r)$ ({file://xn--file-2nnnwmwbptvdvnbeqndhbe%23:~:text=vam%20drag%20-uz9a/}{GR_in_3d.pdf})). For macroscopic $r$, μ=1, yielding the same form as GR's Lense–Thirring (derivable from $J = I\Omega$ for a sphere) ({file://xn--file-2nnnwmwbptvdvnbeqndhbe%23:~:text=in%20the%20vortex%20ther%20model,angular%20velocity%20induced%20by%20a-lej/}{GR_in_3d.pdf}) ({file://xn--file-2nnnwmwbptvdvnbeqndhbe%23:~:text=5c2r%20%20,12-9mc6802b/}{GR_in_3d.pdf}). If VAM did not include μ, a fast-spinning small object (like a proton) would predict absurdly large frame-dragging. Solution: the introduction of μ(r) (making G effectively scale-dependent) fixes this ({file://xn--file-2nnnwmwbptvdvnbeqndhbe%23:~:text=-zn0a/}{GR_in_3d.pdf}). To accurately reproduce Kerr orbital time dilation in all regimes, VAM might require a similar scale or velocity-dependent factor in its time dilation formula to ensure the combined gravitational+frame-dragging time dilation matches GR's. For now, in the regimes tested (satellites, geodesic orbits), no discrepancy has been observed – both models agree within experimental error, so no immediate correction beyond what VAM already implements (the μ factor) is required.
\section*{4. Frame-Dragging (Lense–Thirring Effect)}A spinning mass in GR \grqq drags\textquotedblright spacetime around with it, causing precession of nearby orbits and gyroscopes (the Lense–Thirring effect or frame-dragging). The angular velocity an inertial frame acquires is $\omega_{LT} = \frac{2GJ}{c^2 r^3}$ for a point mass $J$ at distance $r$ (far outside) ({file://file-2nnnwmwbptvdvnbeqndhbe%23:~:text=c2r3/}{GR_in_3d.pdf}).
Observed evidence: Gravity Probe B measured frame-dragging around Earth. GR predicted a gyroscope's axis would precess ~$39.2$ milliarcsec/yr (mas/yr) due to Earth's rotation, while the measured rate was $37.2 \pm 7.2$ mas/yr ({https://arxiv.org/abs/1105.3456#:~:text=Analysis%20of%20the%20data%20from,9%20rad}{[1105.3456] Gravity Probe B: Final Results of a Space Experiment to Test General Relativity}), consistent within ~19% uncertainty. Another test: LAGEOS satellite node regression ~31 mas/yr vs 39 mas/yr GR prediction ({https://arxiv.org/abs/1105.3456#:~:text=Analysis%20of%20the%20data%20from,9%20rad}{[1105.3456] Gravity Probe B: Final Results of a Space Experiment to Test General Relativity}), within 10% after accounting for uncertainties.
VAM Prediction: VAM defines a \grqq hybrid\textquotedblright frame-dragging angular velocity induced by a rotating æther vortex ({file://xn--file-2nnnwmwbptvdvnbeqndhbe%23:~:text=in%20the%20vortex%20ther%20model,angular%20velocity%20induced%20by%20a-lej/}{GR_in_3d.pdf}). For macroscopic scales ($r > r^* \sim 10^{-3}$ m), VAM gives (Equation 12):
\begin{itemize}\item 
$\omega^{VAM}_\text{drag}(r) = \frac{4G M}{5c^2 r},\Omega(r)$ ({file://xn--file-2nnnwmwbptvdvnbeqndhbe%23:~:text=vam%20drag%20-uz9a/}{GR_in_3d.pdf}),
\end{itemize}where $M$ is the object's mass and $\Omega(r)$ its angular velocity at radius $r$. Using $J \approx \frac{2}{5}MR^2\Omega$ (for a sphere) in GR's formula yields $\omega_{LT} \approx \frac{4G M \Omega}{5c^2 r}$ – the same as VAM's expression for $r \ge R$, since $I=2/5MR^2$ ({file://xn--file-2nnnwmwbptvdvnbeqndhbe%23:~:text=5c2r%20%20,12-9mc6802b/}{GR_in_3d.pdf}). Thus, VAM deliberately matches the GR frame-dragging formula in the large-scale limit ({file://file-2nnnwmwbptvdvnbeqndhbe%23:~:text=2,2gj/}{GR_in_3d.pdf}) ({file://file-2nnnwmwbptvdvnbeqndhbe%23:~:text=c2r3/}{GR_in_3d.pdf}).
Table 2 compares the frame-dragging predictions and observations for Earth's field:
Table 2: Frame-Dragging Precession (Earth's Gravitomagnetism)
\begin{table}
    \centering
    \begin{tabular}{lllll}
        \toprule
        \textbf{Object} & \textbf{GR: $d\tau/dt = \sqrt{1-2GM/(Rc^2)}$} & \textbf{VAM: $d\tau/dt = \sqrt{1-v_\phi^2/c^2}$ (tuned)} & \textbf{Observed Effect} & \textbf{Relative Error (VAM vs Obs)} \\
        \midrule
        Earth (R=6.37×10^6 m) & 0.9999999993 (Δ≈7×10^–10) ({https://www.einstein-online.info/en/spotlight/redshift_white_dwarfs/#:~:text=The%20gravitational%20redshift%20was%20first,the%20field%20of%20the%20sun}{Gravitational redshift and White Dwarf stars «  Einstein-Online}) & 0.9999999993 (assuming $v_\phi\approx 11.2$ km/s) & Clock gain of +45 µs/day at orbit (GPS) ({https://www.einstein-online.info/en/spotlight/redshift_white_dwarfs/#:~:text=The%20gravitational%20redshift%20was%20first,the%20field%20of%20the%20sun}{Gravitational redshift and White Dwarf stars «  Einstein-Online}) & ~0% (VAM tuned to match) \\
        Sun (R=6.96×10^8 m) & 0.9999979 (Δ≈2.1×10^–6) ({https://www.sciencedirect.com/science/article/abs/pii/S1384107614000190#:~:text=On%20the%20gravitational%20redshift%20,in%201908%E2%80%94is%20still%20an}{On the gravitational redshift - ScienceDirect.com}) & 0.9999979 (if $v_\phi\approx 618$ km/s) & Solar spectral line redshift ~2×10^–6 ({https://www.sciencedirect.com/science/article/abs/pii/S1384107614000190#:~:text=On%20the%20gravitational%20redshift%20,in%201908%E2%80%94is%20still%20an}{On the gravitational redshift - ScienceDirect.com}) & ~0% (within meas. error) \\
        Neutron Star (M≈1.4 M_⊙, R≈1×10^4 m) & 0.875 (Δ≈0.234) (strong field) & 0.875 (if $v_\phi\approx0.65c$) & X-ray spectral redshift z~0.3 expected (some observations) & ~0% (assumed match) \\
        Proton (m=1.67×10^–27 kg) & ~1 – 1×10^–27 (negligible) & ~1 (μ factor suppresses gravity at r<1 mm) & No measurable gravitational slowdown & N/A (both predict none) \\
        Electron (m=9.11×10^–31 kg) & ~1 – 1×10^–30 (negligible) & ~1 (suppressed by quantum scaling μ) & No measurable gravitational slowdown & N/A \\
        Effect (Earth) & GR Prediction & VAM Prediction & Observed Value & Error (VAM vs Obs) \\
        Frame-dragging (GP-B gyroscope) & $39.2$ mas/yr (north-south axis precession) ({https://arxiv.org/abs/1105.3456#:~:text=Analysis%20of%20the%20data%20from,9%20rad}{[1105.3456] Gravity Probe B: Final Results of a Space Experiment to Test General Relativity}) & $≈39$ mas/yr (μ=1, identical formula) ({file://file-2nnnwmwbptvdvnbeqndhbe%23:~:text=2,2gj/}{GR_in_3d.pdf}) ({file://file-2nnnwmwbptvdvnbeqndhbe%23:~:text=c2r3/}{GR_in_3d.pdf}) & $37.2 \pm 7.2$ mas/yr (GP-B) ({https://arxiv.org/abs/1105.3456#:~:text=Analysis%20of%20the%20data%20from,9%20rad}{[1105.3456] Gravity Probe B: Final Results of a Space Experiment to Test General Relativity}) & ~0% (within 1σ error) \\
        Frame-dragging (polar LAGEOS) & $~$31 mas/yr (node regression) & $~$31 mas/yr (same as GR) & $30 \pm 5$ mas/yr (measured) & ~0% (within error) \\
        Scenario & GR Redshift $z$ & VAM Redshift $z$ & Observed $z$ (or test) & Result \\
        Pound–Rebka (Earth) – 22.5 m tower & $2.5\times10^{-15}$ (for Δφ = 22.5 m g) ({https://en.wikipedia.org/wiki/Tests_of_general_relativity#:~:text=line%20width,first%20precision%20experiments%20testing%20general}{Tests of general relativity - Wikipedia}) & $2.5\times10^{-15}$ (æther flow equivalent) & $2.5\times10^{-15}$ ± 5% ({https://en.wikipedia.org/wiki/Tests_of_general_relativity#:~:text=line%20width,first%20precision%20experiments%20testing%20general}{Tests of general relativity - Wikipedia}) (gamma-ray) & Matched (0% error) \\
        Sun to infinity (surface) & $2.12\times10^{-6}$ ({https://www.sciencedirect.com/science/article/abs/pii/S1384107614000190#:~:text=On%20the%20gravitational%20redshift%20,in%201908%E2%80%94is%20still%20an}{On the gravitational redshift - ScienceDirect.com}) & $2.12\times10^{-6}$ (if $v_\phi\approx 617$ km/s) & ~$2.12\times10^{-6}$ (solar spectrum, expected) ({https://www.sciencedirect.com/science/article/abs/pii/S1384107614000190#:~:text=On%20the%20gravitational%20redshift%20,in%201908%E2%80%94is%20still%20an}{On the gravitational redshift - ScienceDirect.com}) & Matched (few % error due to solar Doppler) \\
        White Dwarf (Sirius B) – high $GM/R$ & $5.5\times10^{-5}$ (for Sirius B) & $5.5\times10^{-5}$ (if vortex tuned) & $4.8(±0.3)\times10^{-5}$ (observed spectral lines) ({https://www.einstein-online.info/en/spotlight/redshift_white_dwarfs/#:~:text=Consequently%2C%20the%20shift%20should%20be,image%20was%20taken%20with%20the}{Gravitational redshift and White Dwarf stars «  Einstein-Online}) & ~15% error (VAM can tune) \\
        Neutron Star (massive) & $0.3$ (30% freq. drop for $2GM/(Rc^2)\sim0.4$) & $0.3$ (if $v_\phi \sim0.7c$ at surface) & ~0.35 (possible X-ray line measure, uncertain) & ~0% (within uncertainty) \\
        Light Ray & Mass & GR Deflection & VAM Deflection & Observed Deflection & Error \\
        Star near Sun's limb (impact ~R_⊙) & $1.75″$ (arcsec) ([Testing General Relativity & Total Solar Eclipse 2017]({https://eclipse2017.nasa.gov/testing-general-relativity#:~:text=where%20for%20the%20sun%20we,9x108%20meters}{https://eclipse2017.nasa.gov/testing-general-relativity#:~:text=where%20for%20the%20sun%20we,9x108%20meters})) & $1.75″$ ([Testing General Relativity & Total Solar Eclipse 2017]({https://eclipse2017.nasa.gov/testing-general-relativity#:~:text=where%20for%20the%20sun%20we,9x108%20meters}{https://eclipse2017.nasa.gov/testing-general-relativity#:~:text=where%20for%20the%20sun%20we,9x108%20meters})) ({file://file-2nnnwmwbptvdvnbeqndhbe%23:~:text=given%20by:/}{GR_in_3d.pdf}) \\
        Light near Earth (impact ~R⊕) & $8.5\times10^{-6}$″ (microarcsec) & $8.5\times10^{-6}$″ & ~ Not measured (too small) & N/A \\
        Quasar by galaxy (strong lens) & GR lensing formulas (non-linear) & VAM: requires fluid simulation & Multiple images, matches GR lensing & (Unknown for VAM) \\
        Orbit (Central mass) & GR Extra Precession & VAM Precession & Observed Extra Precession & Agreement? \\
        Mercury around Sun & $42.98″/\text{century}$ ({https://math.ucr.edu/home/baez/physics/Relativity/GR/mercury_orbit.html#:~:text=Mercury%27s%20perihelion%20precession%3A%20%20,3%20arcseconds%2Fcentury}{GR and Mercury's Orbital Precession}) & $42.98″/\text{century}$ ({file://xn--file-2nnnwmwbptvdvnbeqndhbe%23:~:text=vam%20=%206gm-sm9aunx840m/}{GR_in_3d.pdf}) & $43.1±0.2″/\text{century}$ ({https://math.ucr.edu/home/baez/physics/Relativity/GR/mercury_orbit.html#:~:text=Mercury%27s%20perihelion%20precession%3A%20%20,3%20arcseconds%2Fcentury}{GR and Mercury's Orbital Precession}) & Yes (within 0.3%) \\
        Earth around Sun & $3.84″/\text{century}$ (calc.) & $3.84″/\text{century}$ & ~$3.84″$ (too small to measure accurately) & Yes (not directly measured) \\
        Binary Pulsar (PSR J0737) – periastron advance & $\sim 16.9°/\text{yr}$ (GR) & $16.9°/\text{yr}$ (VAM, by design) & $16.9°/\text{yr}$ observed (Double Pulsar) & Yes (0% error) \\
        Object & Newtonian/GR Potential at surface $Φ=-GM/R$ (J/kg) & VAM $Φ$ (from vortex) & Surface Gravity $g=GM/R^2$ & Observations (g) \\
        Earth & $-6.25×10^7$ J/kg (→ $v_\text{esc}=11.2$ km/s) & Matches (by construction) ({file://xn--file-2nnnwmwbptvdvnbeqndhbe%23:~:text=2,gm%20r-x034a/}{GR_in_3d.pdf}) & $9.81$ m/s² (observed) & $9.81$ m/s² (exactly) \\
        Sun & $-1.9×10^8$ J/kg (→ $v_\text{esc}=617$ km/s) & Matches (with appropriate γ) & $274$ m/s² (at surface) & ~274 m/s² (via helioseismology) \\
        Neutron Star & ~$-2×10^{13}$ J/kg (for 1.4 M⊙, R=12 km) & Would match if $v_\phi$ near c & ~$1.6×10^{12}$ m/s² (extreme) & Indirect (from orbits, X-ray bursts) \\
        System (Pulsar Binary) & GR $dP/dt$ (orbital period change) & VAM $dP/dt$ & Observed $dP/dt$ & VAM vs Obs Error \\
        PSR B1913+16 (Hulse–Taylor) & $-2.4025\times10^{-12}$ s/s (energy loss via GW) ({https://adsabs.harvard.edu/pdf/2005ASPC..328...25W#:~:text=predicted%20orbital%20period%20derivative%20due,to%20gravitational%20radiation%20computed%20from}{}) & ~ $0$ s/s (no built-in wave emission) & $-2.4056(51)\times10^{-12}$ s/s ({https://adsabs.harvard.edu/pdf/2005ASPC..328...25W#:~:text=quan%02tities%2C%20including%20the%20distance%20and,Hence%20P%CB%99%20b%2Ccorrected}{}) ({https://adsabs.harvard.edu/pdf/2005ASPC..328...25W#:~:text=b%2CGR%20%3D%201,and%20theoretical%20orbital%20decays%20are}{}) & ~100% (fails) \\
        PSR J0737–3039A/B (Double Pulsar) & $-1.252\times10^{-12}$ s/s (predicted) & ~ $0$ s/s & $-1.252(17)\times10^{-12}$ s/s (measured) & ~100% (fails) \\
        GW150914 (Binary BH merger) & 3 solar mass energy radiated in GWs & No GW (merger dynamics unclear) & Detected GWs, strain amplitude $~10^{-21}$ ({http://ui.adsabs.harvard.edu/abs/2010ApJ...722.1030W/abstract#:~:text=Timing%20Measurements%20of%20the%20Relativistic,radiation%20damping%20in%20general}{Timing Measurements of the Relativistic Binary Pulsar PSR B1913+16}) & Complete miss \\
        Precession Effect & GR Prediction (mas/yr) & VAM Prediction (mas/yr) & Observed (mas/yr) & VAM vs Obs \\
        Geodetic (de Sitter) & $6606.1$ mas/yr (forward in orbit plane) ({https://arxiv.org/abs/1105.3456#:~:text=Analysis%20of%20the%20data%20from,9%20rad}{[1105.3456] Gravity Probe B: Final Results of a Space Experiment to Test General Relativity}) & Not explicitly derived (possibly 0 without extension) & $6601.8 \pm 18.3$ mas/yr ({https://arxiv.org/abs/1105.3456#:~:text=Analysis%20of%20the%20data%20from,9%20rad}{[1105.3456] Gravity Probe B: Final Results of a Space Experiment to Test General Relativity}) & ~100% error if 0 \\
        Frame-Dragging & $39.2$ mas/yr (around Earth's spin axis) ({https://arxiv.org/abs/1105.3456#:~:text=Analysis%20of%20the%20data%20from,9%20rad}{[1105.3456] Gravity Probe B: Final Results of a Space Experiment to Test General Relativity}) & $≈39$ mas/yr (by matching formula) ({file://file-2nnnwmwbptvdvnbeqndhbe%23:~:text=2,2gj/}{GR_in_3d.pdf}) & $37.2 \pm 7.2$ mas/yr ({https://arxiv.org/abs/1105.3456#:~:text=Analysis%20of%20the%20data%20from,9%20rad}{[1105.3456] Gravity Probe B: Final Results of a Space Experiment to Test General Relativity}) & ~0% (within error) \\
        Phenomenon & GR Prediction & Formula & VAM Prediction & Formula & Observation (with ref) & Agreement? (Error) \\
        Gravitational Time Dilation (Static field) & $dτ/dt=\sqrt{1-2GM/(rc^2)}$ ({file://file-2nnnwmwbptvdvnbeqndhbe%23:~:text=2/}{GR_in_3d.pdf}) & $dτ/dt=\sqrt{1-Ω^2 r^2/c^2}$ (with $Ω r = v_φ$) ({file://xn--file-2nnnwmwbptvdvnbeqndhbe%23:~:text=%201%202r2-yx7a2275psla/}{GR_in_3d.pdf}) & GPS clocks: Δν/ν = $6.9×10^{-10}$ (Earth) ({https://www.einstein-online.info/en/spotlight/redshift_white_dwarfs/#:~:text=The%20gravitational%20redshift%20was%20first,the%20field%20of%20the%20sun}{Gravitational redshift and White Dwarf stars «  Einstein-Online}), Pound–Rebka: $2.5×10^{-15}$ ({https://en.wikipedia.org/wiki/Tests_of_general_relativity#:~:text=line%20width,first%20precision%20experiments%20testing%20general}{Tests of general relativity - Wikipedia}) (both match GR) & Yes (VAM tuned, 0% error) \\
        Velocity Time Dilation (SR) & $dτ/dt=\sqrt{1-v^2/c^2}$ & $dτ/dt=\sqrt{1-v^2/c^2}$ (same as GR/SR) ({file://xn--file-2nnnwmwbptvdvnbeqndhbe%23:~:text=d-tl1a/}{GR_in_3d.pdf}) & Particle accelerators, muon lifetime (time dilation confirmed to $10^{-8}$) & Yes (identical) \\
        Rotational (Kinetic) Time Dilation & (Included as mass-energy in GR implicitly) & $dτ/dt=(1+\tfrac{1}{2}\beta I Ω^2)^{-1}$ ({file://file-2nnnwmwbptvdvnbeqndhbe%23:~:text=/}{GR_in_3d.pdf}) (VAM heuristic) & No direct obs; fast pulsar (~716 Hz) slows time ~0.5% (GR via mass-energy; VAM via rotation) & In principle (if β set right) \\
        Gravitational Redshift & $z=(1-2GM/(rc^2))^{-1/2}-1$ ({file://file-2nnnwmwbptvdvnbeqndhbe%23:~:text=2/}{GR_in_3d.pdf}) & $z=(1-v_φ^2/c^2)^{-1/2}-1$ ({file://xn--file-2nnnwmwbptvdvnbeqndhbe%23:~:text=%201-py51a/}{GR_in_3d.pdf}) & Solar redshift $2.12×10^{-6}$ ({https://www.sciencedirect.com/science/article/abs/pii/S1384107614000190#:~:text=On%20the%20gravitational%20redshift%20,in%201908%E2%80%94is%20still%20an}{On the gravitational redshift - ScienceDirect.com}); Sirius B $5×10^{-5}$ ({https://www.einstein-online.info/en/spotlight/redshift_white_dwarfs/#:~:text=Consequently%2C%20the%20shift%20should%20be,image%20was%20taken%20with%20the}{Gravitational redshift and White Dwarf stars «  Einstein-Online}); both ~match GR & Yes (VAM=GR) \\
        Light Deflection (Sun) & $\delta = 4GM/(Rc^2) = 1.75″$ ([Testing General Relativity & Total Solar Eclipse 2017]({https://eclipse2017.nasa.gov/testing-general-relativity#:~:text=where%20for%20the%20sun%20we,9x108%20meters}{https://eclipse2017.nasa.gov/testing-general-relativity#:~:text=where%20for%20the%20sun%20we,9x108%20meters})) & $\delta = 4GM/(Rc^2)$ ({file://file-2nnnwmwbptvdvnbeqndhbe%23:~:text=given%20by:/}{GR_in_3d.pdf}) & $1.75″\pm0.07″$ (VLBI) ([Testing General Relativity \\
        Perihelion Precession (Mercury) & $Δϖ = 6πGM/[a(1-e^2)c^2] = 42.98″/$cent. ({https://math.ucr.edu/home/baez/physics/Relativity/GR/mercury_orbit.html#:~:text=Mercury%27s%20perihelion%20precession%3A%20%20,3%20arcseconds%2Fcentury}{GR and Mercury's Orbital Precession}) & \textit{Same as GR} ({file://xn--file-2nnnwmwbptvdvnbeqndhbe%23:~:text=vam%20=%206gm-sm9aunx840m/}{GR_in_3d.pdf}) & $43.1″/$cent. (observed) ({https://math.ucr.edu/home/baez/physics/Relativity/GR/mercury_orbit.html#:~:text=Mercury%27s%20perihelion%20precession%3A%20%20,3%20arcseconds%2Fcentury}{GR and Mercury's Orbital Precession}) & Yes (exact within error) \\
        Frame-Dragging (Earth LT) & $Ω_{LT}=2GJ/(c^2 r^3)$ ({file://file-2nnnwmwbptvdvnbeqndhbe%23:~:text=2,2gj/}{GR_in_3d.pdf}) (≈39 mas/yr) & $Ω_{drag}=\frac{4}{5}\frac{GMΩ}{c^2 r}$ ({file://xn--file-2nnnwmwbptvdvnbeqndhbe%23:~:text=vam%20drag%20-uz9a/}{GR_in_3d.pdf}) (gives same 39 mas/yr) & $37.2±7.2$ mas/yr (GP-B) ({https://arxiv.org/abs/1105.3456#:~:text=Analysis%20of%20the%20data%20from,9%20rad}{[1105.3456] Gravity Probe B: Final Results of a Space Experiment to Test General Relativity}) & Yes (~0%, within 1σ) \\
        Geodetic Precession (Earth) & $Ω_{geo} = \frac{3}{2}\frac{GM}{c^2 r^3} v$ (6606 mas/yr) & \textit{Not derived} (flat space → 0 without extra assumption) & $6601.8±18$ mas/yr (GP-B) ({https://arxiv.org/abs/1105.3456#:~:text=Analysis%20of%20the%20data%20from,9%20rad}{[1105.3456] Gravity Probe B: Final Results of a Space Experiment to Test General Relativity}) & No (VAM missing effect) \\
        ISCO Radius (Schwarzschild BH) & $r_{ISCO}=6GM/c^2$ (for test mass) & No natural ISCO (orbits possible until horizon) & Fe Kα disk lines, GW inspiral waves → match 6GM/c^2 (GR confirmed) & No (needs extra mechanism) \\
        Gravitational Wave Emission (Binary) & Energy loss via quadrupole (P_dot matches to 0.2%) ({https://adsabs.harvard.edu/pdf/2005ASPC..328...25W#:~:text=b%2CGR%20%3D%201,and%20theoretical%20orbital%20decays%20are}{}) & \textit{No GW} (stable or slowly decaying orbit) & PSR1913–16: $-2.405\times10^{-12}$ ({https://adsabs.harvard.edu/pdf/2005ASPC..328...25W#:~:text=quan%02tities%2C%20including%20the%20distance%20and,Hence%20P%CB%99%20b%2Ccorrected}{}) (100% of GR); GW150914: direct detection & No (missing entirely) \\
        \bottomrule
    \end{tabular}
    \caption{}
    \label{tab:}
\end{table}As shown, VAM and GR both agree with the observed frame-dragging to within experimental uncertainty for Earth. VAM achieved this by choosing the form of its vortex-induced angular velocity to align with GR in the classical limit ({file://file-2nnnwmwbptvdvnbeqndhbe%23:~:text=2,2gj/}{GR_in_3d.pdf}).
At quantum scales, VAM's $\mu(r)$ deviates from 1 (for $r < r^* \approx 1$ mm) ({file://xn--file-2nnnwmwbptvdvnbeqndhbe%23:~:text=-zn0a/}{GR_in_3d.pdf}), drastically reducing predicted frame-dragging. For example, an electron's intrinsic spin (angular momentum $\hbar/2$) naively would produce enormous frame-dragging if plugged into $\omega_{LT}$ (because $r$ is tiny), but VAM's $\mu(r) = \frac{r_c C_e}{r^2}$ for $r < r^*$ cuts it down ({file://xn--file-2nnnwmwbptvdvnbeqndhbe%23:~:text=-zn0a/}{GR_in_3d.pdf}). This is necessary for consistency – we don't observe atoms pulling nearby inertial frames measurably. VAM's prescription thus far is successful for frame-dragging: no deviations seen.
Where VAM might need improvement: The angular momentum distribution within a real object can affect frame-dragging. Earth is not a point mass; a portion of its mass is closer to the gyroscope than the full radius, contributing differently. GR can handle this via numerical integration of the gravitomagnetic field; VAM's formula assumed a uniform sphere moment of inertia (hence 4/5 factor). Earth's actual $I=0.33MR^2$ yields $\omega_{LT}$ slightly different in the interior. The 5–20% experimental error bars so far haven't required a correction, but a more precise measurement in the future might detect subtle deviations. Potential correction: Refine VAM's frame-dragging formula by integrating the æther vorticity over the object's volume instead of using a single $I$ at radius $R$. This would incorporate mass distribution (density profile) – for Earth, it might reduce the effective coefficient to match the precise GR result. In short, while current tests show agreement, ensuring \textit{exact} agreement would involve adjusting VAM's model of æther swirl inside the mass (non-uniform ω(r) profile) so that the external frame-dragging field matches GR's multipole structure.
\section*{5. Gravitational Redshift (Frequency Shift of Light)}Gravitational redshift is closely tied to time dilation – light emitted from a deeper gravitational potential is observed at lower frequency (longer wavelength) when climbing out to a higher potential.
GR Prediction: $\displaystyle z \equiv \frac{\Delta \nu}{\nu} = \frac{\nu_\text{source} - \nu_\text{observer}}{\nu_{\observer}} = \sqrt{\frac{1}{1-\frac{2GM}{rc^2}}}; - 1$ for a photon emitted at radius $r$ (for small potential, $z\approx \frac{GM}{rc^2}$). For example, light from the Sun's surface (where $2GM_{\odot}/(Rc^2)\approx 2.12\times10^{-6}$) should be redshifted by ~2.12×10^–6 ({https://www.sciencedirect.com/science/article/abs/pii/S1384107614000190#:~:text=On%20the%20gravitational%20redshift%20,in%201908%E2%80%94is%20still%20an}{On the gravitational redshift - ScienceDirect.com}).
VAM Prediction: VAM replaces the gravitational potential with the æther's rotational kinetic term. The VAM gravitational redshift formula (for light from the boundary of a vortex) is ({file://file-2nnnwmwbptvdvnbeqndhbe%23:~:text=zvam%20=/}{GR_in_3d.pdf}):
\begin{itemize}\item 
$z_{VAM} = (1 - v_\phi^2/c^2)^{-1/2} - 1$ ({file://file-2nnnwmwbptvdvnbeqndhbe%23:~:text=zvam%20=/}{GR_in_3d.pdf}),
\end{itemize}which indeed matches the GR form if $v_\phi^2/c^2$ is set equal to $2GM/(rc^2)$ ({file://file-2nnnwmwbptvdvnbeqndhbe%23:~:text=/}{GR_in_3d.pdf}) ({file://file-2nnnwmwbptvdvnbeqndhbe%23:~:text=vam%20z%20=/}{GR_in_3d.pdf}). For small $v_\phi$, expand $(1-\frac{v_\phi^2}{c^2})^{-1/2} \approx 1 + \frac{1}{2}\frac{v_\phi^2}{c^2}$; if $v_\phi^2/2 = GM/r$, then $z \approx GM/(rc^2)$, reproducing GR at first order. In fact, VAM's summary table explicitly equates the two expressions ({file://file-2nnnwmwbptvdvnbeqndhbe%23:~:text=2/}{GR_in_3d.pdf}) ({file://xn--file-2nnnwmwbptvdvnbeqndhbe%23:~:text=%201-py51a/}{GR_in_3d.pdf}). Thus for macroscopic masses, VAM predicts the same gravitational redshift as GR, assuming the æther's tangential speed is calibrated appropriately.
We compile known values in Table 3:
Table 3: Gravitational Redshift of Emitted Light
\begin{table}
    \centering
    \begin{tabular}{lllll}
        \toprule
        \textbf{Object} & \textbf{GR: $d\tau/dt = \sqrt{1-2GM/(Rc^2)}$} & \textbf{VAM: $d\tau/dt = \sqrt{1-v_\phi^2/c^2}$ (tuned)} & \textbf{Observed Effect} & \textbf{Relative Error (VAM vs Obs)} \\
        \midrule
        Earth (R=6.37×10^6 m) & 0.9999999993 (Δ≈7×10^–10) ({https://www.einstein-online.info/en/spotlight/redshift_white_dwarfs/#:~:text=The%20gravitational%20redshift%20was%20first,the%20field%20of%20the%20sun}{Gravitational redshift and White Dwarf stars «  Einstein-Online}) & 0.9999999993 (assuming $v_\phi\approx 11.2$ km/s) & Clock gain of +45 µs/day at orbit (GPS) ({https://www.einstein-online.info/en/spotlight/redshift_white_dwarfs/#:~:text=The%20gravitational%20redshift%20was%20first,the%20field%20of%20the%20sun}{Gravitational redshift and White Dwarf stars «  Einstein-Online}) & ~0% (VAM tuned to match) \\
        Sun (R=6.96×10^8 m) & 0.9999979 (Δ≈2.1×10^–6) ({https://www.sciencedirect.com/science/article/abs/pii/S1384107614000190#:~:text=On%20the%20gravitational%20redshift%20,in%201908%E2%80%94is%20still%20an}{On the gravitational redshift - ScienceDirect.com}) & 0.9999979 (if $v_\phi\approx 618$ km/s) & Solar spectral line redshift ~2×10^–6 ({https://www.sciencedirect.com/science/article/abs/pii/S1384107614000190#:~:text=On%20the%20gravitational%20redshift%20,in%201908%E2%80%94is%20still%20an}{On the gravitational redshift - ScienceDirect.com}) & ~0% (within meas. error) \\
        Neutron Star (M≈1.4 M_⊙, R≈1×10^4 m) & 0.875 (Δ≈0.234) (strong field) & 0.875 (if $v_\phi\approx0.65c$) & X-ray spectral redshift z~0.3 expected (some observations) & ~0% (assumed match) \\
        Proton (m=1.67×10^–27 kg) & ~1 – 1×10^–27 (negligible) & ~1 (μ factor suppresses gravity at r<1 mm) & No measurable gravitational slowdown & N/A (both predict none) \\
        Electron (m=9.11×10^–31 kg) & ~1 – 1×10^–30 (negligible) & ~1 (suppressed by quantum scaling μ) & No measurable gravitational slowdown & N/A \\
        Effect (Earth) & GR Prediction & VAM Prediction & Observed Value & Error (VAM vs Obs) \\
        Frame-dragging (GP-B gyroscope) & $39.2$ mas/yr (north-south axis precession) ({https://arxiv.org/abs/1105.3456#:~:text=Analysis%20of%20the%20data%20from,9%20rad}{[1105.3456] Gravity Probe B: Final Results of a Space Experiment to Test General Relativity}) & $≈39$ mas/yr (μ=1, identical formula) ({file://file-2nnnwmwbptvdvnbeqndhbe%23:~:text=2,2gj/}{GR_in_3d.pdf}) ({file://file-2nnnwmwbptvdvnbeqndhbe%23:~:text=c2r3/}{GR_in_3d.pdf}) & $37.2 \pm 7.2$ mas/yr (GP-B) ({https://arxiv.org/abs/1105.3456#:~:text=Analysis%20of%20the%20data%20from,9%20rad}{[1105.3456] Gravity Probe B: Final Results of a Space Experiment to Test General Relativity}) & ~0% (within 1σ error) \\
        Frame-dragging (polar LAGEOS) & $~$31 mas/yr (node regression) & $~$31 mas/yr (same as GR) & $30 \pm 5$ mas/yr (measured) & ~0% (within error) \\
        Scenario & GR Redshift $z$ & VAM Redshift $z$ & Observed $z$ (or test) & Result \\
        Pound–Rebka (Earth) – 22.5 m tower & $2.5\times10^{-15}$ (for Δφ = 22.5 m g) ({https://en.wikipedia.org/wiki/Tests_of_general_relativity#:~:text=line%20width,first%20precision%20experiments%20testing%20general}{Tests of general relativity - Wikipedia}) & $2.5\times10^{-15}$ (æther flow equivalent) & $2.5\times10^{-15}$ ± 5% ({https://en.wikipedia.org/wiki/Tests_of_general_relativity#:~:text=line%20width,first%20precision%20experiments%20testing%20general}{Tests of general relativity - Wikipedia}) (gamma-ray) & Matched (0% error) \\
        Sun to infinity (surface) & $2.12\times10^{-6}$ ({https://www.sciencedirect.com/science/article/abs/pii/S1384107614000190#:~:text=On%20the%20gravitational%20redshift%20,in%201908%E2%80%94is%20still%20an}{On the gravitational redshift - ScienceDirect.com}) & $2.12\times10^{-6}$ (if $v_\phi\approx 617$ km/s) & ~$2.12\times10^{-6}$ (solar spectrum, expected) ({https://www.sciencedirect.com/science/article/abs/pii/S1384107614000190#:~:text=On%20the%20gravitational%20redshift%20,in%201908%E2%80%94is%20still%20an}{On the gravitational redshift - ScienceDirect.com}) & Matched (few % error due to solar Doppler) \\
        White Dwarf (Sirius B) – high $GM/R$ & $5.5\times10^{-5}$ (for Sirius B) & $5.5\times10^{-5}$ (if vortex tuned) & $4.8(±0.3)\times10^{-5}$ (observed spectral lines) ({https://www.einstein-online.info/en/spotlight/redshift_white_dwarfs/#:~:text=Consequently%2C%20the%20shift%20should%20be,image%20was%20taken%20with%20the}{Gravitational redshift and White Dwarf stars «  Einstein-Online}) & ~15% error (VAM can tune) \\
        Neutron Star (massive) & $0.3$ (30% freq. drop for $2GM/(Rc^2)\sim0.4$) & $0.3$ (if $v_\phi \sim0.7c$ at surface) & ~0.35 (possible X-ray line measure, uncertain) & ~0% (within uncertainty) \\
        Light Ray & Mass & GR Deflection & VAM Deflection & Observed Deflection & Error \\
        Star near Sun's limb (impact ~R_⊙) & $1.75″$ (arcsec) ([Testing General Relativity & Total Solar Eclipse 2017]({https://eclipse2017.nasa.gov/testing-general-relativity#:~:text=where%20for%20the%20sun%20we,9x108%20meters}{https://eclipse2017.nasa.gov/testing-general-relativity#:~:text=where%20for%20the%20sun%20we,9x108%20meters})) & $1.75″$ ([Testing General Relativity & Total Solar Eclipse 2017]({https://eclipse2017.nasa.gov/testing-general-relativity#:~:text=where%20for%20the%20sun%20we,9x108%20meters}{https://eclipse2017.nasa.gov/testing-general-relativity#:~:text=where%20for%20the%20sun%20we,9x108%20meters})) ({file://file-2nnnwmwbptvdvnbeqndhbe%23:~:text=given%20by:/}{GR_in_3d.pdf}) \\
        Light near Earth (impact ~R⊕) & $8.5\times10^{-6}$″ (microarcsec) & $8.5\times10^{-6}$″ & ~ Not measured (too small) & N/A \\
        Quasar by galaxy (strong lens) & GR lensing formulas (non-linear) & VAM: requires fluid simulation & Multiple images, matches GR lensing & (Unknown for VAM) \\
        Orbit (Central mass) & GR Extra Precession & VAM Precession & Observed Extra Precession & Agreement? \\
        Mercury around Sun & $42.98″/\text{century}$ ({https://math.ucr.edu/home/baez/physics/Relativity/GR/mercury_orbit.html#:~:text=Mercury%27s%20perihelion%20precession%3A%20%20,3%20arcseconds%2Fcentury}{GR and Mercury's Orbital Precession}) & $42.98″/\text{century}$ ({file://xn--file-2nnnwmwbptvdvnbeqndhbe%23:~:text=vam%20=%206gm-sm9aunx840m/}{GR_in_3d.pdf}) & $43.1±0.2″/\text{century}$ ({https://math.ucr.edu/home/baez/physics/Relativity/GR/mercury_orbit.html#:~:text=Mercury%27s%20perihelion%20precession%3A%20%20,3%20arcseconds%2Fcentury}{GR and Mercury's Orbital Precession}) & Yes (within 0.3%) \\
        Earth around Sun & $3.84″/\text{century}$ (calc.) & $3.84″/\text{century}$ & ~$3.84″$ (too small to measure accurately) & Yes (not directly measured) \\
        Binary Pulsar (PSR J0737) – periastron advance & $\sim 16.9°/\text{yr}$ (GR) & $16.9°/\text{yr}$ (VAM, by design) & $16.9°/\text{yr}$ observed (Double Pulsar) & Yes (0% error) \\
        Object & Newtonian/GR Potential at surface $Φ=-GM/R$ (J/kg) & VAM $Φ$ (from vortex) & Surface Gravity $g=GM/R^2$ & Observations (g) \\
        Earth & $-6.25×10^7$ J/kg (→ $v_\text{esc}=11.2$ km/s) & Matches (by construction) ({file://xn--file-2nnnwmwbptvdvnbeqndhbe%23:~:text=2,gm%20r-x034a/}{GR_in_3d.pdf}) & $9.81$ m/s² (observed) & $9.81$ m/s² (exactly) \\
        Sun & $-1.9×10^8$ J/kg (→ $v_\text{esc}=617$ km/s) & Matches (with appropriate γ) & $274$ m/s² (at surface) & ~274 m/s² (via helioseismology) \\
        Neutron Star & ~$-2×10^{13}$ J/kg (for 1.4 M⊙, R=12 km) & Would match if $v_\phi$ near c & ~$1.6×10^{12}$ m/s² (extreme) & Indirect (from orbits, X-ray bursts) \\
        System (Pulsar Binary) & GR $dP/dt$ (orbital period change) & VAM $dP/dt$ & Observed $dP/dt$ & VAM vs Obs Error \\
        PSR B1913+16 (Hulse–Taylor) & $-2.4025\times10^{-12}$ s/s (energy loss via GW) ({https://adsabs.harvard.edu/pdf/2005ASPC..328...25W#:~:text=predicted%20orbital%20period%20derivative%20due,to%20gravitational%20radiation%20computed%20from}{}) & ~ $0$ s/s (no built-in wave emission) & $-2.4056(51)\times10^{-12}$ s/s ({https://adsabs.harvard.edu/pdf/2005ASPC..328...25W#:~:text=quan%02tities%2C%20including%20the%20distance%20and,Hence%20P%CB%99%20b%2Ccorrected}{}) ({https://adsabs.harvard.edu/pdf/2005ASPC..328...25W#:~:text=b%2CGR%20%3D%201,and%20theoretical%20orbital%20decays%20are}{}) & ~100% (fails) \\
        PSR J0737–3039A/B (Double Pulsar) & $-1.252\times10^{-12}$ s/s (predicted) & ~ $0$ s/s & $-1.252(17)\times10^{-12}$ s/s (measured) & ~100% (fails) \\
        GW150914 (Binary BH merger) & 3 solar mass energy radiated in GWs & No GW (merger dynamics unclear) & Detected GWs, strain amplitude $~10^{-21}$ ({http://ui.adsabs.harvard.edu/abs/2010ApJ...722.1030W/abstract#:~:text=Timing%20Measurements%20of%20the%20Relativistic,radiation%20damping%20in%20general}{Timing Measurements of the Relativistic Binary Pulsar PSR B1913+16}) & Complete miss \\
        Precession Effect & GR Prediction (mas/yr) & VAM Prediction (mas/yr) & Observed (mas/yr) & VAM vs Obs \\
        Geodetic (de Sitter) & $6606.1$ mas/yr (forward in orbit plane) ({https://arxiv.org/abs/1105.3456#:~:text=Analysis%20of%20the%20data%20from,9%20rad}{[1105.3456] Gravity Probe B: Final Results of a Space Experiment to Test General Relativity}) & Not explicitly derived (possibly 0 without extension) & $6601.8 \pm 18.3$ mas/yr ({https://arxiv.org/abs/1105.3456#:~:text=Analysis%20of%20the%20data%20from,9%20rad}{[1105.3456] Gravity Probe B: Final Results of a Space Experiment to Test General Relativity}) & ~100% error if 0 \\
        Frame-Dragging & $39.2$ mas/yr (around Earth's spin axis) ({https://arxiv.org/abs/1105.3456#:~:text=Analysis%20of%20the%20data%20from,9%20rad}{[1105.3456] Gravity Probe B: Final Results of a Space Experiment to Test General Relativity}) & $≈39$ mas/yr (by matching formula) ({file://file-2nnnwmwbptvdvnbeqndhbe%23:~:text=2,2gj/}{GR_in_3d.pdf}) & $37.2 \pm 7.2$ mas/yr ({https://arxiv.org/abs/1105.3456#:~:text=Analysis%20of%20the%20data%20from,9%20rad}{[1105.3456] Gravity Probe B: Final Results of a Space Experiment to Test General Relativity}) & ~0% (within error) \\
        Phenomenon & GR Prediction & Formula & VAM Prediction & Formula & Observation (with ref) & Agreement? (Error) \\
        Gravitational Time Dilation (Static field) & $dτ/dt=\sqrt{1-2GM/(rc^2)}$ ({file://file-2nnnwmwbptvdvnbeqndhbe%23:~:text=2/}{GR_in_3d.pdf}) & $dτ/dt=\sqrt{1-Ω^2 r^2/c^2}$ (with $Ω r = v_φ$) ({file://xn--file-2nnnwmwbptvdvnbeqndhbe%23:~:text=%201%202r2-yx7a2275psla/}{GR_in_3d.pdf}) & GPS clocks: Δν/ν = $6.9×10^{-10}$ (Earth) ({https://www.einstein-online.info/en/spotlight/redshift_white_dwarfs/#:~:text=The%20gravitational%20redshift%20was%20first,the%20field%20of%20the%20sun}{Gravitational redshift and White Dwarf stars «  Einstein-Online}), Pound–Rebka: $2.5×10^{-15}$ ({https://en.wikipedia.org/wiki/Tests_of_general_relativity#:~:text=line%20width,first%20precision%20experiments%20testing%20general}{Tests of general relativity - Wikipedia}) (both match GR) & Yes (VAM tuned, 0% error) \\
        Velocity Time Dilation (SR) & $dτ/dt=\sqrt{1-v^2/c^2}$ & $dτ/dt=\sqrt{1-v^2/c^2}$ (same as GR/SR) ({file://xn--file-2nnnwmwbptvdvnbeqndhbe%23:~:text=d-tl1a/}{GR_in_3d.pdf}) & Particle accelerators, muon lifetime (time dilation confirmed to $10^{-8}$) & Yes (identical) \\
        Rotational (Kinetic) Time Dilation & (Included as mass-energy in GR implicitly) & $dτ/dt=(1+\tfrac{1}{2}\beta I Ω^2)^{-1}$ ({file://file-2nnnwmwbptvdvnbeqndhbe%23:~:text=/}{GR_in_3d.pdf}) (VAM heuristic) & No direct obs; fast pulsar (~716 Hz) slows time ~0.5% (GR via mass-energy; VAM via rotation) & In principle (if β set right) \\
        Gravitational Redshift & $z=(1-2GM/(rc^2))^{-1/2}-1$ ({file://file-2nnnwmwbptvdvnbeqndhbe%23:~:text=2/}{GR_in_3d.pdf}) & $z=(1-v_φ^2/c^2)^{-1/2}-1$ ({file://xn--file-2nnnwmwbptvdvnbeqndhbe%23:~:text=%201-py51a/}{GR_in_3d.pdf}) & Solar redshift $2.12×10^{-6}$ ({https://www.sciencedirect.com/science/article/abs/pii/S1384107614000190#:~:text=On%20the%20gravitational%20redshift%20,in%201908%E2%80%94is%20still%20an}{On the gravitational redshift - ScienceDirect.com}); Sirius B $5×10^{-5}$ ({https://www.einstein-online.info/en/spotlight/redshift_white_dwarfs/#:~:text=Consequently%2C%20the%20shift%20should%20be,image%20was%20taken%20with%20the}{Gravitational redshift and White Dwarf stars «  Einstein-Online}); both ~match GR & Yes (VAM=GR) \\
        Light Deflection (Sun) & $\delta = 4GM/(Rc^2) = 1.75″$ ([Testing General Relativity & Total Solar Eclipse 2017]({https://eclipse2017.nasa.gov/testing-general-relativity#:~:text=where%20for%20the%20sun%20we,9x108%20meters}{https://eclipse2017.nasa.gov/testing-general-relativity#:~:text=where%20for%20the%20sun%20we,9x108%20meters})) & $\delta = 4GM/(Rc^2)$ ({file://file-2nnnwmwbptvdvnbeqndhbe%23:~:text=given%20by:/}{GR_in_3d.pdf}) & $1.75″\pm0.07″$ (VLBI) ([Testing General Relativity \\
        Perihelion Precession (Mercury) & $Δϖ = 6πGM/[a(1-e^2)c^2] = 42.98″/$cent. ({https://math.ucr.edu/home/baez/physics/Relativity/GR/mercury_orbit.html#:~:text=Mercury%27s%20perihelion%20precession%3A%20%20,3%20arcseconds%2Fcentury}{GR and Mercury's Orbital Precession}) & \textit{Same as GR} ({file://xn--file-2nnnwmwbptvdvnbeqndhbe%23:~:text=vam%20=%206gm-sm9aunx840m/}{GR_in_3d.pdf}) & $43.1″/$cent. (observed) ({https://math.ucr.edu/home/baez/physics/Relativity/GR/mercury_orbit.html#:~:text=Mercury%27s%20perihelion%20precession%3A%20%20,3%20arcseconds%2Fcentury}{GR and Mercury's Orbital Precession}) & Yes (exact within error) \\
        Frame-Dragging (Earth LT) & $Ω_{LT}=2GJ/(c^2 r^3)$ ({file://file-2nnnwmwbptvdvnbeqndhbe%23:~:text=2,2gj/}{GR_in_3d.pdf}) (≈39 mas/yr) & $Ω_{drag}=\frac{4}{5}\frac{GMΩ}{c^2 r}$ ({file://xn--file-2nnnwmwbptvdvnbeqndhbe%23:~:text=vam%20drag%20-uz9a/}{GR_in_3d.pdf}) (gives same 39 mas/yr) & $37.2±7.2$ mas/yr (GP-B) ({https://arxiv.org/abs/1105.3456#:~:text=Analysis%20of%20the%20data%20from,9%20rad}{[1105.3456] Gravity Probe B: Final Results of a Space Experiment to Test General Relativity}) & Yes (~0%, within 1σ) \\
        Geodetic Precession (Earth) & $Ω_{geo} = \frac{3}{2}\frac{GM}{c^2 r^3} v$ (6606 mas/yr) & \textit{Not derived} (flat space → 0 without extra assumption) & $6601.8±18$ mas/yr (GP-B) ({https://arxiv.org/abs/1105.3456#:~:text=Analysis%20of%20the%20data%20from,9%20rad}{[1105.3456] Gravity Probe B: Final Results of a Space Experiment to Test General Relativity}) & No (VAM missing effect) \\
        ISCO Radius (Schwarzschild BH) & $r_{ISCO}=6GM/c^2$ (for test mass) & No natural ISCO (orbits possible until horizon) & Fe Kα disk lines, GW inspiral waves → match 6GM/c^2 (GR confirmed) & No (needs extra mechanism) \\
        Gravitational Wave Emission (Binary) & Energy loss via quadrupole (P_dot matches to 0.2%) ({https://adsabs.harvard.edu/pdf/2005ASPC..328...25W#:~:text=b%2CGR%20%3D%201,and%20theoretical%20orbital%20decays%20are}{}) & \textit{No GW} (stable or slowly decaying orbit) & PSR1913–16: $-2.405\times10^{-12}$ ({https://adsabs.harvard.edu/pdf/2005ASPC..328...25W#:~:text=quan%02tities%2C%20including%20the%20distance%20and,Hence%20P%CB%99%20b%2Ccorrected}{}) (100% of GR); GW150914: direct detection & No (missing entirely) \\
        \bottomrule
    \end{tabular}
    \caption{}
    \label{tab:}
\end{table}\textit{Comments:} Pound–Rebka in 1960 was a direct terrestrial test and confirmed GR's redshift to ~1% ({https://en.wikipedia.org/wiki/Tests_of_general_relativity#:~:text=line%20width,first%20precision%20experiments%20testing%20general}{Tests of general relativity - Wikipedia}). VAM naturally gives the same result because in a 22.5 m height difference, the æther flow speed difference is tiny and can be treated via equivalence principle (in fact, VAM reduces to Newtonian gravity + SR for such weak fields, giving the same prediction). Solar redshift was first observed in 1920s; VAM and GR both yield ~2 ppm effect, but solar convective motions (~20–30 ppm Doppler shifts) complicate precise verification ({https://www.einstein-online.info/en/spotlight/redshift_white_dwarfs/#:~:text=Rebka%2C%20and%20Snider%20at%20Harvard,the%20field%20of%20the%20sun}{Gravitational redshift and White Dwarf stars «  Einstein-Online}). Still, modern spectroscopy (moon reflectance spectra) measure ~$636\pm 3$ m/s equivalent shift vs GR's 633 m/s ({https://www.sciencedirect.com/science/article/abs/pii/S1384107614000190#:~:text=On%20the%20gravitational%20redshift%20,in%201908%E2%80%94is%20still%20an}{On the gravitational redshift - ScienceDirect.com}), confirming it within ~0.5%. VAM's mechanism (pressure deficit from solar vortex) would produce that same shift as long as the solar æther rotation is set by the Sun's mass.
For compact stars, high redshifts provide a sharper test. Sirius B's white dwarf gravitational redshift (~80 km/s) was measured in 1925 as a key early proof; the observed $z = 5\times10^{-5}$ matched GR within ~5% (later refined) ({https://www.einstein-online.info/en/spotlight/redshift_white_dwarfs/#:~:text=Consequently%2C%20the%20shift%20should%20be,image%20was%20taken%20with%20the}{Gravitational redshift and White Dwarf stars «  Einstein-Online}). VAM could accommodate this by assigning Sirius B's æther vortex such that $v_\phi \approx 1700$ km/s at its surface (which is 5000 km radius). This is below $c$ (so physically possible) and would yield the required redshift. Neutron stars are even more extreme: for a typical NS (M1.4 M$\textit{\odot$, R~12 km), $z \approx 0.3$. Some candidate spectral lines in X-ray bursters imply $z\sim0.35$, consistent with GR, though systematic uncertainties are large. VAM's vortex for a neutron star would likely be rotating at $v}\phi$ close to light speed at the surface (e.g. 0.7c–0.8c) to produce such a redshift. Notably, VAM predicts a \grqq natural cutoff\textquotedblright as $v_\phi \to c$ – $z \to \infty$ ({file://xn--file-2nnnwmwbptvdvnbeqndhbe%23:~:text=velocity,redshift%20for%20low%20v,%20and-sd8c/}{GR_in_3d.pdf}), analogous to an event horizon. In other words, if a star's æther vortex had to spin at $c$ at the surface to match gravity, that radius would behave like a horizon (time stops), corresponding to the Schwarzschild radius. This shows VAM can emulate the concept of a black hole event horizon (where $z \to ∞$) via reaching the speed-of-light limit in the fluid ({file://file-2nnnwmwbptvdvnbeqndhbe%23:~:text=this%20expression%20reflects%20the%20modification,perception%20caused%20by%20local%20rotational/}{GR_in_3d.pdf}) ({file://xn--file-2nnnwmwbptvdvnbeqndhbe%23:~:text=velocity,redshift%20for%20low%20v,%20and-sd8c/}{GR_in_3d.pdf}).
Assessment: Gravitational redshift is no problem for VAM – it was explicitly constructed to reproduce the GR result across scales ({file://file-2nnnwmwbptvdvnbeqndhbe%23:~:text=2/}{GR_in_3d.pdf}) ({file://xn--file-2nnnwmwbptvdvnbeqndhbe%23:~:text=%201-py51a/}{GR_in_3d.pdf}). The only caveat is that VAM requires calibration of the swirl speed to each mass. In a unified theory one would want $v_\phi$ to be predicted from fundamental parameters (mass, maybe vortex quantum number) rather than \grqq set to escape velocity.\textquotedblright The VAM framework suggests an object's mass \textit{is} the energy of its æther vortex, so in principle $v_\phi$ is not arbitrary but determined by that energy. In practice, one might need to adjust the æther density or vortex core size to make this exact. For example, if the æther density were different, the same $v_\phi$ would produce a different effective $GM$. Proposed fix (conceptual): derive $v_\phi$ from first principles by equating vortex pressure deficit to the required gravitational potential. The current VAM paper hints at this by introducing a \grqq vorticity–gravity coupling\textquotedblright γ such that Newton's $GM$ is recovered ({file://xn--file-2nnnwmwbptvdvnbeqndhbe%23:~:text=where%20%20in%20m5%2Fs2%20is,this-eg2c/}{GR_in_3d.pdf}) ({file://file-2nnnwmwbptvdvnbeqndhbe%23:~:text=%EF%BF%BD%20g:%20vortex%20coupling%20constant,newtonian%20g%20under%20macroscopic%20limits/}{GR_in_3d.pdf}). Fine-tuning that coupling (essentially calibrating how much circulation corresponds to a given mass) ensures redshift matches exactly. As of now, with the tuning done to match known masses, no discrepancies appear in redshift tests.
\section*{6. Deflection of Light by Gravity}The deflection of light passing near a massive body was one of Einstein's classic tests.
GR Prediction: A light ray grazing the Sun is deflected by an angle $\delta = \frac{4GM}{Rc^2} \approx 1.75$ arcseconds ({https://eclipse2017.nasa.gov/testing-general-relativity#:~:text=where%20for%20the%20sun%20we,9x108%20meters}{Testing General Relativity | Total Solar Eclipse 2017}). This formula was confirmed in the 1919 solar eclipse expedition (Eddington's measurement) to within ~20% (they found ~1.98″ ± 0.30″) ({https://www.americanscientist.org/article/bent-starlight#:~:text=Bent%20Starlight%20,that%20Einstein%20had%20predicted}{Bent Starlight | American Scientist}), and has since been confirmed to high precision by radio VLBI (modern error $<0.1%$).
VAM Prediction: In VAM, light is treated as a wave perturbation in the æther. A massive vortex creates a pressure gradient and an anisotropic refractive index in the æther, causing light to bend ({file://xn--file-2nnnwmwbptvdvnbeqndhbe%23:~:text=the%20vortex%20ther%20model,%20light,bends%20due-ckh/}{GR_in_3d.pdf}) ({file://file-2nnnwmwbptvdvnbeqndhbe%23:~:text=the%20equivalent%20vam%20deflection%20angle,a%20spherical%20vortex%20mass%20is/}{GR_in_3d.pdf}). The derived VAM deflection for a light ray with closest approach $R$ is given as ({file://file-2nnnwmwbptvdvnbeqndhbe%23:~:text=given%20by:/}{GR_in_3d.pdf}):
\begin{itemize}\item 
$\delta_{VAM} = \frac{4GM}{R c^2}$ ({file://file-2nnnwmwbptvdvnbeqndhbe%23:~:text=given%20by:/}{GR_in_3d.pdf}),
\end{itemize}identical in form to GR's result. VAM's explanation is that the wavefront of light experiences faster propagation on the far side of the vortex (lower index) than the near side, bending its path as if influenced by gravity ({file://xn--file-2nnnwmwbptvdvnbeqndhbe%23:~:text=in%20vam,%20this%20results%20from,the%20lights%20propagation%20velocity%20and-wb08e/}{GR_in_3d.pdf}). The key point is that VAM exactly reproduces the GR deflection angle for a given mass (with $G$ appropriately identified as the vortex coupling constant) ({file://xn--file-2nnnwmwbptvdvnbeqndhbe%23:~:text=vam%20=%204gm-n38a/}{GR_in_3d.pdf}) ({file://file-2nnnwmwbptvdvnbeqndhbe%23:~:text=4gm/}{GR_in_3d.pdf}).
Comparison and Observations: Table 4 shows the numbers for light deflection:
Table 4: Light Deflection by Gravity (Sun as example)
\begin{table}
    \centering
    \begin{tabular}{lllll}
        \toprule
        \textbf{Object} & \textbf{GR: $d\tau/dt = \sqrt{1-2GM/(Rc^2)}$} & \textbf{VAM: $d\tau/dt = \sqrt{1-v_\phi^2/c^2}$ (tuned)} & \textbf{Observed Effect} & \textbf{Relative Error (VAM vs Obs)} \\
        \midrule
        Earth (R=6.37×10^6 m) & 0.9999999993 (Δ≈7×10^–10) ({https://www.einstein-online.info/en/spotlight/redshift_white_dwarfs/#:~:text=The%20gravitational%20redshift%20was%20first,the%20field%20of%20the%20sun}{Gravitational redshift and White Dwarf stars «  Einstein-Online}) & 0.9999999993 (assuming $v_\phi\approx 11.2$ km/s) & Clock gain of +45 µs/day at orbit (GPS) ({https://www.einstein-online.info/en/spotlight/redshift_white_dwarfs/#:~:text=The%20gravitational%20redshift%20was%20first,the%20field%20of%20the%20sun}{Gravitational redshift and White Dwarf stars «  Einstein-Online}) & ~0% (VAM tuned to match) \\
        Sun (R=6.96×10^8 m) & 0.9999979 (Δ≈2.1×10^–6) ({https://www.sciencedirect.com/science/article/abs/pii/S1384107614000190#:~:text=On%20the%20gravitational%20redshift%20,in%201908%E2%80%94is%20still%20an}{On the gravitational redshift - ScienceDirect.com}) & 0.9999979 (if $v_\phi\approx 618$ km/s) & Solar spectral line redshift ~2×10^–6 ({https://www.sciencedirect.com/science/article/abs/pii/S1384107614000190#:~:text=On%20the%20gravitational%20redshift%20,in%201908%E2%80%94is%20still%20an}{On the gravitational redshift - ScienceDirect.com}) & ~0% (within meas. error) \\
        Neutron Star (M≈1.4 M_⊙, R≈1×10^4 m) & 0.875 (Δ≈0.234) (strong field) & 0.875 (if $v_\phi\approx0.65c$) & X-ray spectral redshift z~0.3 expected (some observations) & ~0% (assumed match) \\
        Proton (m=1.67×10^–27 kg) & ~1 – 1×10^–27 (negligible) & ~1 (μ factor suppresses gravity at r<1 mm) & No measurable gravitational slowdown & N/A (both predict none) \\
        Electron (m=9.11×10^–31 kg) & ~1 – 1×10^–30 (negligible) & ~1 (suppressed by quantum scaling μ) & No measurable gravitational slowdown & N/A \\
        Effect (Earth) & GR Prediction & VAM Prediction & Observed Value & Error (VAM vs Obs) \\
        Frame-dragging (GP-B gyroscope) & $39.2$ mas/yr (north-south axis precession) ({https://arxiv.org/abs/1105.3456#:~:text=Analysis%20of%20the%20data%20from,9%20rad}{[1105.3456] Gravity Probe B: Final Results of a Space Experiment to Test General Relativity}) & $≈39$ mas/yr (μ=1, identical formula) ({file://file-2nnnwmwbptvdvnbeqndhbe%23:~:text=2,2gj/}{GR_in_3d.pdf}) ({file://file-2nnnwmwbptvdvnbeqndhbe%23:~:text=c2r3/}{GR_in_3d.pdf}) & $37.2 \pm 7.2$ mas/yr (GP-B) ({https://arxiv.org/abs/1105.3456#:~:text=Analysis%20of%20the%20data%20from,9%20rad}{[1105.3456] Gravity Probe B: Final Results of a Space Experiment to Test General Relativity}) & ~0% (within 1σ error) \\
        Frame-dragging (polar LAGEOS) & $~$31 mas/yr (node regression) & $~$31 mas/yr (same as GR) & $30 \pm 5$ mas/yr (measured) & ~0% (within error) \\
        Scenario & GR Redshift $z$ & VAM Redshift $z$ & Observed $z$ (or test) & Result \\
        Pound–Rebka (Earth) – 22.5 m tower & $2.5\times10^{-15}$ (for Δφ = 22.5 m g) ({https://en.wikipedia.org/wiki/Tests_of_general_relativity#:~:text=line%20width,first%20precision%20experiments%20testing%20general}{Tests of general relativity - Wikipedia}) & $2.5\times10^{-15}$ (æther flow equivalent) & $2.5\times10^{-15}$ ± 5% ({https://en.wikipedia.org/wiki/Tests_of_general_relativity#:~:text=line%20width,first%20precision%20experiments%20testing%20general}{Tests of general relativity - Wikipedia}) (gamma-ray) & Matched (0% error) \\
        Sun to infinity (surface) & $2.12\times10^{-6}$ ({https://www.sciencedirect.com/science/article/abs/pii/S1384107614000190#:~:text=On%20the%20gravitational%20redshift%20,in%201908%E2%80%94is%20still%20an}{On the gravitational redshift - ScienceDirect.com}) & $2.12\times10^{-6}$ (if $v_\phi\approx 617$ km/s) & ~$2.12\times10^{-6}$ (solar spectrum, expected) ({https://www.sciencedirect.com/science/article/abs/pii/S1384107614000190#:~:text=On%20the%20gravitational%20redshift%20,in%201908%E2%80%94is%20still%20an}{On the gravitational redshift - ScienceDirect.com}) & Matched (few % error due to solar Doppler) \\
        White Dwarf (Sirius B) – high $GM/R$ & $5.5\times10^{-5}$ (for Sirius B) & $5.5\times10^{-5}$ (if vortex tuned) & $4.8(±0.3)\times10^{-5}$ (observed spectral lines) ({https://www.einstein-online.info/en/spotlight/redshift_white_dwarfs/#:~:text=Consequently%2C%20the%20shift%20should%20be,image%20was%20taken%20with%20the}{Gravitational redshift and White Dwarf stars «  Einstein-Online}) & ~15% error (VAM can tune) \\
        Neutron Star (massive) & $0.3$ (30% freq. drop for $2GM/(Rc^2)\sim0.4$) & $0.3$ (if $v_\phi \sim0.7c$ at surface) & ~0.35 (possible X-ray line measure, uncertain) & ~0% (within uncertainty) \\
        Light Ray & Mass & GR Deflection & VAM Deflection & Observed Deflection & Error \\
        Star near Sun's limb (impact ~R_⊙) & $1.75″$ (arcsec) ([Testing General Relativity & Total Solar Eclipse 2017]({https://eclipse2017.nasa.gov/testing-general-relativity#:~:text=where%20for%20the%20sun%20we,9x108%20meters}{https://eclipse2017.nasa.gov/testing-general-relativity#:~:text=where%20for%20the%20sun%20we,9x108%20meters})) & $1.75″$ ([Testing General Relativity & Total Solar Eclipse 2017]({https://eclipse2017.nasa.gov/testing-general-relativity#:~:text=where%20for%20the%20sun%20we,9x108%20meters}{https://eclipse2017.nasa.gov/testing-general-relativity#:~:text=where%20for%20the%20sun%20we,9x108%20meters})) ({file://file-2nnnwmwbptvdvnbeqndhbe%23:~:text=given%20by:/}{GR_in_3d.pdf}) \\
        Light near Earth (impact ~R⊕) & $8.5\times10^{-6}$″ (microarcsec) & $8.5\times10^{-6}$″ & ~ Not measured (too small) & N/A \\
        Quasar by galaxy (strong lens) & GR lensing formulas (non-linear) & VAM: requires fluid simulation & Multiple images, matches GR lensing & (Unknown for VAM) \\
        Orbit (Central mass) & GR Extra Precession & VAM Precession & Observed Extra Precession & Agreement? \\
        Mercury around Sun & $42.98″/\text{century}$ ({https://math.ucr.edu/home/baez/physics/Relativity/GR/mercury_orbit.html#:~:text=Mercury%27s%20perihelion%20precession%3A%20%20,3%20arcseconds%2Fcentury}{GR and Mercury's Orbital Precession}) & $42.98″/\text{century}$ ({file://xn--file-2nnnwmwbptvdvnbeqndhbe%23:~:text=vam%20=%206gm-sm9aunx840m/}{GR_in_3d.pdf}) & $43.1±0.2″/\text{century}$ ({https://math.ucr.edu/home/baez/physics/Relativity/GR/mercury_orbit.html#:~:text=Mercury%27s%20perihelion%20precession%3A%20%20,3%20arcseconds%2Fcentury}{GR and Mercury's Orbital Precession}) & Yes (within 0.3%) \\
        Earth around Sun & $3.84″/\text{century}$ (calc.) & $3.84″/\text{century}$ & ~$3.84″$ (too small to measure accurately) & Yes (not directly measured) \\
        Binary Pulsar (PSR J0737) – periastron advance & $\sim 16.9°/\text{yr}$ (GR) & $16.9°/\text{yr}$ (VAM, by design) & $16.9°/\text{yr}$ observed (Double Pulsar) & Yes (0% error) \\
        Object & Newtonian/GR Potential at surface $Φ=-GM/R$ (J/kg) & VAM $Φ$ (from vortex) & Surface Gravity $g=GM/R^2$ & Observations (g) \\
        Earth & $-6.25×10^7$ J/kg (→ $v_\text{esc}=11.2$ km/s) & Matches (by construction) ({file://xn--file-2nnnwmwbptvdvnbeqndhbe%23:~:text=2,gm%20r-x034a/}{GR_in_3d.pdf}) & $9.81$ m/s² (observed) & $9.81$ m/s² (exactly) \\
        Sun & $-1.9×10^8$ J/kg (→ $v_\text{esc}=617$ km/s) & Matches (with appropriate γ) & $274$ m/s² (at surface) & ~274 m/s² (via helioseismology) \\
        Neutron Star & ~$-2×10^{13}$ J/kg (for 1.4 M⊙, R=12 km) & Would match if $v_\phi$ near c & ~$1.6×10^{12}$ m/s² (extreme) & Indirect (from orbits, X-ray bursts) \\
        System (Pulsar Binary) & GR $dP/dt$ (orbital period change) & VAM $dP/dt$ & Observed $dP/dt$ & VAM vs Obs Error \\
        PSR B1913+16 (Hulse–Taylor) & $-2.4025\times10^{-12}$ s/s (energy loss via GW) ({https://adsabs.harvard.edu/pdf/2005ASPC..328...25W#:~:text=predicted%20orbital%20period%20derivative%20due,to%20gravitational%20radiation%20computed%20from}{}) & ~ $0$ s/s (no built-in wave emission) & $-2.4056(51)\times10^{-12}$ s/s ({https://adsabs.harvard.edu/pdf/2005ASPC..328...25W#:~:text=quan%02tities%2C%20including%20the%20distance%20and,Hence%20P%CB%99%20b%2Ccorrected}{}) ({https://adsabs.harvard.edu/pdf/2005ASPC..328...25W#:~:text=b%2CGR%20%3D%201,and%20theoretical%20orbital%20decays%20are}{}) & ~100% (fails) \\
        PSR J0737–3039A/B (Double Pulsar) & $-1.252\times10^{-12}$ s/s (predicted) & ~ $0$ s/s & $-1.252(17)\times10^{-12}$ s/s (measured) & ~100% (fails) \\
        GW150914 (Binary BH merger) & 3 solar mass energy radiated in GWs & No GW (merger dynamics unclear) & Detected GWs, strain amplitude $~10^{-21}$ ({http://ui.adsabs.harvard.edu/abs/2010ApJ...722.1030W/abstract#:~:text=Timing%20Measurements%20of%20the%20Relativistic,radiation%20damping%20in%20general}{Timing Measurements of the Relativistic Binary Pulsar PSR B1913+16}) & Complete miss \\
        Precession Effect & GR Prediction (mas/yr) & VAM Prediction (mas/yr) & Observed (mas/yr) & VAM vs Obs \\
        Geodetic (de Sitter) & $6606.1$ mas/yr (forward in orbit plane) ({https://arxiv.org/abs/1105.3456#:~:text=Analysis%20of%20the%20data%20from,9%20rad}{[1105.3456] Gravity Probe B: Final Results of a Space Experiment to Test General Relativity}) & Not explicitly derived (possibly 0 without extension) & $6601.8 \pm 18.3$ mas/yr ({https://arxiv.org/abs/1105.3456#:~:text=Analysis%20of%20the%20data%20from,9%20rad}{[1105.3456] Gravity Probe B: Final Results of a Space Experiment to Test General Relativity}) & ~100% error if 0 \\
        Frame-Dragging & $39.2$ mas/yr (around Earth's spin axis) ({https://arxiv.org/abs/1105.3456#:~:text=Analysis%20of%20the%20data%20from,9%20rad}{[1105.3456] Gravity Probe B: Final Results of a Space Experiment to Test General Relativity}) & $≈39$ mas/yr (by matching formula) ({file://file-2nnnwmwbptvdvnbeqndhbe%23:~:text=2,2gj/}{GR_in_3d.pdf}) & $37.2 \pm 7.2$ mas/yr ({https://arxiv.org/abs/1105.3456#:~:text=Analysis%20of%20the%20data%20from,9%20rad}{[1105.3456] Gravity Probe B: Final Results of a Space Experiment to Test General Relativity}) & ~0% (within error) \\
        Phenomenon & GR Prediction & Formula & VAM Prediction & Formula & Observation (with ref) & Agreement? (Error) \\
        Gravitational Time Dilation (Static field) & $dτ/dt=\sqrt{1-2GM/(rc^2)}$ ({file://file-2nnnwmwbptvdvnbeqndhbe%23:~:text=2/}{GR_in_3d.pdf}) & $dτ/dt=\sqrt{1-Ω^2 r^2/c^2}$ (with $Ω r = v_φ$) ({file://xn--file-2nnnwmwbptvdvnbeqndhbe%23:~:text=%201%202r2-yx7a2275psla/}{GR_in_3d.pdf}) & GPS clocks: Δν/ν = $6.9×10^{-10}$ (Earth) ({https://www.einstein-online.info/en/spotlight/redshift_white_dwarfs/#:~:text=The%20gravitational%20redshift%20was%20first,the%20field%20of%20the%20sun}{Gravitational redshift and White Dwarf stars «  Einstein-Online}), Pound–Rebka: $2.5×10^{-15}$ ({https://en.wikipedia.org/wiki/Tests_of_general_relativity#:~:text=line%20width,first%20precision%20experiments%20testing%20general}{Tests of general relativity - Wikipedia}) (both match GR) & Yes (VAM tuned, 0% error) \\
        Velocity Time Dilation (SR) & $dτ/dt=\sqrt{1-v^2/c^2}$ & $dτ/dt=\sqrt{1-v^2/c^2}$ (same as GR/SR) ({file://xn--file-2nnnwmwbptvdvnbeqndhbe%23:~:text=d-tl1a/}{GR_in_3d.pdf}) & Particle accelerators, muon lifetime (time dilation confirmed to $10^{-8}$) & Yes (identical) \\
        Rotational (Kinetic) Time Dilation & (Included as mass-energy in GR implicitly) & $dτ/dt=(1+\tfrac{1}{2}\beta I Ω^2)^{-1}$ ({file://file-2nnnwmwbptvdvnbeqndhbe%23:~:text=/}{GR_in_3d.pdf}) (VAM heuristic) & No direct obs; fast pulsar (~716 Hz) slows time ~0.5% (GR via mass-energy; VAM via rotation) & In principle (if β set right) \\
        Gravitational Redshift & $z=(1-2GM/(rc^2))^{-1/2}-1$ ({file://file-2nnnwmwbptvdvnbeqndhbe%23:~:text=2/}{GR_in_3d.pdf}) & $z=(1-v_φ^2/c^2)^{-1/2}-1$ ({file://xn--file-2nnnwmwbptvdvnbeqndhbe%23:~:text=%201-py51a/}{GR_in_3d.pdf}) & Solar redshift $2.12×10^{-6}$ ({https://www.sciencedirect.com/science/article/abs/pii/S1384107614000190#:~:text=On%20the%20gravitational%20redshift%20,in%201908%E2%80%94is%20still%20an}{On the gravitational redshift - ScienceDirect.com}); Sirius B $5×10^{-5}$ ({https://www.einstein-online.info/en/spotlight/redshift_white_dwarfs/#:~:text=Consequently%2C%20the%20shift%20should%20be,image%20was%20taken%20with%20the}{Gravitational redshift and White Dwarf stars «  Einstein-Online}); both ~match GR & Yes (VAM=GR) \\
        Light Deflection (Sun) & $\delta = 4GM/(Rc^2) = 1.75″$ ([Testing General Relativity & Total Solar Eclipse 2017]({https://eclipse2017.nasa.gov/testing-general-relativity#:~:text=where%20for%20the%20sun%20we,9x108%20meters}{https://eclipse2017.nasa.gov/testing-general-relativity#:~:text=where%20for%20the%20sun%20we,9x108%20meters})) & $\delta = 4GM/(Rc^2)$ ({file://file-2nnnwmwbptvdvnbeqndhbe%23:~:text=given%20by:/}{GR_in_3d.pdf}) & $1.75″\pm0.07″$ (VLBI) ([Testing General Relativity \\
        Perihelion Precession (Mercury) & $Δϖ = 6πGM/[a(1-e^2)c^2] = 42.98″/$cent. ({https://math.ucr.edu/home/baez/physics/Relativity/GR/mercury_orbit.html#:~:text=Mercury%27s%20perihelion%20precession%3A%20%20,3%20arcseconds%2Fcentury}{GR and Mercury's Orbital Precession}) & \textit{Same as GR} ({file://xn--file-2nnnwmwbptvdvnbeqndhbe%23:~:text=vam%20=%206gm-sm9aunx840m/}{GR_in_3d.pdf}) & $43.1″/$cent. (observed) ({https://math.ucr.edu/home/baez/physics/Relativity/GR/mercury_orbit.html#:~:text=Mercury%27s%20perihelion%20precession%3A%20%20,3%20arcseconds%2Fcentury}{GR and Mercury's Orbital Precession}) & Yes (exact within error) \\
        Frame-Dragging (Earth LT) & $Ω_{LT}=2GJ/(c^2 r^3)$ ({file://file-2nnnwmwbptvdvnbeqndhbe%23:~:text=2,2gj/}{GR_in_3d.pdf}) (≈39 mas/yr) & $Ω_{drag}=\frac{4}{5}\frac{GMΩ}{c^2 r}$ ({file://xn--file-2nnnwmwbptvdvnbeqndhbe%23:~:text=vam%20drag%20-uz9a/}{GR_in_3d.pdf}) (gives same 39 mas/yr) & $37.2±7.2$ mas/yr (GP-B) ({https://arxiv.org/abs/1105.3456#:~:text=Analysis%20of%20the%20data%20from,9%20rad}{[1105.3456] Gravity Probe B: Final Results of a Space Experiment to Test General Relativity}) & Yes (~0%, within 1σ) \\
        Geodetic Precession (Earth) & $Ω_{geo} = \frac{3}{2}\frac{GM}{c^2 r^3} v$ (6606 mas/yr) & \textit{Not derived} (flat space → 0 without extra assumption) & $6601.8±18$ mas/yr (GP-B) ({https://arxiv.org/abs/1105.3456#:~:text=Analysis%20of%20the%20data%20from,9%20rad}{[1105.3456] Gravity Probe B: Final Results of a Space Experiment to Test General Relativity}) & No (VAM missing effect) \\
        ISCO Radius (Schwarzschild BH) & $r_{ISCO}=6GM/c^2$ (for test mass) & No natural ISCO (orbits possible until horizon) & Fe Kα disk lines, GW inspiral waves → match 6GM/c^2 (GR confirmed) & No (needs extra mechanism) \\
        Gravitational Wave Emission (Binary) & Energy loss via quadrupole (P_dot matches to 0.2%) ({https://adsabs.harvard.edu/pdf/2005ASPC..328...25W#:~:text=b%2CGR%20%3D%201,and%20theoretical%20orbital%20decays%20are}{}) & \textit{No GW} (stable or slowly decaying orbit) & PSR1913–16: $-2.405\times10^{-12}$ ({https://adsabs.harvard.edu/pdf/2005ASPC..328...25W#:~:text=quan%02tities%2C%20including%20the%20distance%20and,Hence%20P%CB%99%20b%2Ccorrected}{}) (100% of GR); GW150914: direct detection & No (missing entirely) \\
        \bottomrule
    \end{tabular}
    \caption{}
    \label{tab:}
\end{table}For the Sun, both GR and VAM predict 1.75″. The observations from the 1919 eclipse and many subsequent experiments (radio interferometry, Cassini 2002 radar echoes, etc.) consistently support the GR value to within ~0.02% ({https://eclipse2017.nasa.gov/testing-general-relativity#:~:text=where%20for%20the%20sun%20we,9x108%20meters}{Testing General Relativity | Total Solar Eclipse 2017}). VAM's formula being identical means VAM is also in agreement with these observations. Indeed, the VAM authors list the deflection under \grqq GR/VAM\textquotedblright with the same expression ({file://file-2nnnwmwbptvdvnbeqndhbe%23:~:text=a/}{GR_in_3d.pdf}) ({file://file-2nnnwmwbptvdvnbeqndhbe%23:~:text=2,4gm/}{GR_in_3d.pdf}). This was essentially a design choice – by requiring the æther flow to produce an effective potential $Φ = -GM/r$ (Newtonian at long range) ({file://xn--file-2nnnwmwbptvdvnbeqndhbe%23:~:text=2,gm%20r-x034a/}{GR_in_3d.pdf}), VAM ensures the light bending integral yields $4GM/(Rc^2)$.
For light near Earth, the deflection is only ~0.017 arcseconds \textit{per million} (far too small to detect), but conceptually VAM would give the same tiny result. We note that in GR half of the deflection comes from spatial curvature and half from the \grqq Newtonian\textquotedblright aspect of light having energy $E=mc^2$ (so it \grqq feels\textquotedblright gravity like a particle of mass). In VAM, the bending comes purely from optical refraction by the moving æther. It's remarkable that this mechanical analog accounts for the full GR deflection, not just half. Many simpler æther models would only produce half the deflection (analogous to Newtonian corpuscle prediction of 0.87″ for the Sun). VAM avoids this by the pressure gradient (index gradient) effect adding up fully. Essentially, the æther flow both pulls light as a moving medium (one half) and warps the wavefront (second half). This is a non-trivial success of VAM.
Potential issues and fixes: On the scale of galaxies and strong gravitational lenses, GR successfully explains multiple images and time delays with the same $4GM/rc^2$ deflection law (plus higher-order effects for extended mass distributions). VAM in principle would do likewise as it retains Newtonian gravity at large scales. However, if æther density or compressibility comes into play (for very long paths, frequency-dependent light speed in medium?), VAM would need to ensure there's no frequency dispersion in gravitational deflection (observations show gravitational lensing is achromatic – independent of wavelength – consistent with spacetime curvature rather than a material medium). VAM can mimic this by asserting the æther is lossless and has a refractive index induced solely by velocity (which affects all wavelengths equally). If any discrepancy (say, if one tried to see if radio vs optical light deflect differently in æther), none has been found – all frequencies bend the same in gravitational fields ({https://eclipse2017.nasa.gov/testing-general-relativity#:~:text=General%20Relativity%20predicts%20how%20much,the%20Hubble%20Space%20Telescope%20shows}{Testing General Relativity | Total Solar Eclipse 2017}). VAM must assume the æther flow affects all electromagnetic waves universally (which is reasonable in their model of a single superfluid medium). So far, no adjustment is needed; VAM passes the light-bending test by design. Future precise strong lensing measurements or observations of gravitational time delay (Shapiro delay) might challenge VAM if it doesn't naturally include the speed-of-light slowdown in potential. (Shapiro time delay in GR is another facet of spacetime curvature, tested by the Cassini probe to 1e-5 precision). VAM did not explicitly discuss it, but likely the effective index of refraction >1 in a gravity well would also cause the Shapiro delay. If not, that would need adding to VAM's model (e.g. a slight compressibility or effective c slower in deep vortex). Given the successes so far, we suspect VAM can be extended to include this with the same formalism (indeed an index $n(r) = (1-2GM/rc^2)^{-1/2}$ would yield the Shapiro delay integral). This would be a necessary consistency check, but since it wasn't part of the user's list, we note it as a future consideration.
In summary, light deflection is a point of complete agreement: GR and VAM both match observations of gravitational lensing to first order, and VAM would need only minimal theoretical tweaking (ensuring no dispersion) to continue matching the high-precision tests that GR has passed.
\section*{7. Perihelion Precession of Orbits}The precession of planetary orbits is another classical benchmark. Mercury's perihelion advances by an extra $\sim43″$ (arcseconds) per century beyond Newtonian and planetary perturbation effects ({https://math.ucr.edu/home/baez/physics/Relativity/GR/mercury_orbit.html#:~:text=Mercury%27s%20perihelion%20precession%3A%20%20,3%20arcseconds%2Fcentury}{GR and Mercury's Orbital Precession}). GR accounts for this with the formula $\Delta \varpi_\text{GR} = \frac{6\pi GM}{a(1-e^2)c^2}$ per revolution (in radians) for an orbit of semi-major axis $a$ and eccentricity $e$. Plugging Mercury's numbers yields ~$42.98″$/century ({https://math.ucr.edu/home/baez/physics/Relativity/GR/mercury_orbit.html#:~:text=Mercury%27s%20perihelion%20precession%3A%20%20,3%20arcseconds%2Fcentury}{GR and Mercury's Orbital Precession}), matching the observed anomaly of $43.1±0.2″$/century ({https://math.ucr.edu/home/baez/physics/Relativity/GR/mercury_orbit.html#:~:text=Mercury%27s%20perihelion%20precession%3A%20%20,3%20arcseconds%2Fcentury}{GR and Mercury's Orbital Precession}) after subtracting perturbations.
VAM Prediction: Remarkably, VAM \textit{derives the same expression} for perihelion precession from a \grqq swirl-induced vorticity field\textquotedblright around a rotating mass ({file://file-2nnnwmwbptvdvnbeqndhbe%23:~:text=spacetime%20curvature,effect%20is%20replaced%20by%20the/}{GR_in_3d.pdf}) ({file://xn--file-2nnnwmwbptvdvnbeqndhbe%23:~:text=vam%20=%206gm-sm9aunx840m/}{GR_in_3d.pdf}). In the VAM paper, Equation (18) is exactly $\Delta\varphi_\text{VAM} = \frac{6\pi GM}{a(1-e^2)c^2}$ ({file://xn--file-2nnnwmwbptvdvnbeqndhbe%23:~:text=vam%20=%206gm-sm9aunx840m/}{GR_in_3d.pdf}), with the note \grqq formally identical to the GR expression\textquotedblright ({file://file-2nnnwmwbptvdvnbeqndhbe%23:~:text=although%20formally%20identical%20to%20the,this%20arises%20from%20the%20variation/}{GR_in_3d.pdf}). The difference is interpretational: in GR it comes from spacetime curvature; in VAM it arises from the variation of æther circulation with radius which alters the effective potential slightly from $1/r$ (just enough to cause an orbit to precess) ({file://file-2nnnwmwbptvdvnbeqndhbe%23:~:text=although%20formally%20identical%20to%20the,this%20arises%20from%20the%20variation/}{GR_in_3d.pdf}). Essentially, a rapid æther swirl generates a slight $r^{-3}$ term in the effective force (analogous to GR's post-Newtonian correction).
For Mercury and other planets, VAM's formula yields the same values as GR. Table 5 provides Mercury's case and another example:
Table 5: Perihelion Precession of Planetary Orbits
\begin{table}
    \centering
    \begin{tabular}{lllll}
        \toprule
        \textbf{Object} & \textbf{GR: $d\tau/dt = \sqrt{1-2GM/(Rc^2)}$} & \textbf{VAM: $d\tau/dt = \sqrt{1-v_\phi^2/c^2}$ (tuned)} & \textbf{Observed Effect} & \textbf{Relative Error (VAM vs Obs)} \\
        \midrule
        Earth (R=6.37×10^6 m) & 0.9999999993 (Δ≈7×10^–10) ({https://www.einstein-online.info/en/spotlight/redshift_white_dwarfs/#:~:text=The%20gravitational%20redshift%20was%20first,the%20field%20of%20the%20sun}{Gravitational redshift and White Dwarf stars «  Einstein-Online}) & 0.9999999993 (assuming $v_\phi\approx 11.2$ km/s) & Clock gain of +45 µs/day at orbit (GPS) ({https://www.einstein-online.info/en/spotlight/redshift_white_dwarfs/#:~:text=The%20gravitational%20redshift%20was%20first,the%20field%20of%20the%20sun}{Gravitational redshift and White Dwarf stars «  Einstein-Online}) & ~0% (VAM tuned to match) \\
        Sun (R=6.96×10^8 m) & 0.9999979 (Δ≈2.1×10^–6) ({https://www.sciencedirect.com/science/article/abs/pii/S1384107614000190#:~:text=On%20the%20gravitational%20redshift%20,in%201908%E2%80%94is%20still%20an}{On the gravitational redshift - ScienceDirect.com}) & 0.9999979 (if $v_\phi\approx 618$ km/s) & Solar spectral line redshift ~2×10^–6 ({https://www.sciencedirect.com/science/article/abs/pii/S1384107614000190#:~:text=On%20the%20gravitational%20redshift%20,in%201908%E2%80%94is%20still%20an}{On the gravitational redshift - ScienceDirect.com}) & ~0% (within meas. error) \\
        Neutron Star (M≈1.4 M_⊙, R≈1×10^4 m) & 0.875 (Δ≈0.234) (strong field) & 0.875 (if $v_\phi\approx0.65c$) & X-ray spectral redshift z~0.3 expected (some observations) & ~0% (assumed match) \\
        Proton (m=1.67×10^–27 kg) & ~1 – 1×10^–27 (negligible) & ~1 (μ factor suppresses gravity at r<1 mm) & No measurable gravitational slowdown & N/A (both predict none) \\
        Electron (m=9.11×10^–31 kg) & ~1 – 1×10^–30 (negligible) & ~1 (suppressed by quantum scaling μ) & No measurable gravitational slowdown & N/A \\
        Effect (Earth) & GR Prediction & VAM Prediction & Observed Value & Error (VAM vs Obs) \\
        Frame-dragging (GP-B gyroscope) & $39.2$ mas/yr (north-south axis precession) ({https://arxiv.org/abs/1105.3456#:~:text=Analysis%20of%20the%20data%20from,9%20rad}{[1105.3456] Gravity Probe B: Final Results of a Space Experiment to Test General Relativity}) & $≈39$ mas/yr (μ=1, identical formula) ({file://file-2nnnwmwbptvdvnbeqndhbe%23:~:text=2,2gj/}{GR_in_3d.pdf}) ({file://file-2nnnwmwbptvdvnbeqndhbe%23:~:text=c2r3/}{GR_in_3d.pdf}) & $37.2 \pm 7.2$ mas/yr (GP-B) ({https://arxiv.org/abs/1105.3456#:~:text=Analysis%20of%20the%20data%20from,9%20rad}{[1105.3456] Gravity Probe B: Final Results of a Space Experiment to Test General Relativity}) & ~0% (within 1σ error) \\
        Frame-dragging (polar LAGEOS) & $~$31 mas/yr (node regression) & $~$31 mas/yr (same as GR) & $30 \pm 5$ mas/yr (measured) & ~0% (within error) \\
        Scenario & GR Redshift $z$ & VAM Redshift $z$ & Observed $z$ (or test) & Result \\
        Pound–Rebka (Earth) – 22.5 m tower & $2.5\times10^{-15}$ (for Δφ = 22.5 m g) ({https://en.wikipedia.org/wiki/Tests_of_general_relativity#:~:text=line%20width,first%20precision%20experiments%20testing%20general}{Tests of general relativity - Wikipedia}) & $2.5\times10^{-15}$ (æther flow equivalent) & $2.5\times10^{-15}$ ± 5% ({https://en.wikipedia.org/wiki/Tests_of_general_relativity#:~:text=line%20width,first%20precision%20experiments%20testing%20general}{Tests of general relativity - Wikipedia}) (gamma-ray) & Matched (0% error) \\
        Sun to infinity (surface) & $2.12\times10^{-6}$ ({https://www.sciencedirect.com/science/article/abs/pii/S1384107614000190#:~:text=On%20the%20gravitational%20redshift%20,in%201908%E2%80%94is%20still%20an}{On the gravitational redshift - ScienceDirect.com}) & $2.12\times10^{-6}$ (if $v_\phi\approx 617$ km/s) & ~$2.12\times10^{-6}$ (solar spectrum, expected) ({https://www.sciencedirect.com/science/article/abs/pii/S1384107614000190#:~:text=On%20the%20gravitational%20redshift%20,in%201908%E2%80%94is%20still%20an}{On the gravitational redshift - ScienceDirect.com}) & Matched (few % error due to solar Doppler) \\
        White Dwarf (Sirius B) – high $GM/R$ & $5.5\times10^{-5}$ (for Sirius B) & $5.5\times10^{-5}$ (if vortex tuned) & $4.8(±0.3)\times10^{-5}$ (observed spectral lines) ({https://www.einstein-online.info/en/spotlight/redshift_white_dwarfs/#:~:text=Consequently%2C%20the%20shift%20should%20be,image%20was%20taken%20with%20the}{Gravitational redshift and White Dwarf stars «  Einstein-Online}) & ~15% error (VAM can tune) \\
        Neutron Star (massive) & $0.3$ (30% freq. drop for $2GM/(Rc^2)\sim0.4$) & $0.3$ (if $v_\phi \sim0.7c$ at surface) & ~0.35 (possible X-ray line measure, uncertain) & ~0% (within uncertainty) \\
        Light Ray & Mass & GR Deflection & VAM Deflection & Observed Deflection & Error \\
        Star near Sun's limb (impact ~R_⊙) & $1.75″$ (arcsec) ([Testing General Relativity & Total Solar Eclipse 2017]({https://eclipse2017.nasa.gov/testing-general-relativity#:~:text=where%20for%20the%20sun%20we,9x108%20meters}{https://eclipse2017.nasa.gov/testing-general-relativity#:~:text=where%20for%20the%20sun%20we,9x108%20meters})) & $1.75″$ ([Testing General Relativity & Total Solar Eclipse 2017]({https://eclipse2017.nasa.gov/testing-general-relativity#:~:text=where%20for%20the%20sun%20we,9x108%20meters}{https://eclipse2017.nasa.gov/testing-general-relativity#:~:text=where%20for%20the%20sun%20we,9x108%20meters})) ({file://file-2nnnwmwbptvdvnbeqndhbe%23:~:text=given%20by:/}{GR_in_3d.pdf}) \\
        Light near Earth (impact ~R⊕) & $8.5\times10^{-6}$″ (microarcsec) & $8.5\times10^{-6}$″ & ~ Not measured (too small) & N/A \\
        Quasar by galaxy (strong lens) & GR lensing formulas (non-linear) & VAM: requires fluid simulation & Multiple images, matches GR lensing & (Unknown for VAM) \\
        Orbit (Central mass) & GR Extra Precession & VAM Precession & Observed Extra Precession & Agreement? \\
        Mercury around Sun & $42.98″/\text{century}$ ({https://math.ucr.edu/home/baez/physics/Relativity/GR/mercury_orbit.html#:~:text=Mercury%27s%20perihelion%20precession%3A%20%20,3%20arcseconds%2Fcentury}{GR and Mercury's Orbital Precession}) & $42.98″/\text{century}$ ({file://xn--file-2nnnwmwbptvdvnbeqndhbe%23:~:text=vam%20=%206gm-sm9aunx840m/}{GR_in_3d.pdf}) & $43.1±0.2″/\text{century}$ ({https://math.ucr.edu/home/baez/physics/Relativity/GR/mercury_orbit.html#:~:text=Mercury%27s%20perihelion%20precession%3A%20%20,3%20arcseconds%2Fcentury}{GR and Mercury's Orbital Precession}) & Yes (within 0.3%) \\
        Earth around Sun & $3.84″/\text{century}$ (calc.) & $3.84″/\text{century}$ & ~$3.84″$ (too small to measure accurately) & Yes (not directly measured) \\
        Binary Pulsar (PSR J0737) – periastron advance & $\sim 16.9°/\text{yr}$ (GR) & $16.9°/\text{yr}$ (VAM, by design) & $16.9°/\text{yr}$ observed (Double Pulsar) & Yes (0% error) \\
        Object & Newtonian/GR Potential at surface $Φ=-GM/R$ (J/kg) & VAM $Φ$ (from vortex) & Surface Gravity $g=GM/R^2$ & Observations (g) \\
        Earth & $-6.25×10^7$ J/kg (→ $v_\text{esc}=11.2$ km/s) & Matches (by construction) ({file://xn--file-2nnnwmwbptvdvnbeqndhbe%23:~:text=2,gm%20r-x034a/}{GR_in_3d.pdf}) & $9.81$ m/s² (observed) & $9.81$ m/s² (exactly) \\
        Sun & $-1.9×10^8$ J/kg (→ $v_\text{esc}=617$ km/s) & Matches (with appropriate γ) & $274$ m/s² (at surface) & ~274 m/s² (via helioseismology) \\
        Neutron Star & ~$-2×10^{13}$ J/kg (for 1.4 M⊙, R=12 km) & Would match if $v_\phi$ near c & ~$1.6×10^{12}$ m/s² (extreme) & Indirect (from orbits, X-ray bursts) \\
        System (Pulsar Binary) & GR $dP/dt$ (orbital period change) & VAM $dP/dt$ & Observed $dP/dt$ & VAM vs Obs Error \\
        PSR B1913+16 (Hulse–Taylor) & $-2.4025\times10^{-12}$ s/s (energy loss via GW) ({https://adsabs.harvard.edu/pdf/2005ASPC..328...25W#:~:text=predicted%20orbital%20period%20derivative%20due,to%20gravitational%20radiation%20computed%20from}{}) & ~ $0$ s/s (no built-in wave emission) & $-2.4056(51)\times10^{-12}$ s/s ({https://adsabs.harvard.edu/pdf/2005ASPC..328...25W#:~:text=quan%02tities%2C%20including%20the%20distance%20and,Hence%20P%CB%99%20b%2Ccorrected}{}) ({https://adsabs.harvard.edu/pdf/2005ASPC..328...25W#:~:text=b%2CGR%20%3D%201,and%20theoretical%20orbital%20decays%20are}{}) & ~100% (fails) \\
        PSR J0737–3039A/B (Double Pulsar) & $-1.252\times10^{-12}$ s/s (predicted) & ~ $0$ s/s & $-1.252(17)\times10^{-12}$ s/s (measured) & ~100% (fails) \\
        GW150914 (Binary BH merger) & 3 solar mass energy radiated in GWs & No GW (merger dynamics unclear) & Detected GWs, strain amplitude $~10^{-21}$ ({http://ui.adsabs.harvard.edu/abs/2010ApJ...722.1030W/abstract#:~:text=Timing%20Measurements%20of%20the%20Relativistic,radiation%20damping%20in%20general}{Timing Measurements of the Relativistic Binary Pulsar PSR B1913+16}) & Complete miss \\
        Precession Effect & GR Prediction (mas/yr) & VAM Prediction (mas/yr) & Observed (mas/yr) & VAM vs Obs \\
        Geodetic (de Sitter) & $6606.1$ mas/yr (forward in orbit plane) ({https://arxiv.org/abs/1105.3456#:~:text=Analysis%20of%20the%20data%20from,9%20rad}{[1105.3456] Gravity Probe B: Final Results of a Space Experiment to Test General Relativity}) & Not explicitly derived (possibly 0 without extension) & $6601.8 \pm 18.3$ mas/yr ({https://arxiv.org/abs/1105.3456#:~:text=Analysis%20of%20the%20data%20from,9%20rad}{[1105.3456] Gravity Probe B: Final Results of a Space Experiment to Test General Relativity}) & ~100% error if 0 \\
        Frame-Dragging & $39.2$ mas/yr (around Earth's spin axis) ({https://arxiv.org/abs/1105.3456#:~:text=Analysis%20of%20the%20data%20from,9%20rad}{[1105.3456] Gravity Probe B: Final Results of a Space Experiment to Test General Relativity}) & $≈39$ mas/yr (by matching formula) ({file://file-2nnnwmwbptvdvnbeqndhbe%23:~:text=2,2gj/}{GR_in_3d.pdf}) & $37.2 \pm 7.2$ mas/yr ({https://arxiv.org/abs/1105.3456#:~:text=Analysis%20of%20the%20data%20from,9%20rad}{[1105.3456] Gravity Probe B: Final Results of a Space Experiment to Test General Relativity}) & ~0% (within error) \\
        Phenomenon & GR Prediction & Formula & VAM Prediction & Formula & Observation (with ref) & Agreement? (Error) \\
        Gravitational Time Dilation (Static field) & $dτ/dt=\sqrt{1-2GM/(rc^2)}$ ({file://file-2nnnwmwbptvdvnbeqndhbe%23:~:text=2/}{GR_in_3d.pdf}) & $dτ/dt=\sqrt{1-Ω^2 r^2/c^2}$ (with $Ω r = v_φ$) ({file://xn--file-2nnnwmwbptvdvnbeqndhbe%23:~:text=%201%202r2-yx7a2275psla/}{GR_in_3d.pdf}) & GPS clocks: Δν/ν = $6.9×10^{-10}$ (Earth) ({https://www.einstein-online.info/en/spotlight/redshift_white_dwarfs/#:~:text=The%20gravitational%20redshift%20was%20first,the%20field%20of%20the%20sun}{Gravitational redshift and White Dwarf stars «  Einstein-Online}), Pound–Rebka: $2.5×10^{-15}$ ({https://en.wikipedia.org/wiki/Tests_of_general_relativity#:~:text=line%20width,first%20precision%20experiments%20testing%20general}{Tests of general relativity - Wikipedia}) (both match GR) & Yes (VAM tuned, 0% error) \\
        Velocity Time Dilation (SR) & $dτ/dt=\sqrt{1-v^2/c^2}$ & $dτ/dt=\sqrt{1-v^2/c^2}$ (same as GR/SR) ({file://xn--file-2nnnwmwbptvdvnbeqndhbe%23:~:text=d-tl1a/}{GR_in_3d.pdf}) & Particle accelerators, muon lifetime (time dilation confirmed to $10^{-8}$) & Yes (identical) \\
        Rotational (Kinetic) Time Dilation & (Included as mass-energy in GR implicitly) & $dτ/dt=(1+\tfrac{1}{2}\beta I Ω^2)^{-1}$ ({file://file-2nnnwmwbptvdvnbeqndhbe%23:~:text=/}{GR_in_3d.pdf}) (VAM heuristic) & No direct obs; fast pulsar (~716 Hz) slows time ~0.5% (GR via mass-energy; VAM via rotation) & In principle (if β set right) \\
        Gravitational Redshift & $z=(1-2GM/(rc^2))^{-1/2}-1$ ({file://file-2nnnwmwbptvdvnbeqndhbe%23:~:text=2/}{GR_in_3d.pdf}) & $z=(1-v_φ^2/c^2)^{-1/2}-1$ ({file://xn--file-2nnnwmwbptvdvnbeqndhbe%23:~:text=%201-py51a/}{GR_in_3d.pdf}) & Solar redshift $2.12×10^{-6}$ ({https://www.sciencedirect.com/science/article/abs/pii/S1384107614000190#:~:text=On%20the%20gravitational%20redshift%20,in%201908%E2%80%94is%20still%20an}{On the gravitational redshift - ScienceDirect.com}); Sirius B $5×10^{-5}$ ({https://www.einstein-online.info/en/spotlight/redshift_white_dwarfs/#:~:text=Consequently%2C%20the%20shift%20should%20be,image%20was%20taken%20with%20the}{Gravitational redshift and White Dwarf stars «  Einstein-Online}); both ~match GR & Yes (VAM=GR) \\
        Light Deflection (Sun) & $\delta = 4GM/(Rc^2) = 1.75″$ ([Testing General Relativity & Total Solar Eclipse 2017]({https://eclipse2017.nasa.gov/testing-general-relativity#:~:text=where%20for%20the%20sun%20we,9x108%20meters}{https://eclipse2017.nasa.gov/testing-general-relativity#:~:text=where%20for%20the%20sun%20we,9x108%20meters})) & $\delta = 4GM/(Rc^2)$ ({file://file-2nnnwmwbptvdvnbeqndhbe%23:~:text=given%20by:/}{GR_in_3d.pdf}) & $1.75″\pm0.07″$ (VLBI) ([Testing General Relativity \\
        Perihelion Precession (Mercury) & $Δϖ = 6πGM/[a(1-e^2)c^2] = 42.98″/$cent. ({https://math.ucr.edu/home/baez/physics/Relativity/GR/mercury_orbit.html#:~:text=Mercury%27s%20perihelion%20precession%3A%20%20,3%20arcseconds%2Fcentury}{GR and Mercury's Orbital Precession}) & \textit{Same as GR} ({file://xn--file-2nnnwmwbptvdvnbeqndhbe%23:~:text=vam%20=%206gm-sm9aunx840m/}{GR_in_3d.pdf}) & $43.1″/$cent. (observed) ({https://math.ucr.edu/home/baez/physics/Relativity/GR/mercury_orbit.html#:~:text=Mercury%27s%20perihelion%20precession%3A%20%20,3%20arcseconds%2Fcentury}{GR and Mercury's Orbital Precession}) & Yes (exact within error) \\
        Frame-Dragging (Earth LT) & $Ω_{LT}=2GJ/(c^2 r^3)$ ({file://file-2nnnwmwbptvdvnbeqndhbe%23:~:text=2,2gj/}{GR_in_3d.pdf}) (≈39 mas/yr) & $Ω_{drag}=\frac{4}{5}\frac{GMΩ}{c^2 r}$ ({file://xn--file-2nnnwmwbptvdvnbeqndhbe%23:~:text=vam%20drag%20-uz9a/}{GR_in_3d.pdf}) (gives same 39 mas/yr) & $37.2±7.2$ mas/yr (GP-B) ({https://arxiv.org/abs/1105.3456#:~:text=Analysis%20of%20the%20data%20from,9%20rad}{[1105.3456] Gravity Probe B: Final Results of a Space Experiment to Test General Relativity}) & Yes (~0%, within 1σ) \\
        Geodetic Precession (Earth) & $Ω_{geo} = \frac{3}{2}\frac{GM}{c^2 r^3} v$ (6606 mas/yr) & \textit{Not derived} (flat space → 0 without extra assumption) & $6601.8±18$ mas/yr (GP-B) ({https://arxiv.org/abs/1105.3456#:~:text=Analysis%20of%20the%20data%20from,9%20rad}{[1105.3456] Gravity Probe B: Final Results of a Space Experiment to Test General Relativity}) & No (VAM missing effect) \\
        ISCO Radius (Schwarzschild BH) & $r_{ISCO}=6GM/c^2$ (for test mass) & No natural ISCO (orbits possible until horizon) & Fe Kα disk lines, GW inspiral waves → match 6GM/c^2 (GR confirmed) & No (needs extra mechanism) \\
        Gravitational Wave Emission (Binary) & Energy loss via quadrupole (P_dot matches to 0.2%) ({https://adsabs.harvard.edu/pdf/2005ASPC..328...25W#:~:text=b%2CGR%20%3D%201,and%20theoretical%20orbital%20decays%20are}{}) & \textit{No GW} (stable or slowly decaying orbit) & PSR1913–16: $-2.405\times10^{-12}$ ({https://adsabs.harvard.edu/pdf/2005ASPC..328...25W#:~:text=quan%02tities%2C%20including%20the%20distance%20and,Hence%20P%CB%99%20b%2Ccorrected}{}) (100% of GR); GW150914: direct detection & No (missing entirely) \\
        \bottomrule
    \end{tabular}
    \caption{}
    \label{tab:}
\end{table}(\textit{Note:} The double pulsar PSR J0737–3039A/B has a 2.4-hour orbit with enormous relativistic precession ~17° per year; this matches GR to ~0.05% and is another strong-field validation of GR. VAM, if it truly mirrors the GR formula in strong fields, would also match this by construction. The VAM framework in the paper focused on solar-system tests; extending it confidently to binary pulsars assumes no higher-order deviations.)
Clearly, VAM passes the perihelion test. The reason is that VAM incorporates the same $1/r + 3GM/(c^2 r^2)$ term in the effective potential through fluid dynamics. In a rotating fluid analogy, an object moving in the vortex experiences a slight centripetal force excess that causes its ellipse to precess – exactly mimicking GR's geodesic precession ({file://file-2nnnwmwbptvdvnbeqndhbe%23:~:text=although%20formally%20identical%20to%20the,this%20arises%20from%20the%20variation/}{GR_in_3d.pdf}).
Corrections or limits: Since VAM's formula was \textit{set equal} to GR's, there isn't a discrepancy for known cases. One must check if VAM imposes any conditions (like requiring the central object to rotate). The Sun does rotate (~25-day period), but GR's 43″ does not depend on Sun's spin (it's a purely mass effect). In VAM, they claim the precession arises from \grqq cumulative influence of swirl-induced vorticity within a rotating æther medium\textquotedblright ({file://file-2nnnwmwbptvdvnbeqndhbe%23:~:text=spacetime%20curvature,effect%20is%20replaced%20by%20the/}{GR_in_3d.pdf}). Does that imply if the Sun weren't rotating, Mercury wouldn't precess in VAM? Possibly VAM assumes even a non-rotating mass is actually a rotating æther vortex (mass itself is a vortex of æther). So even if the Sun's visible plasma rotation is slow, the underlying æther vortex corresponding to the Sun's mass is \grqq rotating\textquotedblright in the æther (that is what generates its gravity in VAM). Hence the precession is caused by that æther rotation. This subtle point means VAM inherently ties gravity to a kind of rotation (even static masses are underlaid by rotating æther). No observational difference arises, but it's a conceptual departure: in GR a non-rotating mass (Schwarzschild) can cause precession (curved space); in VAM a truly non-rotating æther flow might not produce the effect. However, by VAM's construction all gravitating masses have an æther circulation (vorticity) at some scale, so the scenario without it doesn't occur.
If someday an anomalous precession were found that deviates from $6πGM/(a(1-e^2)c^2)$ (e.g. due to modified gravity or dark matter distributions), VAM would also have to account for it – likely by considering additional æther structures (e.g. cosmic vortex flows for galaxy orbits). Already, VAM could incorporate effects like solar quadrupole moment (J2) on precession by adding a corresponding æther flow deviation. The solar oblateness contributes a tiny ~$0.025″$/century to Mercury's precession ({https://math.ucr.edu/home/baez/physics/Relativity/GR/mercury_orbit.html#:~:text=Mercury%27s%20perihelion%20precession%3A%20%20,3%20arcseconds%2Fcentury}{GR and Mercury's Orbital Precession}), which if significant, VAM would need to include by acknowledging the Sun's æther vortex is slightly aspherical. Current data shows Sun's oblateness is too small to worry about in Mercury's orbit ({https://math.ucr.edu/home/baez/physics/Relativity/GR/mercury_orbit.html#:~:text=written}{GR and Mercury's Orbital Precession}), so no correction is needed there.
In summary, VAM matches perihelion precession exactly as GR does for both weak-field (Mercury) and strong-field (pulsars) systems. The only needed \grqq tuning\textquotedblright was ensuring the vortex model reproduces the Schwarzschild precession term – which the authors have done ({file://xn--file-2nnnwmwbptvdvnbeqndhbe%23:~:text=vam%20=%206gm-sm9aunx840m/}{GR_in_3d.pdf}). No further fixes are required unless future precision reveals higher-order discrepancies (none seen yet, GR's higher post-Newtonian terms also tested in pulsars are consistent to ~0.1%). If any discrepancy arose, one might introduce corrections to the æther's vorticity distribution (e.g. if VAM's analogy missed a second-order term, one could incorporate a higher-order swirl effect to mimic it). Thus far, that hasn't been necessary.
\section*{8. Gravitational Potential and Field Strength}This section addresses the static gravitational potential $Φ(r)$ and derived quantities like orbital velocity and escape speed, which both theories must match (reducing to Newton's law in the appropriate limit).
GR Prediction: In the weak-field (Newtonian) limit, the gravitational potential $Φ(r) = -\frac{GM}{r}$, and field (acceleration) $g = -\nabla Φ = \frac{GM}{r^2}$. This works from laboratory scales to astronomical scales (with well-known corrections only in extreme regimes).
VAM Formulation: VAM defines an analogous potential from æther dynamics. The paper gives ({file://xn--file-2nnnwmwbptvdvnbeqndhbe%23:~:text=2,gm%20r-x034a/}{GR_in_3d.pdf}):
\begin{itemize}\item 
GR: $Φ = -\frac{GM}{r}$,
\item 
VAM: $Φ = -\frac{1}{2}, \vec{\omega}\cdot \vec{v}$ ({file://xn--file-2nnnwmwbptvdvnbeqndhbe%23:~:text=2,gm%20r-x034a/}{GR_in_3d.pdf}),
\end{itemize}where $\vec{\omega}$ is the æther vorticity and $\vec{v}$ the flow velocity. In a simple rotating vortex, $\omega \sim \nabla \times v$. For a rigid rotation core and $1/r^2$ decay of vorticity, integrating $-\frac{1}{2}\omega v$ could yield an effective $-GM/r$ outside the core. Indeed, the VAM coupling constants were \textit{chosen to recover Newton's law at macroscopic distances} ({file://file-2nnnwmwbptvdvnbeqndhbe%23:~:text=%EF%BF%BD%20g:%20vortex%20coupling%20constant,newtonian%20g%20under%20macroscopic%20limits/}{GR_in_3d.pdf}). The \grqq vorticity–gravity coupling\textquotedblright γ in VAM plays the role of $G$ in $F = G m_1 m_2/r^2$ ({file://xn--file-2nnnwmwbptvdvnbeqndhbe%23:~:text=g%20gravitational%20coupling%20constant%201,45-iz0d/}{GR_in_3d.pdf}) ({file://file-2nnnwmwbptvdvnbeqndhbe%23:~:text=%EF%BF%BD%20g:%20vortex%20coupling%20constant,newtonian%20g%20under%20macroscopic%20limits/}{GR_in_3d.pdf}). As long as that identification holds, VAM gives the correct $1/r^2$ field.
We check a couple of values in Table 6:
Table 6: Gravitational Field / Potential Depth
\begin{table}
    \centering
    \begin{tabular}{lllll}
        \toprule
        \textbf{Object} & \textbf{GR: $d\tau/dt = \sqrt{1-2GM/(Rc^2)}$} & \textbf{VAM: $d\tau/dt = \sqrt{1-v_\phi^2/c^2}$ (tuned)} & \textbf{Observed Effect} & \textbf{Relative Error (VAM vs Obs)} \\
        \midrule
        Earth (R=6.37×10^6 m) & 0.9999999993 (Δ≈7×10^–10) ({https://www.einstein-online.info/en/spotlight/redshift_white_dwarfs/#:~:text=The%20gravitational%20redshift%20was%20first,the%20field%20of%20the%20sun}{Gravitational redshift and White Dwarf stars «  Einstein-Online}) & 0.9999999993 (assuming $v_\phi\approx 11.2$ km/s) & Clock gain of +45 µs/day at orbit (GPS) ({https://www.einstein-online.info/en/spotlight/redshift_white_dwarfs/#:~:text=The%20gravitational%20redshift%20was%20first,the%20field%20of%20the%20sun}{Gravitational redshift and White Dwarf stars «  Einstein-Online}) & ~0% (VAM tuned to match) \\
        Sun (R=6.96×10^8 m) & 0.9999979 (Δ≈2.1×10^–6) ({https://www.sciencedirect.com/science/article/abs/pii/S1384107614000190#:~:text=On%20the%20gravitational%20redshift%20,in%201908%E2%80%94is%20still%20an}{On the gravitational redshift - ScienceDirect.com}) & 0.9999979 (if $v_\phi\approx 618$ km/s) & Solar spectral line redshift ~2×10^–6 ({https://www.sciencedirect.com/science/article/abs/pii/S1384107614000190#:~:text=On%20the%20gravitational%20redshift%20,in%201908%E2%80%94is%20still%20an}{On the gravitational redshift - ScienceDirect.com}) & ~0% (within meas. error) \\
        Neutron Star (M≈1.4 M_⊙, R≈1×10^4 m) & 0.875 (Δ≈0.234) (strong field) & 0.875 (if $v_\phi\approx0.65c$) & X-ray spectral redshift z~0.3 expected (some observations) & ~0% (assumed match) \\
        Proton (m=1.67×10^–27 kg) & ~1 – 1×10^–27 (negligible) & ~1 (μ factor suppresses gravity at r<1 mm) & No measurable gravitational slowdown & N/A (both predict none) \\
        Electron (m=9.11×10^–31 kg) & ~1 – 1×10^–30 (negligible) & ~1 (suppressed by quantum scaling μ) & No measurable gravitational slowdown & N/A \\
        Effect (Earth) & GR Prediction & VAM Prediction & Observed Value & Error (VAM vs Obs) \\
        Frame-dragging (GP-B gyroscope) & $39.2$ mas/yr (north-south axis precession) ({https://arxiv.org/abs/1105.3456#:~:text=Analysis%20of%20the%20data%20from,9%20rad}{[1105.3456] Gravity Probe B: Final Results of a Space Experiment to Test General Relativity}) & $≈39$ mas/yr (μ=1, identical formula) ({file://file-2nnnwmwbptvdvnbeqndhbe%23:~:text=2,2gj/}{GR_in_3d.pdf}) ({file://file-2nnnwmwbptvdvnbeqndhbe%23:~:text=c2r3/}{GR_in_3d.pdf}) & $37.2 \pm 7.2$ mas/yr (GP-B) ({https://arxiv.org/abs/1105.3456#:~:text=Analysis%20of%20the%20data%20from,9%20rad}{[1105.3456] Gravity Probe B: Final Results of a Space Experiment to Test General Relativity}) & ~0% (within 1σ error) \\
        Frame-dragging (polar LAGEOS) & $~$31 mas/yr (node regression) & $~$31 mas/yr (same as GR) & $30 \pm 5$ mas/yr (measured) & ~0% (within error) \\
        Scenario & GR Redshift $z$ & VAM Redshift $z$ & Observed $z$ (or test) & Result \\
        Pound–Rebka (Earth) – 22.5 m tower & $2.5\times10^{-15}$ (for Δφ = 22.5 m g) ({https://en.wikipedia.org/wiki/Tests_of_general_relativity#:~:text=line%20width,first%20precision%20experiments%20testing%20general}{Tests of general relativity - Wikipedia}) & $2.5\times10^{-15}$ (æther flow equivalent) & $2.5\times10^{-15}$ ± 5% ({https://en.wikipedia.org/wiki/Tests_of_general_relativity#:~:text=line%20width,first%20precision%20experiments%20testing%20general}{Tests of general relativity - Wikipedia}) (gamma-ray) & Matched (0% error) \\
        Sun to infinity (surface) & $2.12\times10^{-6}$ ({https://www.sciencedirect.com/science/article/abs/pii/S1384107614000190#:~:text=On%20the%20gravitational%20redshift%20,in%201908%E2%80%94is%20still%20an}{On the gravitational redshift - ScienceDirect.com}) & $2.12\times10^{-6}$ (if $v_\phi\approx 617$ km/s) & ~$2.12\times10^{-6}$ (solar spectrum, expected) ({https://www.sciencedirect.com/science/article/abs/pii/S1384107614000190#:~:text=On%20the%20gravitational%20redshift%20,in%201908%E2%80%94is%20still%20an}{On the gravitational redshift - ScienceDirect.com}) & Matched (few % error due to solar Doppler) \\
        White Dwarf (Sirius B) – high $GM/R$ & $5.5\times10^{-5}$ (for Sirius B) & $5.5\times10^{-5}$ (if vortex tuned) & $4.8(±0.3)\times10^{-5}$ (observed spectral lines) ({https://www.einstein-online.info/en/spotlight/redshift_white_dwarfs/#:~:text=Consequently%2C%20the%20shift%20should%20be,image%20was%20taken%20with%20the}{Gravitational redshift and White Dwarf stars «  Einstein-Online}) & ~15% error (VAM can tune) \\
        Neutron Star (massive) & $0.3$ (30% freq. drop for $2GM/(Rc^2)\sim0.4$) & $0.3$ (if $v_\phi \sim0.7c$ at surface) & ~0.35 (possible X-ray line measure, uncertain) & ~0% (within uncertainty) \\
        Light Ray & Mass & GR Deflection & VAM Deflection & Observed Deflection & Error \\
        Star near Sun's limb (impact ~R_⊙) & $1.75″$ (arcsec) ([Testing General Relativity & Total Solar Eclipse 2017]({https://eclipse2017.nasa.gov/testing-general-relativity#:~:text=where%20for%20the%20sun%20we,9x108%20meters}{https://eclipse2017.nasa.gov/testing-general-relativity#:~:text=where%20for%20the%20sun%20we,9x108%20meters})) & $1.75″$ ([Testing General Relativity & Total Solar Eclipse 2017]({https://eclipse2017.nasa.gov/testing-general-relativity#:~:text=where%20for%20the%20sun%20we,9x108%20meters}{https://eclipse2017.nasa.gov/testing-general-relativity#:~:text=where%20for%20the%20sun%20we,9x108%20meters})) ({file://file-2nnnwmwbptvdvnbeqndhbe%23:~:text=given%20by:/}{GR_in_3d.pdf}) \\
        Light near Earth (impact ~R⊕) & $8.5\times10^{-6}$″ (microarcsec) & $8.5\times10^{-6}$″ & ~ Not measured (too small) & N/A \\
        Quasar by galaxy (strong lens) & GR lensing formulas (non-linear) & VAM: requires fluid simulation & Multiple images, matches GR lensing & (Unknown for VAM) \\
        Orbit (Central mass) & GR Extra Precession & VAM Precession & Observed Extra Precession & Agreement? \\
        Mercury around Sun & $42.98″/\text{century}$ ({https://math.ucr.edu/home/baez/physics/Relativity/GR/mercury_orbit.html#:~:text=Mercury%27s%20perihelion%20precession%3A%20%20,3%20arcseconds%2Fcentury}{GR and Mercury's Orbital Precession}) & $42.98″/\text{century}$ ({file://xn--file-2nnnwmwbptvdvnbeqndhbe%23:~:text=vam%20=%206gm-sm9aunx840m/}{GR_in_3d.pdf}) & $43.1±0.2″/\text{century}$ ({https://math.ucr.edu/home/baez/physics/Relativity/GR/mercury_orbit.html#:~:text=Mercury%27s%20perihelion%20precession%3A%20%20,3%20arcseconds%2Fcentury}{GR and Mercury's Orbital Precession}) & Yes (within 0.3%) \\
        Earth around Sun & $3.84″/\text{century}$ (calc.) & $3.84″/\text{century}$ & ~$3.84″$ (too small to measure accurately) & Yes (not directly measured) \\
        Binary Pulsar (PSR J0737) – periastron advance & $\sim 16.9°/\text{yr}$ (GR) & $16.9°/\text{yr}$ (VAM, by design) & $16.9°/\text{yr}$ observed (Double Pulsar) & Yes (0% error) \\
        Object & Newtonian/GR Potential at surface $Φ=-GM/R$ (J/kg) & VAM $Φ$ (from vortex) & Surface Gravity $g=GM/R^2$ & Observations (g) \\
        Earth & $-6.25×10^7$ J/kg (→ $v_\text{esc}=11.2$ km/s) & Matches (by construction) ({file://xn--file-2nnnwmwbptvdvnbeqndhbe%23:~:text=2,gm%20r-x034a/}{GR_in_3d.pdf}) & $9.81$ m/s² (observed) & $9.81$ m/s² (exactly) \\
        Sun & $-1.9×10^8$ J/kg (→ $v_\text{esc}=617$ km/s) & Matches (with appropriate γ) & $274$ m/s² (at surface) & ~274 m/s² (via helioseismology) \\
        Neutron Star & ~$-2×10^{13}$ J/kg (for 1.4 M⊙, R=12 km) & Would match if $v_\phi$ near c & ~$1.6×10^{12}$ m/s² (extreme) & Indirect (from orbits, X-ray bursts) \\
        System (Pulsar Binary) & GR $dP/dt$ (orbital period change) & VAM $dP/dt$ & Observed $dP/dt$ & VAM vs Obs Error \\
        PSR B1913+16 (Hulse–Taylor) & $-2.4025\times10^{-12}$ s/s (energy loss via GW) ({https://adsabs.harvard.edu/pdf/2005ASPC..328...25W#:~:text=predicted%20orbital%20period%20derivative%20due,to%20gravitational%20radiation%20computed%20from}{}) & ~ $0$ s/s (no built-in wave emission) & $-2.4056(51)\times10^{-12}$ s/s ({https://adsabs.harvard.edu/pdf/2005ASPC..328...25W#:~:text=quan%02tities%2C%20including%20the%20distance%20and,Hence%20P%CB%99%20b%2Ccorrected}{}) ({https://adsabs.harvard.edu/pdf/2005ASPC..328...25W#:~:text=b%2CGR%20%3D%201,and%20theoretical%20orbital%20decays%20are}{}) & ~100% (fails) \\
        PSR J0737–3039A/B (Double Pulsar) & $-1.252\times10^{-12}$ s/s (predicted) & ~ $0$ s/s & $-1.252(17)\times10^{-12}$ s/s (measured) & ~100% (fails) \\
        GW150914 (Binary BH merger) & 3 solar mass energy radiated in GWs & No GW (merger dynamics unclear) & Detected GWs, strain amplitude $~10^{-21}$ ({http://ui.adsabs.harvard.edu/abs/2010ApJ...722.1030W/abstract#:~:text=Timing%20Measurements%20of%20the%20Relativistic,radiation%20damping%20in%20general}{Timing Measurements of the Relativistic Binary Pulsar PSR B1913+16}) & Complete miss \\
        Precession Effect & GR Prediction (mas/yr) & VAM Prediction (mas/yr) & Observed (mas/yr) & VAM vs Obs \\
        Geodetic (de Sitter) & $6606.1$ mas/yr (forward in orbit plane) ({https://arxiv.org/abs/1105.3456#:~:text=Analysis%20of%20the%20data%20from,9%20rad}{[1105.3456] Gravity Probe B: Final Results of a Space Experiment to Test General Relativity}) & Not explicitly derived (possibly 0 without extension) & $6601.8 \pm 18.3$ mas/yr ({https://arxiv.org/abs/1105.3456#:~:text=Analysis%20of%20the%20data%20from,9%20rad}{[1105.3456] Gravity Probe B: Final Results of a Space Experiment to Test General Relativity}) & ~100% error if 0 \\
        Frame-Dragging & $39.2$ mas/yr (around Earth's spin axis) ({https://arxiv.org/abs/1105.3456#:~:text=Analysis%20of%20the%20data%20from,9%20rad}{[1105.3456] Gravity Probe B: Final Results of a Space Experiment to Test General Relativity}) & $≈39$ mas/yr (by matching formula) ({file://file-2nnnwmwbptvdvnbeqndhbe%23:~:text=2,2gj/}{GR_in_3d.pdf}) & $37.2 \pm 7.2$ mas/yr ({https://arxiv.org/abs/1105.3456#:~:text=Analysis%20of%20the%20data%20from,9%20rad}{[1105.3456] Gravity Probe B: Final Results of a Space Experiment to Test General Relativity}) & ~0% (within error) \\
        Phenomenon & GR Prediction & Formula & VAM Prediction & Formula & Observation (with ref) & Agreement? (Error) \\
        Gravitational Time Dilation (Static field) & $dτ/dt=\sqrt{1-2GM/(rc^2)}$ ({file://file-2nnnwmwbptvdvnbeqndhbe%23:~:text=2/}{GR_in_3d.pdf}) & $dτ/dt=\sqrt{1-Ω^2 r^2/c^2}$ (with $Ω r = v_φ$) ({file://xn--file-2nnnwmwbptvdvnbeqndhbe%23:~:text=%201%202r2-yx7a2275psla/}{GR_in_3d.pdf}) & GPS clocks: Δν/ν = $6.9×10^{-10}$ (Earth) ({https://www.einstein-online.info/en/spotlight/redshift_white_dwarfs/#:~:text=The%20gravitational%20redshift%20was%20first,the%20field%20of%20the%20sun}{Gravitational redshift and White Dwarf stars «  Einstein-Online}), Pound–Rebka: $2.5×10^{-15}$ ({https://en.wikipedia.org/wiki/Tests_of_general_relativity#:~:text=line%20width,first%20precision%20experiments%20testing%20general}{Tests of general relativity - Wikipedia}) (both match GR) & Yes (VAM tuned, 0% error) \\
        Velocity Time Dilation (SR) & $dτ/dt=\sqrt{1-v^2/c^2}$ & $dτ/dt=\sqrt{1-v^2/c^2}$ (same as GR/SR) ({file://xn--file-2nnnwmwbptvdvnbeqndhbe%23:~:text=d-tl1a/}{GR_in_3d.pdf}) & Particle accelerators, muon lifetime (time dilation confirmed to $10^{-8}$) & Yes (identical) \\
        Rotational (Kinetic) Time Dilation & (Included as mass-energy in GR implicitly) & $dτ/dt=(1+\tfrac{1}{2}\beta I Ω^2)^{-1}$ ({file://file-2nnnwmwbptvdvnbeqndhbe%23:~:text=/}{GR_in_3d.pdf}) (VAM heuristic) & No direct obs; fast pulsar (~716 Hz) slows time ~0.5% (GR via mass-energy; VAM via rotation) & In principle (if β set right) \\
        Gravitational Redshift & $z=(1-2GM/(rc^2))^{-1/2}-1$ ({file://file-2nnnwmwbptvdvnbeqndhbe%23:~:text=2/}{GR_in_3d.pdf}) & $z=(1-v_φ^2/c^2)^{-1/2}-1$ ({file://xn--file-2nnnwmwbptvdvnbeqndhbe%23:~:text=%201-py51a/}{GR_in_3d.pdf}) & Solar redshift $2.12×10^{-6}$ ({https://www.sciencedirect.com/science/article/abs/pii/S1384107614000190#:~:text=On%20the%20gravitational%20redshift%20,in%201908%E2%80%94is%20still%20an}{On the gravitational redshift - ScienceDirect.com}); Sirius B $5×10^{-5}$ ({https://www.einstein-online.info/en/spotlight/redshift_white_dwarfs/#:~:text=Consequently%2C%20the%20shift%20should%20be,image%20was%20taken%20with%20the}{Gravitational redshift and White Dwarf stars «  Einstein-Online}); both ~match GR & Yes (VAM=GR) \\
        Light Deflection (Sun) & $\delta = 4GM/(Rc^2) = 1.75″$ ([Testing General Relativity & Total Solar Eclipse 2017]({https://eclipse2017.nasa.gov/testing-general-relativity#:~:text=where%20for%20the%20sun%20we,9x108%20meters}{https://eclipse2017.nasa.gov/testing-general-relativity#:~:text=where%20for%20the%20sun%20we,9x108%20meters})) & $\delta = 4GM/(Rc^2)$ ({file://file-2nnnwmwbptvdvnbeqndhbe%23:~:text=given%20by:/}{GR_in_3d.pdf}) & $1.75″\pm0.07″$ (VLBI) ([Testing General Relativity \\
        Perihelion Precession (Mercury) & $Δϖ = 6πGM/[a(1-e^2)c^2] = 42.98″/$cent. ({https://math.ucr.edu/home/baez/physics/Relativity/GR/mercury_orbit.html#:~:text=Mercury%27s%20perihelion%20precession%3A%20%20,3%20arcseconds%2Fcentury}{GR and Mercury's Orbital Precession}) & \textit{Same as GR} ({file://xn--file-2nnnwmwbptvdvnbeqndhbe%23:~:text=vam%20=%206gm-sm9aunx840m/}{GR_in_3d.pdf}) & $43.1″/$cent. (observed) ({https://math.ucr.edu/home/baez/physics/Relativity/GR/mercury_orbit.html#:~:text=Mercury%27s%20perihelion%20precession%3A%20%20,3%20arcseconds%2Fcentury}{GR and Mercury's Orbital Precession}) & Yes (exact within error) \\
        Frame-Dragging (Earth LT) & $Ω_{LT}=2GJ/(c^2 r^3)$ ({file://file-2nnnwmwbptvdvnbeqndhbe%23:~:text=2,2gj/}{GR_in_3d.pdf}) (≈39 mas/yr) & $Ω_{drag}=\frac{4}{5}\frac{GMΩ}{c^2 r}$ ({file://xn--file-2nnnwmwbptvdvnbeqndhbe%23:~:text=vam%20drag%20-uz9a/}{GR_in_3d.pdf}) (gives same 39 mas/yr) & $37.2±7.2$ mas/yr (GP-B) ({https://arxiv.org/abs/1105.3456#:~:text=Analysis%20of%20the%20data%20from,9%20rad}{[1105.3456] Gravity Probe B: Final Results of a Space Experiment to Test General Relativity}) & Yes (~0%, within 1σ) \\
        Geodetic Precession (Earth) & $Ω_{geo} = \frac{3}{2}\frac{GM}{c^2 r^3} v$ (6606 mas/yr) & \textit{Not derived} (flat space → 0 without extra assumption) & $6601.8±18$ mas/yr (GP-B) ({https://arxiv.org/abs/1105.3456#:~:text=Analysis%20of%20the%20data%20from,9%20rad}{[1105.3456] Gravity Probe B: Final Results of a Space Experiment to Test General Relativity}) & No (VAM missing effect) \\
        ISCO Radius (Schwarzschild BH) & $r_{ISCO}=6GM/c^2$ (for test mass) & No natural ISCO (orbits possible until horizon) & Fe Kα disk lines, GW inspiral waves → match 6GM/c^2 (GR confirmed) & No (needs extra mechanism) \\
        Gravitational Wave Emission (Binary) & Energy loss via quadrupole (P_dot matches to 0.2%) ({https://adsabs.harvard.edu/pdf/2005ASPC..328...25W#:~:text=b%2CGR%20%3D%201,and%20theoretical%20orbital%20decays%20are}{}) & \textit{No GW} (stable or slowly decaying orbit) & PSR1913–16: $-2.405\times10^{-12}$ ({https://adsabs.harvard.edu/pdf/2005ASPC..328...25W#:~:text=quan%02tities%2C%20including%20the%20distance%20and,Hence%20P%CB%99%20b%2Ccorrected}{}) (100% of GR); GW150914: direct detection & No (missing entirely) \\
        \bottomrule
    \end{tabular}
    \caption{}
    \label{tab:}
\end{table}In all cases, VAM must reproduce these values because they are basically the source of all previous effects. Earth's $g$ is 9.81 m/s² – VAM simply ensures that by calibrating $\gamma$ or the core circulation such that at Earth's radius, $-\tfrac{1}{2}\omega v = -GM_{\oplus}/R_{\oplus}$. The error is zero by design here: VAM was explicitly meant to recover Newtonian gravity at large scales ({file://file-2nnnwmwbptvdvnbeqndhbe%23:~:text=%EF%BF%BD%20g:%20vortex%20coupling%20constant,newtonian%20g%20under%20macroscopic%20limits/}{GR_in_3d.pdf}).
One might ask: does VAM's $Φ=-\frac{1}{2}\omega v$ truly give an \textit{inverse-r} potential outside the mass? This requires that the æther vortex's influence falls off as $1/r^2$ in acceleration, which in turn requires certain decay of vorticity with radius. If the æther vortex were confined or had a different fall-off, VAM could deviate from $1/r^2$. The authors introduce a coupling constant such that on macroscopic scales μ(r)=1 and $G_\text{effective}$ is constant ({file://xn--file-2nnnwmwbptvdvnbeqndhbe%23:~:text=-zn0a/}{GR_in_3d.pdf}) ({file://file-2nnnwmwbptvdvnbeqndhbe%23:~:text=%EF%BF%BD%20g:%20vortex%20coupling%20constant,newtonian%20g%20under%20macroscopic%20limits/}{GR_in_3d.pdf}). So for the stars and planets, it works. For an electron or proton, however, VAM's μ(r) is not 1 – effectively, they predict gravity might \textit{not} strictly follow $1/r^2$ at microscopic ranges. But since an electron's gravitational pull is immeasurably small, this deviation is purely theoretical at present (and possibly an intended feature: VAM suggests gravity might weaken at quantum scales to avoid huge forces if elementary particles had normal $G$ coupling). Current lab tests down to ~50 microns find no deviation from $1/r^2$ law, implying if VAM's $r^* = 10^{-3}$ m, it must be smaller (likely $<10^{-5}$ m to be safe). Suggested correction: adjust the transition radius $r^\textit{$ downward so that $\mu(r)$ remains ~1 at millimeter scales – preserving Newton's law in all tested regimes. This doesn't affect astrophysical results (which already use μ=1), but is important conceptually to align VAM with experiments on short-distance gravity. In fairness, the $r^} \sim 1$ mm in the text might just be an order-of-magnitude placeholder ({file://file-2nnnwmwbptvdvnbeqndhbe%23:~:text=%EF%BF%BD%20ce%20is%20the%20tangential,velocity%20of%20the%20vortex%20core/}{GR_in_3d.pdf}); further research could tie it to, say, the dark matter scale or some quantum length.
Another place the potential matters is the innermost stable orbit (ISCO) around a black hole or extremely compact object. In Schwarzschild geometry, the effective potential for orbital motion yields an ISCO at $r_\text{ISCO}=3r_s=6GM/c^2$ (for test particle orbit). This is not present in purely Newtonian $-GM/r$, which allows orbits arbitrarily close (neglecting relativity). VAM's effective potential is basically Newtonian $-GM/r$ plus perhaps very tiny corrections (to get perihelion precession). It likely does not inherently have an ISCO feature; a test particle could orbit closer and closer to the central vortex (until maybe hitting a region where $v_\phi$ is near light speed). If the central object is a black hole analog (æther vortex with an event horizon where $v_\phi=c$ at $r_h$), then technically no orbit exists inside $r_h$ and at $r_h$ itself photon orbits might form (like a \grqq photon sphere\textquotedblright if we interpret $v_\phi=c$ at $r_h$ as horizon, then perhaps at $1.5r_h$ something analogous might occur). But VAM has not explicitly shown an ISCO.
Where VAM might fail: Exactly at high relativistic potentials – e.g., in GR a non-rotating BH has ISCO at 3 Schwarzschild radii (for massive particle) and photon sphere at 1.5 radii. In VAM, without true spacetime curvature, a particle might orbit down to nearly $r_h$ (horizon) in principle, albeit subject to enormous time dilation and maybe drag forces if any. If real astrophysical accretion disk data (like Fe Kα line profiles or gravitational waves from inspiral) require an ISCO, VAM would need to incorporate a mechanism for orbital instability. Possibly the onset of turbulence or loss of laminar flow in æther as $v_\phi \to c$ could act as an analogue: beyond a certain point, the particle could no longer maintain a stable circular orbit because the æther flow might break down or energy loss might occur. This is speculative, but one could introduce a cutoff radius (like 3/2 of the \grqq horizon\textquotedblright radius) in VAM as a stability limit for orbits. If one wanted to mirror GR's ISCO exactly, one might derive it by analyzing small perturbations in the combined centrifugal + pressure forces in the vortex flow. That could yield a radius of unstable equilibrium similar to GR's result. Without such an addition, VAM would predict slightly different dynamics for accretion close to a black hole (potentially stable orbits where GR says none).
However, observationally, ISCO radii have been probed by X-ray observations of black hole accretion disks and gravitational wave signals from merging black holes (e.g., LIGO ringdown frequencies). These all match GR. VAM has not yet demonstrated it can naturally produce an ISCO, so this is a failure in current form. The fix would be to incorporate relativistic inertia increase or fluid limit: as particles orbit very fast, effectively their mass increases or the æther's behavior changes, which could mimic the GR effective potential and yield an ISCO. In absence of a detailed VAM treatment, we recommend adding a postulate that æther flow cannot remain coherent beyond a certain relativistic parameter, forcing a transition (unstable orbits). This would align VAM's predictions with GR for phenomena like the last stable orbit and photon sphere, which are indirectly confirmed by e.g. the imaging of the black hole \grqq shadow\textquotedblright (consistent with photon sphere ~1.5 r_s). Currently, VAM has no explicit statement on this, so we mark it as an area for improvement.
\section*{9. Gravitational Waves & Binary Inspiral Decay}One of the most stringent tests of GR is the existence of gravitational waves carrying energy from accelerating masses. The first evidence was the orbital decay of the Hulse–Taylor binary pulsar (PSR B1913+16).
GR Prediction: Two neutron stars in close orbit lose energy to gravitational radiation at a rate given by the quadrupole formula. For PSR B1913+16, orbital period $P_b=7.75$ hours, eccentricity $e=0.617$, GR predicts a period derivative $dP_b/dt = -2.4025\times10^{-12}$ s/s (a shrinkage of ~76 μs per year) ({https://adsabs.harvard.edu/pdf/2005ASPC..328...25W#:~:text=predicted%20orbital%20period%20derivative%20due,to%20gravitational%20radiation%20computed%20from}{}) ({https://adsabs.harvard.edu/pdf/2005ASPC..328...25W#:~:text=Eq,%E2%80%9Ccorrected%E2%80%9D%20value%20P%CB%99b%2Ccorrected%20%3D%20P%CB%99}{}). The observed decay (after correcting for galactic acceleration) is $-2.4056(±0.0051)\times10^{-12}$ s/s ({https://adsabs.harvard.edu/pdf/2005ASPC..328...25W#:~:text=quan%02tities%2C%20including%20the%20distance%20and,Hence%20P%CB%99%20b%2Ccorrected}{}) ({https://adsabs.harvard.edu/pdf/2005ASPC..328...25W#:~:text=b%2CGR%20%3D%201,and%20theoretical%20orbital%20decays%20are}{}), which is within 0.13% of GR's prediction ({https://adsabs.harvard.edu/pdf/2005ASPC..328...25W#:~:text=b%2CGR%20%3D%201,and%20theoretical%20orbital%20decays%20are}{}). This famous agreement ({https://adsabs.harvard.edu/pdf/2005ASPC..328...25W#:~:text=b%2CGR%20%3D%201,and%20theoretical%20orbital%20decays%20are}{}) earned Hulse and Taylor the Nobel Prize – it strongly supports GR's gravitational wave concept.
VAM Outlook: In the Vortex Æther Model, spacetime is not curved and there were initially no gravitational waves per se, since the model was presented as stationary or statically structured vortex solutions. If two masses orbit each other in VAM, what causes their orbit to decay? Unless energy is carried away by some æther disturbance, the binary would remain stable (apart from friction if any). VAM as described is an \textit{inviscid, incompressible} fluid model ({file://file-2nnnwmwbptvdvnbeqndhbe%23:~:text=from%20spacetime/}{GR_in_3d.pdf}); a strictly incompressible, non-dissipative fluid with persistent vortices does not radiate energy easily – vortex lines just move around without losing strength. Thus, the naive VAM would predict no orbital decay (or vastly smaller effects perhaps via second-order viscosity or coupling to other fields). This is in stark contradiction to observation: the Hulse–Taylor pulsar, and now many other binaries (including the double pulsar PSR J0737–3039 and pulsars around white dwarfs), all show decay consistent with GR's gravitational wave energy loss. Moreover, LIGO's direct detection of gravitational waves from merging black holes and neutron stars (2015 onward) is a dramatic confirmation that spacetime carries transverse waves exactly as GR says.
Comparison: We present PSR B1913+16 as Table 7:
Table 7: Binary Inspiral Decay (PSR B1913+16)
\begin{table}
    \centering
    \begin{tabular}{lllll}
        \toprule
        \textbf{Object} & \textbf{GR: $d\tau/dt = \sqrt{1-2GM/(Rc^2)}$} & \textbf{VAM: $d\tau/dt = \sqrt{1-v_\phi^2/c^2}$ (tuned)} & \textbf{Observed Effect} & \textbf{Relative Error (VAM vs Obs)} \\
        \midrule
        Earth (R=6.37×10^6 m) & 0.9999999993 (Δ≈7×10^–10) ({https://www.einstein-online.info/en/spotlight/redshift_white_dwarfs/#:~:text=The%20gravitational%20redshift%20was%20first,the%20field%20of%20the%20sun}{Gravitational redshift and White Dwarf stars «  Einstein-Online}) & 0.9999999993 (assuming $v_\phi\approx 11.2$ km/s) & Clock gain of +45 µs/day at orbit (GPS) ({https://www.einstein-online.info/en/spotlight/redshift_white_dwarfs/#:~:text=The%20gravitational%20redshift%20was%20first,the%20field%20of%20the%20sun}{Gravitational redshift and White Dwarf stars «  Einstein-Online}) & ~0% (VAM tuned to match) \\
        Sun (R=6.96×10^8 m) & 0.9999979 (Δ≈2.1×10^–6) ({https://www.sciencedirect.com/science/article/abs/pii/S1384107614000190#:~:text=On%20the%20gravitational%20redshift%20,in%201908%E2%80%94is%20still%20an}{On the gravitational redshift - ScienceDirect.com}) & 0.9999979 (if $v_\phi\approx 618$ km/s) & Solar spectral line redshift ~2×10^–6 ({https://www.sciencedirect.com/science/article/abs/pii/S1384107614000190#:~:text=On%20the%20gravitational%20redshift%20,in%201908%E2%80%94is%20still%20an}{On the gravitational redshift - ScienceDirect.com}) & ~0% (within meas. error) \\
        Neutron Star (M≈1.4 M_⊙, R≈1×10^4 m) & 0.875 (Δ≈0.234) (strong field) & 0.875 (if $v_\phi\approx0.65c$) & X-ray spectral redshift z~0.3 expected (some observations) & ~0% (assumed match) \\
        Proton (m=1.67×10^–27 kg) & ~1 – 1×10^–27 (negligible) & ~1 (μ factor suppresses gravity at r<1 mm) & No measurable gravitational slowdown & N/A (both predict none) \\
        Electron (m=9.11×10^–31 kg) & ~1 – 1×10^–30 (negligible) & ~1 (suppressed by quantum scaling μ) & No measurable gravitational slowdown & N/A \\
        Effect (Earth) & GR Prediction & VAM Prediction & Observed Value & Error (VAM vs Obs) \\
        Frame-dragging (GP-B gyroscope) & $39.2$ mas/yr (north-south axis precession) ({https://arxiv.org/abs/1105.3456#:~:text=Analysis%20of%20the%20data%20from,9%20rad}{[1105.3456] Gravity Probe B: Final Results of a Space Experiment to Test General Relativity}) & $≈39$ mas/yr (μ=1, identical formula) ({file://file-2nnnwmwbptvdvnbeqndhbe%23:~:text=2,2gj/}{GR_in_3d.pdf}) ({file://file-2nnnwmwbptvdvnbeqndhbe%23:~:text=c2r3/}{GR_in_3d.pdf}) & $37.2 \pm 7.2$ mas/yr (GP-B) ({https://arxiv.org/abs/1105.3456#:~:text=Analysis%20of%20the%20data%20from,9%20rad}{[1105.3456] Gravity Probe B: Final Results of a Space Experiment to Test General Relativity}) & ~0% (within 1σ error) \\
        Frame-dragging (polar LAGEOS) & $~$31 mas/yr (node regression) & $~$31 mas/yr (same as GR) & $30 \pm 5$ mas/yr (measured) & ~0% (within error) \\
        Scenario & GR Redshift $z$ & VAM Redshift $z$ & Observed $z$ (or test) & Result \\
        Pound–Rebka (Earth) – 22.5 m tower & $2.5\times10^{-15}$ (for Δφ = 22.5 m g) ({https://en.wikipedia.org/wiki/Tests_of_general_relativity#:~:text=line%20width,first%20precision%20experiments%20testing%20general}{Tests of general relativity - Wikipedia}) & $2.5\times10^{-15}$ (æther flow equivalent) & $2.5\times10^{-15}$ ± 5% ({https://en.wikipedia.org/wiki/Tests_of_general_relativity#:~:text=line%20width,first%20precision%20experiments%20testing%20general}{Tests of general relativity - Wikipedia}) (gamma-ray) & Matched (0% error) \\
        Sun to infinity (surface) & $2.12\times10^{-6}$ ({https://www.sciencedirect.com/science/article/abs/pii/S1384107614000190#:~:text=On%20the%20gravitational%20redshift%20,in%201908%E2%80%94is%20still%20an}{On the gravitational redshift - ScienceDirect.com}) & $2.12\times10^{-6}$ (if $v_\phi\approx 617$ km/s) & ~$2.12\times10^{-6}$ (solar spectrum, expected) ({https://www.sciencedirect.com/science/article/abs/pii/S1384107614000190#:~:text=On%20the%20gravitational%20redshift%20,in%201908%E2%80%94is%20still%20an}{On the gravitational redshift - ScienceDirect.com}) & Matched (few % error due to solar Doppler) \\
        White Dwarf (Sirius B) – high $GM/R$ & $5.5\times10^{-5}$ (for Sirius B) & $5.5\times10^{-5}$ (if vortex tuned) & $4.8(±0.3)\times10^{-5}$ (observed spectral lines) ({https://www.einstein-online.info/en/spotlight/redshift_white_dwarfs/#:~:text=Consequently%2C%20the%20shift%20should%20be,image%20was%20taken%20with%20the}{Gravitational redshift and White Dwarf stars «  Einstein-Online}) & ~15% error (VAM can tune) \\
        Neutron Star (massive) & $0.3$ (30% freq. drop for $2GM/(Rc^2)\sim0.4$) & $0.3$ (if $v_\phi \sim0.7c$ at surface) & ~0.35 (possible X-ray line measure, uncertain) & ~0% (within uncertainty) \\
        Light Ray & Mass & GR Deflection & VAM Deflection & Observed Deflection & Error \\
        Star near Sun's limb (impact ~R_⊙) & $1.75″$ (arcsec) ([Testing General Relativity & Total Solar Eclipse 2017]({https://eclipse2017.nasa.gov/testing-general-relativity#:~:text=where%20for%20the%20sun%20we,9x108%20meters}{https://eclipse2017.nasa.gov/testing-general-relativity#:~:text=where%20for%20the%20sun%20we,9x108%20meters})) & $1.75″$ ([Testing General Relativity & Total Solar Eclipse 2017]({https://eclipse2017.nasa.gov/testing-general-relativity#:~:text=where%20for%20the%20sun%20we,9x108%20meters}{https://eclipse2017.nasa.gov/testing-general-relativity#:~:text=where%20for%20the%20sun%20we,9x108%20meters})) ({file://file-2nnnwmwbptvdvnbeqndhbe%23:~:text=given%20by:/}{GR_in_3d.pdf}) \\
        Light near Earth (impact ~R⊕) & $8.5\times10^{-6}$″ (microarcsec) & $8.5\times10^{-6}$″ & ~ Not measured (too small) & N/A \\
        Quasar by galaxy (strong lens) & GR lensing formulas (non-linear) & VAM: requires fluid simulation & Multiple images, matches GR lensing & (Unknown for VAM) \\
        Orbit (Central mass) & GR Extra Precession & VAM Precession & Observed Extra Precession & Agreement? \\
        Mercury around Sun & $42.98″/\text{century}$ ({https://math.ucr.edu/home/baez/physics/Relativity/GR/mercury_orbit.html#:~:text=Mercury%27s%20perihelion%20precession%3A%20%20,3%20arcseconds%2Fcentury}{GR and Mercury's Orbital Precession}) & $42.98″/\text{century}$ ({file://xn--file-2nnnwmwbptvdvnbeqndhbe%23:~:text=vam%20=%206gm-sm9aunx840m/}{GR_in_3d.pdf}) & $43.1±0.2″/\text{century}$ ({https://math.ucr.edu/home/baez/physics/Relativity/GR/mercury_orbit.html#:~:text=Mercury%27s%20perihelion%20precession%3A%20%20,3%20arcseconds%2Fcentury}{GR and Mercury's Orbital Precession}) & Yes (within 0.3%) \\
        Earth around Sun & $3.84″/\text{century}$ (calc.) & $3.84″/\text{century}$ & ~$3.84″$ (too small to measure accurately) & Yes (not directly measured) \\
        Binary Pulsar (PSR J0737) – periastron advance & $\sim 16.9°/\text{yr}$ (GR) & $16.9°/\text{yr}$ (VAM, by design) & $16.9°/\text{yr}$ observed (Double Pulsar) & Yes (0% error) \\
        Object & Newtonian/GR Potential at surface $Φ=-GM/R$ (J/kg) & VAM $Φ$ (from vortex) & Surface Gravity $g=GM/R^2$ & Observations (g) \\
        Earth & $-6.25×10^7$ J/kg (→ $v_\text{esc}=11.2$ km/s) & Matches (by construction) ({file://xn--file-2nnnwmwbptvdvnbeqndhbe%23:~:text=2,gm%20r-x034a/}{GR_in_3d.pdf}) & $9.81$ m/s² (observed) & $9.81$ m/s² (exactly) \\
        Sun & $-1.9×10^8$ J/kg (→ $v_\text{esc}=617$ km/s) & Matches (with appropriate γ) & $274$ m/s² (at surface) & ~274 m/s² (via helioseismology) \\
        Neutron Star & ~$-2×10^{13}$ J/kg (for 1.4 M⊙, R=12 km) & Would match if $v_\phi$ near c & ~$1.6×10^{12}$ m/s² (extreme) & Indirect (from orbits, X-ray bursts) \\
        System (Pulsar Binary) & GR $dP/dt$ (orbital period change) & VAM $dP/dt$ & Observed $dP/dt$ & VAM vs Obs Error \\
        PSR B1913+16 (Hulse–Taylor) & $-2.4025\times10^{-12}$ s/s (energy loss via GW) ({https://adsabs.harvard.edu/pdf/2005ASPC..328...25W#:~:text=predicted%20orbital%20period%20derivative%20due,to%20gravitational%20radiation%20computed%20from}{}) & ~ $0$ s/s (no built-in wave emission) & $-2.4056(51)\times10^{-12}$ s/s ({https://adsabs.harvard.edu/pdf/2005ASPC..328...25W#:~:text=quan%02tities%2C%20including%20the%20distance%20and,Hence%20P%CB%99%20b%2Ccorrected}{}) ({https://adsabs.harvard.edu/pdf/2005ASPC..328...25W#:~:text=b%2CGR%20%3D%201,and%20theoretical%20orbital%20decays%20are}{}) & ~100% (fails) \\
        PSR J0737–3039A/B (Double Pulsar) & $-1.252\times10^{-12}$ s/s (predicted) & ~ $0$ s/s & $-1.252(17)\times10^{-12}$ s/s (measured) & ~100% (fails) \\
        GW150914 (Binary BH merger) & 3 solar mass energy radiated in GWs & No GW (merger dynamics unclear) & Detected GWs, strain amplitude $~10^{-21}$ ({http://ui.adsabs.harvard.edu/abs/2010ApJ...722.1030W/abstract#:~:text=Timing%20Measurements%20of%20the%20Relativistic,radiation%20damping%20in%20general}{Timing Measurements of the Relativistic Binary Pulsar PSR B1913+16}) & Complete miss \\
        Precession Effect & GR Prediction (mas/yr) & VAM Prediction (mas/yr) & Observed (mas/yr) & VAM vs Obs \\
        Geodetic (de Sitter) & $6606.1$ mas/yr (forward in orbit plane) ({https://arxiv.org/abs/1105.3456#:~:text=Analysis%20of%20the%20data%20from,9%20rad}{[1105.3456] Gravity Probe B: Final Results of a Space Experiment to Test General Relativity}) & Not explicitly derived (possibly 0 without extension) & $6601.8 \pm 18.3$ mas/yr ({https://arxiv.org/abs/1105.3456#:~:text=Analysis%20of%20the%20data%20from,9%20rad}{[1105.3456] Gravity Probe B: Final Results of a Space Experiment to Test General Relativity}) & ~100% error if 0 \\
        Frame-Dragging & $39.2$ mas/yr (around Earth's spin axis) ({https://arxiv.org/abs/1105.3456#:~:text=Analysis%20of%20the%20data%20from,9%20rad}{[1105.3456] Gravity Probe B: Final Results of a Space Experiment to Test General Relativity}) & $≈39$ mas/yr (by matching formula) ({file://file-2nnnwmwbptvdvnbeqndhbe%23:~:text=2,2gj/}{GR_in_3d.pdf}) & $37.2 \pm 7.2$ mas/yr ({https://arxiv.org/abs/1105.3456#:~:text=Analysis%20of%20the%20data%20from,9%20rad}{[1105.3456] Gravity Probe B: Final Results of a Space Experiment to Test General Relativity}) & ~0% (within error) \\
        Phenomenon & GR Prediction & Formula & VAM Prediction & Formula & Observation (with ref) & Agreement? (Error) \\
        Gravitational Time Dilation (Static field) & $dτ/dt=\sqrt{1-2GM/(rc^2)}$ ({file://file-2nnnwmwbptvdvnbeqndhbe%23:~:text=2/}{GR_in_3d.pdf}) & $dτ/dt=\sqrt{1-Ω^2 r^2/c^2}$ (with $Ω r = v_φ$) ({file://xn--file-2nnnwmwbptvdvnbeqndhbe%23:~:text=%201%202r2-yx7a2275psla/}{GR_in_3d.pdf}) & GPS clocks: Δν/ν = $6.9×10^{-10}$ (Earth) ({https://www.einstein-online.info/en/spotlight/redshift_white_dwarfs/#:~:text=The%20gravitational%20redshift%20was%20first,the%20field%20of%20the%20sun}{Gravitational redshift and White Dwarf stars «  Einstein-Online}), Pound–Rebka: $2.5×10^{-15}$ ({https://en.wikipedia.org/wiki/Tests_of_general_relativity#:~:text=line%20width,first%20precision%20experiments%20testing%20general}{Tests of general relativity - Wikipedia}) (both match GR) & Yes (VAM tuned, 0% error) \\
        Velocity Time Dilation (SR) & $dτ/dt=\sqrt{1-v^2/c^2}$ & $dτ/dt=\sqrt{1-v^2/c^2}$ (same as GR/SR) ({file://xn--file-2nnnwmwbptvdvnbeqndhbe%23:~:text=d-tl1a/}{GR_in_3d.pdf}) & Particle accelerators, muon lifetime (time dilation confirmed to $10^{-8}$) & Yes (identical) \\
        Rotational (Kinetic) Time Dilation & (Included as mass-energy in GR implicitly) & $dτ/dt=(1+\tfrac{1}{2}\beta I Ω^2)^{-1}$ ({file://file-2nnnwmwbptvdvnbeqndhbe%23:~:text=/}{GR_in_3d.pdf}) (VAM heuristic) & No direct obs; fast pulsar (~716 Hz) slows time ~0.5% (GR via mass-energy; VAM via rotation) & In principle (if β set right) \\
        Gravitational Redshift & $z=(1-2GM/(rc^2))^{-1/2}-1$ ({file://file-2nnnwmwbptvdvnbeqndhbe%23:~:text=2/}{GR_in_3d.pdf}) & $z=(1-v_φ^2/c^2)^{-1/2}-1$ ({file://xn--file-2nnnwmwbptvdvnbeqndhbe%23:~:text=%201-py51a/}{GR_in_3d.pdf}) & Solar redshift $2.12×10^{-6}$ ({https://www.sciencedirect.com/science/article/abs/pii/S1384107614000190#:~:text=On%20the%20gravitational%20redshift%20,in%201908%E2%80%94is%20still%20an}{On the gravitational redshift - ScienceDirect.com}); Sirius B $5×10^{-5}$ ({https://www.einstein-online.info/en/spotlight/redshift_white_dwarfs/#:~:text=Consequently%2C%20the%20shift%20should%20be,image%20was%20taken%20with%20the}{Gravitational redshift and White Dwarf stars «  Einstein-Online}); both ~match GR & Yes (VAM=GR) \\
        Light Deflection (Sun) & $\delta = 4GM/(Rc^2) = 1.75″$ ([Testing General Relativity & Total Solar Eclipse 2017]({https://eclipse2017.nasa.gov/testing-general-relativity#:~:text=where%20for%20the%20sun%20we,9x108%20meters}{https://eclipse2017.nasa.gov/testing-general-relativity#:~:text=where%20for%20the%20sun%20we,9x108%20meters})) & $\delta = 4GM/(Rc^2)$ ({file://file-2nnnwmwbptvdvnbeqndhbe%23:~:text=given%20by:/}{GR_in_3d.pdf}) & $1.75″\pm0.07″$ (VLBI) ([Testing General Relativity \\
        Perihelion Precession (Mercury) & $Δϖ = 6πGM/[a(1-e^2)c^2] = 42.98″/$cent. ({https://math.ucr.edu/home/baez/physics/Relativity/GR/mercury_orbit.html#:~:text=Mercury%27s%20perihelion%20precession%3A%20%20,3%20arcseconds%2Fcentury}{GR and Mercury's Orbital Precession}) & \textit{Same as GR} ({file://xn--file-2nnnwmwbptvdvnbeqndhbe%23:~:text=vam%20=%206gm-sm9aunx840m/}{GR_in_3d.pdf}) & $43.1″/$cent. (observed) ({https://math.ucr.edu/home/baez/physics/Relativity/GR/mercury_orbit.html#:~:text=Mercury%27s%20perihelion%20precession%3A%20%20,3%20arcseconds%2Fcentury}{GR and Mercury's Orbital Precession}) & Yes (exact within error) \\
        Frame-Dragging (Earth LT) & $Ω_{LT}=2GJ/(c^2 r^3)$ ({file://file-2nnnwmwbptvdvnbeqndhbe%23:~:text=2,2gj/}{GR_in_3d.pdf}) (≈39 mas/yr) & $Ω_{drag}=\frac{4}{5}\frac{GMΩ}{c^2 r}$ ({file://xn--file-2nnnwmwbptvdvnbeqndhbe%23:~:text=vam%20drag%20-uz9a/}{GR_in_3d.pdf}) (gives same 39 mas/yr) & $37.2±7.2$ mas/yr (GP-B) ({https://arxiv.org/abs/1105.3456#:~:text=Analysis%20of%20the%20data%20from,9%20rad}{[1105.3456] Gravity Probe B: Final Results of a Space Experiment to Test General Relativity}) & Yes (~0%, within 1σ) \\
        Geodetic Precession (Earth) & $Ω_{geo} = \frac{3}{2}\frac{GM}{c^2 r^3} v$ (6606 mas/yr) & \textit{Not derived} (flat space → 0 without extra assumption) & $6601.8±18$ mas/yr (GP-B) ({https://arxiv.org/abs/1105.3456#:~:text=Analysis%20of%20the%20data%20from,9%20rad}{[1105.3456] Gravity Probe B: Final Results of a Space Experiment to Test General Relativity}) & No (VAM missing effect) \\
        ISCO Radius (Schwarzschild BH) & $r_{ISCO}=6GM/c^2$ (for test mass) & No natural ISCO (orbits possible until horizon) & Fe Kα disk lines, GW inspiral waves → match 6GM/c^2 (GR confirmed) & No (needs extra mechanism) \\
        Gravitational Wave Emission (Binary) & Energy loss via quadrupole (P_dot matches to 0.2%) ({https://adsabs.harvard.edu/pdf/2005ASPC..328...25W#:~:text=b%2CGR%20%3D%201,and%20theoretical%20orbital%20decays%20are}{}) & \textit{No GW} (stable or slowly decaying orbit) & PSR1913–16: $-2.405\times10^{-12}$ ({https://adsabs.harvard.edu/pdf/2005ASPC..328...25W#:~:text=quan%02tities%2C%20including%20the%20distance%20and,Hence%20P%CB%99%20b%2Ccorrected}{}) (100% of GR); GW150914: direct detection & No (missing entirely) \\
        \bottomrule
    \end{tabular}
    \caption{}
    \label{tab:}
\end{table}Clearly, VAM as originally formulated does not account for these. This is a major failure of VAM in its current form. The model must be extended to include radiation of gravitational energy through the æther medium. A few possibilities to consider:
\begin{itemize}\item 
Compressible Æther: If the æther is not perfectly incompressible, accelerated masses might launch compression waves or vortex waves in the fluid. For instance, an orbiting binary could induce time-varying pressure fields that propagate outward (analogous to sound waves). If tuned right, these could carry away the same energy as GR's gravitational waves. The gravitational-wave frequency (twice the orbital frequency) could correspond to an æther wave mode. VAM's incompressibility was assumed for simplicity, but relaxing that could allow longitudinal waves. However, gravitational waves in GR are transverse and travel at $c$. Perhaps a small compressibility could mimic transverse-like shear waves in a superfluid (second sound?).
\item 
Vortex Shedding or Turbulence: Two rotating vortices (the æther structures of two masses) orbiting each other might shed tiny vortices or excitations (like how two moving vortices in a fluid can generate spiral waves). If VAM included a bit of viscosity or coupling to some background field (like the \grqq entropy field\textquotedblright mentioned in the paper for thermodynamics), energy could be lost that way. Since VAM does unify some thermodynamics, one might speculate merging vortex knots could emit some \grqq knot excitations\textquotedblright carrying energy.
\item 
Electromagnetic Radiation? The pulsars also emit EM waves, but we account for those (and they are far too small to explain the decay – the power in gravitational waves is many orders larger). So it's not EM.
\end{itemize}Given these, the simplest fix is to postulate a compressible component to the æther. For example, if the æther is a superfluid, it might support sound-like modes (quantum pressure waves). By adjusting the æther's compressibility so that these modes travel at $c$ (the speed of light), one could effectively have gravitational wave analogs. The quadrupole changes of the vortex would then emit those waves. One would need to show that the amplitude and polarization match GR's predictions (two polarization states, tensor character). This is non-trivial, but some analogue gravity systems (e.g. fluid analogues) do generate wave solutions for metric perturbations ({file://file-2nnnwmwbptvdvnbeqndhbe%23:~:text=this%20paper%20introduces%20a%20fluid,general%20relativity%20through%20the%20vortex/}{GR_in_3d.pdf}) ({file://file-2nnnwmwbptvdvnbeqndhbe%23:~:text=approach%20extends%20analogue%20gravity%20programs,1,%202],%20but/}{GR_in_3d.pdf}). VAM in its paper did not elaborate on dynamic perturbations, focusing on static solutions. Therefore, to make VAM viable, introducing gravitational radiation is critical.
Concretely, one could introduce an equation of state to the æther (slight elasticity) such that any time-varying quadrupole in the vortex density launches outward-propagating ripples. These would carry energy and angular momentum away, causing inspiral. The formula for $dP/dt$ in GR is complicated (involving the masses and separation), but since VAM already matches the static fields, making its radiative losses match GR's might be plausible with the right adjustments. Until such a feature is added and shown to match the $10^{-3}$ precision measurements of pulsar decay, VAM remains inconsistent with these observations.
As LIGO/Virgo have now directly observed gravitational waves from merging black holes and neutron stars (with waveforms precisely fitting GR's templates), any alternative like VAM must reproduce those signals. Currently, VAM has no clear mechanism for a ripple in flat space that travels out at light speed and has the quadrupolar form. This is an open problem for VAM. Proposed remedy: incorporate time-dependent æther perturbation equations and show that in the weak-field far zone, they reduce to a wave equation $\nabla^2 \psi - \frac{1}{c^2}\partial^2_t \psi = S(t)$ with source $S$ related to second time derivatives of quadrupole moments (the known GR structure). Without this, VAM cannot claim to pass the binary inspiral test.
In summary, Gravitational wave emission is a domain where VAM fails badly, requiring new physics in the model. We recommend making the æther slightly compressible or introducing a new field that mediates radiation of the vortex motion. This could be analogous to electromagnetic waves emanating from accelerating charges, but here for accelerating mass–vortices. If done correctly, VAM could then match the Hulse–Taylor decay (by calibrating the coupling of the æther wave to the quadrupole moment) and the gravitational wave speeds. Until such a correction is made, GR stands unchallenged in this arena.
\section*{10. Geodetic Precession (De Sitter Precession)}Finally, we consider the geodetic precession (also called de Sitter precession) – the precession of a gyroscope due to moving through curved spacetime, even with a non-rotating central mass. In GR, a gyroscope orbiting a mass $M$ at semi-major axis $a$ has its spin precess by $\Omega_\text{geod} = \frac{3}{2}\frac{GM}{c^2 a^3},\mathbf{v}\times\mathbf{r}$ (vector form). For a polar Earth satellite like Gravity Probe B, this amounted to a predicted ~6606 mas/yr in the orbital plane (geodetic effect) ({https://arxiv.org/abs/1105.3456#:~:text=Analysis%20of%20the%20data%20from,9%20rad}{[1105.3456] Gravity Probe B: Final Results of a Space Experiment to Test General Relativity}). GP-B measured $6601.8 \pm 18.3$ mas/yr ({https://arxiv.org/abs/1105.3456#:~:text=Analysis%20of%20the%20data%20from,9%20rad}{[1105.3456] Gravity Probe B: Final Results of a Space Experiment to Test General Relativity}) – a 0.3% agreement with GR.
VAM Consideration: In VAM, spacetime is Euclidean and only the æther flow might cause precession. If Earth's æther vortex is static (non-rotating frame aside from the rotation we accounted for in Lense–Thirring), what would cause a gyroscope's spin to tilt? There is no spacetime curvature, but perhaps one can attribute geodetic precession to an effect of moving through a spatial gradient of æther velocity. As a gyroscope orbits, one side of it might experience slightly different flow than the other, potentially inducing torque. However, in an ideal incompressible flow that is static in Earth's frame, a free spinning object might not spontaneously precess without some asymmetry.
Another way to derive geodetic precession is via special relativity: a spinning object moving in a gravitational field undergoes Thomas precession due to the combination of orbit velocity and gravitational acceleration. In GR, that exactly yields the geodetic rate. In flat spacetime, one can reproduce de Sitter precession by considering the non-inertial motion of the gyroscope in Earth's gravitational field and doing successive Lorentz boosts (this was done historically by Einstein and others). So even if space is flat, a gyroscope's spin axis will precess by $\approx \frac{3}{2} \frac{GM}{c^2 a}$ per orbital revolution (for a circular orbit). This is essentially a result of the equivalence principle + special relativity. Thus, VAM could emulate geodetic precession if it fully incorporates equivalence principle effects.
However, VAM's current formulation treats gravity as a force (pressure gradient). A gyroscope in orbit is in free fall (just like an accelerometer in a satellite reads near zero). In GR, free fall in curved spacetime still affects the orientation of a spin (via parallel transport on a curved manifold). In VAM, free fall means the gyroscope is moving along with the æther flow (if we imagine the satellite is moving under gravity). Would its spin vector change relative to distant stars? Possibly not, unless some Coriolis-like effect from the æther rotation is present. Earth's æther vortex is primarily rotational (to cause gravity, presumably swirling around Earth's center). A polar orbit gyroscope might pass through regions of different æther velocity direction, which could cause its spin to gradually rotate relative to inertial space.
To be concrete, the GR geodetic precession for Earth is $\approx 6.6″$/yr (or 6600 mas/yr) toward the orbital motion direction. VAM did not explicitly calculate this. If VAM's gravitational field is purely central (no spacetime curvature), one might initially think VAM predicts zero geodetic precession, which is wrong. But there is a way VAM could get a non-zero result: in a rotating reference frame of the æther, perhaps an object's spin experiences a Foucault-like precession. Consider a gyroscope in orbit – from the perspective of the æther co-moving frame, its spin might appear to precess due to the rotation of the frame as it goes around Earth.
It's quite complex, but given GP-B's verification, we must see if VAM can accommodate it. If not, it's a glaring gap. The authors did not mention geodetic precession in their summary table (they only listed frame-dragging) ({file://file-2nnnwmwbptvdvnbeqndhbe%23:~:text=2,2gj/}{GR_in_3d.pdf}). That suggests they might not have derived it.
Comparison: Table 8 highlights geodetic vs frame-dragging for Earth's satellite:
Table 8: Geodetic vs Frame-Dragging Precession (Earth satellite, GP-B)
\begin{table}
    \centering
    \begin{tabular}{lllll}
        \toprule
        \textbf{Object} & \textbf{GR: $d\tau/dt = \sqrt{1-2GM/(Rc^2)}$} & \textbf{VAM: $d\tau/dt = \sqrt{1-v_\phi^2/c^2}$ (tuned)} & \textbf{Observed Effect} & \textbf{Relative Error (VAM vs Obs)} \\
        \midrule
        Earth (R=6.37×10^6 m) & 0.9999999993 (Δ≈7×10^–10) ({https://www.einstein-online.info/en/spotlight/redshift_white_dwarfs/#:~:text=The%20gravitational%20redshift%20was%20first,the%20field%20of%20the%20sun}{Gravitational redshift and White Dwarf stars «  Einstein-Online}) & 0.9999999993 (assuming $v_\phi\approx 11.2$ km/s) & Clock gain of +45 µs/day at orbit (GPS) ({https://www.einstein-online.info/en/spotlight/redshift_white_dwarfs/#:~:text=The%20gravitational%20redshift%20was%20first,the%20field%20of%20the%20sun}{Gravitational redshift and White Dwarf stars «  Einstein-Online}) & ~0% (VAM tuned to match) \\
        Sun (R=6.96×10^8 m) & 0.9999979 (Δ≈2.1×10^–6) ({https://www.sciencedirect.com/science/article/abs/pii/S1384107614000190#:~:text=On%20the%20gravitational%20redshift%20,in%201908%E2%80%94is%20still%20an}{On the gravitational redshift - ScienceDirect.com}) & 0.9999979 (if $v_\phi\approx 618$ km/s) & Solar spectral line redshift ~2×10^–6 ({https://www.sciencedirect.com/science/article/abs/pii/S1384107614000190#:~:text=On%20the%20gravitational%20redshift%20,in%201908%E2%80%94is%20still%20an}{On the gravitational redshift - ScienceDirect.com}) & ~0% (within meas. error) \\
        Neutron Star (M≈1.4 M_⊙, R≈1×10^4 m) & 0.875 (Δ≈0.234) (strong field) & 0.875 (if $v_\phi\approx0.65c$) & X-ray spectral redshift z~0.3 expected (some observations) & ~0% (assumed match) \\
        Proton (m=1.67×10^–27 kg) & ~1 – 1×10^–27 (negligible) & ~1 (μ factor suppresses gravity at r<1 mm) & No measurable gravitational slowdown & N/A (both predict none) \\
        Electron (m=9.11×10^–31 kg) & ~1 – 1×10^–30 (negligible) & ~1 (suppressed by quantum scaling μ) & No measurable gravitational slowdown & N/A \\
        Effect (Earth) & GR Prediction & VAM Prediction & Observed Value & Error (VAM vs Obs) \\
        Frame-dragging (GP-B gyroscope) & $39.2$ mas/yr (north-south axis precession) ({https://arxiv.org/abs/1105.3456#:~:text=Analysis%20of%20the%20data%20from,9%20rad}{[1105.3456] Gravity Probe B: Final Results of a Space Experiment to Test General Relativity}) & $≈39$ mas/yr (μ=1, identical formula) ({file://file-2nnnwmwbptvdvnbeqndhbe%23:~:text=2,2gj/}{GR_in_3d.pdf}) ({file://file-2nnnwmwbptvdvnbeqndhbe%23:~:text=c2r3/}{GR_in_3d.pdf}) & $37.2 \pm 7.2$ mas/yr (GP-B) ({https://arxiv.org/abs/1105.3456#:~:text=Analysis%20of%20the%20data%20from,9%20rad}{[1105.3456] Gravity Probe B: Final Results of a Space Experiment to Test General Relativity}) & ~0% (within 1σ error) \\
        Frame-dragging (polar LAGEOS) & $~$31 mas/yr (node regression) & $~$31 mas/yr (same as GR) & $30 \pm 5$ mas/yr (measured) & ~0% (within error) \\
        Scenario & GR Redshift $z$ & VAM Redshift $z$ & Observed $z$ (or test) & Result \\
        Pound–Rebka (Earth) – 22.5 m tower & $2.5\times10^{-15}$ (for Δφ = 22.5 m g) ({https://en.wikipedia.org/wiki/Tests_of_general_relativity#:~:text=line%20width,first%20precision%20experiments%20testing%20general}{Tests of general relativity - Wikipedia}) & $2.5\times10^{-15}$ (æther flow equivalent) & $2.5\times10^{-15}$ ± 5% ({https://en.wikipedia.org/wiki/Tests_of_general_relativity#:~:text=line%20width,first%20precision%20experiments%20testing%20general}{Tests of general relativity - Wikipedia}) (gamma-ray) & Matched (0% error) \\
        Sun to infinity (surface) & $2.12\times10^{-6}$ ({https://www.sciencedirect.com/science/article/abs/pii/S1384107614000190#:~:text=On%20the%20gravitational%20redshift%20,in%201908%E2%80%94is%20still%20an}{On the gravitational redshift - ScienceDirect.com}) & $2.12\times10^{-6}$ (if $v_\phi\approx 617$ km/s) & ~$2.12\times10^{-6}$ (solar spectrum, expected) ({https://www.sciencedirect.com/science/article/abs/pii/S1384107614000190#:~:text=On%20the%20gravitational%20redshift%20,in%201908%E2%80%94is%20still%20an}{On the gravitational redshift - ScienceDirect.com}) & Matched (few % error due to solar Doppler) \\
        White Dwarf (Sirius B) – high $GM/R$ & $5.5\times10^{-5}$ (for Sirius B) & $5.5\times10^{-5}$ (if vortex tuned) & $4.8(±0.3)\times10^{-5}$ (observed spectral lines) ({https://www.einstein-online.info/en/spotlight/redshift_white_dwarfs/#:~:text=Consequently%2C%20the%20shift%20should%20be,image%20was%20taken%20with%20the}{Gravitational redshift and White Dwarf stars «  Einstein-Online}) & ~15% error (VAM can tune) \\
        Neutron Star (massive) & $0.3$ (30% freq. drop for $2GM/(Rc^2)\sim0.4$) & $0.3$ (if $v_\phi \sim0.7c$ at surface) & ~0.35 (possible X-ray line measure, uncertain) & ~0% (within uncertainty) \\
        Light Ray & Mass & GR Deflection & VAM Deflection & Observed Deflection & Error \\
        Star near Sun's limb (impact ~R_⊙) & $1.75″$ (arcsec) ([Testing General Relativity & Total Solar Eclipse 2017]({https://eclipse2017.nasa.gov/testing-general-relativity#:~:text=where%20for%20the%20sun%20we,9x108%20meters}{https://eclipse2017.nasa.gov/testing-general-relativity#:~:text=where%20for%20the%20sun%20we,9x108%20meters})) & $1.75″$ ([Testing General Relativity & Total Solar Eclipse 2017]({https://eclipse2017.nasa.gov/testing-general-relativity#:~:text=where%20for%20the%20sun%20we,9x108%20meters}{https://eclipse2017.nasa.gov/testing-general-relativity#:~:text=where%20for%20the%20sun%20we,9x108%20meters})) ({file://file-2nnnwmwbptvdvnbeqndhbe%23:~:text=given%20by:/}{GR_in_3d.pdf}) \\
        Light near Earth (impact ~R⊕) & $8.5\times10^{-6}$″ (microarcsec) & $8.5\times10^{-6}$″ & ~ Not measured (too small) & N/A \\
        Quasar by galaxy (strong lens) & GR lensing formulas (non-linear) & VAM: requires fluid simulation & Multiple images, matches GR lensing & (Unknown for VAM) \\
        Orbit (Central mass) & GR Extra Precession & VAM Precession & Observed Extra Precession & Agreement? \\
        Mercury around Sun & $42.98″/\text{century}$ ({https://math.ucr.edu/home/baez/physics/Relativity/GR/mercury_orbit.html#:~:text=Mercury%27s%20perihelion%20precession%3A%20%20,3%20arcseconds%2Fcentury}{GR and Mercury's Orbital Precession}) & $42.98″/\text{century}$ ({file://xn--file-2nnnwmwbptvdvnbeqndhbe%23:~:text=vam%20=%206gm-sm9aunx840m/}{GR_in_3d.pdf}) & $43.1±0.2″/\text{century}$ ({https://math.ucr.edu/home/baez/physics/Relativity/GR/mercury_orbit.html#:~:text=Mercury%27s%20perihelion%20precession%3A%20%20,3%20arcseconds%2Fcentury}{GR and Mercury's Orbital Precession}) & Yes (within 0.3%) \\
        Earth around Sun & $3.84″/\text{century}$ (calc.) & $3.84″/\text{century}$ & ~$3.84″$ (too small to measure accurately) & Yes (not directly measured) \\
        Binary Pulsar (PSR J0737) – periastron advance & $\sim 16.9°/\text{yr}$ (GR) & $16.9°/\text{yr}$ (VAM, by design) & $16.9°/\text{yr}$ observed (Double Pulsar) & Yes (0% error) \\
        Object & Newtonian/GR Potential at surface $Φ=-GM/R$ (J/kg) & VAM $Φ$ (from vortex) & Surface Gravity $g=GM/R^2$ & Observations (g) \\
        Earth & $-6.25×10^7$ J/kg (→ $v_\text{esc}=11.2$ km/s) & Matches (by construction) ({file://xn--file-2nnnwmwbptvdvnbeqndhbe%23:~:text=2,gm%20r-x034a/}{GR_in_3d.pdf}) & $9.81$ m/s² (observed) & $9.81$ m/s² (exactly) \\
        Sun & $-1.9×10^8$ J/kg (→ $v_\text{esc}=617$ km/s) & Matches (with appropriate γ) & $274$ m/s² (at surface) & ~274 m/s² (via helioseismology) \\
        Neutron Star & ~$-2×10^{13}$ J/kg (for 1.4 M⊙, R=12 km) & Would match if $v_\phi$ near c & ~$1.6×10^{12}$ m/s² (extreme) & Indirect (from orbits, X-ray bursts) \\
        System (Pulsar Binary) & GR $dP/dt$ (orbital period change) & VAM $dP/dt$ & Observed $dP/dt$ & VAM vs Obs Error \\
        PSR B1913+16 (Hulse–Taylor) & $-2.4025\times10^{-12}$ s/s (energy loss via GW) ({https://adsabs.harvard.edu/pdf/2005ASPC..328...25W#:~:text=predicted%20orbital%20period%20derivative%20due,to%20gravitational%20radiation%20computed%20from}{}) & ~ $0$ s/s (no built-in wave emission) & $-2.4056(51)\times10^{-12}$ s/s ({https://adsabs.harvard.edu/pdf/2005ASPC..328...25W#:~:text=quan%02tities%2C%20including%20the%20distance%20and,Hence%20P%CB%99%20b%2Ccorrected}{}) ({https://adsabs.harvard.edu/pdf/2005ASPC..328...25W#:~:text=b%2CGR%20%3D%201,and%20theoretical%20orbital%20decays%20are}{}) & ~100% (fails) \\
        PSR J0737–3039A/B (Double Pulsar) & $-1.252\times10^{-12}$ s/s (predicted) & ~ $0$ s/s & $-1.252(17)\times10^{-12}$ s/s (measured) & ~100% (fails) \\
        GW150914 (Binary BH merger) & 3 solar mass energy radiated in GWs & No GW (merger dynamics unclear) & Detected GWs, strain amplitude $~10^{-21}$ ({http://ui.adsabs.harvard.edu/abs/2010ApJ...722.1030W/abstract#:~:text=Timing%20Measurements%20of%20the%20Relativistic,radiation%20damping%20in%20general}{Timing Measurements of the Relativistic Binary Pulsar PSR B1913+16}) & Complete miss \\
        Precession Effect & GR Prediction (mas/yr) & VAM Prediction (mas/yr) & Observed (mas/yr) & VAM vs Obs \\
        Geodetic (de Sitter) & $6606.1$ mas/yr (forward in orbit plane) ({https://arxiv.org/abs/1105.3456#:~:text=Analysis%20of%20the%20data%20from,9%20rad}{[1105.3456] Gravity Probe B: Final Results of a Space Experiment to Test General Relativity}) & Not explicitly derived (possibly 0 without extension) & $6601.8 \pm 18.3$ mas/yr ({https://arxiv.org/abs/1105.3456#:~:text=Analysis%20of%20the%20data%20from,9%20rad}{[1105.3456] Gravity Probe B: Final Results of a Space Experiment to Test General Relativity}) & ~100% error if 0 \\
        Frame-Dragging & $39.2$ mas/yr (around Earth's spin axis) ({https://arxiv.org/abs/1105.3456#:~:text=Analysis%20of%20the%20data%20from,9%20rad}{[1105.3456] Gravity Probe B: Final Results of a Space Experiment to Test General Relativity}) & $≈39$ mas/yr (by matching formula) ({file://file-2nnnwmwbptvdvnbeqndhbe%23:~:text=2,2gj/}{GR_in_3d.pdf}) & $37.2 \pm 7.2$ mas/yr ({https://arxiv.org/abs/1105.3456#:~:text=Analysis%20of%20the%20data%20from,9%20rad}{[1105.3456] Gravity Probe B: Final Results of a Space Experiment to Test General Relativity}) & ~0% (within error) \\
        Phenomenon & GR Prediction & Formula & VAM Prediction & Formula & Observation (with ref) & Agreement? (Error) \\
        Gravitational Time Dilation (Static field) & $dτ/dt=\sqrt{1-2GM/(rc^2)}$ ({file://file-2nnnwmwbptvdvnbeqndhbe%23:~:text=2/}{GR_in_3d.pdf}) & $dτ/dt=\sqrt{1-Ω^2 r^2/c^2}$ (with $Ω r = v_φ$) ({file://xn--file-2nnnwmwbptvdvnbeqndhbe%23:~:text=%201%202r2-yx7a2275psla/}{GR_in_3d.pdf}) & GPS clocks: Δν/ν = $6.9×10^{-10}$ (Earth) ({https://www.einstein-online.info/en/spotlight/redshift_white_dwarfs/#:~:text=The%20gravitational%20redshift%20was%20first,the%20field%20of%20the%20sun}{Gravitational redshift and White Dwarf stars «  Einstein-Online}), Pound–Rebka: $2.5×10^{-15}$ ({https://en.wikipedia.org/wiki/Tests_of_general_relativity#:~:text=line%20width,first%20precision%20experiments%20testing%20general}{Tests of general relativity - Wikipedia}) (both match GR) & Yes (VAM tuned, 0% error) \\
        Velocity Time Dilation (SR) & $dτ/dt=\sqrt{1-v^2/c^2}$ & $dτ/dt=\sqrt{1-v^2/c^2}$ (same as GR/SR) ({file://xn--file-2nnnwmwbptvdvnbeqndhbe%23:~:text=d-tl1a/}{GR_in_3d.pdf}) & Particle accelerators, muon lifetime (time dilation confirmed to $10^{-8}$) & Yes (identical) \\
        Rotational (Kinetic) Time Dilation & (Included as mass-energy in GR implicitly) & $dτ/dt=(1+\tfrac{1}{2}\beta I Ω^2)^{-1}$ ({file://file-2nnnwmwbptvdvnbeqndhbe%23:~:text=/}{GR_in_3d.pdf}) (VAM heuristic) & No direct obs; fast pulsar (~716 Hz) slows time ~0.5% (GR via mass-energy; VAM via rotation) & In principle (if β set right) \\
        Gravitational Redshift & $z=(1-2GM/(rc^2))^{-1/2}-1$ ({file://file-2nnnwmwbptvdvnbeqndhbe%23:~:text=2/}{GR_in_3d.pdf}) & $z=(1-v_φ^2/c^2)^{-1/2}-1$ ({file://xn--file-2nnnwmwbptvdvnbeqndhbe%23:~:text=%201-py51a/}{GR_in_3d.pdf}) & Solar redshift $2.12×10^{-6}$ ({https://www.sciencedirect.com/science/article/abs/pii/S1384107614000190#:~:text=On%20the%20gravitational%20redshift%20,in%201908%E2%80%94is%20still%20an}{On the gravitational redshift - ScienceDirect.com}); Sirius B $5×10^{-5}$ ({https://www.einstein-online.info/en/spotlight/redshift_white_dwarfs/#:~:text=Consequently%2C%20the%20shift%20should%20be,image%20was%20taken%20with%20the}{Gravitational redshift and White Dwarf stars «  Einstein-Online}); both ~match GR & Yes (VAM=GR) \\
        Light Deflection (Sun) & $\delta = 4GM/(Rc^2) = 1.75″$ ([Testing General Relativity & Total Solar Eclipse 2017]({https://eclipse2017.nasa.gov/testing-general-relativity#:~:text=where%20for%20the%20sun%20we,9x108%20meters}{https://eclipse2017.nasa.gov/testing-general-relativity#:~:text=where%20for%20the%20sun%20we,9x108%20meters})) & $\delta = 4GM/(Rc^2)$ ({file://file-2nnnwmwbptvdvnbeqndhbe%23:~:text=given%20by:/}{GR_in_3d.pdf}) & $1.75″\pm0.07″$ (VLBI) ([Testing General Relativity \\
        Perihelion Precession (Mercury) & $Δϖ = 6πGM/[a(1-e^2)c^2] = 42.98″/$cent. ({https://math.ucr.edu/home/baez/physics/Relativity/GR/mercury_orbit.html#:~:text=Mercury%27s%20perihelion%20precession%3A%20%20,3%20arcseconds%2Fcentury}{GR and Mercury's Orbital Precession}) & \textit{Same as GR} ({file://xn--file-2nnnwmwbptvdvnbeqndhbe%23:~:text=vam%20=%206gm-sm9aunx840m/}{GR_in_3d.pdf}) & $43.1″/$cent. (observed) ({https://math.ucr.edu/home/baez/physics/Relativity/GR/mercury_orbit.html#:~:text=Mercury%27s%20perihelion%20precession%3A%20%20,3%20arcseconds%2Fcentury}{GR and Mercury's Orbital Precession}) & Yes (exact within error) \\
        Frame-Dragging (Earth LT) & $Ω_{LT}=2GJ/(c^2 r^3)$ ({file://file-2nnnwmwbptvdvnbeqndhbe%23:~:text=2,2gj/}{GR_in_3d.pdf}) (≈39 mas/yr) & $Ω_{drag}=\frac{4}{5}\frac{GMΩ}{c^2 r}$ ({file://xn--file-2nnnwmwbptvdvnbeqndhbe%23:~:text=vam%20drag%20-uz9a/}{GR_in_3d.pdf}) (gives same 39 mas/yr) & $37.2±7.2$ mas/yr (GP-B) ({https://arxiv.org/abs/1105.3456#:~:text=Analysis%20of%20the%20data%20from,9%20rad}{[1105.3456] Gravity Probe B: Final Results of a Space Experiment to Test General Relativity}) & Yes (~0%, within 1σ) \\
        Geodetic Precession (Earth) & $Ω_{geo} = \frac{3}{2}\frac{GM}{c^2 r^3} v$ (6606 mas/yr) & \textit{Not derived} (flat space → 0 without extra assumption) & $6601.8±18$ mas/yr (GP-B) ({https://arxiv.org/abs/1105.3456#:~:text=Analysis%20of%20the%20data%20from,9%20rad}{[1105.3456] Gravity Probe B: Final Results of a Space Experiment to Test General Relativity}) & No (VAM missing effect) \\
        ISCO Radius (Schwarzschild BH) & $r_{ISCO}=6GM/c^2$ (for test mass) & No natural ISCO (orbits possible until horizon) & Fe Kα disk lines, GW inspiral waves → match 6GM/c^2 (GR confirmed) & No (needs extra mechanism) \\
        Gravitational Wave Emission (Binary) & Energy loss via quadrupole (P_dot matches to 0.2%) ({https://adsabs.harvard.edu/pdf/2005ASPC..328...25W#:~:text=b%2CGR%20%3D%201,and%20theoretical%20orbital%20decays%20are}{}) & \textit{No GW} (stable or slowly decaying orbit) & PSR1913–16: $-2.405\times10^{-12}$ ({https://adsabs.harvard.edu/pdf/2005ASPC..328...25W#:~:text=quan%02tities%2C%20including%20the%20distance%20and,Hence%20P%CB%99%20b%2Ccorrected}{}) (100% of GR); GW150914: direct detection & No (missing entirely) \\
        \bottomrule
    \end{tabular}
    \caption{}
    \label{tab:}
\end{table}We see VAM got frame-dragging right but likely missed geodetic precession. To salvage this, one could incorporate the effect of æther flow curvature on moving spin. One approach: in GR, geodetic precession can be seen as half of the deflection of a parallel transported vector over a closed orbit (the other half being the Thomas precession from special relativity). VAM could try to replicate parallel transport in flat space with a force field: a spin in free fall might still feel a torque if the gravitational field has a gradient. In fact, in a non-inertial frame one has a -Ω × S term (where Ω is some angular velocity of the local frame) causing precession. For a satellite in Earth's gravity, an observer might attribute a fictitious rotation Ω = (1/2)Ω_orbit (this comes out of GR derivations). If VAM can produce an effective rotational reference frame for free-fall, it could get $\frac{1}{2}\omega_\text{orbital}$ per orbit, which is the geodetic effect.
Likely, VAM did not initially include this, so we identify it as a shortcoming. To correct it, we propose: Extend VAM to include the transport of spin. Perhaps define that the æther flow not only drags space but also defines a local inertial frame for spin precession. By requiring consistency with the equivalence principle, one can derive the de Sitter precession in flat space (it's a known result that you don't \textit{need} full GR to get the correct number, just GR to justify it fundamentally). If one treats the gravitational force as affecting the gyroscope's four-velocity but not directly the spin, except that the spin's direction in an inertial frame changes because the frame is rotating relative to distant stars (as the gyroscope orbits), one can get the same precession.
In summary, without modifications, VAM fails to predict geodetic precession (which has been experimentally confirmed). The fix is conceptual: incorporate a \grqq spin connection\textquotedblright for the æther flow. In other words, when analyzing spin dynamics, include a term analogous to the Christoffel spin connection in GR, but built from æther velocity gradients. Possibly, require that a spin undergoes Fermi–Walker transport in the æther velocity field. Doing so should yield the same $\frac{3GM}{2rc^2}$ per revolution precession. This is admittedly an \textit{ad hoc} addition, but it aligns with the idea that VAM must reproduce all aspects of the geodesic equation, not just the path but also how vectors rotate.
Recommendations: The VAM framework could introduce a rule: \textit{Any extended object moving in the æther with velocity $\mathbf{v}$ experiences a precession of its spin given by $\boldsymbol{\Omega}\textit{\text{geo} = -\frac{1}{2}\nabla \times \mathbf{v}}\text{field}$ (evaluated along its trajectory).} For a circular orbit in a static central flow, $\nabla \times \mathbf{v}_\text{field}$ might relate to the angular velocity of the orbit, yielding the correct de Sitter rate. This is speculative, but a starting point for making VAM consistent with Gravity Probe B's geodetic result.\section*{Summary and Conclusions}Summary of Results: The benchmark comparison between General Relativity and the Vortex Æther Model is encapsulated in Table 9, which lists each phenomenon, GR's prediction, VAM's prediction, the observed value, and any discrepancy:
Table 9: GR vs VAM vs Observations – Summary of Key Tests
\begin{table}
    \centering
    \begin{tabular}{lllll}
        \toprule
        \textbf{Object} & \textbf{GR: $d\tau/dt = \sqrt{1-2GM/(Rc^2)}$} & \textbf{VAM: $d\tau/dt = \sqrt{1-v_\phi^2/c^2}$ (tuned)} & \textbf{Observed Effect} & \textbf{Relative Error (VAM vs Obs)} \\
        \midrule
        Earth (R=6.37×10^6 m) & 0.9999999993 (Δ≈7×10^–10) ({https://www.einstein-online.info/en/spotlight/redshift_white_dwarfs/#:~:text=The%20gravitational%20redshift%20was%20first,the%20field%20of%20the%20sun}{Gravitational redshift and White Dwarf stars «  Einstein-Online}) & 0.9999999993 (assuming $v_\phi\approx 11.2$ km/s) & Clock gain of +45 µs/day at orbit (GPS) ({https://www.einstein-online.info/en/spotlight/redshift_white_dwarfs/#:~:text=The%20gravitational%20redshift%20was%20first,the%20field%20of%20the%20sun}{Gravitational redshift and White Dwarf stars «  Einstein-Online}) & ~0% (VAM tuned to match) \\
        Sun (R=6.96×10^8 m) & 0.9999979 (Δ≈2.1×10^–6) ({https://www.sciencedirect.com/science/article/abs/pii/S1384107614000190#:~:text=On%20the%20gravitational%20redshift%20,in%201908%E2%80%94is%20still%20an}{On the gravitational redshift - ScienceDirect.com}) & 0.9999979 (if $v_\phi\approx 618$ km/s) & Solar spectral line redshift ~2×10^–6 ({https://www.sciencedirect.com/science/article/abs/pii/S1384107614000190#:~:text=On%20the%20gravitational%20redshift%20,in%201908%E2%80%94is%20still%20an}{On the gravitational redshift - ScienceDirect.com}) & ~0% (within meas. error) \\
        Neutron Star (M≈1.4 M_⊙, R≈1×10^4 m) & 0.875 (Δ≈0.234) (strong field) & 0.875 (if $v_\phi\approx0.65c$) & X-ray spectral redshift z~0.3 expected (some observations) & ~0% (assumed match) \\
        Proton (m=1.67×10^–27 kg) & ~1 – 1×10^–27 (negligible) & ~1 (μ factor suppresses gravity at r<1 mm) & No measurable gravitational slowdown & N/A (both predict none) \\
        Electron (m=9.11×10^–31 kg) & ~1 – 1×10^–30 (negligible) & ~1 (suppressed by quantum scaling μ) & No measurable gravitational slowdown & N/A \\
        Effect (Earth) & GR Prediction & VAM Prediction & Observed Value & Error (VAM vs Obs) \\
        Frame-dragging (GP-B gyroscope) & $39.2$ mas/yr (north-south axis precession) ({https://arxiv.org/abs/1105.3456#:~:text=Analysis%20of%20the%20data%20from,9%20rad}{[1105.3456] Gravity Probe B: Final Results of a Space Experiment to Test General Relativity}) & $≈39$ mas/yr (μ=1, identical formula) ({file://file-2nnnwmwbptvdvnbeqndhbe%23:~:text=2,2gj/}{GR_in_3d.pdf}) ({file://file-2nnnwmwbptvdvnbeqndhbe%23:~:text=c2r3/}{GR_in_3d.pdf}) & $37.2 \pm 7.2$ mas/yr (GP-B) ({https://arxiv.org/abs/1105.3456#:~:text=Analysis%20of%20the%20data%20from,9%20rad}{[1105.3456] Gravity Probe B: Final Results of a Space Experiment to Test General Relativity}) & ~0% (within 1σ error) \\
        Frame-dragging (polar LAGEOS) & $~$31 mas/yr (node regression) & $~$31 mas/yr (same as GR) & $30 \pm 5$ mas/yr (measured) & ~0% (within error) \\
        Scenario & GR Redshift $z$ & VAM Redshift $z$ & Observed $z$ (or test) & Result \\
        Pound–Rebka (Earth) – 22.5 m tower & $2.5\times10^{-15}$ (for Δφ = 22.5 m g) ({https://en.wikipedia.org/wiki/Tests_of_general_relativity#:~:text=line%20width,first%20precision%20experiments%20testing%20general}{Tests of general relativity - Wikipedia}) & $2.5\times10^{-15}$ (æther flow equivalent) & $2.5\times10^{-15}$ ± 5% ({https://en.wikipedia.org/wiki/Tests_of_general_relativity#:~:text=line%20width,first%20precision%20experiments%20testing%20general}{Tests of general relativity - Wikipedia}) (gamma-ray) & Matched (0% error) \\
        Sun to infinity (surface) & $2.12\times10^{-6}$ ({https://www.sciencedirect.com/science/article/abs/pii/S1384107614000190#:~:text=On%20the%20gravitational%20redshift%20,in%201908%E2%80%94is%20still%20an}{On the gravitational redshift - ScienceDirect.com}) & $2.12\times10^{-6}$ (if $v_\phi\approx 617$ km/s) & ~$2.12\times10^{-6}$ (solar spectrum, expected) ({https://www.sciencedirect.com/science/article/abs/pii/S1384107614000190#:~:text=On%20the%20gravitational%20redshift%20,in%201908%E2%80%94is%20still%20an}{On the gravitational redshift - ScienceDirect.com}) & Matched (few % error due to solar Doppler) \\
        White Dwarf (Sirius B) – high $GM/R$ & $5.5\times10^{-5}$ (for Sirius B) & $5.5\times10^{-5}$ (if vortex tuned) & $4.8(±0.3)\times10^{-5}$ (observed spectral lines) ({https://www.einstein-online.info/en/spotlight/redshift_white_dwarfs/#:~:text=Consequently%2C%20the%20shift%20should%20be,image%20was%20taken%20with%20the}{Gravitational redshift and White Dwarf stars «  Einstein-Online}) & ~15% error (VAM can tune) \\
        Neutron Star (massive) & $0.3$ (30% freq. drop for $2GM/(Rc^2)\sim0.4$) & $0.3$ (if $v_\phi \sim0.7c$ at surface) & ~0.35 (possible X-ray line measure, uncertain) & ~0% (within uncertainty) \\
        Light Ray & Mass & GR Deflection & VAM Deflection & Observed Deflection & Error \\
        Star near Sun's limb (impact ~R_⊙) & $1.75″$ (arcsec) ([Testing General Relativity & Total Solar Eclipse 2017]({https://eclipse2017.nasa.gov/testing-general-relativity#:~:text=where%20for%20the%20sun%20we,9x108%20meters}{https://eclipse2017.nasa.gov/testing-general-relativity#:~:text=where%20for%20the%20sun%20we,9x108%20meters})) & $1.75″$ ([Testing General Relativity & Total Solar Eclipse 2017]({https://eclipse2017.nasa.gov/testing-general-relativity#:~:text=where%20for%20the%20sun%20we,9x108%20meters}{https://eclipse2017.nasa.gov/testing-general-relativity#:~:text=where%20for%20the%20sun%20we,9x108%20meters})) ({file://file-2nnnwmwbptvdvnbeqndhbe%23:~:text=given%20by:/}{GR_in_3d.pdf}) \\
        Light near Earth (impact ~R⊕) & $8.5\times10^{-6}$″ (microarcsec) & $8.5\times10^{-6}$″ & ~ Not measured (too small) & N/A \\
        Quasar by galaxy (strong lens) & GR lensing formulas (non-linear) & VAM: requires fluid simulation & Multiple images, matches GR lensing & (Unknown for VAM) \\
        Orbit (Central mass) & GR Extra Precession & VAM Precession & Observed Extra Precession & Agreement? \\
        Mercury around Sun & $42.98″/\text{century}$ ({https://math.ucr.edu/home/baez/physics/Relativity/GR/mercury_orbit.html#:~:text=Mercury%27s%20perihelion%20precession%3A%20%20,3%20arcseconds%2Fcentury}{GR and Mercury's Orbital Precession}) & $42.98″/\text{century}$ ({file://xn--file-2nnnwmwbptvdvnbeqndhbe%23:~:text=vam%20=%206gm-sm9aunx840m/}{GR_in_3d.pdf}) & $43.1±0.2″/\text{century}$ ({https://math.ucr.edu/home/baez/physics/Relativity/GR/mercury_orbit.html#:~:text=Mercury%27s%20perihelion%20precession%3A%20%20,3%20arcseconds%2Fcentury}{GR and Mercury's Orbital Precession}) & Yes (within 0.3%) \\
        Earth around Sun & $3.84″/\text{century}$ (calc.) & $3.84″/\text{century}$ & ~$3.84″$ (too small to measure accurately) & Yes (not directly measured) \\
        Binary Pulsar (PSR J0737) – periastron advance & $\sim 16.9°/\text{yr}$ (GR) & $16.9°/\text{yr}$ (VAM, by design) & $16.9°/\text{yr}$ observed (Double Pulsar) & Yes (0% error) \\
        Object & Newtonian/GR Potential at surface $Φ=-GM/R$ (J/kg) & VAM $Φ$ (from vortex) & Surface Gravity $g=GM/R^2$ & Observations (g) \\
        Earth & $-6.25×10^7$ J/kg (→ $v_\text{esc}=11.2$ km/s) & Matches (by construction) ({file://xn--file-2nnnwmwbptvdvnbeqndhbe%23:~:text=2,gm%20r-x034a/}{GR_in_3d.pdf}) & $9.81$ m/s² (observed) & $9.81$ m/s² (exactly) \\
        Sun & $-1.9×10^8$ J/kg (→ $v_\text{esc}=617$ km/s) & Matches (with appropriate γ) & $274$ m/s² (at surface) & ~274 m/s² (via helioseismology) \\
        Neutron Star & ~$-2×10^{13}$ J/kg (for 1.4 M⊙, R=12 km) & Would match if $v_\phi$ near c & ~$1.6×10^{12}$ m/s² (extreme) & Indirect (from orbits, X-ray bursts) \\
        System (Pulsar Binary) & GR $dP/dt$ (orbital period change) & VAM $dP/dt$ & Observed $dP/dt$ & VAM vs Obs Error \\
        PSR B1913+16 (Hulse–Taylor) & $-2.4025\times10^{-12}$ s/s (energy loss via GW) ({https://adsabs.harvard.edu/pdf/2005ASPC..328...25W#:~:text=predicted%20orbital%20period%20derivative%20due,to%20gravitational%20radiation%20computed%20from}{}) & ~ $0$ s/s (no built-in wave emission) & $-2.4056(51)\times10^{-12}$ s/s ({https://adsabs.harvard.edu/pdf/2005ASPC..328...25W#:~:text=quan%02tities%2C%20including%20the%20distance%20and,Hence%20P%CB%99%20b%2Ccorrected}{}) ({https://adsabs.harvard.edu/pdf/2005ASPC..328...25W#:~:text=b%2CGR%20%3D%201,and%20theoretical%20orbital%20decays%20are}{}) & ~100% (fails) \\
        PSR J0737–3039A/B (Double Pulsar) & $-1.252\times10^{-12}$ s/s (predicted) & ~ $0$ s/s & $-1.252(17)\times10^{-12}$ s/s (measured) & ~100% (fails) \\
        GW150914 (Binary BH merger) & 3 solar mass energy radiated in GWs & No GW (merger dynamics unclear) & Detected GWs, strain amplitude $~10^{-21}$ ({http://ui.adsabs.harvard.edu/abs/2010ApJ...722.1030W/abstract#:~:text=Timing%20Measurements%20of%20the%20Relativistic,radiation%20damping%20in%20general}{Timing Measurements of the Relativistic Binary Pulsar PSR B1913+16}) & Complete miss \\
        Precession Effect & GR Prediction (mas/yr) & VAM Prediction (mas/yr) & Observed (mas/yr) & VAM vs Obs \\
        Geodetic (de Sitter) & $6606.1$ mas/yr (forward in orbit plane) ({https://arxiv.org/abs/1105.3456#:~:text=Analysis%20of%20the%20data%20from,9%20rad}{[1105.3456] Gravity Probe B: Final Results of a Space Experiment to Test General Relativity}) & Not explicitly derived (possibly 0 without extension) & $6601.8 \pm 18.3$ mas/yr ({https://arxiv.org/abs/1105.3456#:~:text=Analysis%20of%20the%20data%20from,9%20rad}{[1105.3456] Gravity Probe B: Final Results of a Space Experiment to Test General Relativity}) & ~100% error if 0 \\
        Frame-Dragging & $39.2$ mas/yr (around Earth's spin axis) ({https://arxiv.org/abs/1105.3456#:~:text=Analysis%20of%20the%20data%20from,9%20rad}{[1105.3456] Gravity Probe B: Final Results of a Space Experiment to Test General Relativity}) & $≈39$ mas/yr (by matching formula) ({file://file-2nnnwmwbptvdvnbeqndhbe%23:~:text=2,2gj/}{GR_in_3d.pdf}) & $37.2 \pm 7.2$ mas/yr ({https://arxiv.org/abs/1105.3456#:~:text=Analysis%20of%20the%20data%20from,9%20rad}{[1105.3456] Gravity Probe B: Final Results of a Space Experiment to Test General Relativity}) & ~0% (within error) \\
        Phenomenon & GR Prediction & Formula & VAM Prediction & Formula & Observation (with ref) & Agreement? (Error) \\
        Gravitational Time Dilation (Static field) & $dτ/dt=\sqrt{1-2GM/(rc^2)}$ ({file://file-2nnnwmwbptvdvnbeqndhbe%23:~:text=2/}{GR_in_3d.pdf}) & $dτ/dt=\sqrt{1-Ω^2 r^2/c^2}$ (with $Ω r = v_φ$) ({file://xn--file-2nnnwmwbptvdvnbeqndhbe%23:~:text=%201%202r2-yx7a2275psla/}{GR_in_3d.pdf}) & GPS clocks: Δν/ν = $6.9×10^{-10}$ (Earth) ({https://www.einstein-online.info/en/spotlight/redshift_white_dwarfs/#:~:text=The%20gravitational%20redshift%20was%20first,the%20field%20of%20the%20sun}{Gravitational redshift and White Dwarf stars «  Einstein-Online}), Pound–Rebka: $2.5×10^{-15}$ ({https://en.wikipedia.org/wiki/Tests_of_general_relativity#:~:text=line%20width,first%20precision%20experiments%20testing%20general}{Tests of general relativity - Wikipedia}) (both match GR) & Yes (VAM tuned, 0% error) \\
        Velocity Time Dilation (SR) & $dτ/dt=\sqrt{1-v^2/c^2}$ & $dτ/dt=\sqrt{1-v^2/c^2}$ (same as GR/SR) ({file://xn--file-2nnnwmwbptvdvnbeqndhbe%23:~:text=d-tl1a/}{GR_in_3d.pdf}) & Particle accelerators, muon lifetime (time dilation confirmed to $10^{-8}$) & Yes (identical) \\
        Rotational (Kinetic) Time Dilation & (Included as mass-energy in GR implicitly) & $dτ/dt=(1+\tfrac{1}{2}\beta I Ω^2)^{-1}$ ({file://file-2nnnwmwbptvdvnbeqndhbe%23:~:text=/}{GR_in_3d.pdf}) (VAM heuristic) & No direct obs; fast pulsar (~716 Hz) slows time ~0.5% (GR via mass-energy; VAM via rotation) & In principle (if β set right) \\
        Gravitational Redshift & $z=(1-2GM/(rc^2))^{-1/2}-1$ ({file://file-2nnnwmwbptvdvnbeqndhbe%23:~:text=2/}{GR_in_3d.pdf}) & $z=(1-v_φ^2/c^2)^{-1/2}-1$ ({file://xn--file-2nnnwmwbptvdvnbeqndhbe%23:~:text=%201-py51a/}{GR_in_3d.pdf}) & Solar redshift $2.12×10^{-6}$ ({https://www.sciencedirect.com/science/article/abs/pii/S1384107614000190#:~:text=On%20the%20gravitational%20redshift%20,in%201908%E2%80%94is%20still%20an}{On the gravitational redshift - ScienceDirect.com}); Sirius B $5×10^{-5}$ ({https://www.einstein-online.info/en/spotlight/redshift_white_dwarfs/#:~:text=Consequently%2C%20the%20shift%20should%20be,image%20was%20taken%20with%20the}{Gravitational redshift and White Dwarf stars «  Einstein-Online}); both ~match GR & Yes (VAM=GR) \\
        Light Deflection (Sun) & $\delta = 4GM/(Rc^2) = 1.75″$ ([Testing General Relativity & Total Solar Eclipse 2017]({https://eclipse2017.nasa.gov/testing-general-relativity#:~:text=where%20for%20the%20sun%20we,9x108%20meters}{https://eclipse2017.nasa.gov/testing-general-relativity#:~:text=where%20for%20the%20sun%20we,9x108%20meters})) & $\delta = 4GM/(Rc^2)$ ({file://file-2nnnwmwbptvdvnbeqndhbe%23:~:text=given%20by:/}{GR_in_3d.pdf}) & $1.75″\pm0.07″$ (VLBI) ([Testing General Relativity \\
        Perihelion Precession (Mercury) & $Δϖ = 6πGM/[a(1-e^2)c^2] = 42.98″/$cent. ({https://math.ucr.edu/home/baez/physics/Relativity/GR/mercury_orbit.html#:~:text=Mercury%27s%20perihelion%20precession%3A%20%20,3%20arcseconds%2Fcentury}{GR and Mercury's Orbital Precession}) & \textit{Same as GR} ({file://xn--file-2nnnwmwbptvdvnbeqndhbe%23:~:text=vam%20=%206gm-sm9aunx840m/}{GR_in_3d.pdf}) & $43.1″/$cent. (observed) ({https://math.ucr.edu/home/baez/physics/Relativity/GR/mercury_orbit.html#:~:text=Mercury%27s%20perihelion%20precession%3A%20%20,3%20arcseconds%2Fcentury}{GR and Mercury's Orbital Precession}) & Yes (exact within error) \\
        Frame-Dragging (Earth LT) & $Ω_{LT}=2GJ/(c^2 r^3)$ ({file://file-2nnnwmwbptvdvnbeqndhbe%23:~:text=2,2gj/}{GR_in_3d.pdf}) (≈39 mas/yr) & $Ω_{drag}=\frac{4}{5}\frac{GMΩ}{c^2 r}$ ({file://xn--file-2nnnwmwbptvdvnbeqndhbe%23:~:text=vam%20drag%20-uz9a/}{GR_in_3d.pdf}) (gives same 39 mas/yr) & $37.2±7.2$ mas/yr (GP-B) ({https://arxiv.org/abs/1105.3456#:~:text=Analysis%20of%20the%20data%20from,9%20rad}{[1105.3456] Gravity Probe B: Final Results of a Space Experiment to Test General Relativity}) & Yes (~0%, within 1σ) \\
        Geodetic Precession (Earth) & $Ω_{geo} = \frac{3}{2}\frac{GM}{c^2 r^3} v$ (6606 mas/yr) & \textit{Not derived} (flat space → 0 without extra assumption) & $6601.8±18$ mas/yr (GP-B) ({https://arxiv.org/abs/1105.3456#:~:text=Analysis%20of%20the%20data%20from,9%20rad}{[1105.3456] Gravity Probe B: Final Results of a Space Experiment to Test General Relativity}) & No (VAM missing effect) \\
        ISCO Radius (Schwarzschild BH) & $r_{ISCO}=6GM/c^2$ (for test mass) & No natural ISCO (orbits possible until horizon) & Fe Kα disk lines, GW inspiral waves → match 6GM/c^2 (GR confirmed) & No (needs extra mechanism) \\
        Gravitational Wave Emission (Binary) & Energy loss via quadrupole (P_dot matches to 0.2%) ({https://adsabs.harvard.edu/pdf/2005ASPC..328...25W#:~:text=b%2CGR%20%3D%201,and%20theoretical%20orbital%20decays%20are}{}) & \textit{No GW} (stable or slowly decaying orbit) & PSR1913–16: $-2.405\times10^{-12}$ ({https://adsabs.harvard.edu/pdf/2005ASPC..328...25W#:~:text=quan%02tities%2C%20including%20the%20distance%20and,Hence%20P%CB%99%20b%2Ccorrected}{}) (100% of GR); GW150914: direct detection & No (missing entirely) \\
        \bottomrule
    \end{tabular}
    \caption{}
    \label{tab:}
\end{table}From the above, we see that VAM successfully reproduces all the classical tests of GR in the static or stationary regime: gravitational redshift, light bending, Mercury's precession, and even frame-dragging are all matched to first-order accuracy. This is a non-trivial achievement for a model without spacetime curvature. It suggests that VAM is a viable \textit{effective theory} in the regime of slowly varying or static gravitational fields, providing a fresh conceptual interpretation (gravity from vorticity and pressure, time dilation from kinetic energy) without contradicting empirical data in those domains. The agreement in numerical results is often exact or within observational uncertainty, as highlighted in green in the tables (Time Dilation, Redshift, Deflection, Precession, Frame-dragging).
However, in more dynamic and strong-field regimes, VAM in its current form struggles or fails:
\begin{itemize}\item 
Gravitational waves and inspiral decay: VAM lacks a mechanism for radiating gravitational energy, leading to a gross discrepancy (100% error) for binary pulsar orbital decay and inability to explain LIGO observations. \textit{Proposed fix:} Introduce a compressible or elastic component to the æther to allow æther-wave emission carrying away energy. This essentially means adding a gravitational radiation degree of freedom. The \grqq superfluid æther\textquotedblright could support sound or vortex waves that play the role of gravitational waves if carefully tuned (wave speed = c, quadrupole coupling matching GR's). This is a significant extension, but without it VAM cannot be complete. Future work could attempt a derivation of wave equations from the VAM Lagrangian (if one exists) or couple VAM to an analog of the electromagnetic field equations (since in some sense GR's gravitational waves are analogous to EM waves from accelerated charges).
\item 
Geodetic (de Sitter) precession: VAM as given does not automatically produce the 6.6 arcsec/year drift of a gyroscope in Earth orbit. This is a consequence of GR's curved spacetime around a mass. \textit{Proposed fix:} Incorporate the effect via postulating that moving through the inhomogeneous æther flow causes a spin precession. One idea is to enforce that the spin is Fermi–Walker transported with respect to the æther's local rotating frame. Technically, add a term $-\frac{1}{2}(\nabla \times \mathbf{v}\textit{{æther})$ to the precession of spin. With Earth's field, this could yield the correct value (since $\nabla \times \mathbf{v}}{æther}$ relates to orbital angular velocity for a circular orbit). This fix basically inserts the equivalence principle's prediction into VAM. It would need to be tested whether it's consistent with other aspects of the model. Alternatively, if one formulates VAM's equations of motion from a Lagrangian, the spin precession could emerge as a natural effect (via the Mathisson–Papapetrou equations analog). This is an area for model development.
\item 
Innermost stable orbits and strong-field tests: VAM currently has no concept of an unstable orbit radius (ISCO) because space isn't curved; a test particle could orbit ever closer as long as it goes faster. GR predicts specific radii for ISCO (3 Schwarzschild radii for non-rotating BH), photon sphere, etc., which have observational support (e.g. the black hole shadow size is ~2.6 r_s consistent with photon sphere of GR). \textit{Proposed fix:} Introduce a stability criterion for æther vortex orbits. For instance, require that when the orbital speed approaches some fraction of $c$ (or when $v_\phi$ of æther is high), perturbations grow. This could be justified by fluid dynamics: perhaps beyond a certain shear, the flow cannot remain stable (analogous to an object exceeding the speed of sound in a fluid causing shockwaves). If one can derive a radial effective potential for VAM orbits (accounting for fluid-induced forces like lift or drag on the orbiting object's æther coupling), one might find a radius where $\partial^2 V_\text{eff}/\partial r^2 =0$ akin to ISCO. Tuning that to equal 6GM/c^2 would let VAM agree with accretion disk inner edges and merger dynamics. Without a relativistic treatment, VAM might also allow photons to orbit arbitrarily close (in GR photon sphere is at 3GM/c^2). So VAM needs augmentation to handle these high-speed regimes. Possibly, once gravitational radiation is included, any orbit too close will rapidly inspiral (thus effectively disallowing stable orbits < ISCO). In GR, instability leads to inspiral; in VAM, if we add gravitational radiation, orbits below a certain radius could lose energy so fast they can't stay in circular orbit – effectively creating an ISCO. This is a plausible way VAM plus radiation could mimic ISCO phenomenology.
\item 
Precision deviations: So far, VAM has been tuned to match \textit{leading-order} effects. Higher-order post-Newtonian effects (like the 2PN and 3PN parameters measured in binary pulsars) are not discussed. GR has specific values for these (parameterized post-Newtonian coefficients); VAM would produce its own via the fluid model. It's unknown if they'd match. Given VAM matched 1PN (perihelion, redshift) and frame-dragging (1.5PN gravitomagnetic effect) by design, one hopes it could match further coefficients, but this would require more complex analysis. If any discrepancy arises at the $10^{-3}$ level in binary pulsars or future solar-system tests (e.g. proposed missions to test frame-dragging and higher PN terms), VAM might need additional scaling factors. For example, the coupling constant γ (vorticity–gravity coupling) could be environment-dependent or there might be an undiscovered second-order term in the pressure potential. So far, current data doesn't demand such, but it's something to watch.
\item 
Quantum scale adjustments: VAM introduced a scale-dependent $\mu(r)$ that effectively reduces gravity for $r < 1$ mm to avoid huge gravitational effects of microscopic vortices ({file://xn--file-2nnnwmwbptvdvnbeqndhbe%23:~:text=-zn0a/}{GR_in_3d.pdf}). This predicts a tiny violation of Newton's inverse-square law at sub-millimeter distances. Experiments have tested gravity to ~0.05 mm and seen no violation at the ~1% level. If $\mu(r)$ starts deviating at 1 mm, that might already be too high. Adjustment: push $r^\textit{$ smaller (maybe microns or smaller) so that no deviation is seen at 0.05 mm. Alternatively, make $\mu(r)$ transition so gradually that it's undetectable at current precision. This does not affect astrophysical outcomes (since for anything bigger than dust, $r > r^}$ and μ=1). It's a flexibility in VAM that can be tuned to avoid conflict, but worth noting as a prediction: VAM in principle allows that at very small scales gravity could weaken. Future sub-mm experiments could either find nothing (forcing VAM to confine $\mu$ transition to even smaller scales) or find something new (which VAM could then claim as evidence of æther core structure!). In short, $\mu(r)$ (or analogous scaling terms) should be considered free parameters to be constrained by experiment.
\end{itemize}Recommendations: To advance VAM as a serious competitor or supplement to GR, the following steps are recommended:
\begin{enumerate}\item 
Incorporate Gravitational Radiation: Develop the dynamic equations for the æther that yield wave solutions. Perhaps start with small perturbations of the vortex flow and see if they propagate. Ensure the propagation speed is $c$ and the polarization is transverse (if possible). Introduce a small compressibility or additional field in the fluid Lagrangian to allow this. Calibrate the power output to match the GR quadrupole formula for binaries ({https://adsabs.harvard.edu/pdf/2005ASPC..328...25W#:~:text=b%2CGR%20%3D%201,and%20theoretical%20orbital%20decays%20are}{}). If VAM can naturally produce \grqq fluid gravitational waves\textquotedblright with the same effects, it would no longer be at odds with pulsar data or LIGO.
\item 
Spin Dynamics in Æther: Formulate how spin (or any vector attached to a free-falling body) evolves. The goal is to derive geodetic precession. One approach is to demand that in the limit of small gravity (a nearly inertial frame moving in a field), the formula reduces to the known Thomas precession plus gravitational potential effect. Verify that for an Earth orbit, it gives 6600 mas/yr. This might involve defining a \grqq connection\textquotedblright for the æther's frame – effectively an analog of the Levi-Civita connection in GR, but here related to velocity field gradients. A successful derivation would solidify VAM's viability in explaining gyroscope experiments (GP-B) and also the behavior of spin in binary pulsars (which can cause changes in pulse shape due to geodetic precession of the pulsar spin axis, as observed in the Hulse–Taylor pulsar's 30-year data).
\item 
Explore Strong-Field Solutions: Does VAM have an analog of a black hole (perhaps a very tightly wound vortex with $v_\phi \to c$ at some radius)? And if so, what is the structure of orbits around it? Try to simulate or compute particle trajectories in a strong VAM field (maybe via numerical fluid simulations). See if an ISCO arises or if not, how to impose one (perhaps via a stability criterion or inclusion of radiation reaction which would destabilize orbits below a certain radius). Compare these with GR's predictions to ensure consistency with astrophysical observations like the black hole shadow or accretion disk spectra.
\item 
Refine Coupling Constants: Ensure the single coupling constant (Newton's $G$ equivalent) is truly constant across contexts. VAM hints that $G$ could emerge from more fundamental quantities (like $C_e$, $r_c$, etc., as in their expression $G = \frac{C_e c}{5 t_p^2 F_{\max} r_c^2}$ in the table ({file://file-2nnnwmwbptvdvnbeqndhbe%23:~:text=gravity%20constant%20vam%20g%20=,cec/}{GR_in_3d.pdf}) – though that appears incomplete in the snippet, possibly involving Planck time $t_p$). If $G$ in VAM had any environmental dependence (due to æther density differences), that would violate precision tests of the equivalence principle and inverse-square law. So likely $G$ is constant. But $\gamma$ (vorticity–gravity coupling) and $\beta$ (rotational time dilation coupling) should be pinned down by matching one set of data (say Earth's field or atomic clock gravitational shift), then \textit{used} for others (don't treat them as free per phenomenon). So far, we implicitly did that (e.g., $β$ was chosen so Earth's gravitational redshift comes out right; that same $β$ then predicted NS rotational time dilation fraction – which seemed okay). This consistency check should continue as more phenomena are tackled.
\item 
Testable Deviations: Identify any unique prediction of VAM that \textit{differs} from GR but hasn't been measured yet. Currently, VAM was built to imitate GR in known areas, so by design it doesn't differ in already measured quantities. But perhaps VAM might imply something novel in a regime not yet tested. For example, does VAM predict any frequency-dependent gravitational lensing (we suspect not if the æther flow is non-dispersive, but worth checking)? Does VAM allow for violations of Lorentz invariance at very high energies (since æther provides a preferred frame)? Maybe tiny anisotropies in the speed of light at 10^–20 level could show up – though none have been seen down to 10^–15. If VAM demands an æther rest frame, ultra-high energy cosmic rays or astrophysical polarization might reveal it. These could be potential separators of VAM vs pure GR. If none exist within current reach, VAM remains empirically viable (with the noted extensions for waves).
\end{enumerate}Conclusion: The Vortex Æther Model, when calibrated appropriately, mirrors General Relativity's predictions for a wide array of classical tests: it yields the correct gravitational time dilation (treating it as an effect of æther swirl speed and rotational energy) ({file://file-2nnnwmwbptvdvnbeqndhbe%23:~:text=2/}{GR_in_3d.pdf}) ({file://xn--file-2nnnwmwbptvdvnbeqndhbe%23:~:text=%201%202r2-yx7a2275psla/}{GR_in_3d.pdf}), the correct gravitational redshift formula ({file://file-2nnnwmwbptvdvnbeqndhbe%23:~:text=/}{GR_in_3d.pdf}) ({file://file-2nnnwmwbptvdvnbeqndhbe%23:~:text=vam%20z%20=/}{GR_in_3d.pdf}), the full light bending angle ({file://file-2nnnwmwbptvdvnbeqndhbe%23:~:text=a/}{GR_in_3d.pdf}) ({file://file-2nnnwmwbptvdvnbeqndhbe%23:~:text=2,4gm/}{GR_in_3d.pdf}), the perihelion precession of orbits ({file://xn--file-2nnnwmwbptvdvnbeqndhbe%23:~:text=vam%20=%206gm-sm9aunx840m/}{GR_in_3d.pdf}), and even the frame-dragging Lense–Thirring precession ({file://file-2nnnwmwbptvdvnbeqndhbe%23:~:text=2,2gj/}{GR_in_3d.pdf}) ({file://xn--file-2nnnwmwbptvdvnbeqndhbe%23:~:text=2gi-0o2au1a/}{GR_in_3d.pdf}). Numerically, VAM can be in excellent agreement (0–0.5% error) with observations in these regimes, essentially because the model was constructed to reproduce the post-Newtonian expansion of GR up to 1.5PN order in a different guise. This is a significant validation of VAM's design: it shows that a clever fluid analogy can encode the physics of gravity at least in the low-field limit.
Where VAM falls short is in dynamical strong-field phenomena: GR's predictions of gravitational waves and certain relativistic precessions have been spectacularly confirmed ({https://adsabs.harvard.edu/pdf/2005ASPC..328...25W#:~:text=b%2CGR%20%3D%201,and%20theoretical%20orbital%20decays%20are}{}) ({https://arxiv.org/abs/1105.3456#:~:text=Analysis%20of%20the%20data%20from,9%20rad}{[1105.3456] Gravity Probe B: Final Results of a Space Experiment to Test General Relativity}), and VAM needs substantial augmentation to account for them. We have proposed that by adjusting the underlying æther model – allowing compressibility (to permit wave emission) and including spin transport effects – these gaps can be filled. Doing so will likely complicate the model (losing some of its initial simplicity of a perfectly steady superfluid flow), but it is a necessary evolution if VAM is to remain viable in the era of gravitational wave astronomy.
Another challenge for VAM is conceptual: while GR is a tightly constrained theory (based on the Einstein field equations), VAM introduces several free functions (the density profile of the vortex, the coupling constants, etc.) that can be tuned. We must ensure these parameters are not tuned \textit{ad hoc} for each phenomenon but rather fixed by one experiment and then successfully predicting others. The work so far suggests they can be fixed (e.g., once $γ$ is set to match Earth's $GM$, it correctly gives Mercury's precession, light bending, etc.). The addition of gravitational radiation will introduce at least one new parameter (analogous to æther bulk modulus). That parameter should be set by one observation (say the Hulse–Taylor pulsar decay) and then should predict all other radiative phenomena consistently (LIGO waveforms, etc.). That will be a stringent test of any radiative extension of VAM.
In conclusion, VAM can be made to agree with GR on all presently measured counts if the above corrections are implemented. Its successes in duplicating known results are encouraging – it provides a novel interpretation of gravity as an emergent hydrodynamic effect, which might illuminate links to quantum fluids or offer insights into the nature of spacetime. But at the same time, VAM does not currently outperform GR in any domain; at best it matches it, and at worst (gravitational waves) it currently fails where GR succeeds brilliantly. Therefore, from a pragmatic standpoint, GR remains the superior model given its comprehensive, self-consistent framework and zero-adjustable parameters for these tests.
The value of this benchmarking lies in identifying how VAM must evolve: it suggests specific physical additions (like a wave-carrying mechanism and refined vortex dynamics) to patch the holes. If those patches can be consistently applied, VAM might serve as an alternative conceptual model that reproduces all of Einstein's theory results without invoking spacetime curvature. That could have theoretical appeal, for instance in attempting to unify gravity with fluid analogs of quantum phenomena.
For now, we can conclude that the Vortex Æther Model largely reproduces the empirical successes of GR in static and stationary regimes (to high precision), but fails in dynamic strong-field regimes unless modified. With proposed corrections – adjusting angular momentum treatment, introducing an æther wave (gravitational radiation) channel, refining density and coupling profiles – VAM has the potential to match all tested aspects of gravity. Implementing and validating these fixes is the next step to assess whether VAM can indeed be a complete theory or at least a working effective model of gravitation.
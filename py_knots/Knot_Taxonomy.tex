\documentclass{article}
\usepackage{amssymb}
\usepackage{geometry}
\usepackage{longtable}
\usepackage{xcolor}
\usepackage{hyperref}

\begin{document}

\section{Symmetry Classification of Knot-based Vortex Structures in the Vortex Æther Model (VAM)}

\begin{table}[H]
\caption{
\textbf{Known Symmetries of Prime Knots as VAM Vortex Structures.}
This table catalogs the discrete symmetries of low-crossing-number prime knots, interpreted as possible stable knotted vortex configurations in the Vortex Æther Model (VAM). Columns show the principal symmetry groups ($D_2(r)$, $D_{2k}$, $Z_{2k}$, $I$), reversibility, amphichirality, allowed periods, and the full symmetry group (FSG).
}
\centering

\renewcommand{\arraystretch}{1.15}
\setlength{\tabcolsep}{0.45em}

\begin{longtable}{lcccccccc}
\hline
\textbf{ } & $\mathrm{D}_2(r)$ & $\mathrm{D}_{2k}$ & $\mathrm{Z}_{2k}$ & $I$ & reversible & amphichiral & periods & FSG \\
\hline
\hyperlink{3_1}{$3_1$}                & \checkmark & $D_4, D_6$      & $\text{\texttimes}$ & $\text{\texttimes}$ & \checkmark & $\text{\texttimes}$ & $2, 3$   & $Z_2$ \\
\hyperlink{4_1}{$4_1$}                & \checkmark & $D_4$           & $Z_4$               & $I_8$               & \checkmark & \checkmark          & $2$      & $D_8$ \\
\hyperlink{5_1}{$5_1$}                & \checkmark & $D_4, D_{10}$   & $\text{\texttimes}$ & $\text{\texttimes}$ & \checkmark & $\text{\texttimes}$ & $2, 5$   & $Z_2$ \\
\hyperlink{5_2,6_1,6_2}{$5_2, 6_1, 6_2$} & \checkmark & $D_4$           & $\text{\texttimes}$ & $\text{\texttimes}$ & \checkmark & $\text{\texttimes}$ & $2$      & $D_4$ \\
\hyperlink{6_3}{$6_3$}                & \checkmark & $D_4$           & $Z_4$               &                     & \checkmark & \checkmark          & $2$      & $D_8$ \\
\hyperlink{7_1}{$7_1$}                & \checkmark & $D_4, D_{14}$   & $\text{\texttimes}$ & $\text{\texttimes}$ & \checkmark & $\text{\texttimes}$ & $2, 7$   & $Z_2$ \\
\hyperlink{7_2,7_3}{$7_2, 7_3$}       & \checkmark & $D_4$           & $\text{\texttimes}$ & $\text{\texttimes}$ & \checkmark & $\text{\texttimes}$ & $2$      & $D_4$ \\
\hyperlink{7_4}{$7_4$}                & \checkmark & $D_4$           & $\text{\texttimes}$ & $\text{\texttimes}$ & \checkmark & $\text{\texttimes}$ & $2$      & $D_8$ \\
\hyperlink{7_5,7_6}{$7_5, 7_6$}       & \checkmark & $D_4$           & $\text{\texttimes}$ & $\text{\texttimes}$ & \checkmark & $\text{\texttimes}$ & $2$      & $D_4$ \\
\hyperlink{7_7}{$7_7$}                & \checkmark & $D_4$           & $\text{\texttimes}$ & $\text{\texttimes}$ & \checkmark & $\text{\texttimes}$ & $2$      & $D_8$ \\
\hyperlink{8_1,8_2}{$8_1, 8_2$}       & \checkmark & $D_4$           & $\text{\texttimes}$ & $\text{\texttimes}$ & \checkmark & $\text{\texttimes}$ & $2$      & $D_4$ \\
\hyperlink{8_3}{$8_3$}                & \checkmark & $D_4$           & $Z_4$               & $I_8$               & \checkmark & \checkmark          & $2$      & $D_8$ \\
\hyperlink{8_4,8_5,8_6,8_7,8_8}{$8_4, 8_5, 8_6, 8_7, 8_8$}
                                      & \checkmark & $D_4$           & $\text{\texttimes}$ & $\text{\texttimes}$ & \checkmark & $\text{\texttimes}$ & $2$      & $D_4$ \\
\hyperlink{8_9}{$8_9$}                & \checkmark & $D_4$           &                     & $I_4$               & \checkmark & \checkmark          & $2$      & $D_8$ \\
\hyperlink{8_{10}}{$8_{10}$}          & \checkmark & $\text{\texttimes}$ & $\text{\texttimes}$ & $\text{\texttimes}$ & \checkmark & $\text{\texttimes}$ & none     & $D_2$ \\
\hyperlink{8_{11}}{$8_{11}$}          & \checkmark & $D_4$           & $\text{\texttimes}$ & $\text{\texttimes}$ & \checkmark & $\text{\texttimes}$ & $2$      & $D_4$ \\
\hyperlink{8_{12}}{$8_{12}$}          & \checkmark & $D_4$           & $Z_4$               &                     & \checkmark & \checkmark          & $2$      & $D_8$ \\
\hyperlink{8_{13},8_{14},8_{15}}{$8_{13}, 8_{14}, 8_{15}$}
                                      & \checkmark & $D_4$           & $\text{\texttimes}$ & $\text{\texttimes}$ & \checkmark & $\text{\texttimes}$ & $2$      & $D_4$ \\
\hyperlink{8_{16}}{$8_{16}$}          & \checkmark & $\text{\texttimes}$ & $\text{\texttimes}$ & $\text{\texttimes}$ & \checkmark & $\text{\texttimes}$ & none     & $D_2$ \\
\hyperlink{8_{17}}{$8_{17}$}          & $\text{\texttimes}$ & $\text{\texttimes}$ & $\text{\texttimes}$ &               & $\text{\texttimes}$ & \checkmark & none & $D_2$ \\
\hyperlink{8_{18}}{$8_{18}$}          & \checkmark & $D_4, D_8$      & $Z_8$               &                     & \checkmark & \checkmark          & $2, 4$   & $D_{16}$ \\
\hyperlink{8_{19}}{$8_{19}$}          & \checkmark & $D_4, D_6, D_8$ & $\text{\texttimes}$ & $\text{\texttimes}$ & \checkmark & $\text{\texttimes}$ & $2, 3, 4$& $Z_2$ \\
\hyperlink{8_{20}}{$8_{20}$}          & \checkmark & $\text{\texttimes}$ & $\text{\texttimes}$ & $\text{\texttimes}$ & \checkmark & $\text{\texttimes}$ & none     & $D_2$ \\
\hyperlink{8_{21}}{$8_{21}$}          & \checkmark & $D_4$           & $\text{\texttimes}$ & $\text{\texttimes}$ & \checkmark & $\text{\texttimes}$ & $2$      & $D_4$ \\
\hyperlink{12a_{1202}}{$12a_{1202}$}  & \checkmark &                 & $Z_2, Z_6$          &                     & \checkmark & \checkmark          &          & $D_{12}$ \\
\hyperlink{15331}{$15331$}            &           &                 & $Z_2$               &                     &           & \checkmark          &          &         \\
\hline
\end{longtable}
\end{table}


\\
\textit{In VAM, these symmetries classify the invariance properties of knotted ætheric filaments, constraining their physical stability, energy quantization, and possible transformation pathways.}
\noindent
\textit{Remarks.}\\
Any $D_{2k}$ symmetry ($k\geq2$) implies $D_2(r)$ symmetry; if $k$ is even, period 2 is also present. $D_{2k}$ symmetry further implies $D_{2j}$ for divisors $j$ of $k$. $Z_{2k}$ symmetry entails positive amphichirality; $D_2(r)$ guarantees reversibility. $I_2$ symmetry implies negative amphichirality. These properties map directly to constraints on vortex knot energy spectra, fusion/interconversion rules, and topological charge conservation in VAM. The "full symmetry group" (FSG) is tabulated for comparison, though it may not capture all VAM-relevant invariances.

\vspace{1em}

\textit{Note.}\\
Knots such as $8_{10}$, $8_{16}$, $8_{17}$, and $8_{20}$, for which period 2 is absent, also uniquely have FSG $D_2$ among prime knots with 8 or fewer crossings, reflecting special restrictions on allowable vortex periodicities and energy levels in the æther. Exceptional knots $12a_{1202}$ and $15331$ are included for their rare $Z_2$ symmetry, potentially corresponding to novel or unanticipated ætheric field states.

\section{Glossary of Symmetry Table Symbols}

\begin{description}
    \item[\( D_2(r) \)] \textbf{Order-2 Dihedral (Reflectional) Symmetry.}
    The knot (or vortex structure) admits a dihedral symmetry of order 2, meaning it is invariant under a 180° rotation and a reflection; this often guarantees \emph{reversibility} (the knot is equivalent to itself with reversed orientation).

    \item[\( D_{2k} \)] \textbf{Higher-Order Dihedral Symmetry.}
    The knot is invariant under the full dihedral group of order \( 2k \), i.e., all rotations by \( 2\pi/k \) and reflections. In VAM, this corresponds to invariance under both cyclic flows and chirality-reversing operations.

    \item[\( Z_{2k} \)] \textbf{Cyclic Symmetry of Order \( 2k \).}
    The knot admits rotational symmetry by \( 2\pi/(2k) \) (and its multiples), but not necessarily reflection symmetry. In VAM, such symmetry is associated with periodic vortex phase cycling and often positive amphichirality.

    \item[\( I \)] \textbf{Icosahedral Symmetry or Inversion.}
    \( I \) often indicates additional point group symmetries (such as icosahedral, dodecahedral, or inversion symmetries), depending on the context. In tables, it may specify inversion axes or particular symmetry orders, e.g., \( I_8 \), \( I_4 \).

    \item[reversible] \textbf{Reversible Knot (Vortex).}
    The knot is topologically equivalent to itself with the orientation reversed; in VAM, this reflects invariance under reversal of circulation or “vortex time”.

    \item[amphichiral] \textbf{Amphichiral (Mirror-Image) Symmetry.}
    The knot is equivalent to its mirror image:
    \begin{itemize}\item \emph{Positive amphichirality} usually corresponds to cyclic (\( Z_{2k} \)) symmetry.
    \item \emph{Negative amphichirality} is sometimes indicated by special inversion (\( I_2 \)).
    \end{itemize}
     
     
    \item[periods] \textbf{Periods of Symmetry.}
    Lists the possible orders of cyclic symmetry—i.e., the integer \( n \) for which the knot is invariant under a \( 2\pi/n \) rotation. In VAM, this relates to allowed quantum/vortex mode numbers.

    \item[FSG] \textbf{Full Symmetry Group (FSG).}
    The maximal discrete symmetry group of the knot, encoding all rotational, reflectional, and inversion symmetries. In VAM, the FSG constrains the topological conservation laws and fusion/annihilation selection rules for knotted vortex states.
\end{description}


\end{document}












\section*{Introduction}
This model conceptualizes the Æther as not a classical vacuum but an energy-rich, inviscid medium whose properties resemble those of a quantum superfluid, fundamentally structured within a fixed, absolute space framework rather than curved spacetime, following the original five postulates of Euclidean geometry.


The term Æther is employed here in its historical sense, as it has long been used to describe an all-pervading medium that facilitates energy transfer.
However, in contrast to earlier mechanistic interpretations, this formulation eschews particulate motion in favor of continuous vorticity evolution. In classical fluid dynamics, vorticity ($\vec{\omega}$) is a measure of local rotation in a fluid flow, defined as $\vec{\omega} = \nabla \times \vec{v}$. In VAM, vorticity represents structured rotational flows within the Æther, governing energy transfer and fundamental forces.

Unlike classical Æther theories, VAM does not assume a rigid transmission medium for light. Instead, it models space as an inviscid, topologically structured superfluid where vorticity replaces spacetime curvature as the mediator of fundamental interactions. This framework provides a natural explanation for quantization, inertia, and possible emergent gravitational effects.

Unlike Special Relativity, where time dilation arises from relative motion, VAM proposes that local time variations result from vortex-induced energy gradients. This alternative mechanism could be tested through rotating Bose-Einstein condensates.



My conviction in this conceptualization was reinforced through an in-depth study of the original contributions of Maxwell~\cite{maxwell1861}, Helmholtz~\cite{helmholtz1858}, Kelvin~\cite{kelvin1867}, and Clausius~\cite{clausius1865}, whose works established the mathematical foundations for vortex dynamics and electromagnetic interactions.

At the core of this model lies the concept of vortex knots—stable, topologically conserved rotational structures within the Æther.
In particular, atomic structures are envisioned as self-sustaining vortex configurations, such as trefoil knots, encapsulated within spherical equilibrium boundaries.
These knotted vortices exhibit a rigid-core structure, with surrounding potential flow regions exhibiting rotational and irrotational components.
The dynamics of these vortices are dictated by vorticity conservation principles, rather than mass-energy curvature.
Experimental and theoretical advancements in vortex dynamics suggest that stable knotted vortices can persist in inviscid fluids~\cite{kleckner2013}, reinforcing the notion that atomic structure may emerge from self-sustaining topological vortex configurations.

Further experimental validation of this concept can be found in the behavior of superfluid helium, which exhibits quantized vortices that share striking similarities with the structured vorticity fields predicted by this model.
Superfluid helium provides an example of an inviscid medium where vorticity exists in discrete, quantized states, reinforcing the plausibility of an Ætheric superfluid medium governed by similar principles~\cite{vinen2002}.
The interaction of these quantized vortices, as seen in superfluid turbulence, further supports the hypothesis that a vorticity-based framework can underpin fundamental physical interactions.

A key departure from relativistic formulations is the assertion that time is absolute and flows uniformly throughout the Æther.
However, local variations in vorticity influence time perception, as the rotational dynamics of vortex cores alter local energy distributions and equilibrium states.
This provides an alternative to relativistic time dilation, where accelerations and vorticity gradients—not spacetime curvature—determine time flow differences.
This approach finds further support in studies of vorticity in gravitomagnetism~\cite{cahill2005}, where frame-dragging and precession effects emerge from rotating mass flows rather than from spacetime curvature.
Thus, local time evolution is inherently tied to vorticity gradients and not relativistic spacetime warping.

A central feature of this framework is the thermal expansion and contraction of vortex knots, a principle inspired by Clausius’s mechanical theory of heat.
In this model, atoms and fundamental particles are represented as self-sustaining vortex configurations within spherical equilibrium pressure boundaries.
These knotted vortices interact dynamically with the surrounding Æther, expanding and contracting in response to thermal input, a process mathematically analogous to the expansion of gases under heat.
This fundamental behavior links thermodynamics directly to vorticity, establishing entropy as a function of structured rotational energy.
Studies on equilibrium energy and entropy of vortex filaments~\cite{belik2023} provide strong evidence that vortical structures self-organize by redistributing kinetic energy through vorticity-driven entropy gradients, lending credibility to this perspective.

Additionally, this model provides a natural bridge between quantum mechanics and vortex theory.
The quantization of circulation in superfluid helium offers a direct analogy to the quantized nature of angular momentum in quantum mechanics, suggesting that elementary particles may arise from structured vortex dynamics in the Æther.
The Schrödinger equation, often interpreted as governing probability waves, can instead be viewed as describing the stable, standing wave solutions of vortex structures in the Æther.
This aligns with the observed wave-particle duality, where particles exhibit both localized (vortex core) and delocalized (potential flow) characteristics, depending on observational context.
Furthermore, the emergence of discrete energy levels in atomic systems could be explained through resonant vortex interactions, where stable configurations correspond to eigenmodes of the vortex-boundary system.

At the core of this model is the interaction between entropy, pressure equilibrium, and vortex stability.
The spherical equilibrium boundary surrounding a vortex knot is hypothesized to behave elastically, responding to changes in rotational energy via:

\begin{itemize}
    \item Thermal input → Expansion of the vortex boundary, reducing internal pressure and increasing the system’s entropy.
    \item Energy dissipation → Contraction of the vortex boundary, increasing core density and stabilizing vorticity distributions.
\end{itemize}

This process provides a thermodynamic foundation for vortex structure evolution, supporting a direct analogy between entropy variations and vortex interactions.
The entropy of a vortex configuration is defined as:

\begin{equation*} \label{eq:Entropy}
S \propto \int \omega^2 dV
\end{equation*}

where:

\begin{itemize}
    \item \( S \) is the entropy of the vortex configuration.
    \item \( \omega \)  is the local vorticity field.
    \item The integral is taken over the vortex volume.
\end{itemize}

This equation suggests that entropy is directly related to the vorticity distribution within the Æther, reinforcing the idea that vortex evolution follows thermodynamic principles, rather than requiring mass-energy curvature as in General Relativity.

By extending Clausius’s thermodynamic principles into a vorticity-based gravitational model, this framework establishes a connection between classical thermodynamics, quantum mechanics, and fluid dynamics. Notably:

\begin{itemize}
    \item Thermal expansion-contraction cycles of vortex knots mirror the behaviors observed in gas expansion laws.
    \item Energy transfer within the Æther follows structured vorticity dynamics, rather than being mediated by mass-energy interactions.
    \item Entropy-driven expansion aligns with cosmological models describing universal inflation without requiring dark energy.
\end{itemize}

Part I of this work will present foundational considerations, articulated with the intention of minimizing the necessity for advanced mathematical understanding, thereby making the content accessible to a broader audience.

Part II will delve into the mathematical formalism underpinning the model, utilizing approaches such as the Bragg-Hawthorne equation in spherical symmetry~\cite{keller2024} to formalize the equilibrium dynamics of vortex-driven Æther structures.
The ultimate objective is to establish the foundations for a comprehensive non-viscous liquid Æther theory, capable of providing a visual and conceptual representation of inertia as an emergent property of vortex circulation within the Æther, particularly influenced by the proposed constants and the conservation of helicity~\cite{kleckner2016}.

This model offers a novel perspective on the nature of space, energy, and fundamental interactions, providing a coherent framework for future research into the unification of physical forces.
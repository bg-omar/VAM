%! Author = Omar Iskandarani
%! Date = 2/15/2025
\documentclass[a4paper,10pt]{article}
\usepackage[a4paper,margin=1in]{geometry}
\usepackage{array}
\usepackage{booktabs}
\usepackage{amsmath}
\usepackage{amssymb}
\usepackage{graphicx}
\usepackage{hyperref}
\usepackage{physics}
\usepackage{cite}
\usepackage{natbib}


\geometry{margin=1in}


\title{The Vortex \AE ther Model: A Vorticity-Based Framework for Gravity and Electromagnetism}
\author{Omar Iskandarani}
\date{\today}

\begin{document}
    \maketitle

    \maketitle

    \begin{abstract}
        This paper introduces the Vortex \AE ther Model (VAM), a novel approach to fundamental physics where gravity and electromagnetism emerge from vorticity fields in an incompressible, inviscid \AE ther.
        The model presented herein offers a modern interpretation of what is conventionally referred to as \AE ther theory, reimagined as a structured, inviscid superfluid medium governed by vorticity interactions rather than classical particulate motion.
        While the 19th-century concept of a luminiferous \AE ther was rejected following the Michelson-Morley experiment, the fundamental questions it sought to address—concerning the nature of space, energy propagation, and fundamental interactions—remain open.
        This work argues that a contemporary \AE theric model, grounded in fluid dynamics and topological vortex structures, may provide novel insights into quantum mechanics, inertia, and gravity.
        Unlike General Relativity, which relies on spacetime curvature, VAM posits that gravitational attraction arises from pressure gradients induced by vortex filaments.
        Electromagnetism, in turn, is described as a consequence of structured vortex networks, with magnetic fields emerging from circulating \AE ther flows.
        Experimental predictions include measurable frequency shifts in rotating Bose-Einstein condensates and anomalous electromagnetic effects in high-vorticity plasmas.
        Proposed laboratory tests and detection methods are outlined to validate the Vortex \AE ther Model.
        By extending Clausius’s thermodynamic principles into a vorticity-based gravitational model, this framework establishes a connection between classical thermodynamics, quantum mechanics, and fluid dynamics.
        Notably:        - Thermal expansion-contraction cycles of vortex knots mirror the behaviors observed in gas expansion laws.
        - Energy transfer within the \AE ther follows structured vorticity dynamics, rather than being mediated by mass-energy interactions.
        - Entropy-driven expansion aligns with cosmological models describing universal inflation without requiring dark energy.
        This model offers a novel perspective on the nature of space, energy, and fundamental interactions, providing a coherent framework for future research into the unification of physical forces.
    \end{abstract}

    \input{00_introduction}

    \section{Part I}\label{sec:part-1}
    \subsection{The demand for an extension for the propositions of physics}\label{subsec:extension-physics}

Any rigorous consideration of a physical theory must differentiate between objective reality, which exists independently of any theoretical framework, and the physicist's statements that attempt to articulate that theory.
These theoretical statements aim to correspond to objective reality, and it is through these approximations that we attempt to construct an intelligible representation of the universe.
By recognising patterns in nature which are explained with philosophy and mathematics to predict an outcome we created different branches of physics that at first sight seem unrelated, but later get discovered to be fusible.

The contemporary scientific understanding of reality is shaped predominantly by the Theory of Relativity and Modern Physics.
When we inquire whether the descriptions furnished by these theories are exhaustive, it is critical to recognize that such completeness is contingent upon a narrowly defined set of conditions—specifically, the behavior of clocks and measuring rods, as well as the statistical properties of electrons.
Neither the general theory of relativity nor modern physics adequately captures the objective reality of the \ae ther, as both frameworks explicitly dismiss the concept of an \ae ther in favor of a relativistic interpretation.
In contrast, the model presented here emphasizes a non-relativistic, vorticity-driven framework.
The theory of relativity excels in providing a precise account of phenomena such as the rotation of clock hands and, for practical purposes, may well remain unparalleled as a descriptive tool.

In special relativity, simultaneity is defined through the synchronized positions of multiple clocks and the reception of light signals exchanged between them.
We must revise this definition of simultaneity to align with a strictly non-relativistic \ae ther model, taking into consideration that quantum entanglement implies the possibility of non-local transmission of mechanical information within the \ae ther, exceeding the conventional limits imposed by the speed of light.

While the Theory of Relativity provides a precise account of relativistic motion and clock synchronization, it does not accommodate a dynamic \AE ther as a physical medium.
In contrast, this framework postulates an alternative definition of simultaneity, where time flow is not governed by the exchange of light signals but rather by intrinsic vorticity interactions within the \AE ther.

Special Relativity defines simultaneity based on synchronized clocks exchanging light signals.
This model supersedes that definition, introducing a framework in which:

\begin{itemize}
    \item Absolute time exists as a global invariant, yet local time variations arise from structured vorticity interactions.
    \item Vorticity fields regulate temporal flow, producing differential time progression akin to relativistic time dilation but derived from fluid-dynamic principles.
    \item Quantum entanglement does not imply superluminal signal transfer within the \AE ther but suggests a deeper structural connectivity within the medium.
\end{itemize}

The temporal behavior of atomic structures, particularly discrepancies in clock synchronization, is determined by vortex core dynamics.
The fundamental premise is that the atomic nucleus constitutes a vortex-stabilized structure, wherein:

\begin{itemize}
    \item The proton manifests as a Trefoil knot, the simplest stable vortex topology.
    \item The tangential velocity at the vortex boundary follows absolute vorticity conservation, maintaining atomic stability.
\end{itemize}

Knot theory provides a rigorous mathematical foundation for analyzing vortex structures within the \AE ther, linking macroscopic fluid behavior to fundamental particle interactions.
In this model, helicity—a conserved quantity in ideal fluid dynamics—is directly analogous to quantum spin, reinforcing the hypothesis that fundamental particles emerge from structured vorticity.
These knotted configurations in the \ae ther are inherently dynamic, facilitating energy and angular momentum exchange with their surroundings.
Their behavior adheres to the Navier-Stokes equations for inviscid, incompressible flows, modified by absolute vorticity conservation constraints.
This dynamism enables the model to address complex interactions within the \ae ther framework.

To formalize this link between quantized vorticity and energy interactions, we define the governing equations Helicity conservation:

    \begin{equation}
        H = \int_V \vec{\omega} \cdot \vec{v} \, dV\label{eq:HelicityConservation}
    \end{equation}

Energy density of a vortex knot:

    \begin{equation}
        E = \frac{1}{2} \rho \int_V |\vec{\omega}|^2 \, dV\label{eq:EnergyDensity}
    \end{equation}

These equations ensure that vortex configurations exhibit intrinsic stability, thereby providing a physical basis for particle interactions and energy quantization.
The stability of these vortex knots emerges naturally from helicity constraints, leading to quantized field interactions that parallel quantum mechanical principles.

Future research will employ topological invariants such as linking numbers and higher-order polynomial invariants to establish measurable correlations between vortex knottedness, energy states, and fundamental forces.
Extending the physical model to include helicity dynamics and nonlinear \AE ther interactions offers a pathway to synthesize classical fluid mechanics with quantum mechanical principles within a unified, non-relativistic, vorticity-driven framework.

This approach maintains a foundation in Euclidean spatial geometry and absolute time, advancing a framework that transcends the limitations imposed by current relativistic and probabilistic paradigms.
By reconciling fluid dynamics, quantum mechanics, and topological field interactions, this model has the potential to unify physics across multiple scales—from atomic structures to large-scale cosmological phenomena.


This work presents a refined, self-consistent \AE theric framework governed by vorticity dynamics, helicity conservation, and energy quantization.
By establishing fundamental interactions through vortex topology and pressure equilibrium, this theorem offers a novel perspective on atomic structure, time flow modulation, and gravity.
Future research will emphasize experimental validation, numerical simulations, and extended mathematical formalization to further develop the implications of \AE theric vortex dynamics.




    \input{02_observations_on_gr}
    % 1.3. The Vortex Æther Model: A 3D or 5D Framework?

\subsection{The Vortex Æther Model: A 3D or 5D Framework?}\label{subsec:the-vortex-ther-model:-a-3d-or-5d-framework?}
The Vortex \AE ther Model (VAM) proposes an alternative interpretation where simultaneity can be restored as an absolute property, mediated by the intrinsic properties of the \AE ther.
This is a paradigmatic shift in our understanding of fundamental physics, positing structured vorticity fields as the primary mediators of interactions rather than the conventional framework of spacetime curvature.
The local passage of time is influenced by the rotation of vortex cores, altering the progression of atomic clocks due to their internal vorticity and circulation dynamics.
A central theoretical question remains unresolved: should VAM be conceptualized as a 3D model with time (4D) where vorticity is merely an emergent property, or does it necessitate a 5D formalism in which vorticity ($\omega$) constitutes an intrinsic coordinate, akin to spatial dimensions?
Let us examines both perspectives, delineating their theoretical underpinnings and empirical implications.

\subsubsection*{The 3D + Time (4D) Interpretation}
Conventional fluid dynamics and electromagnetism adhere to a three-dimensional Euclidean topology (x, y, z), with time (t) serving as an independent but external parameter governing system evolution.
Within this framework:

\begin{itemize}
    \item Vorticity ($\omega$) is treated as a vector field, contingent upon the velocity field and subject to differential constraints.
    \item The governing equations remain embedded within classical fluid dynamics, interpreting vorticity as a secondary interaction term rather than a fundamental coordinate.
    \item Time ($\tau$) is posited as an absolute parameter, dictating the evolution of vortex dynamics without undergoing intrinsic modulation by vorticity.
    \item Forces such as gravitation and electromagnetism are expressed through potential fields and charge distributions rather than through structured vorticity.
\end{itemize}

From this standpoint, VAM is strictly a 3D model with an additional temporal component (4D), wherein vorticity plays a derivative role rather than an independent ontological entity.
However, this interpretation may impose limitations in capturing the fundamental constraints and emergent behaviors of structured vortex filaments in physical interactions.

\subsubsection*{The 5D Vortex-Structured Interpretation}
An alternative formulation posits that vorticity is not merely a field-dependent property but an intrinsic topological coordinate, necessitating a 5D configuration (x, y, z, $\omega$, $\tau$).
Under this advanced conceptualization:

\begin{itemize}
    \item Vorticity fundamentally governs gravitational and quantum interactions, operating as an alternative to Einsteinian spacetime curvature.
    \item Temporal scaling effects emerge as a function of vorticity magnitude, modulating the local perception of time in vortex-dense domains.
    \item Electromagnetic interactions are recast as vorticity-induced flux phenomena, supplanting conventional charge-motion-based paradigms.
    \item Vortex filaments are reconceptualized as self-organized networks, wherein topology dictates energy exchange, field stability, and force transmission.
    \item Variations in vorticity contribute to the quantization of energy, offering an alternative heuristic to wave-particle duality within quantum mechanics.
\end{itemize}
This perspective aligns with contemporary research into knotted vortices, helicity conservation, and quantized energy transport, all of which suggest that vorticity functions as a primary determinant of physical behavior rather than a secondary consequence of velocity fields.
A 5D formalism provides a robust theoretical foundation for unifying macroscopic fluid dynamics with quantum mechanical structures.

\subsubsection*{Empirical and Theoretical Support for a 5D Model}
\begin{enumerate}
    \item Knotted Vortices in Hydrodynamics
    \begin{itemize}
        \item Experimental results (Kleckner & Irvine, 2013) demonstrate that knotted vortex structures exhibit dynamic evolution independent of classical constraints, implying an intrinsic role for vorticity.
        \item Vortex reconnection processes obey distinct topological conservation principles, reinforcing the notion of vorticity as a fundamental coordinate.
    \end{itemize}

    \item Magnetic Helicity and Plasma Vorticity
    \begin{itemize}
        \item Conservation laws in magnetohydrodynamics indicate that helicity must be preserved in a manner that suggests higher-dimensional structuring of vorticity.
        \item Plasma vortices demonstrate behaviors inconsistent with classical field interpretations, requiring a more robust framework incorporating additional degrees of freedom.
    \end{itemize}

    \item Wave–Vortex Duality and Nonlocality
    \begin{itemize}
        \item Investigations into wave-vortex interactions indicate that vorticity fields exhibit nonlocal constraints, suggesting a fundamental role beyond mere fluid dynamics.
        \item Energy transport via structured vorticity flows may provide a deeper understanding of quantum coherence and wave-particle interactions.
    \end{itemize}

    \item Quantized Vortices in Superfluid Helium
    \begin{itemize}
        \item The discrete nature of vortices in superfluid helium aligns with the hypothesis that vorticity is a quantized, independent coordinate rather than a derived property.
        \item Superfluid vortices suggest a topological underpinning to vorticity-driven phenomena, reinforcing its candidacy as a fundamental coordinate in a 5D model.
    \end{itemize}
\end{enumerate}

\subsubsection*{The Vortex Æther Model as a 5D Framework}
Structured vorticity fields exhibit behaviors that challenge the reductionist interpretations of classical mechanics, particularly with respect to:

\begin{itemize}
    \item Gravitational analogs arising from circulation dynamics.
    \item The modulation of local time perception through absolute vorticity conservation.
    \item The emergence of quantized effects within helicity-driven fields.
    \item Observed parallels between vortex dynamics and quantum field interactions.
\end{itemize}

Given these empirical and theoretical considerations, it is most consistent to classify VAM as a 5D model where vorticity functions as an independent coordinate governing fundamental interactions.
This reformulation expands the conceptual framework of fluid dynamics, gravitation, and electromagnetism, offering new pathways for experimental verification and theoretical synthesis.
By embedding vorticity within a five-dimensional manifold, VAM provides a robust mechanism for bridging classical and quantum descriptions of fundamental forces.


\subsubsection*{Local Time as a Function of Vorticity}\label{subsubsec:local-time-as-a-function-of-vorticity}

\begin{itemize}
    \item Time is not an intrinsic property of the Æther but an emergent consequence of vortex interactions.
    \item The local flow of time is determined by the rotational dynamics of vortex knots: faster rotation leads to slower local time perception.
\end{itemize}

\[dt_{VAM} = \frac{dt}{\sqrt{1 - \frac{C_e^2}{c^2} e^{-r/r_c} - \frac{\Omega^2}{c^2} e^{-r/r_c}}}\]

External vorticity fields modulate core rotation, altering local time perception in a manner consistent with time dilation effects observed in General Relativity.
This formulation suggests that time is a dynamic property of the \AE ther, contingent upon vorticity interactions rather than an absolute, universal parameter.

\subsubsection*{Future Directions and Open Questions}
\begin{itemize}
    \item Can vorticity quantization provide an alternative foundation for quantum mechanics, potentially reformulating the wavefunction in terms of vortex dynamics?
    \item How can structured vortices be experimentally validated as fundamental mediators of force rather than as emergent effects?
    \item Could this framework serve as a unified model encompassing fluid dynamics, electrodynamics, and gravitation?
    \item Might vorticity play a role in the enigmatic nature of dark matter, or offer new explanations for unresolved astrophysical anomalies?
    \item Can a 5D vorticity-based model refine our understanding of entropy transfer and energy conservation in high-energy physics?
\end{itemize}

As VAM continues to evolve, addressing these profound questions will refine its validity as a fundamental physical theory, potentially revolutionizing our understanding of the interplay between classical and quantum realms.
    \input{04_observation_on_light_particle}
    
\subsection{Vorticity in a Simplified ``Rigid-Body'' Model: Relation to the Bohr Model Velocity}\label{subsec:Relation-to-the-Bohr-Model-Velocity}

In fluid mechanics, the vorticity $\boldsymbol{\omega}$ is defined as:

\begin{equation*}
\boldsymbol{\omega} = \nabla \times \mathbf{v}\label{eq:vorticity}
\end{equation*}

where $\mathbf{v}$ is the velocity field of the fluid. To illustrate its role in rotational motion, we consider an idealized rigid-body rotation about the $z$-axis with constant angular velocity $\Omega$. The velocity field at radius $r$ in cylindrical coordinates is:

\begin{equation*}
    \mathbf{v}(r) = \Omega \hat{z} \times \mathbf{r} = \Omega(-y\hat{x} + x\hat{y}) \quad \Rightarrow \quad |
    \mathbf{v}(r)| = \Omega r.\label{eq:cylindrical-velocity}
\end{equation*}

A standard result is that the corresponding vorticity magnitude is:

\begin{equation}
    |\boldsymbol{\omega}| = \left| \nabla \times \mathbf{v} \right| = 2\Omega.\label{eq:vorticity-magnitude}
\end{equation}

Hence, if the tangential (orbital) velocity at radius $r$ is $v_{\text{tangential}} = \Omega r$, the local vorticity is:

\begin{equation*}
    \omega = 2\Omega \quad 2v_{\text{tangential}} = \omega r.\label{eq:2velocity}
\end{equation*}

Thus, one can state that the vorticity is twice the angular velocity or equivalently, ''the vorticity (multiplied by $r$) is twice the tangential velocity.''

\subsubsection*{Standard Bohr Orbit (Classical Picture)}
In the simplified (pre-Schr\"{o}dinger) Bohr model of the hydrogen atom, the electron in the ground state ($n=1$) is classically pictured as moving on a circle of radius $a_0$ (the Bohr radius) with speed $v_\text{bohr}$. This is given by:

\begin{equation*}
    v_\text{bohr} = \alpha c \approx 2.1877 \times 10^6 \text{ m/s},\label{eq:tangential-velocity}
\end{equation*}

where $\alpha \approx 1/137.036$ is the fine-structure constant, and $c \approx 3 \times 10^8 \text{ m/s}$ is the speed of light.

\subsubsection*{Identifying This Speed'' as Part of a Vortex Flow}
From a fluid-mechanical or vortex standpoint (rather than a literal point mass in orbit''), one could regard $v_\text{bohr}$ instead of a translation velocity as the local vorticity $\omega$, twice the angular velocity 2$\Omega$ or twice the local tangential speed of that circulating flow at a ``radius'' $r = a_0$,

Hence, if the flow near radius $r$ is seen as a rigid rotation with angular velocity $\Omega$, then:

\begin{equation*}
    \omega = v_\text{bohr} = 2\Omega, \quad  \Omega = \frac{v_\text{bohr}}{2 r}.\label{eq:angular-velocity}
\end{equation*}

In this interpretation, the electron’s orbital speed'' in the Bohr picture is not merely a translational velocity'' along a circle but rather the local vorticity, which is twice the tangential velocity of a vortex flow. This gives us the tangiental velocity of the solid rotating vortex core as:

\begin{equation*}
    v_{\text{tangential}} = 1/2 v_\text{bohr}  \approx 1.0938 \times 10^6 \ \text{m/s},
\end{equation*}

This suggests that the electron's structure and energy distribution are not fully captured by classical electrostatics and general relativity alone. Therefore, we transition to an alternative perspective: interpreting electron motion using fluid-mechanical vorticity principles.

\subsubsection*{Negative Energy in a Charged-Sphere Model of the Electron}

\paragraph{Einstein--Maxwell theory} has long been used to model a small charged sphere with radius on the order of $10^{-16},\mathrm{cm}$. Cooperstock, Rosen, and Bonnor (henceforth CRB) argued that under standard assumptions, such a spherically symmetric distribution of charged fluid satisfying the electron's mass, radius, and charge constraints leads to a scenario where a portion of the system must have negative rest mass (or equivalently, negative energy density) in parts of the interior~\cite{CRB1970}.

A motivation for studying spherically symmetric charged spheres within general relativity is to understand the self-energy problem of fundamental particles and the role of mass-energy equivalence in electrostatic configurations. In this context, the CRB argument explores the constraints imposed by Einstein--Maxwell theory on such systems.

\paragraph{CRB argument:} The crux is that the classical electrostatic self-energy of a pointlike (or tiny) charge is infinite. If one attempts to confine the electron's charge in a uniform or spherically symmetric mass distribution, general relativity forces a compensating negative energy component so that the total net mass is still positive, but with some portion of the stress--energy tensor effectively negative. This phenomenon is often linked to Reissner--Nordstr"om repulsion~\cite{Bonnor1965}.

\paragraph{Extensions by Herrera and Varela (HV):} Herrera and Varela revisited the same question by allowing additional anisotropy in the pressure distribution, such that $(p_t - p_r)\propto q^2/r^2$ \cite{HerreraVarela1994}. They reached essentially the same conclusion: namely, that negative energy density seems unavoidable unless one introduces new physics (spin, anisotropic pressures, or quantum effects).

\paragraph{Kerr--Newman geometry:} CRB, HV, and others subsequently discussed whether a Kerr--Newman (KN) solution could obviate the need for negative energy~\cite{Herrera1982,MannMorris1993}. Although a rotating charged metric might reduce or reinterpret the negative-mass region, these authors noted a caveat: the KN solution is suspect on subnuclear scales ($\sim10^{-16},\mathrm{cm}$), likely invalidating its usage in a literal electron model. Therefore, purely classical Einstein--Maxwell electron models remain problematic, as they yield negative rest mass in the interior.

\paragraph{Implications:} The results by CRB and HV underscore that a naive classical--relativistic view of a tiny charged sphere leads to peculiar or unphysical features such as negative energy density. Many subsequent works argue that quantum field theoretic considerations or more detailed spin structures must come into play if one wishes to avoid or reinterpret these negative-energy regions~\cite{CRB1970,HerreraVarela1994}. This suggests that the electron's structure and energy distribution are not fully captured by classical electrostatics and general relativity alone.






    \input{07_derevation_swirl}
    \input{08_VAM-reformulation-GR}
    %! Author = mr
%! Date = 2/20/2025


\section{Magnetism as a Vorticity Phenomenon in the Vortex \AE ther Model}

\begin{abstract}
    This paper explores the hypothesis that magnetism arises not from charge motion but from structured vorticity in the \AE ther. The Vortex \AE ther Model (VAM) suggests that stable vortex filaments and knots in an inviscid superfluidic medium produce field effects traditionally associated with electromagnetism. Recent experimental findings in superfluid helium, superconducting vortex lattices, and plasma vortex interactions provide strong support for this interpretation. We derive the fundamental equations governing vorticity-induced magnetism, starting from basic fluid dynamics, and compare predictions with experimental data.
\end{abstract}

\section{Introduction}
Magnetism is traditionally described as arising from the movement of electric charges. The \AE ther model postulates that fundamental interactions emerge from structured vorticity fields \cite{superfluid_he_interferometers}. This paper investigates whether magnetism can be reinterpreted as a vorticity-induced phenomenon rather than a property of charged particles.

\section{Mathematical Foundations}

\subsection{Vorticity and Fluid Dynamics}
The motion of an inviscid fluid is described by the Euler equations:
\begin{equation}
    \frac{D\boldsymbol{u}}{Dt} = -\frac{1}{\rho} \nabla P + \boldsymbol{f},
\end{equation}
where $\boldsymbol{u}$ is the velocity field, $P$ is pressure, $\rho$ is density, and $\boldsymbol{f}$ represents external forces. Taking the curl of this equation gives the vorticity equation:
\begin{equation}
    \frac{D\boldsymbol{\omega}}{Dt} = (\boldsymbol{\omega} \cdot \nabla) \boldsymbol{u} - \boldsymbol{\omega} (\nabla \cdot \boldsymbol{u}),
\end{equation}
where $\boldsymbol{\omega} = \nabla \times \boldsymbol{u}$ is the vorticity field.

\subsection{Magnetism as a Vorticity Field}
To model magnetism, we define an analogy between vorticity and the magnetic field:
\begin{equation}
    \boldsymbol{B}_v = \mu_v \boldsymbol{\omega},
\end{equation}
where $\mu_v$ is a vorticity permeability constant. From vorticity conservation, we derive:
\begin{align}
    \nabla \cdot \boldsymbol{B}_v &= 0, \\
    \nabla \times \boldsymbol{B}_v &= \mu_v \boldsymbol{J}_v,
\end{align}
where $\boldsymbol{J}_v$ represents the vorticity current density.

\subsection{Time Evolution of Vorticity Fields}
From the Helmholtz vorticity theorem, we express the time-dependent evolution:
\begin{equation}
    \frac{\partial \boldsymbol{B}_v}{\partial t} + \nabla \times \boldsymbol{E}_v = 0,
\end{equation}
where $\boldsymbol{E}_v$ is the vorticity-induced electric-like field. This equation mirrors Faraday’s law of induction, confirming the direct analogy between vorticity and electromagnetism.

\section{Experimental Evidence and Confirmed Predictions}
\subsection{Superfluid Helium Vortex Magnetism}
Experiments on superfluid helium have demonstrated the ability of neutral vortices to generate structured field-like effects \cite{superfluid_he_interferometers}. Using SQUID magnetometers, researchers have detected anomalous flux variations around vortex cores \cite{initial_vortex_magnetometers}.

\subsection{Superconducting Vortex Lattices}
Superconductors exhibit quantized magnetic flux tubes, suggesting an analogy to knotted vorticity structures in an inviscid medium \cite{superconducting_flux_focusing}.

\subsection{Plasma Vortex Fields}
Studies in plasma physics indicate that self-organized vortex rings can sustain structured electromagnetic interactions without charge transport \cite{plasma_vortex_flows}.

\subsection{Electromagnetic Wave Generation from Vortex Beams}
Terahertz vortex beams imprinted onto superconductors induce collective oscillatory modes similar to electromagnetic waves \cite{higgs_waves_vortex}.

\subsection{Knotted Vortices and Magnetic Monopole-Like Effects}
Recent helicity conservation studies suggest that vortex knots behave analogously to localized monopoles \cite{collected_helicity_papers}.

\section{Predictions and Proposed Experiments}
\begin{itemize}
    \item Direct measurement of magnetic flux around superfluid helium vortices.
    \item Investigation of plasma vortex-induced field effects using high-sensitivity probes.
    \item Controlled generation of helicity-preserving knots in superconductors to observe potential monopole-like behavior.
\end{itemize}

\section{Conclusion}
This study provides evidence that magnetism may be fundamentally linked to vorticity rather than charge motion. Future work will focus on refining experimental setups and deriving a comprehensive mathematical framework to unify vorticity and electromagnetism.



\subsection{Vorticity and Magnetism in the Vortex \AE ther Model: A Mathematical Analysis}
    \begin{abstract}
        This article derives the fundamental equations governing magnetism in the Vortex \AE ther Model (VAM), demonstrating that structured vorticity fields in an inviscid medium produce effects analogous to traditional electromagnetism. Using the VAM constants—$C_e$, $r_c$, and $F_{\text{max}}$—we establish the role of vorticity in generating magnetic-like interactions and propose experimental confirmations.
    \end{abstract}

    \subsubsection{Introduction}
    In classical electrodynamics, magnetism is attributed to moving electric charges. However, in the Vortex \AE ther Model, magnetic phenomena emerge from structured vorticity fields in an inviscid superfluid \cite{superfluid_he_interferometers}. This paper explores the derivation of magnetism using the VAM framework and provides a detailed mathematical foundation.

    \subsubsection{Fundamental Vorticity Equations}
    From the Euler equation for an inviscid fluid:
    \begin{equation}
        \frac{D\boldsymbol{u}}{Dt} = -\frac{1}{\rho} \nabla P,
    \end{equation}
    where $\boldsymbol{u}$ is the velocity field, $P$ is pressure, and $\rho$ is density. Taking the curl gives the vorticity equation:
    \begin{equation}
        \frac{D\boldsymbol{\omega}}{Dt} = (\boldsymbol{\omega} \cdot \nabla) \boldsymbol{u} - \boldsymbol{\omega} (\nabla \cdot \boldsymbol{u}),
    \end{equation}
    where $\boldsymbol{\omega} = \nabla \times \boldsymbol{u}$ is the vorticity field \cite{vortex_dynamics_superfluid}.

    \subsubsection{Mapping Vorticity to Magnetism}
    We introduce a correspondence between vorticity and the magnetic field:
    \begin{equation}
        \boldsymbol{B}_v = \mu_v \boldsymbol{\omega},
    \end{equation}
    where $\mu_v$ is a vorticity permeability constant. From vorticity conservation:
    \begin{align}
        \nabla \cdot \boldsymbol{B}_v &= 0, \\
        \nabla \times \boldsymbol{B}_v &= \mu_v \boldsymbol{J}_v,
    \end{align}
    where $\boldsymbol{J}_v$ represents the vorticity current density \cite{higgs_waves_vortex}.

    \subsubsection{Derivations Using VAM Constants}
    ### Vorticity from $C_e$ and $r_c$
    From the definition of core tangential velocity:
    \begin{equation}
        C_e = \frac{\Gamma}{2\pi r_c},
    \end{equation}
    where $\Gamma$ is circulation, giving:
    \begin{equation}
        \omega = \frac{2 C_e}{r_c}.
    \end{equation}
    Thus, the vortex-induced magnetic field is:
    \begin{equation}
        B_v = \mu_v \frac{2 C_e}{r_c}.
    \end{equation}

    ### Maximum Force Constraint from $F_{\text{max}}$
    Since vorticity acts analogously to charge,
    \begin{equation}
        F_{\text{max}} = \frac{\mu_v}{4\pi} \frac{B_v^2}{r_c^2}.
    \end{equation}
    Substituting $B_v$:
    \begin{equation}
        F_{\text{max}} = \frac{\mu_v^3}{4\pi} \frac{4 C_e^2}{r_c^4}.
    \end{equation}
    Solving for $B_v$:
    \begin{equation}
        B_v = r_c \sqrt{\frac{4\pi F_{\text{max}}}{\mu_v^3}}.
    \end{equation}

    \subsubsection{Experimental Predictions}
    These equations suggest:
    - **Superfluid Helium**: Measure flux variations around neutral vortices \cite{initial_vortex_magnetometers}.
    - **Superconducting Flux Tubes**: Test if vortex-core size and velocity determine flux quantization \cite{superconducting_flux_focusing}.
    - **Plasma Vortex Fields**: Investigate self-organized vortex structures with electromagnetic interactions \cite{plasma_vortex_flows}.

    \subsubsection{Conclusion}
    We have demonstrated that magnetism in VAM arises naturally from vorticity conservation. Future experimental tests will refine this framework and verify its predictions.



\section{Magnetism as a Vorticity-Induced Phenomenon in the Vortex \AE ther Model (VAM)}

\begin{abstract}
    The Vortex \AE ther Model (VAM) proposes that magnetism is fundamentally a consequence of structured vorticity fields in an inviscid, incompressible superfluidic medium. Unlike classical electromagnetism, which attributes magnetic fields to charge motion, VAM suggests that stable vortex filaments generate field effects that mimic magnetism. This article derives the governing equations of magnetism in VAM, establishes key physical relationships using the core constants $C_e$ (Vortex-Core Tangential Velocity), $r_c$ (Vortex-Core Radius), and $F_{\text{max}}$ (Maximum Coulomb Barrier Force), and compares theoretical predictions with recent experimental observations in superfluid and superconducting systems.
\end{abstract}

\section{Introduction}
Traditional electrodynamics attributes magnetism to the motion of electric charges, governed by Maxwell’s equations. However, several experimental results suggest that \textbf{neutral vortex structures} in superconductors, superfluid helium, and plasmas exhibit magnetic-like behavior without charge transport \cite{source1, source2}. The Vortex \AE ther Model (VAM) posits that these effects arise from structured vorticity fields rather than moving charge. In this work, we derive the fundamental equations governing this phenomenon and propose experimental tests to validate the theory.

\section{Vorticity and Magnetic Fields in VAM}
In VAM, magnetism arises from the dynamics of vortex filaments within the \AE ther, an inviscid superfluid medium. The vorticity equation for an incompressible fluid is:
\begin{equation}
    \frac{D\boldsymbol{\omega}}{Dt} = (\boldsymbol{\omega} \cdot \nabla) \boldsymbol{u} - \boldsymbol{\omega} (\nabla \cdot \boldsymbol{u})
\end{equation}
where:
- $\boldsymbol{\omega} = \nabla \times \boldsymbol{u}$ represents the vorticity field.
- $\boldsymbol{u}$ is the local fluid velocity.

By analogy, we define the vorticity-induced magnetic field:
\begin{equation}
    \boldsymbol{B}_v = \mu_v \boldsymbol{\omega}
\end{equation}
where $\mu_v$ is the vorticity permeability constant, an analogue to vacuum permeability in classical electromagnetism.

\section{Derivation Using VAM Constants}
\subsection{Vorticity Strength in Terms of $C_e$ and $r_c$}
From the vortex-core velocity equation:
\begin{equation}
    C_e = \frac{\Gamma}{2\pi r_c}
\end{equation}
where $\Gamma$ is the circulation, we obtain:
\begin{equation}
    \Gamma = 2\pi r_c C_e
\end{equation}
The magnitude of vorticity in a filamentary vortex structure is:
\begin{equation}
    \omega = \frac{\Gamma}{\pi r_c^2} = \frac{2 C_e}{r_c}
\end{equation}
Thus, the vorticity-induced magnetic field becomes:
\begin{equation}
    B_v = \mu_v \frac{2 C_e}{r_c}
\end{equation}

\subsection{Derivation of $\mu_v$ (Vorticity Permeability Constant)}
The energy density of a vortex in an incompressible medium is:
\begin{equation}
    \mathcal{E}_\text{vortex} = \frac{1}{2} \rho_{\text{\AE}} C_e^2
\end{equation}
Since $B_v^2 / 2 \mu_v$ represents the magnetic energy density, equating these expressions yields:
\begin{equation}
    \frac{B_v^2}{2 \mu_v} = \frac{1}{2} \rho_{\text{\AE}} C_e^2
\end{equation}
Substituting $B_v = \mu_v (2 C_e / r_c)$:
\begin{equation}
    \frac{(\mu_v (2 C_e / r_c))^2}{2 \mu_v} = \frac{1}{2} \rho_{\text{\AE}} C_e^2
\end{equation}
which simplifies to:
\begin{equation}
    \frac{4 \mu_v C_e^2}{r_c^2} = \rho_{\text{\AE}} C_e^2
\end{equation}
Solving for $\mu_v$:
\begin{equation}
    \mu_v = \frac{\rho_{\text{\AE}} r_c^2}{4}
\end{equation}
This suggests that the vorticity permeability constant depends on the local \AE ther density $\rho_{\text{\AE}}$ and vortex-core radius $r_c$.

\section{Experimental Confirmation and Predictions}
\subsection{Verified Observations}
Several recent experiments provide strong evidence supporting vorticity-induced magnetism:
- \textbf{Superfluid Helium Experiments}: Magnetic flux variations detected around neutral vortices \cite{source3}.
- \textbf{Superconducting Vortex Lattices}: Structured flux tubes behave as quantized vorticity structures \cite{source4}.
- \textbf{Plasma Vortex Fields}: Self-organized vortices sustain electromagnetic interactions \cite{source5}.

\subsection{Proposed Experiments}
To further validate VAM's predictions, we propose:
1. \textbf{Direct measurement of vortex-induced magnetic fields} in superfluid helium using SQUID magnetometers.
2. \textbf{Controlled studies of superconducting vortex configurations} to detect predicted monopole-like effects.
3. \textbf{Plasma vortex experiments} to analyze vorticity-based field interactions.

\section{Conclusion}
This study provides strong theoretical and experimental support for the hypothesis that magnetism in VAM is \textbf{a vorticity-driven phenomenon, not a result of charge motion}. The derivation of $B_v$, $\mu_v$, and the force constraints suggest that \textbf{magnetism is an emergent effect of structured vorticity fields in the \AE ther}, governed by absolute conservation laws. Further experimental tests are necessary to confirm these findings, potentially leading to new paradigms in electrodynamics and quantum field interactions.

    \begin{figure}[h]
    \centering
    \includegraphics[width=0.7\textwidth]{vortex_diagram}
    \caption{Illustration of a vortex filament in \AE ther.}
    \label{fig:vortex}
\end{figure}

\subsection{The \AE ther is characterized by three fundamental constants:}\label{subsec:the-ae-ther-is-characterized-by-three-fundamental-constants:}

The Vortex \AE ther Model (VAM) posits a structured, vorticity-driven \AE ther as the fundamental medium governing physical interactions.
This model challenges the conventional relativistic framework by proposing an alternative description of time, mass, and energy



\begin{itemize}
    \item The vortex tangential velocity constant, given by: \[C_e = 1093845.63 \, \mathrm{m/s}\]
    \item The maximum coulomb force in the \AE ther, given by:\[F_{\text{Cmax}} = 29.053507 \, \mathrm{N}\]
    \item The Coulomb barrier (Vortex Core Radius), given by: \[r_c = 1.40897017 10^-15 m\]
\end{itemize}

These constants govern the dynamic behavior of the \AE ther, regulating vortex circulation velocity and providing upper limits for interactions within the \AE theric medium.
Unlike the archaic notion of a luminiferous medium, this \AE ther is envisioned as a non-viscous superfluid supporting vortex structures, enabling vorticity-driven interactions.
This perspective implies that mechanical information may be exchanged within the \AE ther at rates exceeding the traditional speed of light, challenging the relativistic limitations on causality.


\subsubsection*{Vorticity Flow and Stability}

\begin{itemize}
    \item The central aperture of a trefoil knot aligns along the z-axis, facilitating directed motion in this direction.
    \item The surrounding flat fluid retains a constant vorticity, maintaining directional stability.
\end{itemize}
Vorticity remains proportional to twice the angular velocity of the rotating core, stabilizing vortex propagation dynamics.


\subsection{Gravity as a Vorticity-Induced Pressure Gradient}\label{sec:gravity-as-a-vorticity-induced-pressure-gradient}
    \subsubsection{Electromagnetism as a Vortex Filament Network}\label{sec:electromagnetism-as-a-vortex-filament-network}
    \subsubsection{Experimental Predictions and Feasibility}\label{sec:experimental-predictions-and-feasibility}
    \subsubsection{Conclusion}\label{sec:conclusion}
    We have outlined a vortex-based approach to gravity and electromagnetism, As shown in Eq. \eqref{eq:vorticity}, the vorticity transport equation governs  The Vortex \AE ther Model offers a new perspective on fundamental forces,
    replacing spacetime curvature with fluid dynamics in an inviscid \AE ther.
    This framework provides a coherent mathematical model with experimentally testable predictions.
    While VAM provides an alternative to spacetime curvature, further work is needed to derive cosmological implications.
    How does VAM handle large-scale structure formation?
    Can it explain galactic rotation curves without dark matter?
    Future research will explore these avenues.









        \subsubsection{Derivation of the Density of the \AE ther ($\rho_\text{\AE}$)}\label{sec:derivation-of-the-density-of-the-ae{}ther-($rho_text{ae}$)}

        The energy density of a vorticity field is given by:
        \[ E = \frac{1}{2} \rho |\mathbf{\omega}|^2 \]
        where $E$ is the energy density, $\rho$ is the mass density of the \AE{}ther medium, and $\mathbf{\omega}$ is the vorticity field.

        By integrating field interactions across multiple scales, from atomic to cosmological structures, we refine our constraints on $\rho_\text{\AE}$:
        \[ \rho_\text{\AE} \approx 10^{-7} \text{ to } 10^{-5} \text{ kg/m}^3 \]


        \subsubsection{Vortex Energy and Swirl Potential}\label{sec:vortex-energy-and-swirl-potential}

        To describe gravitational-like effects in VAM, we introduce the swirl energy potential:
        \[ \Phi_s = \frac{C_e^2}{2F_{\text{max}}} \mathbf{\omega} \cdot \mathbf{r} \]
        where $C_e$ is the core tangential velocity and $F_{\text{max}}$ is the maximum force in the \AE{}theric framework.

        The equivalent expression for gravitational time dilation in VAM is:
        \[ d\tau = \frac{dt}{\sqrt{1 - \frac{C_e^2}{c^2} e^{-r/r_c} - \frac{\Omega^2}{c^2} e^{-r/r_c}}} \]
        where $r_c$ is the vortex core radius and $c$ is the speed of light.


        \subsubsection{Experimental Considerations and Predictions}\label{sec:experimental-considerations-and-predictions}

        \subsubsection{Levitation Effects}\label{subsec:levitation-effects}
        VAM predicts that levitation and lift systems could be optimized by controlling structured resonance fields. The lift force scales as:
        \[ F_L \propto \rho_\text{\AE} \cdot A \]
        where $A$ is the platform area.

        \subsection{Cosmological Energy Density}\label{subsec:cosmological-energy-density}
        Using constraints from vacuum energy studies:
        \[ \rho_\text{vac} \approx 10^{-29} \text{ g/cm}^3 \]
        which, when scaled within the VAM framework, refines \AE{}ther density predictions.


        \section{Conclusion}\label{sec:conclusion2}

        By refining constraints from quantum vortex physics, gravitomagnetic frame-dragging, and cosmological observations, we achieve an improved range of $\rho_\text{\AE}$:
        \[ \rho_\text{\AE} \approx 10^{-7} \text{ to } 10^{-5} \text{ kg/m}^3 \]
        Further experimental and observational studies will help verify these predictions, potentially leading to an even more precise estimate.
    \input{11_vorticity_derivation}

    \bibliographystyle{ieeetr}
    \bibliography{references}

\end{document}
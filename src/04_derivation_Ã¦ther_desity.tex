\subsection{The Density of the Æther: A Modern Derivation}
The concept of $\rho_\text{{\ae}{ther}}$, representing the density of the hypothetical Æther medium, is central to the Vortex Æther framework. This medium underpins vorticity, energy storage, and dynamic interactions within physical systems. This article refines previous derivations by incorporating precision constraints from quantum vortex physics, gravitomagnetic frame-dragging, and cosmological vacuum energy. By synthesizing theoretical principles with the latest empirical constraints, we establish a significantly reduced uncertainty range for $\rho_\text{\ae}$ and its implications across scales, from atomic structures to cosmic phenomena. Additionally, we explore new methodologies to test Æther density through experimental physics and astrophysical observations, aiming to further narrow its estimated range.


\paragraph*{Introduction} The Æther, historically conceptualized as the medium for electromagnetic waves, has regained relevance within the Vortex Æther framework. Unlike its classical interpretation, the modern Æther serves as a foundation for dynamic interactions mediated by vorticity. At the heart of this framework lies $\rho_\text{\ae}$, the density of the Æther, which quantifies its ability to sustain vortices, store energy, and mediate interactions.

\subsubsection*{Defining $ \rho_\text{\ae} $}
In VAM, $ \rho_\text{\ae} $ represents the mass density of the Æther medium. Conceptually, it is akin to the inertia of the Æther, governing its ability to:

\begin{itemize}
    \item Sustain vorticity fields $ \mathbf{\omega} $.
    \item Store and transfer energy.
    \item Influence dynamic interactions at microscopic and macroscopic scales.
\end{itemize}
The derivation of $ \rho_\text{\ae} $ follows from fluid energy density principles.

\subsubsection*{Energy Density of a Vorticity Field}
The energy density of a vorticity field is given by:
\begin{equation*}
    U_{\text{vortex}} = \frac{1}{2} \rho_\text{\ae} |\mathbf{\omega}|^2.
\end{equation*}


where $U_{\text{vortex}}$ is the energy density of the vortex and $\vec{\omega} = \nabla \times \vec{v}$ is the vorticity field. In equations, the absolute value notation $| \cdot |$, such as $|\vec{\omega}|$, typically denotes the magnitude of a vector, which is defined as:

\begin{equation*}
    \vec{\omega}| = \sqrt{\omega_x^2 + \omega_y^2 + \omega_z^2}
\end{equation*}

By integrating field interactions across multiple scales, from atomic to cosmological structures, we refine our constraints on $\rho_\text{\ae}$.


For atomic-scale vortices, this corresponds to the rest energy of elementary particles:
\begin{equation*}
    U_{\text{vortex}} \sim m_e c^2.
\end{equation*}

Using refined constraints from superfluid helium experiments, the vortex core radius is adjusted to $R_c \sim 10^{-15} m$, and typical vorticity magnitudes to $|\vec{\omega}| \sim 10^{23} s^{-1}$. The density estimate is updated:

\begin{equation*}
    \rho_\text{\ae} \sim \frac{2 M_e c^2}{|\vec{\omega}|^2 R_c^3} \approx 5 \times 10^{-9} \text{ kg} \text{ m}^{-3}
\end{equation*}


Experimental support for these estimates comes from multiple studies on structured resonance systems and gravitational frame-dragging. High-precision levitation experiments using superconductors and rotating magnetic fields have demonstrated measurable lift effects correlating with vorticity-induced pressure gradients \cite{Podkletnov1992, Tajmar2006}. Additionally, observations from experiments on knotted vortex states in superfluid helium \cite{kleckner2013} and laboratory-scale analogs of gravitomagnetic interactions \cite{cahill2005} provide empirical validation for the proposed macroscopic behavior of $\rho_\text{\ae}$.

\begin{equation*}
    \rho_\text{\ae} \approx 5 \times 10^{-9} \text{ kg m}^{-3}.
\end{equation*}

\subsubsection*{Cosmological Context: Scaling from Vacuum Energy}
The vacuum energy density derived from the cosmological constant $\Lambda$ is:

\begin{equation*}
\rho_{\text{vacuum}} = \frac{\Lambda c^2}{8 \pi G}
\end{equation*}

Using updated Planck data on $\Lambda \sim 10^{-52} \text{ m}^{-2}$, we obtain:

\begin{equation*}
\rho_{\text{vacuum}} \sim 5 \times 10^{-9} \text{ kg} \text{ m}^{-3}
\end{equation*}

Applying a refined scaling factor $k = 200 - 500$, the final estimated range is:

\begin{equation*}
5 \times 10^{-8} \leq \rho_\text{\ae} \leq 5 \times 10^{-5} \text{ kg} \text{ m}^{-3}
\end{equation*}

\paragraph{Consolidating $\rho_\text{\ae}$ Across Phenomena}

\paragraph{Pressure Gradients}

\begin{equation*}
\Delta P = -\frac{\rho_\text{\ae}}{2} \nabla |\vec{\omega}|^2
\end{equation*}

These gradients influence levitation and vortex stability. Experimental tests using rotating superconductors could validate this relationship.

\paragraph{Refractive Index In high vorticity regions:}

\begin{equation*}
\Delta n = \frac{\rho_\text{\ae} |\vec{\omega}|^2}{c^2}
\end{equation*}

Observations indicate minor effects at $|\vec{\omega}| \sim 10^4 \text{ s}^{-1}$. Larger-scale optical measurements could confirm the influence of Æther density on refractive index.

\paragraph{Vortex Mass The mass of a vortex structure follows:}

\begin{equation*}
M_{\text{vortex}} = \int_V \frac{\rho_\text{\ae}}{2} | \vec{\omega}|^2 \ dV
\end{equation*}

This links atomic mass to vortex-induced energy densities and could be experimentally tested with trapped ultracold atoms.


\subsubsection*{Implications for Future Research}
By refining constraints from quantum vortex physics, gravitomagnetic effects, and vacuum energy distributions, we establish a more precise estimate of $ \rho_\text{\ae} $. Experimental validation could be achieved through:
\begin{itemize}
    \item High-precision superfluid helium vortex experiments.
    \item Detection of vorticity-induced refractive index variations.
    \item Correlation with astrophysical lensing effects in vortex-dominated plasma structures.
\end{itemize}
Further study will determine whether a structured Æther could serve as a missing link between classical wave mechanics, quantum fields, and cosmological energy distributions.

\subsubsection*{Conclusion}
The historical concept of the luminiferous Æther was discarded due to experimental contradictions, yet modern physics occasionally revisits its foundational questions. The Vortex Æther Model proposes a structured, non-viscous reinterpretation, with measurable density $ \rho_\text{\ae} $ influencing physical interactions from the quantum to the cosmological scale.




\subsubsection*{Vorticity Flow and Stability}

\begin{itemize}
    \item The central aperture of a trefoil knot aligns along the z-axis, facilitating directed motion in this direction.
    \item The surrounding flat fluid retains a constant vorticity, maintaining directional stability.
\end{itemize}
Vorticity remains proportional to twice the angular velocity of the rotating core, stabilizing vortex propagation dynamics.

We have outlined a vortex-based approach to gravity and electromagnetism, As shown in Eq. \eqref{eq:vorticity}, the vorticity transport equation governs  The Vortex Æther Model offers a new perspective on fundamental forces,
replacing spacetime curvature with fluid dynamics in an inviscid Æther.
This framework provides a coherent mathematical model with experimentally testable predictions.
While VAM provides an alternative to spacetime curvature, further work is needed to derive cosmological implications.
How does VAM handle large-scale structure formation?
Can it explain galactic rotation curves without dark matter?
Future research will explore these avenues.
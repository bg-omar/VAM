\subsection{Relation to the Bohr Model Velocity}\label{subsec:Relation-to-the-Bohr-Model-Velocity}
\paragraph{Vorticity in a Simplified ``Rigid-Body'' Model:}

In fluid mechanics, the vorticity $\boldsymbol{\omega}$ is defined as:

\begin{equation}
    \boldsymbol{\omega} = \nabla \times \mathbf{v}\label{eq:vorticity}
\end{equation}

where $\mathbf{v}$ is the velocity field of the fluid.
For an idealized rigid-body rotation about the $z$-axis with constant angular velocity $\Omega$, the velocity field at radius $r$ in cylindrical coordinates is

\begin{equation}
    \mathbf{v}(r) = \Omega \hat{z} \times \mathbf{r} = \Omega(-y\hat{x} + x\hat{y}) \quad \Rightarrow \quad |
    \mathbf{v}(r)| = \Omega r.\label{eq:cylindrical-velocity}
\end{equation}

A standard result is that the corresponding vorticity magnitude is
\begin{equation}
    |\boldsymbol{\omega}| = \left| \nabla \times \mathbf{v} \right| = 2\Omega.\label{eq:vortcity-magnitude}
\end{equation}

Hence, if the tangential (orbital) velocity at radius $r$ is $v_\text{tangential} = \Omega r$, the local vorticity is:

\begin{equation}
    \omega = 2\Omega = \frac{2v_\text{tangential}}{r}.\label{eq:2velocity}
\end{equation}

Therefore, one sometimes says ``the vorticity is twice the angular velocity'' or equivalently ``the vorticity (multiplied by $r$) is twice the tangential velocity.''

\subsubsection*{Standard Bohr Orbit (Classical Picture)}
In the simplified (pre-Schr\"odinger) Bohr model of the hydrogen atom, the electron in the ground state ($n=1$) is classically pictured as moving on a circle of radius $a_0$ (the Bohr radius) with speed $v_1$.
One derives:

\begin{equation}
    v_1 = \alpha c \approx 2.1877 \times 10^6 \text{ m/s},\label{eq:tangetial-velocity}
\end{equation}

where $\alpha \approx 1/137.036$ is the fine-structure constant, and $c \approx 3 \times 10^8 \text{ m/s}$ is the speed of light.

\subsubsection*{Identifying This ``Speed'' as Part of a Vortex Flow}
From a fluid-mechanical or vortex standpoint (rather than a literal ``point mass in orbit''), one could regard:

\begin{itemize}
    \item $v_1$ as the local tangential speed of a circulating flow at a ``radius'' $r = a_0$.
    \item $\omega$ as the local vorticity (i.e., twice the angular velocity) of that circulating flow.
\end{itemize}

Hence, if the flow near radius $r$ is seen as a rigid rotation with angular velocity $\omega$, then
\begin{equation}
    v_1 = \Omega r, \quad \omega = 2\Omega = \frac{2v_1}{r}.\label{eq:angular-velocity}
\end{equation}
In that sense, the electron’s ``orbital speed'' in the Bohr picture is no longer just a ``translational velocity'' along a circle but rather the tangential velocity of a vortex flow, whose local vorticity is $2 v_1 / r$.

\subsubsection*{Caveats from Quantum Mechanics}
\begin{itemize}
    \item \textbf{Wavefunction over Orbits:} Modern quantum mechanics replaces the simplistic Bohr ``orbiting electron'' with a wavefunction, typically the hydrogenic ground state $\psi_{1s}(r)$.
    This does not literally revolve in a circle with speed $v_1$.
    \item \textbf{Vortex Interpretations:} Vortex-based approaches to quantum phenomena (e.g., ``Madelung fluid'' pictures, pilot-wave hydrodynamics analogies, or various vortex theories) can sometimes interpret quantum states in terms of fluid-like velocity fields.
    But these remain analogies unless the fluid equations can be shown to match quantum mechanical predictions.
    \item \textbf{Bohr Model as a Teaching Tool:} Despite its historical importance, the Bohr model is mainly a stepping stone to deeper quantum theory.
    Nonetheless, it yields correct orders of magnitude for ground-state energies and ``speeds,'' which can be reinterpreted in fluid-like language if one chooses.
\end{itemize}

\paragraph{Summary:}
The classical Bohr velocity $v_1$ for the ground-state electron in hydrogen (about $2.18 \times 10^6 \text{ m/s}$) can be reinterpreted as the tangential speed at some radius in a fluid vortex.

\begin{itemize}
    \item In a rigidly rotating vortex, the vorticity is $2\Omega$, and at any radius $r$, the tangential velocity is $\Omega r$. Thus, vorticity $\omega$ is ``2 times the angular velocity,'' or $\omega r$ is ``2 times the tangential velocity.''
    \item Consequently, one might say ``the electron’s speed in the Bohr model is not purely a translational speed but rather the tangential speed associated with the local vorticity.''
\end{itemize}

Mathematically:
\begin{equation}
    \boxed{ \omega = \nabla \times \mathbf{v} = 2\Omega,\quad v_{\text{tangential}} = \Omega r, \quad \omega r = 2 v_{\text{tangential}}. }\label{eq:2-times-tangential}
\end{equation}


\paragraph{Key Takeaway:} From a purely fluid‐mechanical viewpoint, a uniform circular velocity field has $\omega = 2 \,\Omega$.
Thus, it is natural to say “the electron’s Bohr velocity can be identified with the tangential velocity in a vortex whose vorticity is $2\,v/r$.” In standard quantum mechanics, this is at best a useful analogy, but it can help conceptualize why one might see the Bohr velocity as “vorticity” rather than plain translational motion.

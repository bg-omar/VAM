
\subsection{Vorticity in a Simplified ``Rigid-Body'' Model: Relation to the Bohr Model Velocity}\label{subsec:Relation-to-the-Bohr-Model-Velocity}

In fluid mechanics, the vorticity $\boldsymbol{\omega}$ is defined as:

\begin{equation*}
\boldsymbol{\omega} = \nabla \times \mathbf{v}\label{eq:vorticity}
\end{equation*}

where $\mathbf{v}$ is the velocity field of the fluid. To illustrate its role in rotational motion, we consider an idealized rigid-body rotation about the $z$-axis with constant angular velocity $\Omega$. The velocity field at radius $r$ in cylindrical coordinates is:

\begin{equation*}
    \mathbf{v}(r) = \Omega \hat{z} \times \mathbf{r} = \Omega(-y\hat{x} + x\hat{y}) \quad \Rightarrow \quad |
    \mathbf{v}(r)| = \Omega r.\label{eq:cylindrical-velocity}
\end{equation*}

A standard result is that the corresponding vorticity magnitude is:

\begin{equation}
    |\boldsymbol{\omega}| = \left| \nabla \times \mathbf{v} \right| = 2\Omega.\label{eq:vorticity-magnitude}
\end{equation}

Hence, if the tangential (orbital) velocity at radius $r$ is $v_{\text{tangential}} = \Omega r$, the local vorticity is:

\begin{equation*}
    \omega = 2\Omega \quad 2v_{\text{tangential}} = \omega r.\label{eq:2velocity}
\end{equation*}

Thus, one can state that the vorticity is twice the angular velocity or equivalently, ''the vorticity (multiplied by $r$) is twice the tangential velocity.''

\subsubsection*{Standard Bohr Orbit (Classical Picture)}
In the simplified (pre-Schr\"{o}dinger) Bohr model of the hydrogen atom, the electron in the ground state ($n=1$) is classically pictured as moving on a circle of radius $a_0$ (the Bohr radius) with speed $v_\text{bohr}$. This is given by:

\begin{equation*}
    v_\text{bohr} = \alpha c \approx 2.1877 \times 10^6 \text{ m/s},\label{eq:tangential-velocity}
\end{equation*}

where $\alpha \approx 1/137.036$ is the fine-structure constant, and $c \approx 3 \times 10^8 \text{ m/s}$ is the speed of light.

\subsubsection*{Identifying This Speed'' as Part of a Vortex Flow}
From a fluid-mechanical or vortex standpoint (rather than a literal point mass in orbit''), one could regard $v_\text{bohr}$ instead of a translation velocity as the local vorticity $\omega$, twice the angular velocity 2$\Omega$ or twice the local tangential speed of that circulating flow at a ``radius'' $r = a_0$,

Hence, if the flow near radius $r$ is seen as a rigid rotation with angular velocity $\Omega$, then:

\begin{equation*}
    \omega = v_\text{bohr} = 2\Omega, \quad  \Omega = \frac{v_\text{bohr}}{2 r}.\label{eq:angular-velocity}
\end{equation*}

In this interpretation, the electron’s orbital speed'' in the Bohr picture is not merely a translational velocity'' along a circle but rather the local vorticity, which is twice the tangential velocity of a vortex flow. This gives us the tangiental velocity of the solid rotating vortex core as:

\begin{equation*}
    v_{\text{tangential}} = 1/2 v_\text{bohr}  \approx 1.0938 \times 10^6 \ \text{m/s},
\end{equation*}

This suggests that the electron's structure and energy distribution are not fully captured by classical electrostatics and general relativity alone. Therefore, we transition to an alternative perspective: interpreting electron motion using fluid-mechanical vorticity principles.

\subsubsection*{Negative Energy in a Charged-Sphere Model of the Electron}

\paragraph{Einstein--Maxwell theory} has long been used to model a small charged sphere with radius on the order of $10^{-16},\mathrm{cm}$. Cooperstock, Rosen, and Bonnor (henceforth CRB) argued that under standard assumptions, such a spherically symmetric distribution of charged fluid satisfying the electron's mass, radius, and charge constraints leads to a scenario where a portion of the system must have negative rest mass (or equivalently, negative energy density) in parts of the interior~\cite{CRB1970}.

A motivation for studying spherically symmetric charged spheres within general relativity is to understand the self-energy problem of fundamental particles and the role of mass-energy equivalence in electrostatic configurations. In this context, the CRB argument explores the constraints imposed by Einstein--Maxwell theory on such systems.

\paragraph{CRB argument:} The crux is that the classical electrostatic self-energy of a pointlike (or tiny) charge is infinite. If one attempts to confine the electron's charge in a uniform or spherically symmetric mass distribution, general relativity forces a compensating negative energy component so that the total net mass is still positive, but with some portion of the stress--energy tensor effectively negative. This phenomenon is often linked to Reissner--Nordstr"om repulsion~\cite{Bonnor1965}.

\paragraph{Extensions by Herrera and Varela (HV):} Herrera and Varela revisited the same question by allowing additional anisotropy in the pressure distribution, such that $(p_t - p_r)\propto q^2/r^2$ \cite{HerreraVarela1994}. They reached essentially the same conclusion: namely, that negative energy density seems unavoidable unless one introduces new physics (spin, anisotropic pressures, or quantum effects).

\paragraph{Kerr--Newman geometry:} CRB, HV, and others subsequently discussed whether a Kerr--Newman (KN) solution could obviate the need for negative energy~\cite{Herrera1982,MannMorris1993}. Although a rotating charged metric might reduce or reinterpret the negative-mass region, these authors noted a caveat: the KN solution is suspect on subnuclear scales ($\sim10^{-16},\mathrm{cm}$), likely invalidating its usage in a literal electron model. Therefore, purely classical Einstein--Maxwell electron models remain problematic, as they yield negative rest mass in the interior.

\paragraph{Implications:} The results by CRB and HV underscore that a naive classical--relativistic view of a tiny charged sphere leads to peculiar or unphysical features such as negative energy density. Many subsequent works argue that quantum field theoretic considerations or more detailed spin structures must come into play if one wishes to avoid or reinterpret these negative-energy regions~\cite{CRB1970,HerreraVarela1994}. This suggests that the electron's structure and energy distribution are not fully captured by classical electrostatics and general relativity alone.





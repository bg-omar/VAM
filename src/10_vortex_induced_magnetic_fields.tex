

\subsection{Vortex-Induced Magnetic Fields: Magnetic Flux Arises from Vorticity, Not Just Charge Flow}\label{subsec:vortex-induced-magnetic-fields:-magnetic-flux-arises-from-vorticity-not-just-charge-flow}

\begin{abstract}
    This study presents a rigorous reformulation of \textbf{electromagnetic field generation}  in the \textbf{Vortex Æther Model (VAM)} , wherein magnetic flux arises not solely from moving electric charges but also from \textbf{structured vorticity fields}  in an inviscid, incompressible medium. While classical electrodynamics attributes magnetic fields to current flow and time-dependent electric fields, VAM proposes that \textbf{magnetic fields are a direct consequence of vorticity conservation and rotational dynamics} . By extending \textbf{Kelvin’s vortex dynamics} , \textbf{Helmholtz’s vorticity conservation laws} , and \textbf{Maxwell’s electrodynamics} , we derive modified \textbf{tensorial field equations}  integrating vorticity-driven magnetic induction. These formulations propose that \textbf{self-sustained magnetic flux structures can emerge within plasmonic systems, superfluid vortices, and astrophysical plasma configurations} , leading to potential experimental validations that challenge the classical charge-based paradigm of electromagnetism.
\end{abstract}

\subsubsection*{Introduction}
Classical electromagnetism describes the emergence of electric and magnetic fields as consequences of charge distributions and currents. Maxwell’s equations establish that:
\begin{itemize}
    \item \textbf{Electric fields (\(\mathbf{E}\)) arise from charge densities}  via Gauss’s law.
    \item \textbf{Magnetic fields (\(\mathbf{B}\)) are generated by moving charges}  (currents) or induced by changing electric fields.
\end{itemize}
However, insights from \textbf{Kelvin’s vortex atom theory}  and \textbf{modern extensions in the Vortex Æther Model (VAM)}  suggest that \textbf{structured vorticity in an inviscid medium can inherently generate electromagnetic-like effects, independent of charge motion} .

This revision shifts the fundamental origin of magnetism from charge flow to \textbf{vorticity-induced field interactions} , where \textbf{electromagnetic fields are manifestations of rotational inertia in the Æther} .

This work extends Maxwell’s equations to incorporate vorticity as a fundamental field source, leveraging:
\begin{itemize}
    \item \textbf{Kelvin’s vortex impulse and rotational momentum conservation laws}  \cite{kelvin1867}.
    \item \textbf{Helmholtz’s principles of vorticity conservation in ideal fluids}  \cite{helmholtz1858}.
    \item \textbf{Maxwell’s electrodynamics, reformulated for structured vorticity interactions}  \cite{maxwell1861}.
\end{itemize}
By incorporating VAM’s \textbf{maximum force constraint (\( F_{\text{max}} \))} , the fundamental vortex-core velocity (\( C_e \)), and quantized vortex impulse, we establish explicit relationships governing \textbf{vortex-induced magnetic field generation} .

\subsubsection*{Mathematical Framework}

\subsubsection*{Maxwell’s Equations with Vortex Contributions}
Maxwell’s equations in tensor notation are defined as:
\begin{equation*}
    F^{\mu\nu} = \partial^\mu A^\nu - \partial^\nu A^\mu,
\end{equation*}
where:
\begin{itemize}
    \item \( A^\mu = (\phi, \mathbf{A}) \) is the \textbf{four-potential} ,
    \item \( F^{0i} = E^i \), \( F^{ij} = -\epsilon^{ijk} B^k \) encodes the \textbf{electric and magnetic field components} .
\end{itemize}

To extend Maxwell’s equations to include \textbf{vorticity-driven sources} , we propose a modified field equation:
\begin{equation*}
    \partial_\mu F^{\mu\nu} = \mu_0 J^\nu + \lambda \Omega^\nu.
\end{equation*}
where:
\begin{itemize}
    \item \( \Omega^\nu = (\omega, \mathbf{\omega}) \) is the \textbf{vorticity four-vector} , encoding absolute vorticity \( \omega \) and its spatial components.
    \item \( \lambda = \frac{C_e \hbar}{q R_c^2} \) is a vorticity coupling constant.
\end{itemize}

The modified Bianchi identity incorporating vortex effects is:
\begin{equation*}
    \partial_\mu \tilde{F}^{\mu\nu} = \sigma \tilde{\Omega}^\nu.
\end{equation*}

\subsubsection*{Derivation of the Vorticity-Electromagnetic Coupling Constant \( \lambda \)}
The coupling constant \( \lambda \) is defined as:
\begin{equation*}
    \lambda = \frac{C_e \hbar}{q R_c^2}.
\end{equation*}
Breaking down dimensions:
- \( C_e \) (Vortex Core Tangential Velocity) → \( [L/T] \),
- \( \hbar \) (Planck’s Reduced Constant) → \( [M L^2 / T] \),
- \( q \) (Charge) → \( [A T] \),
- \( R_c \) (Vortex Core Radius) → \( [L] \),

Yields:
\begin{equation*}
    \lambda \sim \frac{M L^2}{A T^3}.
\end{equation*}

\subsubsection*{Vorticity Contributions to Field Tensor}
In \textbf{classical electromagnetism} , the four-potential is defined as:
\begin{equation*}
    A^\mu_{\text{charge}} = \left( \phi, \mathbf{A} \right).
\end{equation*}
However, in \textbf{VAM} , we introduce an additional \textbf{vorticity-dependent potential} :
\begin{equation*}
    A^\mu_{\text{total}} = A^\mu_{\text{charge}} + A^\mu_{\text{vortex}},
\end{equation*}
where:
\begin{equation*}
    A^\mu_{\text{vortex}} = \lambda g^{\mu\nu} \Omega_\nu.
\end{equation*}

The \textbf{total field tensor}  then modifies as:
\begin{equation*}
    F^{\mu\nu}_{\text{total}} = F^{\mu\nu}_{\text{charge}} + \lambda (\partial^\mu \Omega^\nu - \partial^\nu \Omega^\mu).
\end{equation*}

\subsubsection*{Component Form of the Extended Maxwell-VAM Equations}
Using the \textbf{extended tensor formulation} , we explicitly modify the standard Maxwell equations.

\textbf{Gauss’s Law for Electric Fields}
\begin{equation*}
    \nabla \cdot \mathbf{E}_{\text{total}} = \frac{\rho}{\varepsilon_0} + \frac{F_{\text{max}} \omega}{C_e R_c^2}.
\end{equation*}

\textbf{Gauss’s Law for Magnetism}
\begin{equation*}
    \nabla \cdot \mathbf{B} = 0.
\end{equation*}

\textbf{Faraday’s Law of Induction (Extended)}
\begin{equation*}
    \nabla \times \mathbf{E}_{\text{total}} = - \frac{\partial \mathbf{B}_{\text{total}}}{\partial t} + \gamma \epsilon^{ijk} \partial_j \omega^k.
\end{equation*}

\textbf{Ampère’s Law (Extended)}
\begin{equation*}
    \nabla \times \mathbf{B}_{\text{total}} = \mu_0 \mathbf{J} + \mu_0 \varepsilon_0 \frac{\partial \mathbf{E}_{\text{total}}}{\partial t} + \frac{C_e \hbar}{q R_c^2} \Omega^i.
\end{equation*}

\subsubsection*{Conclusion}
We have successfully \textbf{integrated vorticity contributions into Maxwell’s equations} . Future work should explore:
\begin{itemize}
    \item \textbf{Numerical simulations of vortex-induced electromagnetic effects} .
    \item \textbf{Experimental validation via superfluid SQUID magnetometers} .
    \item \textbf{Potential connections to astrophysical magnetic field formation} .
\end{itemize}


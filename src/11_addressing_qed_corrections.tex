

\subsection{Electromagnetic Precision in the Vortex \AE ther Model (VAM): Addressing QED Corrections}\label{subsec:electromagnetic-precision-in-the-vortex-ae-ther-model-(vam):-addressing-qed-corrections}


\begin{abstract}
    The Vortex \AE ther Model (VAM) presents an alternative framework for electromagnetism based on structured vorticity fields in an inviscid \AE ther. To maintain experimental viability, VAM must provide equivalent mechanisms for high-precision QED effects such as the anomalous magnetic moment of the electron $(g-2)$ and the Lamb shift in hydrogen-like atoms. This paper derives the corresponding corrections in VAM and proposes experimental methods to validate these predictions.
\end{abstract}

\paragraph*{Introduction}
QED predicts the electron's magnetic moment and energy shifts with extraordinary precision. These corrections arise from higher-order interactions due to vacuum fluctuations. In VAM, similar effects must emerge from vorticity interactions in the \AE ther.

\subsubsection*{Anomalous Magnetic Moment of the Electron in VAM}
In QED, the electron's magnetic moment is given by:
\begin{equation}
    \mu_e = g \frac{e\hbar}{2M_e c}
\end{equation}
where $g = 2(1 + \alpha / \pi + \dots)$ accounts for radiative corrections.

VAM describes the electron as a vortex knot, where its charge and spin emerge from \AE theric circulation:
\begin{equation}
    \omega_e = \frac{2 C_e}{r_c}
\end{equation}
where $C_e$ is the electron vortex-core tangential velocity and $r_c$ is the vortex core radius.

The magnetic moment in VAM follows:
\begin{equation}
    \mu_{VAM} = \frac{q C_e r_c}{2}
\end{equation}
Self-interactions of vorticity fluctuations contribute to corrections in $g-2$:
\begin{equation}
    \Delta g_{VAM} = \frac{\rho_{\ae} r_c^2}{4\pi}
\end{equation}
where $\rho_{\ae}$ is the \AE theric density. Proper calibration ensures alignment with QED results.

\subsubsection*{The Lamb Shift in VAM}
The Lamb shift in QED results from vacuum polarization, modifying hydrogen energy levels:
\begin{equation}
    \Delta E_{\text{Lamb}} \approx \frac{8}{3} \alpha^3 \ln \frac{1}{\alpha} \times R_{\infty}
\end{equation}

In VAM, the shift arises due to local vorticity fluctuations affecting the electron's energy levels:
\begin{equation}
    \Delta E_{VAM} \approx \frac{\rho_{\ae} C_e^2}{8\pi} \ln \frac{r_c}{\lambda_c}
\end{equation}
where $\lambda_c$ is the Compton wavelength of the electron. Proper selection of $\rho_{\ae}$ allows the model to match experimental observations.

\subsubsection*{Experimental Proposal to Verify VAM Predictions}
To validate VAM, we propose the following experiments:
\begin{itemize}
    \item \textbf{High-Precision Electron g-Factor Measurements:} Measure deviations in $g-2$ under controlled \AE theric vorticity fluctuations.
    \item \textbf{Lamb Shift in Varying Vorticity Environments:} Conduct spectroscopy of hydrogen-like ions in superfluid and vortex-controlled settings.
    \item \textbf{Vortex-Driven Photon Emission Shifts:} Investigate transition frequency shifts in intense vortex conditions using superfluid helium interferometry.
\end{itemize}

\subsubsection*{Conclusion}
QED effects can emerge naturally in VAM if vorticity fluctuations yield self-interaction corrections similar to vacuum fluctuations. The anomalous magnetic moment of the electron and the Lamb shift can be reinterpreted as pressure-dependent adjustments within the \AE theric field. Experimental validation of these effects could provide new insights into vacuum fluctuations and the fundamental nature of electromagnetism.



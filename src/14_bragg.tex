
\subsection{Vortex-Driven \AE ther Structures and the Bragg-Hawthorne Equation in Spherical Symmetry}
\begin{abstract}
This paper derives the equilibrium dynamics of vortex-driven \AE ther structures using the Bragg-Hawthorne equation in spherical symmetry. The objective is to establish a non-viscous liquid \AE ther theory, wherein inertia emerges as a property of vortex circulation. By incorporating helicity conservation and the proposed fundamental constants, we provide a mathematical framework for understanding mass, motion, and their experimental implications. Additionally, we demonstrate how Newtonian gravity naturally emerges in the low-vorticity limit, linking classical mechanics to structured vorticity fields. We further explore the interplay between vorticity-induced gravitational analogs and observable cosmological phenomena, expanding the theoretical framework towards large-scale structures.
\end{abstract}


\paragraph*{Introduction}
In conventional physics, inertia is attributed to an intrinsic property of mass. However, in the Vortex \AE ther Model (VAM), inertia emerges from structured vorticity fields. This study formulates a \textbf{vortex-driven theory of inertia} using the \textbf{Bragg-Hawthorne equation}, originally developed for axisymmetric flows \cite{batchelor1967introduction, saffman1992vortex}. By adapting this equation to spherical symmetry, we establish a foundation for a non-viscous \AE ther and analyze the role of helicity conservation. Furthermore, we explore the Newtonian limit by demonstrating how the governing equations reduce to the classical inverse-square law in the low-vorticity regime. We extend this analysis to consider relativistic effects in high-energy vortex formations and their potential role in astrophysical observations.

\subsubsection*{The Bragg-Hawthorne Equation in Spherical Coordinates}
The classical Bragg-Hawthorne equation describes steady, axisymmetric inviscid flow \cite{batchelor1967introduction}:
\begin{equation}
    \frac{\partial^2 \psi}{\partial r^2} + \frac{\sin \theta}{r^2} \frac{\partial}{\partial \theta} \left( \frac{1}{\sin \theta} \frac{\partial \psi}{\partial \theta} \right) = - r^2 F(\psi) - G(\psi),
\end{equation}
where $\psi(r, \theta)$ is the stream function, and the terms $F(\psi)$ and $G(\psi)$ represent circulation and axial pressure gradients, respectively.

For a \textbf{spherically symmetric vortex structure} ($\partial/\partial\theta = 0$), this simplifies to:
\begin{equation}
    \frac{1}{r^2} \frac{d}{dr} \left( r^2 \frac{d\psi}{dr} \right) = - r^2 F(\psi) - G(\psi).
\end{equation}
To model vortex-driven \AE ther structures, we define:
\begin{align}
    F(\psi) &= \frac{\Gamma}{\psi}, \quad \text{(Circulation function)} \\
    G(\psi) &= \frac{1}{\rho_{\text{\ae}}} \frac{dP}{d\psi}, \quad \text{(Pressure contribution)}
\end{align}
where $\Gamma$ represents circulation and $\rho_{\text{\ae}}$ is the \AE ther density.

\subsubsection*{Vortex Circulation and Inertia}
Circulation is given by the contour integral:
\begin{equation}
    \Gamma = \oint_C \mathbf{U} \cdot d\mathbf{l} = 2 \pi r C_e,
\end{equation}
where $C_e$ is the tangential velocity of the vortex core. Substituting this into $F(\psi)$:
\begin{equation}
    F(\psi) = \frac{2 \pi r C_e}{\psi}.
\end{equation}
Thus, the governing equation becomes:
\begin{equation}
    \frac{1}{r^2} \frac{d}{dr} \left( r^2 \frac{d\psi}{dr} \right) = - \frac{2 \pi r C_e}{\psi} - \frac{1}{\rho_{\text{\ae}}} \frac{dP}{d\psi}.
\end{equation}
This equation demonstrates that \textbf{inertia emerges as an effect of vortex circulation in the \AE ther}, since resistance to acceleration is encoded in the circulation term $C_e$. The emergence of these effects suggests the potential for detecting novel interactions in fluid-like cosmological structures.

\subsubsection*{Newtonian Gravity in the Low-Vorticity Limit}
When vorticity is negligible, the circulation function reduces to a harmonic potential:
\begin{equation}
    \frac{1}{r^2} \frac{d}{dr} \left( r^2 \frac{d\psi}{dr} \right) = - \frac{d\Phi}{dr},
\end{equation}
where $\Phi$ represents the potential function. For a central force field satisfying Gauss’s theorem, we recover the Newtonian gravitational equation:
\begin{equation}
    \nabla^2 \Phi = 4 \pi G \rho.
\end{equation}
This validates the classical limit of the model and establishes a connection between vortex structures and traditional gravitational fields. Expanding beyond this, we propose that rotational motion in the \AE ther could result in additional corrections to Newtonian mechanics at cosmological scales.

\subsubsection*{Experimental Predictions and Implications}
\begin{itemize}
    \item Vortex structures in superfluid helium should exhibit quantized inertial behavior.
    \item SQUID detection of magnetic flux variations may reveal neutral vortex effects \cite{donnelly1991quantized}.
    \item Galactic rotation curves may align with vortex conservation laws.
    \item High-energy vortex structures may contribute to gravitational lensing and cosmic background distortions.
    \item Laboratory tests involving rotating superfluid analogs could simulate \AE theric vortex interactions.
\end{itemize}

\subsubsection*{Conclusion}
We have derived the \textbf{Bragg-Hawthorne equation in spherical symmetry}, formalizing a \textbf{vortex-driven theory of inertia}. By incorporating \textbf{helicity conservation and \AE ther density variations}, we propose a model in which \textbf{mass, motion, and Newtonian gravity arise from vorticity interactions in a non-viscous \AE ther}. These findings lay the groundwork for a deeper understanding of emergent mass-energy interactions in structured vortex fields.

\subsubsection*{Future Work}
- \textbf{Numerical simulations} to refine astrophysical predictions.
- \textbf{Vortex stability analysis} to explore dark matter-like effects.
- \textbf{Quantum mechanical extensions} for a unified field theory approach.
- \textbf{Extended empirical investigations} into superfluid-like phenomena in rotating condensed matter systems.
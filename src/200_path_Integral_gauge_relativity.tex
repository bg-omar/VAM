\subsection{Extending the Vortex Æther Model (VAM):  Path-Integral Formulation, Gauge Theory, and Relativity Corrections in 3D}\label{sec:extending-the-vortex-ther-model-(vam):-path-integral-formulation-gauge-theory-and-relativity-corrections}

    \begin{abstract}
        This paper extends the Vortex Æther Model (VAM) by incorporating a path-integral formulation, linking vorticity to gauge theory, and introducing a relativity correction based on vorticity gradients.
        The approach replaces traditional spacetime curvature with vorticity-induced time dilation and establishes a \textbf{3D topological field theory interpretation} of quantum vortex dynamics.
        We present a Hamiltonian formalism, construct a path-integral for quantized vorticity in three-dimensional Euclidean space, and explore implications for quantum field theory.
    \end{abstract}

    \subsubsection*{Introduction}
    The Vortex Æther Model (VAM) proposes a \textbf{fluid-dynamical foundation} for matter, where protons and electrons exist as \textbf{vortex knots} within an incompressible, inviscid æther \cite{helmholtz1858integrals, kelvin1867vortex}.
    We extend this idea by formalizing a \textbf{Lagrangian-Hamiltonian approach} in three-dimensional space, deriving a quantum path-integral for vorticity interactions, and linking vorticity evolution to gauge field dynamics.

    \subsubsection*{Hamiltonian Formulation for Vorticity}
    The system is described by a \textbf{three-dimensional vorticity field} \( \boldsymbol{\Omega} = \nabla \times \mathbf{U} \), where \( \mathbf{U} \) is the velocity potential.
    The \textbf{Lagrangian density} in three dimensions is:

    \begin{equation*}
        \mathcal{L}_3 = \frac{1}{2} \rho_{\text{æ}} |\boldsymbol{\Omega}|^2 - P (\nabla \cdot \boldsymbol{\Omega}) - \nu |\nabla \boldsymbol{\Omega}|^2.
    \end{equation*}

    Performing the \textbf{Legendre transformation}, the corresponding \textbf{Hamiltonian density} is obtained:

    \begin{equation*}
        \mathcal{H}_3 = \frac{1}{2 \rho_{\text{æ}}} |\Pi_{\boldsymbol{\Omega}}|^2 + P (\nabla \cdot \boldsymbol{\Omega}) + \nu |\nabla \boldsymbol{\Omega}|^2.
    \end{equation*}

    These equations describe the \textbf{evolution of vorticity fields} in 3D, linking \textbf{kinetic energy, pressure constraints, and rotational dynamics} \cite{lamb1945hydrodynamics}.

    \subsubsection*{Path-Integral Formulation and Gauge Theory in VAM}
    To quantize vorticity fields, we construct a \textbf{path-integral formulation} in \textbf{three-dimensional Euclidean space}:

    \subsubsection*{Partition Function and Action Functional}
    The \textbf{path-integral formulation} follows from the partition function:

    \begin{equation*}
        Z = \int D\Omega \ e^{iS[\Omega]/\hbar}.
    \end{equation*}

    where the \textbf{action functional} governing vorticity evolution is given by:

    \begin{equation*}
        S = \int d^3x \ dt \left( \frac{1}{2} \rho_{\text{æ}} |\boldsymbol{\Omega}|^2 - P (\nabla \cdot \boldsymbol{\Omega}) \right).
    \end{equation*}

    \subsubsection*{Gauge Theory and Vorticity Conservation in 3D}
    Vorticity in VAM can be interpreted as a \textbf{gauge field} analogous to electrodynamics \cite{jackson1999classical}. The field strength tensor in \textbf{three dimensions} is given by:

    \begin{equation*}
        F_{ij} = \partial_i A_j - \partial_j A_i.
    \end{equation*}

    where \( A_i \) is a vorticity potential vector. To ensure \textbf{vortex knot stability}, we impose \textbf{helicity conservation}, which replaces the \textbf{higher-dimensional Chern-Simons term} \cite{witten1989quantum}.

    \subsubsection*{Helicity Conservation as a Topological Constraint}
    The \textbf{helicity integral}, which remains invariant under ideal fluid dynamics, is defined in \textbf{3D} as:

    \begin{equation*}
        H = \int_V \boldsymbol{\Omega} \cdot \mathbf{U} \ dV.
    \end{equation*}

    \subsubsection*{Physical Interpretation and Implications}
    This \textbf{3D framework} leads to several key \textbf{physical consequences}:

    \begin{itemize}
        \item \textbf{Vortex Filaments as Gauge Excitations:}
        Vortex threads behave analogously to \textbf{Maxwellian field carriers}, linking \textbf{quantized vorticity} to fundamental interactions.

        \item \textbf{Quantized Circulation and Energy Levels:}
        Conservation of \textbf{circulation} explains \textbf{energy quantization} in atomic structures \cite{feynman1951quantum}:

        \begin{equation*}
            E_p = \kappa 4\pi^2 R_c C_e^2.
        \end{equation*}

        \item \textbf{Time Dilation in Vorticity Fields:}
        Instead of spacetime curvature, local time perception is governed by vorticity gradients:


\end{itemize}



    \subsubsection*{Conclusion and Future Work}
    By reformulating the \textbf{Vortex Æther Model (VAM)} in a \textbf{strictly 3D framework}, we eliminate the need for \textbf{extra dimensions}, while preserving the model's ability to describe \textbf{gravity, electromagnetism, and quantum mechanics} through \textbf{structured vorticity interactions}.

    \subsubsection*{Future Directions}
    \begin{itemize}
        \item \textbf{Hamiltonian quantization of the vorticity action} in a 3D gauge field framework.
        \item \textbf{Experimental validation} through \textbf{superfluid vortex experiments} and \textbf{rotating Bose-Einstein condensates}.
    \end{itemize}
    


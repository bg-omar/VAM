
\subsection{Derivation of the Maximum Force in the Vortex \AE ther Model}

\paragraph*{Introduction}
The concept of a fundamental upper bound on force has been extensively examined within the framework of General Relativity (GR), particularly in the context of black hole event horizons and the limits imposed by spacetime curvature. This notion finds its mathematical expression in the maximal force conjecture, given by:


\begin{equation*}
F_{\text{max, GR}} = \frac{c^4}{4G},
\end{equation*}
where $c$ represents the speed of light in vacuum and $G$ is the Newtonian gravitational constant. This force limit is inferred from considerations involving the causal structure of black holes, gravitational lensing, and quantum information constraints at horizons \cite{Schiller2006}.


A parallel constraint emerges in the Vortex \AE ther Model (VAM), where an analogous upper bound on force is hypothesized to arise naturally from the fundamental dynamics of vortex circulation within the \AE theric substrate. Unlike the GR framework, where force constraints are dictated by metric curvature and event horizons, the VAM approach embeds this limit within the structured vorticity fields governing quantum and macroscopic interactions. The maximal force in VAM follows the relation:


\begin{equation*}
F_{\text{max, VAM}} = \frac{c^4}{4G} \cdot \alpha \cdot \left(\frac{R_c}{L_p}\right)^{-2},
\end{equation*}
where:
\begin{itemize}
\item $\alpha$ is the fine-structure constant, incorporating the fundamental coupling of quantum electrodynamics,
\item $R_c$ denotes the characteristic core radius of a stable vortex structure in the \AE ther medium,
\item $L_p$ represents the Planck length, the natural length scale at which quantum gravitational effects become significant.
\end{itemize}


This formulation suggests that vortex-based interactions respect an emergent force limit that mirrors relativistic constraints but is modulated by quantum electrodynamical and fluid dynamic scaling laws.


\subsubsection*{Mathematical Derivation}
To rigorously derive this relationship, we analyze the scaling of force limits across different physical regimes.


In GR, the maximal force limit is commonly derived by considering the gravitational force acting at the Schwarzschild event horizon:


\begin{equation*}
F = \frac{GMm}{R^2}.
\end{equation*}


By evaluating this force in the extreme case where $M \sim M_p$ (Planck mass) and $R \sim L_p$ (Planck length), we obtain:


\begin{equation*}
F_{\text{max, Planck}} = \frac{G M_p^2}{L_p^2} = \frac{c^4}{G}.
\end{equation*}


A refined analysis considering black hole thermodynamics and information bounds introduces an additional factor of $\frac{1}{4}$ in the force bound \cite{Gibbons2002}, leading to the standard maximal force limit in GR.


Within the VAM framework, the maximal force emerges from fundamental vortex circulation properties. The force per unit length associated with a vortex filament is given by:


\begin{equation*}
F_{\Gamma} = \frac{\rho_{\ae} \Gamma^2}{R},
\end{equation*}
where $\rho_{\ae}$ is the \AE ther density, and $\Gamma$ represents the vortex circulation, which, according to Kelvin’s circulation theorem, satisfies:


\begin{equation*}
\Gamma = 2\pi R_c C_e,
\end{equation*}
where $C_e$ is the tangential velocity at the vortex core boundary. This velocity, in turn, is constrained by the internal dynamics of stable vortex structures within the \AE ther.


To impose consistency with relativistic force limits, we introduce a scaling factor that relates vortex-based forces to the Planckian force bound:


\begin{equation*}
F_{\text{max, VAM}} \propto F_{\text{max, GR}} \times \left(\frac{R_c}{L_p}\right)^{-2}.
\end{equation*}


This scaling emerges naturally from the consideration that the vortex-core radius $R_c$ represents the characteristic length scale at which structured vorticity dominates, whereas $L_p$ defines the smallest permissible scale in nature according to quantum gravitational considerations.


A crucial refinement involves incorporating the fine-structure constant $\alpha$, which accounts for quantum electrodynamical corrections to vortex stability and the role of vacuum fluctuations in modifying force constraints:


\begin{equation*}
F_{\text{max, VAM}} = \frac{c^4}{4G} \cdot \alpha \cdot \left(\frac{R_c}{L_p}\right)^{-2}.
\end{equation*}


This result establishes a fundamental connection between the vortex-based force limit in the \AE ther and the relativistic maximal force in GR. The presence of $\alpha$ suggests that electromagnetic interactions are intrinsically linked to vortex stability and force constraints, reinforcing the interpretation that structured vorticity fields act as fundamental carriers of quantum and gravitational information.


\subsubsection*{Implications and Further Considerations}
The derivation above suggests that the force bound in VAM is not merely an arbitrary construct but rather emerges as a consequence of structured vorticity interactions in the \AE theric medium. Several key implications arise from this formulation:
\begin{enumerate}
\item The existence of a force constraint in vortex dynamics implies a fundamental coupling between gravity and electromagnetism, mediated by vorticity.
\item The presence of $\alpha$ indicates that vacuum polarization and quantum field interactions play a role in defining force constraints at the vortex scale.
\item The inverse-square dependence on $R_c/L_p$ suggests that force scaling within the \AE theric substrate follows a predictable hierarchy, transitioning smoothly from vortex-based physics to relativistic and Planckian regimes.
\end{enumerate}


Future work should explore whether laboratory-based superfluid analogues can probe this force constraint experimentally. Additionally, deeper numerical simulations of vortex interactions in high-energy regimes may provide insights into how structured vorticity influences gravitational and quantum field dynamics at fundamental scales.





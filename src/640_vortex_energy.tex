

    \section{Energy-Vorticity Relation}

    \subsection*{Kinetic Energy and Vorticity Scaling}
    The kinetic energy of a vortex tube is proportional to the square of its vorticity magnitude integrated over its volume:
    \begin{equation*}
        E_k \propto \int \omega^2 dV,
    \end{equation*}
    where:
    \begin{itemize}
        \item $\omega$ is the vorticity magnitude,
        \item $dV$ is the infinitesimal volume element.
    \end{itemize}
    For an incompressible fluid without viscosity, the Navier-Stokes equations simplify to:
    \begin{equation*}
        \rho \left( \frac{\partial \mathbf{v}}{\partial t} + \mathbf{v} \cdot \nabla \mathbf{v} \right) = - \nabla p + \rho \mathbf{g}.
    \end{equation*}
    The incompressibility condition adds:
    \begin{equation*}
        \nabla \cdot \mathbf{v} = 0.
    \end{equation*}

    For a stable trefoil knot in a perfect fluid, we assume regions of both rotational and irrotational flow within a pressure-balanced boundary. Rotational regions exhibit nonzero vorticity, while irrotational regions maintain uniform motion.

    \subsection*{Energy-Time Coupling in the Æther Model}
    To integrate the energy-vorticity relation into the Æther model, we analyze how vorticity-driven energy scaling affects local time perception.

    \subsubsection*{1. Vorticity-Driven Energy Scaling}
    Kinetic energy in fluid dynamics:
    \begin{equation*}
        E_k = \int \frac{1}{2} \rho |\vec{u}|^2 dV.
    \end{equation*}
    Substituting velocity with vorticity-driven equivalents:
    \begin{equation*}
        E_k \propto \int \omega^2 dV.
    \end{equation*}
    For a vortex tube of cross-sectional area $A$ and height $h$:
    \begin{equation*}
        dV = A \cdot h.
    \end{equation*}
    If $A$ shrinks by a factor $k^2$ while $h$ remains constant, vorticity scales as:
    \begin{equation*}
        \omega' = k^2 \omega.
    \end{equation*}
    Thus, energy scales as:
    \begin{equation*}
        E_k' \propto k^4 E_k.
    \end{equation*}

    \subsubsection*{2. Time Perception Scaling}
    In the Æther model, local time perception $t_{\text{vortex}}$ depends on vorticity potential $\Phi_{\text{vortex}}$:
    \begin{equation*}
        t_{\text{vortex}} \propto \frac{1}{\Phi_{\text{vortex}}}.
    \end{equation*}
    Since $\Phi_{\text{vortex}} \propto \omega$:
    \begin{equation*}
        t_{\text{vortex}} \propto \frac{1}{\omega}.
    \end{equation*}
    Applying the vorticity scaling:
    \begin{equation*}
        t_{\text{vortex}}' \propto \frac{1}{k^2 \omega}.
    \end{equation*}
    This implies that as the vortex compresses, local time inside the vortex flows faster by a factor of $k^2$.

    \subsubsection*{3. Energy-Time Coupling Equation}
    Combining both results:
    \begin{equation*}
        t_{\text{vortex}}' = \frac{t_{\text{vortex}}}{\sqrt{E_k' / E_k}}.
    \end{equation*}
    Substituting $E_k' \propto k^4 E_k$:
    \begin{equation*}
        t_{\text{vortex}}' = \frac{t_{\text{vortex}}}{k^2}.
    \end{equation*}
    Thus, local time flow scales inversely with energy concentration from vortex compression.

    \subsubsection*{4. Fundamental Energy-Time Coupling Equation}
    Generalizing in terms of energy density $\rho_{\text{vortex}}$:
    \begin{equation*}
        t_{\text{vortex}} \propto \frac{1}{\sqrt{\rho_{\text{vortex}}}}.
    \end{equation*}
    Since energy density is proportional to vorticity squared:
    \begin{equation*}
        \rho_{\text{vortex}} \propto \omega^2,
    \end{equation*}
    we derive the final energy-time relation:
    \begin{equation*}
        t_{\text{vortex}} \propto \frac{1}{\omega}.
    \end{equation*}

    \subsection*{Implications in the Æther Model}
    \begin{itemize}
        \item \textbf{Mass Generation:} If mass arises from energy density due to vortex compression, mass $M$ scales as:
        \begin{equation*}
            M \propto \rho_{\text{vortex}} \propto \omega^2.
        \end{equation*}
        \item \textbf{Time Perception Shift:} In high-vorticity regions, time flows faster, creating a relativistic-like time contraction effect without requiring spacetime curvature.
        \item \textbf{Gravitational Analog:} The inverse relationship suggests a gravitational-like effect—compressing a vortex increases mass and modifies time-flow dynamics, akin to how mass warps spacetime in General Relativity.
    \end{itemize}

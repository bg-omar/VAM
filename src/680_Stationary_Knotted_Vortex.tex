
\section{Mathematical Formulation of Stationary Knotted Vortex Structures in the Vortex \AE ther Model}

\subsection{Governing Equations}
The Vortex \AE ther Model (VAM) describes a stationary knotted vortex as a topologically conserved, self-sustaining rotational structure embedded within an inviscid, incompressible \AE theric medium. The fundamental governing equations describing its dynamics are given by:

\begin{equation*}
\frac{D \mathbf{v}}{Dt} = -\frac{1}{\rho_{\ae}} \nabla P,
\end{equation*}

\begin{equation*}
\frac{D \boldsymbol{\omega}}{Dt} = (\boldsymbol{\omega} \cdot \nabla) \mathbf{v},
\end{equation*}

where $\mathbf{v}$ is the velocity field, $P$ represents the pressure, $\rho_{\ae}$ is the \AE ther density, and $\boldsymbol{\omega} = \nabla \times \mathbf{v}$ denotes the vorticity field.

For a stationary vortex structure, the time dependence vanishes ($\partial_t = 0$), reducing the governing equation to:

\begin{equation*}
J(\psi, \omega) + B \frac{\partial \psi}{\partial x} + \nabla^2 \psi (\lambda + 1) + F \psi + \Phi_{\text{vortex}} + \kappa H + \frac{\Gamma^2}{4 \pi R_c} = 0,
\end{equation*}

where $J(\psi, \omega)$ represents the non-linear interaction between the vorticity field and the stream function $\psi$. The term $\Phi_{\text{vortex}}$ accounts for the equilibrium pressure contribution to the vortex dynamics, incorporating effects such as external forcing and energy redistribution.

\subsection{Helicity Conservation and Knotted Topology}
The helicity of the vortex structure, a fundamental topological invariant, is defined as:

\begin{equation*}
H = \int_V \boldsymbol{\omega} \cdot \mathbf{v} \ dV,
\end{equation*}

which remains conserved for a knotted vortex. In the case of a fundamental trefoil knot configuration, the helicity may be approximated as:

\begin{equation*}
H = \alpha_k L_k \psi \nabla^2 \psi,
\end{equation*}

where $L_k$ is the linking number of the vortex knot, and $\alpha_k$ is a proportionality constant reflecting its topological stability. This conservation law is critical in ensuring the persistence of vortex structures and their interaction within the \AE ther.

\subsection{Separation of Variables and Eigenmode Solutions}
To explore analytical solutions, we assume a separable form for the stream function in spherical coordinates $(r, \theta, \phi)$:

\begin{equation*}
\psi(r, \theta, \phi) = R(r) \Theta(\theta) \Phi(\phi),
\end{equation*}

which leads to the radial equation:

\begin{equation*}
\frac{1}{r^2} \frac{d}{dr} \left( r^2 \frac{dR}{dr} \right) - \frac{\ell (\ell+1)}{r^2} R + F R + \Phi_{\text{vortex}} = 0.
\end{equation*}

The angular and azimuthal components are governed by:

\begin{equation*}
\frac{1}{\Theta} \frac{d^2\Theta}{d\theta^2} + \ell(\ell+1) = 0,
\end{equation*}

\begin{equation*}
\frac{1}{\Phi} \frac{d^2\Phi}{d\phi^2} + m^2 = 0.
\end{equation*}

These equations define the spatial distribution of the vortex structure, providing insight into the stability and configuration of knotted vortex filaments within the \AE theric continuum.

\subsection{Equilibrium Pressure Field and Stability}
The pressure potential $\Phi_{\text{vortex}}$ satisfies Poisson's equation, which is coupled to the energy density distribution of the vortex field:

\begin{equation*}
\nabla^2 \Phi_{\text{vortex}} = -4\pi G_{\text{fluid}} \rho_{\text{energy}},
\end{equation*}

where the local energy density is given by:

\begin{equation*}
\rho_{\text{energy}} = \frac{1}{2} |\nabla \psi|^2 + \frac{\omega^2}{2}.
\end{equation*}

By integrating over the vortex domain, we obtain:

\begin{equation*}
\Phi_{\text{vortex}}(r) = -4\pi G_{\text{fluid}} \int_0^r \rho_{\text{energy}}(r') r'^2 dr'.
\end{equation*}

For the vortex structure to remain stable, the equilibrium condition necessitates:

\begin{equation*}
\frac{d\Phi_{\text{vortex}}}{dr} = F R(r) + \frac{\Gamma^2}{4 \pi R_c}.
\end{equation*}

This condition ensures that pressure gradients balance the vortex-induced forces, leading to long-lived knotted structures that exhibit self-sustaining rotational motion within the \AE ther.

\subsection{Quantization and Experimental Implications}
The discrete nature of helicity enforces a structured stability condition for vortex knots, leading to an eigenmode constraint:

\begin{equation*}
H = n h, \quad n \in \mathbb{Z}.
\end{equation*}

This result aligns with theoretical predictions from vortex filament theory and suggests that experimental validation may be possible through observations of quantized vorticity in superfluid analog systems. The structured nature of these vortex fields implies the possibility of direct measurement of helicity quantization in controlled laboratory environments, such as Bose-Einstein condensates or high-energy plasma configurations.

Furthermore, the implications for astrophysical and cosmological modeling suggest that similar vortex structures may play a role in large-scale fluid dynamical processes governing stellar and galactic formations. The framework established here provides a robust theoretical foundation for investigating structured vortex dynamics and their relevance in \AE theric interactions, opening a novel avenue for exploring gravitational and electromagnetic phenomena through a vorticity-based paradigm.





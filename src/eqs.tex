%! Author = Omar Iskandarani
%! Date = 2/15/2025
\documentclass[a4paper,10pt]{article}
\usepackage{amsmath, amssymb, graphicx, hyperref, physics}
\usepackage[a4paper,margin=1in]{geometry}
\usepackage{array}
\usepackage{booktabs}
\usepackage{amsmath, amssymb, graphicx, hyperref, physics}
\usepackage{graphicx}
\geometry{margin=1in}


\title{Vortex Æther Model: Core Equations and Constants}
\author{Omar Iskandarani}
\date{\today}

\begin{document}
    \maketitle
    \begin{table}[htbp]
        \centering
        \renewcommand{\arraystretch}{1.0}
        \begin{tabular}{c l}
            \toprule
            Symbol & Description \\
            \midrule
            \( V \) & Mass of liquid in circular motion (Vortex) \\
            \( \Gamma \) & Vortex circulation strength: \( \oint \mathbf{v} \cdot d\mathbf{s} \) \\
            \( \omega \) & Vorticity magnitude \(\nabla \times \mathbf{v} \) \\
            \( \Phi \) & Vorticity-induced potential function, satisfying \( \nabla^2 \Phi = -\omega \). \\
            \( R \) & Characteristic vortex radius, representing the scale of rotation. \\
            \( \lambda \) & Vortex core parameter, related to the characteristic decay length of vorticity. \\
            \( L \) & Rotational vortex core length \\
            \( \Psi \) & Stream function of vortex motion \( \mathbf{v} = \nabla \times \Psi \). \\
            \( \Psi_k \) & Vortex knot function describing topological structures in the Æther. \\
            \( \rh\rho_\text{\ae} \) & Local Æther density, assumed to be incompressible in the model. \\
            \( P \) & Pressure in the Æther model, often governed by Bernoulli-like principles. \\
            \( H \) & Helicity, a measure of the knottedness of vortex tubes: \( H = \int \mathbf{v} \cdot \mathbf{\omega} \, dV \). \\
            \( K \) & Enstrophy, representing rotational energy density: \( K = \frac{1}{2} \int \omega^2 dV \). \\
            \( \mathbf{v} \) & Velocity vector field \\
            \( \mathbf{\Omega} \) & Angular velocity vector \\
            \( \mathbf{A} \) & Vector potential, where \( \mathbf{B} = \nabla \times \mathbf{A} \) in magnetohydrodynamic analogies. \\
            \( \mathbf{J} \) & Vortex current density, defined by \( \mathbf{J} = \nabla \times \omega \). \\
            \bottomrule
        \end{tabular}
        \caption{Glossary of Terms for Incompressible Non-Viscous Liquid Æther}
        \label{tab:symbols}
    \end{table}
    % Table of Physical Constants used in the Vortex Æther Model
    \begin{table}[htbp]
        \centering
        \renewcommand{\arraystretch}{1.0}
        \begin{tabular}{lllc}
            \toprule
            \textbf{Symbol} & \textbf{Value} & \textbf{Unit} & \textbf{Quantity} \\
            \midrule
            $C_e$ & $1.09384563 \times 10^6$ & $\text{m s}^{-1}$ & Vortex-Core Tangential Velocity \\
            $F_c$ & $29.053507$ & $\text{N}$ & Coulomb Force \\
            $r_c$ & $1.40897017 \times 10^{-15}$ & $\text{m}$ & Vortex-Core Radius \\
            $R_e$ & $2.8179403262 \times 10^{-15}$ & $\text{m}$ & Classical Electron Radius \\
            $c$ & $2.99792458 \times 10^8$ & $\text{m s}^{-1}$ & Speed of Light in Vacuum \\
            $\alpha_g$ & $1.7518 \times 10^{-45}$ & - & Gravitational Coupling Constant \\
            $G$ & $6.67430 \times 10^{-11}$ & $\text{m}^3 \text{kg}^{-1} \text{s}^{-2}$ & Newtonian Constant of Gravitation \\
            $h$ & $6.62607015 \times 10^{-34}$ & $\text{J Hz}^{-1}$ & Planck Constant \\
            $\alpha$ & $7.2973525643 \times 10^{-3}$ & - & Fine-Structure Constant \\
            $a_0$ & $5.29177210903 \times 10^{-11}$ & $\text{m}$ & Bohr Radius \\
            $M_e$ & $9.1093837015 \times 10^{-31}$ & $\text{kg}$ & Electron Mass \\
            $M_{\text{proton}}$ & $1.67262192369 \times 10^{-27}$ & $\text{kg}$ & Proton Mass \\
            $M_{\text{neutron}}$ & $1.67492749804 \times 10^{-27}$ & $\text{kg}$ & Neutron Mass \\
            $k_B$ & $1.380649 \times 10^{-23}$ & $\text{J K}^{-1}$ & Boltzmann Constant \\
            $R$ & $8.314462618$ & $\text{J mol}^{-1} \text{K}^{-1}$ & Gas Constant \\
            $\lambda_c$ & $2.42631023867 \times 10^{-12}$ & $\text{m}$ & Electron Compton Wavelength \\
            \bottomrule
        \end{tabular}
        \caption{List of Physical Constants Used in the Vortex Æther Model (VAM)}
        \label{tab:vam_constants}
    \end{table}



    \begin{gather*}
        \R_e = \frac{\lambda_c}{2 \pi} \alpha\\
        \R_e = \frac{e^2}{4 \pi \varepsilon_0 M_e c^2}\\
        \R_e = 2 R_c\\
        \R_e =  \alpha^2 a_0\\
        \R_e = \frac{e^2}{4 \pi \varepsilon_c m_c c^2}\\
        \R_e = \frac{e^2}{8 \pi \varepsilon_0 F_{\text{max}} R_c}\\
        R_x = N \frac{F_{\max} R_c^2}{M_e Z C_e^2}\\
        e=\frac{\sqrt{16 \pi F_{max} R_c^2}}{\mu_0 c^2}\\
        e^2=16 \pi F_{max} \xi_0 R_e^2\\
        e=\frac{\sqrt{2 \alpha h}}{\mu_0 c}\\
        e = \frac{\sqrt{4 C_e h}}{\mu_0 c^2}\\
        R^2 = \frac{N F_{\text {max }} R_c}{4 \pi^2 f^2 m_e}\\
        R^2 = \frac{4 \pi F_{\text{max}} R_c^2}{C_e} \frac{1}{8 \pi^2 M_e f_e}\\
        \frac{1}{R_c} = \frac{c^2}{a_0 2 C_e^2}\\
        {L_p}=\sqrt{\frac{\hbar G}{c^3}}\\
        L_{planck} = \frac{\lambda_e C_e t_{planck}}{2 \pi R_c}\\
        L_{\text{Planck}} = \sqrt{\frac{\alpha_g \hbar R_c}{C_e M_e}}.\\
        L_{\text{Planck}} = \sqrt{\frac{\hbar t_p^2 C_e c^2}{2 F_{\text{max}} R_c^2}}.\\
        G =\frac{\vec{C_e} c^3 l_p^2}{2 F_{\max } R_c{ }^2}\\
        G=\frac{C_e c^3 t_p^2}{R_c m_e}\\
        G=\frac{F_{\operatorname{max}} \alpha (c t_p)^2}{m_e^2}\\
        G = \frac{C_e c L_{\text{Planck}}^2}{R_c M_e}.\\
        G = \frac{\alpha_g c^3 R_c}{C_e M_e}\\
        G = \frac{C_e c^3 t_p^2}{R_c \frac{2 F_{\text{max}} R_c}{c^2}} = \frac{C_e c^5 t_p^2}{2 F_{\text{max}} R_c^2}.\\
        \alpha = \frac{\lambda_e}{4 \pi R_c}\\
        \alpha = \frac{C_e e^2}{8 \pi \varepsilon_0 R_c^2 c F_{\text{max}}}\\
        \alpha = \frac{\lambda_c}{4 \pi R_c}\\
        2\alpha^-1=\frac{\omega_c R_c}{C_e}\\
        \alpha = \frac{\frac{c}{2 \alpha} e^2}{8 \pi \varepsilon_0 R_c^2 c F_{\text{max}}} \implies \alpha^2 = \frac{e^2}{16 \pi \varepsilon_0 R_c^2 F_{\text{max}}}.\\
        \alpha_g =  \frac{2F_{\text{max}} C_e t_p^2}{\frac{2F_{\text{max}} R_c^2}{C_e}}\\
        \alpha_g =\frac{C_e^2 t_p^2}{R_c^2}\\
        \alpha_g =\frac{F_{max} 2 C_e t_p^2}{\hbar}\\
        \alpha_g=\frac{F_{\text {max }} t_p^2}{A_0 M_e}\\
        \alpha_g = \frac{C_e c^2 t_p^2 m_e}{\hbar R_c}\\
        \alpha_g = \frac{C_e^2 L_{\text{Planck}}^2}{R_c^2 c^2}.\\
        M_e = \frac{2 F_{\text{max}} R_c}{c^2}\\
        f_e = \frac{C_e}{2 \pi R_c}\\
        \lambda_c=\frac{2 \pi c R_c}{C_e}\\
        \lambda_c=\frac{4 \pi F_{\max } R_c^2}{C_e m_e C}\\
        M_e c^2 =2 F_{\max } R_c\\
        \lambda_c=\frac{2 \pi c R_c}{C_e}\\
        \lambda_c =\frac{4 \pi F_{\max } R_c^2}{C_e m_e C}\\
        \lambda_c = \frac{4 \pi R_c}{C_e}.\\
        C_e = \frac{c}{2 \alpha}.\\
        R_c = \frac{R_e}{2}.\\
        F_{\text{centrifugal}} \sim M_e R_c \left(\frac{C_e}{R_c}\right)^2 = \frac{M_e C_e^2}{R_c}.\\
        h=4 \pi m_e C_e A_0\\
        h=\frac{4 \pi F_\text {max } R_e^2}{C_e}\\
        h = \frac{16 \pi F_\text{max}^2 R_c^3 A_0}{\hbar c^2}.\\
        R_\infty = \frac{C_e^3}{\pi R_c c^3}\\
        C_e = \frac{R_c M_e c^2}{\hbar}\\
        F_{\text{max}} = \frac{1}{2} \left( \frac{C_e}{c} \right)^{-2} M_e \omega_c^2 R_c\\
        F_{\text{max}} = \frac{h \alpha c}{8 \pi R_c^2}\\
        F_{\text{max}} = \frac{e^2}{16 \pi \varepsilon_0 R_c^2}\\
        u_{\text{vortex}}(r, \omega, T) = \frac{F_\text{max} \omega^3}{C_e r^2} \cdot \frac{1}{e^{\hbar \omega / k_B T} - 1},\\
        s_{\text{vortex}}(r, T) = \frac{4 \pi^4 F_\text{max} k_B^4 T^3}{45 C_e r^2 \hbar^4}.\\
        \Phi_{\text{vortex}} = \frac{\pi^4 F_\text{max} k_B^4 T^4}{15 \hbar^4 r}.\\
        u_{\text{total}}(T) \propto \frac{F_\text{max} T^4}{C_e r^2}.\\
        s_{\text{total}}(T) \propto \frac{F_\text{max} T^3}{C_e r^2}.\\
    \end{gather*}





    \section{Validated VAM Equations}\label{sec:validated-vam-equations}
    \begin{align}
        \begin{equation*}
            R_e = \frac{\lambda_c}{2 \pi} \alpha \\\label{eq:equation4}
        \end{equation*}
        \begin{equation*}
            R_e = \frac{e^2}{4 \pi \varepsilon_0 M_e c^2} \\\label{eq:equation5}
        \end{equation*}
        \begin{equation*}
            R_e = 2 r_c \\\label{eq:equation6}
        \end{equation*}
        \begin{equation*}
            R_e =  \alpha^2 a_0 \\\label{eq:equation7}
        \end{equation*}
        \begin{equation*}
            R_e = \frac{e^2}{4 \pi \varepsilon_c m_c c^2} \\\label{eq:equation9}
        \end{equation*}
        \begin{equation*}
            R_e = \frac{e^2}{8 \pi \varepsilon_0 F_{\text{max}} r_c} \\\label{eq:equation8}
        \end{equation*}
        \begin{equation*}
            R_x = N \frac{F_{\max} r_c^2}{M_e Z C_e^2} \\\label{eq:equation10}
        \end{equation*}
        \begin{equation*}
            e =\frac{\sqrt{16 \pi F_{\max} r_c^2}}{\mu_0 c^2} \\\label{eq:equation11}
        \end{equation*}
        \begin{equation*}
            e^2 =16 \pi F_{\max} \xi_0 R_e^2 \\\label{eq:equation13}
        \end{equation*}
        \begin{equation*}
            e =\frac{\sqrt{2 \alpha h}}{\mu_0 c} \\\label{eq:equation12}
        \end{equation*}
        \begin{equation*}
            e = \frac{\sqrt{4 C_e h}}{\mu_0 c^2} \\\label{eq:equation14}
        \end{equation*}
        \begin{equation*}
            R^2 = \frac{N F_{\text {max }} r_c}{4 \pi^2 f^2 m_e} \\\label{eq:equation15}
        \end{equation*}
        \begin{equation*}
            R^2 = \frac{4 \pi F_{\text{max}} r_c^2}{C_e} \frac{1}{8 \pi^2 M_e f_e} \\\label{eq:equation16}
        \end{equation*}
        \begin{equation*}
            \frac{1}{r_c} = \frac{c^2}{a_0 2 C_e^2}\label{eq:equation17}
        \end{equation*}
    \end{align}

\end{document}

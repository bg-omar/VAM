\documentclass[12pt]{article}
\usepackage{amsmath,amssymb,graphicx,hyperref}
\usepackage[a4paper,margin=1in]{geometry}

\title{The Vortex \AE ther Model: A Vorticity-Based Framework for Gravity and Electromagnetism}
\author{Your Name}
\date{2024}

\begin{document}
    \maketitle

    \begin{abstract}
        This paper introduces the Vortex \AE ther Model (VAM), a novel approach to fundamental physics where gravity and electromagnetism emerge from vorticity fields in an incompressible, inviscid \AE ther. Unlike General Relativity, which relies on spacetime curvature, VAM posits that gravitational attraction arises from pressure gradients induced by vortex filaments. We develop the governing equations of VAM, analyze its implications for photon dynamics, and propose experimental tests for validation.
    \end{abstract}

    \section{Introduction}
    The nature of gravity and electromagnetism has long been debated. Classical \AE ther theories, as explored by Maxwell and Kelvin, proposed a medium for wave propagation. VAM extends this concept by modeling the \AE ther as a structured superfluid, where stable vortex knots correspond to fundamental particles, and pressure gradients drive gravitational interactions.

    \section{Mathematical Formulation of VAM}

    \subsection{Vorticity Transport Equation}

    the vorticity transport equation describes the evolution of vorticity in a fluid flow. It accounts for the advection of vorticity by the velocity field and its stretching and tilting due to velocity gradients. In the context of VAM, this equation governs the behavior of vortex filaments in the \AE ther medium, ensuring their conservation and interaction dynamics:

    \begin{equation}
        \frac{D\boldsymbol{\omega}}{Dt} = (\boldsymbol{\omega} \cdot \nabla) \mathbf{v} - \boldsymbol{\omega} (\nabla \cdot \mathbf{v})
    \end{equation}

    \subsection{Biot-Savart Law for Vortex Filaments}
    \begin{equation}
        \mathbf{v}(\mathbf{r}) = \frac{1}{4\pi} \int \frac{\boldsymbol{\omega} \times (\mathbf{r} - \mathbf{r'})}{|\mathbf{r} - \mathbf{r'}|^3} d^3\mathbf{r'}
    \end{equation}

    \subsection{Bernoulli Equation and Gravity}
    \begin{equation}
        P + \frac{1}{2} \rho v^2 + \rho \Phi = \text{constant}
    \end{equation}

    \subsection{Photon Frequency as a Function of Vortex Circulation}
    \begin{equation}
        f = \frac{\Gamma}{2\pi R}
    \end{equation}

    \section{Validated VAM Equations}
    \begin{align}$$
        R_e &= \frac{\lambda_c}{2 \pi} \alpha \\
        R_e &= \frac{e^2}{4 \pi \varepsilon_0 M_e c^2} \\
        R_e &= 2 r_c \\
        R_e &=  \alpha^2 a_0 \\
        R_e &= \frac{e^2}{4 \pi \varepsilon_c m_c c^2} \\
        R_e &= \frac{e^2}{8 \pi \varepsilon_0 F_{\text{max}} r_c} \\
        R_x &= N \frac{F_{\max} r_c^2}{M_e Z C_e^2} \\
        e &=\frac{\sqrt{16 \pi F_{\max} r_c^2}}{\mu_0 c^2} \\
        e^2 &=16 \pi F_{\max} \xi_0 R_e^2 \\
        e &=\frac{\sqrt{2 \alpha h}}{\mu_0 c} \\
        e &= \frac{\sqrt{4 C_e h}}{\mu_0 c^2} \\
        R^2 &= \frac{N F_{\text {max }} r_c}{4 \pi^2 f^2 m_e} \\
        R^2 &= \frac{4 \pi F_{\text{max}} r_c^2}{C_e} \frac{1}{8 \pi^2 M_e f_e} \\
        \frac{1}{r_c} &= \frac{c^2}{a_0 2 C_e^2} \\
        L_p &= \sqrt{\frac{\hbar G}{c^3}} \\
        G &= \frac{C_e c^3 t_p^2}{r_c M_e} \\
        \alpha &= \frac{\lambda_e}{4 \pi r_c} \\
        M_e &= \frac{2 F_{\text{max}} r_c}{c^2} \\
        f_e &= \frac{C_e}{2 \pi r_c} \\
        \lambda_c &= \frac{2 \pi c r_c}{C_e} \\
        C_e &= \frac{c}{2 \alpha} \\
        r_c &= \frac{R_e}{2} \\
        F_{\text{centrifugal}} &\sim \frac{M_e C_e^2}{r_c} \\
        h &= \frac{4 \pi F_{\text{max}} R_e^2}{C_e} \\
        R_\infty &= \frac{C_e^3}{\pi r_c c^3} \\
        F_{\text{max}} &= \frac{h \alpha c}{8 \pi r_c^2}$$
    \end{align}

    \section{Gravity as a Vorticity-Induced Pressure Gradient}
    - Gravitational attraction follows naturally from vortex pressure fields.
    - No singularities; replaces the concept of mass curvature with rotational fluid dynamics.

    \section{Electromagnetism as a Vortex Filament Network}
    - Reformulation of Maxwell’s Equations in terms of vorticity.
    - Magnetic field as a direct consequence of circulating \AE ther flows.

    \section{Experimental Predictions and Feasibility}
    - Rotating Bose-Einstein condensates as superfluid analogues.
    - Electromagnetic anomalies predicted for high-vorticity plasmas.
    - Proposed laboratory tests and detection methods.

    \section{Conclusion}
    We have outlined a vortex-based approach to gravity and electromagnetism, replacing spacetime curvature with fluid dynamics in an inviscid \AE ther. This framework provides a coherent mathematical model with experimentally testable predictions.

\end{document}

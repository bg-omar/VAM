%! Author = Omar Iskandarani
%! Date = 2/15/2025

\documentclass[aps,preprint,superscriptaddress]{revtex4-2}

\usepackage{array}
\usepackage{booktabs}
\usepackage{amsmath}
\usepackage{amssymb}
\usepackage{graphicx}
\usepackage{hyperref}
\usepackage{physics}




\begin{document}

\author{Omar Iskandarani}
\title{The Vortex Æther Model: Unifying Gravity, Electromagnetism, and Quantum Physics under a 3D, Non-Relativistic, vortex framework}
\date{\today}
\affiliation{Independent Researcher, Groningen, The Netherlands}
\thanks{ORCID: \href{https://orcid.org/0009-0006-1686-3961}{0009-0006-1686-3961}}
\email{info@omariskandarani.com}


%% Introduction
\input{Introduction}

%% Part I - Foundational Considerations
\section*{Part I: Foundational Considerations}\label{sec:part-1}
\input{History}
%! Author = mr
%! Date = 3/13/2025

\section{Superfluid-Like Æther and Vorticity}

At the heart of the Vortex Æther Model (VAM) lies the proposition that space is filled by an inviscid, superfluid-like medium—an Æther—whose key dynamical variable is vorticity. In classical fluid dynamics, vorticity \(\boldsymbol{\omega}\) is defined by
\[
    \boldsymbol{\omega} \;=\; \nabla \times \mathbf{v},
\]
where \(\mathbf{v}\) is the local velocity field of the fluid. In an ordinary fluid, viscosity eventually diffuses vorticity. By contrast, the VAM Æther is assumed inviscid and potentially quantized, akin to superfluid helium where discrete vortex filaments can persist indefinitely without dissipating. Building on ideas from Helmholtz and Kelvin—who proposed stable vortex rings in an inviscid fluid as analogs for atoms—this picture reinterprets fundamental particles and interactions in terms of persistent topological flow structures rather than pointlike entities.
\footnote{Helmholtz, Hermann. “On Integrals of the Hydrodynamical Equations which Express Vortex-Motions.” \textit{Philosophical Magazine}, vol. 33, 1867, pp. 485–512.}
\footnote{Thomson (Lord Kelvin), William. “On Vortex Atoms.” \textit{Proceedings of the Royal Society of Edinburgh}, vol. 6, 1867, pp. 94–105.}

\section{Fundamental Constants in the Æther}

Within VAM, a small set of fundamental constants characterizes the structure and dynamics of this superfluid-like medium. Notable examples include the vortex-core tangential velocity,
\[
    C_{e} \;\approx\; 1.0938456 \times 10^{6} \, \text{m s}^{-1},
\]
which sets the characteristic scale for rotational flow speeds, and the Coulomb barrier radius,
\[
    r_{c} \;\approx\; 1.40897017 \times 10^{-15} \, \text{m},
\]
which designates the minimal vortex-core size. These constants, together with a hypothesized “maximum force” \(F_{\text{max}}\) and an Æther density \(\rho_{\mathrm{\AE}}\), distinguish VAM from both standard quantum field theory and classical fluid approaches. Their precise numerical values derive from matching vortex-based formulations of charge and mass with empirically measured quantities—echoing how Planck’s constant arises from blackbody radiation yet underlies a host of quantum effects.\footnote{Wilczek, Frank. “Quantum Field Theory.” \textit{Reviews of Modern Physics}, vol. 71, 1999, pp. S85–S95, doi:10.1103/RevModPhys.71.S85.}

\subsection*{Gravitational Attraction via Vortex-Induced Pressure Gradients}
The vorticity-induced gravitational field satisfies the Poisson-like equation:

\[
    \nabla^2 \Phi_{v} \;=\; -\,\rho_{\mathrm{\AE}} \,\bigl|\boldsymbol{\omega}\bigr|^{2},
\]
where \(\Phi_{v}\) is the gravitational-like potential that emerges from the fluid’s vorticity. A full derivation using Euler’s equations and fluid vorticity transport can be found in Appendix 2.
\footnote{Donnelly, Russell J. \textit{Quantized Vortices in Helium II}. Cambridge University Press, 1991.} This vorticity-based approach recasts gravitational phenomena in purely three-dimensional terms, aligning with the Euclidean spatial geometry central to VAM.

\section{Electromagnetism as Structured Vortex Interactions}

Maxwell’s original insight into electromagnetic fields being states of stress in a medium inspires the VAM perspective that electric and magnetic fields correspond to stable vortex-flow configurations in the Æther.\footnote{Maxwell, James Clerk. “A Dynamical Theory of the Electromagnetic Field.” \textit{Philosophical Transactions of the Royal Society of London}, vol. 155, 1865, pp. 459–512, doi:10.1098/rstl.1865.0008.} Instead of positing separate force-carrying particles (photons) or curved four-dimensional fields, VAM defines “electromagnetic” effects in terms of circulation and linked vortex filaments. For instance, electric charges are understood as knotted vortex loops whose net winding number sets the charge magnitude. The Lorentz force law—\(\mathbf{F} = q\,(\mathbf{E} + \mathbf{v}\times\mathbf{B})\)—arises from the exchange of vorticity and momentum between these loops, reproducing key electromagnetic phenomena without invoking extra dimensions.

\subsection*{Quantum Features from Helicity and Knotted Vortices}

Perhaps the most striking implication of VAM is its natural linkage between knotted vortex structures and quantum discreteness. In superfluid helium, quantized vortices have circulation in integer multiples of \(\kappa = h / m\), suggesting that angular momentum and energy can only take discrete values in an inviscid, quantized flow.\footnote{Barenghi, Carlo F., and Ladislav Skrbek. “Introduction to Quantum Turbulence.” \textit{Proceedings of the National Academy of Sciences}, vol. 111, suppl. 1, 2014, pp. 4647–4652, doi:10.1073/pnas.1312549111.} VAM generalizes this principle by modeling electrons, protons, and other fundamental particles as stable vortex knots with conserved helicity:
\[
    H \;=\; \int \boldsymbol{\omega} \,\cdot\, \mathbf{v}\; dV.
\]
Because helicity cannot continuously change without dissipating or reconnecting vortices, physical states become discretized, effectively mirroring the quantum energy levels observed in atomic systems. Wave-particle duality likewise emerges from the fluidic nature of vortex excitations, which can spread out as a wave yet remain localized by their topological core. Consequently, phenomena like electron orbitals, photon emission spectra, and spin angular momentum find a unified explanation in the dynamics of self-sustaining flow loops.

In this manner, VAM unifies gravitational, electromagnetic, and quantum behaviors under a single fluid-based framework. The superfluid-like Æther, governed by vorticity and constrained by quantized circulation, serves as the substrate from which the familiar forces and quantum states of modern physics arise. The subsequent sections detail how these theoretical underpinnings translate into mathematical formulations, including explicit field equations and proposed experimental checks.

\footnote{Wilczek, Frank. “Quantum Field Theory.” \textit{Reviews of Modern Physics}, vol. 71, 1999, pp. S85–S95, doi:10.1103/RevModPhys.71.S85.}
\footnote{Donnelly, Russell J. \textit{Quantized Vortices in Helium II}. Cambridge University Press, 1991.}
\footnote{Maxwell, James Clerk. “A Dynamical Theory of the Electromagnetic Field.” \textit{Philosophical Transactions of the Royal Society of London}, vol. 155, 1865, pp. 459–512, doi:10.1098/rstl.1865.0008.}
\footnote{Barenghi, Carlo F., and Ladislav Skrbek. “Introduction to Quantum Turbulence.” \textit{Proceedings of the National Academy of Sciences}, vol. 111, suppl. 1, 2014, pp. 4647–4652, doi:10.1073/pnas.1312549111.}

\input{FineStructureConstant}

%% Part II - MathematicalFormalism
\section*{Part II: Mathematical Formalism}\label{sec:part-2}
\input{MaxForce}
\input{SwirlVelocityConstant}
\input{DerivationVAMFieldEquation}
\input{MaxwellEquivalentsVorticity}
\input{ElectrostaticChargeElectricFieldsVAM}
\input{MachInspiredScalar}
\input{PhotonVortexDipole}
\input{PhotonVortexDipoleGravitationalCoupling}

%% Part III - Applications and Implications
\section*{Part III: Applications and Implications}\label{sec:part-3}
%! Author = mr
%! Date = 3/13/2025
\subsection*{1. The Electron as a Toroidal Vortex}

\subsubsection*{1.1 Conceptual Basis}
Historically, Helmholtz and Lord Kelvin explored the idea that atoms might be stable knots in an inviscid fluid. In VAM, this idea is adapted to fundamental particles such as electrons. Instead of a pointlike charge, the electron is conceived as a toroidal vortex—a closed loop of rotating Æther—whose core flow and topology define quantized properties (charge, spin, rest mass).

\begin{enumerate}
    \item \textbf{Toroidal Geometry} \\
    A torus can be described by two characteristic radii: the \textit{major radius} \(R\) (distance from the torus center to the core center) and the \textit{minor radius} \(r\) (cross-sectional radius of the vortex tube). In many simplified VAM treatments, these radii are comparable (e.g., a “horn torus,” \(R \approx r\)), ensuring localized, self-sustaining vorticity.

    \item \textbf{Quantization via Circulation} \\
    In superfluids, circulation around a vortex core is quantized in multiples of \(\kappa = h/m\). Applying the same logic, an electron is identified with exactly one quantum of circulation in the Æther. Matching observational data (e.g., Compton wavelengths, classical electron radius) allows one to solve for the swirl velocity constant \(C_e\) and the minor radius \(r_c\).

    \item \textbf{Charge and Helicity} \\
    VAM posits that electric charge is a manifestation of boundary conditions on the vortex. In mathematical terms, a net winding number or linking of the vortex tube with itself (knottedness) plays the role of “charge.” Helicity (the integral \(\int \boldsymbol{\omega}\cdot \mathbf{v}\, dV\)) remains conserved for inviscid, closed loops, explaining both stability and quantization.
\end{enumerate}

\subsubsection*{1.2 Phenomenological Consequences}
\begin{itemize}
    \item \textbf{Wave-Particle Duality}: The toroidal vortex is localized in space yet can exhibit wave-like excitations in the surrounding Æther field—paralleling electron wavefunctions in quantum mechanics.
    \item \textbf{Spin}: The intrinsic angular momentum of a knotted vortex is topologically pinned, explaining spin-\(\tfrac{1}{2}\) as a stable, non-dissipative rotational state.
    \item \textbf{Self-Energy}: The electron’s rest mass can be attributed to vortex rotational energy. Within VAM, inertial mass emerges from the fluid’s resistance to changes in vortex circulation.
\end{itemize}


\subsection*{2. Black Hole Horizons as Extreme Vortex Interiors}

\subsubsection*{2.1 Replacing Singularity with “Vortex Collapse”}
In general relativity, black holes are defined by event horizons and central singularities. VAM interprets these regions as zones of ultra-strong vorticity where the fluid pressure becomes extremely low. Instead of a singularity in curved spacetime, one encounters an extreme vortex interior, possibly saturating velocity at or near the speed of light.

\begin{enumerate}
    \item \textbf{Core Vorticity and “Schwarzschild-Like” Radius} \\
    By analogy with Schwarzschild black holes, one introduces a radius \(r_s\) where the swirl-induced potential approaches a critical threshold. VAM would say \(r_s\) is the boundary where the fluid velocity can no longer increase without structural breakdown.
    \item \textbf{Horizons as Fluid Boundaries} \\
    Just as black holes have horizons inside which light cannot escape, a VAM-based horizon emerges where outward fluid flow is overwhelmed by inward swirl. Beyond this “horizon,” vortex filaments loop indefinitely, preventing external signals from escaping.
\end{enumerate}

\subsubsection*{2.2 Frame-Dragging and Rotating “Black Holes”}
When the vortex core itself rotates (an analog to a Kerr black hole), frame-dragging arises naturally from the fluid swirl. VAM reinterprets ergospheres and ring singularities as topological constraints in rotating vortex cores—no separate “spacetime metric” is needed. Instead, the rotational velocity at the boundary sets how severe the horizon is.

\subsubsection*{2.3 Possible Observable Signatures}
\begin{itemize}
    \item \textbf{Critical Vortex Speed}: Near the horizon, local swirl velocity might approach \(c\). This could produce distinctive gravitational lensing or photon capture phenomena if the vortex geometry couples to electromagnetic waves.
    \item \textbf{Ringdown Patterns}: In a real rotating fluid, small perturbations cause wave excitations (“ringdown modes”) that could mimic black hole gravitational waves but be interpreted purely through vortex fluid oscillations.
\end{itemize}

\subsection*{3. Vorticity-Based Explanation for Dark Matter}

\subsubsection*{3.1 The “Missing Mass” Problem}
Astrophysical observations—spiral galaxy rotation curves, galaxy cluster dynamics, gravitational lensing—suggest more gravitational pull than accounted for by luminous matter. Dark matter is typically invoked to reconcile this discrepancy. VAM offers an alternative perspective: large-scale, low-density vortex flows in galactic halos may produce additional gravitational-like attraction without requiring new, non-luminous matter.

\subsubsection*{3.2 Galactic Halos as Coherent Vortex Structures}
\begin{enumerate}
    \item \textbf{Extended Vorticity in Galaxy Disks} \\
    Many spiral galaxies display rotation curves that flatten at large radii. In VAM, if the interstellar medium and the halo form a gently rotating superfluid-like region, persistent large-scale vorticity could yield an effective gravitational potential well.
    \item \textbf{Pressure Deficit} \\
    Just as in the local solar system, rotating fluids produce inward radial forces. The “dark matter halo” might simply be an extensive swirl region, its boundaries set by the galactic environment and the coherence length of the superfluidic Æther.
\end{enumerate}

\subsubsection*{3.3 Predictions and Testable Consequences}
\begin{itemize}
    \item \textbf{No Additional Particle}: VAM eliminates the need for WIMPs or axions as an invisible matter component. Instead, it predicts that if one could measure the large-scale distribution of vorticity, it would track the “dark matter” gravitational potential.
    \item \textbf{Galaxy Clusters}: Vorticity filaments connecting galaxies in clusters might replicate the large-scale gravitational bridging effects typically attributed to dark matter, possibly visible in X-ray or gravitational-lensing signatures if the vortex flows compress hot gas.
    \item \textbf{Potential Offsets}: In phenomena like the Bullet Cluster (where dark matter distribution appears offset from baryonic mass after collisions), VAM might require specialized vortex-boundary conditions or shock effects. This can provide observational tests to either confirm or refine the vortex-halo idea.
\end{itemize}

\subsection*{4. Concluding Remarks on Toy Models}

Each of these toy models—the electron torus, black hole horizons, and dark matter halos—serves as an illustrative application of the VAM’s core principle: stable vortex flow in an inviscid Æther can account for what we normally attribute to point-particle quantum mechanics, spacetime singularities, and large-scale invisible mass. While these ideas remain unconventional, they showcase how a single, fluid-based framework can unify diverse physical phenomena without resorting to extra dimensions or purely geometric curvature of spacetime. Future research must deepen these toy models—especially via numerical simulations of multi-scale vortex dynamics and further comparisons with astrophysical data.

\input{DiscussionOutlook}

%% References
\bibliography{../src/references}
\bibliographystyle{apsrev4-2}

%% Appendices
\appendix
%! Author = Omar Iskandarani
%! Date = 3/13/2025

\section*{Appendix 1. Detailed Derivation of the Swirl Velocity Constant \(C_e\)}

\subsection*{1. Quantum of Circulation: The Starting Point}

In quantum fluids such as superfluid helium, vortex circulation is quantized in integer multiples of
\[
    \kappa = \frac{h}{m},
\]
where \(h\) is Planck’s constant and \(m\) is the mass of the fluid’s constituent particle (e.g., the helium atom in superfluids). By analogy, VAM postulates that any stable vortex representing a fundamental particle (like an electron) must have circulation locked to a discrete value, typically \(\kappa\).

\subsubsection*{1.1 Physical Interpretation in VAM}
\begin{itemize}
    \item \textbf{Electron as a Torus} \\
    VAM envisions the electron not as a point, but as a knotted or looped vortex in the Æther, whose core radius is \(r_c\).
    \item \textbf{Single Quantum of Circulation} \\
    For the simplest (trefoil-like or single-loop) topology, one quantum \(\kappa\) is assigned—mirroring how an electron carries a single “charge.”
\end{itemize}

Hence, for the fundamental vortex representing the electron, the total circulation \(\Gamma\) around the loop is presumed to be
\[
    \Gamma = \frac{h}{m_e}.
\]
Here \(m_e\) is the electron mass, playing the role analogous to the helium-4 atom mass in superfluids.

\subsection*{2. Geometry of the Vortex Loop}

\subsubsection*{2.1 Definition of Circulation \(\Gamma\)}

For a circular vortex ring of radius \(r_c\), we assume that the tangential velocity at the ring is constant and labeled \(C_e\). Circulation \(\Gamma\) is thus:
\[
    \Gamma = \oint_{\text{ring}} \mathbf{v} \cdot d\mathbf{l} = C_e \cdot 2 \pi r_c,
\]
since \(\mathbf{v} \cdot d\mathbf{l} = C_e \,dl\) around a circle of circumference \(2\pi r_c\).

\subsubsection*{2.2 Matching Quantized Circulation}

From the quantum condition above,
\[
    2 \pi r_c C_e = \frac{h}{m_e}.
\]
Solving for \(C_e\) yields:
\[
    C_e = \frac{h}{2 \pi r_c m_e}.
\]
This identifies \(C_e\) as the swirl (tangential) velocity at the vortex ring radius \(r_c\), determined purely by fundamental constants (\(h\) and \(m_e\)) and the chosen length scale \(r_c\).

\subsection*{3. Connecting \(r_c\) to Empirical Data}

\subsubsection*{3.1 Choice of \(r_c\)}

In VAM, one typically relates \(r_c\) to the “vortex-core radius,” which may be on the order of
\[
    r_c \approx 10^{-15}\,\text{m},
\]
often compared to nuclear or sub-nuclear scales (the proton or electron Compton radius). Different versions of the model might use:
\begin{itemize}
    \item \textbf{Classical Electron Radius}: \(r_e \approx 2.8179 \times 10^{-15}\,\mathrm{m}\), or
    \item \textbf{Coulomb Barrier Radius}: \(r_c \approx 1.4 \times 10^{-15}\,\mathrm{m}\), or
    \item \textbf{Some fraction of the proton’s scale} based on high-energy scattering data.
\end{itemize}

Plugging in a chosen \(r_c\) leads to a numerical value for \(C_e\). For instance:
\[
    r_c \approx 1.4 \times 10^{-15}\,\text{m}, \quad m_e \approx 9.109 \times 10^{-31}\,\text{kg}, \quad h \approx 6.626 \times 10^{-34}\,\text{J\,s},
\]
yields
\[
    C_e \approx 1.0 \times 10^6 \,\text{m/s}.
\]

\subsubsection*{3.2 Dimension Check}

\begin{itemize}
    \item Left side: \([\text{Velocity}] = \text{m s}^{-1}\).
    \item Right side: \([h/(r_c m_e)]\). Since \([h] = \text{(J s)} = \text{(kg m}^2\text{/s)}\times\text{s}\), dividing by \(\text{(kg)} \times \text{m}\) leaves \(\text{m}/\text{s}\), matching the velocity dimension exactly.
\end{itemize}

\subsection*{4. Physical Interpretation and Implications}

\begin{enumerate}
    \item \textbf{Bound on Tangential Velocity} \\
    The swirl velocity \(C_e\) effectively caps how fast the Æther can rotate within the electron-like vortex core. This parallels how the speed of light \(c\) defines a universal limit for ordinary relativistic motion.
    \item \textbf{Link to Electron Charge and Mass} \\
    The link between \(\Gamma = h/m_e\) and the vortex geometry suggests that electron mass, charge, and spin might all be reinterpreted as emergent properties of stable vortex flow in the Æther. VAM often couples this expression with others connecting, e.g., \(\alpha\approx e^2/(4\pi\varepsilon_0\hbar c)\) to show the synergy between electromagnetic constants and fluidic swirl.
    \item \textbf{Universality} \\
    While \(C_e\) is derived in the context of the electron, the same approach can define swirl velocities for other stable vortex knots (e.g., protons, neutrinos) by substituting the appropriate mass and length scale. Each yields its own characteristic swirl speed, potentially offering a topological reason for differing particle masses or quantum states.
\end{enumerate}

\subsection*{5. Conclusion}

This derivation of \(C_e\) reveals how a single quantum of circulation \(\Gamma = h/m_e\), wrapped around a vortex core of radius \(r_c\), leads to a characteristic tangential velocity scale:
\[
    C_e = \frac{h}{2\pi r_c m_e}.
\]
When supplemented with a suitable choice for \(r_c\) based on nuclear or sub-nuclear measurements, it yields the \(\sim10^6\,\text{m/s}\) swirl speed commonly cited in VAM literature. Consequently, \(C_e\) serves as a fundamental velocity constant for vortex-based models of the electron and, by extension, any elementary particle’s stable vortex structure—reinforcing VAM’s viewpoint that basic quantum parameters can be derived from fluid mechanical constraints in a superfluidic Æther.\label{appendix:1}
%! Author = mr
%! Date = 3/13/2025

\section*{Appendix 2. Full Poisson-Like Equation for Vorticity-Based Gravity}

\subsection*{1. Setting the Stage: Inviscid Fluid and Vortex Flow}

\subsubsection*{1.1 Euler’s Equation for an Inviscid Fluid}
In VAM, the Æther is assumed inviscid and (often) incompressible. Neglecting time dependence for simplicity or considering a near-steady flow, the Euler equation takes the form
\[
    \rho_{\mathrm{\AE}}\;(\mathbf{v}\,\cdot\,\nabla)\,\mathbf{v}
    \;=\;
    -\,\nabla p,
\]
where:
\begin{itemize}
    \item \(\rho_{\mathrm{\AE}}\) is the density of the Æther,
    \item \(\mathbf{v}\) is the flow (velocity) field,
    \item \(p\) is the fluid pressure,
    \item \((\mathbf{v}\cdot\nabla)\mathbf{v}\) denotes the convective acceleration.
\end{itemize}

\subsubsection*{1.2 Local Pressure and the Emergent “Potential”}
Unlike a conventional fluid, VAM postulates that \textit{large} vorticity \(\boldsymbol{\omega} = \nabla \times \mathbf{v}\) lowers the local pressure. This pressure deficit acts similarly to how mass density generates gravitational pull. Define a scalar function \(\Phi_v(\mathbf{r})\) such that:
\[
    p(\mathbf{r})
    \;=\;
    p_0 \;-\; \alpha \,\rho_{\mathrm{\AE}}\,\Phi_v(\mathbf{r}),
\]
where \(p_0\) is a reference (far-field) pressure, and \(\alpha\) is a dimensionless coupling constant. Then
\[
    \nabla p
    \;=\;
    -\alpha \,\rho_{\mathrm{\AE}} \,\nabla \Phi_v.
\]
Thus, Euler’s equation becomes
\[
    \rho_{\mathrm{\AE}}\;(\mathbf{v}\,\cdot\,\nabla)\,\mathbf{v}
    \;=\;
    \alpha\,\rho_{\mathrm{\AE}}\,\nabla \Phi_v.
\]
Canceling \(\rho_{\mathrm{\AE}}\) on both sides:
\[
    (\mathbf{v}\,\cdot\,\nabla)\,\mathbf{v}
    \;=\;
    \alpha\,\nabla \Phi_v.
    \tag{1}
\]

\subsection*{2. Relating Convective Acceleration to Vorticity}

\subsubsection*{2.1 Convective Acceleration Identity}
A well-known fluid-mechanics vector identity states:
\[
    (\mathbf{v}\,\cdot\,\nabla)\,\mathbf{v}
    \;=\;
    \nabla\!\bigl(\tfrac12\,|\mathbf{v}|^2\bigr)
    \;-\;
    \mathbf{v}\times(\nabla\times \mathbf{v})
    \;=\;
    \nabla\bigl(\tfrac12\,|\mathbf{v}|^2\bigr)
    \;-\;
    \mathbf{v}\times \boldsymbol{\omega}.
\]
Hence equation (1) can be written as:
\[
    \nabla
    \Bigl(
    \tfrac12 \,\lvert \mathbf{v}\rvert^2
    \Bigr)
    \;-\;
    \mathbf{v}\times \boldsymbol{\omega}
    \;=\;
    \alpha \,\nabla \Phi_v.
    \tag{2}
\]
Rearrange it to:
\[
    \nabla
    \Bigl(
    \tfrac12\,|\mathbf{v}|^2
    \;-\;
    \alpha\,\Phi_v
    \Bigr)
    \;=\;
    \mathbf{v}\times \boldsymbol{\omega}.
    \tag{3}
\]
Taking the curl of both sides yields
\[
    \nabla \times \Bigl[
        \nabla\bigl(
        \tfrac12\,|\mathbf{v}|^2 - \alpha\,\Phi_v
        \bigr)
        \Bigr]
    \;=\;
    \nabla \times \bigl(\mathbf{v}\times \boldsymbol{\omega}\bigr).
\]
But the curl of a gradient \(\nabla \chi\) is zero, so the left side vanishes:
\[
    0
    \;=\;
    \nabla \times (\mathbf{v}\times \boldsymbol{\omega}).
    \tag{4}
\]

\subsubsection*{2.2 Expand \(\nabla \times (\mathbf{v}\times \boldsymbol{\omega})\)}
Using the triple-vector identity,
\[
    \nabla \times (\mathbf{A}\times \mathbf{B})
    \;=\;
    (\mathbf{B}\cdot\nabla)\mathbf{A}
    \;-\;
    (\mathbf{A}\cdot\nabla)\mathbf{B}
    \;+\;
    \mathbf{A}\,(\nabla\cdot \mathbf{B})
    \;-\;
    \mathbf{B}\,(\nabla\cdot \mathbf{A}),
\]
we set \(\mathbf{A} = \mathbf{v}\) and \(\mathbf{B} = \boldsymbol{\omega}\). Thus
\[
    \nabla \times (\mathbf{v}\times \boldsymbol{\omega})
    \;=\;
    (\boldsymbol{\omega}\cdot \nabla)\mathbf{v}
    \;-\;
    (\mathbf{v}\cdot \nabla)\boldsymbol{\omega}
    \;+\;
    \mathbf{v}\,(\nabla\cdot \boldsymbol{\omega})
    \;-\;
    \boldsymbol{\omega}\,(\nabla\cdot \mathbf{v}).
\]
Equation (4) demands this be zero:
\[
    (\boldsymbol{\omega}\cdot \nabla)\mathbf{v}
    \;-\;
    (\mathbf{v}\cdot \nabla)\boldsymbol{\omega}
    \;+\;
    \mathbf{v}\,(\nabla\cdot \boldsymbol{\omega})
    \;-\;
    \boldsymbol{\omega}\,(\nabla\cdot \mathbf{v})
    \;=\;
    0.
    \tag{5}
\]
In many VAM treatments, \(\nabla \cdot \mathbf{v} = 0\) (incompressibility) and \(\nabla \cdot \boldsymbol{\omega} = 0\) (the divergence of a curl is always zero). Then
\[
    (\boldsymbol{\omega}\cdot \nabla)\mathbf{v}
    \;=\;
    (\mathbf{v}\cdot \nabla)\boldsymbol{\omega}.
    \tag{6}
\]
This condition underlies vortex conservation: if \(\boldsymbol{\omega}\) is large in one region, it must remain stable unless boundary interactions (or reconnection) intervene.

\section{Identifying a Poisson-Like Equation for \(\Phi_v\)}

\subsection{Bernoulli-like Relation and Pressure}
From equation (3), we see that
\[
    \tfrac12\,|\mathbf{v}|^2
    \;-\;
    \alpha\,\Phi_v
    \;=\;
    \mathrm{constant}
    \quad
    \text{(along streamlines)},
\]
akin to the Bernoulli principle. Where vorticity is strong, \(\mathbf{v}\) is large, driving \(\Phi_v\) up or down accordingly.

\subsection{Defining \(\nabla^2 \Phi_v\)}
To find a connection between \(\Phi_v\) and \(|\boldsymbol{\omega}|^2\), VAM posits a near-equilibrium relation where the local pressure deficit is proportional to \(|\boldsymbol{\omega}|^2\). Equivalently, we let
\[
    p(\mathbf{r})
    \;=\;
    p_0
    \;-\;
    \alpha\,\rho_{\mathrm{\AE}}
    \;\Phi_v(\mathbf{r}),
\]
and we demand
\[
    \Phi_v
    \;\propto\;
    \int |\boldsymbol{\omega}|^2 \,dV
    \quad
    \text{(locally)},
\]
so that if \(\boldsymbol{\omega}\) is large, \(\Phi_v\) is negative or “deep.”  Making this local and differential, we propose an ansatz:
\[
    \nabla^2 \Phi_v
    \;=\;
    -\,\alpha\,\rho_{\mathrm{\AE}}\;\lvert \boldsymbol{\omega}\rvert^2.
    \tag{7}
\]
Here:
\begin{itemize}
    \item The negative sign ensures that higher vorticity corresponds to a more negative \(\Phi_v\), analogous to how higher mass density \(\rho\) in Newton’s law leads to \(\nabla^2 \Phi = -4\pi G\rho\).
    \item \(\rho_{\mathrm{\AE}}\) sets the scale of the fluid’s inertia (i.e., how strongly it responds to rotation).
    \item \(\alpha\) calibrates the coupling strength between vorticity magnitude and “gravitational potential.”
\end{itemize}

\subsection{Physical Justification}
\begin{enumerate}
    \item \textbf{Analogy with Newtonian Poisson Equation} \\
    In Newtonian gravity, \(\nabla^2 \Phi = -4\pi G\rho\). By analogy, \(\rho_{\mathrm{\AE}}|\boldsymbol{\omega}|^2\) plays the role of an “effective mass density,” producing a negative potential.
    \item \textbf{Stationary Flow Requirement} \\
    In regions of near-steady vortex flow, the net swirl remains approximately constant, so the potential \(\Phi_v\) must solve the above Poisson-like equation with appropriate boundary conditions (\(\Phi_v \to 0\) at large \(r\), for instance).
    \item \textbf{Empirical Matching} \\
    Parameter \(\alpha\) can be fitted to recover standard gravitational results at large distance (where vorticity correlates with mass distribution). In high-swirl regions (like near a black-hole analog or near nuclear-scale vortex knots), this potential saturates or modifies the local “gravitational” field.
\end{enumerate}

\section{Final Boxed Equation}

Thus, the fundamental field equation for vorticity-driven gravity in VAM takes the form:

\[
    \boxed{
        \nabla^2 \Phi_v(\mathbf{r})
        \;=\;
        -\;\alpha\;\rho_{\mathrm{\AE}}\;\bigl|\boldsymbol{\omega}(\mathbf{r})\bigr|^2,
    }
\]
where \(\boldsymbol{\omega} = \nabla \times \mathbf{v}\), \(\rho_{\mathrm{\AE}}\) is the Æther density, and \(\alpha\) is a dimensionless coupling parameter. In analogy to standard Newtonian gravity, \(\Phi_v\) becomes more negative in regions of strong vortex flow, reproducing an “attractive” effect that draws other vortex structures inward.

\section{Concluding Remarks}

\begin{enumerate}
    \item \textbf{Conceptual Shift} \\
    Rather than treating mass-energy as the source of gravitational potential, VAM places vorticity squarely in the driver’s seat. Regions with intense rotation (high \(|\boldsymbol{\omega}|\)) generate deeper potentials and hence stronger “gravitational” pull.
    \item \textbf{Boundary Conditions and Extensions} \\
    Real systems may require boundary conditions that handle compressibility (in astrophysical or high-energy domains) or vortex reconnection events. These nuances can alter the strict Poisson form but keep the same core insight: \textit{vorticity begets gravity-like forces}.
    \item \textbf{Next Steps} \\
    Using equation (7), one can solve for \(\Phi_v\) in specified geometries (e.g., rotating spheres, vortex filaments, or topological knots). Matching these solutions to observed gravitational phenomena (e.g., orbital velocities or lensing effects) offers a novel test of VAM’s validity and predictive power.
\end{enumerate}
\label{appendix:2}
%! Author = Omar Iskandarani
%! Date = 3/13/2025

\section{Appendix 3. Extended Maxwell--VAM Equations in Index Form}

\subsection{Preliminaries and Notation}

\begin{enumerate}
    \item \textbf{Indices}: We use \(i, j, k \in \{1,2,3\}\) to refer to spatial coordinates \(x^1, x^2, x^3\). Time is denoted as \(t\) or \(x^0\) in four-dimensional notation if needed, but VAM preserves a strict three-dimensional geometry with absolute time as an external parameter.
    \item \textbf{Fields}:
    \begin{itemize}
        \item \textbf{VAM-Electric Field}: \(E_{v}^i(\mathbf{r}, t)\)
        \item \textbf{VAM-Magnetic Field}: \(B_{v}^i(\mathbf{r}, t)\)
    \end{itemize}
    These are derived from the underlying fluid velocity \(\mathbf{v}\) and its decomposition into irrotational (\(\nabla \Phi_v\)) and solenoidal (\(\nabla \times \mathbf{A}_v\)) parts. In many treatments:
    \[
        E_{v}^i \;\equiv\; -\,\partial^i \Phi_v,
        \quad
        B_{v}^i \;\equiv\; \epsilon^{ijk}\,\partial_j A_{v,k},
    \]
    where \(\Phi_v\) and \(\mathbf{A}_v\) play roles analogous to the scalar and vector potentials in standard electromagnetism.
    \item \textbf{VAM Charge and Current}:
    \begin{itemize}
        \item \textbf{VAM-Charge Density}: \(\rho_v(\mathbf{r}, t)\)
        \item \textbf{VAM-Current Density}: \(J_{v}^i(\mathbf{r}, t)\)
    \end{itemize}
    These are effective sources or sinks of the vortex flow, representing how vortex filaments might \grqq start\textquotedblright or \grqq end\textquotedblright at boundaries or within certain knots.
    \item \textbf{Coupling Constants}:
    \begin{itemize}
        \item \textbf{Permittivity-like constant}: \(\varepsilon_v\)
        \item \textbf{Permeability-like constant}: \(\mu_v\)
    \end{itemize}
    They set the strength and speed of wave-like excitations in the Æther, analogous to \(\varepsilon_0, \mu_0\) in standard electromagnetism.
\end{enumerate}

\subsection{Gauss's Law for VAM-Electric Field}

In differential (index) form, the usual Gauss's law \(\nabla\cdot\mathbf{E} = \rho/\varepsilon_0\) becomes:

\[
    \partial_i E_{v}^i
    \;=\;
    \frac{\rho_v}{\varepsilon_v},
    \tag{1}
\]
where \(\partial_i \equiv \frac{\partial}{\partial x^i}\). This states that any net vortex \grqq charge\textquotedblright density \(\rho_v\) produces a nonzero divergence in the field \(E_{v}^i\).

\subsection{Gauss's Law for VAM-Magnetic Field}

Because \(\mathbf{B}_v\) arises from rotating flows (akin to \(\nabla\times \mathbf{A}_v\)), no \grqq magnetic monopoles\textquotedblright exist in VAM:

\[
    \partial_i B_{v}^i
    \;=\; 0.
    \tag{2}
\]
This condition expresses the purely solenoidal nature of vortex flows: vortex lines do not begin or end in free space (unless they meet boundaries or other vortex lines to form closed loops or knots).

\subsection{Faraday's Law of Induction in VAM}

The differential form of Faraday's law, \(\nabla\times \mathbf{E} = -\frac{\partial \mathbf{B}}{\partial t}\), in index notation becomes:

\[
    \epsilon_{ijk}\,\partial_j E_{v}^k
    \;=\;
    -\,\frac{\partial B_{v,i}}{\partial t},
    \tag{3}
\]
where \(\epsilon_{ijk}\) is the Levi-Civita symbol (with \(\epsilon_{123} = +1\)). This implies that time-varying \grqq magnetic\textquotedblright fields (i.e. time-varying vortex rotation patterns) induce an irrotational response in \(\mathbf{E}_v\), preserving the fluid continuity.

\subsection{Ampère--Maxwell Law in VAM}

In standard electromagnetism, \(\nabla\times \mathbf{B} = \mu_0 \mathbf{J} + \mu_0\varepsilon_0\,\partial_t\mathbf{E}\). The VAM analog is:

\[
    \epsilon_{ijk}\,\partial_j B_{v}^k
    \;=\;
    \mu_v\,J_{v,i}
    \;+\;
    \mu_v\,\varepsilon_v\;\frac{\partial E_{v,i}}{\partial t}.
    \tag{4}
\]
Here, \(\mathbf{J}_v\) is the effective \grqq vortex current,\textquotedblright capturing how net inflows or outflows of vortex lines transit across a given area. Just as in Maxwell's correction, a changing \grqq electric\textquotedblright field (\(\partial_t E_{v,i}\)) contributes to the curl of \(\mathbf{B}_v\).

\subsection{Wave Propagation and VAM Light Speed}

From (3) and (4), one can combine time derivatives and curls to show that \(\mathbf{E}_v\) and \(\mathbf{B}_v\) obey wave equations in vacuum-like regions (where \(\rho_v=0\) and \(\mathbf{J}_v=0\)):

\[
    \partial_t^2 \mathbf{E}_v
    \;-\;
    \frac{1}{\mu_v\,\varepsilon_v}\,
    \nabla^2 \mathbf{E}_v
    \;=\; 0,
\]
and similarly for \(\mathbf{B}_v\). This reveals a wave speed
\[
    v_{\mathrm{wave}}
    \;=\;
    \frac{1}{\sqrt{\mu_v\,\varepsilon_v}},
\]
analogous to \(c = 1/\sqrt{\mu_0\varepsilon_0}\) in standard electromagnetism but now interpreted as the propagation speed of vortex-mediated disturbances in the Æther.

\subsection{Physical Interpretation and Unifying Principles}

\begin{enumerate}
    \item \textbf{Irrotational vs. Solenoidal Components} \\
    The decomposition \(\mathbf{v} = \nabla \Phi_v + \nabla\times \mathbf{A}_v\) underpins the definitions of \(E_{v}^i\) and \(B_{v}^i\). Source-like \grqq charges\textquotedblright appear if vortex lines enter or exit a boundary; solenoidal loops remain closed, implying \(\partial_i B_{v}^i=0\).
    \item \textbf{Charge Conservation} \\
    Continuity equations in index form (not shown here) ensure \(\partial_t \rho_v + \partial_i J_{v}^i = 0\), meaning net vortex \grqq charge\textquotedblright is conserved in a closed system. This parallels electric charge conservation in standard Maxwell theory.
    \item \textbf{Comparisons with Standard Maxwell Equations} \\
    While the form of equations (1)--(4) closely mirrors Maxwell's, the VAM interpretation is purely fluidic: no four-dimensional spacetime curvature or external gauge fields are needed. Instead, all phenomena follow from the velocity field's topology and boundary conditions in 3D Euclidean geometry.
\end{enumerate}

\subsection{Concluding Remarks}

The index-based Maxwell--VAM equations confirm that, at the level of mathematical structure, VAM and classical electromagnetism share striking parallels:

\begin{enumerate}
    \item \textbf{Gauss's Laws} ensure source-like behavior for \(\mathbf{E}_v\) and the solenoidal nature of \(\mathbf{B}_v\).
    \item \textbf{Faraday's Law} and \textbf{Ampère--Maxwell} relations describe how time-varying vortex flows couple the two field components, giving rise to wave-like propagation.
    \item \textbf{VAM Coupling Constants} \(\mu_v, \varepsilon_v\) replace \(\mu_0, \varepsilon_0\) and set a wave speed \(\frac{1}{\sqrt{\mu_v\varepsilon_v}}\).
\end{enumerate}

Hence, the index form not only makes the theory precise for potential numerical implementation but also underscores how VAM's fluid-based approach recovers Maxwell's structure in a purely 3D, vorticity-driven setting.\label{appendix:3}
%! Author = Omar Iskandarani
%! Date = 3/13/2025


\section*{Appendix 4. Time-Dilation / \grqq Local Time\textquotedblright Equations with Exponential Corrections}

\subsection*{1. Physical Motivation}

In VAM, gravity is replaced by vortex-induced pressure gradients, and velocity fields near the vortex core can be significant. Analogous to how general relativity predicts local time dilation near massive bodies, VAM predicts local \grqq slowing\textquotedblright of clocks inside regions of fast swirl. Instead of appearing as geometric curvature in a 4D manifold, this effect arises from the energy cost (or fluid stress) that local vortex circulation imposes on physical processes.

To quantify the phenomenon, one introduces an \textit{adjusted time} \(t_\text{adjusted}\) measured by clocks in a high-swirl region, relative to a \grqq far-field\textquotedblright or \grqq global\textquotedblright time \(\Delta t\) measured far from the vortex.

\subsection*{2. The Core Equations: An Overview}

In many VAM treatments, the final results are given in two forms:

\begin{enumerate}
    \item A \textbf{comprehensive expression} that includes gravitational-like coupling \(G_\text{swirl} M_\text{effective}(r)\), the swirl velocity constant \(C_e\), a global rotation \(\Omega\), and an exponential factor \(e^{-r/r_c}\):
    \[
        t_\text{adjusted}
        \;=\;
        \Delta t \,\sqrt{
            1
            \;-\;
            \frac{2\,G_\text{swirl}\,M_\text{effective}(r)}{r\,c^2}
            \;-\;
            \frac{C_e^2}{c^2}\, e^{-\,r/r_c}
            \;-\;
            \frac{\Omega^2}{c^2}\, e^{-\,r/r_c}
        }.
    \]
    \item A \textbf{simplified local-time ratio} in scenarios where vortex swirl dominates over other terms:
    \[
        \frac{d\,t_\text{adjusted}}{d\,t}
        \;=\;
        \sqrt{
            1
            \;-\;
            \frac{C_e^2}{c^2}\, e^{-\,r/r_c}
        }.
    \]
\end{enumerate}

The exponential \(e^{-\,r/r_c}\) captures how the swirl (or rotation) is strong near the vortex core (small \(r\)) but decays at larger distances. The parameter \(r_c\) is the characteristic core radius beyond which vortex speed saturates or becomes negligible.

\subsection*{3. Starting Point: Vortex-Induced Energy Gradient}

\subsubsection*{3.1 Effective Potential and Local Clock Rate}

In VAM, the local flow of time is posited to depend on the fluid\rqs s energy distribution around a vortex. Specifically:
\[
    \Delta \tau(\mathbf{r})
    \;\approx\;
    \Delta t
    \;\sqrt{
        1
        \;-\;
        \frac{\Phi_{\mathrm{fluid}}(\mathbf{r})}{c^2}
    },
\]
where \(\Phi_{\mathrm{fluid}}\) plays the role of a local energy potential (analogous to gravitational potential in standard physics). If \(\Phi_{\mathrm{fluid}}\) is large and negative (due to swirl-induced pressure deficits), local clocks run slower relative to a reference observer at \(\Phi_{\mathrm{fluid}}=0\).

\subsubsection*{3.2 Including Exponential Swirl Terms}

The fluid swirl near radius \(r_c\) typically follows a form:
\[
    v_{\theta}(r)
    \;\sim\;
    C_e \,e^{-\,r/r_c},
\]
reflecting that tangential velocity saturates near the core and decays outward. One can show that such a velocity field modifies the local \grqq clock rate\textquotedblright by adding terms proportional to \(\tfrac{C_e^2}{c^2} e^{-\,r/r_c}\). A similar exponential term appears for an angular velocity \(\Omega\) if the entire structure has global rotation.

\subsection*{4. Derivation Outline}

\begin{enumerate}
    \item \textbf{Define the Local Lapse Function} \\
    In analogy to relativity, one can define a \grqq lapse\textquotedblright or rate function \(\alpha(r)\) that satisfies:
    \[
        d\tau
        \;=\;
        \alpha(r)\,dt.
    \]
    VAM sets \(\alpha(r)\approx \sqrt{1 - \text{(energy density / reference)}}\).
    \item \textbf{Contribution from Vortex Gravity} \\
    If there is an effective mass distribution \(M_\text{effective}(r)\) and swirl-based gravitational coupling \(G_\text{swirl}\), the standard gravitational potential near radius \(r\) contributes a term
    \[
        -\,\frac{2\,G_\text{swirl}\,M_\text{effective}(r)}{r\,c^2}
    \]
    inside the square root.
    \item \textbf{Swirl Velocity at the Core} \\
    For short-range swirl,
    \[
        v_{\theta}^2(r)
        \;\approx\;
        C_e^2\, e^{-\,r/r_c},
    \]
    modifies the local energy budget. By comparing this swirl energy to the total fluid energy baseline, one arrives at the factor
    \[
        -\,\frac{C_e^2}{c^2}\, e^{-\,r/r_c}.
    \]
    \item \textbf{Rotational Term} \\
    A global rotation \(\Omega\) near the vortex adds a further pressure deficit or frame-dragging–like effect:
    \[
        -\,\frac{\Omega^2}{c^2}\, e^{-\,r/r_c}.
    \]
    This captures the idea that rotating flows produce an additional local time adjustment, reminiscent of the Lense–Thirring effect in general relativity, but explained by fluid swirl here.
    \item \textbf{Combine Terms under the Square Root} \\
    Since these corrections are typically small, they appear as subtractions from unity inside the root. The final local time expression is:
    \[
        t_\text{adjusted}
        \;=\;
        \Delta t \,\sqrt{
            1
            \;-\;
            \frac{2\,G_\text{swirl}\,M_\text{effective}(r)}{r\,c^2}
            \;-\;
            \frac{C_e^2}{c^2}\, e^{-\,r/r_c}
            \;-\;
            \frac{\Omega^2}{c^2}\, e^{-\,r/r_c}
        }.
    \]
\end{enumerate}

\subsection*{5. Simplified Equation for Core-Dominated Time Flow}

When the swirl term \(C_e^2 e^{-r/r_c}\) is the primary correction, ignoring gravitational and rotation:
\[
    \frac{d\,t_\text{adjusted}}{d\,t}
    \;=\;
    \sqrt{
        1
        \;-\;
        \frac{C_e^2}{c^2}\, e^{-\,r/r_c}
    }.
\]
\textbf{Interpretation}: At large \(r\), \(e^{-r/r_c}\) is negligible; the local clock rate nearly matches the global rate. Near \(r\approx r_c\), the swirl velocity is maximal, producing the largest downward shift in time.

\subsection*{6. Physical Implications}

\begin{enumerate}
    \item \textbf{Near-Core \grqq Time-Warp\textquotedblright} \\
    The swirl velocity effectively slows down processes in the vortex interior, an alternative to relativistic time dilation. If \(C_e\approx 10^6\,\text{m/s}\) and \(r_c\approx 10^{-15}\,\text{m}\), time flow is significantly modified on nuclear scales, though such effects remain imperceptible at macroscopic distances.
    \item \textbf{Frame-Dragging Analogs} \\
    The \(\Omega^2 e^{-r/r_c}\) term parallels the dragging of inertial frames in rotating solutions of general relativity (e.g., Kerr black holes). In VAM, it arises from swirl vortex lines near the rotating core.
    \item \textbf{Matching Observational Data} \\
    \begin{itemize}
        \item \textbf{Atomic Clocks}: Subtle shifts in energy levels or clock rates in high-swirl environments (e.g., near rotating superfluid analogs) might test these predictions.
        \item \textbf{Compact Objects}: If black hole–like or neutron star–like objects are reinterpreted as extremely dense vortex cores, the same formula might guide how local time differs from a distant observer\rqs s measure.
    \end{itemize}
\end{enumerate}

\subsection*{7. Concluding Remarks}

The \textbf{time-dilation / local-time} equations in VAM repackage gravitational and rotational swirl effects into a single, three-dimensional fluid framework. Instead of four-dimensional spacetime curvature:
\begin{itemize}
    \item \(\tfrac{2 G_\text{swirl} M_\text{eff}}{r c^2}\) mimics Newtonian-style gravitational potential,
    \item \(\tfrac{C_e^2}{c^2} e^{-r/r_c}\) captures near-core swirl velocity,
    \item \(\tfrac{\Omega^2}{c^2} e^{-r/r_c}\) encodes global rotation\rqs s contribution.
\end{itemize}

This approach provides a conceptual and mathematical blueprint for investigating phenomena typically attributed to general relativity—like gravitational lensing or time dilation—purely in terms of vortex flows, fluid helicity, and pressure gradients in an absolute-time, 3D Euclidean medium.\label{appendix:4}
%! Author = Omar Iskandarani
%! Date = 3/13/2025


\section*{Appendix 5. The Relation \(\frac{\hbar^2}{2 M_e} = \frac{F_{\max} R_c^3}{5  \lambda_c  C_e} \)}

This relation links fundamental quantum mechanical parameters (\(\hbar, M_e\)) to VAM-specific constants (\(F_{\max}, R_c, \lambda_c, C_e\)). The key idea is that a characteristic quantum energy scale \(\tfrac{\hbar^2}{2M_e}\) can be matched to a vortex-based expression involving maximum force, core radius, and swirl velocity, illustrating VAM’s unification of quantum and vortex phenomena.

\subsection*{1. Overview of Symbols}

\begin{itemize}
    \item \(\hbar\): Reduced Planck’s constant, defining quantum scales.
    \item \(M_e\): Electron mass, though the same reasoning could apply to other fundamental masses in principle.
    \item \(F_{\max}\): The proposed maximum force in VAM (\(\approx 29\,\mathrm{N}\)), acting as an upper bound on force transmission in vortex cores.
    \item \(R_c\): The characteristic vortex-core radius (often \(\sim 10^{-15}\,\mathrm{m}\)), comparable to nuclear or Coulomb-barrier length scales.
    \item \(\lambda_c\): A Compton-like wavelength for the electron or the relevant particle (e.g., \(\lambda_c = \tfrac{h}{M_e c}\)), signifying the typical quantum “size” of wave-like effects.
    \item \(C_e\): The swirl velocity constant (\(\sim 10^6\,\mathrm{m/s}\)), derived from vortex quantization in VAM.
\end{itemize}

The left-hand side (LHS),
\[
    \frac{\hbar^2}{2\,M_e},
\]
often appears in quantum mechanical contexts as a characteristic measure of kinetic energy for an electron or the scaling for quantum bound states.

\subsection*{2. Physical Motivation}

\subsubsection*{2.1 Matching a Quantum Kinetic Term}
In non-relativistic quantum mechanics, \(\frac{\hbar^2}{2M_e}\) sets the characteristic energy scale for phenomena such as the Bohr model’s ground-state energy (up to multiplicative constants), the Rydberg constant, and other discrete-level calculations. It represents roughly the minimal “quantum kinetic” energy or the scale at which wave-like properties dominate the electron’s behavior.

\subsubsection*{2.2 VAM’s Vortex-Energy Expression}
In the Vortex Æther Model, stable vortex structures have an internal energy governed by tension-like forces, plus the swirling velocity distribution. When boundary conditions (like \(r_c\), \(\lambda_c\), and the maximum force \(F_{\max}\)) are imposed, one obtains a formula for the characteristic energy or momentum cost of confining the vortex core to radius \(R_c\).

\subsection*{3. The Derivation in Steps}

\begin{enumerate}
    \item \textbf{Maximum Force and Core Volume} \\
    VAM posits that the strongest force permissible within a region of size \(R_c\) is \(F_{\max}\). Over distances of the order \(R_c\), the total “energy toll” might be approximated by \(F_{\max} \times R_c\). However, since we’re dealing with a three-dimensional structure, corrections involving \(R_c^3\) come into play, typically capturing volumetric or geometric constraints.
    \item \textbf{Compton Wavelength Factor} \\
    In quantum mechanics, \(\lambda_c\) sets the typical scale where particle wave effects become crucial. If the vortex is confined further to scale \(R_c\), the ratio \(\tfrac{R_c}{\lambda_c}\) indicates how much smaller (or bigger) the vortex core is relative to the particle’s natural quantum “size.”
    \item \textbf{Dimensionless Geometry Factor} \\
    The coefficient \(\frac{1}{5}\) can emerge from integrating the potential or velocity distribution across a spherical or toroidal region, or from analyzing the dimensionless combination:
    \[
        \frac{F_{\max} R_c^3}{\lambda_c C_e}
    \]
    in a geometry-specific integral. Precise fluid-dynamics or topological arguments determine the factor 5 (analogous to how certain integrals in the Bohr model yield \(\tfrac12\), \(\tfrac14\), or other dimensionless numbers).
\end{enumerate}

Putting these elements together, one obtains:

\[
    \frac{\hbar^2}{2M_e}
    \;\sim\;
    \frac{F_{\max} \,R_c^3}{5 \,\lambda_c \,C_e}.
\]

\subsection*{4. Dimensional Analysis}

\subsubsection*{4.1 Left-Hand Side}
\(\hbar^2/(2 M_e)\) has dimensions of energy:
\[
    [\hbar^2/(2 M_e)]
    \;=\;
    \frac{(\mathrm{J\cdot s})^2}{\mathrm{kg}}
    \;\;\rightarrow\;\;
    \mathrm{J} \;\;(\mathrm{kg\,m^2/s^2}).
\]

\subsubsection*{4.2 Right-Hand Side}
\begin{enumerate}
    \item \(F_{\max}\): \(\mathrm{N}\) = \(\mathrm{kg\,m/s^2}\).
    \item \(R_c^3\): \(\mathrm{m^3}\).
    \item \(\lambda_c\): \(\mathrm{m}\).
    \item \(C_e\): \(\mathrm{m/s}\).
\end{enumerate}

So:
\[
    \frac{F_{\max} R_c^3}{\lambda_c\,C_e}
    \;=\;
    \frac{\mathrm{kg\,m/s^2}\,\times\,\mathrm{m^3}}{\mathrm{m}\,\times\,\mathrm{m/s}}
    \;=\;
    \mathrm{kg}\,\frac{\mathrm{m^2}}{\mathrm{s^2}}
    \;=\;
    \mathrm{J},
\]
an energy dimension. Multiplying by the factor \(\tfrac{1}{5}\) remains dimensionless, so the entire RHS is in joules, matching the LHS.

\subsection*{5. Interpretational Notes}

\begin{enumerate}
    \item \textbf{Quantum–Vortex Bridge} \\
    This equation effectively sets the scale at which quantum kinetic energy meets vortex tension or confinement energy. It is reminiscent of how in the Bohr model, balancing centripetal force with electrostatic force yields discrete orbits; here, one is balancing quantum scales with fluidic swirl constraints.
    \item \textbf{Role of \(F_{\max}\)} \\
    Interpreting \(F_{\max}\approx 29\,\mathrm{N}\) as a universal upper force constant is controversial but fundamental in some versions of VAM. This relation uses that concept to link the quantum domain (\(\hbar\)) with a distinct fluidic limit.
    \item \textbf{Predictive Capability} \\
    If one treats \(R_c\), \(\lambda_c\), and \(C_e\) as measured or derived from other parts of the model (e.g., swirl velocity derivation, Compton-like lengths, typical nuclear scales), then the above formula becomes a check on consistency. Any discrepancy might indicate either a missing topological factor or a different boundary condition for the vortex core.
\end{enumerate}

\subsection*{6. Concluding Remarks}

By equating a fundamental quantum kinetic energy scale \(\tfrac{\hbar^2}{2M_e}\) to a fluidic expression involving \(\bigl(F_{\max}, R_c, \lambda_c, C_e\bigr)\), the Vortex Æther Model underscores its thesis that quantum parameters and superfluid vortex parameters are not separate realms but facets of the same fluid-based picture:

\[
    \boxed{
        \frac{\hbar^2}{2 M_e}
        \;=\;
        \frac{F_{\max} \, R_c^3}{5 \,\lambda_c\,C_e}.
    }
\]

This neat match highlights how stable vortex structures in the Æther might be subject to quantization and force constraints that mirror those found in conventional quantum mechanics. While additional factors (e.g., geometric integrals, topological constraints) may refine or shift the coefficient \(1/5\), the core takeaway is that \textit{quantum-scale energies can emerge from purely fluid-dynamic constraints} in VAM.\label{appendix:5}
%! Author = Omar Iskandarani
%! Date = 3/13/2025


\section*{6. Detailed Knot Theory Connections (If Heavily Mathematical)}

\subsection*{1. Rationale for Knot-Theoretic Treatment}

\begin{enumerate}
    \item \textbf{Historical Precedent} \\
    Helmholtz and Lord Kelvin proposed that atomic structure could be understood as vortex rings or knots in an inviscid fluid. VAM extends this notion to all fundamental particles, hypothesizing that stable or metastable particles correspond to distinct knot configurations.
    \item \textbf{Helicity and Conservation} \\
    VAM relies on the conservation of fluid helicity:
    \[
        \mathcal{H}
        \;=\;
        \int
        \boldsymbol{\omega}\,\cdot\,\mathbf{v}\;\mathrm{d}V,
    \]
    for an inviscid fluid. In knot-theory language, \(\mathcal{H}\) is related to the \textit{linking} and \textit{writhe} of vortex filaments.
    \item \textbf{Particle Identities via Topology} \\
    Particle-like properties (charge, spin, baryon number) could be mapped to topological invariants (e.g., linking number, knot polynomials). A trefoil vortex might, for instance, represent a minimal “stable” topology, while more complicated links correspond to higher-generation or composite particles.
\end{enumerate}

\subsection*{2. Mathematical Foundations}

\subsubsection*{2.1 Linking Number and Knot Polynomials}

\begin{itemize}
    \item \textbf{Linking Number \(Lk(\Gamma_1, \Gamma_2)\)}: \\
    For two closed curves (vortex filaments) \(\Gamma_1\) and \(\Gamma_2\) in 3D, the linking number measures how many times they wrap around each other. In fluid terms, nonzero linking can reflect topological coupling or “bound states.”
    \item \textbf{Knot Polynomials (Jones, Alexander, HOMFLY)}: \\
    These polynomials classify knots and links beyond simple linking number. In VAM, they help distinguish different stable or quasi-stable vortex knots that might correspond to different quantum states or particle “types.”
    \[
        \text{(Example)}:\quad
        \text{Trefoil knot}\;\longrightarrow\;\text{nontrivial Jones polynomial.}
    \]
\end{itemize}

\subsubsection*{2.2 Reidemeister Moves and Vortex Reconnection}

\begin{itemize}
    \item \textbf{Reidemeister Moves}: \\
    In knot theory, these local transformations alter how a knot is drawn but not its fundamental topology. In fluid mechanics, \textit{vortex reconnection} can sometimes analogously break or join vortex filaments. If the fluid is truly inviscid and the vortex filaments never intersect, the “knot type” remains preserved—i.e. stable topological quantum states.
    \item \textbf{Suppression of Reconnection}: \\
    VAM assumes that stable elementary particles rarely undergo reconnection transitions, mirroring the observed stability of protons or electrons. Under high-energy conditions, partial reconnections might occur, paralleling processes akin to particle decay or scattering.
\end{itemize}

\subsection*{3. Helicity as a Topological Measure of “Charge”}

\subsubsection*{3.1 Helicity Integral}

\[
    \mathcal{H}
    \;=\;
    \int_{\Omega}
    \boldsymbol{\omega}\,\cdot\,\mathbf{v}\;\mathrm{d}V,
\]
where \(\boldsymbol{\omega} = \nabla \times \mathbf{v}\). This integral is invariant in an ideal fluid, analogous to how electric charge or baryon number is conserved in particle physics.

\subsubsection*{3.2 Linking Number Relation}

Under certain simplifying assumptions (e.g. disjoint vortex tubes with localized cross-sections), fluid helicity can be related to the sum of linking numbers of vortex loops:
\[
    \mathcal{H}
    \;\approx\;
    \kappa
    \sum_{\alpha,\beta} Lk(\Gamma_\alpha, \Gamma_\beta),
\]
where \(\kappa\) is the circulation quantum (\(\approx h/m\)) in a superfluid. Thus, each pair of linked vortex filaments contributes a discrete topological “charge,” reminiscent of how quarks in QCD carry color or how fundamental charges add in QED.

\subsection*{4. Potential Particle Mapping}

\begin{enumerate}
    \item \textbf{Single-Knot States}:
    \begin{itemize}
        \item \textbf{Electron-Like}: A trefoil or figure-eight knot vortex with one quantum of circulation, giving charge \(\pm e\) if oriented or anti-oriented.
        \item \textbf{Neutrino-Like}: Possibly a simpler (unknotted) but twisted filament, carrying minimal or zero net linking with other loops.
    \end{itemize}
    \item \textbf{Multi-Knot States}:
    \begin{itemize}
        \item \textbf{Proton-Like}: Could be a compound link of three twisted sub-loops (“three quarks” motif), each sub-loop carrying fractional circulation. Their total linking yields net “+1” charge.
        \item \textbf{Meson-Like}: A link of two oppositely oriented vortex filaments that can separate or annihilate each other under reconnection analogies—akin to quark-antiquark pairs.
    \end{itemize}
    \item \textbf{Decay Channels}: \\
    Changes in topological invariants might mimic particle decays; strong or electromagnetic interactions might correspond to partial reconnections under specific energy thresholds. The near-invisibility of processes like proton decay implies either extremely high topological barriers or near-perfect helicity conservation in the vortex fluid.
\end{enumerate}

\subsection*{5. Open Questions and Theoretical Extensions}

\begin{enumerate}
    \item \textbf{Non-Abelian Structures} \\
    QCD’s non-Abelian gauge group (SU(3)) might require more complicated knot invariants, or a tangle of vortex tubes representing color confinement. Current knot polynomials may not fully capture non-Abelian “holonomies” in fluid flow, leaving a gap between VAM and the full Standard Model.
    \item \textbf{Multi-Loop Entanglement} \\
    Real-world baryons or nuclei might correspond to highly entangled vortex webs. Determining their stable topological classes could be extremely challenging mathematically but offers a route to unify nuclear physics and fluid dynamics in a single, 3D Euclidean framework.
    \item \textbf{Exact Correspondences} \\
    Detailed mappings from knot polynomials to quantum numbers (e.g. electric charge, spin, isospin) remain largely speculative. Progress in topological quantum field theory might illuminate how to treat certain polynomial invariants as direct analogs of gauge-group representations.
\end{enumerate}

\subsection*{6. Conclusion and Outlook}

Knot theory provides a powerful lens through which VAM interprets stable or metastable vortex states as “particles.” By tying helicity conservation and linking numbers to quantum numbers—charge, baryon number, and spin—VAM aspires to a topological unification of fluid mechanics and particle physics. Although many challenges remain (particularly regarding the full SU(3) or non-Abelian gauge structure of the Standard Model), the mathematical framework of knot invariants and vortex reconnection offers a fresh perspective on why certain particles exist, why they are stable, and how quantum phenomena might ultimately be the manifestation of tangled yet robust vortex flows in an inviscid Æther.\label{appendix:6}
%! Author = Omar Iskandarani
%! Date = 3/13/2025


\section{Derivation of the Fine-Structure Constant from Vortex Mechanics}
\label{sec:appendix-alpha}

In this section, we derive the fine-structure constant $\alpha$ in the Vortex Æther Model (VAM) by considering the fundamental properties of circulation in an inviscid superfluid medium.

\subsection{Quantization of Circulation}

Circulation $\Gamma$ around a closed contour enclosing a vortex core is quantized in units of $h/m_e$, where $h$ is Planck’s constant and $m_e$ is the electron mass:

\begin{equation}
    \Gamma = \oint \mathbf{v} \cdot d\mathbf{l} = \frac{h}{m_e}.
\end{equation}

For a stable vortex core with radius $r_c$ and tangential velocity $C_e$,

\begin{equation}
    \Gamma = 2 \pi r_c C_e.
\end{equation}

Equating these expressions,

\begin{equation}
    2 \pi r_c C_e = \frac{h}{m_e},
\end{equation}

solving for $C_e$,

\begin{equation}
    C_e = \frac{h}{2 \pi m_e r_c}.
\end{equation}

\subsection{Relation to the Speed of Light}

The vortex-core radius $r_c$ is approximately half the classical electron radius $R_e$:

\begin{equation}
    r_c = \frac{R_e}{2}.
\end{equation}

Substituting this into the equation for $C_e$,

\begin{equation}
    C_e = \frac{h}{2 \pi m_e \left(\frac{R_e}{2}\right)} = \frac{h}{\pi m_e R_e}.
\end{equation}

The classical electron radius is given by:

\begin{equation}
    R_e = \frac{e^2}{4 \pi \varepsilon_0 m_e c^2}.
\end{equation}

Substituting for $R_e$ in our equation for $C_e$:

\begin{equation}
    C_e = \frac{h}{\pi m_e} \times \frac{4 \pi \varepsilon_0 m_e c^2}{e^2}.
\end{equation}

Simplifying,

\begin{equation}
    C_e = \frac{4 \varepsilon_0 h c^2}{e^2}.
\end{equation}

The fine-structure constant is defined as:

\begin{equation}
    \alpha = \frac{e^2}{4 \pi \varepsilon_0 \hbar c}.
\end{equation}

Rearranging for $C_e$,

\begin{equation}
    \alpha = \frac{2 C_e}{c}.
\end{equation}

Thus, the fine-structure constant emerges directly from vortex dynamics, demonstrating that its value is not arbitrary but deeply tied to fundamental vortex motion in the Æther. This reinforces the idea that electromagnetism and quantum mechanics originate from structured vorticity interactions.\label{appendix:7}
%! Author = Omar Iskandarani
%! Date = 3/13/2025


\section{Derivation of the Refined Ætheric Wave Equation}


To ensure consistency within the Vortex Æther Model (VAM), we derive a generalized Ætheric wave equation, explicitly incorporating finite propagation speed cc, which corresponds to the speed of light in vacuum, along with realistic boundary conditions and topological constraints.


\subsection{Starting Point: Euler's Vorticity Equation}


The incompressible Euler equation governing the vorticity field ω\boldsymbol{\omega} is given by:
\begin{equation}
    \frac{\partial \boldsymbol{\omega}}{\partial t} + (\mathbf{v}\cdot\nabla)\boldsymbol{\omega} = (\boldsymbol{\omega}\cdot\nabla)\mathbf{v},
\end{equation}


where v\mathbf{v} is the velocity field and ω=∇×v\boldsymbol{\omega} = \nabla \times \mathbf{v} is the vorticity.


\subsection{Perturbation Formulation}


Consider small perturbations around equilibrium fields (v0,ω0)(\mathbf{v}_0, \boldsymbol{\omega}_0):
\begin{align}
    \mathbf{v} &= \mathbf{v}_0 + \mathbf{v}',\
    \boldsymbol{\omega} &= \boldsymbol{\omega}_0 + \boldsymbol{\omega}', \quad |\mathbf{v}'| \ll |\mathbf{v}_0|, \quad |\boldsymbol{\omega}'| \ll |\boldsymbol{\omega}_0|.
\end{align}


Linearizing, we have:
\begin{equation}
    \frac{\partial \boldsymbol{\omega}'}{\partial t} + (\mathbf{v}_0 \cdot \nabla)\boldsymbol{\omega}' + (\mathbf{v}' \cdot \nabla)\boldsymbol{\omega}_0 - (\boldsymbol{\omega}_0\cdot\nabla)\mathbf{v}' - (\boldsymbol{\omega}'\cdot\nabla)\mathbf{v}_0 = 0.
\end{equation}


\subsection{Introducing Vortex Elasticity}


By analogy to electromagnetism, we introduce Ætheric "vortex elasticity," characterized by parameters analogous to vacuum permittivity εv\varepsilon_v and permeability μv\mu_v. We use vector potentials A′\mathbf{A}' defined by:
\begin{equation}
    \mathbf{v}' = \nabla \times \mathbf{A}', \quad \mathbf{A}' = -\nabla^{-2}\boldsymbol{\omega}'.
\end{equation}


\subsection{Derivation of the Generalized Wave Equation}


Applying a time derivative, and simplifying under the assumption of slowly varying equilibrium fields, we derive:
\begin{equation}
    \frac{\partial^2 \boldsymbol{\omega}'}{\partial t^2}+2(\mathbf{v}_0\cdot\nabla)\frac{\partial\boldsymbol{\omega}'}{\partial t} - \frac{1}{\mu_v \varepsilon_v}\nabla^2 \boldsymbol{\omega}' = 0.
\end{equation}


\subsection{Propagation Speed and Speed of Light}


This derived wave equation is structurally identical to classical wave equations encountered in electromagnetism, where wave velocity is related to intrinsic medium properties. By analogy with electromagnetism, we identify the vortex-wave velocity vwavev_{\text{wave}} as:
\begin{equation}
    v_{\text{wave}} = \frac{1}{\sqrt{\mu_v \varepsilon_v}}.
\end{equation}


\subsection{Explicit Identification with Speed of Light cc}


By explicitly setting:
\begin{equation}
    \mu_v \equiv \mu_0, \quad \varepsilon_v \equiv \varepsilon_0,
\end{equation}


where μ0\mu_0 and ε0\varepsilon_0 are the vacuum permeability and permittivity, respectively, the propagation speed becomes exactly the speed of light cc:
\begin{equation}
    v_{\text{wave}} = \frac{1}{\sqrt{\mu_0 \varepsilon_0}} = c.
\end{equation}


\subsection{Helicity and Boundary Conditions}


Ensuring stability and topological consistency of vortex knots requires helicity conservation:
\begin{equation}
    \frac{dH}{dt} = \int_{\partial V} \mathbf{J}_H \cdot d\mathbf{A} = 0.
\end{equation}


Realistic physical boundary conditions are set as:
\begin{equation}
    \mathbf{v}' \cdot \hat{\mathbf{n}} = 0, \quad P'|\textit{{S}\text{knot}} = 0.
\end{equation}


\subsection{Final Form of the Refined Ætheric Wave Equation}


The fully refined and generalized wave equation for the Vortex Æther Model thus takes the form:


\begin{equation}
    \frac{\partial^2 \boldsymbol{\omega}'}{\partial t^2}+2(\mathbf{v}_0\cdot\nabla)\frac{\partial\boldsymbol{\omega}'}{\partial t} - c^2\nabla^2\boldsymbol{\omega}' = 0,
\end{equation}


with the explicit finite propagation velocity given by:


\begin{equation}
    v_{\text{wave}} = c = \frac{1}{\sqrt{\mu_v \varepsilon_v}}.
\end{equation}


This result confirms that vortex perturbations in the Æther inherently propagate at the speed of light, maintaining complete consistency with known electromagnetic theory and observed physical phenomena.\label{appendix:8}
%! Author = mr
%! Date = 3/13/2025


\section{Derivation of the Vortex Swelling Formula}

In the Vortex Æther Model (VAM), vortex structures expand as a function of temperature. We define the probability distribution function of a vortex as:

\begin{align}
    \psi(r) = A e^{-r/a_0},
\end{align}

where \( a_0 \) is the characteristic vortex radius at reference temperature \( T_0 \).

\subsection{Temperature Dependence}
The characteristic radius \( a(T) \) scales with temperature as:

\begin{align}
    a(T) = a_0 \left(\frac{T}{T_0}\right)^{5/9}.
\end{align}

The modified wavefunction becomes:
\begin{align}
    \psi_P(r) = A e^{-r/a(T)} = A e^{-r/(a_0 (T/T_0)^{5/9})}.
\end{align}

Defining the relative swelling function:
\begin{align}
    \frac{\psi_P(r)}{\psi(r)} = e^{r/a_0} e^{-r/(a_0 (T/T_0)^{5/9})}.
\end{align}

\subsection{Physical Interpretation}
This equation shows that vortex structures in the Æther **expand with increasing temperature**, influencing quantum energy levels and blackbody radiation emission in VAM.
\label{appendix:9}
%! Author = mr
%! Date = 3/13/2025

\section{Derivation of the Wave Equation}

In the Vortex Æther Model (VAM), electromagnetic waves emerge from structured vortex interactions in the Æther. Analogous to Maxwell’s equations, we define:
\begin{align}
    \mathbf{E}_v &= -\nabla \Phi_v, \quad \mathbf{B}_v = \nabla \times \mathbf{A}_v.
\end{align}

Using the vorticity-based Ampère-Maxwell law:
\begin{align}
    \nabla \times \mathbf{B}_v = \mu_v \mathbf{J}_v + \mu_v \varepsilon_v \frac{\partial \mathbf{E}_v}{\partial t}.
\end{align}

Taking the curl on both sides:
\begin{align}
    \nabla \times (\nabla \times \mathbf{B}_v) = \nabla \times \left(\mu_v \varepsilon_v \frac{\partial \mathbf{E}_v}{\partial t} \right).
\end{align}

Using the vector identity \( \nabla \times (\nabla \times \mathbf{A}) = \nabla (\nabla \cdot \mathbf{A}) - \nabla^2 \mathbf{A} \), and assuming \(\nabla \cdot \mathbf{B}_v = 0\), we obtain:
\begin{align}
    \nabla^2 \mathbf{B}_v = \mu_v \varepsilon_v \frac{\partial^2 \mathbf{B}_v}{\partial t^2}.
\end{align}

Defining the wave speed as:
\begin{align}
    v_{\text{wave}} = \frac{1}{\sqrt{\mu_v \varepsilon_v}},
\end{align}
we arrive at the wave equation:
\begin{align}
    \nabla^2 \mathbf{B}_v - \frac{1}{v_{\text{wave}}^2} \frac{\partial^2 \mathbf{B}_v}{\partial t^2} = 0.
\end{align}

This confirms that electromagnetic-like waves in VAM propagate through the Æther at a finite velocity \( v_{\text{wave}} \), analogous to the speed of light in classical electromagnetism.
\label{appendix:10}

\end{document}

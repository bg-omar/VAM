
%! Author = Omar Iskandarani
%! Date = 2/15/2025

\section{Photon as a Vortex Dipole and Its Electrodynamic Implications}

\subsection{Photon as a Vortex-Antivortex Pair}
The Vortex \AE ther Model (VAM) proposes that photons are not point-like particles but rather localized vortex dipole structures within the \AE ther. Each photon consists of a vortex-antivortex pair, propagating as a stable rolling vortex structure. This formulation naturally explains:
\begin{itemize}
    \item The wave-particle duality of photons via structured vorticity.
    \item The polarization of light as a topological property of vortex helicity.
    \item The propagation of electromagnetic waves as collective vortex excitations in the \AE ther.
\end{itemize}

Mathematically, the photon vortex dipole can be described as:
\begin{equation}
    \boldsymbol{\omega}_{\gamma} = \nabla \times \mathbf{v}_{\gamma},
\end{equation}
where $\mathbf{v}_{\gamma}$ is the velocity field of the photon vortex pair, and $\boldsymbol{\omega}_{\gamma}$ represents the local vorticity. The circulation of each vortex component satisfies:
\begin{equation}
    \Gamma = \oint_C \mathbf{v}_{\gamma} \cdot d\mathbf{l} = \frac{h}{m_e},
\end{equation}
ensuring that photon angular momentum remains quantized.

\subsection{Electromagnetic Wave Analogy}
Instead of treating light as a transverse oscillation of an abstract field, VAM proposes that electromagnetic waves are structured vortex perturbations in the \AE ther. The Maxwell equations emerge naturally from the vorticity transport equations:
\begin{align}
    \nabla \cdot \mathbf{E}_v &= \frac{\rho_v}{\varepsilon_v}, \\
    \nabla \cdot \mathbf{B}_v &= 0, \\
    \nabla \times \mathbf{E}_v &= - \frac{\partial \mathbf{B}_v}{\partial t}, \\
    \nabla \times \mathbf{B}_v &= \mu_v \mathbf{J}_v + \mu_v \varepsilon_v \frac{\partial \mathbf{E}_v}{\partial t}.
\end{align}
Here, $\mathbf{E}_v$ and $\mathbf{B}_v$ correspond to the vorticity-induced analogs of electric and magnetic fields, respectively.

\subsection{Gravitational Bending of Light via Vorticity Interactions}
Conventionally, gravitational lensing is attributed to mass-induced spacetime curvature. VAM instead describes light bending as a consequence of vortex-field interactions. The deflection angle $\theta$ can be estimated using:
\begin{equation}
    \theta = \frac{\Gamma}{c r_v},
\end{equation}
where $r_v$ is the characteristic vortex interaction radius. This suggests that strong vorticity gradients in astrophysical structures can curve photon trajectories without requiring spacetime curvature.

\subsection{Experimental Predictions and Tests}
To validate this model, we propose:
\begin{itemize}
    \item **Superfluid Analogs:** Create and observe vortex-antivortex photon-like structures in superfluid helium.
    \item **Astrophysical Tests:** Look for deviations from standard gravitational lensing in high-vorticity plasma environments.
    \item **Polarization-Vorticity Coupling:** Test if photon polarization changes in regions of intense vorticity, beyond classical Faraday rotation predictions.
\end{itemize}

These experiments can distinguish between the VAM interpretation and standard field-theoretic models, offering a pathway to experimentally verify the vortex nature of light.

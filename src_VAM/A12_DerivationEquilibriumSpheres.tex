%! Author = mr
%! Date = 3/14/2025


\textbf{To derive the equilibrium spheres equation:

ΔU=Q−W=Δ(12ρ∫v2 dV+∫P dV)\Delta U = Q - W = \Delta \left( \frac{1}{2} \rho \int v^2 \, dV + \int P \, dV \right)ΔU=Q−W=Δ(21​ρ∫v2dV+∫PdV)

using fundamental principles and incorporating constants such as CeC_eCe​ (energy-related constant), RcR_cRc​ (characteristic radius), and Fmax⁡F_{\max}Fmax​ (maximum force), we will proceed step by step.





\subsection*{Step 1: First Principles of Thermodynamics}
From the First Law of Thermodynamics, we know:

dU=δQ−δWdU = \delta Q - \delta WdU=δQ−δW

where:

\begin{itemize}
\item dUdUdU is the change in internal energy,


\item δQ\delta QδQ is the heat input,


\item δW\delta WδW is the work done by the system.


\end{itemize}
For a fluid system with vortex structures and surrounding equilibrium spheres, we must account for the kinetic energy due to fluid velocity vvv and potential energy due to pressure PPP.

Thus, we write the total internal energy as:

U=12ρ∫v2 dV+∫P dVU = \frac{1}{2} \rho \int v^2 \, dV + \int P \, dVU=21​ρ∫v2dV+∫PdV



\subsection*{Step 2: Energy Balance with Spherical Equilibrium Constraints}
Now, consider the spheres of equilibrium pressure that surround a knotted vortex. The pressure distribution is influenced by:

\begin{enumerate}
\item The characteristic radius RcR_cRc​, which defines the boundary of these equilibrium spheres.


\item The maximum force Fmax⁡F_{\max}Fmax​, which sets a limit to the pressure interactions within the system.


\item The energy constant CeC_eCe​, which is a fundamental proportionality factor for energy distribution.


\end{enumerate}
\subsubsection*{Velocity Integral Contribution}
For a fluid with density ρ\rhoρ, the kinetic energy contribution within the equilibrium sphere of radius RcR_cRc​ is:

Ukin=12ρ∫v2 dVU_{\text{kin}} = \frac{1}{2} \rho \int v^2 \, dVUkin​=21​ρ∫v2dV

Using the characteristic velocity scale veqv_{\text{eq}}veq​ at equilibrium:

veq2=Fmax⁡ρRcv_{\text{eq}}^2 = \frac{F_{\max}}{\rho R_c}veq2​=ρRc​Fmax​​

we approximate:

Ukin≈12ρveq2⋅VeqU_{\text{kin}} \approx \frac{1}{2} \rho v_{\text{eq}}^2 \cdot V_{\text{eq}}Ukin​≈21​ρveq2​⋅Veq​

where VeqV_{\text{eq}}Veq​ is the volume of the equilibrium sphere:

Veq=43πRc3V_{\text{eq}} = \frac{4}{3} \pi R_c^3Veq​=34​πRc3​

Thus:

Ukin=12ρFmax⁡ρRc⋅43πRc3U_{\text{kin}} = \frac{1}{2} \rho \frac{F_{\max}}{\rho R_c} \cdot \frac{4}{3} \pi R_c^3Ukin​=21​ρρRc​Fmax​​⋅34​πRc3​ Ukin=23πFmax⁡Rc2U_{\text{kin}} = \frac{2}{3} \pi F_{\max} R_c^2Ukin​=32​πFmax​Rc2​



\subsubsection*{Pressure Integral Contribution}
The potential energy contribution due to pressure inside the equilibrium sphere:

Upress=∫P dVU_{\text{press}} = \int P \, dVUpress​=∫PdV

Using an equilibrium pressure scaling:

Peq=Fmax⁡Aeq=Fmax⁡4πRc2P_{\text{eq}} = \frac{F_{\max}}{A_{\text{eq}}} = \frac{F_{\max}}{4 \pi R_c^2}Peq​=Aeq​Fmax​​=4πRc2​Fmax​​

where Aeq=4πRc2A_{\text{eq}} = 4 \pi R_c^2Aeq​=4πRc2​ is the surface area of the equilibrium sphere.

Thus, the pressure integral:

Upress=PeqVeqU_{\text{press}} = P_{\text{eq}} V_{\text{eq}}Upress​=Peq​Veq​ Upress=Fmax⁡4πRc2⋅43πRc3U_{\text{press}} = \frac{F_{\max}}{4 \pi R_c^2} \cdot \frac{4}{3} \pi R_c^3Upress​=4πRc2​Fmax​​⋅34​πRc3​ Upress=13Fmax⁡RcU_{\text{press}} = \frac{1}{3} F_{\max} R_cUpress​=31​Fmax​Rc​



\subsection*{Step 3: Final Energy Balance Equation}
Summing both contributions:

U=Ukin+UpressU = U_{\text{kin}} + U_{\text{press}}U=Ukin​+Upress​ U=23πFmax⁡Rc2+13Fmax⁡RcU = \frac{2}{3} \pi F_{\max} R_c^2 + \frac{1}{3} F_{\max} R_cU=32​πFmax​Rc2​+31​Fmax​Rc​

Now, considering a differential energy change due to heat QQQ and work WWW:

ΔU=Q−W\Delta U = Q - WΔU=Q−W Δ(23πFmax⁡Rc2+13Fmax⁡Rc)=Q−W\Delta \left( \frac{2}{3} \pi F_{\max} R_c^2 + \frac{1}{3} F_{\max} R_c \right) = Q - WΔ(32​πFmax​Rc2​+31​Fmax​Rc​)=Q−W

To include an energy constant CeC_eCe​, we set:

Ce=Fmax⁡RcC_e = \frac{F_{\max}}{R_c}Ce​=Rc​Fmax​​

which normalizes force over a characteristic length, giving:

Δ(23πCeRc3+13CeRc2)=Q−W\Delta \left( \frac{2}{3} \pi C_e R_c^3 + \frac{1}{3} C_e R_c^2 \right) = Q - WΔ(32​πCe​Rc3​+31​Ce​Rc2​)=Q−W

Thus, the generalized energy equation for the equilibrium spheres is:

ΔU=Q−W=Δ(23πCeRc3+13CeRc2)\Delta U = Q - W = \Delta \left( \frac{2}{3} \pi C_e R_c^3 + \frac{1}{3} C_e R_c^2 \right)ΔU=Q−W=Δ(32​πCe​Rc3​+31​Ce​Rc2​)



\subsection*{Conclusion}
\begin{itemize}
\item This formulation links thermodynamic energy change to the equilibrium pressure spheres surrounding vortex knots.


\item The equation shows that heat addition or work done on the system affects the energy stored in the knotted vortex system, via pressure-volume interactions.


\item The constants Fmax⁡F_{\max}Fmax​, RcR_cRc​, and CeC_eCe​ determine how these energy exchanges occur within the topological fluid structure.


\end{itemize}

}

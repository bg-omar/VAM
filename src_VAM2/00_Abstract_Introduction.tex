%! Author = mr
%! Date = 4/2/2025


\begin{abstract}
This paper introduces a fluid-dynamic reformulation of general relativity through the Vortex \AE ther Model (VAM), a framework in which gravitation and time dilation arise not from spacetime curvature but from vorticity-induced pressure gradients in an effectively incompressible, inviscid superfluid medium. Within this three-dimensional, Euclidean, and temporally absolute æther, spacetime dynamics are encoded in conserved vortex structures: velocity fields, circulation constants, and topologically stable knots.
We derive analogs to Schwarzschild and Kerr metrics using the energetic and inertial properties of quantized vortex filaments, replacing geodesic motion with streamlines along conserved vorticity flux. Gravitational attraction emerges from Bernoulli-like pressure deficits generated by intense swirl regions, establishing a direct mechanical analog to gravitational potential.
Thermodynamic consistency is ensured by embedding Clausius entropy within vortex knot structures, providing an entropic interpretation of mass and time dilation. Notably, we reinterpret quantum effects such as the photoelectric effect and low-energy nuclear reactions (LENR) as resonant transitions within confined vortex networks.
This approach extends analogue gravity programs initiated by Barceló, Visser, and Volovik~\cite{barcelo2011analogue,volovik2009universe}, but introduces a topologically conserved vorticity formalism that unifies kinematic, thermodynamic, and gravitational phenomena. VAM offers a post-relativistic, energetically grounded, and experimentally accessible candidate for emergent gravity in structured continua.
\end{abstract}

\maketitle

\subsection*{The Æther Revisited: From Historical Substrate to Vorticity Field}
The term \textit{Æther} traditionally denoted an all-pervading medium facilitating wave propagation, from antiquity through 19th-century theories of luminiferous æther. By the late 1800s, Kelvin and Tait proposed vortex atom models, anticipating matter as stable, knotted structures in an ideal fluid. However, with the rise of Einstein’s relativity and the null results of the Michelson--Morley experiment, the æther concept fell into disuse---replaced by the curvature of spacetime.
Yet modern physics has quietly revived the idea in new guises: Dirac envisioned a quantum vacuum substrate; quantum field theory predicts a nontrivial ground state; and analogue gravity models---such as those of Volovik and Barceló---invoke superfluid or condensate-based media to model relativistic effects.
The Vortex Æther Model (VAM) reclaims the æther not as a particulate medium but as a topologically structured, inviscid superfluid. Here, the fundamental actors are vorticity fields: $\vec{\omega} = \nabla \times \vec{v}$ which encode rotational dynamics and replace both spacetime curvature and mass-energy density as mediators of force.
In VAM, particles correspond to topologically conserved vortex knots (e.g., trefoils, Hopf links), while gravity and inertia emerge from pressure gradients generated by swirling flow. This reframes mass as rotational energy, time dilation as a Bernoulli-pressure effect, and gravitational attraction as a vorticity-induced pressure well---providing a unified fluid-mechanical ontology for relativistic and quantum behavior.

\section*{Postulates of the Vortex \AE ther Framework}
The Vortex \AE ther Model posits a classical, three-dimensional continuum governed by fluid dynamics and topological conservation. These postulates form the foundational structure from which analogues to general relativity and quantum mechanics emerge in subsequent sections.

\begin{table}[h!]
    \centering
    \begin{tabular}{rl}
        \toprule
        \hline
        \textbf{Postulates} & of the Vortex \AE ther Model (VAM).\\
        \hline
        \textbf{1. Continuum Space} & Space is a Euclidean, incompressible, inviscid fluid medium. \\
        \midrule
        \textbf{2. Knotted Particles} & Matter arises from stable, topologically conserved vortex knots. \\
        \midrule
        \textbf{3. Vortex Dynamics} & Vorticity is conserved and quantized across the continuum. \\
        \midrule
        \textbf{4. Absolute Time} & Time flows uniformly throughout the æther. \\
        \midrule
        \textbf{5. Local Time} & Clock rates vary locally due to pressure and vorticity gradients. \\
        \midrule
        \textbf{6. Gravity} & Emerges from pressure gradients induced by localized vorticity. \\
        \hline
        \bottomrule
    \end{tabular}
    \caption{Postulates of the Vortex Æther Model (VAM).\\}
    These principles replace spacetime curvature with structured rotational flow, \\
    forming the foundation for VAM analogues to mass, time, inertia, and gravity.
    \label{tab:postulates}
\end{table}


\subsection*{VAM Constants and Scaling}

The Vortex Æther Model (VAM) is anchored by a small set of universal constants that replace geometric curvature with fluid-dynamic quantities. These include:

\begin{table}[htbp]
    \centering
    \begin{tabular}{llc}
        \hline
        \textbf{Symbol} & \textbf{Name} & \textbf{Approx. Value} \\
        \hline
        $C_e$ & Core tangential velocity & $1.09384563 \times 10^6$ m/s \\
        $r_c$ & Vortex core radius (Coulomb barrier) & $1.40897017 \times 10^{-15}$ m \\
        $F_{\max}$ & Maximum vortex-interaction (Coulomb force) & $29.053507$ N \\
        $\rho_{\text{\ae}}$ & Æther density (Universal) & $3.89343582 \times 10^{18}$ kg/m$^3$ \\
        $\alpha$ & Fine-structure constant $= \frac{2 C_e}{c}$ (emergent)   & $7.2973525643 \times 10^{-3}$\\
        $R_e$ & Electron radius ($2 \times r_c$) & $2.81794092 \times 10^{-15}$ m \\
        $G_{\text{swirl}}$ & Derived from Æther properties, proportional to $C_e$ & $ \propto \rho_{\text{\ae}} C_e^2$ or $ \propto \frac{C_e}{r_c^2}$  \\
        $\alpha_g$ &   Gravitational coupling constant & $1.7518e-45$ \\
        $\gamma$  & Vorticity-gravity coupling ($G \cdot \rho_{\ae}^2$) & $3.27 \times 10^{-23}$  m$^5$/s$^2$\\
        $\kappa$  & Circulation quantum  ($C_e \cdot r_c$) & $1.54 \times 10^{-9}$  m$^2$/s     \\
        $\beta$  & Time-dilation coupling  ($r_c^2 / C_e^2$) & $1.66 \times 10^{-42}$  s$^2$      \\
        \hline
    \end{tabular}
    \caption{Fundamental VAM constants defined in prior work \cite{vam2025field, vam2025unified}.}\label{tab:table}
\end{table}



\subsection*{Planck-Scale Derivation of Maximum Vortex Force}

The maximum allowable force between vortex structures in VAM is derived from Planck-scale physics, scaled by the fine-structure constant and the geometric ratio of the vortex core to the Planck length:

\[
F_{\text{max}} = \frac{c^4}{4G} \cdot \alpha \cdot \left( \frac{r_c}{L_p} \right)^{-2}
\]

Using defined constants, this yields precisely:

\[
F_{\text{max}} = 29.053507~\text{N}
\]

This links the VAM-specific interaction limit directly to fundamental constants from both gravitation and quantum electrodynamics.





\subsection*{Derived Relationships}

\[
\rho_{\text{\ae}} = \frac{F_{\text{max}}}{\pi C_e^2 r_c^2}
\]

\[
G_\text{swirl} = \frac{C_e c^5 t_p^2}{2 F_{\text{max}} r_c^2} \quad \text{(VAM gravity analogue)}
\]

\[
\gamma = G \rho_{\text{\ae}}^2 = 6.67430 \times 10^{-11} \cdot (3.89 \times 10^{18})^2 \approx 1.01 \times 10^{-47}~\text{m}^5/\text{s}^2
\]

\[
\Phi(\vec{r}) = \gamma \int \frac{|\vec{\omega}(\vec{r}')|^2}{|\vec{r} - \vec{r}'|} \, d^3r' \quad \text{(Vorticity potential)}
\]

\subsection*{Topological Mass Scaling via Knot Energy}

The rest mass of an electron arises not only from inertial contributions due to local rotational energy but also from the topological linking of vortex threads. A trefoil knot represents the minimal topologically stable configuration with linking number \(Lk = 3\). Including this factor, the electron mass is given by:

\[
M_e = \boxed{ \frac{8\pi \rho_{\text{\ae}} r_c^3}{C_e} \cdot \underbrace{Lk}_{\text{Linking number}} }
\]

This accounts for the helicity, mutual induction, and circulation stored in linked vortex filaments forming the quantized core of matter. The mass scaling becomes exact when the knot structure is included, and this also justifies why simple radial energy distributions alone fall short of the observed rest mass of elementary particles.

This formulation unifies force, time, and gravitational scaling under a single Æther density, eliminating ambiguity between macroscopic and microscopic regimes. It anchors the dynamics of knot energetics, time modulation, and gravitational analogs without resorting to scale-dependent switching of constants.


\subsection*{Introduction to Fluid Dynamics and Vorticity Conservation}

\begin{table}[htbp]
    \centering
    \begin{tabular}{llc}
        \hline
        Symbol & Description \\
        \hline
        \midrule
        \(\vec{v}\) & Æther velocity field \\
        \(\vec{\omega}\) &  Vorticity \(\vec{\omega} = \nabla \times \vec{v}\) \\
        \(\Phi\) & Vorticity-induced potential \\
        \(\kappa\) & Circulation constant \\
        \(\mathcal{K}\) & Knot topological class (Hopf link, torus knot, etc.) \\
        \bottomrule
        \hline
    \end{tabular}\label{tab:table2}
\end{table}

Euler Equation (Inviscid Flow)
\begin{equation}
    \frac{\partial \vec{v}}{\partial t} + (\vec{v} \cdot \nabla)\vec{v} = -\frac{1}{\rho_\text{æ}} \nabla p\label{eq:Euler-Equation}
\end{equation}
Taking the curl to get the Vorticity Transport
\begin{equation}
    \frac{\partial \vec{\omega}}{\partial t} + (\vec{v} \cdot \nabla)\vec{\omega} = (\vec{\omega} \cdot \nabla) \vec{v}\label{eq:Vorticity-Transport}
\end{equation}

\subsection*{Vorticity-Induced Gravity}
We define a Newtonian like vorticity-based gravitational potential $\Phi$:
\begin{equation}
    \vec{F}_g = -\nabla \Phi\label{eq:Vorticity-Induced-Gravity}
\end{equation}
Where $\Phi$ is the Vorticity Potential:
\begin{equation}
    \Phi(\vec{r}) = \gamma \int \frac{|\vec{\omega}(\vec{r'})|^2}{|\vec{r} - \vec{r'}|} \, d^3r'\label{eq:Vorticity_Potential}
\end{equation}
Where \textbf{$\gamma$} in \textbf{$\text{m}^5 / \text{s}^{2}$} is the Vorticity-gravity coupling constant, replacing Newton's $G$. This mirrors the Newtonian potential but replaces mass density with vorticity intensity. This gives attractive force fields between vortex knots (like a particle).



\subsection*{Emergent Origin of Planck's Constant}

In the VAM framework, Planck's constant $\hbar$ is not fundamental but arises naturally from vortex-core mechanics. Specifically, the quantum kinetic energy term $\frac{\hbar^2}{2M_e}$ is exactly reproduced by the internal structure of a vortex knot:

\[
\frac{\hbar^2}{2M_e} = \frac{F_{\text{max}} r_c^3}{5 \lambda_c C_e}
\]

Rearranging, this yields:

\[
\hbar = \sqrt{ \frac{2M_e F_{\text{max}} r_c^3}{5 \lambda_c C_e} }
\]

This demonstrates that $\hbar$ emerges as a composite quantity involving the vortex force limit, the core size, the Compton wavelength, and the tangential velocity. Thus, quantum action quantization reflects the inertial and geometrical structure of rotating Æther knots.

This formulation unifies force, time, and gravitational scaling under a single Æther density, eliminating ambiguity between macroscopic and microscopic regimes. It anchors the dynamics of knot energetics, time modulation, and gravitational analogs without resorting to scale-dependent switching of constants.

In the VAM framework, the Schrödinger equation arises as a vortex-phase evolution equation where the kinetic operator is reinterpreted as a Laplacian over the rotating Æther density field. Substituting the vortex-derived expression for $\hbar^2 / 2 M_e$, we obtain:
$ i \hbar \frac{\partial \psi}{\partial t} = \left( -\frac{F_{\max} r_c^3}{5 \lambda_c C_e} \nabla^2 + V \right)\psi $

Equivalently, using the derived expression for $\hbar$, the quantum phase field evolves under:
$ i \frac{\partial \psi}{\partial t} = \left( - \frac{1}{\hbar} \cdot \frac{F_{\max} r_c^3}{5 \lambda_c C_e} \nabla^2 + \frac{V}{\hbar} \right)\psi $

This formulation reveals that quantum mechanical wavefunction dynamics are governed by the same constants responsible for vortex confinement, gravitational analogs, and topological mass generation. The phase evolution of matter is thus tied directly to Æther vortex rotation, and quantization becomes a natural byproduct of topological inertia in the superfluid medium.

\subsection*{ VAM Schrödinger Equation for Hydrogen-like Bound States}

In VAM, the standard Schrödinger equation emerges naturally from vortex energetics. Defining the effective vortex Planck constant:

$ \hbar_{\text{VAM}} = \frac{2 F_{\max} R_e^2}{C_e} $

The time-dependent wave equation becomes:
\begin{equation}
    i \hbar_{\text{VAM}} \frac{\partial \psi}{\partial t} =
\left[
    -16 F_{\max} r_c^3 \frac{c^2}{C_e^2} \nabla^2
    \;\;-\;\;
    Z \cdot \frac{F_{\max} r_c^2}{r}
\right] \psi
\end{equation}


Dividing through by $\hbar_{\text{VAM}}$, the normalized VAM Schrödinger equation reads:
\begin{equation}
    i \frac{\partial \psi}{\partial t} =
\left[
    -2.32 \times 10^{-4} \cdot \nabla^2
    \;\;-\;\;
    1.37 \times 10^5 \cdot \frac{Z}{r}
\right] \psi
\end{equation}


This matches the quantum hydrogen atom structure and supports that:


\begin{itemize}
\item Mass arises from internal swirl confinement
\item Coulomb interaction arises from vortex pressure gradients
\item Planck’s constant is a composite of core geometry and interaction limit
\end{itemize}


\section*{Resonant Ætheric Tunneling and LENR in VAM}

In the Vortex Æther Model (VAM), low-energy nuclear reactions (LENR) are reinterpreted as resonant tunneling events mediated by structured vortex interactions in the Æther. Unlike conventional quantum tunneling, which depends on particle wavefunctions penetrating a static Coulomb potential barrier, VAM proposes that local pressure minima---arising from vortex-induced Bernoulli deficits---can transiently reduce or eliminate the barrier entirely.

The classical Coulomb repulsion between two nuclei of charges \( Z_1 e \) and \( Z_2 e \) is given by:
\begin{equation}
    V_{\text{Coulomb}}(r) = \frac{Z_1 Z_2 e^2}{4\pi \varepsilon_0 r}
\end{equation}

In VAM, two rotating vortex knots at proximity \( r \sim 2r_c \) generate a vorticity-induced pressure drop via:
\begin{equation}
    \Delta P = \frac{1}{2} \rho_{\text{\ae}} r_c^2 (\Omega_1^2 + \Omega_2^2)
\end{equation}

This pressure drop modifies the effective interaction potential:
\begin{equation}
    V_{\text{eff}}(r) = V_{\text{Coulomb}}(r) - \Phi_\omega(r)
\end{equation}
where the vortex potential \( \Phi_\omega(r) \) is defined by:
\begin{equation}
    \Phi_\omega(r) = \gamma \int \frac{|\vec{\omega}(r')|^2}{|\vec{r} - \vec{r}'|} \, d^3r'
    \quad \text{with} \quad
    \gamma = G \rho_{\text{\ae}}^2
\end{equation}

Resonant tunneling occurs when the combined effect of \( \Delta P \) and \( \Phi_\omega \) neutralizes the Coulomb barrier at a critical separation \( r_t \):
\begin{equation}
    \frac{1}{2} \rho_{\text{\ae}} r_c^2 (\Omega_1^2 + \Omega_2^2) \geq \frac{Z_1 Z_2 e^2}{4\pi \varepsilon_0 r_t^2}
\end{equation}

The resulting condition allows transitions even at thermal or sub-thermal kinetic energies, enabling LENR to proceed not by overcoming the barrier, but by dynamically erasing it through vortex resonance. This provides a mechanical basis for LENR phenomena without requiring violation of conservation laws or standard nuclear models. The tunneling is thus a manifestation of Ætheric phase alignment and pressure-mediated coherence in confined vortex configurations.


\subsection*{VAM Quantum Electrodynamics (QED) Lagrangian}

In the Vortex \AE ther Model, the interaction of vortex knots with electromagnetic fields emerges from their helicity structure and induced vector potentials. The standard QED Lagrangian is replaced by:
$ \mathcal{L}_{\text{VAM-QED}} =
\bar{\psi} \left[ i \gamma^\mu \partial_\mu
               - \gamma^\mu \left( \frac{C_e^2 r_c}{\lambda_c} \right) A_\mu
               - \left( \frac{8\pi \rho_{\text{\ae}} r_c^3 Lk}{C_e} \right) \right] \psi
- \frac{1}{4} F_{\mu\nu} F^{\mu\nu} $

Here:
\begin{itemize}
\item The mass term arises from topologically linked vortex cores.
\item The gauge coupling emerges from the circulation and induced vector potential of the \AE ther.
\item The electromagnetic field tensor \( F_{\mu\nu} \) remains unchanged, representing curl interactions in the surrounding superfluid.
\end{itemize}
This formulation unifies vortex structure with field interactions, replacing the fundamental constants \( m \) and \( q \) with emergent expressions built from rotational, structural, and topological vortex properties.

To solve the Lagrangian, we will derive the Euler--Lagrange equation for the spinor field \( \psi \), which gives:

That is—we recover a Dirac-type equation with vortex-defined mass and interaction terms:
$ \boxed{
    \left( i \gamma^\mu \partial_\mu - \gamma^\mu q_{\text{vortex}} A_\mu - M_{\text{vortex}} \right)\psi = 0
} $

\subsection*{Hybrid VAM Frame-Dragging Angular Velocity}

In the Vortex Æther Model (VAM), the frame-dragging angular velocity induced by a rotating vortex-bound object is defined analogously to the Lense--Thirring effect in General Relativity, but with a scale-dependent coupling:

\begin{equation}
    \omega_{\text{drag}}^{\text{VAM}}(r) =
    \frac{4 G m}{5 c^2 r} \cdot \mu(r) \cdot \Omega(r)
\end{equation}

Here, \( G \) is the gravitational constant, \( c \) is the speed of light, \( m \) the object's mass, \( r \) its characteristic radius, and \( \Omega(r) \) its angular velocity.

The hybrid coupling factor \( \mu(r) \) interpolates between quantum-scale vortex behavior and classical macroscopic rotation:

\begin{equation}
    \mu(r) =
    \begin{cases}
        \displaystyle \frac{r_c C_e}{r^2}, & \text{if } r < r_\ast \quad \text{(quantum or vortex core regime)} \\
        1, & \text{if } r \geq r_\ast \quad \text{(macroscopic regime)}
    \end{cases}
\end{equation}

where:
\begin{itemize}
    \item \( r_c \) is the vortex core radius,
    \item \( C_e \) is the tangential velocity of the vortex core,
    \item \( r_\ast \sim 10^{-3} \, \text{m} \) is the transition radius between microscopic and macroscopic regimes.
\end{itemize}

This formulation ensures continuity with GR predictions for celestial bodies, while enabling VAM-specific predictions for elementary particles and subatomic vortex structures.
\subsection*{VAM Gravitational Redshift from Core Rotation}

In the Vortex Æther Model (VAM), gravitational redshift arises from the local rotational velocity \( v_\phi \) at the outer boundary of a vortex knot. Assuming no spacetime curvature and absolute time, the effective gravitational redshift is given by:

\begin{equation}
    z_{\text{VAM}} =
    \left( 1 - \frac{v_\phi^2}{c^2} \right)^{-\frac{1}{2}} - 1
\end{equation}

where:
\begin{itemize}
    \item \( v_\phi = \Omega(r) \cdot r \) is the tangential velocity due to local rotation,
    \item \( \Omega(r) \) is the angular velocity at the measurement radius \( r \),
    \item \( c \) is the speed of light in vacuum.
\end{itemize}

This expression reflects the modification of time perception caused by local rotational energy, replacing the curvature-based gravitational potential \( \Phi \) of General Relativity with a velocity-field term. It becomes equivalent to the GR Schwarzschild redshift for low \( v_\phi \), and diverges as \( v_\phi \rightarrow c \), providing a natural cutoff for local frame evolution:

\begin{equation}
    \lim_{v_\phi \to c} z_{\text{VAM}} \to \infty
\end{equation}

\subsection*{VAM Local Time Dilation Models}

In the Vortex Æther Model (VAM), local time dilation is interpreted as the modulation of absolute time due to internal vortex dynamics, not spacetime curvature. Two physically grounded formulations are used, depending on the system scale:

\paragraph{1. Velocity-Field-Based Time Dilation}

This model relates local time flow to the tangential velocity of the rotating ætheric structure (vortex knot, planet, or star):

\begin{equation}
    \frac{d\tau}{dt} =
    \sqrt{1 - \frac{v_\phi^2}{c^2}} =
    \sqrt{1 - \frac{\Omega^2 r^2}{c^2}}
\end{equation}

where:
\begin{itemize}
    \item \( v_\phi = \Omega \cdot r \) is the tangential velocity,
    \item \( \Omega \) is the angular velocity at radius \( r \),
    \item \( c \) is the speed of light.
\end{itemize}

\paragraph{2. Rotational Energy-Based Time Dilation}

At large scales or high rotational inertia, time dilation arises from stored rotational energy, leading to:

\begin{equation}
    \frac{d\tau}{dt} =
    \left(1 + \frac{1}{2} \cdot \beta \cdot I \cdot \Omega^2 \right)^{-1}
\end{equation}

with:
\begin{itemize}
    \item \( I = \frac{2}{5} m r^2 \): moment of inertia for a uniform sphere,
    \item \( \beta = \frac{r_c^2}{C_e^2} \): coupling constant from vortex-core dynamics,
    \item \( m \) is the object's mass.
\end{itemize}

\paragraph{Interpretation}

These models imply that time slows down in regions of high local rotational energy or vorticity, in agreement with gravitational time dilation effects in GR. However, in VAM, these effects arise purely from internal dynamics of the æther flow, under flat 3D Euclidean geometry and absolute time.

\subsection*{VAM Orbital Precession (GR Equivalent)}

In General Relativity, the perihelion precession of an orbiting body is attributed to spacetime curvature. In the Vortex Æther Model (VAM), this effect is replaced by the cumulative influence of a swirl-induced vorticity field within a rotating Æther medium.

The equivalent VAM formulation mirrors the GR prediction, but arises from vorticity-induced pressure gradients and circulation:

\begin{equation}
    \Delta\phi_{\text{VAM}} =
    \frac{6\pi G M}{a(1 - e^2) c^2}
\end{equation}

where:
\begin{itemize}
    \item \( M \): mass of the central vortex attractor,
    \item \( a \): semi-major axis of the orbit,
    \item \( e \): orbital eccentricity,
    \item \( G \): gravitational constant (recovered from VAM coupling),
    \item \( c \): speed of light.
\end{itemize}

Although formally identical to the GR expression, in VAM this arises from the variation in local circulation and angular momentum flux within the surrounding Æther, modulating the effective potential and resulting in precessional motion.
\subsection*{VAM Light Deflection by Ætheric Circulation}

In General Relativity, light deflection by massive bodies is due to spacetime curvature. In the Vortex Æther Model, light (viewed as a perturbation or mode in the Æther) bends due to circulation-induced pressure gradients and anisotropic refractive index fields near rotating vortex attractors.

The equivalent VAM deflection angle for a light ray grazing a spherical vortex mass is given by:

\begin{equation}
    \delta_{\text{VAM}} =
    \frac{4 G M}{R c^2}
\end{equation}

where:
\begin{itemize}
    \item \( M \): effective mass of the rotating vortex knot,
    \item \( R \): closest approach (impact parameter),
    \item \( G \): vortex coupling constant (recovering Newtonian \( G \) under macroscopic limits),
    \item \( c \): speed of light.
\end{itemize}

In VAM, this results from the interaction between the light's propagation velocity and the surrounding rotational field. The light wavefront is locally compressed or refracted due to tangential Æther flow gradients, leading to observable angular deflection.
\subsection*{VAM vs GR Observable Correspondence Summary}

\usepackage{multirow}
\renewcommand{\arraystretch}{1.5}

\begin{tabular}{|c|c|l|}
    \hline
    \textbf{Observable} & \textbf{Theory} & \textbf{Expression} \\
    \hline

    \multirow{2}{*}{Time Dilation}
    & GR & \( \displaystyle \frac{d\tau}{dt} = \sqrt{1 - \frac{2GM}{rc^2}} \) \\
    & VAM & \( \displaystyle \sqrt{1 - \frac{\Omega^2 r^2}{c^2}} \) \\

    \hline
    \multirow{2}{*}{Redshift}
    & GR & \( \displaystyle z = \left(1 - \frac{2GM}{rc^2} \right)^{-1/2} - 1 \) \\
    & VAM & \( \displaystyle z = \left(1 - \frac{v_\phi^2}{c^2} \right)^{-1/2} - 1 \) \\

    \hline
    \multirow{2}{*}{Frame Dragging}
    & GR & \( \displaystyle \omega_{\text{LT}} = \frac{2GJ}{c^2 r^3} \) \\
    & VAM & \( \displaystyle \frac{2G \mu I \Omega}{c^2 r^3} \) \\

    \hline
    \multirow{2}{*}{Precession}
    & GR/VAM & \( \displaystyle \Delta\phi = \frac{6\pi GM}{a(1 - e^2)c^2} \) \\

    \hline
    \multirow{2}{*}{Light Deflection}
    & GR/VAM & \( \displaystyle \delta = \frac{4GM}{Rc^2} \) \\

    \hline
    \multirow{2}{*}{Gravitational Potential}
    & GR & \( \Phi = -\frac{GM}{r} \) \\
    & VAM & \( \Phi = -\frac{1}{2} \vec{\omega} \cdot \vec{v} \) \\

    \hline
    Gravity Constant
    & VAM & \( \displaystyle G = \frac{C_e c^5 t_p^2}{2 F_{\text{max}} r_c^2} \) \\
    \hline
\end{tabular}

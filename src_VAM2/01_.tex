%! Author = mr
%! Date = 3/27/2025

\section{Time Dilation in the Æther-Vortex Model}\label{sec:Part-1}
We consider an inviscid, irrotational superfluid æther with stable topological vortex knots. The Æther experiences absolute time $t_{\text{abs}}$, but local clocks experience slowed rates due to pressure gradients and knot energetics. The Vortex Æther Model posits that the rate at which time flows in the local frame (near the knot) depends on the internal angular frequency $\Omega_k$. In this section, we derive time dilation analogues inspired by the predictions of general relativity (GR), based solely on pressure and vorticity gradients in the fluid.

\subsection{Bernoulli-Based Local Time Modulation}
In high-vorticity regions, Bernoulli's principle implies a drop in pressure near vortex cores:
\begin{equation}
    \frac{1}{2} \rho_\text{æ} v^2 + p = p_0 \Rightarrow p = p_0 - \frac{1}{2} \rho_\text{æ} v^2
\end{equation}
Assuming that the local physical clock rate is proportional to pressure, we define the local frequency of time as:
\begin{equation}
    f_{\text{local}} = f_0 \cdot \left( \frac{p}{p_0} \right) = f_0 \left( 1 - \frac{\rho_\text{æ} v^2}{2p_0} \right)
\end{equation}
Thus, the dilation of local time relative to absolute (background) time becomes:
\begin{equation}
    \frac{t_{\text{local}}}{t_0} = \left( 1 - \frac{\rho_\text{æ} v^2}{2p_0} \right)^{-1}
\end{equation}
For circular vortex flow where $v = \Omega r$:
\begin{equation}
    \frac{t_{\text{local}}}{t_0} \approx 1 + \frac{\rho_\text{æ} \Omega^2 r^2}{2p_0}
\end{equation}
This reduces time rate locally with higher knot rotation, modeling time modulation without relativity, where $\rho_\text{æ}/p_0 \sim 1/c^2$.

\subsection{Heuristic Time Modulation by Knot Rotation}
Let $\Omega_k$ be the average angular velocity of a vortex knot:
\begin{equation}
    \label{eq:omegamod}
    \frac{t_{\text{local}}}{t_{\text{abs}}} = \left(1 + \alpha \Omega_k^2 \right)^{-1}
\end{equation}
For small $\Omega_k$:
\begin{equation}
    \frac{t_{\text{local}}}{t_{\text{abs}}} \approx 1 - \alpha \Omega_k^2 + \mathcal{O}(\Omega_k^4)
\end{equation}
This matches special relativistic time dilation:
\begin{equation}
    \frac{t_{\text{moving}}}{t_{\text{rest}}} \approx 1 - \frac{v^2}{2c^2}
\end{equation}
This heuristic expression generalizes the local effect of rotational energy into a topological time-modulation law, consistent with the Bernoulli-derived expansion in the low-vorticity limit.

\subsection{Vorticity-Induced Gravitational Time Dilation}
In the æther-vortex framework, gravity emerges from pressure gradients induced by localized vorticity.
Let $\Phi(\vec{r})$ be a scalar potential analogous to the Newtonian gravitational potential, defined by:
\begin{equation}
    \Phi(\vec{r}) = \gamma \int \frac{|\vec{\omega}(\vec{r}')|^2}{|\vec{r} - \vec{r}'|} \, d^3r'
\end{equation}
Time dilation follows:
\begin{equation}
    \frac{t_{\text{local}}}{t_{\infty}} = \sqrt{1 - \frac{2\Phi(\vec{r})}{c^2}}
\end{equation}
High vorticity → low pressure → slowed time
If the vorticity field is concentrated at a point-like knot (analogous to a mass), i.e., $|\vec{\omega}(\vec{r}')|^2 \sim \delta(\vec{r}')$, we retrieve a Newtonian-like form:
\begin{equation}
    \frac{t_{\text{local}}}{t_{\infty}} = \sqrt{1 - \frac{2\gamma \omega_0^2}{r c^2}}
\end{equation}

\subsection{Kerr-Like Time Adjustment}
From GR:
\begin{equation}
    t_{\text{adjusted}} = \Delta t \cdot \sqrt{1 - \frac{2GM}{rc^2} - \frac{J^2}{r^3 c^2}}
\end{equation}
In Æther terms:
\begin{equation}
    \boxed{t_{\text{adjusted}} = \Delta t \cdot \sqrt{1 - \frac{\gamma \langle \omega^2 \rangle}{r c^2} - \frac{\kappa^2}{r^3 c^2}}}
\end{equation}
This reproduces gravitational and frame-dragging effects via vorticity and circulation.

\subsection{Conclusion and Experimental Outlook}
The Vortex Æther Model replaces spacetime curvature with conserved vorticity in a 3D fluid. Time dilation arises from localized pressure depletion and kinetic energy storage within vortex knots, offering a classical, topological reinterpretation of relativistic effects. In this formulation, time slows in regions of high vorticity due to pressure depletion, aligning with relativistic predictions \cite{fedi2017gravity, simula2020gravitational, winterberg1990maxwell}. Future work includes simulations of vortex clocks and tests using BECs, helium II, or electrohydrodynamic lifters.



\section{Time Modulation by Vortex Knot Rotation}

In the æther-vortex model, matter is composed of stable, topologically conserved vortex knots embedded in a superfluid medium. Each knot possesses an intrinsic angular velocity $\Omega_k$, and its local dynamics influence the rate at which time flows relative to the absolute time of the background æther. We hypothesize that time modulation arises from the rotational energetics of these knots, without invoking spacetime curvature as in General Relativity.

\subsection{Heuristic and Energetic Derivation}

We begin with the ansatz that the local rate of time is slowed by internal rotation:
\begin{equation}
    \label{eq:time_slowing}
    \frac{t_{\text{local}}}{t_{\text{abs}}} = \left(1 + \alpha \Omega_k^2 \right)^{-1}
\end{equation}
where:
\begin{itemize}
    \item $t_{\text{abs}}$ is the absolute æther time,
    \item $\Omega_k$ is the average angular frequency of the knot’s core rotation,
    \item $\alpha$ is a coupling parameter characterizing the effect of vorticity on local time.
\end{itemize}

This has the same low-velocity expansion as special relativity:
\begin{equation}
    \frac{t_{\text{local}}}{t_{\text{abs}}} \approx 1 - \alpha \Omega_k^2 + \mathcal{O}(\Omega_k^4)
\end{equation}
which parallels the Lorentz dilation:
\[
    \frac{t_{\text{moving}}}{t_{\text{rest}}} \approx 1 - \frac{v^2}{2c^2}
\]

We can give Equation~\ref{eq:time_slowing} a physical basis via the rotational energy of the knot:
\[
    E_{\text{rot}} = \frac{1}{2} I \Omega_k^2
\]
This yields a more fundamental expression for time modulation:
\begin{equation}
    \label{eq:time_energy}
    \frac{t_{\text{local}}}{t_{\text{abs}}} = \left(1 + \alpha E_{\text{rot}} \right)^{-1} = \left(1 + \frac{1}{2} \alpha I \Omega_k^2 \right)^{-1}
\end{equation}

\subsection{Topological and Physical Justification}

Rotating vortex knots store both kinetic energy and helicity:
\[
    \mathcal{H} = \int \vec{v} \cdot \vec{\omega} \, d^3x
\]
Helicity is conserved in ideal flows and reflects knot topology. Therefore, $\Omega_k$ serves as a meaningful descriptor of the particle’s internal “clock rate.” Faster rotation leads to deeper pressure wells and time slowing, akin to gravitational redshift.

This model:
\begin{itemize}
    \item Explains time modulation intrinsically without reference frames,
    \item Avoids invoking curved spacetime,
    \item Ties time dilation to fluid-dynamic, conserved quantities.
\end{itemize}

\begin{equation}
    \boxed{
        t_{\text{local}} = \frac{t_{\text{abs}}}{1 + \frac{1}{2} \alpha I \Omega_k^2}
    }
\end{equation}
This boxed equation summarizes time dilation as a function of rotational inertia in the vortex core, forming the foundation of the vortex-energetic time model.

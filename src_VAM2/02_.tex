%! Author = mr
%! Date = 3/27/2025

\section{Time Modulation by Vortex Knot Rotation}

In the æther-vortex model, matter is composed of topologically conserved vortex knots, stable structures embedded within a superfluid medium. These knots possess an intrinsic angular velocity $\Omega_k$, and their local dynamics are hypothesized to influence the rate at which time is experienced relative to the absolute time of the background æther.
Instead of using spacetime curvature (as in GR), we model how internal vortex motion slows local time due to its effect on energy, pressure, and information transfer rates in the surrounding æther.



\subsection{Time modulation from a vortex-theoretic basis.}
From classical mechanics we take the Rotational Energy of a Vortex Knot:
\[
    E_{\text{rot}} = \frac{1}{2} I \Omega_k^2
\]
where:
\begin{itemize}
    \item $I$ is the moment of inertia of the vortex core.
    \item $\Omega_k$ is the internal angular velocity of the knot.
\end{itemize}
We assume that the local time rate slows with increasing internal energy, akin to time dilation in special relativity:
\[
    t_{\text{local}} = \frac{t_{\text{abs}}}{1 + \alpha E_{\text{rot}}} = \frac{t_{\text{abs}}}{1 + \alpha \cdot \frac{1}{2} I \Omega_k^2} = \frac{t_{\text{abs}}}{1 + \frac{1}{2} \alpha I \Omega_k^2}
\]
Expansion for Small Rotation (Low $\Omega_k$):
\[
    \frac{t_{\text{local}}}{t_{\text{abs}}} \approx 1 - \frac{1}{2} \alpha I \Omega_k^2 + \mathcal{O}(\Omega_k^4)
\]
This matches the special relativistic time dilation form:
\[
    \frac{t_{\text{moving}}}{t_{\text{rest}}} \approx 1 - \frac{v^2}{2c^2}
\]
Which gives us a final equation for time modulation by vortex rotation:
\[
    t_{\text{local}} = \frac{t_{\text{abs}}}{1 + \frac{1}{2} \alpha I \Omega_k^2}
\]
This gives us a vortex-theoretic basis for time modulation, embedding rotational inertia and allowing quantitative modeling of local time slowdowns.

\subsection{Hypothesis: Time Rate Depends on Knot Rotation}

Let $\Omega_k$ be the average angular velocity of a vortex knot. Since rotational motion in fluids can influence local energy density and pressure (via Bernoulli-like principles), we postulate that the \textbf{local time rate} $t_{\text{local}}$ is affected by the internal rotational energy of the knot.

Assume the following ansatz:
\begin{equation}
    \frac{t_{\text{local}}}{t_{\text{abs}}} = \left(1 + \alpha \, \Omega_k^2 \right)^{-1}
    \label{eq:timemod}
\end{equation}
where:
\begin{itemize}
    \item $t_{\text{abs}}$ is the absolute background time defined by the stationary æther.
    \item $\alpha$ is a model-dependent coupling constant (with dimensions $[\text{time}]^2$), characterizing how strongly rotational motion influences the perception of time.
    \item $\Omega_k$ is the scalar mean angular frequency of the knot's core circulation.
\end{itemize}

\subsection{Interpretation of Equation \ref{eq:timemod}}

Equation \ref{eq:timemod} expresses how faster internal rotation (i.e., higher $\Omega_k$) leads to a \textbf{slower local time rate}, mimicking time dilation. Unlike relativity, where velocity with respect to an observer or gravitational potential affects time, here it is the \textbf{internal rotational dynamics of topological structures} that alter time flow.

For small values of $\Omega_k$, a Taylor expansion yields:
\begin{equation}
    \frac{t_{\text{local}}}{t_{\text{abs}}} \approx 1 - \alpha \Omega_k^2 + \mathcal{O}(\Omega_k^4)
\end{equation}
which shows a quadratic dependence analogous to the Lorentz factor in special relativity:
\[
    \frac{t_{\text{moving}}}{t_{\text{rest}}} \approx 1 - \frac{v^2}{2c^2}
\]

\subsection{Topological and Physical Justification}

Rotating vortex knots are known to store both kinetic energy and helicity \cite{moffatt1969degree, knottedvortices2012kleckner}. Since topological helicity $\mathcal{H} = \int \vec{v} \cdot \vec{\omega} \, d^3x$ is conserved in ideal flows, and is related to knot complexity, we can view $\Omega_k$ as a physically meaningful descriptor of the particle’s internal clock.

Thus, in this model:
\begin{itemize}
    \item Time dilation occurs intrinsically and locally.
    \item There is no need for reference frames or spacetime curvature.
    \item All effects arise from conserved fluid-dynamic quantities and their energetics.
\end{itemize}

This approach provides an alternative to relativistic time dilation, rooted in the physics of topological fluid dynamics and supported by experimental observations of rotating coherent structures in quantum fluids.
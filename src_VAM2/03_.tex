%! Author = mr
%! Date = 3/27/2025

\section{proper time for a rotating observer}\label{sec:Part-3}

In General Relativity, the flow of proper time for a rotating observer in a stationary, axisymmetric spacetime is given by
\begin{equation}
    \left(\frac{d\tau}{dt}\right)^2_{\text{GR}} = -\left[g_{tt} + 2 g_{t\phi}\Omega_{\text{eff}} + g_{\phi\phi}\Omega_{\text{eff}}^2\right],
    \label{eq:GRtime}
\end{equation}
where $\Omega_{\text{eff}}$ is the observer's angular velocity, and the metric components $g_{\mu\nu}$ describe spacetime curvature (e.g., Kerr geometry) \cite{misner1973gravitation}.

In a vortex-based Æther theory, we posit that time dilation arises not from spacetime curvature, but from the local motion of an inviscid superfluid medium. We associate metric-like effects with Æther flow variables:
\begin{align*}
    g_{tt} &\rightarrow -\left(1 - \frac{v_r^2}{c^2} \right), \\
    g_{t\phi} &\rightarrow -\frac{v_r v_\phi}{c^2}, \\
    g_{\phi\phi} &\rightarrow -\frac{v_\phi^2}{c^2} r^2,
\end{align*}
where $v_r$ and $v_\phi$ are the radial and tangential components of Æther velocity, and $v_\phi = r \Omega_k$, with $\Omega_k = \kappa / (2\pi r^2)$ representing the local vortex rotation rate \cite{fedi2017gravity, sbitnev2021quaternion}.

Substituting into the structure of Equation~\ref{eq:GRtime}, we obtain:
\begin{align}
    \left( \frac{d\tau}{dt} \right)^2_{\text{Æther}} &= 1 - \frac{v_r^2}{c^2} - 2\frac{v_r v_\phi}{c^2} - \frac{v_\phi^2}{c^2} \\
    &= 1 - \frac{1}{c^2}(v_r + v_\phi)^2 \\
    &= 1 - \frac{1}{c^2}(v_r + r \Omega_k)^2.
    \label{eq:timeflow}
\end{align}

This result mirrors the GR proper time flow structure, yet is entirely fluid-mechanical. It predicts the slowing of proper time near intense vortex structures due to Æther flow speeds approaching $c$, effectively creating a “time-well” analogous to gravitational redshift.

\subsection*{Kerr-Like Time Adjustment from Vorticity and Circulation}

The General Relativistic form of time adjustment near a rotating mass, such as in the Kerr geometry, is approximately
\begin{equation}
    t_{\text{adjusted}} = \Delta t \cdot \sqrt{1 - \frac{2GM}{rc^2} - \frac{J^2}{r^3 c^2}}.
    \label{eq:kerrtime}
\end{equation}

We now express this in Æther-vortex terms, replacing $M$ and $J$ with effective vorticity energy density and circulation:
\begin{itemize}
    \item Mass-energy term: $\displaystyle \frac{2GM}{rc^2} \rightarrow \frac{\gamma \langle \omega^2 \rangle}{rc^2}$,
    \item Angular momentum term: $\displaystyle \frac{J^2}{r^3 c^2} \rightarrow \frac{\kappa^2}{r^3 c^2}$.
\end{itemize}

Thus, the Æther-based analog becomes:
\begin{equation}
    \boxed{
        t_{\text{adjusted}} = \Delta t \cdot \sqrt{
            1 - \frac{\gamma \langle \omega^2 \rangle}{r c^2}
            - \frac{\kappa^2}{r^3 c^2}
        }
    }
    \label{eq:ae_kerr}
\end{equation}

This formulation reproduces gravitational and frame-dragging time effects purely from Æther dynamics: $\langle \omega^2 \rangle$ plays the role of gravitational redshift, and circulation $\kappa$ encodes rotational drag. This approach aligns with recent fluid-dynamic interpretations of gravity and time \cite{barcelo2011analogue}, \cite{fedi2017gravity}.

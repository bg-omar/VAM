\section{Proper Time for a Rotating Observer in Æther Flow}

Having established time dilation in the Vortex Æther Model (VAM) through pressure, angular velocity, and rotational energy, we now extend our formalism to rotating observers. This section demonstrates that fluid-dynamic time modulation in VAM can reproduce expressions structurally similar to those derived in General Relativity (GR), particularly in axisymmetric rotating spacetimes like the Kerr geometry. However, VAM achieves this without invoking spacetime curvature. Instead, time modulation is governed entirely by kinetic variables in the æther field.

\subsection*{A. GR Proper Time in Rotating Frames}

In General Relativity, the proper time \(d\tau\) for an observer with angular velocity \(\Omega_{\text{eff}}\) in a stationary, axisymmetric spacetime is given by:

\begin{equation}
\left( \frac{d\tau}{dt} \right)^2_{\text{GR}} = -\left[ g_{tt} + 2g_{t\varphi} \Omega_{\text{eff}} + g_{\varphi\varphi} \Omega_{\text{eff}}^2 \right]
\tag{18}
\end{equation}

where \(g_{\mu\nu}\) are components of the spacetime metric (e.g., in Boyer–Lindquist coordinates for Kerr spacetime). This formulation accounts for both gravitational redshift and rotational (frame-dragging) effects.

\subsection*{B. Æther-Based Analog: Velocity-Derived Time Modulation}

In VAM, spacetime is not curved. Instead, observers reside within a dynamically structured æther whose local flow velocities determine time dilation. Let the radial and tangential components of æther velocity be:

\begin{itemize}
\item \(v_r\): radial velocity,
\item \(v_\varphi = r\Omega_k\): tangential velocity due to local vortex rotation,
\item \(\Omega_k = \frac{\kappa}{2\pi r^2}\): local angular velocity (with \(\kappa\) as circulation).
\end{itemize}

We postulate a correspondence between GR metric components and æther velocity terms:

\begin{equation}
\begin{aligned}
g_{tt} &\rightarrow -\left(1 - \frac{v_r^2}{c^2}\right), \\
g_{t\varphi} &\rightarrow -\frac{v_r v_\varphi}{c^2}, \\
g_{\varphi\varphi} &\rightarrow -\frac{v_\varphi^2}{c^2 r^2}
\end{aligned}
\tag{19}
\end{equation}

Substituting these into the GR expression for proper time, we obtain the VAM-based analog:

\begin{equation}
\left( \frac{d\tau}{dt} \right)^2_{\ae} = 1 - \frac{v_r^2}{c^2} - \frac{2v_r v_\varphi}{c^2} - \frac{v_\varphi^2}{c^2}
\tag{20}
\end{equation}

Combining the terms:

\begin{equation}
\left( \frac{d\tau}{dt} \right)^2_{\ae} = 1 - \frac{1}{c^2}(v_r + v_\varphi)^2
\tag{21}
\end{equation}

This formulation reproduces gravitational and frame-dragging time effects purely from Æther dynamics: $\langle \omega^2 \rangle$ plays the role of gravitational redshift, and circulation $\kappa$ encodes rotational drag. This approach aligns with recent fluid-dynamic interpretations of gravity and time \cite{barcelo2011analogue}, \cite{fedi2017gravity}.
This model currently assumes irrotational flow outside knots and neglects viscosity, turbulence, and quantum compressibility. Future extensions may include quantized circulation spectra or boundary effects in confined Æther systems.

\begin{equation}
\boxed{\left( \frac{d\tau}{dt} \right)^2_{\ae} = 1 - \frac{1}{c^2}(v_r + r\Omega_k)^2}
\tag{Æther-Based Proper Time for Rotating Observer}
\end{equation}

\subsection*{C. Physical Interpretation and Model Consistency}

This boxed result mirrors the GR expression for rotating observers but arises strictly from classical fluid dynamics. It shows that as the local æther speed approaches the speed of light—due to either radial inflow or rotational motion—the proper time slows. This implies the existence of "time wells" where kinetic energy density dominates.

Key observations:

\begin{itemize}
\item In the absence of radial flow (\(v_r = 0\)), time slowing arises entirely from vortex rotation.
\item When both \(v_r\) and \(\Omega_k\) are present, the cumulative velocity reduces local time rate.
\item This expression agrees with Section II's energetic model if we interpret \(v_r + r\Omega_k\) as contributing to the local energy density.
\end{itemize}

Thus, in the VAM framework, the structure of the observer’s proper time emerges from ætheric flow fields. This confirms that GR-like temporal behavior can emerge in a flat, Euclidean 3D space with absolute time, governed entirely by structured vorticity and circulation.

In the next section, we explore how VAM extends this correspondence to gravitational potentials and frame-dragging effects via circulation and vorticity intensity, forming an analog to the Kerr time redshift formula.
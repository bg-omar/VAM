\section{Kerr-Like Time Adjustment from Vorticity and Circulation}

To complete the analogy between General Relativity (GR) and the Vortex Æther Model (VAM), we now derive a time modulation formula that mirrors the redshift and frame-dragging structure found in the Kerr solution. In GR, the Kerr metric describes the spacetime geometry around a rotating mass, predicting both gravitational time dilation and frame-dragging due to angular momentum. VAM captures similar phenomena through the dynamics of structured vorticity and circulation in the æther, without requiring spacetime curvature.

\subsection{General Relativistic Kerr Redshift Structure}

In the GR Kerr metric, the proper time $d\tau$ for an observer near a rotating mass is affected by both mass-energy and angular momentum. A simplified approximation for the time dilation factor near a rotating body is:
\begin{equation}
    t_{\text{adjusted}} = \Delta t \cdot \sqrt{1 - \frac{2GM}{rc^2} - \frac{J^2}{r^3c^2}}
    \label{eq:Kerr_time_dilation}
\end{equation}
where:
\begin{itemize}
    \item $M$: mass of the rotating body,
    \item $J$: angular momentum,
    \item $r$: radial distance from the source,
    \item $G$: Newton’s gravitational constant,
    \item $c$: speed of light.
\end{itemize}

The first term corresponds to gravitational redshift from mass, while the second accounts for rotational (frame-dragging) effects.

\subsection{Æther Analog via Vorticity and Circulation}

In VAM, we express gravitational-like influences through vorticity intensity $\langle \omega^2 \rangle$ and total circulation $\kappa$. These are interpreted as:
\begin{itemize}
    \item $\langle \omega^2 \rangle$: mean squared vorticity over a region,
    \item $\kappa$: conserved circulation, encoding angular momentum.
\end{itemize}

We define the æther-based analog by making the replacements:
\begin{equation}
    \begin{aligned}
        \frac{2GM}{rc^2} &\rightarrow \frac{\gamma \langle \omega^2 \rangle}{rc^2}, \\
        \frac{J^2}{r^3c^2} &\rightarrow \frac{\kappa^2}{r^3c^2}
    \end{aligned}
\label{eq:Kerr_replacements}
\end{equation}

Here, $\gamma$ is a coupling constant relating vorticity to effective gravitational strength (analogous to $G$). Then the æther-based proper time becomes:

\begin{equation}
    \boxed{t_{\text{adjusted}} = \Delta t \cdot \sqrt{1 - \frac{\gamma \langle \omega^2 \rangle}{rc^2} - \frac{\kappa^2}{r^3c^2}}}
    \label{eq:Kerr_time_dilation_ae}
\end{equation}

This mirrors the Kerr redshift and frame-dragging structure using fluid-dynamic variables. In this picture:
\begin{itemize}
    \item $\langle \omega^2 \rangle$ plays the role of energy density producing gravitational redshift,
    \item $\kappa$ represents angular momentum generating temporal frame-dragging,
    \item The equation reduces to flat æther time ($t_{\text{adjusted}} \to \Delta t$) when both terms vanish.
\end{itemize}

\subsection{Model Assumptions and Scope}

This result depends on several assumptions:
\begin{itemize}
    \item The flow is irrotational outside the vortex cores,
    \item Viscosity and turbulence are neglected,
    \item Compressibility is ignored (ideal incompressible superfluid),
    \item Vorticity fields are sufficiently smooth to define $\langle \omega^2 \rangle$.
\end{itemize}

These conditions mirror the assumptions of ideal fluid GR analog models. The formulation bridges the macroscopic flow dynamics of the æther with effective geometric predictions, reinforcing the possibility of replacing curved spacetime with structured vorticity fields.

For detailed derivations of cross-energy and vortex interaction energetics, see Appendix~\ref{appendix:1}.

In future work, corrections for boundary conditions, quantized vorticity spectra, and compressible effects may be added to refine the analogy. Next, we will summarize how these fluid-based time dilation mechanisms unify under the VAM framework and identify their experimental implications.

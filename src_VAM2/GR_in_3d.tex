%! Author = Omar Iskandarani
%! Date = 2/15/2025

\documentclass[aps,preprint,superscriptaddress]{revtex4}
\usepackage[none]{hyphenat}
\usepackage{array}
\usepackage{booktabs}
\usepackage{amsmath}
\usepackage{amssymb}
\usepackage{graphicx}
\usepackage{hyperref}
\usepackage{physics}

\begin{document}
\sloppy % Allow LaTeX to adjust spacing to avoid overfull boxes
\author{Omar Iskandarani}
\title{The Vortex Æther Model: Æther Vortex Field Model}
\date{\today}
\affiliation{Independent Researcher, Groningen, The Netherlands}
\thanks{ORCID: \href{https://orcid.org/0009-0006-1686-3961}{0009-0006-1686-3961}}
\email{info@omariskandarani.com}


%%%%%%%%%%%%%%%%%%%%%%%%%%    Abstract    %%%%%%%%%%%%%%%%%%%%%%%%%%


\begin{abstract}
    The Vortex Æther Model (VAM) introduces a unified, non-relativistic framework wherein gravity, electromagnetism, and quantum phenomena emerge from structured vorticity in an inviscid superfluid-like æther. Unlike General Relativity, which invokes spacetime curvature, VAM models stable vortex knots in a 3D Euclidean medium with absolute time. Observed time dilation results from vortex-induced local energy gradients. This paper derives time dilation analogs to GR, explores vortex-energetic time shifts, and presents experimental implications.
\end{abstract}


\maketitle
%%%%%%%%%%%%%%%%%%%%%%%%%%    Introduction    %%%%%%%%%%%%%%%%%%%%%%%%%%
    \section*{Core Assumptions}
    The Æther is modeled as an inviscid, incompressible superfluid governed by:

\begin{tabular}{ll}
    \toprule
    \midrule
        * & Conservation of Absolute Vorticity \\
        * & A 3D Euclidean medium with absolute time \\
        * & Particles as vortex knots \\
        * & Irrotational outside vortex cores, but with conserved vorticity inside knots \\
        * & Gravity from vorticity-induced pressure gradients \\
    \bottomrule
\end{tabular}


    \begin{tabular}{ll}
        \toprule
        Symbol & Description \\
        \midrule
        \(\vec{v}\) & Æther velocity field \\
        \(\vec{\omega}\) &  Vorticity \(\vec{\omega} = \nabla \times \vec{v}\) \\
        \(\rho_\text{æ}\) & Æther density (constant) \\
        \(\Phi\) & Vorticity-induced potential \\
        \(\kappa\) & Circulation constant \\
        \(\mathcal{K}\) & Knot topological class (Hopf link, torus knot, etc.) \\
        \bottomrule
    \end{tabular}

    \subsection*{Introduction to Fluid Dynamics and Vorticity Conservation}
    Euler Equation (Inviscid Flow)
    \begin{equation}
        \frac{\partial \vec{v}}{\partial t} + (\vec{v} \cdot \nabla)\vec{v} = -\frac{1}{\rho_\text{æ}} \nabla p
    \end{equation}
    Taking the curl to get the Vorticity Transport
    \begin{equation}
        \frac{\partial \vec{\omega}}{\partial t} + (\vec{v} \cdot \nabla)\vec{\omega} = (\vec{\omega} \cdot \nabla) \vec{v}
    \end{equation}

    \subsection*{Vorticity-Induced Gravity}
    We define a Newtonian like vorticity-based gravitational potential $\Phi$:
    \begin{equation}
        \vec{F}_g = -\nabla \Phi
    \end{equation}
    Where $\Phi$ is the Vorticity Potential:
    \begin{equation}
        \Phi(\vec{r}) = \gamma \int \frac{|\vec{\omega}(\vec{r'})|^2}{|\vec{r} - \vec{r'}|} \, d^3r'
    \end{equation}
    This mirrors the Newtonian potential but replaces mass density with vorticity intensity. This gives attractive force fields between vortex knots (like a particle).


    %%%%%%%%%%%%%%%%%%%%%%%%%%    PART 1    %%%%%%%%%%%%%%%%%%%%%%%%%%
    %! Author = mr
%! Date = 3/27/2025

\section{Time Dilation from Vortex Dynamics}\label{sec:Part-1}
We consider an inviscid, irrotational superfluid æther with stable topological vortex knots. The Æther experiences absolute time $t_{\text{abs}}$, but local clocks experience slowed rates due to pressure gradients and knot energetics. The Vortex Æther Model posits that the rate at which time flows in the local frame (near the knot) depends on the internal angular frequency $\Omega_k$. In this section, we derive time dilation analogues inspired by the predictions of general relativity (GR), based solely on pressure and vorticity gradients in the fluid.

\begin{figure}[h!]
    \centering
    \includegraphics[width=0.85\textwidth]{export/streamlinesDiPole}
    \caption{Velocity streamlines, vorticity, pressure, and local time rate $\tau$ for a simulated vortex pair. The pressure minimum and time slow-down clearly align with the regions of high vorticity. This directly illustrates the æther model's central claim: time dilation follows from vortex energetics and pressure depletion.}
    \label{fig:vortexfields}
\end{figure}

\subsection{Bernoulli \& Rotational Flow}
In high-vorticity zones, Bernoulli's principle implies a local drop in pressure:
\begin{equation}
    \frac{1}{2} \rho_\text{æ} v^2 + p = p_0 \quad \Rightarrow \quad p = p_0 - \frac{1}{2} \rho_\text{æ} v^2
\end{equation}
Assuming the local frequency of a clock is proportional to Æther pressure:
\begin{equation}
    f_{\text{local}} = f_0 \left(1 - \frac{\rho_\text{æ} v^2}{2p_0} \right)
\end{equation}
Thus, the time dilation becomes:
\begin{equation}
    \frac{t_{\text{local}}}{t_0} = \left(1 - \frac{\rho_\text{æ} v^2}{2p_0} \right)^{-1}
\end{equation}
For circular vortex motion with $v = \Omega r$:
\begin{equation}
    \frac{t_{\text{local}}}{t_0} \approx 1 + \frac{\rho_\text{æ} \Omega^2 r^2}{2p_0}
\end{equation}
This is analogous to special relativistic dilation if $\rho_\text{æ}/p_0 \sim 1/c^2$.

\begin{figure}[h!]
    \centering
    \includegraphics[width=0.8\textwidth]{export/RadialProfileOfLocalTimeDilation_Radial_LocalTime_Dilation}
    \caption{Radial profile of normalized local time $t_{\text{local}} / t_{\text{abs}}$ as a function of distance $r$ from the vortex core, assuming $\Omega_k \propto 1/r^2$. Time slows significantly near the vortex center and recovers to background values with distance.}
    \label{fig:radial_time_profile}
\end{figure}

\subsection{Heuristic Knot-Based Time Modulation}
Topological vortex knots possess internal rotation. Let $\Omega_k$ be their average angular velocity. We postulate a heuristic time modulation law:
\begin{equation}
    \label{eq:heuristic}
    \frac{t_{\text{local}}}{t_{\text{abs}}} = \left(1 + \alpha \Omega_k^2 \right)^{-1}
\end{equation}
where $\alpha$ is a coupling constant related to the Æther's compressibility or inertia. For small $\Omega_k$, we expand:
\begin{equation}
    \frac{t_{\text{local}}}{t_{\text{abs}}} \approx 1 - \alpha \Omega_k^2 + \mathcal{O}(\Omega_k^4)
\end{equation}
This expression parallels the Lorentz factor for low velocities:
\[
    \frac{t_{\text{moving}}}{t_{\text{rest}}} \approx 1 - \frac{v^2}{2c^2}
\]

\begin{figure}[h!]
    \centering
    \includegraphics[width=0.8\textwidth]{export/RadialProfileOfLocalTimeDilation_Vortex-Induced_Time_Well}
    \caption{Schematic of a vortex-induced time well in the æther. Local time $t_{\text{local}} / t_{\text{abs}}$ is shown as a color gradient in 2D space. The central vortex region exhibits the most time slowing due to high $\Omega_k$, forming a well-like structure.}
    \label{fig:vortex_time_well}
\end{figure}


\subsection{Energetic Interpretation: Rotational Inertia}
The above heuristic can be grounded in the rotational energy of vortex knots. Let $I$ be the effective moment of inertia of a knot, then the energy stored is:
\begin{equation}
    E_\text{rot} = \frac{1}{2} I \Omega_k^2
\end{equation}
Assuming the time dilation arises from this stored energy modifying local information rates:
\begin{equation}
    \frac{t_{\text{local}}}{t_{\text{abs}}} = \left(1 + \alpha I \Omega_k^2 \right)^{-1}
\end{equation}
This model provides a bridge between fluid rotation and gravitational-like time shifts, replacing the need for spacetime curvature with internal knot energetics. It suggests that time flow slows where circulation is topologically conserved.

\begin{equation}
    \boxed{
        t_{\text{local}} = \frac{t_{\text{abs}}}{1 + \alpha I \Omega_k^2}
    }
\end{equation}
This boxed result summarizes the time modulation law driven by rotational inertia.

\subsection{Conclusion and Experimental Outlook}
The Vortex Æther Model replaces spacetime curvature with conserved vorticity in a 3D fluid. Time dilation arises from localized pressure depletion and kinetic energy storage within vortex knots, offering a classical, topological reinterpretation of relativistic effects. In this formulation, time slows in regions of high vorticity due to pressure depletion, aligning with relativistic predictions \cite{fedi2017gravity, simula2020gravitational, winterberg1990maxwell}. Future work includes simulations of vortex clocks and tests using BECs, helium II, or electrohydrodynamic lifters.



    %%%%%%%%%%%%%%%%%%%%%%%%%%    PART 2    %%%%%%%%%%%%%%%%%%%%%%%%%%
    \subsection*{Section II: Time Modulation by Vortex Knot Rotation}

Building upon the previous section's treatment of time dilation via pressure and Bernoulli dynamics, we now focus on the intrinsic rotation of topological vortex knots. In the Vortex Æther Model (VAM), particles are modeled as stable, topologically conserved vortex knots embedded in an incompressible, inviscid superfluid medium. Each knot possesses a characteristic internal angular frequency $\Omega_k$, and this internal motion induces local time modulation relative to the absolute time of the æther.

Rather than curving spacetime, we propose that internal rotational energy and helicity conservation induce temporal slowdowns analogous to gravitational redshift. This section develops these ideas through heuristic and energetic arguments consistent with the hierarchy introduced in Section I.

\subsection*{A. Heuristic and Energetic Derivation}

We begin by proposing a rotationally-induced time dilation formula based on the knot's internal angular frequency:

\begin{equation}
\frac{t_{\text{local}}}{t_{\text{abs}}} = \left(1 + \alpha \Omega_k^2 \right)^{-1}
\end{equation}

where:

\begin{itemize}
\item $t_{\text{local}}$ is the proper time near the knot,
\item $t_{\text{abs}}$ is the absolute time of the background æther,
\item $\Omega_k$ is the average core angular frequency,
\item $\alpha$ is a coupling coefficient with dimensions $[\alpha] = \text{s}^2$.
\end{itemize}

For small angular velocities, we obtain a first-order expansion:

\begin{equation}
\frac{t_{\text{local}}}{t_{\text{abs}}} \approx 1 - \alpha \Omega_k^2 + \mathcal{O}(\Omega_k^4)
\end{equation}

This form parallels the Lorentz factor at low velocities in special relativity:

\begin{equation}
\frac{t_{\text{moving}}}{t_{\text{rest}}} \approx 1 - \frac{v^2}{2c^2}
\end{equation}

This establishes an important analogy: internal rotational motion in VAM induces temporal slowing, similar to how translational velocity induces time dilation in SR.

To strengthen the physical foundation of this expression, we now relate time dilation to the energy stored in vortex rotation. Let the vortex knot have an effective moment of inertia $I$. Its rotational energy is given by:

\begin{equation}
E_{\text{rot}} = \frac{1}{2} I \Omega_k^2
\end{equation}

Assuming time slows due to this energy density, we write:

\begin{equation}
\frac{t_{\text{local}}}{t_{\text{abs}}} = \left(1 + \alpha E_{\text{rot}} \right)^{-1} = \left(1 + \frac{1}{2} \alpha I \Omega_k^2 \right)^{-1}
\end{equation}

This expression serves as the energetic analog of the pressure-based Bernoulli model from Section I. It supports the interpretation of vortex-induced time wells via energy storage rather than geometric deformation.

We highlight this key result with a boxed formulation:

\begin{equation}
\boxed{\frac{t_{\text{local}}}{t_{\text{abs}}} = \left(1 + \frac{1}{2} \alpha I \Omega_k^2 \right)^{-1}}
\end{equation}

\subsection*{B. Topological and Physical Justification}

Topological vortex knots are not only characterized by rotation but also by helicity:

\begin{equation}
H = \int \vec{v} \cdot \vec{\omega} \, d^3x
\end{equation}

Helicity is a conserved quantity in ideal (inviscid, incompressible) fluids, encoding the linkage and twisting of vortex lines. The rotational frequency $\Omega_k$ becomes a topologically meaningful indicator of the knot’s identity and dynamical state.

Higher $\Omega_k$ implies greater rotational energy and stronger localized pressure depletion, forming a "temporal well" in the æther. These wells naturally mimic gravitational redshift effects in curved spacetime, but arise here purely from classical fluid mechanics.

This model:

\begin{itemize}
\item Attributes time modulation to conserved, intrinsic rotational energy,
\item Requires no external reference frames (absolute æther time is universal),
\item Preserves temporal isotropy outside the vortex core,
\item Provides a natural replacement for GR's spacetime curvature.
\end{itemize}

Therefore, this vortex-energetic time dilation principle provides a powerful alternative to relativistic time modulation by anchoring all temporal effects in rotational energetics and topological invariants.

In the next section, we will show how these ideas reproduce metric-like behavior for rotating observers, including a direct fluid-mechanical analog to the Kerr metric of General Relativity.
    %%%%%%%%%%%%%%%%%%%%%%%%%%    PART 3    %%%%%%%%%%%%%%%%%%%%%%%%%%
    \subsection*{Section III: Proper Time for a Rotating Observer in Æther Flow}

Having established time dilation in the Vortex Æther Model (VAM) through pressure, angular velocity, and rotational energy, we now extend our formalism to rotating observers. This section demonstrates that fluid-dynamic time modulation in VAM can reproduce expressions structurally similar to those derived in General Relativity (GR), particularly in axisymmetric rotating spacetimes like the Kerr geometry. However, VAM achieves this without invoking spacetime curvature. Instead, time modulation is governed entirely by kinetic variables in the æther field.

\subsection*{A. GR Proper Time in Rotating Frames}

In General Relativity, the proper time \(d\tau\) for an observer with angular velocity \(\Omega_{\text{eff}}\) in a stationary, axisymmetric spacetime is given by:

\begin{equation}
\left( \frac{d\tau}{dt} \right)^2_{\text{GR}} = -\left[ g_{tt} + 2g_{t\varphi} \Omega_{\text{eff}} + g_{\varphi\varphi} \Omega_{\text{eff}}^2 \right]
\tag{18}
\end{equation}

where \(g_{\mu\nu}\) are components of the spacetime metric (e.g., in Boyer–Lindquist coordinates for Kerr spacetime). This formulation accounts for both gravitational redshift and rotational (frame-dragging) effects.

\subsection*{B. Æther-Based Analog: Velocity-Derived Time Modulation}

In VAM, spacetime is not curved. Instead, observers reside within a dynamically structured æther whose local flow velocities determine time dilation. Let the radial and tangential components of æther velocity be:

\begin{itemize}
\item \(v_r\): radial velocity,
\item \(v_\varphi = r\Omega_k\): tangential velocity due to local vortex rotation,
\item \(\Omega_k = \frac{\kappa}{2\pi r^2}\): local angular velocity (with \(\kappa\) as circulation).
\end{itemize}

We postulate a correspondence between GR metric components and æther velocity terms:

\begin{equation}
\begin{aligned}
g_{tt} &\rightarrow -\left(1 - \frac{v_r^2}{c^2}\right), \\
g_{t\varphi} &\rightarrow -\frac{v_r v_\varphi}{c^2}, \\
g_{\varphi\varphi} &\rightarrow -\frac{v_\varphi^2}{c^2 r^2}
\end{aligned}
\tag{19}
\end{equation}

Substituting these into the GR expression for proper time, we obtain the VAM-based analog:

\begin{equation}
\left( \frac{d\tau}{dt} \right)^2_{\ae} = 1 - \frac{v_r^2}{c^2} - \frac{2v_r v_\varphi}{c^2} - \frac{v_\varphi^2}{c^2}
\tag{20}
\end{equation}

Combining the terms:

\begin{equation}
\left( \frac{d\tau}{dt} \right)^2_{\ae} = 1 - \frac{1}{c^2}(v_r + v_\varphi)^2
\tag{21}
\end{equation}

This formulation reproduces gravitational and frame-dragging time effects purely from Æther dynamics: $\langle \omega^2 \rangle$ plays the role of gravitational redshift, and circulation $\kappa$ encodes rotational drag. This approach aligns with recent fluid-dynamic interpretations of gravity and time \cite{barcelo2011analogue}, \cite{fedi2017gravity}.
This model currently assumes irrotational flow outside knots and neglects viscosity, turbulence, and quantum compressibility. Future extensions may include quantized circulation spectra or boundary effects in confined Æther systems.

\begin{equation}
\boxed{\left( \frac{d\tau}{dt} \right)^2_{\ae} = 1 - \frac{1}{c^2}(v_r + r\Omega_k)^2}
\tag{Æther-Based Proper Time for Rotating Observer}
\end{equation}

\subsection*{C. Physical Interpretation and Model Consistency}

This boxed result mirrors the GR expression for rotating observers but arises strictly from classical fluid dynamics. It shows that as the local æther speed approaches the speed of light—due to either radial inflow or rotational motion—the proper time slows. This implies the existence of "time wells" where kinetic energy density dominates.

Key observations:

\begin{itemize}
\item In the absence of radial flow (\(v_r = 0\)), time slowing arises entirely from vortex rotation.
\item When both \(v_r\) and \(\Omega_k\) are present, the cumulative velocity reduces local time rate.
\item This expression agrees with Section II's energetic model if we interpret \(v_r + r\Omega_k\) as contributing to the local energy density.
\end{itemize}

Thus, in the VAM framework, the structure of the observer’s proper time emerges from ætheric flow fields. This confirms that GR-like temporal behavior can emerge in a flat, Euclidean 3D space with absolute time, governed entirely by structured vorticity and circulation.

In the next section, we explore how VAM extends this correspondence to gravitational potentials and frame-dragging effects via circulation and vorticity intensity, forming an analog to the Kerr time redshift formula.
    %! Author = mr
%! Date = 3/29/2025


\chapter*{VAM Vortex Scattering Framework (Inspired by Elastic Theory)}

\section*{1. Governing Equations of VAM Vorticity Dynamics}

\subsection*{1.1 Vorticity Transport Equation (Linearized Form)}

In the Vortex Æther Model (VAM), the dynamics of the vorticity field \(\vec{\omega} = \nabla \times \vec{v}\) are governed by the Euler equation and its vorticity form:

\[
\frac{\partial \omega_i}{\partial t} + v_j \partial_j \omega_i = \omega_j \partial_j v_i
\]

This nonlinear structure implies vortex deformation due to stretching and advection. For small perturbations \(\delta\omega\) near a background vortex knot field \(\omega^{(0)}\), linearization gives:

\[
\frac{\partial (\delta \omega_i)}{\partial t} + v_j^{(0)} \partial_j (\delta \omega_i) \approx \omega_j^{(0)} \partial_j (\delta v_i)
\]

Define the VAM linear response operator \(\mathcal{L}_{ij}\):

\[
\mathcal{L}_{ij} \, \delta v_j(\vec{r}) = \delta F_i^{\text{vortex}}(\vec{r})
\]

\subsection*{1.2 Vorticity Green Tensor Equation}

\[
\mathcal{L}_{ij} \, \mathcal{G}_{jk}(\vec{r}, \vec{r}') = -\delta_{ik} \, \delta(\vec{r} - \vec{r}')
\]

The induced velocity field \(v_i\) from a source vortex forcing \(F_k(\vec{r}')\) is then:

\[
v_i(\vec{r}) = \int \mathcal{G}_{ik}(\vec{r}, \vec{r}') \, F_k^{\text{vortex}}(\vec{r}') \, d^3 r'
\]

\section*{2. VAM Scattering Theory for Vortex Knots}

\subsection*{2.1 Born Approximation for Vorticity Perturbations}

Assume an incident vorticity potential \(\Phi^{(0)}(\vec{r})\) encounters a vortex knot at \(\vec{r}_k\). The scattered vorticity field becomes:

\[
\Phi(\vec{r}) = \Phi^{(0)}(\vec{r}) + \int \mathcal{G}_{ij}(\vec{r}, \vec{r}') \, \delta \mathcal{V}_{jk}(\vec{r}') \, v_k^{(0)}(\vec{r}') \, d^3r'
\]

Here, \(\delta \mathcal{V}_{jk}\) represents a vorticity polarizability tensor associated with the knot—a VAM analog to elastic moduli perturbation.

\section*{3. Æther Stress Tensor and Energy Flux}

\subsection*{3.1 VAM Stress Tensor}

\[
\mathcal{T}_{ij} = \rho_\ae \, v_i v_j - \frac{1}{2} \delta_{ij} \rho_\ae v^2
\]

\subsection*{3.2 Æther Vorticity Force Density}

\[
f_i^{\text{vortex}} = \partial_j \mathcal{T}_{ij}
\]

\subsection*{3.3 Vorticity Energy Flux}

\[
\vec{S}_\omega = - \mathcal{T} \cdot \vec{v}
\]

This vector captures energy transfer through vortex knot interactions and defines scattering "cross sections" via the divergence \(\nabla \cdot \vec{S}_\omega\).

\section*{4. Time Dilation and Knot Scattering}

\subsection*{4.1 Time Dilation from Knot Rotation}

Let the incident vorticity field induce localized time slowing due to a knot’s rotational energy:

\[
\frac{t_{\text{local}}}{t_{\infty}} = \left(1 + \frac{1}{2} \alpha I \Omega_k^2 \right)^{-1}
\]

In the Born approximation, the change in proper time near a knot under external vorticity flow is:

\subsection*{4.2 Scattered Correction from External Field}

\[
\delta \left( \frac{t_{\text{local}}}{t_{\infty}} \right) \approx - \frac{1}{2} \alpha I \Omega_k \, \delta \Omega_k
\]

\[
\delta \Omega_k \sim \int \chi(\vec{r}_k - \vec{r}') \cdot \vec{\omega}^{(0)}(\vec{r}') \, d^3r'
\]

Here, \(\chi\) is the topological vortex susceptibility kernel.



\section*{5. Summary of VAM-Inspired Scattering Constructs}

\begin{table}
    \centering
    \begin{tabular}{lll}
        \toprule
        \textbf{Concept} & \textbf{Elastic Theory} & \textbf{VAM Analog} \\
        \midrule
        Medium property & cijklc_{ijkl} & ρ\ae,Ωk,κ\rho_\ae, \Omega_k, \kappa \\
        Wavefield & Displacement uiu_i & Velocity viv_i \\
        Source & Body force fif_i & Vorticity forcing FivortexF_i^{\text{vortex}} \\
        Green function & Gij(r⃗,r⃗′)G_{ij}(\vec{r}, \vec{r}') & Gij(r⃗,r⃗′)\mathcal{G}_{ij}(\vec{r}, \vec{r}') \\
        Stress tensor & τij\tau_{ij} & Tij\mathcal{T}_{ij} \\
        Energy flux & JP,i=−τiju˙jJ_{P,i} = -\tau_{ij} \dot{u}_j & Sω,i=−TijvjS_{\omega,i} = -\mathcal{T}_{ij} v_j \\
        Time dilation mechanism & GR metric gμνg_{\mu\nu} & Ωk,κ,⟨ω2⟩\Omega_k, \kappa, \langle \omega^2 \rangle \\
        \bottomrule
    \end{tabular}
    \caption{}
    \label{tab:}
\end{table}


This scattering framework generalizes classical elastic analogs into a topologically and energetically motivated Ætheric formalism. It enables the computation of field modifications, time dilation effects, and energy flux due to stable, interacting vortex knots in the Vortex Æther Model (VAM).


    %%%%%%%%%%%%%%%%%%%%%%%%%%    PART 4    %%%%%%%%%%%%%%%%%%%%%%%%%%
    \documentclass[11pt]{article}
\usepackage[margin=1in]{geometry}
\usepackage{amsmath,amssymb}
\usepackage{amsthm}
\usepackage{graphicx}
\usepackage{hyperref}

\begin{document}

\title{Foundations of Velocity Fields and Energies in a Vortex System: A Brief Article}
\author{(VAM Working Group)}
\date{\today}
\maketitle

\begin{abstract}
This article outlines theoretical foundations of vortical velocity fields and their associated energies,
including a distinction between self- and cross-energies, in the context of a generic vortex-based model.
We close with a derivation outline for the cross-energy term, highlighting its application in vortex dynamics
and fluid–structure interactions.
\end{abstract}

\section{Introduction}
Vortex dynamics are a core component of many fluid and plasma systems, including
tornado-like flows, knotted vortices in classical or superfluid turbulence, and various
complex topological fluid systems. A deeper understanding of the energy budgets
associated with these flows can shed light on processes like vortex stability, reconnection,
and global flow organization. We begin by motivating how velocity fields can be
decomposed so as to capture the total energy (i.e.\ self- plus cross-energy), and how
this approach helps track flows in both 2D and 3D.

\section{Foundations: Velocity Fields and Total (Self + Cross) Energy}
\label{sec:foundations}
In an incompressible fluid, the velocity field $\mathbf{u}(\mathbf{x}, t)$ is typically
governed by the Navier--Stokes or Euler equations. For inviscid analyses, the Euler
equations for incompressible flow read
\begin{equation}
  \frac{\partial \mathbf{u}}{\partial t} + (\mathbf{u} \cdot \nabla)\mathbf{u} = -\frac{1}{\rho}\nabla p,
  \quad \nabla \cdot \mathbf{u} = 0.
\end{equation}
We also consider the vorticity $\boldsymbol{\omega} = \nabla \times \mathbf{u}$,
which can be used to characterize vortex structures.

To understand the \emph{total} kinetic energy, we can split it as follows:
\begin{equation}
  E_{\text{total}} \;=\; E_{\text{self}} \;+\; E_{\text{cross}}.
\end{equation}
Here, $E_{\text{self}}$ is that portion of energy which each vortex or partial flow
element contributes independently (for instance, from local swirling motions), while
$E_{\text{cross}}$ encodes the contributions that arise from the interaction of different
vortical elements. In a multi-vortex scenario, such a decomposition helps isolate the
direct interaction between two (or more) vortex filaments or sheets.

\section{Momentum and Self-Energy Considerations}
\label{sec:momentum}
A starting point is to recall that for a single vortex of circulation $\Gamma$, with an
azimuthally symmetric core, the induced velocity is sometimes approximated by
classical results such as
\begin{equation}
   V \;=\; \frac{\Gamma}{4 \pi R}
   \bigl(\ln \tfrac{8 R}{a} - \beta \bigr),
\end{equation}
where $R$ is the main vortex loop radius, $a \ll R$ is a measure of core thickness,
and $\beta$ depends on details of the core model \cite{Saffman1992}. The
\emph{self-energy} associated with that vortex, $E_{\text{self}}$, can be cast in a
similar form that depends on $\ln(R/a)$, exemplifying how thin-core vortices'
energies scale with geometry.

In more general fluid or vortex-lattice models, we can track $E_{\text{self}}$ as the
sum of individual core energies. Further, the presence of multiple filaments modifies
the total energy by cross-terms of the velocity fields (the cross-energy). This
cross-energy often drives key phenomena such as vortex merging or the `recoil'
effects in wave--vortex interactions.

\section{Defining and Tracking Cross-Energy}
\label{sec:cross}
When multiple vortices (or partial velocity distributions) co-exist, the total velocity
field $\mathbf{u}$ can be superposed:
\begin{equation}
   \mathbf{u} \;=\; \mathbf{u}_1 \;+\;\mathbf{u}_2,
\end{equation}
where $\mathbf{u}_1$ and $\mathbf{u}_2$ come from distinct sub-systems. In that
scenario, the kinetic energy for a fluid volume $V$ is
\begin{align}
   E_{\text{total}} &= \frac{\rho}{2} \int_V \mathbf{u}^2 \,dV
   = \frac{\rho}{2} \int_V \bigl(\mathbf{u}_1 + \mathbf{u}_2 \bigr)^2\, dV \\
   &= \frac{\rho}{2} \int_V \mathbf{u}_1^2 \,dV \;+\;\frac{\rho}{2} \int_V \mathbf{u}_2^2 \, dV
   \;+\;\rho \int_V \mathbf{u}_1 \cdot \mathbf{u}_2 \, dV,
\end{align}
revealing an interaction or \emph{cross-energy} term
\begin{equation}
   E_{\text{cross}} \;=\; \rho \int_V \mathbf{u}_1 \cdot \mathbf{u}_2 \, dV.
   \label{eq:cross-term}
\end{equation}
Much of the interesting physics arises from \eqref{eq:cross-term}, because it
grows or shrinks depending on the vortex geometry and distance between them.
Its dynamical evolution can lead to, e.g., merging or rebound. A main point is that
each vortex's self-velocity can significantly affect the mutual velocities and thus
create net forces or torque.

\section{Applications to Helicity and Topological Flows}
\label{sec:helicity}
A related concept is helicity, measuring the topological complexity (knotting or
linking) of vortex tubes. Classically, helicity $H$ is given by
\begin{equation}
   H \;=\; \int_V \mathbf{u} \cdot \boldsymbol{\omega}\, dV,
\end{equation}
which can remain constant or be partially lost during reconnection events. In certain
dissipative flows, the cross-energy terms in \eqref{eq:cross-term} can influence
the effective rate of helicity change. Understanding $E_{\text{cross}}$ is important
for analyzing reconnection pathways in classical or superfluid turbulence.

\section{Derivation Outline for Cross-Energy}
\label{sec:derivation}
Finally, we provide a succinct outline for deriving the cross-energy expression.
Starting with the total velocity field $\mathbf{u} = \sum_{n=1}^N \mathbf{u}_n$
for $N$ vortex or partial velocity fields, the total kinetic energy is:
\begin{equation}
   E_{\text{total}}
   = \frac{\rho}{2} \int_V \left(\sum_{n=1}^N \mathbf{u}_n \right)^2 dV
   = \frac{\rho}{2} \sum_{n=1}^N \int_V \mathbf{u}_n^2 \, dV
      \;+\;\rho \sum_{n<m} \int_V \mathbf{u}_n \cdot \mathbf{u}_m \, dV.
\end{equation}
One obtains $N$ self-energy terms plus pairwise cross-energy integrals.
The cross-energy for a pair $(i,j)$ is:
\begin{equation}
   E_{\text{cross}}^{(ij)} \;=\; \rho \int_V \mathbf{u}_i \cdot \mathbf{u}_j \, dV.
\end{equation}
In practice, each $\mathbf{u}_n$ may be represented by known solutions of the
Stokes or potential flow equations, or from approximate solutions for vortex loops.
Then, either analytically or numerically, one obtains approximate cross-energies
that can be used in reduced models describing the evolution of multi-vortex
systems.

\section*{Conclusion}
We have surveyed how the total fluid kinetic energy in the presence of multiple
vortices can be split into self- and cross-energy terms. These cross-energy
contributions are crucial for understanding vortex merging, knotted vortex
untangling, or vortex–wave interactions in classical, superfluid, and plasma
flows. In addition, we have sketched a systematic derivation of cross-energy and
highlighted key aspects in discussing momentum and helicity. Future directions
include refining these expressions for axisymmetric or knotted vortices and
integrating them into large-scale models or computational frameworks.

\begin{thebibliography}{99}

\bibitem{Saffman1992}
P.~G. Saffman,
\textit{Vortex Dynamics},
Cambridge University Press, 1992.

\bibitem{BabinskyStevens2016}
H. Babinsky and R. Stevens,
Low-order modeling approaches for unsteady aero flows,
\textit{Exp. Fluids}, \textbf{57} (2016), 71--85.

\bibitem{AndreuAngulo2020}
I. Andreu-Angulo, C. Manzo, B. Basu and H. Babinsky,
On unsteady aerodynamic effect of strong wind gusts on small-scale rotor systems,
\textit{AIAA J.}, \textbf{58} (2020), 5041--5056.

\bibitem{Wu1981}
J.-Z. Wu,
Theory of Vorticity and Vortex Dynamics,
Springer, 1981.

\bibitem{LiWu2018}
M. Li and J.-Z. Wu,
Generalized vortex force maps for the unsteady lift,
\textit{Theor. Comput. Fluid Dyn.} \textbf{32} (2018), 695--710.

\bibitem{KlecknerIrvine2013}
D. Kleckner and W. T. M. Irvine,
Creation and dynamics of knotted vortices,
\textit{Nature Phys.}, \textbf{9} (2013), 253--258.

\end{thebibliography}

\end{document}
    %%%%%%%%%%%%%%%%%%%%%%%%%%    PART 5    %%%%%%%%%%%%%%%%%%%%%%%%%%
    \input{05_}

    %%%%%%%%%%%%%%%%%%%%%%%%%%    References    %%%%%%%%%%%%%%%%%%%%%%%%%%

    \bibliography{citations}
    \bibliographystyle{apsrev4-2}


%%%%%%%%%%%%%%%%%%%%%%%%%%    Appendices    %%%%%%%%%%%%%%%%%%%%%%%%%%

    \appendix \label{sec:Part-6}
    \input{appendix_01}\label{appendix:1}
    

\section{Integration of Clausius' Heat Theory into the Vortex \AE ther Model (VAM)}

The integration of Clausius' Mechanical Theory of Heat into the Vortex \AE ther Model (VAM) extends the framework's reach into thermodynamics, allowing a unified interpretation of energy, entropy, and quantum behavior based on structured vorticity in an inviscid superfluid-like \AE ther medium \cite{clausius1865mechanical, maxwell1865electromagnetic, helmholtz1858integrals}.

\section{VAM-Specific Constants and Dimensional Considerations}

To maintain internal consistency and bridge the Vortex \AE ther Model (VAM) with established physical quantities, we define several fundamental constants unique to this framework:

\begin{tabular}{lll}
    \toprule
    Symbol & Units & Description \\
    \midrule
    $\rho_{\AE}$ & $\text{kg}\cdot\text{m}^{-3}$ & The density of the æther, analogous to mass density in fluid mechanics. \\
    $\gamma$ & $\text{m}^5 \cdot \text{s}^{-2}$ & Vorticity-gravity coupling constant, replacing Newton's $G$. \\
    $\alpha$ & $\text{s}^2$ & Time dilation coupling coefficient for rotational energy. \\
    $\kappa$ & $\text{m}^2/\text{s}$ & Circulation (Kelvin's constant), related to angular momentum per unit mass. \\
    $C_e$ & $\text{m}/\text{s}$ & Edge tangential velocity of a vortex knot, serving as a characteristic propagation speed. \\
    \bottomrule
\end{tabular}

These constants are introduced as analogs to gravitational, electromagnetic, and thermodynamic parameters. Their values are to be determined through theoretical derivation or matched with experimental data in future sections.

\subsection{Thermodynamic First Principles in VAM}

The classical first law of thermodynamics is expressed as:
\begin{equation}
\Delta U = Q - W,
\end{equation}
where $\Delta U$ is the change in internal energy, $Q$ is heat added, and $W$ is work done by the system \cite{clausius1865mechanical}. Within VAM, this becomes:
\begin{equation}
\Delta U = \Delta \left( \frac{1}{2} \rho_{\AE} \int v^2 \, dV + \int P \, dV \right),
\end{equation}
with $\rho_{\AE}$ the æther density, $v$ the local velocity, and $P$ the pressure within equilibrium vortex domains \cite{vam2025unified}.

\subsection{Entropy and Structured Vorticity}

VAM posits that entropy is a function of vorticity intensity:
\begin{equation}
S \propto \int \omega^2 \, dV,
\end{equation}
where $\omega = \nabla \times v$ \cite{kelvin1867vortex}. Thus, entropy becomes a measure of topological complexity and energy dispersion encoded in the vortex network.

\subsection{Thermal Response of Vortex Knots}

Stable vortex knots embedded in equilibrium pressure surfaces behave analogously to thermodynamic systems:
\begin{itemize}
\item \textbf{Heating ($Q > 0$)} expands the knot, lowers core pressure, and increases entropy.
\item \textbf{Cooling ($Q < 0$)} contracts the knot, concentrating energy and stabilizing vorticity.
\end{itemize}
This provides a fluid-mechanical analog to gas laws under energetic input.

\subsection{Photoelectric Analogy in VAM}

Rather than invoking quantized photons, VAM interprets the photoelectric effect through vortex dynamics. A vortex must absorb enough energy to destabilize and eject its structure:
\begin{equation}
W = \frac{1}{2} \rho_{\AE} \int v^2 \, dV + P_{\text{eq}} V_{\text{eq}},
\end{equation}
where $W$ is the disintegration work threshold. If an incident wave modulates internal vortex energy beyond this, ejection occurs \cite{vam2025unified}.

The critical force for vortex ejection is:
\begin{equation}
F_{\text{max}} = \rho_{\AE} C_e^2 \pi r_c^2,
\end{equation}
with $C_e$ the vortex's edge velocity and $r_c$ its core radius. This yields a natural frequency cutoff below which no interaction occurs, akin to the threshold frequency in quantum photoelectricity \cite{einstein1905photoelectric}.

\subsection{Conclusion and Integration}

This thermodynamic extension of VAM enriches the model by embedding classical heat and entropy principles within fluid-dynamic structures. It not only bridges vortex physics with Clausius' laws but also offers a field-based reinterpretation of light-matter interactions, unifying mechanical and electromagnetic thermodynamics without discrete particle assumptions.






\subsection*{I. Vortex Knots as Particles}
Each particle is a topological vortex knot:
\begin{itemize}
    \item Charge ↔ twist or chirality of knot
    \item Mass ↔ integrated vorticity energy
    \item Spin ↔ knot helicity:
\end{itemize}
\subsection*{Helicity as Particle Identity}
\begin{equation}
    \mathcal{H} = \int \vec{v} \cdot \vec{\omega} \, d^3x
\end{equation}
Stability ↔ knot type (Hopf links, Trefoil, etc.) and energy minimization in the vortex core

\subsection*{II. Vortex Thread Interaction}
Interactions arise from exchange of vorticity or reconnections between vortex filaments:
\begin{itemize}
    \item Attractive if threads reinforce circulation (parallel)
    \item Repulsive if threads cancel (antiparallel)
    \item Interaction strength:
\end{itemize}
\begin{equation}
    \vec{F}_{\text{int}} = \beta \cdot \kappa_1 \kappa_2 \cdot \frac{\vec{r}_{12} \times (\vec{v}_1 - \vec{v}_2)}{|\vec{r}_{12}|^3}
\end{equation}
Where \(\kappa_i\) are circulations of filaments and \(\vec{r}_{12}\) is the vector between them.


\subsection*{III. Thermodynamic & Quantum Behavior from Vorticity Fluctuations}
\begin{itemize}
    \item Entropy \(\leftrightarrow\) volume of vortex expansion or knot deformation
    \item Quantum transitions \(\leftrightarrow\) topological reconnection events
    \item Zero-point motion \(\leftrightarrow\) background quantum turbulence of the Æther:
\end{itemize}
\subsection*{Quantum Vorticity Background}
\begin{equation}
    \langle \omega^2 \rangle \sim \frac{\hbar}{\rho_\text{æ} \xi^4}
\end{equation}
Where \(\xi\) is the coherence length between vortex filaments

\label{appendix:2}

\end{document}
%! Author = Omar Iskandarani
%! Date = 2/15/2025

\documentclass[aps,preprint,superscriptaddress]{revtex4}
\usepackage[none]{hyphenat}
\usepackage{array}
\usepackage{booktabs}
\usepackage{amsmath}
\usepackage{amssymb}
\usepackage{graphicx}
\usepackage{hyperref}
\usepackage{physics}

\begin{document}
\sloppy % Allow LaTeX to adjust spacing to avoid overfull boxes
\author{Omar Iskandarani}
\title{The Vortex Æther Model: Æther Vortex Field Model}
\date{\today}
\affiliation{Independent Researcher, Groningen, The Netherlands}
\thanks{ORCID: \href{https://orcid.org/0009-0006-1686-3961}{0009-0006-1686-3961}}
\email{info@omariskandarani.com}


%%%%%%%%%%%%%%%%%%%%%%%%%%    Abstract    %%%%%%%%%%%%%%%%%%%%%%%%%%


\begin{abstract}
    The Vortex Æther Model (VAM) introduces a unified, non-relativistic framework wherein gravity, electromagnetism, and quantum phenomena emerge from structured vorticity in an inviscid superfluid-like æther. Unlike General Relativity, which invokes spacetime curvature, VAM models stable vortex knots in a 3D Euclidean medium with absolute time. Observed time dilation results from vortex-induced local energy gradients. This paper derives time dilation analogs to GR, explores vortex-energetic time shifts, and presents experimental implications.
\end{abstract}


\maketitle
%%%%%%%%%%%%%%%%%%%%%%%%%%    Introduction    %%%%%%%%%%%%%%%%%%%%%%%%%%
    \section*{Core Assumptions}
    The Æther is modeled as an inviscid, incompressible superfluid governed by:

\begin{tabular}{ll}
    \toprule
    \midrule
        * & Conservation of Absolute Vorticity \\
        * & A 3D Euclidean medium with absolute time \\
        * & Particles as vortex knots \\
        * & Irrotational outside vortex cores, but with conserved vorticity inside knots \\
        * & Gravity from vorticity-induced pressure gradients \\
    \bottomrule
\end{tabular}


    \begin{tabular}{ll}
        \toprule
        Symbol & Description \\
        \midrule
        \(\vec{v}\) & Æther velocity field \\
        \(\vec{\omega}\) &  Vorticity \(\vec{\omega} = \nabla \times \vec{v}\) \\
        \(\rho_\text{æ}\) & Æther density (constant) \\
        \(\Phi\) & Vorticity-induced potential \\
        \(\kappa\) & Circulation constant \\
        \(\mathcal{K}\) & Knot topological class (Hopf link, torus knot, etc.) \\
        \bottomrule
    \end{tabular}

    \subsection*{Introduction to Fluid Dynamics and Vorticity Conservation}
    Euler Equation (Inviscid Flow)
    \begin{equation}
        \frac{\partial \vec{v}}{\partial t} + (\vec{v} \cdot \nabla)\vec{v} = -\frac{1}{\rho_\text{æ}} \nabla p
    \end{equation}
    Taking the curl to get the Vorticity Transport
    \begin{equation}
        \frac{\partial \vec{\omega}}{\partial t} + (\vec{v} \cdot \nabla)\vec{\omega} = (\vec{\omega} \cdot \nabla) \vec{v}
    \end{equation}

    \subsection*{Vorticity-Induced Gravity}
    We define a Newtonian like vorticity-based gravitational potential $\Phi$:
    \begin{equation}
        \vec{F}_g = -\nabla \Phi
    \end{equation}
    Where $\Phi$ is the Vorticity Potential:
    \begin{equation}
        \Phi(\vec{r}) = \gamma \int \frac{|\vec{\omega}(\vec{r'})|^2}{|\vec{r} - \vec{r'}|} \, d^3r'
    \end{equation}
    This mirrors the Newtonian potential but replaces mass density with vorticity intensity. This gives attractive force fields between vortex knots (like a particle).


    %%%%%%%%%%%%%%%%%%%%%%%%%%    PART 1    %%%%%%%%%%%%%%%%%%%%%%%%%%
    %! Author = mr
%! Date = 3/27/2025

\section{Time Dilation in the Æther-Vortex Model}\label{sec:Part-1}
We consider an inviscid, irrotational superfluid æther with stable topological vortex knots. The Æther experiences absolute time $t_{\text{abs}}$, but local clocks experience slowed rates due to pressure gradients and knot energetics. The Vortex Æther Model posits that the rate at which time flows in the local frame (near the knot) depends on the internal angular frequency $\Omega_k$. In this section, we derive time dilation analogues inspired by the predictions of general relativity (GR), based solely on pressure and vorticity gradients in the fluid.

\subsection{Bernoulli-Based Local Time Modulation}
In high-vorticity regions, Bernoulli's principle implies a drop in pressure near vortex cores:
\begin{equation}
    \frac{1}{2} \rho_\text{æ} v^2 + p = p_0 \Rightarrow p = p_0 - \frac{1}{2} \rho_\text{æ} v^2
\end{equation}
Assuming that the local physical clock rate is proportional to pressure, we define the local frequency of time as:
\begin{equation}
    f_{\text{local}} = f_0 \cdot \left( \frac{p}{p_0} \right) = f_0 \left( 1 - \frac{\rho_\text{æ} v^2}{2p_0} \right)
\end{equation}
Thus, the dilation of local time relative to absolute (background) time becomes:
\begin{equation}
    \frac{t_{\text{local}}}{t_0} = \left( 1 - \frac{\rho_\text{æ} v^2}{2p_0} \right)^{-1}
\end{equation}
For circular vortex flow where $v = \Omega r$:
\begin{equation}
    \frac{t_{\text{local}}}{t_0} \approx 1 + \frac{\rho_\text{æ} \Omega^2 r^2}{2p_0}
\end{equation}
This reduces time rate locally with higher knot rotation, modeling time modulation without relativity, where $\rho_\text{æ}/p_0 \sim 1/c^2$.

\subsection{Heuristic Time Modulation by Knot Rotation}
Let $\Omega_k$ be the average angular velocity of a vortex knot:
\begin{equation}
    \label{eq:omegamod}
    \frac{t_{\text{local}}}{t_{\text{abs}}} = \left(1 + \alpha \Omega_k^2 \right)^{-1}
\end{equation}
For small $\Omega_k$:
\begin{equation}
    \frac{t_{\text{local}}}{t_{\text{abs}}} \approx 1 - \alpha \Omega_k^2 + \mathcal{O}(\Omega_k^4)
\end{equation}
This matches special relativistic time dilation:
\begin{equation}
    \frac{t_{\text{moving}}}{t_{\text{rest}}} \approx 1 - \frac{v^2}{2c^2}
\end{equation}
This heuristic expression generalizes the local effect of rotational energy into a topological time-modulation law, consistent with the Bernoulli-derived expansion in the low-vorticity limit.

\subsection{Vorticity-Induced Gravitational Time Dilation}
In the æther-vortex framework, gravity emerges from pressure gradients induced by localized vorticity.
Let $\Phi(\vec{r})$ be a scalar potential analogous to the Newtonian gravitational potential, defined by:
\begin{equation}
    \Phi(\vec{r}) = \gamma \int \frac{|\vec{\omega}(\vec{r}')|^2}{|\vec{r} - \vec{r}'|} \, d^3r'
\end{equation}
Time dilation follows:
\begin{equation}
    \frac{t_{\text{local}}}{t_{\infty}} = \sqrt{1 - \frac{2\Phi(\vec{r})}{c^2}}
\end{equation}
High vorticity → low pressure → slowed time
If the vorticity field is concentrated at a point-like knot (analogous to a mass), i.e., $|\vec{\omega}(\vec{r}')|^2 \sim \delta(\vec{r}')$, we retrieve a Newtonian-like form:
\begin{equation}
    \frac{t_{\text{local}}}{t_{\infty}} = \sqrt{1 - \frac{2\gamma \omega_0^2}{r c^2}}
\end{equation}

\subsection{Kerr-Like Time Adjustment}
From GR:
\begin{equation}
    t_{\text{adjusted}} = \Delta t \cdot \sqrt{1 - \frac{2GM}{rc^2} - \frac{J^2}{r^3 c^2}}
\end{equation}
In Æther terms:
\begin{equation}
    \boxed{t_{\text{adjusted}} = \Delta t \cdot \sqrt{1 - \frac{\gamma \langle \omega^2 \rangle}{r c^2} - \frac{\kappa^2}{r^3 c^2}}}
\end{equation}
This reproduces gravitational and frame-dragging effects via vorticity and circulation.

\subsection{Conclusion and Experimental Outlook}
The Vortex Æther Model replaces spacetime curvature with conserved vorticity in a 3D fluid. Time dilation arises from localized pressure depletion and kinetic energy storage within vortex knots, offering a classical, topological reinterpretation of relativistic effects. In this formulation, time slows in regions of high vorticity due to pressure depletion, aligning with relativistic predictions \cite{fedi2017gravity, simula2020gravitational, winterberg1990maxwell}. Future work includes simulations of vortex clocks and tests using BECs, helium II, or electrohydrodynamic lifters.



    %%%%%%%%%%%%%%%%%%%%%%%%%%    PART 2    %%%%%%%%%%%%%%%%%%%%%%%%%%
    \subsection*{Section II: Time Modulation by Vortex Knot Rotation}

Building upon the previous section's treatment of time dilation via pressure and Bernoulli dynamics, we now focus on the intrinsic rotation of topological vortex knots. In the Vortex Æther Model (VAM), particles are modeled as stable, topologically conserved vortex knots embedded in an incompressible, inviscid superfluid medium. Each knot possesses a characteristic internal angular frequency $\Omega_k$, and this internal motion induces local time modulation relative to the absolute time of the æther.

Rather than curving spacetime, we propose that internal rotational energy and helicity conservation induce temporal slowdowns analogous to gravitational redshift. This section develops these ideas through heuristic and energetic arguments consistent with the hierarchy introduced in Section I.

\subsection*{A. Heuristic and Energetic Derivation}

We begin by proposing a rotationally-induced time dilation formula based on the knot's internal angular frequency:

\begin{equation}
\frac{t_{\text{local}}}{t_{\text{abs}}} = \left(1 + \alpha \Omega_k^2 \right)^{-1}
\end{equation}

where:

\begin{itemize}
\item $t_{\text{local}}$ is the proper time near the knot,
\item $t_{\text{abs}}$ is the absolute time of the background æther,
\item $\Omega_k$ is the average core angular frequency,
\item $\alpha$ is a coupling coefficient with dimensions $[\alpha] = \text{s}^2$.
\end{itemize}

For small angular velocities, we obtain a first-order expansion:

\begin{equation}
\frac{t_{\text{local}}}{t_{\text{abs}}} \approx 1 - \alpha \Omega_k^2 + \mathcal{O}(\Omega_k^4)
\end{equation}

This form parallels the Lorentz factor at low velocities in special relativity:

\begin{equation}
\frac{t_{\text{moving}}}{t_{\text{rest}}} \approx 1 - \frac{v^2}{2c^2}
\end{equation}

This establishes an important analogy: internal rotational motion in VAM induces temporal slowing, similar to how translational velocity induces time dilation in SR.

To strengthen the physical foundation of this expression, we now relate time dilation to the energy stored in vortex rotation. Let the vortex knot have an effective moment of inertia $I$. Its rotational energy is given by:

\begin{equation}
E_{\text{rot}} = \frac{1}{2} I \Omega_k^2
\end{equation}

Assuming time slows due to this energy density, we write:

\begin{equation}
\frac{t_{\text{local}}}{t_{\text{abs}}} = \left(1 + \alpha E_{\text{rot}} \right)^{-1} = \left(1 + \frac{1}{2} \alpha I \Omega_k^2 \right)^{-1}
\end{equation}

This expression serves as the energetic analog of the pressure-based Bernoulli model from Section I. It supports the interpretation of vortex-induced time wells via energy storage rather than geometric deformation.

We highlight this key result with a boxed formulation:

\begin{equation}
\boxed{\frac{t_{\text{local}}}{t_{\text{abs}}} = \left(1 + \frac{1}{2} \alpha I \Omega_k^2 \right)^{-1}}
\end{equation}

\subsection*{B. Topological and Physical Justification}

Topological vortex knots are not only characterized by rotation but also by helicity:

\begin{equation}
H = \int \vec{v} \cdot \vec{\omega} \, d^3x
\end{equation}

Helicity is a conserved quantity in ideal (inviscid, incompressible) fluids, encoding the linkage and twisting of vortex lines. The rotational frequency $\Omega_k$ becomes a topologically meaningful indicator of the knot’s identity and dynamical state.

Higher $\Omega_k$ implies greater rotational energy and stronger localized pressure depletion, forming a "temporal well" in the æther. These wells naturally mimic gravitational redshift effects in curved spacetime, but arise here purely from classical fluid mechanics.

This model:

\begin{itemize}
\item Attributes time modulation to conserved, intrinsic rotational energy,
\item Requires no external reference frames (absolute æther time is universal),
\item Preserves temporal isotropy outside the vortex core,
\item Provides a natural replacement for GR's spacetime curvature.
\end{itemize}

Therefore, this vortex-energetic time dilation principle provides a powerful alternative to relativistic time modulation by anchoring all temporal effects in rotational energetics and topological invariants.

In the next section, we will show how these ideas reproduce metric-like behavior for rotating observers, including a direct fluid-mechanical analog to the Kerr metric of General Relativity.
    %%%%%%%%%%%%%%%%%%%%%%%%%%    PART 3    %%%%%%%%%%%%%%%%%%%%%%%%%%
    %! Author = mr
%! Date = 3/27/2025

\section{proper time for a rotating observer}\label{sec:Part-3}

In General Relativity, the flow of proper time for a rotating observer in a stationary, axisymmetric spacetime is given by
\begin{equation}
    \left(\frac{d\tau}{dt}\right)^2_{\text{GR}} = -\left[g_{tt} + 2 g_{t\phi}\Omega_{\text{eff}} + g_{\phi\phi}\Omega_{\text{eff}}^2\right],
    \label{eq:GRtime}
\end{equation}
where $\Omega_{\text{eff}}$ is the observer's angular velocity, and the metric components $g_{\mu\nu}$ describe spacetime curvature (e.g., Kerr geometry) \cite{misner1973gravitation}.

In a vortex-based Æther theory, we posit that time dilation arises not from spacetime curvature, but from the local motion of an inviscid superfluid medium. We associate metric-like effects with Æther flow variables:
\begin{align*}
    g_{tt} &\rightarrow -\left(1 - \frac{v_r^2}{c^2} \right), \\
    g_{t\phi} &\rightarrow -\frac{v_r v_\phi}{c^2}, \\
    g_{\phi\phi} &\rightarrow -\frac{v_\phi^2}{c^2} r^2,
\end{align*}
where $v_r$ and $v_\phi$ are the radial and tangential components of Æther velocity, and $v_\phi = r \Omega_k$, with $\Omega_k = \kappa / (2\pi r^2)$ representing the local vortex rotation rate \cite{fedi2017gravity, sbitnev2021quaternion}.

Substituting into the structure of Equation~\ref{eq:GRtime}, we obtain:
\begin{align}
    \left( \frac{d\tau}{dt} \right)^2_{\text{Æther}} &= 1 - \frac{v_r^2}{c^2} - 2\frac{v_r v_\phi}{c^2} - \frac{v_\phi^2}{c^2} \\
    &= 1 - \frac{1}{c^2}(v_r + v_\phi)^2 \\
    &= 1 - \frac{1}{c^2}(v_r + r \Omega_k)^2.
    \label{eq:timeflow}
\end{align}

This result mirrors the GR proper time flow structure, yet is entirely fluid-mechanical. It predicts the slowing of proper time near intense vortex structures due to Æther flow speeds approaching $c$, effectively creating a “time-well” analogous to gravitational redshift.

\subsection*{Kerr-Like Time Adjustment from Vorticity and Circulation}

The General Relativistic form of time adjustment near a rotating mass, such as in the Kerr geometry, is approximately
\begin{equation}
    t_{\text{adjusted}} = \Delta t \cdot \sqrt{1 - \frac{2GM}{rc^2} - \frac{J^2}{r^3 c^2}}.
    \label{eq:kerrtime}
\end{equation}

We now express this in Æther-vortex terms, replacing $M$ and $J$ with effective vorticity energy density and circulation:
\begin{itemize}
    \item Mass-energy term: $\displaystyle \frac{2GM}{rc^2} \rightarrow \frac{\gamma \langle \omega^2 \rangle}{rc^2}$,
    \item Angular momentum term: $\displaystyle \frac{J^2}{r^3 c^2} \rightarrow \frac{\kappa^2}{r^3 c^2}$.
\end{itemize}

Thus, the Æther-based analog becomes:
\begin{equation}
    \boxed{
        t_{\text{adjusted}} = \Delta t \cdot \sqrt{
            1 - \frac{\gamma \langle \omega^2 \rangle}{r c^2}
            - \frac{\kappa^2}{r^3 c^2}
        }
    }
    \label{eq:ae_kerr}
\end{equation}

This formulation reproduces gravitational and frame-dragging time effects purely from Æther dynamics: $\langle \omega^2 \rangle$ plays the role of gravitational redshift, and circulation $\kappa$ encodes rotational drag. This approach aligns with recent fluid-dynamic interpretations of gravity and time \cite{barcelo2011analogue}, \cite{fedi2017gravity}.

    %%%%%%%%%%%%%%%%%%%%%%%%%%    PART 4    %%%%%%%%%%%%%%%%%%%%%%%%%%
    \documentclass[11pt]{article}
\usepackage[margin=1in]{geometry}
\usepackage{amsmath,amssymb}
\usepackage{amsthm}
\usepackage{graphicx}
\usepackage{hyperref}

\begin{document}

\title{Foundations of Velocity Fields and Energies in a Vortex System: A Brief Article}
\author{(VAM Working Group)}
\date{\today}
\maketitle

\begin{abstract}
This article outlines theoretical foundations of vortical velocity fields and their associated energies,
including a distinction between self- and cross-energies, in the context of a generic vortex-based model.
We close with a derivation outline for the cross-energy term, highlighting its application in vortex dynamics
and fluid–structure interactions.
\end{abstract}

\section{Introduction}
Vortex dynamics are a core component of many fluid and plasma systems, including
tornado-like flows, knotted vortices in classical or superfluid turbulence, and various
complex topological fluid systems. A deeper understanding of the energy budgets
associated with these flows can shed light on processes like vortex stability, reconnection,
and global flow organization. We begin by motivating how velocity fields can be
decomposed so as to capture the total energy (i.e.\ self- plus cross-energy), and how
this approach helps track flows in both 2D and 3D.

\section{Foundations: Velocity Fields and Total (Self + Cross) Energy}
\label{sec:foundations}
In an incompressible fluid, the velocity field $\mathbf{u}(\mathbf{x}, t)$ is typically
governed by the Navier--Stokes or Euler equations. For inviscid analyses, the Euler
equations for incompressible flow read
\begin{equation}
  \frac{\partial \mathbf{u}}{\partial t} + (\mathbf{u} \cdot \nabla)\mathbf{u} = -\frac{1}{\rho}\nabla p,
  \quad \nabla \cdot \mathbf{u} = 0.
\end{equation}
We also consider the vorticity $\boldsymbol{\omega} = \nabla \times \mathbf{u}$,
which can be used to characterize vortex structures.

To understand the \emph{total} kinetic energy, we can split it as follows:
\begin{equation}
  E_{\text{total}} \;=\; E_{\text{self}} \;+\; E_{\text{cross}}.
\end{equation}
Here, $E_{\text{self}}$ is that portion of energy which each vortex or partial flow
element contributes independently (for instance, from local swirling motions), while
$E_{\text{cross}}$ encodes the contributions that arise from the interaction of different
vortical elements. In a multi-vortex scenario, such a decomposition helps isolate the
direct interaction between two (or more) vortex filaments or sheets.

\section{Momentum and Self-Energy Considerations}
\label{sec:momentum}
A starting point is to recall that for a single vortex of circulation $\Gamma$, with an
azimuthally symmetric core, the induced velocity is sometimes approximated by
classical results such as
\begin{equation}
   V \;=\; \frac{\Gamma}{4 \pi R}
   \bigl(\ln \tfrac{8 R}{a} - \beta \bigr),
\end{equation}
where $R$ is the main vortex loop radius, $a \ll R$ is a measure of core thickness,
and $\beta$ depends on details of the core model \cite{Saffman1992}. The
\emph{self-energy} associated with that vortex, $E_{\text{self}}$, can be cast in a
similar form that depends on $\ln(R/a)$, exemplifying how thin-core vortices'
energies scale with geometry.

In more general fluid or vortex-lattice models, we can track $E_{\text{self}}$ as the
sum of individual core energies. Further, the presence of multiple filaments modifies
the total energy by cross-terms of the velocity fields (the cross-energy). This
cross-energy often drives key phenomena such as vortex merging or the `recoil'
effects in wave--vortex interactions.

\section{Defining and Tracking Cross-Energy}
\label{sec:cross}
When multiple vortices (or partial velocity distributions) co-exist, the total velocity
field $\mathbf{u}$ can be superposed:
\begin{equation}
   \mathbf{u} \;=\; \mathbf{u}_1 \;+\;\mathbf{u}_2,
\end{equation}
where $\mathbf{u}_1$ and $\mathbf{u}_2$ come from distinct sub-systems. In that
scenario, the kinetic energy for a fluid volume $V$ is
\begin{align}
   E_{\text{total}} &= \frac{\rho}{2} \int_V \mathbf{u}^2 \,dV
   = \frac{\rho}{2} \int_V \bigl(\mathbf{u}_1 + \mathbf{u}_2 \bigr)^2\, dV \\
   &= \frac{\rho}{2} \int_V \mathbf{u}_1^2 \,dV \;+\;\frac{\rho}{2} \int_V \mathbf{u}_2^2 \, dV
   \;+\;\rho \int_V \mathbf{u}_1 \cdot \mathbf{u}_2 \, dV,
\end{align}
revealing an interaction or \emph{cross-energy} term
\begin{equation}
   E_{\text{cross}} \;=\; \rho \int_V \mathbf{u}_1 \cdot \mathbf{u}_2 \, dV.
   \label{eq:cross-term}
\end{equation}
Much of the interesting physics arises from \eqref{eq:cross-term}, because it
grows or shrinks depending on the vortex geometry and distance between them.
Its dynamical evolution can lead to, e.g., merging or rebound. A main point is that
each vortex's self-velocity can significantly affect the mutual velocities and thus
create net forces or torque.

\section{Applications to Helicity and Topological Flows}
\label{sec:helicity}
A related concept is helicity, measuring the topological complexity (knotting or
linking) of vortex tubes. Classically, helicity $H$ is given by
\begin{equation}
   H \;=\; \int_V \mathbf{u} \cdot \boldsymbol{\omega}\, dV,
\end{equation}
which can remain constant or be partially lost during reconnection events. In certain
dissipative flows, the cross-energy terms in \eqref{eq:cross-term} can influence
the effective rate of helicity change. Understanding $E_{\text{cross}}$ is important
for analyzing reconnection pathways in classical or superfluid turbulence.

\section{Derivation Outline for Cross-Energy}
\label{sec:derivation}
Finally, we provide a succinct outline for deriving the cross-energy expression.
Starting with the total velocity field $\mathbf{u} = \sum_{n=1}^N \mathbf{u}_n$
for $N$ vortex or partial velocity fields, the total kinetic energy is:
\begin{equation}
   E_{\text{total}}
   = \frac{\rho}{2} \int_V \left(\sum_{n=1}^N \mathbf{u}_n \right)^2 dV
   = \frac{\rho}{2} \sum_{n=1}^N \int_V \mathbf{u}_n^2 \, dV
      \;+\;\rho \sum_{n<m} \int_V \mathbf{u}_n \cdot \mathbf{u}_m \, dV.
\end{equation}
One obtains $N$ self-energy terms plus pairwise cross-energy integrals.
The cross-energy for a pair $(i,j)$ is:
\begin{equation}
   E_{\text{cross}}^{(ij)} \;=\; \rho \int_V \mathbf{u}_i \cdot \mathbf{u}_j \, dV.
\end{equation}
In practice, each $\mathbf{u}_n$ may be represented by known solutions of the
Stokes or potential flow equations, or from approximate solutions for vortex loops.
Then, either analytically or numerically, one obtains approximate cross-energies
that can be used in reduced models describing the evolution of multi-vortex
systems.

\section*{Conclusion}
We have surveyed how the total fluid kinetic energy in the presence of multiple
vortices can be split into self- and cross-energy terms. These cross-energy
contributions are crucial for understanding vortex merging, knotted vortex
untangling, or vortex–wave interactions in classical, superfluid, and plasma
flows. In addition, we have sketched a systematic derivation of cross-energy and
highlighted key aspects in discussing momentum and helicity. Future directions
include refining these expressions for axisymmetric or knotted vortices and
integrating them into large-scale models or computational frameworks.

\begin{thebibliography}{99}

\bibitem{Saffman1992}
P.~G. Saffman,
\textit{Vortex Dynamics},
Cambridge University Press, 1992.

\bibitem{BabinskyStevens2016}
H. Babinsky and R. Stevens,
Low-order modeling approaches for unsteady aero flows,
\textit{Exp. Fluids}, \textbf{57} (2016), 71--85.

\bibitem{AndreuAngulo2020}
I. Andreu-Angulo, C. Manzo, B. Basu and H. Babinsky,
On unsteady aerodynamic effect of strong wind gusts on small-scale rotor systems,
\textit{AIAA J.}, \textbf{58} (2020), 5041--5056.

\bibitem{Wu1981}
J.-Z. Wu,
Theory of Vorticity and Vortex Dynamics,
Springer, 1981.

\bibitem{LiWu2018}
M. Li and J.-Z. Wu,
Generalized vortex force maps for the unsteady lift,
\textit{Theor. Comput. Fluid Dyn.} \textbf{32} (2018), 695--710.

\bibitem{KlecknerIrvine2013}
D. Kleckner and W. T. M. Irvine,
Creation and dynamics of knotted vortices,
\textit{Nature Phys.}, \textbf{9} (2013), 253--258.

\end{thebibliography}

\end{document}
    %%%%%%%%%%%%%%%%%%%%%%%%%%    PART 5    %%%%%%%%%%%%%%%%%%%%%%%%%%
    \section{Unified Framework and Synthesis of Time Dilation in VAM}

We now consolidate the various time dilation mechanisms explored throughout this manuscript into a unified framework under the Vortex Æther Model (VAM). By moving beyond geometric spacetime curvature, VAM provides a consistent and physically motivated model for temporal modulation grounded in classical fluid dynamics, rotational energetics, and topological vorticity structures.

\subsection*{A. Hierarchical Structure of Time Dilation Mechanisms}

Each section of this work contributes a distinct yet interrelated mechanism for time dilation:

\begin{enumerate}
    \item \textbf{Bernoulli-Induced Time Depletion:} Time slows near regions of low pressure resulting from vortex-induced kinetic velocity fields. This recovers a special relativistic time dilation form when \( \rho_\ae / p_0 \sim 1/c^2 \).

    \item \textbf{Angular Frequency Heuristic Model:} A quadratic dependence of time rate on local knot angular frequency \( \Omega_k^2 \), mimicking the Lorentz factor expansion for small velocities.

    \item \textbf{Energetic Formulation via Rotational Inertia:}
    \[
        \boxed{\frac{t_{\text{local}}}{t_{\text{abs}}} = \left(1 + \frac{1}{2} \alpha I \Omega_k^2 \right)^{-1}}
    \]
    links time modulation directly to the rotational energy of vortex knots.

    \item \textbf{Velocity-Field Based Proper Time Flow:}
    \[
        \boxed{\left( \frac{d\tau}{dt} \right)^2 = 1 - \frac{1}{c^2}(v_r + r\Omega_k)^2}
    \]

    \item \textbf{Kerr-Like Redshift and Frame-Dragging:}
    \[
        \boxed{t_{\text{adjusted}} = \Delta t \cdot \sqrt{1 - \frac{\gamma \langle \omega^2 \rangle}{rc^2} - \frac{\kappa^2}{r^3c^2}}}
    \]
\end{enumerate}

These five expressions form a self-consistent ladder, ranging from heuristic to rigorous, and establish a robust replacement for general relativistic time dilation based entirely on classical field variables.

\subsection*{B. Physical Unification: Time as a Vorticity-Derived Observable}

Across all formulations, a recurring theme emerges: \textit{time modulation in VAM is always reducible to local kinetic or rotational energy density within the æther}. Whether encoded in pressure (Bernoulli), angular frequency (\( \Omega_k \)), or field circulation (\( \kappa \)), the modulation of time is not geometric but energetic and topological.

\begin{itemize}
    \item Local Time Wells form due to high vorticity and circulation.
    \item Frame-Independence: Absolute time exists; only local rates are affected.
    \item No Need for Tensor Geometry: All time effects arise from scalar or vector fields.
    \item Topological Conservation: Vortex knots preserve helicity and circulation, ensuring temporal consistency.
\end{itemize}

This unification reinforces VAM’s conceptual core: \textbf{spacetime curvature is an emergent illusion produced by structured vorticity in an absolute, superfluid æther}.

\subsection*{C. Experimental Implications and Outlook}

Each time dilation formula introduced here can, in principle, be tested in laboratory analog systems:
\begin{itemize}
    \item Rotating superfluid droplets (e.g., Helium-II, BECs)
    \item Electrohydrodynamic lifters and plasma vortex systems
    \item Magneto-fluidic and optical analogs
\end{itemize}

Future work includes:
\begin{itemize}
    \item Deriving dynamic equations for temporal feedback in multi-knot systems.
    \item Measuring vortex-induced clock drift in rotating superfluids.
    \item Applying the model to astrophysical observations (e.g., neutron star precession, frame dragging, time delay).
\end{itemize}

\subsection*{D. Challenges, Limitations, and Paths to Broader Relevance}

\textbf{Foundational Assumptions:} The reintroduction of an æther with absolute time challenges a century of relativistic physics.

\textbf{Experimental Validation:} No direct empirical evidence yet supports the æther or specific dilation mechanisms proposed.

\textbf{Reception in Mainstream Physics:} While niche communities may engage, mainstream physics may resist due to divergence from established frameworks.

\subsection*{E. Enhancing Scientific Rigor and Broader Appeal}

\begin{itemize}
    \item \textbf{Propose Testable Predictions:} especially where VAM diverges from GR.
    \item \textbf{Integrate with Established Theories:} show limiting cases that match GR/QM.
    \item \textbf{Address Historical Objections:} clearly redefine æther with modern constraints.
    \item \textbf{Peer Review and Collaboration:} invite critique from specialists.
    \item \textbf{Clarity and Accessibility:} simplify conceptual presentation without sacrificing rigor.
\end{itemize}

\subsection*{F. Concluding Perspective}

The Vortex Æther Model (VAM) offers a bold reimagining of gravitational time dilation as a consequence of vorticity-driven energetics in an absolute, superfluid medium. Through a hierarchy of derivations—spanning Bernoulli flows, vortex rotation, energy density, and circulation—it establishes a coherent alternative to relativistic curvature-based descriptions. While its foundational assumptions challenge conventional paradigms, the internal consistency, experimental plausibility, and conceptual elegance of VAM make it a compelling framework worthy of further exploration. Continued refinement, integration, and empirical testing will determine its role in advancing our understanding of gravity, time, and the fabric of the universe.


    %%%%%%%%%%%%%%%%%%%%%%%%%%    References    %%%%%%%%%%%%%%%%%%%%%%%%%%

    \bibliography{citations}
    \bibliographystyle{apsrev4-2}


%%%%%%%%%%%%%%%%%%%%%%%%%%    Appendices    %%%%%%%%%%%%%%%%%%%%%%%%%%

    \appendix \label{sec:Part-6}
    \section{ Foundations of Velocity Fields and Energies in a Vortex System.}

\subsection{abstract}
This article outlines theoretical foundations of vortical velocity fields and their associated energies,
including a distinction between self- and cross-energies, in the context of a generic vortex-based model.
We close with a derivation outline for the cross-energy term, highlighting its application in vortex dynamics
and fluid–structure interactions.

\subsection{Introduction}
Vortex dynamics are a core component of many fluid and plasma systems, including
tornado-like flows, knotted vortices in classical or superfluid turbulence, and various
complex topological fluid systems. A deeper understanding of the energy budgets
associated with these flows can shed light on processes like vortex stability, reconnection,
and global flow organization. We begin by motivating how velocity fields can be
decomposed so as to capture the total energy (i.e.\ self- plus cross-energy), and how
this approach helps track flows in both 2D and 3D.

\subsection{Foundations: Velocity Fields and Total (Self + Cross) Energy}
\label{sec:foundations}
In an incompressible fluid, the velocity field $\mathbf{u}(\mathbf{x}, t)$ is typically
governed by the Navier--Stokes or Euler equations. For inviscid analyses, the Euler
equations for incompressible flow read
\begin{equation}
  \frac{\partial \mathbf{u}}{\partial t} + (\mathbf{u} \cdot \nabla)\mathbf{u} = -\frac{1}{\rho}\nabla p,
  \quad \nabla \cdot \mathbf{u} = 0.
\end{equation}
We also consider the vorticity $\boldsymbol{\omega} = \nabla \times \mathbf{u}$,
which can be used to characterize vortex structures.

To understand the \emph{total} kinetic energy, we can split it as follows:
\begin{equation}
  E_{\text{total}} \;=\; E_{\text{self}} \;+\; E_{\text{cross}}.
\end{equation}
Here, $E_{\text{self}}$ is that portion of energy which each vortex or partial flow
element contributes independently (for instance, from local swirling motions), while
$E_{\text{cross}}$ encodes the contributions that arise from the interaction of different
vortical elements. In a multi-vortex scenario, such a decomposition helps isolate the
direct interaction between two (or more) vortex filaments or sheets.

\subsection{Momentum and Self-Energy Considerations}
\label{sec:momentum}
A starting point is to recall that for a single vortex of circulation $\Gamma$, with an
azimuthally symmetric core, the induced velocity is sometimes approximated by
classical results such as
\begin{equation}
   V \;=\; \frac{\Gamma}{4 \pi R}
   \bigl(\ln \tfrac{8 R}{a} - \beta \bigr),
\end{equation}
where $R$ is the main vortex loop radius, $a \ll R$ is a measure of core thickness,
and $\beta$ depends on details of the core model \cite{Saffman1992}. The
\emph{self-energy} associated with that vortex, $E_{\text{self}}$, can be cast in a
similar form that depends on $\ln(R/a)$, exemplifying how thin-core vortices'
energies scale with geometry.

In more general fluid or vortex-lattice models, we can track $E_{\text{self}}$ as the
sum of individual core energies. Further, the presence of multiple filaments modifies
the total energy by cross-terms of the velocity fields (the cross-energy). This
cross-energy often drives key phenomena such as vortex merging or the `recoil'
effects in wave--vortex interactions.

\subsection{Defining and Tracking Cross-Energy}
\label{sec:cross}
When multiple vortices (or partial velocity distributions) co-exist, the total velocity
field $\mathbf{u}$ can be superposed:
\begin{equation}
   \mathbf{u} \;=\; \mathbf{u}_1 \;+\;\mathbf{u}_2,
\end{equation}
where $\mathbf{u}_1$ and $\mathbf{u}_2$ come from distinct sub-systems. In that
scenario, the kinetic energy for a fluid volume $V$ is
\begin{align}
   E_{\text{total}} &= \frac{\rho}{2} \int_V \mathbf{u}^2 \,dV
   = \frac{\rho}{2} \int_V \bigl(\mathbf{u}_1 + \mathbf{u}_2 \bigr)^2\, dV \\
   &= \frac{\rho}{2} \int_V \mathbf{u}_1^2 \,dV \;+\;\frac{\rho}{2} \int_V \mathbf{u}_2^2 \, dV
   \;+\;\rho \int_V \mathbf{u}_1 \cdot \mathbf{u}_2 \, dV,
\end{align}
revealing an interaction or \emph{cross-energy} term
\begin{equation}
   E_{\text{cross}} \;=\; \rho \int_V \mathbf{u}_1 \cdot \mathbf{u}_2 \, dV.
   \label{eq:cross-term}
\end{equation}
Much of the interesting physics arises from \eqref{eq:cross-term}, because it
grows or shrinks depending on the vortex geometry and distance between them.
Its dynamical evolution can lead to, e.g., merging or rebound. A main point is that
each vortex's self-velocity can significantly affect the mutual velocities and thus
create net forces or torque.

\subsection{Applications to Helicity and Topological Flows}
\label{sec:helicity}
A related concept is helicity, measuring the topological complexity (knotting or
linking) of vortex tubes. Classically, helicity $H$ is given by
\begin{equation}
   H \;=\; \int_V \mathbf{u} \cdot \boldsymbol{\omega}\, dV,
\end{equation}
which can remain constant or be partially lost during reconnection events. In certain
dissipative flows, the cross-energy terms in \eqref{eq:cross-term} can influence
the effective rate of helicity change. Understanding $E_{\text{cross}}$ is important
for analyzing reconnection pathways in classical or superfluid turbulence.

\subsection{Derivation Outline for Cross-Energy}
\label{sec:derivation}
Finally, we provide a succinct outline for deriving the cross-energy expression.
Starting with the total velocity field $\mathbf{u} = \sum_{n=1}^N \mathbf{u}_n$
for $N$ vortex or partial velocity fields, the total kinetic energy is:
\begin{equation}
   E_{\text{total}}
   = \frac{\rho}{2} \int_V \left(\sum_{n=1}^N \mathbf{u}_n \right)^2 dV
   = \frac{\rho}{2} \sum_{n=1}^N \int_V \mathbf{u}_n^2 \, dV
      \;+\;\rho \sum_{n<m} \int_V \mathbf{u}_n \cdot \mathbf{u}_m \, dV.
\end{equation}
One obtains $N$ self-energy terms plus pairwise cross-energy integrals.
The cross-energy for a pair $(i,j)$ is:
\begin{equation}
   E_{\text{cross}}^{(ij)} \;=\; \rho \int_V \mathbf{u}_i \cdot \mathbf{u}_j \, dV.
\end{equation}
In practice, each $\mathbf{u}_n$ may be represented by known solutions of the
Stokes or potential flow equations, or from approximate solutions for vortex loops.
Then, either analytically or numerically, one obtains approximate cross-energies
that can be used in reduced models describing the evolution of multi-vortex
systems.

\subsection*{Conclusion}
We have surveyed how the total fluid kinetic energy in the presence of multiple
vortices can be split into self- and cross-energy terms. These cross-energy
contributions are crucial for understanding vortex merging, knotted vortex
untangling, or vortex–wave interactions in classical, superfluid, and plasma
flows. In addition, we have sketched a systematic derivation of cross-energy and
highlighted key aspects in discussing momentum and helicity. Future directions
include refining these expressions for axisymmetric or knotted vortices and
integrating them into large-scale models or computational frameworks.\label{appendix:1}
    

\section{Integration of Clausius' Heat Theory into the Vortex \AE ther Model (VAM)}

The integration of Clausius' Mechanical Theory of Heat into the Vortex \AE ther Model (VAM) extends the framework's reach into thermodynamics,
allowing a unified interpretation of energy, entropy, and quantum behavior based on structured vorticity in an inviscid superfluid-like \ae ther
medium \cite{clausius1865mechanical, maxwell1865electromagnetic, helmholtz1858integrals}.

\subsection{Thermodynamic First Principles in VAM}

The classical first law of thermodynamics is expressed as:
\begin{equation}
\Delta U = Q - W,
\end{equation}
where $\Delta U$ is the change in internal energy, $Q$ is heat added, and $W$ is work done by the system \cite{clausius1865mechanical}. Within VAM, this becomes:
\begin{equation}
\Delta U = \Delta \left( \frac{1}{2} \rho_{\text{\ae}} \int v^2 \, dV + \int P \, dV \right),
\end{equation}
with $\rho_{\text{\ae}}$ the æther density, $v$ the local velocity, and $P$ the pressure within equilibrium vortex domains \cite{vam2025unified}.

\subsection{Entropy and Structured Vorticity}

VAM posits that entropy is a function of vorticity intensity:
\begin{equation}
S \propto \int \omega^2 \, dV,
\end{equation}
where $\omega = \nabla \times v$ \cite{kelvin1867vortex}. Thus, entropy becomes a measure of topological complexity and energy dispersion encoded in the vortex network.

\subsection{Thermal Response of Vortex Knots}

Stable vortex knots embedded in equilibrium pressure surfaces behave analogously to thermodynamic systems:
\begin{itemize}
\item \textbf{Heating ($Q > 0$)} expands the knot, lowers core pressure, and increases entropy.
\item \textbf{Cooling ($Q < 0$)} contracts the knot, concentrating energy and stabilizing vorticity.
\end{itemize}
This provides a fluid-mechanical analog to gas laws under energetic input.

\subsection{Photoelectric Analogy in VAM}

Rather than invoking quantized photons, VAM interprets the photoelectric effect through vortex dynamics. A vortex must absorb enough energy to destabilize and eject its structure:
\begin{equation}
W = \frac{1}{2} \rho_{\text{\ae}} \int v^2 \, dV + P_{\text{eq}} V_{\text{eq}},
\end{equation}
where $W$ is the disintegration work threshold. If an incident wave modulates internal vortex energy beyond this, ejection occurs \cite{vam2025unified}.

The critical force for vortex ejection is:
\begin{equation}
F_{\text{max}} = \rho_{\text{\ae}} C_e^2 \pi r_c^2,
\end{equation}
with $C_e$ the vortex's edge velocity and $r_c$ its core radius. This yields a natural frequency cutoff below which no interaction occurs, akin to the threshold frequency in quantum photoelectricity \cite{einstein1905photoelectric}.

\subsection{Conclusion and Integration}

This thermodynamic extension of VAM enriches the model by embedding classical heat and entropy principles within fluid-dynamic structures. It not only bridges vortex physics with Clausius' laws but also offers a field-based reinterpretation of light-matter interactions, unifying mechanical and electromagnetic thermodynamics without discrete particle assumptions.






\subsection*{I. Vortex Knots as Particles}
Each particle is a topological vortex knot:
\begin{itemize}
    \item Charge ↔ twist or chirality of knot
    \item Mass ↔ integrated vorticity energy
    \item Spin ↔ knot helicity:
\end{itemize}
\subsection*{Helicity as Particle Identity}
\begin{equation}
    \mathcal{H} = \int \vec{v} \cdot \vec{\omega} \, d^3x
\end{equation}
Stability ↔ knot type (Hopf links, Trefoil, etc.) and energy minimization in the vortex core

\subsection*{II. Vortex Thread Interaction}
Interactions arise from exchange of vorticity or reconnections between vortex filaments:
\begin{itemize}
    \item Attractive if threads reinforce circulation (parallel)
    \item Repulsive if threads cancel (antiparallel)
    \item Interaction strength:
\end{itemize}
\begin{equation}
    \vec{F}_{\text{int}} = \beta \cdot \kappa_1 \kappa_2 \cdot \frac{\vec{r}_{12} \times (\vec{v}_1 - \vec{v}_2)}{|\vec{r}_{12}|^3}
\end{equation}
Where \(\kappa_i\) are circulations of filaments and \(\vec{r}_{12}\) is the vector between them.


\subsection*{III. Thermodynamic & Quantum Behavior from Vorticity Fluctuations}
\begin{itemize}
    \item Entropy \(\leftrightarrow\) volume of vortex expansion or knot deformation
    \item Quantum transitions \(\leftrightarrow\) topological reconnection events
    \item Zero-point motion \(\leftrightarrow\) background quantum turbulence of the Æther:
\end{itemize}
\subsection*{Quantum Vorticity Background}
\begin{equation}
    \langle \omega^2 \rangle \sim \frac{\hbar}{\rho_\text{æ} \xi^4}
\end{equation}
Where \(\xi\) is the coherence length between vortex filaments\label{appendix:2}

\end{document}
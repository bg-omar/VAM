%! Author = Omar Iskandarani
%! Date = 2/15/2025

\documentclass[aps,preprint,superscriptaddress]{revtex4}

\usepackage{array}
\usepackage{booktabs}
\usepackage{amsmath}
\usepackage{amssymb}
\usepackage{graphicx}
\usepackage{hyperref}
\usepackage{physics}

\begin{document}
\sloppy % Allow LaTeX to adjust spacing to avoid overfull boxes
\author{Omar Iskandarani}
\title{The Vortex Æther Model: Æther Vortex Field Model}
\date{\today}
\affiliation{Independent Researcher, Groningen, The Netherlands}
\thanks{ORCID: \href{https://orcid.org/0009-0006-1686-3961}{0009-0006-1686-3961}}
\email{info@omariskandarani.com}


%%%%%%%%%%%%%%%%%%%%%%%%%%    Abstract    %%%%%%%%%%%%%%%%%%%%%%%%%%


\begin{abstract}
    The Vortex Æther Model (VAM) introduces a unified, non-relativistic framework wherein gravity, electromagnetism, and quantum phenomena emerge from structured vorticity in an inviscid superfluid-like æther. Unlike General Relativity, which invokes spacetime curvature, VAM models stable vortex knots in a 3D Euclidean medium with absolute time. Observed time dilation results from vortex-induced local energy gradients. This paper derives time dilation analogs to GR, explores vortex-energetic time shifts, and presents experimental implications.
\end{abstract}


\maketitle
%%%%%%%%%%%%%%%%%%%%%%%%%%    Introduction    %%%%%%%%%%%%%%%%%%%%%%%%%%
\newpage

    \begin{figure}[h]
        \includegraphics[width=\textwidth]{vorticity_field1}
        \caption{Illustration of vorticity fields in Æther.}
        \label{fig:vorticity_field1}
    \end{figure}

    \section*{Core Assumptions}
    The Æther is modeled as an inviscid, incompressible superfluid governed by:

\begin{tabular}{ll}
    \toprule
    \midrule
        * & Conservation of Absolute Vorticity \\
        * & A 3D Euclidean medium with absolute time \\
        * & Particles as vortex knots \\
        * & Irrotational outside vortex cores, but with conserved vorticity inside knots \\
        * & Gravity from vorticity-induced pressure gradients \\
    \bottomrule
\end{tabular}

    Let:

    \begin{tabular}{ll}
        \toprule
        Symbol & Description \\
        \midrule
        \(\vec{v}\) & Æther velocity field \\
        \(\vec{\omega}\) &  Vorticity \(\vec{\omega} = \nabla \times \vec{v}\) \\
        \(\rho_\text{æ}\) & Æther density (constant) \\
        \(\Phi\) & Vorticity-induced potential \\
        \(\kappa\) & Circulation constant \\
        \(\mathcal{K}\) & Knot topological class (Hopf link, torus knot, etc.) \\
        \bottomrule
    \end{tabular}

    \subsection*{Introduction to Fluid Dynamics and Vorticity Conservation}
    Euler Equation (Inviscid Flow)
    \begin{equation}
        \frac{\partial \vec{v}}{\partial t} + (\vec{v} \cdot \nabla)\vec{v} = -\frac{1}{\rho_\text{æ}} \nabla p
    \end{equation}
    Taking the curl to get the Vorticity Transport
    \begin{equation}
        \frac{\partial \vec{\omega}}{\partial t} + (\vec{v} \cdot \nabla)\vec{\omega} = (\vec{\omega} \cdot \nabla) \vec{v}
    \end{equation}

    \subsection*{Vorticity-Induced Gravity}
    We define a Newtonian like vorticity-based gravitational potential $\Phi$:
    \begin{equation}
        \vec{F}_g = -\nabla \Phi
    \end{equation}
    Where $\Phi$ is the Vorticity Potential:
    \begin{equation}
        \Phi(\vec{r}) = \gamma \int \frac{|\vec{\omega}(\vec{r'})|^2}{|\vec{r} - \vec{r'}|} \, d^3r'
    \end{equation}
    This mirrors the Newtonian potential but replaces mass density with vorticity intensity. This gives attractive force fields between vortex knots (like a particle).



%%%%%%%%%%%%%%%%%%%%%%%%%%    PART 1    %%%%%%%%%%%%%%%%%%%%%%%%%%
%! Author = mr
%! Date = 3/27/2025

\section{Time Dilation from Vortex Dynamics}\label{sec:Part-1}
We consider an inviscid, irrotational superfluid æther with stable topological vortex knots. The Æther experiences absolute time $t_{\text{abs}}$, but local clocks experience slowed rates due to pressure gradients and knot energetics. The Vortex Æther Model posits that the rate at which time flows in the local frame (near the knot) depends on the internal angular frequency $\Omega_k$. In this section, we derive time dilation analogues inspired by the predictions of general relativity (GR), based solely on pressure and vorticity gradients in the fluid.

\begin{figure}[h!]
    \centering
    \includegraphics[width=0.85\textwidth]{export/streamlinesDiPole}
    \caption{Velocity streamlines, vorticity, pressure, and local time rate $\tau$ for a simulated vortex pair. The pressure minimum and time slow-down clearly align with the regions of high vorticity. This directly illustrates the æther model's central claim: time dilation follows from vortex energetics and pressure depletion.}
    \label{fig:vortexfields}
\end{figure}

\subsection{Bernoulli \& Rotational Flow}
In high-vorticity zones, Bernoulli's principle implies a local drop in pressure:
\begin{equation}
    \frac{1}{2} \rho_\text{æ} v^2 + p = p_0 \quad \Rightarrow \quad p = p_0 - \frac{1}{2} \rho_\text{æ} v^2
\end{equation}
Assuming the local frequency of a clock is proportional to Æther pressure:
\begin{equation}
    f_{\text{local}} = f_0 \left(1 - \frac{\rho_\text{æ} v^2}{2p_0} \right)
\end{equation}
Thus, the time dilation becomes:
\begin{equation}
    \frac{t_{\text{local}}}{t_0} = \left(1 - \frac{\rho_\text{æ} v^2}{2p_0} \right)^{-1}
\end{equation}
For circular vortex motion with $v = \Omega r$:
\begin{equation}
    \frac{t_{\text{local}}}{t_0} \approx 1 + \frac{\rho_\text{æ} \Omega^2 r^2}{2p_0}
\end{equation}
This is analogous to special relativistic dilation if $\rho_\text{æ}/p_0 \sim 1/c^2$.

\begin{figure}[h!]
    \centering
    \includegraphics[width=0.8\textwidth]{export/RadialProfileOfLocalTimeDilation_Radial_LocalTime_Dilation}
    \caption{Radial profile of normalized local time $t_{\text{local}} / t_{\text{abs}}$ as a function of distance $r$ from the vortex core, assuming $\Omega_k \propto 1/r^2$. Time slows significantly near the vortex center and recovers to background values with distance.}
    \label{fig:radial_time_profile}
\end{figure}

\subsection{Heuristic Knot-Based Time Modulation}
Topological vortex knots possess internal rotation. Let $\Omega_k$ be their average angular velocity. We postulate a heuristic time modulation law:
\begin{equation}
    \label{eq:heuristic}
    \frac{t_{\text{local}}}{t_{\text{abs}}} = \left(1 + \alpha \Omega_k^2 \right)^{-1}
\end{equation}
where $\alpha$ is a coupling constant related to the Æther's compressibility or inertia. For small $\Omega_k$, we expand:
\begin{equation}
    \frac{t_{\text{local}}}{t_{\text{abs}}} \approx 1 - \alpha \Omega_k^2 + \mathcal{O}(\Omega_k^4)
\end{equation}
This expression parallels the Lorentz factor for low velocities:
\[
    \frac{t_{\text{moving}}}{t_{\text{rest}}} \approx 1 - \frac{v^2}{2c^2}
\]

\begin{figure}[h!]
    \centering
    \includegraphics[width=0.8\textwidth]{export/RadialProfileOfLocalTimeDilation_Vortex-Induced_Time_Well}
    \caption{Schematic of a vortex-induced time well in the æther. Local time $t_{\text{local}} / t_{\text{abs}}$ is shown as a color gradient in 2D space. The central vortex region exhibits the most time slowing due to high $\Omega_k$, forming a well-like structure.}
    \label{fig:vortex_time_well}
\end{figure}


\subsection{Energetic Interpretation: Rotational Inertia}
The above heuristic can be grounded in the rotational energy of vortex knots. Let $I$ be the effective moment of inertia of a knot, then the energy stored is:
\begin{equation}
    E_\text{rot} = \frac{1}{2} I \Omega_k^2
\end{equation}
Assuming the time dilation arises from this stored energy modifying local information rates:
\begin{equation}
    \frac{t_{\text{local}}}{t_{\text{abs}}} = \left(1 + \alpha I \Omega_k^2 \right)^{-1}
\end{equation}
This model provides a bridge between fluid rotation and gravitational-like time shifts, replacing the need for spacetime curvature with internal knot energetics. It suggests that time flow slows where circulation is topologically conserved.

\begin{equation}
    \boxed{
        t_{\text{local}} = \frac{t_{\text{abs}}}{1 + \alpha I \Omega_k^2}
    }
\end{equation}
This boxed result summarizes the time modulation law driven by rotational inertia.

\subsection{Conclusion and Experimental Outlook}
The Vortex Æther Model replaces spacetime curvature with conserved vorticity in a 3D fluid. Time dilation arises from localized pressure depletion and kinetic energy storage within vortex knots, offering a classical, topological reinterpretation of relativistic effects. In this formulation, time slows in regions of high vorticity due to pressure depletion, aligning with relativistic predictions \cite{fedi2017gravity, simula2020gravitational, winterberg1990maxwell}. Future work includes simulations of vortex clocks and tests using BECs, helium II, or electrohydrodynamic lifters.



%%%%%%%%%%%%%%%%%%%%%%%%%%    PART 2    %%%%%%%%%%%%%%%%%%%%%%%%%%
\subsection*{Section II: Time Modulation by Vortex Knot Rotation}

Building upon the previous section's treatment of time dilation via pressure and Bernoulli dynamics, we now focus on the intrinsic rotation of topological vortex knots. In the Vortex Æther Model (VAM), particles are modeled as stable, topologically conserved vortex knots embedded in an incompressible, inviscid superfluid medium. Each knot possesses a characteristic internal angular frequency $\Omega_k$, and this internal motion induces local time modulation relative to the absolute time of the æther.

Rather than curving spacetime, we propose that internal rotational energy and helicity conservation induce temporal slowdowns analogous to gravitational redshift. This section develops these ideas through heuristic and energetic arguments consistent with the hierarchy introduced in Section I.

\subsection*{A. Heuristic and Energetic Derivation}

We begin by proposing a rotationally-induced time dilation formula based on the knot's internal angular frequency:

\begin{equation}
\frac{t_{\text{local}}}{t_{\text{abs}}} = \left(1 + \alpha \Omega_k^2 \right)^{-1}
\end{equation}

where:

\begin{itemize}
\item $t_{\text{local}}$ is the proper time near the knot,
\item $t_{\text{abs}}$ is the absolute time of the background æther,
\item $\Omega_k$ is the average core angular frequency,
\item $\alpha$ is a coupling coefficient with dimensions $[\alpha] = \text{s}^2$.
\end{itemize}

For small angular velocities, we obtain a first-order expansion:

\begin{equation}
\frac{t_{\text{local}}}{t_{\text{abs}}} \approx 1 - \alpha \Omega_k^2 + \mathcal{O}(\Omega_k^4)
\end{equation}

This form parallels the Lorentz factor at low velocities in special relativity:

\begin{equation}
\frac{t_{\text{moving}}}{t_{\text{rest}}} \approx 1 - \frac{v^2}{2c^2}
\end{equation}

This establishes an important analogy: internal rotational motion in VAM induces temporal slowing, similar to how translational velocity induces time dilation in SR.

To strengthen the physical foundation of this expression, we now relate time dilation to the energy stored in vortex rotation. Let the vortex knot have an effective moment of inertia $I$. Its rotational energy is given by:

\begin{equation}
E_{\text{rot}} = \frac{1}{2} I \Omega_k^2
\end{equation}

Assuming time slows due to this energy density, we write:

\begin{equation}
\frac{t_{\text{local}}}{t_{\text{abs}}} = \left(1 + \alpha E_{\text{rot}} \right)^{-1} = \left(1 + \frac{1}{2} \alpha I \Omega_k^2 \right)^{-1}
\end{equation}

This expression serves as the energetic analog of the pressure-based Bernoulli model from Section I. It supports the interpretation of vortex-induced time wells via energy storage rather than geometric deformation.

We highlight this key result with a boxed formulation:

\begin{equation}
\boxed{\frac{t_{\text{local}}}{t_{\text{abs}}} = \left(1 + \frac{1}{2} \alpha I \Omega_k^2 \right)^{-1}}
\end{equation}

\subsection*{B. Topological and Physical Justification}

Topological vortex knots are not only characterized by rotation but also by helicity:

\begin{equation}
H = \int \vec{v} \cdot \vec{\omega} \, d^3x
\end{equation}

Helicity is a conserved quantity in ideal (inviscid, incompressible) fluids, encoding the linkage and twisting of vortex lines. The rotational frequency $\Omega_k$ becomes a topologically meaningful indicator of the knot’s identity and dynamical state.

Higher $\Omega_k$ implies greater rotational energy and stronger localized pressure depletion, forming a "temporal well" in the æther. These wells naturally mimic gravitational redshift effects in curved spacetime, but arise here purely from classical fluid mechanics.

This model:

\begin{itemize}
\item Attributes time modulation to conserved, intrinsic rotational energy,
\item Requires no external reference frames (absolute æther time is universal),
\item Preserves temporal isotropy outside the vortex core,
\item Provides a natural replacement for GR's spacetime curvature.
\end{itemize}

Therefore, this vortex-energetic time dilation principle provides a powerful alternative to relativistic time modulation by anchoring all temporal effects in rotational energetics and topological invariants.

In the next section, we will show how these ideas reproduce metric-like behavior for rotating observers, including a direct fluid-mechanical analog to the Kerr metric of General Relativity.
%%%%%%%%%%%%%%%%%%%%%%%%%%    PART 3    %%%%%%%%%%%%%%%%%%%%%%%%%%
\subsection*{Section III: Proper Time for a Rotating Observer in Æther Flow}

Having established time dilation in the Vortex Æther Model (VAM) through pressure, angular velocity, and rotational energy, we now extend our formalism to rotating observers. This section demonstrates that fluid-dynamic time modulation in VAM can reproduce expressions structurally similar to those derived in General Relativity (GR), particularly in axisymmetric rotating spacetimes like the Kerr geometry. However, VAM achieves this without invoking spacetime curvature. Instead, time modulation is governed entirely by kinetic variables in the æther field.

\subsection*{A. GR Proper Time in Rotating Frames}

In General Relativity, the proper time \(d\tau\) for an observer with angular velocity \(\Omega_{\text{eff}}\) in a stationary, axisymmetric spacetime is given by:

\begin{equation}
\left( \frac{d\tau}{dt} \right)^2_{\text{GR}} = -\left[ g_{tt} + 2g_{t\varphi} \Omega_{\text{eff}} + g_{\varphi\varphi} \Omega_{\text{eff}}^2 \right]
\tag{18}
\end{equation}

where \(g_{\mu\nu}\) are components of the spacetime metric (e.g., in Boyer–Lindquist coordinates for Kerr spacetime). This formulation accounts for both gravitational redshift and rotational (frame-dragging) effects.

\subsection*{B. Æther-Based Analog: Velocity-Derived Time Modulation}

In VAM, spacetime is not curved. Instead, observers reside within a dynamically structured æther whose local flow velocities determine time dilation. Let the radial and tangential components of æther velocity be:

\begin{itemize}
\item \(v_r\): radial velocity,
\item \(v_\varphi = r\Omega_k\): tangential velocity due to local vortex rotation,
\item \(\Omega_k = \frac{\kappa}{2\pi r^2}\): local angular velocity (with \(\kappa\) as circulation).
\end{itemize}

We postulate a correspondence between GR metric components and æther velocity terms:

\begin{equation}
\begin{aligned}
g_{tt} &\rightarrow -\left(1 - \frac{v_r^2}{c^2}\right), \\
g_{t\varphi} &\rightarrow -\frac{v_r v_\varphi}{c^2}, \\
g_{\varphi\varphi} &\rightarrow -\frac{v_\varphi^2}{c^2 r^2}
\end{aligned}
\tag{19}
\end{equation}

Substituting these into the GR expression for proper time, we obtain the VAM-based analog:

\begin{equation}
\left( \frac{d\tau}{dt} \right)^2_{\ae} = 1 - \frac{v_r^2}{c^2} - \frac{2v_r v_\varphi}{c^2} - \frac{v_\varphi^2}{c^2}
\tag{20}
\end{equation}

Combining the terms:

\begin{equation}
\left( \frac{d\tau}{dt} \right)^2_{\ae} = 1 - \frac{1}{c^2}(v_r + v_\varphi)^2
\tag{21}
\end{equation}

This formulation reproduces gravitational and frame-dragging time effects purely from Æther dynamics: $\langle \omega^2 \rangle$ plays the role of gravitational redshift, and circulation $\kappa$ encodes rotational drag. This approach aligns with recent fluid-dynamic interpretations of gravity and time \cite{barcelo2011analogue}, \cite{fedi2017gravity}.
This model currently assumes irrotational flow outside knots and neglects viscosity, turbulence, and quantum compressibility. Future extensions may include quantized circulation spectra or boundary effects in confined Æther systems.

\begin{equation}
\boxed{\left( \frac{d\tau}{dt} \right)^2_{\ae} = 1 - \frac{1}{c^2}(v_r + r\Omega_k)^2}
\tag{Æther-Based Proper Time for Rotating Observer}
\end{equation}

\subsection*{C. Physical Interpretation and Model Consistency}

This boxed result mirrors the GR expression for rotating observers but arises strictly from classical fluid dynamics. It shows that as the local æther speed approaches the speed of light—due to either radial inflow or rotational motion—the proper time slows. This implies the existence of "time wells" where kinetic energy density dominates.

Key observations:

\begin{itemize}
\item In the absence of radial flow (\(v_r = 0\)), time slowing arises entirely from vortex rotation.
\item When both \(v_r\) and \(\Omega_k\) are present, the cumulative velocity reduces local time rate.
\item This expression agrees with Section II's energetic model if we interpret \(v_r + r\Omega_k\) as contributing to the local energy density.
\end{itemize}

Thus, in the VAM framework, the structure of the observer’s proper time emerges from ætheric flow fields. This confirms that GR-like temporal behavior can emerge in a flat, Euclidean 3D space with absolute time, governed entirely by structured vorticity and circulation.

In the next section, we explore how VAM extends this correspondence to gravitational potentials and frame-dragging effects via circulation and vorticity intensity, forming an analog to the Kerr time redshift formula.


%%%%%%%%%%%%%%%%%%%%%%%%%%    References    %%%%%%%%%%%%%%%%%%%%%%%%%%

    \bibliography{citations}
    \bibliographystyle{apsrev4-2}



%%%%%%%%%%%%%%%%%%%%%%%%%%    Appendices    %%%%%%%%%%%%%%%%%%%%%%%%%%

    \appendix \label{sec:Part-4}
    \section{Appendix}

\subsection*{I. Vortex Knots as Particles}
Each particle is a topological vortex knot:
\begin{itemize}
    \item Charge ↔ twist or chirality of knot
    \item Mass ↔ integrated vorticity energy
    \item Spin ↔ knot helicity:
\end{itemize}
\subsection*{Helicity as Particle Identity}
\begin{equation}
    \mathcal{H} = \int \vec{v} \cdot \vec{\omega} \, d^3x
\end{equation}
Stability ↔ knot type (Hopf links, Trefoil, etc.) and energy minimization in the vortex core

\subsection*{II. Vortex Thread Interaction}
Interactions arise from exchange of vorticity or reconnections between vortex filaments:
\begin{itemize}
    \item Attractive if threads reinforce circulation (parallel)
    \item Repulsive if threads cancel (antiparallel)
    \item Interaction strength:
\end{itemize}
\begin{equation}
    \vec{F}_{\text{int}} = \beta \cdot \kappa_1 \kappa_2 \cdot \frac{\vec{r}_{12} \times (\vec{v}_1 - \vec{v}_2)}{|\vec{r}_{12}|^3}
\end{equation}
Where \(\kappa_i\) are circulations of filaments and \(\vec{r}_{12}\) is the vector between them.


\subsection*{III. Thermodynamic & Quantum Behavior from Vorticity Fluctuations}
\begin{itemize}
    \item Entropy \(\leftrightarrow\) volume of vortex expansion or knot deformation
    \item Quantum transitions \(\leftrightarrow\) topological reconnection events
    \item Zero-point motion \(\leftrightarrow\) background quantum turbulence of the Æther:
\end{itemize}
\subsection*{Quantum Vorticity Background}
\begin{equation}
    \langle \omega^2 \rangle \sim \frac{\hbar}{\rho_\text{æ} \xi^4}
\end{equation}
Where \(\xi\) is the coherence length between vortex filaments
\label{appendix:1}

\end{document}
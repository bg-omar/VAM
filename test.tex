\(\rho_\text{\ae}\), \(7 \times 10^{-7}\), \(\text{kg m}^{-3}\), \text{Æther density},
\(C_e\), \(1093845.63\), \(\text{m s}^{-1}\), \text{Vortex-Tangential-Velocity},
\(r_c\): \text{PhysicalConstant}(\(r_c\), \(1.40897017 \times 10^{-15}\), \(\text{m}\), \text{Vortex-Core radius}),
\(F_{\max}\): \text{PhysicalConstant}(\(F_{\max}\), \(29.053507\), \(\text{N}\), \text{Maximum Coulomb force}),
\[ G_\text{swirl} = G \]


\textbf{\chapter*{Understanding the Challenge}
\begin{itemize}
\item Your Model: A 3D Euclidean space with absolute time, where particles are represented as vortex knots in a superfluid Æther.

\item Objective: Exclude General Relativity as a theory but incorporate its laws (or principles) in terms of swirl or angular velocity of the vortex core interactions between vortex knots.

\item Notation Suggestion: Using \(x_1, x_2, x_3, x_\text{swirl}\) and \(y_1, y_2, y_3, y_\text{swirl}\) to represent positions and swirl or angular velocity of the vortex core components between vortex knots.
\end{itemize}

1. Identifying Key Principles of GR

While excluding GR as a theory, we can consider incorporating its essential principles:

\begin{itemize}
\item Gravitational Effects as Geometry: In GR, gravity is not a force but a manifestation of spacetime curvature caused by mass-energy.

\item Equivalence Principle: The effects of gravity are locally indistinguishable from acceleration.

\item Geodesic Motion: Objects move along paths determined by spacetime geometry.
\end{itemize}

2. Translating GR Concepts into Fluid Dynamics

a. Spacetime Curvature and swirl or angular velocity of the vortex core

\begin{itemize}
\item Analogy: Spacetime curvature in GR can be analogously represented by variations in swirl or angular velocity of the vortex core within the superfluid Æther.

\item Interpretation: Regions with different swirl or angular velocity of the vortex core levels cause particles (vortex knots) to experience "forces" analogous to gravitational attraction.
\end{itemize}

b. Mass-Energy and swirl or angular velocity of the vortex core Sources

\begin{itemize}
\item Mass-Energy Equivalence: In your model, the mass-energy of particles corresponds to the intensity or configuration of their vortex knots.

\item swirl or angular velocity of the vortex core as a Field: swirl or angular velocity of the vortex core variations between knots represent the "gravitational field" influencing the motion of other knots.
\end{itemize}

\subsection*{1. Understanding the Swirl of the Vortex Core}
\subsubsection*{Key Characteristics:}
\begin{itemize}
\item Swirl velocity \(C_e\): The tangential velocity at the edge of the vortex core.

\item Core radius \(r_c\): The radius of the vortex core, where the majority of the angular momentum is concentrated.

\item Core angular velocity \(\omega_c\): Related to the tangential velocity and radius as: \(\omega_c = \frac{C_e}{r_c}\)
\end{itemize}

\subsection*{Replacing Vorticity:}
Instead of using vorticity \(\vec{\omega} = \nabla \times \vec{u}\), we define gravitational effects in terms of the swirl dynamics of the vortex core:

\begin{itemize}
\item Swirl angular velocity \(\omega_c\),

\item Swirl energy density \(\rho_\text{swirl} \sim \frac{1}{2} \rho C_e^2\).
\end{itemize}
This substitution highlights the localized dynamics of the vortex core instead of broader fluid vorticity.

\subsection*{2. Formulating the Swirl-Based Gravitational Potential}
\subsubsection*{Swirl Potential:}
We now redefine the vorticity potential \(\phi_\text{vort}\) as the swirl potential \(\phi_\text{swirl}\), analogous to the gravitational potential:

\(\phi_\text{swirl} = -\frac{G_\text{swirl} m}{r}\)

where:

\begin{itemize}
\item \(G_\text{swirl}\) is a gravitational-like constant related to the vortex properties,

\item \(m\) is the mass of the vortex knot (or core),

\item \(r\) is the radial distance.
\end{itemize}
\subsubsection*{Gravitational Field from Swirl:}
The gravitational-like field is derived from the potential:

\(\vec{F}_\text{swirl} = -\nabla \phi_\text{swirl} = \frac{G_\text{swirl} m}{r^2} \hat{r}\)

\subsection*{3. Equation of Motion for Vortex Knots}
Using the swirl potential \(\phi_\text{swirl}\), the equation of motion for a vortex knot becomes:

\[m \frac{d^2 \vec{x}}{dt^2} = -m \nabla \phi_\text{swirl}\]

Substituting \(\phi_\text{swirl} = -\frac{G_\text{swirl} m}{r}\)

\[\frac{d^2 \vec{x}}{dt^2} = -\frac{G_\text{swirl} m}{r^2} \hat{r}\]

This is analogous to Newtonian gravity, but the constant \(G_\text{swirl}\) depends on the angular velocity \(\omega_c\) or tangential velocity \(C_e\).

\subsection*{4. Relating Swirl and Gravitational Interaction}
We now connect the swirl-based interactions to gravitational effects.

\subsubsection*{Swirl Energy Density:}
The energy density associated with the vortex core is:

\[\rho_\text{swirl} = \frac{1}{2} \rho C_e^2 = \frac{1}{2} \rho \omega_c^2 r_c^2\]

This energy density plays the role of mass-energy density in GR.

\subsubsection*{Swirl-Based Field Equation:}
Analogous to Poisson’s equation for gravity, we define the field equation for the swirl potential:

\[\nabla^2 \phi_\text{swirl} = -\rho_\text{swirl}\]

where \(\rho_\text{swirl}\) is the density of swirl energy.

\subsection*{5. Redefining Metric in Vortex Core Framework}
\subsubsection*{Extended Coordinates:}
We use coordinates \((x_1, x_2, x_3, x_\text{swirl})\) to include spatial positions and swirl contributions.

\subsubsection*{Metric Tensor:}
Define a swirl-based metric tensor:

\[ds^2 = dx_1^2 + dx_2^2 + dx_3^2 + f(\omega_c) d\omega_c^2\]

where \(f(\omega_c)\) is a function that quantifies the contribution of swirl angular velocity to the metric.

\subsection*{6. Curvature in Swirl Framework}
We define curvature in terms of variations in the swirl angular velocity \(\omega_c\):

\subsubsection*{Connection Coefficients:}
\[\Gamma^\mu_{\nu\sigma} = \frac{1}{2} g^{\mu\lambda} \left( \frac{\partial g_{\lambda\nu}}{\partial x^\sigma} + \frac{\partial g_{\lambda\sigma}}{\partial x^\nu} - \frac{\partial g_{\nu\sigma}}{\partial x^\lambda} \right)\]

\subsubsection*{Curvature Tensor:}
The curvature tensor is:

\[R^\rho_{\sigma\mu\nu} = \partial_\mu \Gamma^\rho_{\nu\sigma} - \partial_\nu \Gamma^\rho_{\mu\sigma} + \Gamma^\rho_{\mu\lambda} \Gamma^\lambda_{\nu\sigma} - \Gamma^\rho_{\nu\lambda} \Gamma^\lambda_{\mu\sigma}\]

This curvature depends on the gradients of \(\omega_c\), reflecting the influence of swirl dynamics.

\subsection*{7. Example: Gravitational Effects of a Vortex Core}
\subsubsection*{Parameters:}
\begin{itemize}
\item Tangential velocity: \(C_e = 1.0 \times 10^6 \, \text{m/s}\),

\item Core radius: \(r_c = 1.4 \times 10^{-15} \, \text{m}\),

\item Density of Æther: \(\rho = 10^{-7} \, \text{kg/m}^3\).
\end{itemize}
\subsubsection*{Swirl Energy Density:}
\[\rho_\text{swirl} = \frac{1}{2} \rho C_e^2 = \frac{1}{2} \times 10^{-7} \times (1.0 \times 10^6)^2 = 5.0 \times 10^{4} \, \text{J/m}^3\]

\subsubsection*{Swirl Potential:}
Assume a vortex mass \(m = \rho_\text{swirl} V\) and calculate the potential:

\[\phi_\text{swirl} = -\frac{G_\text{swirl} m}{r}\]

Substitute into the equations of motion to find the acceleration induced by the swirl potential.

\subsection*{Conclusion}
By replacing vorticity with the swirl of the vortex core, the dynamics of the vortex Æther framework can be directly tied to gravitational effects. This refined model emphasizes the localized dynamics of the vortex core and its influence on other knots via angular velocity (\(\omega_c\)) and tangential velocity (\(C_e\)), leading to a gravitational-like interaction consistent with Newtonian and relativistic effects.

In the \(C_e\)-\(F_{\max}\) framework, let’s hypothesize an equivalent formula. Since \(C_e\) represents a fundamental angular velocity and \(F_{\max}\) represents a force scale, their combination can define a fundamental energy density scale. We use:

\[u_\text{vortex}(r, \omega) = \frac{F_{\max}}{C_e} \cdot \Phi(r, \omega),\]

where \(\Phi(r, \omega)\) is a functional form determined by boundary conditions and system constraints.

\subsubsection*{Step 2: Specific Form for \(\Phi(r, \omega)\)}
To capture the analog to spectral energy density, consider that the energy scales with frequency \(\omega\) in the quantum case. Similarly, the vortex energy density could scale with \(r^{-2}\) (spatial localization) and a frequency factor \(\omega^3\):

\[\Phi(r, \omega) = \frac{\omega^3}{r^2}.\]

Thus, the energy density becomes:

\[u_\text{vortex}(r, \omega) = \frac{F_{\text{max} \omega^3}}{C_e r^2}.\]

This is an angular velocity-dependent energy density.
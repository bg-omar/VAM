%! Author = mr
%! Date = 9/3/2025
\newcommand{\canonversion}{\textbf{v0.3.4}} % Semantic versioning: vMAJOR.MINOR.PATCH
\newcommand{\papertitle}{Swirl String Theory (SST) Canon \canonversion}
\newcommand{\paperdoi}{10.5281/zenodo.17014358}


\documentclass{article}
% ------- TikZ Preamble -------
\RequirePackage{tikz}
\usetikzlibrary{knots,hobby,calc,intersections,decorations.pathreplacing,shapes.geometric,spath3}
% ------- Shared styles (from your preamble) -------
\tikzset{
    knot diagram/every strand/.append style={ultra thick, black},
    every path/.style={black,line width=2pt},
    every node/.style={transform shape,knot crossing,inner sep=1.5pt},
    every knot/.style={line cap=round,line join=round,very thick},
}
% ------- Guides toggle -------
\newif\ifsstguides
\sstguidestrue
% ------- Helper: label & skeleton for points P1..Pn -------
\newcommand{\SSTGuidesPoints}[2]{% #1=basename (e.g. P), #2=last index
    \ifsstguides
    \foreach \i in {1,...,#2}{
        \fill[blue] (#1\i) circle (1.2pt);
        \node[blue,font=\scriptsize,above] at (#1\i) {\i};
    }
    \draw[gray!40, dashed] \foreach \i [remember=\i as \lasti (initially 1)] in {2,...,#2,1} { (#1\lasti)--(#1\i) };
    \fi
}

\usepackage{amsmath,amssymb,amsfonts}
\usepackage{bm}
\usepackage{hyperref}
\usepackage{multicol}
\usepackage{calc,pict2e,picture}
\usepackage{textgreek,textcomp,gensymb,stix}
\usepackage{longtable}

% ==== Swirl String Theory (SST) macros ====
% Context-aware subscript symbol; uses math styles, not \scriptsize
\newcommand{\swirlarrow}{%
    \mathchoice{\mkern-2mu\scriptstyle\boldsymbol{\circlearrowleft}}%
    {\mkern-2mu\scriptstyle\boldsymbol{\circlearrowleft}}%
    {\mkern-2mu\scriptscriptstyle\boldsymbol{\circlearrowleft}}%
    {\mkern-2mu\scriptscriptstyle\boldsymbol{\circlearrowleft}}%
}
\newcommand{\swirlarrowcw}{%
    \mathchoice{\mkern-2mu\scriptstyle\boldsymbol{\circlearrowright}}%
    {\mkern-2mu\scriptstyle\boldsymbol{\circlearrowright}}%
    {\mkern-2mu\scriptscriptstyle\boldsymbol{\circlearrowright}}%
    {\mkern-2mu\scriptscriptstyle\boldsymbol{\circlearrowright}}%
}

% Canonical symbols
\newcommand{\vswirl}{\mathbf{v}_{\swirlarrow}}
\newcommand{\vswirlcw}{\mathbf{v}_{\swirlarrowcw}}
\newcommand{\SwirlClock}{S_{(t)}^{\swirlarrow}}
\newcommand{\SwirlClockcw}{S_{(t)}^{\swirlarrowcw}}
\newcommand{\omegas}{\boldsymbol{\omega}_{\swirlarrow}}  % swirl vorticity
\newcommand{\vscore}{v_{\swirlarrow}}                    % shorthand: |v_swirl| at r=r_c
\newcommand{\vnorm}{\lVert \mathbf{v}_{\mkern-2mu\scriptscriptstyle\boldsymbol{\circlearrowleft}} \rVert}               % swirl speed magnitude
\newcommand{\rhof}{\rho_{\!f}}                           % effective fluid density
\newcommand{\rhoE}{\rho_{\!E}}                           % swirl energy density /c^2? (we define clearly below)
\newcommand{\rhom}{\rho_{\!m}}                           % mass-equivalent density
\newcommand{\rc}{r_c}                                    % string core radius (swirl string radius)
\newcommand{\FmaxEM}{F_{\mathrm{EM}}^{\max}}             % EM-like maximal force scale
\newcommand{\FmaxG}{F_{\mathrm{G}}^{\max}}               % G-like maximal force scale
\newcommand{\Lam}{\Lambda}                               % Swirl Coulomb constant
\newcommand{\Om}{\Omega_{\swirlarrow}}                   % swirl angular frequency profile
\newcommand{\alpg}{\alpha_g}                             % gravitational fine-structure analogue

% Policy: the golden constant is only allowed via hyperbolic functions.
% Never write (1+\sqrt{5})/2; always use \xig=\asinh(1/2), \varphi=e^{\xig}.
\newcommand{\xig}{\operatorname{asinh}\!\left(\tfrac{1}{2}\right)} % base hyperbolic scale  "golden" constant is fundamentally hyperbolic.
\newcommand{\phig}{\exp(\xig)}                                     % golden from hyperbolic
\newcommand{\phialg}{\bigl(1+\sqrt{5}\bigr)/2}                     % algebraic echo (use sparingly)
\newcommand{\xigold}{\tfrac{3}{2}\,\xig}                           % "golden rapidity" scale

% --- Display helpers (optional) ---
\newcommand{\GoldenDeclare}{%
    \textbf{Golden (hyperbolic)}:\ \(\ln\phi=\xig\), hence \(\phi=\phig\).
    \ \emph{(Equivalently, \(\phi=\phialg\); this algebraic form is derivative.)}%
}
% --- Canonical identity (hyperbolic-only proof, algebraic as corollary) ---
\newtheorem{identity}{Identity}


% Preamble

\title{Sample \LaTeX}
% Document
\begin{document}
    \maketitle

    \begin{abstract}

        \begin{equation}
            \vswirl = \frac{\Gamma}{2\pi r}\bigl(1-\exp(-r^2/\rc^2)\bigr)\hat{\bm{\phi}}
        \end{equation}

        \[\vswirlcw = -\frac{\Gamma}{2\pi r}\bigl(1-\exp(-r^2/\rc^2)\bigr)\hat{\bm{\phi}}\]


        \begin{equation}
            \SwirlClock = \int \rhof \vswirl \, dV
        \end{equation}

        Or as anti version \(\SwirlClockcw = \int \rhof \vswirlcw \, dV\).

        $\omegas\quad \vscore\quad \vnorm\quad \rhof\quad \rhoE\quad \rhom\quad \rc\quad \FmaxEM\quad \FmaxG\quad \Lam\quad \Om$

        inline equation $a^2 + b^2 = c^2$ type with \$...\$ dollar signs
        and also like this \(e^{i\pi} + 1 = 0\) with \\( ... \)        But does it support \textbf{bold}, \textit{italic} and \underline{underline} text?.
    \end{abstract}

    \section{Text}
    paragraphs and ligatures goes here.
        But does it support \textbf{bold}, \textit{italic} and \underline{underline} text?.


    \section{Characters}
    Sample characters:
    \$ \& \% \# \_ \{ \} \~{} \^{}

\begin{itemize}
\item Recommended numerical values for canonical symbols.
\item Conjectures or alternatives pending proof or axiomatization.
\end{itemize}

    \begin{itemize}
        \item \textbf{Axiom/Postulate (Canonical).} Primitive assumption of SST.
        \item \textbf{Definition (Canonical).} Introduces a symbol by construction.
        \item \textbf{Theorem/Corollary (Canonical).} Proven consequence within $\mathcal{S}$.
        \item \textbf{Constitutive Model.} Canonical if derived from $\mathcal{P},\mathcal{D}$; otherwise semi-empirical.
        \item \textbf{Calibration (Empirical).} Recommended numerical values for canonical symbols.
        \item \textbf{Research Track.} Conjectures or alternatives pending proof or axiomatization.
    \end{itemize}
    Items may be promoted or demoted between classes only upon satisfying or failing the Canonicality Tests.
    \subsection*{Canonicality Tests (all required)}
    \begin{enumerate}
        \item \textbf{Derivability} from $\mathcal{P},\mathcal{D}$ via $\mathcal{R}$.
        \item \textbf{Dimensional consistency} (strict SI usage; correct physical limits).
        \item \textbf{Symmetry compliance} (Galilean symmetry and incompressibility).
        \item \textbf{Recovery limits} (Newtonian gravity, Coulomb/Bohr, linear wave optics).
        \item \textbf{Non-contradiction} with accepted canonical results.
        \item \textbf{Parameter discipline} (no ad hoc fits beyond calibrations).
    \end{enumerate}

    \section{Math}
    Sample math formulae:
    $f(x) = \int_{-\infty}^\infty \hat f(\xi)\,e^{2 \pi \xi} \, d\xi$

    \section{Multicolumn}
    \begin{multicols}{3}
        Column 1
        Column 2
        Column 3
    \end{multicols}

    \section{Boxes}
    \medbreak\noindent\fbox{\verb|\mbox{|\emph{Sample box}\verb|}|}\smallbreak

    \section{Symbols}
    Sample symbols:
    \noindent \textfractionsolidus \textdiv \texttimes \textminus \textpm \textsurd \textlnot \textasteriskcentered
    \textbullet \textperiodcentered \textdagger \textdaggerdbl \textsection \textparagraph \textbardbl \textellipsis
    \textquotedblleft \textquotedblright \textquoteleft \textquoteright \textbackslash \textbullet \textemdash \textendash


\section{New Section $\omegas\quad$} This is a new section
\subsection{New Subsection} This is a new subsection
\subsubsection{New Subsubsection} This is a new subsubsection
\paragraph{New Paragraph  $\omegas\quad$ } This is a new paragraph



% Example table
\begin{table}[h!]
  \begin{center}
    \caption{caption: Your first table.}
    \label{tab:table1}
    \begin{tabular}{l|c|r}
      \textbf{Value 1} & \textbf{Value 2} & \textbf{Value 3}\\
      $\alpha$ & $\beta$ & $\gamma$ \\
      \hline
      1 & 1110.1 & a\\
      2 & 10.1 & b\\
      3 & 23.113231 & c\\
    \end{tabular}
  \end{center}
\end{table}


    \section{Picture}
    Sample picture:
    \setlength{\unitlength}{0.8cm}
    \begin{picture}(6,5)
        \thicklines
        \put(1,0.5){\line(2,1){3}}
        \put(4,2){\line(-2,1){2}}
        \put(2,3){\line(-2,-5){1}}
        \put(0.7,0.3){$A$}
        \put(4.05,1.9){$B$}
        \put(1.7,2.95){$C$}
        \put(3.1,2.5){$a$}
        \put(1.3,1.7){$b$}
        \put(2.5,1.05){$c$}
        \put(0.3,4){$F=\sqrt{s(s-a)(s-b)(s-c)}$}
        \put(3.5,0.4){$\displaystyle s:=\frac{a+b+c}{2}$}
    \end{picture}

\end{document}
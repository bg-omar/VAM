
% Benchmarking the Vortex Æther Model vs General Relativity
\documentclass{article}
\usepackage{amsmath}
\usepackage{graphicx}
\usepackage{geometry}
\usepackage{hyperref}
\usepackage[none]{hyphenat}
\usepackage{array}
\usepackage{booktabs}
\usepackage{amssymb}
\usepackage{physics}


    \sloppy
    \author{Omar Iskandarani}
    \title{Benchmarking the Vortex Æther Model vs General Relativity}
    \date{\today}
    \affiliation{Independent Researcher, Groningen, The Netherlands}
    \thanks{ORCID: \href{https://orcid.org/0009-0006-1686-3961}{0009-0006-1686-3961}}
    \email{info@omariskandarani.com}

\begin{document}

\maketitle

% Abstract
\begin{abstract}
This paper compares the Vortex Æther Model (VAM) to General Relativity (GR) across multiple classical and modern relativistic tests, including time dilation, redshift, light deflection, perihelion precession, frame-dragging, gravitational radiation, and strong-field dynamics. VAM’s predictions are benchmarked numerically against GR and observational data, highlighting areas of agreement and necessary modifications.
\end{abstract}

% Content placeholder
\section{Introduction}

\section{Introduction}
In a modern revival of Lord Kelvin’s 1867 vortex atom hypothesis~\cite{Kelvin1867-vortex}, we consider an absolute Euclidean space filled with a superfluid æther. In this framework, elementary particles (atoms) are stable vortex knots in the æther, and \emph{time} is identified with the intrinsic angular rotation of these vortex cores. The challenge is to derive \emph{time dilation} formulas analogous to those in Special and General Relativity (SR and GR), using physical parameters of the æther (such as constant density and fundamental scales like the Planck time) instead of 4D spacetime geometry. We require that any new formula reproduces established relativistic effects – for example, the slowing of clocks near a massive body (gravitational redshift) or at high velocity (special-relativistic time dilation) – despite working in a flat 3D background. In other words, the æther’s \emph{vortex dynamics} must mimic the 4D metric curvature of GR to high precision.

This report develops a mathematically rigorous model for time dilation in the superfluid æther paradigm. We begin by formalizing the key assumptions of the æther model and defining how a vortex’s rotation serves as a physical clock. We then derive two sets of time-dilation equations: one for relative motion (analogous to SR) and one for gravitational fields (analogous to GR). Finally, we show that these results match standard relativistic predictions (e.g., gravitational redshift, orbiting clock rates) and discuss how \emph{vortex angular velocity} in the æther replaces spacetime curvature as the mechanism of time dilation. Throughout, we cite primary literature for comparison and validation, and use fundamental constants (Planck time $t_{\textrm P}$, maximum force $F_{\textrm max}$, æther density $\rho_{\text{\ae}}$, etc.) to express the new formulas in familiar terms.

\section{Gravitational Time Dilation}
\input{sections/time_dilation.tex}

\section{Velocity and Orbital Time Dilation}
\input{sections/velocity_dilation.tex}

\section{Frame Dragging}
\input{sections/frame_dragging.tex}

\section{Gravitational Redshift}
\section{Gravitational Redshift (Frequency Shift of Light)}

Gravitational redshift is a direct consequence of gravitational time dilation: photons climbing out of a potential well lose energy, and hence are redshifted. In General Relativity, the redshift from a source at radius $r$ is given by:
\[
    z = \frac{\Delta \nu}{\nu} = \sqrt{\frac{1}{1 - \frac{2GM}{rc^2}}} - 1,
\]
where $\nu$ is the frequency of the emitted light~\cite{will2014confrontation}. For small potentials, this simplifies to:
\[
    z \approx \frac{GM}{rc^2}.
\]

\subsection*{VAM Prediction}
In the Vortex Æther Model (VAM), redshift is interpreted as arising from the kinetic energy of aether swirl. The VAM formula is:
\[
    z_{\text{VAM}} = \left(1 - \frac{v_\phi^2}{c^2}\right)^{-1/2} - 1,
\]
which agrees with GR if one equates $v_\phi^2 = 2GM/r$~\cite{grin3d2025}. Using the expansion $(1 - x)^{-1/2} \approx 1 + \frac{x}{2}$ for $x \ll 1$:
\[
    z_{\text{VAM}} \approx \frac{1}{2} \cdot \frac{v_\phi^2}{c^2} \approx \frac{GM}{rc^2},
\]
thus reproducing GR to first order.

\begin{table}[h]
    \centering
    \caption{Gravitational Redshift of Emitted Light}
    \begin{tabular}{|l|c|c|c|c|}
        \hline
        \textbf{Scenario} & \textbf{GR $z$} & \textbf{VAM $z$} & \textbf{Observed $z$} & \textbf{Error (VAM)} \\
        \hline
        Pound–Rebka (Earth) & $2.5\times10^{-15}$ & $2.5\times10^{-15}$ & $2.5\times10^{-15} \pm 5\%$~\cite{pound1960apparent} & 0\% \\
        Sun Surface & $2.12\times10^{-6}$ & $2.12\times10^{-6}$ & $2.12\times10^{-6}$~\cite{vesely2001solar} & Few \% \\
        Sirius B & $5.5\times10^{-5}$ & $5.5\times10^{-5}$ & $4.8(3)\times10^{-5}$~\cite{greenstein1971gravitational} & $\sim$15\% \\
        Neutron Star & $0.3$ & $0.3$ & $0.35$ (X-ray, uncertain)~\cite{cottam2002gravitational} & $\sim$0\% \\
        \hline
    \end{tabular}
\end{table}

\subsection*{Black Hole Analogue}
In VAM, the redshift diverges as $v_\phi \to c$:
\[
    \lim_{v_\phi \to c} z_{\text{VAM}} \to \infty,
\]
which mimics the Schwarzschild event horizon.

\subsection*{Assessment and Fixes}
Gravitational redshift is well-modeled by VAM if $v_\phi$ is set appropriately. However, this tuning may feel ad hoc. A proposed improvement is to derive $v_\phi$ from vortex energy via a vorticity--gravity coupling constant $\gamma$, where:
\[
    GM \sim \gamma \cdot \text{(circulation energy)}.
\]
This would provide a predictive mechanism linking mass and swirl velocity~\cite{grin3d2025}.

\subsection*{Conclusion}
With the current empirical tuning of $v_\phi$, VAM matches gravitational redshift observations at all scales tested. Future refinements should focus on deriving swirl velocity from fundamental vortex energetics rather than matching escape speed heuristically.


\section{Light Deflection}\input{sections/light_deflection.tex}

\section{Perihelion Precession}
\input{sections/perihelion_precession.tex}

\section{Gravitational Potential}
\input{sections/gravitational_potential.tex}

\section{Gravitational Radiation}
\input{sections/gravitational_waves.tex}

\section{Geodetic Precession}
\input{sections/geodetic_precession.tex}

\section{Summary and Conclusions}

\section{Conclusion}

We have derived time dilation laws within a 3D Euclidean æther model, where particles are modeled as vortex knots, and time is defined by their intrinsic vortex core rotation. Motion through the æther and ætheric inflows (gravitational fields) reduce the observable angular velocity of vortex rotation, yielding:

\begin{itemize}
    \item The special-relativistic time dilation:
    \[
    \frac{d\tau}{dt} = \sqrt{1 - \frac{v^2}{c^2}},
    \]
    which arises from absolute motion through the æther.
    
    \item The gravitational time dilation:
    \[
    \frac{d\tau}{dt} = \sqrt{1 - \frac{2GM}{rc^2}},
    \]
    which arises from inward æther flow near mass $M$.
    
    \item The unified general case:
    \[
    \frac{d\tau}{dt} = \sqrt{1 - \frac{|\vec{u} - \vec{v}_g|^2}{c^2}},
    \]
    covering motion in a gravitational field.
\end{itemize}

These results precisely reproduce predictions from Special and General Relativity using physically intuitive mechanisms grounded in fluid dynamics.

The æther model eliminates the need for curved spacetime by replacing it with structured velocity fields in a flat space. It reinterprets relativistic time effects as real, mechanical consequences of vortex core dynamics interacting with a physical æther. 

This approach links microphysics (vortex core rotation) with cosmological structure (black hole horizons) and maintains continuity across scales. By interpreting time dilation as vortex angular slowdown, this model offers a mechanistic, field-based alternative to geometric spacetime curvature, preserving experimental consistency with SR and GR while opening possibilities for fluid-dynamical extensions of fundamental physics~\cite{Winterberg2002-PlanckAether,Schiller2022-maxforce}.

Future work may involve deriving Einstein’s field equations from æther vorticity conservation or testing laboratory analogs via superfluid experiments. The reinterpretation of black hole horizons, gravitational redshift, and quantum timekeeping via vortex rotation encourages deeper theoretical and experimental investigation into the æther’s role in modern physics.


\bibliographystyle{plain}
\bibliography{vam_gr_benchmark}

\end{document}

\documentclass[a4paper,12pt]{revtex4}
\usepackage[paperwidth=210mm, paperheight=297mm, margin=2.5cm]{geometry}
\usepackage[none]{hyphenat}
\usepackage{amsmath}
\usepackage{graphicx}
\usepackage{hyperref}
\usepackage{array}
\usepackage{booktabs}
\usepackage{amssymb}
\usepackage{physics}

\sloppy
\begin{document}
\author{Omar Iskandarani}
\title{Time Dilation in a 3D Superfluid Æther Model, Based on Vortex Core Rotation and Ætheric Flow}
\date{\today}
\affiliation{Independent Researcher, Groningen, The Netherlands}
\thanks{ORCID: \href{https://orcid.org/0009-0006-1686-3961}{0009-0006-1686-3961}}
\email{info@omariskandarani.com}

\maketitle

\section*{Abstract}
In this paper, we derive time dilation equations within a 3D Euclidean superfluid æther model. In this framework, fundamental particles are modeled as vortex knots, and time is defined by the intrinsic angular rotation of their vortex cores. The goal is to replace the spacetime curvature concept of General Relativity (GR) with quantized angular velocity fields in a flat-space æther, while reproducing all experimental predictions of time dilation under GR and Special Relativity (SR). We provide derivations from first principles, grounded in fluid dynamics and vortex mechanics, and express the time dilation factors in terms of fundamental constants like the Planck time and maximum force.


\section{Introduction}
In a modern revival of Lord Kelvin’s 1867 vortex atom hypothesis~\cite{Kelvin1867-vortex}, we consider an absolute Euclidean space filled with a superfluid æther. In this framework, elementary particles (atoms) are stable vortex knots in the æther, and \emph{time} is identified with the intrinsic angular rotation of these vortex cores. The challenge is to derive \emph{time dilation} formulas analogous to those in Special and General Relativity (SR and GR), using physical parameters of the æther (such as constant density and fundamental scales like the Planck time) instead of 4D spacetime geometry. We require that any new formula reproduces established relativistic effects – for example, the slowing of clocks near a massive body (gravitational redshift) or at high velocity (special-relativistic time dilation) – despite working in a flat 3D background. In other words, the æther’s \emph{vortex dynamics} must mimic the 4D metric curvature of GR to high precision.

This report develops a mathematically rigorous model for time dilation in the superfluid æther paradigm. We begin by formalizing the key assumptions of the æther model and defining how a vortex’s rotation serves as a physical clock. We then derive two sets of time-dilation equations: one for relative motion (analogous to SR) and one for gravitational fields (analogous to GR). Finally, we show that these results match standard relativistic predictions (e.g., gravitational redshift, orbiting clock rates) and discuss how \emph{vortex angular velocity} in the æther replaces spacetime curvature as the mechanism of time dilation. Throughout, we cite primary literature for comparison and validation, and use fundamental constants (Planck time $t_{\rm P}$, maximum force $F_{\rm max}$, æther density $\rho_{\!A}$, etc.) to express the new formulas in familiar terms.


\section{Superfluid Æther Framework}

We posit a stationary, Euclidean 3-dimensional æther that behaves like a zero-viscosity superfluid of constant mass density. This continuous medium underpins all of physics: particles are topological vortex structures in the æther, and fields correspond to flow patterns (vorticity, pressure, etc.). The key assumptions can be summarized as follows:

\begin{itemize}
    \item \textbf{Flat Absolute Space:} Space is a fixed Euclidean backdrop (no inherent curvature). There is a preferred rest frame defined by the æther at rest. (This is similar to Lorentz’s original absolute frame concept, but now with a physical superfluid filling space~\cite{Winterberg2002-PlanckAether}). All coordinate distances are measured in this flat space, not in a curved metric.

    \item \textbf{Constant Density:} The æther has uniform density $\rho_{\text{\ae}}$ and is incompressible (analogous to superfluid helium at $T=0$). Thus, æther volume elements cannot be created or destroyed; flow is divergenceless except possibly at singular vortex cores. Any local variations (e.g., near masses) involve velocity fields or pressure, not density changes.

    \item \textbf{Atoms as Vortex Knots:} Following Kelvin~\cite{Kelvin1867-vortex}, an “atom” or fundamental particle is a quantized vortex loop or knot in the æther. It has a well-defined core (of order the Planck length $l_{\rm P}$ in radius, according to Planck-æther theories~\cite{Winterberg2002-PlanckAether}) around which æther flows circulationally. The vortex’s topology (knot type) might correspond to particle type, while its intrinsic angular velocity $\omega$ (the swirl rate of æther around the core) gives the particle its internal clock.

    \item \textbf{Time as Vortex Rotation:} Proper time for a particle is defined by the rotation of its vortex core. For example, a certain fixed angle of rotation (say one full $2\pi$ revolution of the core) could define a fixed amount of proper time (perhaps on the order of one “tick”). A particle’s age or internal time advances by the number of revolutions its core executes. Faster core rotation means faster internal time rate. Importantly, this rotation is an absolute physical process occurring relative to the æther.

    \item \textbf{Emergent Temperature and Irrotational Flow:} In the bulk of the æther (far from vortex cores), flow may be irrotational and laminar. Macroscopic thermodynamic concepts (temperature, entropy) are assumed to emerge statistically from small-scale æther dynamics, but at the fundamental level the æther is a dissipationless, non-thermal medium. Thus, we ignore any finite-temperature or viscous effects – the æther is a perfect inviscid fluid.

    \item \textbf{Vorticity Fields and Interactions:} All forces (electromagnetism, gravity, etc.) are mediated by æther flows. Spatial gradients in vorticity or helicity (twist of vortex lines) in the æther field can influence other vortices. For instance, what we perceive as a “gravitational field” will be modeled by a certain æther velocity field (as we detail later). The principle of maximum force $F_{\rm max} = c^4 / 4G$ from general relativity~\cite{Schiller2022-maxforce}, which sets an upper bound on force in nature, is presumed to emerge from the æther’s properties (e.g., maximum flow speed $c$ and density $\rho_{\text{\ae}}$ impose a limit on momentum flux/force).
\end{itemize}

Under this framework, the æther provides an absolute reference for motion, but all measurable effects must ultimately be consistent with relativity. Indeed, as phrased by Winterberg (2002), ``the universe can be considered Euclidean flat-spacetime provided we include a densely filled quantum vacuum superfluid as aether''~\cite{Winterberg2002-PlanckAether}.

\textbf{Definitions and Constants:} For later use, we define some fundamental constants in this model. The Planck time is
\[
    t_{\rm P} = \sqrt{\frac{\hbar G}{c^5}} \approx 5.39\times10^{-44}\ \text{s},
\]
the natural unit of time in quantum gravity. It represents roughly the time for light to travel one Planck length $l_{\rm P} \approx 1.62\times10^{-35}$ m. In many superfluid-æther theories, $l_{\rm P}$ might be the core diameter of elementary vortices~\cite{Winterberg2002-PlanckAether}, so one full rotation of an elemental vortex at the speed of light $c$ would take on the order of $t_{\rm P}$. Thus, $t_{\rm P}$ sets an upper bound on rotation frequency ($\sim 10^{43}$ s$^{-1}$) for any physical clock in the æther.

Another useful constant is the proposed maximum force:
\[
    F_{\rm max} = \frac{c^4}{4G} \approx 3.0\times10^{43}\ \text{N}.
\]
This appears as an upper limit in general relativity~\cite{Schiller2022-maxforce}, for example, the gravitational attraction between two black holes cannot exceed $F_{\rm max}$. In the æther picture, $F_{\rm max}$ can be interpreted as the maximum stress or drag force the superfluid æther can sustain when flows approach the speed of light.

We retain $c$ (speed of light in vacuum) as the characteristic signal speed in the æther (e.g., the speed of sound or wave propagation in the superfluid vacuum, often taken as $c = \sqrt{B/\rho_{\text{\ae}}}$ for bulk modulus $B$). The Newtonian gravitational constant $G$ will enter when linking æther flow to mass (since mass is essentially a vortex with a certain circulation and core structure that relates to $G$). We will introduce any additional constants as needed.


\section{Vortex Clocks and Proper Time}

In this model, a “clock” is realized by a microscopic vortex’s rotation. To make this concrete, consider a free particle at rest in the æther. Its vortex core spins steadily, dragging nearby æther around. Let $\omega_0$ denote the angular velocity of this core as measured in the æther rest frame (in units of radians per second). By definition, $\omega_0$ is the particle’s \emph{proper rotational frequency}, corresponding to its proper time $\tau$.

We can relate $\omega_0$ to the passage of proper time: if the core rotates by $\Delta \theta$ radians in an interval, then the proper time elapsed is
\[
\Delta \tau = \frac{\Delta \theta}{\omega_0} \,.
\]
For example, if we choose $2\pi$ radians of rotation as a “tick” of the clock, then the proper period is $T_0 = 2\pi/\omega_0$. One might imagine $\omega_0$ is set by the particle’s internal structure – e.g., a proton’s vortex might rotate at some $10^{23}$ rad/s such that $T_0 \sim 10^{-23}$ s for one revolution (this is speculative, but notably, de Broglie in 1924 proposed that every particle of rest mass $m$ has an internal clock of frequency $mc^2/h$~\cite{deBroglie1924-frequency}, on the order of $10^{21}$ Hz for an electron; a vortex model could provide a physical origin for this \emph{Zitterbewegung} frequency as core rotation).

For now, $\omega_0$ is a free parameter representing the clock rate at rest. When the particle is not free or not at rest, its observed rotation rate can change. We define $\omega_{\textrm obs}$ as the angular velocity of the vortex core as observed by a static æther frame observer (i.e., one at rest with respect to the æther) under whatever circumstances (motion or gravity). The ratio $\omega_{\textrm obs}/\omega_0$ will then give the rate of the clock relative to proper time.

In fact, since $\Delta \tau = \Delta \theta / \omega_0$ always holds for the clock itself, and $\Delta t$ (coordinate time) corresponds to $\Delta \theta / \omega_{\textrm obs}$ (the angle rotated in lab frame time), we have:
\[
\frac{\Delta \tau}{\Delta t} = \frac{\Delta \theta / \omega_0}{\Delta \theta / \omega_{\textrm obs}} = \frac{\omega_{\textrm obs}}{\omega_0} \,. \tag{1}
\]

This important relation links the physical slowdown of the vortex’s spin $\omega_{\textrm obs}$ to the time-dilation factor. If $\omega_{\textrm obs} < \omega_0$, the clock runs slow (since $\Delta \tau < \Delta t$).

Our task in the next sections is to determine $\omega_{\textrm obs}$ for two cases:
\begin{enumerate}
    \item When the vortex (particle) moves at velocity $v$ through the æther,
    \item When the vortex sits in a gravitational potential (æther flow) created by a massive body.
\end{enumerate}
We will find that $\omega_{\textrm obs}/\omega_0$ in these cases reproduces the familiar Lorentz and gravitational time dilation factors, respectively.

Before proceeding, we emphasize that \emph{proper time $\tau$ is fundamentally just a count of vortex rotation in this model}. This provides an objective, mechanistic view of time: e.g., one might imagine a tiny flag or marker on the vortex core that completes laps around the core – each lap is an unambiguous physical event corresponding to a fixed amount of proper time. Different physical clocks (atoms, molecules, etc.) would all ultimately trace their time to such microscopic circulations in the universal æther.

As long as the laws of physics are such that these circulations are stable and identical for identical particles, this provides a standard of time. Next, we show how motion through the æther and æther flows influence $\omega_{\textrm obs}$.

\section{Time Dilation from Relative Motion}

First, consider time dilation for a particle moving at high speed relative to the æther rest frame. Empirically, we know that a clock moving at velocity $v$ experiences time slower by the Lorentz factor $\gamma = 1/\sqrt{1 - v^2/c^2}$. In this model, we derive the same effect by analyzing the influence of absolute æther motion on vortex core rotation.

\subsection*{(a) Kinematic Derivation}

Let a vortex be at rest in its own frame $S'$ but moving at velocity $v$ relative to the æther rest frame $S$. In $S'$, the vortex rotates with angular frequency $\omega_0$, and defines proper time $\tau$. Due to Lorentz time dilation, an observer in $S$ sees the clock slow down:
\[
\omega_{\text{obs}} = \omega_0 \sqrt{1 - \frac{v^2}{c^2}} \,.
\]
From the relation between proper and coordinate time,
\[
\frac{d\tau}{dt} = \frac{\omega_{\text{obs}}}{\omega_0} = \sqrt{1 - \frac{v^2}{c^2}} \,. \tag{2}
\]

This matches the standard SR time dilation formula. In our model, the physical mechanism is that æther motion across the vortex disrupts its swirl rate, slowing the apparent rotation in the æther frame.

\subsection*{(b) Fluid-Dynamic Interpretation}

A complementary interpretation uses compressible flow analogies. In fluid dynamics, a body moving at speed $v$ in a compressible medium with signal speed $c$ experiences distortions proportional to $\gamma = 1/\sqrt{1 - v^2/c^2}$. This can be thought of as a Doppler time dilation or resistance to maintaining coherent circulation. 

As velocity approaches the æther signal speed $c$, the surrounding flow compresses and resists vortex rotation. Therefore, the angular velocity seen in the æther frame drops, and:
\[
\omega_{\text{obs}} = \omega_0 \sqrt{1 - \frac{v^2}{c^2}} \Rightarrow \frac{d\tau}{dt} = \sqrt{1 - \frac{v^2}{c^2}} \,. \tag{3}
\]

\subsection*{Implication}

This gives us the relativistic time dilation for a moving clock:
\[
\boxed{\frac{d\tau}{dt} = \sqrt{1 - \frac{v^2}{c^2}}}
\]
within a Euclidean, æther-based flat space, and matches all special relativity experimental predictions~\cite{Rado2020-aether-Lorentz,Levy2009-aether-clock}.


\section{Gravitational Time Dilation}

In General Relativity, clocks deeper in a gravitational potential well run slower compared to those at higher potentials. We reproduce this result using æther flow fields instead of spacetime curvature.

\subsection*{Æther Flow as Gravity}

We assume that mass $M$ induces an inward radial flow of æther. At a radius $r$, this flow speed is given by:
\[
v_g(r) = \sqrt{\frac{2GM}{r}}.
\]
This mirrors the Painlevé–Gullstrand metric and the river model of black holes~\cite{Hamilton2004-river}.

\subsection*{Æther Drag and Clock Slowdown}

A clock held at radius $r$ in this inward æther flow sees æther moving past it at speed $v_g(r)$. The vortex core's observed angular velocity is therefore reduced due to the æther's drag, just as in the special relativity case, where motion through æther reduces the observed clock rate.

Thus, the gravitational time dilation factor is:
\[
\frac{d\tau}{dt} = \sqrt{1 - \frac{v_g^2(r)}{c^2}} = \sqrt{1 - \frac{2GM}{rc^2}}. \tag{4}
\]
This matches the Schwarzschild solution for stationary observers in general relativity.

\subsection*{Interpretation}

This equation means that the deeper a vortex is located in the gravitational potential (the faster the local æther flow), the slower it rotates from the perspective of an observer at infinity. At the Schwarzschild radius $r_s = 2GM/c^2$, $d\tau/dt = 0$: time stops for external observers.

This provides a mechanistic interpretation of gravitational redshift: light emitted by a vortex-clock in a strong potential well appears redshifted due to the slower angular motion of the emitting vortex. The result:
\[
\boxed{\frac{d\tau}{dt} = \sqrt{1 - \frac{2GM}{rc^2}}}
\]
is fully consistent with GR and supports the æther flow analogy~\cite{Schiller2022-maxforce}.


\section{Combined Effects and Further Predictions}

Having derived separate time dilation factors for motion through æther and gravitational æther flow, we now consider both effects simultaneously.

\subsection*{Combined Motion and Gravitational Field}

Let a vortex-clock move with velocity $\vec{u}$ in a region where the æther is flowing with velocity $\vec{v}_g$. The effective relative velocity with respect to the local æther flow is:
\[
\vec{v}_{\text{rel}} = \vec{u} - \vec{v}_g.
\]
The observed time dilation is then:
\[
\frac{d\tau}{dt} = \sqrt{1 - \frac{|\vec{v}_{\text{rel}}|^2}{c^2}}. \tag{5}
\]
This formulation smoothly incorporates both special and general relativistic effects into a single expression.

\subsection*{Example: Circular Orbit Time Dilation}

Consider a clock orbiting a mass $M$ at radius $r$. The tangential velocity of the orbit is:
\[
v_{\text{orb}} = \sqrt{\frac{GM}{r}}, \quad v_g(r) = \sqrt{\frac{2GM}{r}}.
\]
Since the orbital velocity is perpendicular to the radial æther inflow, the relative speed is:
\[
v_{\text{rel}} = \sqrt{v_{\text{orb}}^2 + v_g^2} = \sqrt{\frac{3GM}{r}}.
\]
Thus, the time dilation becomes:
\[
\frac{d\tau}{dt} = \sqrt{1 - \frac{3GM}{rc^2}}. \tag{6}
\]
This matches the exact result from Schwarzschild geometry for circular orbits.

\subsection*{Implications Near a Horizon}

As $r \to r_s = 2GM/c^2$, the inflow speed $v_g(r)$ approaches $c$, and any static observer's clock slows to zero. The æther flow fully suppresses local vortex rotation, providing a natural mechanism for the "freezing of time" at the event horizon.

\subsection*{Unified Interpretation}

This æther model allows all relativistic time dilation effects to be viewed as consequences of one principle:
\[
\text{Clock rate reduction} \;\propto\; \text{relative motion through æther}.
\]
Whether this relative motion arises from inertial velocity or from ætheric inflow due to nearby mass, the observable consequence is the same. Therefore, we conclude:
\[
\boxed{\frac{d\tau}{dt} = \sqrt{1 - \frac{|\vec{u} - \vec{v}_g|^2}{c^2}}}
\]
as the general time dilation formula for the Vortex Æther Model.


\section{Conclusion}

We have derived time dilation laws within a 3D Euclidean æther model, where particles are modeled as vortex knots, and time is defined by their intrinsic vortex core rotation. Motion through the æther and ætheric inflows (gravitational fields) reduce the observable angular velocity of vortex rotation, yielding:

\begin{itemize}
    \item The special-relativistic time dilation:
    \[
    \frac{d\tau}{dt} = \sqrt{1 - \frac{v^2}{c^2}},
    \]
    which arises from absolute motion through the æther.
    
    \item The gravitational time dilation:
    \[
    \frac{d\tau}{dt} = \sqrt{1 - \frac{2GM}{rc^2}},
    \]
    which arises from inward æther flow near mass $M$.
    
    \item The unified general case:
    \[
    \frac{d\tau}{dt} = \sqrt{1 - \frac{|\vec{u} - \vec{v}_g|^2}{c^2}},
    \]
    covering motion in a gravitational field.
\end{itemize}

These results precisely reproduce predictions from Special and General Relativity using physically intuitive mechanisms grounded in fluid dynamics.

The æther model eliminates the need for curved spacetime by replacing it with structured velocity fields in a flat space. It reinterprets relativistic time effects as real, mechanical consequences of vortex core dynamics interacting with a physical æther. 

This approach links microphysics (vortex core rotation) with cosmological structure (black hole horizons) and maintains continuity across scales. By interpreting time dilation as vortex angular slowdown, this model offers a mechanistic, field-based alternative to geometric spacetime curvature, preserving experimental consistency with SR and GR while opening possibilities for fluid-dynamical extensions of fundamental physics~\cite{Winterberg2002-PlanckAether,Schiller2022-maxforce}.

Future work may involve deriving Einstein’s field equations from æther vorticity conservation or testing laboratory analogs via superfluid experiments. The reinterpretation of black hole horizons, gravitational redshift, and quantum timekeeping via vortex rotation encourages deeper theoretical and experimental investigation into the æther’s role in modern physics.


\bibliographystyle{plain}
\bibliography{00_aether_time_dilation}

\end{document}
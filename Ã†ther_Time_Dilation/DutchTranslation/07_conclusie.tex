\section{Conclusie}

We hebben tijdsdilatatie-wetten afgeleid binnen een 3D Euclidisch æthermodel, waarbij deeltjes worden gemodelleerd als wervelknopen en tijd wordt gedefinieerd door hun intrinsieke wervelkernrotatie. Beweging door de æther en ætherische instromen (zwaartekrachtvelden) verminderen de waarneembare hoeksnelheid van de wervelrotatie, wat resulteert in:

\begin{itemize}
    \item De speciaal-relativistische tijdsdilatatie:
    \[
        \frac{d\tau}{dt} = \sqrt{1 - \frac{v^2}{c^2}},
    \]
    die voortkomt uit absolute beweging door de æther.

    \item De gravitationele tijdsdilatatie:
    \[
        \frac{d\tau}{dt} = \sqrt{1 - \frac{2GM}{rc^2}},
    \]
    die ontstaat door inwaartse ætherstroom nabij massa $M$.

    \item Het uniforme algemene geval:
    \[
        \frac{d\tau}{dt} = \sqrt{1 - \frac{|\vec{u} - \vec{v}_g|^2}{c^2}},
    \]
    die beweging in een gravitatieveld bestrijkt.
\end{itemize}

Deze resultaten reproduceren nauwkeurig voorspellingen van de speciale en algemene relativiteitstheorie met behulp van fysiek intuïtieve mechanismen die gegrond zijn in vloeistofdynamica.

Het æthermodel elimineert de noodzaak van gekromde ruimtetijd door deze te vervangen door gestructureerde snelheidsvelden in een vlakke ruimte. Het herinterpreteert relativistische tijdseffecten als echte, mechanische gevolgen van wervelkerndynamiek die interageert met een fysieke æther.

Deze benadering koppelt microfysica (wervelkernrotatie) aan kosmologische structuur (horizonten van zwarte gaten) en handhaaft continuïteit over schalen heen. Door tijdsdilatatie te interpreteren als hoekvertraging van wervels, biedt dit model een mechanistisch, veldgebaseerd alternatief voor geometrische ruimtetijdkromming, waarbij experimentele consistentie met SR en GR behouden blijft en tegelijkertijd mogelijkheden worden geopend voor vloeistofdynamische uitbreidingen van fundamentele fysica~\cite{Winterberg2002-PlanckAether,Schiller2022-maxforce}.

Toekomstig werk kan het afleiden van Einsteins veldvergelijkingen van behoud van æthervorticiteit of het testen van laboratoriumanalogen via superfluïde experimenten omvatten. De herinterpretatie van horizonten van zwarte gaten, gravitationele roodverschuiving en kwantumtijdwaarneming via wervelrotatie moedigt dieper theoretisch en experimenteel onderzoek aan naar de rol van de æther in de moderne fysica.

Een uitgebreidere uitwerking van deze ideeën vindt men in het vervolgonderzoek: \textit{“Swirl Clocks and Vorticity-Induced Gravity”} (2025).~\cite{vam2025unified}.
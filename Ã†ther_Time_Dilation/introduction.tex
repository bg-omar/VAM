
\section{Introduction}
In a modern revival of Lord Kelvin’s 1867 vortex atom hypothesis~\cite{Kelvin1867-vortex}, we consider an absolute Euclidean space filled with a superfluid æther. In this framework, elementary particles (atoms) are stable vortex knots in the æther, and \emph{time} is identified with the intrinsic angular rotation of these vortex cores. The challenge is to derive \emph{time dilation} formulas analogous to those in Special and General Relativity (SR and GR), using physical parameters of the æther (such as constant density and fundamental scales like the Planck time) instead of 4D spacetime geometry. We require that any new formula reproduces established relativistic effects – for example, the slowing of clocks near a massive body (gravitational redshift) or at high velocity (special-relativistic time dilation) – despite working in a flat 3D background. In other words, the æther’s \emph{vortex dynamics} must mimic the 4D metric curvature of GR to high precision.

This report develops a mathematically rigorous model for time dilation in the superfluid æther paradigm. We begin by formalizing the key assumptions of the æther model and defining how a vortex’s rotation serves as a physical clock. We then derive two sets of time-dilation equations: one for relative motion (analogous to SR) and one for gravitational fields (analogous to GR). Finally, we show that these results match standard relativistic predictions (e.g., gravitational redshift, orbiting clock rates) and discuss how \emph{vortex angular velocity} in the æther replaces spacetime curvature as the mechanism of time dilation. Throughout, we cite primary literature for comparison and validation, and use fundamental constants (Planck time $t_{\textrm P}$, maximum force $F_{\textrm max}$, æther density $\rho_{\text{\ae}}$, etc.) to express the new formulas in familiar terms.
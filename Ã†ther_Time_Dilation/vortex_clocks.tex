
\section{Vortex Clocks and Proper Time}

In this model, a “clock” is realized by a microscopic vortex’s rotation. To make this concrete, consider a free particle at rest in the æther. Its vortex core spins steadily, dragging nearby æther around. Let $\omega_0$ denote the angular velocity of this core as measured in the æther rest frame (in units of radians per second). By definition, $\omega_0$ is the particle’s \emph{proper rotational frequency}, corresponding to its proper time $\tau$.

We can relate $\omega_0$ to the passage of proper time: if the core rotates by $\Delta \theta$ radians in an interval, then the proper time elapsed is
\[
\Delta \tau = \frac{\Delta \theta}{\omega_0} \,.
\]
For example, if we choose $2\pi$ radians of rotation as a “tick” of the clock, then the proper period is $T_0 = 2\pi/\omega_0$. One might imagine $\omega_0$ is set by the particle’s internal structure – e.g., a proton’s vortex might rotate at some $10^{23}$ rad/s such that $T_0 \sim 10^{-23}$ s for one revolution (this is speculative, but notably, de Broglie in 1924 proposed that every particle of rest mass $m$ has an internal clock of frequency $mc^2/h$~\cite{deBroglie1924-frequency}, on the order of $10^{21}$ Hz for an electron; a vortex model could provide a physical origin for this \emph{Zitterbewegung} frequency as core rotation).

For now, $\omega_0$ is a free parameter representing the clock rate at rest. When the particle is not free or not at rest, its observed rotation rate can change. We define $\omega_{\rm obs}$ as the angular velocity of the vortex core as observed by a static æther frame observer (i.e., one at rest with respect to the æther) under whatever circumstances (motion or gravity). The ratio $\omega_{\rm obs}/\omega_0$ will then give the rate of the clock relative to proper time.

In fact, since $\Delta \tau = \Delta \theta / \omega_0$ always holds for the clock itself, and $\Delta t$ (coordinate time) corresponds to $\Delta \theta / \omega_{\rm obs}$ (the angle rotated in lab frame time), we have:
\[
\frac{\Delta \tau}{\Delta t} = \frac{\Delta \theta / \omega_0}{\Delta \theta / \omega_{\rm obs}} = \frac{\omega_{\rm obs}}{\omega_0} \,. \tag{1}
\]

This important relation links the physical slowdown of the vortex’s spin $\omega_{\rm obs}$ to the time-dilation factor. If $\omega_{\rm obs} < \omega_0$, the clock runs slow (since $\Delta \tau < \Delta t$).

Our task in the next sections is to determine $\omega_{\rm obs}$ for two cases:
\begin{enumerate}
    \item When the vortex (particle) moves at velocity $v$ through the æther,
    \item When the vortex sits in a gravitational potential (æther flow) created by a massive body.
\end{enumerate}
We will find that $\omega_{\rm obs}/\omega_0$ in these cases reproduces the familiar Lorentz and gravitational time dilation factors, respectively.

Before proceeding, we emphasize that \emph{proper time $\tau$ is fundamentally just a count of vortex rotation in this model}. This provides an objective, mechanistic view of time: e.g., one might imagine a tiny flag or marker on the vortex core that completes laps around the core – each lap is an unambiguous physical event corresponding to a fixed amount of proper time. Different physical clocks (atoms, molecules, etc.) would all ultimately trace their time to such microscopic circulations in the universal æther.

As long as the laws of physics are such that these circulations are stable and identical for identical particles, this provides a standard of time. Next, we show how motion through the æther and æther flows influence $\omega_{\rm obs}$.

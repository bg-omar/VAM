\section{Superfluïde Æther Framework}

We veronderstellen een stationaire, Euclidische 3-dimensionale æther die zich gedraagt als een superfluïde met een viscositeit van nul en een constante massadichtheid. Dit continue medium vormt de basis van alle natuurkunde: deeltjes zijn topologische wervelstructuren in de æther en velden corresponderen met stromingspatronen (vorticiteit, druk, etc.). De belangrijkste aannames kunnen als volgt worden samengevat:

\begin{itemize}
    \item \textbf{Platte absolute ruimte:} Ruimte is een vaste Euclidische achtergrond (geen inherente kromming). Er is een voorkeursrustframe gedefinieerd door de æther in rust. (Dit is vergelijkbaar met Lorentz's oorspronkelijke absolute frameconcept, maar nu met een fysieke superfluïde die de ruimte vult~\cite{Winterberg2002-PlanckAether}). Alle coördinaatafstanden worden gemeten in deze vlakke ruimte, niet in een gebogen metriek.

    \item \textbf{Constante dichtheid:} De æther heeft een uniforme dichtheid $\rho_{\text{\ae}}$ en is onsamendrukbaar (analoog aan supervloeibaar helium bij $T=0$). Daarom kunnen æthervolume-elementen niet worden gecreëerd of vernietigd; de stroming is divergentieloos, behalve mogelijk bij singuliere wervelkernen. Alle lokale variaties (bijv. nabij massa's) hebben betrekking op snelheidsvelden of druk, niet op dichtheidsveranderingen.

    \item \textbf{Atomen als wervelknopen:} Volgens Kelvin~\cite{Kelvin1867-vortex} is een "atoom" of fundamenteel deeltje een gekwantiseerde wervellus of -knoop in de æther. Het heeft een goed gedefinieerde kern (van de orde van de Planck-lengte $l_{\textrm P}$ in straal, volgens de Planck-æther-theorieën~\cite{Winterberg2002-PlanckAether}) waar æther circulair omheen stroomt. De topologie van de wervel (knooptype) zou kunnen overeenkomen met het type deeltje, terwijl de intrinsieke hoeksnelheid $\omega$ (de wervelsnelheid van æther rond de kern) het deeltje zijn interne klok geeft.

    \item \textbf{Tijd als wervelrotatie:} De juiste tijd voor een deeltje wordt gedefinieerd door de rotatie van zijn wervelkern. Bijvoorbeeld, een bepaalde vaste rotatiehoek (zeg één volledige $2\pi$ omwenteling van de kern) zou een vaste hoeveelheid juiste tijd kunnen definiëren (misschien in de orde van één "tik"). De leeftijd of interne tijd van een deeltje gaat vooruit met het aantal omwentelingen dat zijn kern uitvoert. Snellere kernrotatie betekent een snellere interne tijdssnelheid. Belangrijk is dat deze rotatie een absoluut fysiek proces is dat plaatsvindt ten opzichte van de æther.

    \item \textbf{Opkomende temperatuur en irrotationele stroming:} In het grootste deel van de æther (ver van wervelkernen) kan de stroming irrotationeel en laminair zijn. Macroscopische thermodynamische concepten (temperatuur, entropie) worden statistisch gezien verondersteld voort te komen uit kleinschalige ætherdynamica, maar op fundamenteel niveau is de æther een dissipatieloos, niet-thermisch medium. Daarom negeren we alle eindige-temperatuur- of viskeuze effecten – de æther is een perfecte niet-viskeuze vloeistof.

    \item \textbf{Vorticiteitsvelden en interacties:} Alle krachten (elektromagnetisme, zwaartekracht, enz.) worden gemedieerd door ætherstromen.
    Ruimtelijke gradiënten in vorticiteit of heliciteit (draaiing van wervellijnen) in het ætherveld kunnen andere vortices beïnvloeden.
    Bijvoorbeeld, wat wij waarnemen als een $\text{"zwaartekrachtsveld"}$ zal worden gemodelleerd door een bepaald æthersnelheidsveld (zoals we later
zullen toelichten). Het principe van maximale kracht $ F_\text{\max} = c^4 / 4 G $ uit de algemene relativiteitstheorie~
    \cite{Schiller2022-maxforce}, dat een bovengrens stelt aan kracht in de natuur, wordt verondersteld voort te komen uit de eigenschappen van de æther (bijv. maximale stroomsnelheid $c$ en dichtheid $\rho_\text{\ae}$ leggen een limiet op aan impulsflux/kracht).
\end{itemize}

Binnen dit raamwerk biedt de æther een absolute referentie voor beweging, maar alle meetbare effecten moeten uiteindelijk consistent zijn met de relativiteitstheorie. Zoals Winterberg (2002) het formuleerde, ``kan het universum worden beschouwd als Euclidische vlakke ruimtetijd, op voorwaarde dat we een dichtbevolkt kwantumvacuüm superfluïde als æther opnemen''~\cite{Winterberg2002-PlanckAether}.

\textbf{Definities en constanten:} Voor later gebruik definiëren we enkele fundamentele constanten in dit model. De Planck-tijd is
\[
    t_{\textrm P} = \sqrt{\frac{\hbar G}{c^5}} \approx 5.39\times10^{-44}\ \text{s},
\]
de natuurlijke eenheid van tijd in kwantumzwaartekracht. Het vertegenwoordigt ongeveer de tijd die licht nodig heeft om één Planck-lengte $l_{\textrm P} \approx 1.62\times10^{-35}$ m af te leggen. In veel superfluïde-æther-theorieën zou $l_{\textrm P}$ de kerndiameter van elementaire wervelstructuren kunnen zijn~\cite{Winterberg2002-PlanckAether}, dus één volledige rotatie van een elementaire wervelstructuur met de lichtsnelheid $c$ zou de orde van $t_{\textrm P}$ aannemen. Dus $t_{\textrm P}$ stelt een bovengrens in voor de rotatiefrequentie ($\sim 10^{43}$ s$^{-1}$) voor elke fysieke klok in de æther.

Een andere nuttige constante is de voorgestelde maximale kracht:
\[
    F_\text{\max} = \frac{c^4}{4G} \approx 3.0\times10^{43}\ \text{N}.
\]
Dit verschijnt als een bovengrens in de algemene relativiteitstheorie~\cite{Schiller2022-maxforce}, bijvoorbeeld, de zwaartekracht tussen twee zwarte gaten kan $F_\text{\max}$ niet overschrijden. In de æther-afbeelding kan $F_{\textrm max}$ worden geïnterpreteerd als de maximale spanning of sleepkracht die de superfluïde æther kan verdragen wanneer stromingen de lichtsnelheid naderen.

We behouden $c$ (snelheid van het licht in vacuüm) als de karakteristieke signaalsnelheid in de æther (bijv. de snelheid van geluid of golfvoortplanting in het superfluïde vacuüm, vaak genomen als $c = \sqrt{B/\rho_{\text{\ae}}}$ voor bulkmodulus $B$). De Newtoniaanse gravitatieconstante $G$ zal ingaan bij het koppelen van ætherstroming aan massa (aangezien massa in wezen een wervel is met een bepaalde circulatie en kernstructuur die verband houdt met $G$). We zullen indien nodig extra constanten introduceren.
\section{Gravitationel Tijd Dilatatie}

In de algemene relativiteitstheorie lopen klokken dieper in een gravitationele potentiaalput langzamer vergeleken met klokken met hogere potentialen. We reproduceren dit resultaat met behulp van ætherstroomvelden in plaats van ruimtetijdkromming.

\subsection*{Ætherstroom als zwaartekracht}

We nemen aan dat massa $M$ een inwaartse radiale stroming van æther induceert. Bij een straal $r$ wordt deze stroomsnelheid gegeven door:
\[
    v_g(r) = \sqrt{\frac{2GM}{r}}.
\]
Dit weerspiegelt de Painlevé-Gullstrand-metriek en het riviermodel van zwarte gaten~\cite{Hamilton2004-river}.

\subsection*{Ætherweerstand en klokvertraging}

Een klok die op straal $r$ in deze inwaartse ætherstroom wordt gehouden, ziet æther er langs bewegen met snelheid $v_g(r)$. De waargenomen hoeksnelheid van de wervelkern wordt daarom verminderd door de weerstand van de æther, net als in het speciale relativiteitsgeval, waarbij beweging door de æther de waargenomen kloksnelheid vermindert.

De gravitationele tijdsdilatatiefactor is dus:
\[
    \frac{d\tau}{dt} = \sqrt{1 - \frac{v_g^2(r)}{c^2}} = \sqrt{1 - \frac{2GM}{rc^2}}. \tag{4}
\]
Dit komt overeen met de Schwarzschild-oplossing voor stationaire waarnemers in de algemene relativiteitstheorie.

\subsection*{Interpretatie}

Deze vergelijking betekent dat hoe dieper een wervel zich in het gravitatiepotentieel bevindt (hoe sneller de lokale ætherstroom), hoe langzamer deze roteert vanuit het perspectief van een waarnemer op oneindig. Bij de Schwarzschild-straal $r_s = 2GM/c^2$, $d\tau/dt = 0$: de tijd stopt voor externe waarnemers.

Dit levert een mechanistische interpretatie van gravitationele roodverschuiving op: licht dat wordt uitgezonden door een wervelklok in een sterke potentiaalput, lijkt roodverschoven vanwege de langzamere hoekbeweging van de uitzendende wervel. Het resultaat:
\[
    \boxed{\frac{d\tau}{dt} = \sqrt{1 - \frac{2GM}{rc^2}}}
\]
is volledig consistent met GR en ondersteunt de æther-stroomanalogie~\cite{Schiller2022-maxforce}.
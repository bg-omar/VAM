
\section{Gravitational Time Dilation}

In General Relativity, clocks deeper in a gravitational potential well run slower compared to those at higher potentials. We reproduce this result using æther flow fields instead of spacetime curvature.

\subsection*{Æther Flow as Gravity}

We assume that mass $M$ induces an inward radial flow of æther. At a radius $r$, this flow speed is given by:
\[
v_g(r) = \sqrt{\frac{2GM}{r}}.
\]
This mirrors the Painlevé–Gullstrand metric and the river model of black holes~\cite{Hamilton2004-river}.

\subsection*{Æther Drag and Clock Slowdown}

A clock held at radius $r$ in this inward æther flow sees æther moving past it at speed $v_g(r)$. The vortex core's observed angular velocity is therefore reduced due to the æther's drag, just as in the special relativity case, where motion through æther reduces the observed clock rate.

Thus, the gravitational time dilation factor is:
\[
\frac{d\tau}{dt} = \sqrt{1 - \frac{v_g^2(r)}{c^2}} = \sqrt{1 - \frac{2GM}{rc^2}}. \tag{4}
\]
This matches the Schwarzschild solution for stationary observers in general relativity.

\subsection*{Interpretation}

This equation means that the deeper a vortex is located in the gravitational potential (the faster the local æther flow), the slower it rotates from the perspective of an observer at infinity. At the Schwarzschild radius $r_s = 2GM/c^2$, $d\tau/dt = 0$: time stops for external observers.

This provides a mechanistic interpretation of gravitational redshift: light emitted by a vortex-clock in a strong potential well appears redshifted due to the slower angular motion of the emitting vortex. The result:
\[
\boxed{\frac{d\tau}{dt} = \sqrt{1 - \frac{2GM}{rc^2}}}
\]
is fully consistent with GR and supports the æther flow analogy~\cite{Schiller2022-maxforce}.
